%%%%%%%%%%%%%%%%%%%%%%%%%%%%%%%%%%%%%%%%%
% kaobook
% LaTeX Template
% Version 1.3 (December 9, 2021)
%
% This template originates from:
% https://www.LaTeXTemplates.com
%
% For the latest template development version and to make contributions:
% https://github.com/fmarotta/kaobook
%
% Authors:
% Federico Marotta (federicomarotta@mail.com)
% Based on the doctoral thesis of Ken Arroyo Ohori (https://3d.bk.tudelft.nl/ken/en)
% and on the Tufte-LaTeX class.
% Modified for LaTeX Templates by Vel (vel@latextemplates.com)
%
% License:
% CC0 1.0 Universal (see included MANIFEST.md file)
%
%%%%%%%%%%%%%%%%%%%%%%%%%%%%%%%%%%%%%%%%%

%----------------------------------------------------------------------------------------
%  PACKAGES AND OTHER DOCUMENT CONFIGURATIONS
%----------------------------------------------------------------------------------------

\documentclass[
  b5paper, % Page size
  fontsize=10pt, % Base font size
  twoside=true, % Use different layouts for even and odd pages (in particular, if twoside=true, the margin column will be always on the outside)
  %open=any, % If twoside=true, uncomment this to force new chapters to start on any page, not only on right (odd) pages
  %chapterentrydots=true, % Uncomment to output dots from the chapter name to the page number in the table of contents
  numbers=noenddot, % Comment to output dots after chapter numbers; the most common values for this option are: enddot, noenddot and auto (see the KOMAScript documentation for an in-depth explanation)
]{kaobook}

\renewcommand{\marginlayout}{%
  \newgeometry{
    top=27.4\vscale,
    bottom=27.4\vscale,
    inner=24.8\hscale,
    textwidth=147\hscale,
    marginparsep=6.2\hscale,
    marginparwidth=0\hscale,
  }%
  \recalchead
}

% Choose the language
\ifxetexorluatex
  \usepackage{polyglossia}
  \setmainlanguage{english}
\else
  \usepackage[english]{babel} % Load characters and hyphenation
\fi
\usepackage[english=british]{csquotes}  % English quotes

% Load packages for testing
\usepackage{blindtext}
%\usepackage{showframe} % Uncomment to show boxes around the text area, margin, header and footer
%\usepackage{showlabels} % Uncomment to output the content of \label commands to the document where they are used

% Load the bibliography package
\usepackage{kaobiblio}
\addbibresource{main.bib} % Bibliography file

% Load mathematical packages for theorems and related environments
\usepackage[framed=true]{kaotheorems}

% Load the package for hyperreferences
\usepackage{kaorefs}

\usepackage{mathpartir}
\usepackage{dirtree}
\usepackage{xspace}
\usepackage{tikz}
\usepackage{fancyvrb}
\usepackage{bussproofs}
\usepackage{proof}
\usepackage{upquote}
\usepackage{arydshln}
\usepackage{graphicx}
\usepackage{amsthm}

\usetikzlibrary{arrows}
\tikzset{
  treenode/.style = {align=center, inner sep=0pt, text centered},
  arn_u/.style = {treenode, circle, black, draw=black, text width=0.8cm, line width=0.25mm},
  arn_w/.style = {treenode, circle, white, draw=white, text width=0.8cm, line width=0.25mm},
  arn_nc/.style = {treenode, circle, black, draw=white, text width=0.8cm, line width=0.25mm},
}

\newcommand{\shorteq}{\code{=}}

\newcommand{\Lang}{}
\newcommand{\plang}{}
\newcommand{\code}[1]{\texttt{#1}}
\newcommand{\cx}{\code{x}}
\newcommand{\cy}{\code{y}}
\newcommand{\cz}{\code{z}}
\newcommand{\cf}{\code{f}}
\newcommand{\cv}{\code{v}}
\newcommand{\ca}{\code{a}}
\newcommand{\cb}{\code{b}}
\newcommand{\cX}{\code{X}}
\newcommand{\cY}{\code{Y}}
\newcommand{\cZ}{\code{Z}}
\newcommand{\inred}{\color{red}}

\newcommand{\chnum}[1]{\hyperref[ch:#1]{\ref{ch:#1}}}
\newcommand{\refchs}[2]{Chapters \chnum{#1} and \chnum{#2}}
\newcommand{\refchss}[3]{Chapters \chnum{#1}, \chnum{#2}, and \chnum{#3}}
\newcommand{\refex}[1]{Exercise~\hyperref[thm:#1]{\ref{thm:#1}}}
\newcommand{\labex}[1]{\labthm{#1}}
\newcommand{\pto}{\mathrel{\ooalign{\hfil$\mapstochar$\hfil\cr$\to$\cr}}}
\newcommand{\finto}{\stackrel{\mbox{\tiny fin}}{\pto}}
\newcommand{\embox}[1]{\mbox{\emph{#1}}}
\newcommand{\dom}[1]{\embox{Domain}(#1)}
\newcommand{\iadd}{+_{{}_{\mathbb{Z}}}}
\newcommand{\isub}{-_{{}_{\mathbb{Z}}}}

\newcommand{\fundef}[3]{\textsf{def}\ #1(#2)\shorteq#3}
\newcommand{\clov}[3]{\langle\efun{#1}{#2},#3\rangle}
\newcommand{\tclov}[3]{\langle\etfun{#1}{#2},#3\rangle}
\newcommand{\contv}[2]{\langle#1,#2\rangle}
\newcommand{\exprv}[2]{(#1,#2)}
\newcommand{\eadd}[2]{#1+#2}
\newcommand{\esub}[2]{#1-#2}
\newcommand{\ebind}[3]{\textsf{val}\ #1\shorteq#2\ \textsf{in}\ #3}
\newcommand{\evcc}[2]{\textsf{vcc}\ #1\ \textsf{in}\ #2}
\newcommand{\efun}[2]{\lambda#1.#2}
\newcommand{\efunt}[3]{\lambda#1{:}#2.#3}
\newcommand{\eapp}[2]{#1\ #2}
\newcommand{\eappfo}[2]{#1(#2)}
\newcommand{\erec}[4]{\textsf{def}\ #1(#2)\shorteq#3\ \textsf{in}\ #4}
\newcommand{\eifz}[3]{\textsf{if0}\ #1\ #2\ #3}
\newcommand{\eif}[3]{\textsf{if}\ #1\ #2\ #3}
\newcommand{\eskip}{\textsf{skip}}
\newcommand{\ewhilez}[2]{\textsf{while0}\ #1\ #2}
\newcommand{\ewhile}[2]{\textsf{while}\ #1\ #2}
\newcommand{\eref}[1]{\textsf{box}\ #1}
\newcommand{\ederef}[1]{!#1}
\newcommand{\eset}[2]{#1{:}\shorteq#2}
\newcommand{\eseq}[2]{#1;#2}
\newcommand{\erect}[6]{\textsf{def}\ #1(#2{:}#3){:}#4\shorteq#5\ \textsf{in}\ #6}
\newcommand{\etdef}[6]{\textsf{type}\ #1=#2@#3+#4@#5\ \textsf{in}\ #6}
\newcommand{\ematch}[7]{#1\ \textsf{match}\ #2(#3)\rightarrow #4,#5(#6)\rightarrow #7}
\newcommand{\etfun}[2]{\Lambda#1.#2}
\newcommand{\etapp}[2]{#1[#2]}
\newcommand{\true}{\textsf{true}}
\newcommand{\false}{\textsf{false}}

\newcommand{\tnum}{\textsf{num}}
\newcommand{\tbool}{\textsf{bool}}
\newcommand{\tfun}{\textsf{fun}}
\newcommand{\tarrow}[2]{#1\rightarrow#2}
\newcommand{\tforall}[2]{\forall#1.#2}
\newcommand{\ttop}{\textsf{top}}
\newcommand{\tbot}{\textsf{bottom}}

\newcommand{\mtk}{\square}
\newcommand{\evalk}[2]{#1\vdash#2::}
\newcommand{\evalkd}[1]{\evalk{\sigma}{#1}}
\newcommand{\evalke}[1]{\evalk{\emptyset}{#1}}
\newcommand{\addk}{(+)::}
\newcommand{\subk}{(-)::}
\newcommand{\appk}{(@)::}

\newcommand{\mts}{\blacksquare}
\newcommand{\conss}[1]{#1::}

\newcommand{\eval}[3]{#1\vdash#2\Rightarrow#3}
\newcommand{\evaln}[3]{\ensuremath{#2} evaluates to \ensuremath{#3} under
\ensuremath{#1}\xspace}
\newcommand{\evald}[2]{\eval{\sigma}{#1}{#2}}
\newcommand{\evaldn}[2]{\evaln{\sigma}{#1}{#2}}
\newcommand{\evale}[2]{\eval{\emptyset}{#1}{#2}}

\newcommand{\typeof}[3]{#1\vdash#2:#3}
\newcommand{\typeofd}[2]{\typeof{\Gamma}{#1}{#2}}
\newcommand{\typeofe}[2]{\typeof{\emptyset}{#1}{#2}}
\newcommand{\typeofn}[3]{the type of \ensuremath{#2} is
\ensuremath{#3} under \ensuremath{#1}\xspace}
\newcommand{\typeofdn}[2]{\typeofn{\Gamma}{#1}{#2}}
\newcommand{\typeofnc}[3]{The type of \ensuremath{#2} is
\ensuremath{#3} under \ensuremath{#1}\xspace}
\newcommand{\typeofdnc}[2]{\typeofnc{\Gamma}{#1}{#2}}

\newcommand{\wft}[2]{#1\vdash#2}
\newcommand{\wftd}[1]{\wft{\Gamma}{#1}}
\newcommand{\wftn}[2]{\ensuremath{#2} is well-formed under \ensuremath{#1}}
\newcommand{\wftdn}[1]{\wftn{\Gamma}{#1}}

\newcommand{\subt}[2]{#1<:#2}
\newcommand{\subtn}[2]{\ensuremath{#1} is a subtype of \ensuremath{#2}}

\newcommand{\stricte}[2]{#1\Downarrow#2}
\newcommand{\stricten}[2]{\ensuremath{#1} strictly evaluates to \ensuremath{#2}}

\newcommand{\seval}[5]{#1,#2\vdash#3\Rightarrow#4,#5}
\newcommand{\sevaln}[5]{
  \ensuremath{#3} evaluates to \ensuremath{#4} and changes the store from
  \ensuremath{#2} to \ensuremath{#5} under \ensuremath{#1}\xspace
}
\newcommand{\sevald}[4]{\seval{\sigma}{M_{#1}}{#2}{#3}{M_{#4}}}
\newcommand{\sevaldn}[4]{\sevaln{\sigma}{M_{#1}}{#2}{#3}{M_{#4}}}
\newcommand{\sevale}[3]{\seval{\emptyset}{\emptyset}{#1}{#2}{#3}}

\newcommand{\semanticrule}[2]{
\vspace{1em}
\textbf{Rule \textsc{#1}}\\
#2
\vspace{1em}
}
\newcommand{\typerule}[2]{
\vspace{1em}
\textbf{Rule \textsc{#1}}\\
#2
\vspace{1em}
}
\newcommand{\tand}{\text{ and }}

\graphicspath{{images/}} % Paths in which to look for images

\makeindex[columns=3, title=Alphabetical Index, intoc] % Make LaTeX produce the files required to compile the index

\makeglossaries % Make LaTeX produce the files required to compile the glossary
% \input{glossary.tex} % Include the glossary definitions

% \makenomenclature % Make LaTeX produce the files required to compile the nomenclature

% Reset sidenote counter at chapters
%\counterwithin*{sidenote}{chapter}

%----------------------------------------------------------------------------------------

\begin{document}

%----------------------------------------------------------------------------------------
%  BOOK INFORMATION
%----------------------------------------------------------------------------------------

% \titlehead{The \texttt{kaobook} class}
% \subject{Use this document as a template}

% \title[Example and documentation of the {\normalfont\texttt{kaobook}} class]{Example and documentation \\ of the {\normalfont\texttt{kaobook}} class}
% \subtitle{Customise this page according to your needs}
\title{Introduction to Programming Languages}

% \author[Federico Marotta]{Federico Marotta\thanks{A \LaTeX\ lover}}
\author{Jaemin Hong and Sukyoung Ryu}

% \date{\today}
\date{}

% \publishers{An Awesome Publisher}

%----------------------------------------------------------------------------------------

\frontmatter % Denotes the start of the pre-document content, uses roman numerals

%----------------------------------------------------------------------------------------
%  OPENING PAGE
%----------------------------------------------------------------------------------------

%\makeatletter
%\extratitle{
%  % In the title page, the title is vspaced by 9.5\baselineskip
%  \vspace*{9\baselineskip}
%  \vspace*{\parskip}
%  \begin{center}
%    % In the title page, \huge is set after the komafont for title
%    \usekomafont{title}\huge\@title
%  \end{center}
%}
%\makeatother

%----------------------------------------------------------------------------------------
%  COPYRIGHT PAGE
%----------------------------------------------------------------------------------------

\makeatletter
\uppertitleback{\@titlehead} % Header

% \lowertitleback{
%   \textbf{Disclaimer}\\
%   You can edit this page to suit your needs. For instance, here we have a no copyright statement, a colophon and some other information. This page is based on the corresponding page of Ken Arroyo Ohori's thesis, with minimal changes.

%   \medskip

%   \textbf{No copyright}\\
%   \cczero\ This book is released into the public domain using the CC0 code. To the extent possible under law, I waive all copyright and related or neighbouring rights to this work.

%   To view a copy of the CC0 code, visit: \\\url{http://creativecommons.org/publicdomain/zero/1.0/}

%   \medskip

%   \textbf{Colophon} \\
%   This document was typeset with the help of \href{https://sourceforge.net/projects/koma-script/}{\KOMAScript} and \href{https://www.latex-project.org/}{\LaTeX} using the \href{https://github.com/fmarotta/kaobook/}{kaobook} class.

%   The source code of this book is available at:\\\url{https://github.com/fmarotta/kaobook}

%   (You are welcome to contribute!)

%   \medskip

%   \textbf{Publisher} \\
%   First printed in May 2019 by \@publishers
% }
\lowertitleback{
  \copyright2022 Jaemin Hong and Sukyoung Ryu

  \medskip

  All rights reserved. No part of this book may be reproduced in any form by any
  electronic of mechanical means (including photocopying, recording, or
  information storage and retrieval) without permission in writing from the
  authors.
}
\makeatother

%----------------------------------------------------------------------------------------
%  DEDICATION
%----------------------------------------------------------------------------------------

% \dedication{
%   The harmony of the world is made manifest in Form and Number, and the heart and soul and all the poetry of Natural Philosophy are embodied in the concept of mathematical beauty.\\
%   \flushright -- D'Arcy Wentworth Thompson
% }

%----------------------------------------------------------------------------------------
%  OUTPUT TITLE PAGE AND PREVIOUS
%----------------------------------------------------------------------------------------

% Note that \maketitle outputs the pages before here

\maketitle

%----------------------------------------------------------------------------------------
%  PREFACE
%----------------------------------------------------------------------------------------

\setchapterpreamble[u]{\margintoc}
\chapter{Acknowledgement}
\labch{acknowledgement}

The contents of this book are based on the KAIST \textit{Programming Languages}
course. We thank PLT\sidenote{\url{https://racket-lang.org/people.html}} since
the course referred to many materials from PLT in the beginning.
We also thank every student who took the
course before. We have learned many things from the interaction with the
students, and those lessons have affected various parts of the book. In
addition, we thank all the previous and current teaching assistants of the
course. They gave opinions to the course and wrote some of the exercises in the
book. Especially, Jihyeok Park highly contributed to the course, and Jihee Park
helped us edit the exercises.

We would be delighted to receive comments and corrections, which may be sent to
\code{jaemin.hong@kaist.ac.kr}. We thank in advance everyone who will contribute
to the book in the future.


%----------------------------------------------------------------------------------------
%  TABLE OF CONTENTS & LIST OF FIGURES/TABLES
%----------------------------------------------------------------------------------------

\begingroup % Local scope for the following commands

% Define the style for the TOC, LOF, and LOT
%\setstretch{1} % Uncomment to modify line spacing in the ToC
%\hypersetup{linkcolor=blue} % Uncomment to set the colour of links in the ToC
\setlength{\textheight}{230\hscale} % Manually adjust the height of the ToC pages

% Turn on compatibility mode for the etoc package
\etocstandarddisplaystyle % "toc display" as if etoc was not loaded
\etocstandardlines % "toc lines" as if etoc was not loaded

\tableofcontents % Output the table of contents

% \listoffigures % Output the list of figures

% Comment both of the following lines to have the LOF and the LOT on different pages
% \let\cleardoublepage\bigskip
% \let\clearpage\bigskip

% \listoftables % Output the list of tables

\endgroup

%----------------------------------------------------------------------------------------
%  MAIN BODY
%----------------------------------------------------------------------------------------

\mainmatter % Denotes the start of the main document content, resets page numbering and uses arabic numbers
\setchapterstyle{kao} % Choose the default chapter heading style

\setchapterpreamble[u]{\margintoc}
\chapter{Introduction}
\labch{introduction}

What is a programming language?

The simplest answer is ``it is a language used for programming.'' However, this
answer does not help us understand programming languages. We need a better
question to get a better answer.

What does a programming language consist of?

There is a good answer for this question: ``in a narrow sense, a programming
language consists of syntax and semantics, and in a broad sense, it additionally
has a standard library and an ecosystem.''

Syntax and semantics are principal concepts to understand programming languages.
Syntax determines how a language looks like, and semantics fills the inside. If
we consider a programming language as a human, we can say that syntax is one’s
appearance, and semantics is one’s thoughts. Programmers write programs
according to syntax. Syntax decides characters used in source code. Once programs
are written, semantics decides what each program does. Without semantics, all
the programs are useless. Programs can work as being expected only after
semantics determines the meaning of them. A programming language with syntax and
semantics is complete. Programmers using that language can write programs with
the syntax and execute the programs with the semantics. From a theoretical
perspective, syntax and semantics are all of a programming language.

For programmers, syntax and semantics are not the only elements of a programming
language. First, the standard library of a language is another element. The
standard library provides various utilities required by applications: data
structures like lists and maps, functions handling file and network IO, and so
on. The standard library is like clothes for humans. A human without clothes is
a human; a programming language without a standard library is a programming
language. At the same time, clothes are important to humans as they make bodies
warm and protect bodies from dangerous objects. Similarly, a standard library is
important to a programming language as it supplies diverse functionalities for
applications. Each person wears clothes different from others, and each language
puts different things from other languages in its standard library. Some
languages include lots of utilities in their standard libraries, while others
include much less. Some languages treat lists and maps as built-in concepts in
their semantics, while others define them with other primitives in their standard libraries.
Programmers avoid using a language without a standard library because such a
language increases the effort to write programs.

Another important element to programmers is the ecosystem of a programming
language. The ecosystem includes everything related to the language: developers
and companies using the language, third-party libraries written in the language,
and so on. It is like a society for humans. If many programmers and companies
use a programming language, one can easily get help and find complementary
materials by using the same language. There will be more chances of cooperative
work and employment, too. Third-party libraries also take important roles in
software development. The standard library offers only general facilities and
often lacks domain-specific features. When a required functionality cannot be
found in the standard library, a third-party library can provide the exact
functionality. For these reasons, the ecosystem of a programming language is
important to programmers.

Practically, the standard library and the ecosystem of a language are important
elements. Unlike syntax and semantics, they are not essential. A programming
language can exist even without its standard library and ecosystem. However,
developers take standard libraries and ecosystems into account as well as syntax and
semantics to choose languages they use. From a practical perspective, a
programming language consists of syntax, semantics, a standard library, and an
ecosystem.

This book is not for helping readers use a specific programming language. It
does not recommend a specific programming language, either. This book helps
readers learn new programming languages easily. You can acquaint any programming
languages once you completely read and understand this book. Obviously, this
goal cannot be achieved if the book discusses various languages separately. It
is possible only by discussing the underlying principles of every programming
language.

The principles of programming languages can be found from their semantics. Each
language seems very different from the others, but it is actually not the case.
Precisely speaking, their insides are quite the same, while their appearances
look different. They look different because their syntax and standard libraries,
which determine the appearances, are different. However, their insides, the
semantics, fundamentally share the same mathematical principles. If you
understand essential concepts residing in the semantics of multiple languages,
it is easy to understand and learn new languages.

People who know the key principles and can separate the elements of a language
can easily learn programming languages. As an analogy, consider a man learning
how to use a computer. It is a big problem if he cannot distinguish a keyboard
from a computer. For example, he thinks ``to say hello, my right index finger
presses the keyboard, my left middle finger presses the keyboard, my right ring
finger presses the keyboard three times.'' If the layout of the keyboard changes,
he should learn the whole computer again. On the other hand, if he knows that a
keyboard is just a tool to input text, he will less suffer from the change of
the keyboard layout. As he thinks ``to say hello, I press H, E, L, L, and O,'' he
does not need to learn the whole computer again. Of course, he should learn the
new keyboard layout, but it will be much easier. In addition, it is
straightforward to apply his knowledge to do new things. For example, he will
easily figure out ``to say lol, I press L, O, and L.'' If he does not distinguish
a keyboard from a computer, he cannot find any common principles between saying
hello and saying lol. Learning programming languages is the same. People who
cannot distinguish syntax and semantics believe that they should learn the whole
language again when the syntax changes. On the other hand, people who can
distinguish syntax and semantics know that semantics remains the same even if
syntax may vary. They know that understanding the principles of semantics is
important to learn languages. Becoming familiar with the
new syntax is all they need to use a new language fluently.

This book explains the semantics of principal concepts in programming languages.
\refch{introduction-to-scala}, \refch{immutability},
\refch{functions}, and \refch{pattern-matching}
introduce the Scala programming language. This book
uses Scala to implement interpreters of languages introduced in the book.
\refch{syntax-and-semantics} explains syntax and
semantics. Then, the book finally introduces various features of programming languages.
\begin{itemize}
    \item \refch{identifiers} introduces identifiers.
    \item \refch{first-order-functions},
      \refch{first-class-functions}, and \refch{lambda-calculus} introduce functions.
    \item \refch{recursion} introduces recursion.
    \item \refch{mutable-boxes} and \refch{mutable-variables} introduce mutation.
    \item \refch{lazy-evaluation} introduces lazy evaluation.
\end{itemize}

\section{Exercises}

\begin{enumerate}
\item Write the name of a programming language that you have used.
  What are the pros and cons of the language?
\item Write the names of two programming languages you know and compare them.
\end{enumerate}


\pagelayout{wide} % No margins
\addpart{Scala}
\pagelayout{margin} % Restore margins

\setchapterpreamble[u]{\margintoc}
\chapter{Introduction to Scala}
\labch{introduction-to-scala}

This book uses Scala as an implementation language, and
this chapter thus introduces the Scala programming language. Scala stands for a
\textbf{sca}lable \textbf{la}nguage~\cite{programming-in-scala}. It is a
multi-paradigm language that allows both functional and object-oriented styles.
This book focuses on the functional nature of Scala. In this chapter, we will
see what functional programming is and why this book uses functional
programming. In addition, we will install Scala and write simple programs in
Scala.

\section{Functional Programming}

What is \textit{functional programming}?\index{functional programming}
According to Wikipedia,

\begin{quote}
Functional programming is a programming paradigm that treats computation as the
evaluation of mathematical functions and avoids changing-state and mutable data.
\end{quote}

According to the book Functional Programming in Scala~\cite{fp-in-scala},

\begin{quote}
Functional programming (FP) is based on a simple premise with far-reaching
implications: we construct our programs using only pure functions---in other words,
functions that have no side effects.
\end{quote}

The above two sentences are enough to describe functional programming.

First, consider the phrase ``the evaluation of mathematical functions.''
From the perspective of functional programming, a
program is a single mathematical expression and the execution of the program is
finding a value denoted by the expression.
Each expression consists of zero or more subexpressions and evaluates to a value.

Let us discuss how
functional programming is different from imperative programming with
code examples.

\begin{verbatim}
int x = 1;
int y = 2;
if (y < 3)
    x = x + 4;
else
    x = x - 5;
\end{verbatim}

The above code is written in C, which represents imperative languages. Imperative
programming mimics a way in which computers operate. During the execution of a
program, a state, which can be interpreted as the memory of a computer,
exists and the execution modifies the state. The execution of the above C program has
the following steps:

\begin{enumerate}
\item A state that both \code{x} and \code{y} are uninitialized
\item A state that \code{x} is \code{1} and \code{y} is uninitialized
\item A state that \code{x} is \code{1} and \code{y} is \code{2}
\item Since \code{y < 3} is \code{true} under the state of the third step, go to
the next line.
\item A state that \code{x} is \code{5} and \code{y} is \code{2}
\end{enumerate}

The state keeps changes throughout the execution of the program. Each line
modifies the current state rather than resulting in some value.

\begin{verbatim}
let x = 1 in
let y = 2 in
if y < 3 then x + 4 else x - 5
\end{verbatim}

The above code is written in OCaml, which represents functional languages. A
program is an expression and the result of the execution is the result of
evaluating the expression. The execution does not require the notion of a state.
The execution of the above OCaml program has the following steps:

\begin{enumerate}
\item Given the fact that \code{x} equals \code{1}, evaluate
\code{let y = 2 in if y < 3 then x + 4 else x - 5}.
\item Given the fact that \code{x} equals \code{1} and \code{y} equals \code{2},
evaluate \code{if y < 3 then x + 4 else x - 5}.
\item Given the fact that \code{x} equals \code{1} and \code{y} equals \code{2},
evaluating \code{y < 3} yields \code{true}, and the next step is to evaluate
\code{x + 4}.
\item Given the fact that \code{x} equals \code{1} and \code{y} equals \code{2},
evaluate \code{x + 4}.
\item The result is \code{5}.
\end{enumerate}

There is no state. Each expression consists of subexpressions. The result of
an expression is determined by the results of its subexpressions.

Since the programs are simple, two programs look similar, but it is important to
understand two different perspectives of what a program is.

Now, look at the phrases ``avoids changing-state and mutable data'' and ``using
only pure functions.'' Functional programming avoids mutable variables, mutable data
structures, and mutable objects. The term \textit{mutable}\index{mutable}
means being able to change. Its opposite is \textit{immutable}\index{immutable},
which means not being able to change. States change throughout the execution of programs.
In functional programming, states do not exist since things never change.
Due to the lack of states, a function always does the same stuff
and always returns the same value for the same arguments. Such functions
are called pure functions.

In practice, especially for large-scale projects, using only immutable things in
the whole code is often inefficient. Most real-world functional languages
provide mutation via language constructs like \code{var} of Scala, \code{ref} of
OCaml, and \code{set!} and \code{box} of Racket. However,
functional programming uses immutable things in most cases. Even without
mutation, we can still express most programs without difficulties.

As we have seen so far,
immutability is the most important concept of functional programming.
Immutability allows modular programming and eases the reasoning of programs.
Because of immutability, programs that have to be trustworthy or require parallel computing
are good applications of functional programming.
\refch{immutability} will discuss the advantages of immutability in detail and
how to write interesting programs without mutation.

There are other important characteristics of functional programming as well as
immutability. Use of first-class functions and pattern matching also take the
key roles in functional programming. Both first-class functions and pattern matching
are valuable as they help abstraction. First-class functions allow programmers
to abstact computation; pattern matching allows programmers to abstract
data. Because of the ability of abstraction,
programs whose input has complex and abstract structures like source code
are typically written in functional languages.
\refch{functions} and \refch{pattern-matching} will respectively discuss first-class
functions and pattern matching in Scala.

This book implements interpreters and type checkers. They take source code as
input and process the input according to the mathematical semantics of programming
languages. It is important
to reason about the correctness of interpreters and type checkers. These
properties exactly match the strengths of functional programming. It is why
this book uses functional programming and Scala.

Before moving on to the next section,
let us see how people use functional programming in industry.

Akka\sidenote{\url{https://akka.io/}} is a concurrent,
distributed computing library written in Scala. Many companies have been using Akka.
Apache Spark,\sidenote{\url{https://spark.apache.org/}} a well-known library for data
processing, also is written in Scala.
Play\sidenote{\url{https://www.playframework.com/}}
is a widely-used web framework based on Akka.

Facebook has developed Infer,\sidenote{\url{https://fbinfer.com/}} a static analyzer for Java,
C, C++, and Objective-C, in OCaml. Facebook and other companies including Amazon
and Mozila use Infer to find bugs statically in their programs. Facebook has
developed also Flow,\sidenote{\url{https://flow.org/}} a static type checker for JavaScript.
Jane Street\sidenote{\url{https://www.janestreet.com/}} is a financial company well-known in
the programming language community and has developed its own software in OCaml.
According to the OCaml website,\sidenote{\url{http://ocaml.org/learn/companies.html}}
various companies including Docker use OCaml.

Haskell Wiki\sidenote{\url{http://wiki.haskell.org/Haskell_in_industry}} describes that Google,
Facebook, Microsoft, Nvidia, and many other companies use Haskell.

Erlang is a functional language for concurrent and parallel computing. Elixir
operates on the Erlang virtual machine and is used for the same purpose as Erlang.
An article from Code
Sync\sidenote{\url{https://codesync.global/media/successful-companies-using-elixir-and-erlang/}}
said that various companies including WhatsApp, Pinterest, and Goldman Sachs
use Erlang and Elixir.

\section{Installation}

As Scala programs are compiled to Java bytecode, which runs on the Java Virtual
Machine (\acrshort{jvmLabel}), you must
install Java before installing Scala. Java has various versions. Scala 2.13,
which is used in this book, needs JDK 8 or higher. JDK 8 is the most recommended
one. The Scala website\sidenote{\url{https://docs.scala-lang.org/overviews/jdk-compatibility/overview.html}}
discusses compatibility issues regarding the other versions.

The Oracle website\sidenote{\url{https://www.oracle.com/java/technologies/javase/javase-jdk8-downloads.html}}
provides an installation file for JDK 8.

You can donwload an installation file for Scala 2.13 from the Scala
website.\sidenote{\url{https://www.scala-lang.org/download/}} Note that you need a
file in the ``Other resources'' section at the bottom of the page.
On macOS, you may use Homebrew instead. By installing Scala, you can use the
Scala REPL, interpreter, and compiler. \refsec{scala-repl},
\refsec{scala-interpreter}, and \refsec{scala-compiler} will discuss
their usages respectively.

Another thing to install is SBT. SBT is a build tool for Scala. An installation
file for SBT is available at the SBT
website.\sidenote{\url{https://www.scala-sbt.org/download.html}}
\refsec{sbt} will discuss the usage of SBT.

\section{REPL}
\labsec{scala-repl}

Once you install Scala, you can launch Scala REPL by typing \code{scala} in
your command line.

\begin{verbatim}
$ scala
Welcome to Scala 2.13.5.
Type in expressions for evaluation. Or try :help.

scala>
\end{verbatim}

The term \acrshort{replLabel} stands for \textbf{r}ead, \textbf{e}val, \textbf{p}rint, and
\textbf{l}oop.
It is a program that iterativley reads code from a user, evaluates the code,
and prints the result. REPL is not a place to write a program but is a good
place to write short code and see how it works.

If you input an integer to REPL, it will evaluate the integer and show the
result.

\begin{verbatim}
scala> 0
val res0: Int = 0
\end{verbatim}

It means that the expression \code{0} evaluates to the value \code{0} and the type of
\code{0} is \code{Int}.
You can try some arithmetic expressions as well.

\begin{verbatim}
scala> 1 + 2
val res1: Int = 3
\end{verbatim}

A boolean is \code{true} or \code{false} in Scala.

\begin{verbatim}
scala> true
val res2: Boolean = true
\end{verbatim}

You can also use basic logical operators.

\begin{verbatim}
scala> true && false
val res3: Boolean = false
\end{verbatim}

String literals require double quotation marks.

\begin{verbatim}
scala> "hello"
val res4: String = hello
\end{verbatim}

Operations regarding strings can be done by calling methods.

\begin{verbatim}
scala> "hello".length
val res5: Int = 5

scala> "hello".substring(0, 4)
val res6: String = hell
\end{verbatim}

Strings in Scala provide the same methods as those in
Java.\sidenote{\url{https://docs.oracle.com/javase/8/docs/api/java/lang/String.html}}

The \code{println} function prints a given message into the console.

\begin{verbatim}
scala> println("Hello world!")
Hello world!
\end{verbatim}

Note that there is no result of \code{println("Hello world!")}. Actually,
\code{println("Hello world!")} evaluates to \code{()}, which is called unit.
Unit implies that the result does not have any meaningful information. It is
similar to \code{None} in Python and \code{undefined} in JavaScript. At the
same time, functions returning unit are similar to functions whose return types
are \code{void} in C or Java. Since unit does not have meaningful information,
REPL does not show the result when it is unit.

The remainder of this section introduces basic features of Scala, such as
variables and functions, with REPL.

\subsection{Variables}

The syntax of a variable definition is as follows:

\begin{verbatim}
val [name]: [type] = [expression]
\end{verbatim}

It defines a variable whose name is \code{[name]}.
The result of the expression becomes the value denoted by the variable and
must belong to the type.

\begin{verbatim}
scala> val x: Int = 1
val x: Int = 1
\end{verbatim}

If the type of the result does not match a given type, the variable will not be
defined due to a type mismatch.

\begin{verbatim}
scala> val y: Boolean = 2
                        ^
       error: type mismatch;
        found   : Int(2)
        required: Boolean
\end{verbatim}

You can omit the \code{: [type]} part and use the following syntax instead:

\begin{verbatim}
val [name] = [expression]
\end{verbatim}

In this case, a type mimatch never happens, and the
type of the variable becomes the same as the type of its value.
People usually omit the type annotations of local variables.

\begin{verbatim}
scala> val x = 3
val x: Int = 3
\end{verbatim}

Variables defined by \code{val} cannot be mutated, i.e. their values never
change. Reassignment will incur an error. We call such variables immutable
variables.

\begin{verbatim}
scala> x = 4
         ^
       error: reassignment to val
\end{verbatim}

Sometimes, mutable variables, i.e. variables whose values can change, are useful.
Scala provides mutable variables as well as immutable variables. You need to use
\code{var} instead of \code{val} to define mutable variables. You may or may not write
the type of a variable.

\begin{verbatim}
scala> var z = 5
var z: Int = 5

scala> z = 6
// mutated z

scala> z
val res8: Int = 6
\end{verbatim}

To assign a new value to a mutable variable, the value must conform to the
type of the variable. Otherwise, a type mismatch will happen.

\begin{verbatim}
scala> z = true
           ^
       error: type mismatch;
        found   : Boolean(true)
        required: Int
\end{verbatim}

\subsection{Functions}

The syntax of a function definition is as follows:

\begin{verbatim}
def [name]([name]: [type], …): [type] = [expression]
\end{verbatim}

Many programming languages require \code{return} to specify the return value of a function.
On the other hand, functions in Scala are like functions in mathematics: \code{return} is unnecessary.
The return value of a function is the result of the body expression, which is the expression
at the right side of \code{=} in the definition. The type annotation after each parameter specifies
the type of the parameter. The type after the parentheses is the return type, which must be the
same as the type of the return value.

\begin{verbatim}
scala> def add(x: Int, y: Int): Int = x + y
def add(x: Int, y: Int): Int

scala> add(3, 7)
val res9: Int = 10
\end{verbatim}

The return types of functions can be omitted.

\begin{verbatim}
scala> def add(x: Int, y: Int) = x + y
def add(x: Int, y: Int): Int
\end{verbatim}

However, parameter types cannot be omitted.

\begin{verbatim}
scala> def add(x, y) = x + y
                ^
       error: ':' expected but ',' found.
\end{verbatim}

To write multiple expressions including variable and functions definitions
in the body of a function, we put expressions separated by line breaks
inside curly braces.
Each line will be evaluated in the order, and the result of the last line will
be the return value.

\begin{verbatim}
scala> def quadruple(x: Int): Int = {
     |   val y = x + x
     |   y + y
     | }
def quadruple(x: Int): Int
\end{verbatim}

Inside \code{quadruple}, the variable \code{y} is defined and used for the
computation of
the return value.\sidenote{Vertical bars (\code{|}) at the beginning of lines are
not part of code. They have been automatically inserted by REPL.}

Multiple expressions inside curly braces are collectively treated as a single
expression. We call such an expression a sequenced expression. Like any other
expressions, a sequenced expression can occur anywhere an expression is needed.
For example, it can be used to define a variable.

\begin{verbatim}
scala> val a = {
     |   val x = 1 + 1
     |   x + x
     | }
val a: Int = 4
\end{verbatim}

There are many other things related to functions: recursion, first-class
functions, closures, and anonymous functions. \refch{immutability} will
discuss recursion, and \refch{functions} will discuss the other topics.

\subsection{Conditionals}

A conditional expression performs computation depending on a certain
condition, i.e. a boolean value. The syntax of a conditional expression is as
follows:

\begin{verbatim}
if ([expression]) [expression] else [expression]
\end{verbatim}

The first expression is the condition; the second expression is the true branch;
the last expression is the false branch.

\begin{verbatim}
scala> if (true) 1 else 2
val res10: Int = 1
\end{verbatim}

A conditional expression evaluates to a value. It is more similar to the ternary
operator \code{? :} in C than a if statement.
We do not need to make a variable
mutable to initialize the variable with a conditional value.

\begin{verbatim}
scala> val x = if (true) 1 else 2
val x: Int = 1
\end{verbatim}

On the other hand, people write code like below in languages like C.

\begin{verbatim}
int x;
if (true)
    x = 1;
else
    x = 2;
\end{verbatim}

Conditional expressions in Scala are more expressive than the ternary operator
in C because we can make complex computation a single expression with expression
sequencing, which is impossible in C.

\begin{verbatim}
scala> if (true) {
     |   val x = 2
     |   x + x
     | } else {
     |   val x = 3
     |   x * x
     | }
val res11: Int = 4
\end{verbatim}

\subsection{Lists}

A list is a collection of zero or more elements. A list maintains the order between
its elements. Lists in Scala are immutable. Once a list is created, its elements
never change. There are two ways to create a new list in Scala:

\begin{itemize}
  \item \code{List([expression], …, [expression])}
  \item \code{[expression] :: … :: [expression] :: Nil}
\end{itemize}

The type of a list whose elements have type \code{T} is \code{List[T]}.

\begin{verbatim}
scala> List(1, 2, 3)
val res12: List[Int] = List(1, 2, 3)

scala> 1 :: 2 :: 3 :: Nil
val res13: List[Int] = List(1, 2, 3)
\end{verbatim}

\code{List(…)} is more convenient than \code{::} for creating a new list from
scratch. However, \code{::} is more flexible since it can prepend a new element
in front of an existing list.\sidenote{It does not mutate the existing list to
prepend the new element. It creates a new list with the element and the list.}

\begin{verbatim}
scala> val l = List(1, 2, 3)
val l: List[Int] = List(1, 2, 3)

scala> 0 :: l
val res14: List[Int] = List(0, 1, 2, 3)
\end{verbatim}

The \code{length} method computes the length of a list; parentheses are used
to fetch the element at a specific index.\sidenote{The first index is \code{0}.}

\begin{verbatim}
scala> l.length
val res15: Int = 3

scala> l(0)
val res16: Int = 1
\end{verbatim}

In functional programming, accessing an arbitrary element of a list by an index
is rare. We use pattern matching in most cases. The syntax of pattern matching
for a list is as follows:\sidenote{The order between the cases can vary, which
means that the \code{::} case may come first.}

\begin{verbatim}
[expression] match {
  case Nil => [expression]
  case [name] :: [name] => [expression]
}
\end{verbatim}

The expression in front of \code{match} is the target of pattern matching.
If it is an empty list, it matches \code{case Nil}. The expression
of the \code{Nil} case will be evaluated.
Otherwise, it is a nonempty list and matches \code{case [name] :: [name]}. The first
name denotes the head\sidenote{the first element} of the list, and the second
name denotes the tail\sidenote{a list consisting of all the elements except the
head} of the list. The expression of the \code{::} case will be evaluated.

The following function takes a list of integers as an argument and returns the
head. The return value is zero when the list is empty.

\begin{verbatim}
scala> def headOrZero(l: List[Int]): Int = l match {
     |   case Nil => 0
     |   case h :: t => h
     | }
def headOrZero(l: List[Int]): Int

scala> headOrZero(List(1, 2, 3))
val res17: Int = 1

scala> headOrZero(List())
val res18: Int = 0
\end{verbatim}

\refch{immutability} will show use of pattern matching for lists in recursive
functions, and \refch{pattern-matching} will discuss pattern matching in detail.

\subsection{Tuples}

A tuple contains two or more elements and maintains the order between its
elements. We use parentheses to create a new tuple:

\begin{verbatim}
([expression], …, [expression])
\end{verbatim}

The type of a tuple whose elements have types from \code{T1} to \code{Tn}
respectively is \code{(T1, …, Tn)}. For example, the type of a tuple
whose first element is \code{Int} and second element is \code{Boolean} is
\code{(Int, Boolean)}.

\begin{verbatim}
scala> (1, true)
val res19: (Int, Boolean) = (1,true)
\end{verbatim}

To fetch the \code{i}-th element of a tuple, we can use \verb+._i+.\sidenote{The
first index is 1.}

\begin{verbatim}
scala> (1, true)._1
val res20: Int = 1
\end{verbatim}

Tuples look similar to lists but have important differences from
lists. First, a tuple's elements can have different types, while a list's
elements cannot. For example, a tuple of the type \code{(Int, Boolean)} has
one integer and one boolean, while a list of the type \code{List[Int]} can have
only integers. We say that tuples are heterogenous, while lists are homogeneous.
Second, a list allows accessing an arbitrary index of a list,
while a tuple does not. For example, \code{l(f())} is possible where \code{l} is
a list and \code{f} returns an integer, while there is no way to access the
\code{f()}-th element of a tuple since the return value of \code{f} is unknown
before execution.

We use lists and tuples for different purposes. Lists are appropriate when the
number of elements can vary and an arbitrary index should be accessible.
For instance, a list should be used to represent a collection of the heights of
students in a certain class.

\begin{verbatim}
List(189, 167, 156, 170, 183)
\end{verbatim}

It allows us to fetch the height of the \code{i}-th student.

On the other hand, tuples are appropriate when the number of elements
are fixed and each index has a specific meaning. For instance,
a tuple can represent the information of a single student, where the information
consists of one's name, one's height, and whether one has payed the school
expense or not.

\begin{verbatim}
("John Doe", 173, true)
\end{verbatim}

We can use \verb+._1+ to find the name, \verb+._2+ to find the height, and
\verb+._3+ to check whether one has payed.

We call a length-2 tuple a pair and a length-3 tuple a triple. Also, we can
consider unit as a length-0 tuple.

\subsection{Maps}

A map is a collection of pairs, where each pair consists of a key and a value.
Its provides the corresponding value when a key is given.
Maps in Scala are immutable as well. Below is the syntax to create a new map:

\begin{verbatim}
Map([expression] -> [expression], …)
\end{verbatim}

The type of a map whose keys have type \code{T} and values have type \code{S} is
\code{Map[T, S]}.

\begin{verbatim}
scala> val m = Map(1 -> "one", 2 -> "two", 3 -> "three")
val m: Map[Int,String] = Map(1 -> one, 2 -> two, 3 -> three)
\end{verbatim}

To find the value corresponding to a certain key, we use parentheses.

\begin{verbatim}
scala> m(2)
val res21: String = two
\end{verbatim}

Maps provide various
methods.\sidenote{\url{https://www.scala-lang.org/api/current/scala/collection/immutable/Map.html}}

\subsection{Classes and Objects}

An object is a value with fields and methods. Fields store values, and methods
are operations related to the object. A class is a blueprint of objects. We can
easily create multiple objects of the same structure by defining a single class.
This book uses only ``case'' classes of Scala. Case classes are similar to
classes but more convenient, e.g. automatic support for pretty printing and
pattern matching.

The syntax of a class definition is as follows:

\begin{verbatim}
case class [name]([name]: [type], …)
\end{verbatim}

The first name is the name of a new class. The names inside
the parentheses are the names of the fields of the class. A class definition
must specify the types of its fields.

\begin{verbatim}
scala> case class Student(name: String, height: Int)
class Student
\end{verbatim}

Creating new objects is similar to a function call.

\begin{verbatim}
scala> val s = Student("John Doe", 173)
val s: Student = Student(John Doe,173)
\end{verbatim}

Fields can be accessed by \code{.[name]}.

\begin{verbatim}
scala> s.name
val res22: String = John Doe
\end{verbatim}

Objects in Scala are immutable by default. If we add \code{var} to a field when
defining a class, the field becomes mutable.

\begin{verbatim}
scala> case class Student(name: String, var height: Int)
class Student

scala> val s = Student("John Doe", 173)
val s: Student = Student(John Doe,173)

scala> s.height = 180
// mutated s.height

scala> s.height
val res23: Int = 180
\end{verbatim}

\section{Interpreter}
\labsec{scala-interpreter}

An \textit{interpreter}\index{interpreter} is a program that takes source code as input and runs the code.
The Scala interpreter takes Scala source code as input. To use the interpreter,
we need to save source code into a file. Make a file with the following code, and
save it as \code{Hello.scala}.

\begin{verbatim}
println("Hello world!")
\end{verbatim}

You can excute the interpreter by typing \code{scala} with the name of a file in
your command line. Here, we need to say \code{scala Hello.scala}.

\begin{verbatim}
$ scala Hello.scala
Hello world!
\end{verbatim}

You can write multiple lines in a single file. Fix \code{Hello.scala} like
below.

\begin{verbatim}
val x = 2
println(x)
val y = x * x
println(y)
\end{verbatim}

Then, excute the interpreter again.

\begin{verbatim}
$ scala Hello.scala
2
4
\end{verbatim}

\section{Compiler}
\labsec{scala-compiler}

A \textit{compiler}\index{compiler} is a program that takes source code as input and translates it into
another language. Usually, the target language is a low-level language like
machine code or bytecode of a particular virtual machine.
The Scala compiler takes Scala source code as input and translates it into Java
bytecode. Once code is compiled, we can run the generated bytecode with the JVM.

For compilation, we need to define the main method of a program. The main method
is the entrypoint of every program running on the JVM.
Make a file with the following code, and save it as \code{Hello.scala}.

\begin{verbatim}
object Hello {
  def main(args: Array[String]): Unit = {
    println("Hello world!")
  }
}
\end{verbatim}

You can make the compiler compile the code by typing \code{scalac} with the name
of the file in your command line.

\begin{verbatim}
$ scalac Hello.scala
\end{verbatim}

After compilation, you will be able to find the \code{Hello.class} file
in the same directory. The file contains Java bytecode.

You can run the bytecode with the JVM by the \code{scala} command. In this time,
you should write only the class name.

\begin{verbatim}
$ scala Hello
Hello world!
\end{verbatim}

You can change the behavior of a program by modifying the main method.
Each time you modify, you need to re-compile the program to re-generate the
bytecode.

Running bytecode is much more efficient than interpreting Scala source code.
You can easily notice that \code{scala Hello} takes much less than \code{scala
Hello.scala} even though their results are the same.

Scala has two sorts of errors: compile-time errors and run-time errors.
Compile-time errors occur during compilation, i.e. while running \code{scalac}.
If the compiler finds things that
might go wrong at run time, it raises errors and aborts the compilation. For
example, an expression adding an integer to a boolean results in a compile-time
error because such an addition cannot succeed at run time.

\begin{verbatim}
true + 1
\end{verbatim}
\vspace{-1em}
\begin{mdframed}[hidealllines=true,backgroundcolor=red!10,innerleftmargin=3pt,innerrightmargin=3pt,leftmargin=-3pt,rightmargin=-3pt]
\begin{verbatim}
error: type mismatch;
 found   : Int(1)
 required: String
true + 1
     ^
\end{verbatim}
\vspace{-2em}
\begin{flushright}
\scriptsize\textsf{Compile-time error}
\end{flushright}
\end{mdframed}

Unfortunately, some bad behaviors cannot be detected by the compiler.
The compiler does not generate any errors for those behaviors.
Such problems will incur run-time errors during execution, i.e. while running
\code{scala},
and terminate the execution abnormally. Division by zero is one
example of run time errors.

\begin{verbatim}
1 / 0
\end{verbatim}
\vspace{-1em}
\begin{mdframed}[hidealllines=true,backgroundcolor=red!10,innerleftmargin=3pt,innerrightmargin=3pt,leftmargin=-3pt,rightmargin=-3pt]
\begin{verbatim}
java.lang.ArithmeticException: / by zero
\end{verbatim}
\vspace{-2em}
\begin{flushright}
\scriptsize\textsf{Run-time error}
\end{flushright}
\end{mdframed}

\section{SBT}
\labsec{sbt}

SBT is a build tool for Scala. Build tools help programmers work on large
projects with many files and libraries by tracking dependencies between files
and managing libraries. There are various build tools in the world, and SBT is
the most popular one for Scala.

You can create a new Scala project by the \code{sbt new} command.

\begin{verbatim}
$ sbt new scala/scala-seed.g8
[info] welcome to sbt 1.4.7
[info] loading global plugins from ~/.sbt/1.0/plugins
[info] set current project to ~/ (in build file:~/)
[info] set current project to ~/ (in build file:~/)


A minimal Scala project.

name [Scala Seed Project]: hello

Template applied in ~/hello
\end{verbatim}

After the creation, the directory structure is as follows:

\dirtree{%
  .1 hello.
  .2 build.sbt.
  .2 project.
  .3 Dependencies.scala.
  .3 build.properties.
  .2 src.
  .3 main.
  .4 scala.
  .5 example.
  .6 Hello.scala.
  .3 test.
  .4 scala.
  .5 example.
  .6 HelloSpec.scala.
}

The \code{build.sbt} file configures the project. It manages the version of Scala
used for the project, third-party libraries used in the project, and many other
things.
Source files are in the \code{src} directory. Files in \code{main} are
main source files, while files in \code{test} are only for testing.
You can add files into the \code{src/main/scala} directory and edit them to write code.

An SBT console can be started by the \code{sbt} command. The current working
directory of your shell should be the base directory of the project.

\begin{verbatim}
$ sbt
[info] welcome to sbt 1.4.7
[info] loading global plugins from ~/.sbt/1.0/plugins
[info] loading project definition from ~/hello/project
[info] loading settings for project root from build.sbt ...
[info] set current project to hello (in build file:~/hello/)

[info] sbt server started at
local:///~/.sbt/1.0/server/d4cd702f998423203dfe/sock
[info] started sbt server
sbt:hello>
\end{verbatim}

You can compile, run, and test the project by executing SBT commands in the
console.

\begin{itemize}
  \item \code{compile}: compile the project.
  \item \code{run}: run the project (re-compile if necessary).
  \item \code{test}: test the project (re-compile if necessary).
  \item \code{exit}: terminate the console.
\end{itemize}

\begin{verbatim}
sbt:hello> compile
[info] compiling 1 Scala source to ~/hello/target/scala-2.13

| => root / Compile / compileIncremental 0s
[success] Total time: 4 s
sbt:hello> test
[info] compiling 1 Scala source to ~/hello/target/scala-2.13
[info] HelloSpec:
[info] The Hello object
[info] - should say hello
[info] Run completed in 455 milliseconds.
[info] Total number of tests run: 1
[info] Suites: completed 1, aborted 0
[info] Tests: succeeded 1, failed 0, canceled 0, ignored 0
[info] All tests passed.
[success] Total time: 2 s
sbt:hello> run
[info] running example.Hello
hello
[success] Total time: 0 s
sbt:hello> exit
[info] shutting down sbt server
\end{verbatim}

To learn SBT more, refer to the SBT website.\sidenote{\url{https://www.scala-sbt.org/learn.html}}


%----------------------------------------------------------------------------------------

\backmatter % Denotes the end of the main document content
\setchapterstyle{plain} % Output plain chapters from this point onwards

%----------------------------------------------------------------------------------------
%  BIBLIOGRAPHY
%----------------------------------------------------------------------------------------

% The bibliography needs to be compiled with biber using your LaTeX editor, or on the command line with 'biber main' from the template directory

\defbibnote{bibnote}{Here are the references in citation order.\par\bigskip} % Prepend this text to the bibliography
\printbibliography[heading=bibintoc, title=Bibliography, prenote=bibnote] % Add the bibliography heading to the ToC, set the title of the bibliography and output the bibliography note

%----------------------------------------------------------------------------------------
%  NOMENCLATURE
%----------------------------------------------------------------------------------------

% The nomenclature needs to be compiled on the command line with 'makeindex main.nlo -s nomencl.ist -o main.nls' from the template directory

% \nomenclature{$c$}{Speed of light in a vacuum inertial frame}
% \nomenclature{$h$}{Planck constant}

% \renewcommand{\nomname}{Notation} % Rename the default 'Nomenclature'
% \renewcommand{\nompreamble}{The next list describes several symbols that will be later used within the body of the document.} % Prepend this text to the nomenclature

% \printnomenclature % Output the nomenclature

%----------------------------------------------------------------------------------------
%  GREEK ALPHABET
%   Originally from https://gitlab.com/jim.hefferon/linear-algebra
%----------------------------------------------------------------------------------------

% \vspace{1cm}

% {\usekomafont{chapter}Greek Letters with Pronunciations} \\[2ex]
% \begin{center}
%   \newcommand{\pronounced}[1]{\hspace*{.2em}\small\textit{#1}}
%   \begin{tabular}{l l @{\hspace*{3em}} l l}
%     \toprule
%     Character & Name & Character & Name \\
%     \midrule
%     $\alpha$ & alpha \pronounced{AL-fuh} & $\nu$ & nu \pronounced{NEW} \\
%     $\beta$ & beta \pronounced{BAY-tuh} & $\xi$, $\Xi$ & xi \pronounced{KSIGH} \\
%     $\gamma$, $\Gamma$ & gamma \pronounced{GAM-muh} & o & omicron \pronounced{OM-uh-CRON} \\
%     $\delta$, $\Delta$ & delta \pronounced{DEL-tuh} & $\pi$, $\Pi$ & pi \pronounced{PIE} \\
%     $\epsilon$ & epsilon \pronounced{EP-suh-lon} & $\rho$ & rho \pronounced{ROW} \\
%     $\zeta$ & zeta \pronounced{ZAY-tuh} & $\sigma$, $\Sigma$ & sigma \pronounced{SIG-muh} \\
%     $\eta$ & eta \pronounced{AY-tuh} & $\tau$ & tau \pronounced{TOW (as in cow)} \\
%     $\theta$, $\Theta$ & theta \pronounced{THAY-tuh} & $\upsilon$, $\Upsilon$ & upsilon \pronounced{OOP-suh-LON} \\
%     $\iota$ & iota \pronounced{eye-OH-tuh} & $\phi$, $\Phi$ & phi \pronounced{FEE, or FI (as in hi)} \\
%     $\kappa$ & kappa \pronounced{KAP-uh} & $\chi$ & chi \pronounced{KI (as in hi)} \\
%     $\lambda$, $\Lambda$ & lambda \pronounced{LAM-duh} & $\psi$, $\Psi$ & psi \pronounced{SIGH, or PSIGH} \\
%     $\mu$ & mu \pronounced{MEW} & $\omega$, $\Omega$ & omega \pronounced{oh-MAY-guh} \\
%     \bottomrule
%   \end{tabular} \\[1.5ex]
%   Capitals shown are the ones that differ from Roman capitals.
% \end{center}

%----------------------------------------------------------------------------------------
%  GLOSSARY
%----------------------------------------------------------------------------------------

% The glossary needs to be compiled on the command line with 'makeglossaries main' from the template directory

\newacronym[longplural={algebraic data types}]{adtLabel}{ADT}{algebraic data type}
\newacronym[longplural={real-eval-print loop}]{replLabel}{REPL}{read-eval-print loop}
\newacronym[longplural={Java Virtual Machines}]{jvmLabel}{JVM}{Java Virtual Machine}
\newacronym[longplural={Backus-Naur forms}]{bnfLabel}{BNF}{Backus-Naur form}
\newacronym[longplural={abstract syntax trees}]{astLabel}{AST}{abstract syntax tree}
\newacronym[longplural={call-by-value}]{cbvLabel}{CBV}{call-by-value}
\newacronym[longplural={call-by-reference}]{cbrLabel}{CBR}{call-by-reference}
\newacronym[longplural={call-by-name}]{cbnLabel}{CBN}{call-by-name}
\newacronym[longplural={continuation-passing style}]{cpsLabel}{CPS}{continuation-passing style}
\newacronym[longplural={garbage collection}]{gcLabel}{GC}{garbage collection}
\newacronym[longplural={last in, first out}]{lifoLabel}{LIFO}{last in, first out}
\newacronym[longplural={use-after-free}]{uafLabel}{UAF}{use-after-free}

\setglossarystyle{listgroup} % Set the style of the glossary (see https://en.wikibooks.org/wiki/LaTeX/Glossary for a reference)
\printglossary[title=Special Terms, toctitle=List of Terms] % Output the glossary, 'title' is the chapter heading for the glossary, toctitle is the table of contents heading

%----------------------------------------------------------------------------------------
%  INDEX
%----------------------------------------------------------------------------------------

% The index needs to be compiled on the command line with 'makeindex main' from the template directory

\printindex % Output the index

%----------------------------------------------------------------------------------------
%  BACK COVER
%----------------------------------------------------------------------------------------

% If you have a PDF/image file that you want to use as a back cover, uncomment the following lines

%\clearpage
%\thispagestyle{empty}
%\null%
%\clearpage
%\includepdf{cover-back.pdf}

%----------------------------------------------------------------------------------------

\end{document}
