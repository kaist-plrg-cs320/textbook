%%%%%%%%%%%%%%%%%%%%%%%%%%%%%%%%%%%%%%%%%
% kaobook
% LaTeX Template
% Version 1.3 (December 9, 2021)
%
% This template originates from:
% https://www.LaTeXTemplates.com
%
% For the latest template development version and to make contributions:
% https://github.com/fmarotta/kaobook
%
% Authors:
% Federico Marotta (federicomarotta@mail.com)
% Based on the doctoral thesis of Ken Arroyo Ohori (https://3d.bk.tudelft.nl/ken/en)
% and on the Tufte-LaTeX class.
% Modified for LaTeX Templates by Vel (vel@latextemplates.com)
%
% License:
% CC0 1.0 Universal (see included MANIFEST.md file)
%
%%%%%%%%%%%%%%%%%%%%%%%%%%%%%%%%%%%%%%%%%

%----------------------------------------------------------------------------------------
%  PACKAGES AND OTHER DOCUMENT CONFIGURATIONS
%----------------------------------------------------------------------------------------

\documentclass[
  b5paper, % Page size
  fontsize=10pt, % Base font size
  twoside=true, % Use different layouts for even and odd pages (in particular, if twoside=true, the margin column will be always on the outside)
  %open=any, % If twoside=true, uncomment this to force new chapters to start on any page, not only on right (odd) pages
  %chapterentrydots=true, % Uncomment to output dots from the chapter name to the page number in the table of contents
  numbers=noenddot, % Comment to output dots after chapter numbers; the most common values for this option are: enddot, noenddot and auto (see the KOMAScript documentation for an in-depth explanation)
]{kaobook}

\renewcommand{\marginlayout}{%
  \newgeometry{
    top=27.4\vscale,
    bottom=27.4\vscale,
    inner=24.8\hscale,
    textwidth=147\hscale,
    marginparsep=6.2\hscale,
    marginparwidth=0\hscale,
  }%
  \recalchead
}

% Choose the language
\ifxetexorluatex
  \usepackage{polyglossia}
  \setmainlanguage{english}
\else
  \usepackage[english]{babel} % Load characters and hyphenation
\fi
\usepackage[english=british]{csquotes}  % English quotes

% Load packages for testing
\usepackage{blindtext}
%\usepackage{showframe} % Uncomment to show boxes around the text area, margin, header and footer
%\usepackage{showlabels} % Uncomment to output the content of \label commands to the document where they are used

% Load the bibliography package
\usepackage{kaobiblio}
\addbibresource{main.bib} % Bibliography file

% Load mathematical packages for theorems and related environments
\usepackage[framed=true]{kaotheorems}

% Load the package for hyperreferences
\usepackage{kaorefs}

\usepackage{mathpartir}
\usepackage{dirtree}
\usepackage{xspace}
\usepackage{tikz}
\usepackage{fancyvrb}
\usepackage{bussproofs}
\usepackage{proof}
\usepackage{upquote}
\usepackage{arydshln}
\usepackage{graphicx}
\usepackage{amsthm}

\usetikzlibrary{arrows}
\tikzset{
  treenode/.style = {align=center, inner sep=0pt, text centered},
  arn_u/.style = {treenode, circle, black, draw=black, text width=0.8cm, line width=0.25mm},
  arn_w/.style = {treenode, circle, white, draw=white, text width=0.8cm, line width=0.25mm},
  arn_nc/.style = {treenode, circle, black, draw=white, text width=0.8cm, line width=0.25mm},
}

\newcommand{\shorteq}{\code{=}}

\newcommand{\Lang}{}
\newcommand{\plang}{}
\newcommand{\code}[1]{\texttt{#1}}
\newcommand{\cx}{\code{x}}
\newcommand{\cy}{\code{y}}
\newcommand{\cz}{\code{z}}
\newcommand{\cf}{\code{f}}
\newcommand{\cv}{\code{v}}
\newcommand{\ca}{\code{a}}
\newcommand{\cb}{\code{b}}
\newcommand{\cX}{\code{X}}
\newcommand{\cY}{\code{Y}}
\newcommand{\cZ}{\code{Z}}
\newcommand{\inred}{\color{red}}

\newcommand{\chnum}[1]{\hyperref[ch:#1]{\ref{ch:#1}}}
\newcommand{\refchs}[2]{Chapters \chnum{#1} and \chnum{#2}}
\newcommand{\refchss}[3]{Chapters \chnum{#1}, \chnum{#2}, and \chnum{#3}}
\newcommand{\refex}[1]{Exercise~\hyperref[thm:#1]{\ref{thm:#1}}}
\newcommand{\labex}[1]{\labthm{#1}}
\newcommand{\pto}{\mathrel{\ooalign{\hfil$\mapstochar$\hfil\cr$\to$\cr}}}
\newcommand{\finto}{\stackrel{\mbox{\tiny fin}}{\pto}}
\newcommand{\embox}[1]{\mbox{\emph{#1}}}
\newcommand{\dom}[1]{\embox{Domain}(#1)}
\newcommand{\iadd}{+_{{}_{\mathbb{Z}}}}
\newcommand{\isub}{-_{{}_{\mathbb{Z}}}}

\newcommand{\fundef}[3]{\textsf{def}\ #1(#2)\shorteq#3}
\newcommand{\clov}[3]{\langle\efun{#1}{#2},#3\rangle}
\newcommand{\tclov}[3]{\langle\etfun{#1}{#2},#3\rangle}
\newcommand{\contv}[2]{\langle#1,#2\rangle}
\newcommand{\exprv}[2]{(#1,#2)}
\newcommand{\eadd}[2]{#1+#2}
\newcommand{\esub}[2]{#1-#2}
\newcommand{\ebind}[3]{\textsf{val}\ #1\shorteq#2\ \textsf{in}\ #3}
\newcommand{\evcc}[2]{\textsf{vcc}\ #1\ \textsf{in}\ #2}
\newcommand{\efun}[2]{\lambda#1.#2}
\newcommand{\efunt}[3]{\lambda#1{:}#2.#3}
\newcommand{\eapp}[2]{#1\ #2}
\newcommand{\eappfo}[2]{#1(#2)}
\newcommand{\erec}[4]{\textsf{def}\ #1(#2)\shorteq#3\ \textsf{in}\ #4}
\newcommand{\eifz}[3]{\textsf{if0}\ #1\ #2\ #3}
\newcommand{\eif}[3]{\textsf{if}\ #1\ #2\ #3}
\newcommand{\eskip}{\textsf{skip}}
\newcommand{\ewhilez}[2]{\textsf{while0}\ #1\ #2}
\newcommand{\ewhile}[2]{\textsf{while}\ #1\ #2}
\newcommand{\eref}[1]{\textsf{box}\ #1}
\newcommand{\ederef}[1]{!#1}
\newcommand{\eset}[2]{#1{:}\shorteq#2}
\newcommand{\eseq}[2]{#1;#2}
\newcommand{\erect}[6]{\textsf{def}\ #1(#2{:}#3){:}#4\shorteq#5\ \textsf{in}\ #6}
\newcommand{\etdef}[6]{\textsf{type}\ #1=#2@#3+#4@#5\ \textsf{in}\ #6}
\newcommand{\ematch}[7]{#1\ \textsf{match}\ #2(#3)\rightarrow #4,#5(#6)\rightarrow #7}
\newcommand{\etfun}[2]{\Lambda#1.#2}
\newcommand{\etapp}[2]{#1[#2]}
\newcommand{\true}{\textsf{true}}
\newcommand{\false}{\textsf{false}}

\newcommand{\tnum}{\textsf{num}}
\newcommand{\tbool}{\textsf{bool}}
\newcommand{\tfun}{\textsf{fun}}
\newcommand{\tarrow}[2]{#1\rightarrow#2}
\newcommand{\tforall}[2]{\forall#1.#2}
\newcommand{\ttop}{\textsf{top}}
\newcommand{\tbot}{\textsf{bottom}}

\newcommand{\mtk}{\square}
\newcommand{\evalk}[2]{#1\vdash#2::}
\newcommand{\evalkd}[1]{\evalk{\sigma}{#1}}
\newcommand{\evalke}[1]{\evalk{\emptyset}{#1}}
\newcommand{\addk}{(+)::}
\newcommand{\subk}{(-)::}
\newcommand{\appk}{(@)::}

\newcommand{\mts}{\blacksquare}
\newcommand{\conss}[1]{#1::}

\newcommand{\eval}[3]{#1\vdash#2\Rightarrow#3}
\newcommand{\evaln}[3]{\ensuremath{#2} evaluates to \ensuremath{#3} under
\ensuremath{#1}\xspace}
\newcommand{\evald}[2]{\eval{\sigma}{#1}{#2}}
\newcommand{\evaldn}[2]{\evaln{\sigma}{#1}{#2}}
\newcommand{\evale}[2]{\eval{\emptyset}{#1}{#2}}

\newcommand{\typeof}[3]{#1\vdash#2:#3}
\newcommand{\typeofd}[2]{\typeof{\Gamma}{#1}{#2}}
\newcommand{\typeofe}[2]{\typeof{\emptyset}{#1}{#2}}
\newcommand{\typeofn}[3]{the type of \ensuremath{#2} is
\ensuremath{#3} under \ensuremath{#1}\xspace}
\newcommand{\typeofdn}[2]{\typeofn{\Gamma}{#1}{#2}}
\newcommand{\typeofnc}[3]{The type of \ensuremath{#2} is
\ensuremath{#3} under \ensuremath{#1}\xspace}
\newcommand{\typeofdnc}[2]{\typeofnc{\Gamma}{#1}{#2}}

\newcommand{\wft}[2]{#1\vdash#2}
\newcommand{\wftd}[1]{\wft{\Gamma}{#1}}
\newcommand{\wftn}[2]{\ensuremath{#2} is well-formed under \ensuremath{#1}}
\newcommand{\wftdn}[1]{\wftn{\Gamma}{#1}}

\newcommand{\subt}[2]{#1<:#2}
\newcommand{\subtn}[2]{\ensuremath{#1} is a subtype of \ensuremath{#2}}

\newcommand{\stricte}[2]{#1\Downarrow#2}
\newcommand{\stricten}[2]{\ensuremath{#1} strictly evaluates to \ensuremath{#2}}

\newcommand{\seval}[5]{#1,#2\vdash#3\Rightarrow#4,#5}
\newcommand{\sevaln}[5]{
  \ensuremath{#3} evaluates to \ensuremath{#4} and changes the store from
  \ensuremath{#2} to \ensuremath{#5} under \ensuremath{#1}\xspace
}
\newcommand{\sevald}[4]{\seval{\sigma}{M_{#1}}{#2}{#3}{M_{#4}}}
\newcommand{\sevaldn}[4]{\sevaln{\sigma}{M_{#1}}{#2}{#3}{M_{#4}}}
\newcommand{\sevale}[3]{\seval{\emptyset}{\emptyset}{#1}{#2}{#3}}

\newcommand{\semanticrule}[2]{
\vspace{1em}
\textbf{Rule \textsc{#1}}\\
#2
\vspace{1em}
}
\newcommand{\typerule}[2]{
\vspace{1em}
\textbf{Rule \textsc{#1}}\\
#2
\vspace{1em}
}
\newcommand{\tand}{\text{ and }}

\graphicspath{{images/}} % Paths in which to look for images

\makeindex[columns=3, title=Alphabetical Index, intoc] % Make LaTeX produce the files required to compile the index

\makeglossaries % Make LaTeX produce the files required to compile the glossary
% \input{glossary.tex} % Include the glossary definitions

% \makenomenclature % Make LaTeX produce the files required to compile the nomenclature

% Reset sidenote counter at chapters
%\counterwithin*{sidenote}{chapter}

%----------------------------------------------------------------------------------------

\begin{document}

%----------------------------------------------------------------------------------------
%  BOOK INFORMATION
%----------------------------------------------------------------------------------------

% \titlehead{The \texttt{kaobook} class}
% \subject{Use this document as a template}

% \title[Example and documentation of the {\normalfont\texttt{kaobook}} class]{Example and documentation \\ of the {\normalfont\texttt{kaobook}} class}
% \subtitle{Customise this page according to your needs}
\title{Introduction to Programming Languages}

% \author[Federico Marotta]{Federico Marotta\thanks{A \LaTeX\ lover}}
\author{Jaemin Hong and Sukyoung Ryu}

% \date{\today}
\date{}

% \publishers{An Awesome Publisher}

%----------------------------------------------------------------------------------------

\frontmatter % Denotes the start of the pre-document content, uses roman numerals

%----------------------------------------------------------------------------------------
%  OPENING PAGE
%----------------------------------------------------------------------------------------

%\makeatletter
%\extratitle{
%  % In the title page, the title is vspaced by 9.5\baselineskip
%  \vspace*{9\baselineskip}
%  \vspace*{\parskip}
%  \begin{center}
%    % In the title page, \huge is set after the komafont for title
%    \usekomafont{title}\huge\@title
%  \end{center}
%}
%\makeatother

%----------------------------------------------------------------------------------------
%  COPYRIGHT PAGE
%----------------------------------------------------------------------------------------

\makeatletter
\uppertitleback{\@titlehead} % Header

% \lowertitleback{
%   \textbf{Disclaimer}\\
%   You can edit this page to suit your needs. For instance, here we have a no copyright statement, a colophon and some other information. This page is based on the corresponding page of Ken Arroyo Ohori's thesis, with minimal changes.

%   \medskip

%   \textbf{No copyright}\\
%   \cczero\ This book is released into the public domain using the CC0 code. To the extent possible under law, I waive all copyright and related or neighbouring rights to this work.

%   To view a copy of the CC0 code, visit: \\\url{http://creativecommons.org/publicdomain/zero/1.0/}

%   \medskip

%   \textbf{Colophon} \\
%   This document was typeset with the help of \href{https://sourceforge.net/projects/koma-script/}{\KOMAScript} and \href{https://www.latex-project.org/}{\LaTeX} using the \href{https://github.com/fmarotta/kaobook/}{kaobook} class.

%   The source code of this book is available at:\\\url{https://github.com/fmarotta/kaobook}

%   (You are welcome to contribute!)

%   \medskip

%   \textbf{Publisher} \\
%   First printed in May 2019 by \@publishers
% }
\lowertitleback{
  \copyright2022 Jaemin Hong and Sukyoung Ryu

  \medskip

  All rights reserved. No part of this book may be reproduced in any form by any
  electronic of mechanical means (including photocopying, recording, or
  information storage and retrieval) without permission in writing from the
  authors.
}
\makeatother

%----------------------------------------------------------------------------------------
%  DEDICATION
%----------------------------------------------------------------------------------------

% \dedication{
%   The harmony of the world is made manifest in Form and Number, and the heart and soul and all the poetry of Natural Philosophy are embodied in the concept of mathematical beauty.\\
%   \flushright -- D'Arcy Wentworth Thompson
% }

%----------------------------------------------------------------------------------------
%  OUTPUT TITLE PAGE AND PREVIOUS
%----------------------------------------------------------------------------------------

% Note that \maketitle outputs the pages before here

\maketitle

%----------------------------------------------------------------------------------------
%  PREFACE
%----------------------------------------------------------------------------------------

\setchapterpreamble[u]{\margintoc}
\chapter{Acknowledgement}
\labch{acknowledgement}

The contents of this book are based on the KAIST \textit{Programming Languages}
course. We thank PLT\sidenote{\url{https://racket-lang.org/people.html}} since
the course referred to many materials from PLT in the beginning.
We also thank every student who took the
course before. We have learned many things from the interaction with the
students, and those lessons have affected various parts of the book. In
addition, we thank all the previous and current teaching assistants of the
course. They gave opinions to the course and wrote some of the exercises in the
book. Especially, Jihyeok Park highly contributed to the course, and Jihee Park
helped us edit the exercises.

We would be delighted to receive comments and corrections, which may be sent to
\code{jaemin.hong@kaist.ac.kr}. We thank in advance everyone who will contribute
to the book in the future.


%----------------------------------------------------------------------------------------
%  TABLE OF CONTENTS & LIST OF FIGURES/TABLES
%----------------------------------------------------------------------------------------

\begingroup % Local scope for the following commands

% Define the style for the TOC, LOF, and LOT
%\setstretch{1} % Uncomment to modify line spacing in the ToC
%\hypersetup{linkcolor=blue} % Uncomment to set the colour of links in the ToC
\setlength{\textheight}{230\hscale} % Manually adjust the height of the ToC pages

% Turn on compatibility mode for the etoc package
\etocstandarddisplaystyle % "toc display" as if etoc was not loaded
\etocstandardlines % "toc lines" as if etoc was not loaded

\tableofcontents % Output the table of contents

% \listoffigures % Output the list of figures

% Comment both of the following lines to have the LOF and the LOT on different pages
% \let\cleardoublepage\bigskip
% \let\clearpage\bigskip

% \listoftables % Output the list of tables

\endgroup

%----------------------------------------------------------------------------------------
%  MAIN BODY
%----------------------------------------------------------------------------------------

\mainmatter % Denotes the start of the main document content, resets page numbering and uses arabic numbers
\setchapterstyle{kao} % Choose the default chapter heading style

\setchapterpreamble[u]{\margintoc}
\chapter{Introduction}
\labch{introduction}

What is a programming language?

The simplest answer is ``it is a language used for programming.'' However, this
answer does not help us understand programming languages. We need a better
question to get a better answer.

What does a programming language consist of?

There is a good answer for this question: ``in a narrow sense, a programming
language consists of syntax and semantics, and in a broad sense, it additionally
has a standard library and an ecosystem.''

Syntax and semantics are principal concepts to understand programming languages.
Syntax determines how a language looks like, and semantics fills the inside. If
we consider a programming language as a human, we can say that syntax is one’s
appearance, and semantics is one’s thoughts. Programmers write programs
according to syntax. Syntax decides characters used in source code. Once programs
are written, semantics decides what each program does. Without semantics, all
the programs are useless. Programs can work as being expected only after
semantics determines the meaning of them. A programming language with syntax and
semantics is complete. Programmers using that language can write programs with
the syntax and execute the programs with the semantics. From a theoretical
perspective, syntax and semantics are all of a programming language.

For programmers, syntax and semantics are not the only elements of a programming
language. First, the standard library of a language is another element. The
standard library provides various utilities required by applications: data
structures like lists and maps, functions handling file and network IO, and so
on. The standard library is like clothes for humans. A human without clothes is
a human; a programming language without a standard library is a programming
language. At the same time, clothes are important to humans as they make bodies
warm and protect bodies from dangerous objects. Similarly, a standard library is
important to a programming language as it supplies diverse functionalities for
applications. Each person wears clothes different from others, and each language
puts different things from other languages in its standard library. Some
languages include lots of utilities in their standard libraries, while others
include much less. Some languages treat lists and maps as built-in concepts in
their semantics, while others define them with other primitives in their standard libraries.
Programmers avoid using a language without a standard library because such a
language increases the effort to write programs.

Another important element to programmers is the ecosystem of a programming
language. The ecosystem includes everything related to the language: developers
and companies using the language, third-party libraries written in the language,
and so on. It is like a society for humans. If many programmers and companies
use a programming language, one can easily get help and find complementary
materials by using the same language. There will be more chances of cooperative
work and employment, too. Third-party libraries also take important roles in
software development. The standard library offers only general facilities and
often lacks domain-specific features. When a required functionality cannot be
found in the standard library, a third-party library can provide the exact
functionality. For these reasons, the ecosystem of a programming language is
important to programmers.

Practically, the standard library and the ecosystem of a language are important
elements. Unlike syntax and semantics, they are not essential. A programming
language can exist even without its standard library and ecosystem. However,
developers take standard libraries and ecosystems into account as well as syntax and
semantics to choose languages they use. From a practical perspective, a
programming language consists of syntax, semantics, a standard library, and an
ecosystem.

This book is not for helping readers use a specific programming language. It
does not recommend a specific programming language, either. This book helps
readers learn new programming languages easily. You can acquaint any programming
languages once you completely read and understand this book. Obviously, this
goal cannot be achieved if the book discusses various languages separately. It
is possible only by discussing the underlying principles of every programming
language.

The principles of programming languages can be found from their semantics. Each
language seems very different from the others, but it is actually not the case.
Precisely speaking, their insides are quite the same, while their appearances
look different. They look different because their syntax and standard libraries,
which determine the appearances, are different. However, their insides, the
semantics, fundamentally share the same mathematical principles. If you
understand essential concepts residing in the semantics of multiple languages,
it is easy to understand and learn new languages.

People who know the key principles and can separate the elements of a language
can easily learn programming languages. As an analogy, consider a man learning
how to use a computer. It is a big problem if he cannot distinguish a keyboard
from a computer. For example, he thinks ``to say hello, my right index finger
presses the keyboard, my left middle finger presses the keyboard, my right ring
finger presses the keyboard three times.'' If the layout of the keyboard changes,
he should learn the whole computer again. On the other hand, if he knows that a
keyboard is just a tool to input text, he will less suffer from the change of
the keyboard layout. As he thinks ``to say hello, I press H, E, L, L, and O,'' he
does not need to learn the whole computer again. Of course, he should learn the
new keyboard layout, but it will be much easier. In addition, it is
straightforward to apply his knowledge to do new things. For example, he will
easily figure out ``to say lol, I press L, O, and L.'' If he does not distinguish
a keyboard from a computer, he cannot find any common principles between saying
hello and saying lol. Learning programming languages is the same. People who
cannot distinguish syntax and semantics believe that they should learn the whole
language again when the syntax changes. On the other hand, people who can
distinguish syntax and semantics know that semantics remains the same even if
syntax may vary. They know that understanding the principles of semantics is
important to learn languages. Becoming familiar with the
new syntax is all they need to use a new language fluently.

This book explains the semantics of principal concepts in programming languages.
\refch{introduction-to-scala}, \refch{immutability},
\refch{functions}, and \refch{pattern-matching}
introduce the Scala programming language. This book
uses Scala to implement interpreters of languages introduced in the book.
\refch{syntax-and-semantics} explains syntax and
semantics. Then, the book finally introduces various features of programming languages.
\begin{itemize}
    \item \refch{identifiers} introduces identifiers.
    \item \refch{first-order-functions},
      \refch{first-class-functions}, and \refch{lambda-calculus} introduce functions.
    \item \refch{recursion} introduces recursion.
    \item \refch{mutable-boxes} and \refch{mutable-variables} introduce mutation.
    \item \refch{lazy-evaluation} introduces lazy evaluation.
\end{itemize}

\section{Exercises}

\begin{enumerate}
\item Write the name of a programming language that you have used.
  What are the pros and cons of the language?
\item Write the names of two programming languages you know and compare them.
\end{enumerate}


\pagelayout{wide} % No margins
\addpart{Scala}
\pagelayout{margin} % Restore margins

\setchapterpreamble[u]{\margintoc}
\chapter{Introduction to Scala}
\labch{introduction-to-scala}

This book uses Scala as an implementation language, and
this chapter thus introduces the Scala programming language. Scala stands for a
\textbf{sca}lable \textbf{la}nguage~\cite{programming-in-scala}. It is a
multi-paradigm language that allows both functional and object-oriented styles.
This book focuses on the functional nature of Scala. In this chapter, we will
see what functional programming is and why this book uses functional
programming. In addition, we will install Scala and write simple programs in
Scala.

\section{Functional Programming}

What is \textit{functional programming}?\index{functional programming}
According to Wikipedia,

\begin{quote}
Functional programming is a programming paradigm that treats computation as the
evaluation of mathematical functions and avoids changing-state and mutable data.
\end{quote}

According to the book Functional Programming in Scala~\cite{fp-in-scala},

\begin{quote}
Functional programming (FP) is based on a simple premise with far-reaching
implications: we construct our programs using only pure functions---in other words,
functions that have no side effects.
\end{quote}

The above two sentences are enough to describe functional programming.

First, consider the phrase ``the evaluation of mathematical functions.''
From the perspective of functional programming, a
program is a single mathematical expression and the execution of the program is
finding a value denoted by the expression.
Each expression consists of zero or more subexpressions and evaluates to a value.

Let us discuss how
functional programming is different from imperative programming with
code examples.

\begin{verbatim}
int x = 1;
int y = 2;
if (y < 3)
    x = x + 4;
else
    x = x - 5;
\end{verbatim}

The above code is written in C, which represents imperative languages. Imperative
programming mimics a way in which computers operate. During the execution of a
program, a state, which can be interpreted as the memory of a computer,
exists and the execution modifies the state. The execution of the above C program has
the following steps:

\begin{enumerate}
\item A state that both \code{x} and \code{y} are uninitialized
\item A state that \code{x} is \code{1} and \code{y} is uninitialized
\item A state that \code{x} is \code{1} and \code{y} is \code{2}
\item Since \code{y < 3} is \code{true} under the state of the third step, go to
the next line.
\item A state that \code{x} is \code{5} and \code{y} is \code{2}
\end{enumerate}

The state keeps changes throughout the execution of the program. Each line
modifies the current state rather than resulting in some value.

\begin{verbatim}
let x = 1 in
let y = 2 in
if y < 3 then x + 4 else x - 5
\end{verbatim}

The above code is written in OCaml, which represents functional languages. A
program is an expression and the result of the execution is the result of
evaluating the expression. The execution does not require the notion of a state.
The execution of the above OCaml program has the following steps:

\begin{enumerate}
\item Given the fact that \code{x} equals \code{1}, evaluate
\code{let y = 2 in if y < 3 then x + 4 else x - 5}.
\item Given the fact that \code{x} equals \code{1} and \code{y} equals \code{2},
evaluate \code{if y < 3 then x + 4 else x - 5}.
\item Given the fact that \code{x} equals \code{1} and \code{y} equals \code{2},
evaluating \code{y < 3} yields \code{true}, and the next step is to evaluate
\code{x + 4}.
\item Given the fact that \code{x} equals \code{1} and \code{y} equals \code{2},
evaluate \code{x + 4}.
\item The result is \code{5}.
\end{enumerate}

There is no state. Each expression consists of subexpressions. The result of
an expression is determined by the results of its subexpressions.

Since the programs are simple, two programs look similar, but it is important to
understand two different perspectives of what a program is.

Now, look at the phrases ``avoids changing-state and mutable data'' and ``using
only pure functions.'' Functional programming avoids mutable variables, mutable data
structures, and mutable objects. The term \textit{mutable}\index{mutable}
means being able to change. Its opposite is \textit{immutable}\index{immutable},
which means not being able to change. States change throughout the execution of programs.
In functional programming, states do not exist since things never change.
Due to the lack of states, a function always does the same stuff
and always returns the same value for the same arguments. Such functions
are called pure functions.

In practice, especially for large-scale projects, using only immutable things in
the whole code is often inefficient. Most real-world functional languages
provide mutation via language constructs like \code{var} of Scala, \code{ref} of
OCaml, and \code{set!} and \code{box} of Racket. However,
functional programming uses immutable things in most cases. Even without
mutation, we can still express most programs without difficulties.

As we have seen so far,
immutability is the most important concept of functional programming.
Immutability allows modular programming and eases the reasoning of programs.
Because of immutability, programs that have to be trustworthy or require parallel computing
are good applications of functional programming.
\refch{immutability} will discuss the advantages of immutability in detail and
how to write interesting programs without mutation.

There are other important characteristics of functional programming as well as
immutability. Use of first-class functions and pattern matching also take the
key roles in functional programming. Both first-class functions and pattern matching
are valuable as they help abstraction. First-class functions allow programmers
to abstact computation; pattern matching allows programmers to abstract
data. Because of the ability of abstraction,
programs whose input has complex and abstract structures like source code
are typically written in functional languages.
\refch{functions} and \refch{pattern-matching} will respectively discuss first-class
functions and pattern matching in Scala.

This book implements interpreters and type checkers. They take source code as
input and process the input according to the mathematical semantics of programming
languages. It is important
to reason about the correctness of interpreters and type checkers. These
properties exactly match the strengths of functional programming. It is why
this book uses functional programming and Scala.

Before moving on to the next section,
let us see how people use functional programming in industry.

Akka\sidenote{\url{https://akka.io/}} is a concurrent,
distributed computing library written in Scala. Many companies have been using Akka.
Apache Spark,\sidenote{\url{https://spark.apache.org/}} a well-known library for data
processing, also is written in Scala.
Play\sidenote{\url{https://www.playframework.com/}}
is a widely-used web framework based on Akka.

Facebook has developed Infer,\sidenote{\url{https://fbinfer.com/}} a static analyzer for Java,
C, C++, and Objective-C, in OCaml. Facebook and other companies including Amazon
and Mozila use Infer to find bugs statically in their programs. Facebook has
developed also Flow,\sidenote{\url{https://flow.org/}} a static type checker for JavaScript.
Jane Street\sidenote{\url{https://www.janestreet.com/}} is a financial company well-known in
the programming language community and has developed its own software in OCaml.
According to the OCaml website,\sidenote{\url{http://ocaml.org/learn/companies.html}}
various companies including Docker use OCaml.

Haskell Wiki\sidenote{\url{http://wiki.haskell.org/Haskell_in_industry}} describes that Google,
Facebook, Microsoft, Nvidia, and many other companies use Haskell.

Erlang is a functional language for concurrent and parallel computing. Elixir
operates on the Erlang virtual machine and is used for the same purpose as Erlang.
An article from Code
Sync\sidenote{\url{https://codesync.global/media/successful-companies-using-elixir-and-erlang/}}
said that various companies including WhatsApp, Pinterest, and Goldman Sachs
use Erlang and Elixir.

\section{Installation}

As Scala programs are compiled to Java bytecode, which runs on the Java Virtual
Machine (\acrshort{jvmLabel}), you must
install Java before installing Scala. Java has various versions. Scala 2.13,
which is used in this book, needs JDK 8 or higher. JDK 8 is the most recommended
one. The Scala website\sidenote{\url{https://docs.scala-lang.org/overviews/jdk-compatibility/overview.html}}
discusses compatibility issues regarding the other versions.

The Oracle website\sidenote{\url{https://www.oracle.com/java/technologies/javase/javase-jdk8-downloads.html}}
provides an installation file for JDK 8.

You can donwload an installation file for Scala 2.13 from the Scala
website.\sidenote{\url{https://www.scala-lang.org/download/}} Note that you need a
file in the ``Other resources'' section at the bottom of the page.
On macOS, you may use Homebrew instead. By installing Scala, you can use the
Scala REPL, interpreter, and compiler. \refsec{scala-repl},
\refsec{scala-interpreter}, and \refsec{scala-compiler} will discuss
their usages respectively.

Another thing to install is SBT. SBT is a build tool for Scala. An installation
file for SBT is available at the SBT
website.\sidenote{\url{https://www.scala-sbt.org/download.html}}
\refsec{sbt} will discuss the usage of SBT.

\section{REPL}
\labsec{scala-repl}

Once you install Scala, you can launch Scala REPL by typing \code{scala} in
your command line.

\begin{verbatim}
$ scala
Welcome to Scala 2.13.5.
Type in expressions for evaluation. Or try :help.

scala>
\end{verbatim}

The term \acrshort{replLabel} stands for \textbf{r}ead, \textbf{e}val, \textbf{p}rint, and
\textbf{l}oop.
It is a program that iterativley reads code from a user, evaluates the code,
and prints the result. REPL is not a place to write a program but is a good
place to write short code and see how it works.

If you input an integer to REPL, it will evaluate the integer and show the
result.

\begin{verbatim}
scala> 0
val res0: Int = 0
\end{verbatim}

It means that the expression \code{0} evaluates to the value \code{0} and the type of
\code{0} is \code{Int}.
You can try some arithmetic expressions as well.

\begin{verbatim}
scala> 1 + 2
val res1: Int = 3
\end{verbatim}

A boolean is \code{true} or \code{false} in Scala.

\begin{verbatim}
scala> true
val res2: Boolean = true
\end{verbatim}

You can also use basic logical operators.

\begin{verbatim}
scala> true && false
val res3: Boolean = false
\end{verbatim}

String literals require double quotation marks.

\begin{verbatim}
scala> "hello"
val res4: String = hello
\end{verbatim}

Operations regarding strings can be done by calling methods.

\begin{verbatim}
scala> "hello".length
val res5: Int = 5

scala> "hello".substring(0, 4)
val res6: String = hell
\end{verbatim}

Strings in Scala provide the same methods as those in
Java.\sidenote{\url{https://docs.oracle.com/javase/8/docs/api/java/lang/String.html}}

The \code{println} function prints a given message into the console.

\begin{verbatim}
scala> println("Hello world!")
Hello world!
\end{verbatim}

Note that there is no result of \code{println("Hello world!")}. Actually,
\code{println("Hello world!")} evaluates to \code{()}, which is called unit.
Unit implies that the result does not have any meaningful information. It is
similar to \code{None} in Python and \code{undefined} in JavaScript. At the
same time, functions returning unit are similar to functions whose return types
are \code{void} in C or Java. Since unit does not have meaningful information,
REPL does not show the result when it is unit.

The remainder of this section introduces basic features of Scala, such as
variables and functions, with REPL.

\subsection{Variables}

The syntax of a variable definition is as follows:

\begin{verbatim}
val [name]: [type] = [expression]
\end{verbatim}

It defines a variable whose name is \code{[name]}.
The result of the expression becomes the value denoted by the variable and
must belong to the type.

\begin{verbatim}
scala> val x: Int = 1
val x: Int = 1
\end{verbatim}

If the type of the result does not match a given type, the variable will not be
defined due to a type mismatch.

\begin{verbatim}
scala> val y: Boolean = 2
                        ^
       error: type mismatch;
        found   : Int(2)
        required: Boolean
\end{verbatim}

You can omit the \code{: [type]} part and use the following syntax instead:

\begin{verbatim}
val [name] = [expression]
\end{verbatim}

In this case, a type mimatch never happens, and the
type of the variable becomes the same as the type of its value.
People usually omit the type annotations of local variables.

\begin{verbatim}
scala> val x = 3
val x: Int = 3
\end{verbatim}

Variables defined by \code{val} cannot be mutated, i.e. their values never
change. Reassignment will incur an error. We call such variables immutable
variables.

\begin{verbatim}
scala> x = 4
         ^
       error: reassignment to val
\end{verbatim}

Sometimes, mutable variables, i.e. variables whose values can change, are useful.
Scala provides mutable variables as well as immutable variables. You need to use
\code{var} instead of \code{val} to define mutable variables. You may or may not write
the type of a variable.

\begin{verbatim}
scala> var z = 5
var z: Int = 5

scala> z = 6
// mutated z

scala> z
val res8: Int = 6
\end{verbatim}

To assign a new value to a mutable variable, the value must conform to the
type of the variable. Otherwise, a type mismatch will happen.

\begin{verbatim}
scala> z = true
           ^
       error: type mismatch;
        found   : Boolean(true)
        required: Int
\end{verbatim}

\subsection{Functions}

The syntax of a function definition is as follows:

\begin{verbatim}
def [name]([name]: [type], …): [type] = [expression]
\end{verbatim}

Many programming languages require \code{return} to specify the return value of a function.
On the other hand, functions in Scala are like functions in mathematics: \code{return} is unnecessary.
The return value of a function is the result of the body expression, which is the expression
at the right side of \code{=} in the definition. The type annotation after each parameter specifies
the type of the parameter. The type after the parentheses is the return type, which must be the
same as the type of the return value.

\begin{verbatim}
scala> def add(x: Int, y: Int): Int = x + y
def add(x: Int, y: Int): Int

scala> add(3, 7)
val res9: Int = 10
\end{verbatim}

The return types of functions can be omitted.

\begin{verbatim}
scala> def add(x: Int, y: Int) = x + y
def add(x: Int, y: Int): Int
\end{verbatim}

However, parameter types cannot be omitted.

\begin{verbatim}
scala> def add(x, y) = x + y
                ^
       error: ':' expected but ',' found.
\end{verbatim}

To write multiple expressions including variable and functions definitions
in the body of a function, we put expressions separated by line breaks
inside curly braces.
Each line will be evaluated in the order, and the result of the last line will
be the return value.

\begin{verbatim}
scala> def quadruple(x: Int): Int = {
     |   val y = x + x
     |   y + y
     | }
def quadruple(x: Int): Int
\end{verbatim}

Inside \code{quadruple}, the variable \code{y} is defined and used for the
computation of
the return value.\sidenote{Vertical bars (\code{|}) at the beginning of lines are
not part of code. They have been automatically inserted by REPL.}

Multiple expressions inside curly braces are collectively treated as a single
expression. We call such an expression a sequenced expression. Like any other
expressions, a sequenced expression can occur anywhere an expression is needed.
For example, it can be used to define a variable.

\begin{verbatim}
scala> val a = {
     |   val x = 1 + 1
     |   x + x
     | }
val a: Int = 4
\end{verbatim}

There are many other things related to functions: recursion, first-class
functions, closures, and anonymous functions. \refch{immutability} will
discuss recursion, and \refch{functions} will discuss the other topics.

\subsection{Conditionals}

A conditional expression performs computation depending on a certain
condition, i.e. a boolean value. The syntax of a conditional expression is as
follows:

\begin{verbatim}
if ([expression]) [expression] else [expression]
\end{verbatim}

The first expression is the condition; the second expression is the true branch;
the last expression is the false branch.

\begin{verbatim}
scala> if (true) 1 else 2
val res10: Int = 1
\end{verbatim}

A conditional expression evaluates to a value. It is more similar to the ternary
operator \code{? :} in C than a if statement.
We do not need to make a variable
mutable to initialize the variable with a conditional value.

\begin{verbatim}
scala> val x = if (true) 1 else 2
val x: Int = 1
\end{verbatim}

On the other hand, people write code like below in languages like C.

\begin{verbatim}
int x;
if (true)
    x = 1;
else
    x = 2;
\end{verbatim}

Conditional expressions in Scala are more expressive than the ternary operator
in C because we can make complex computation a single expression with expression
sequencing, which is impossible in C.

\begin{verbatim}
scala> if (true) {
     |   val x = 2
     |   x + x
     | } else {
     |   val x = 3
     |   x * x
     | }
val res11: Int = 4
\end{verbatim}

\subsection{Lists}

A list is a collection of zero or more elements. A list maintains the order between
its elements. Lists in Scala are immutable. Once a list is created, its elements
never change. There are two ways to create a new list in Scala:

\begin{itemize}
  \item \code{List([expression], …, [expression])}
  \item \code{[expression] :: … :: [expression] :: Nil}
\end{itemize}

The type of a list whose elements have type \code{T} is \code{List[T]}.

\begin{verbatim}
scala> List(1, 2, 3)
val res12: List[Int] = List(1, 2, 3)

scala> 1 :: 2 :: 3 :: Nil
val res13: List[Int] = List(1, 2, 3)
\end{verbatim}

\code{List(…)} is more convenient than \code{::} for creating a new list from
scratch. However, \code{::} is more flexible since it can prepend a new element
in front of an existing list.\sidenote{It does not mutate the existing list to
prepend the new element. It creates a new list with the element and the list.}

\begin{verbatim}
scala> val l = List(1, 2, 3)
val l: List[Int] = List(1, 2, 3)

scala> 0 :: l
val res14: List[Int] = List(0, 1, 2, 3)
\end{verbatim}

The \code{length} method computes the length of a list; parentheses are used
to fetch the element at a specific index.\sidenote{The first index is \code{0}.}

\begin{verbatim}
scala> l.length
val res15: Int = 3

scala> l(0)
val res16: Int = 1
\end{verbatim}

In functional programming, accessing an arbitrary element of a list by an index
is rare. We use pattern matching in most cases. The syntax of pattern matching
for a list is as follows:\sidenote{The order between the cases can vary, which
means that the \code{::} case may come first.}

\begin{verbatim}
[expression] match {
  case Nil => [expression]
  case [name] :: [name] => [expression]
}
\end{verbatim}

The expression in front of \code{match} is the target of pattern matching.
If it is an empty list, it matches \code{case Nil}. The expression
of the \code{Nil} case will be evaluated.
Otherwise, it is a nonempty list and matches \code{case [name] :: [name]}. The first
name denotes the head\sidenote{the first element} of the list, and the second
name denotes the tail\sidenote{a list consisting of all the elements except the
head} of the list. The expression of the \code{::} case will be evaluated.

The following function takes a list of integers as an argument and returns the
head. The return value is zero when the list is empty.

\begin{verbatim}
scala> def headOrZero(l: List[Int]): Int = l match {
     |   case Nil => 0
     |   case h :: t => h
     | }
def headOrZero(l: List[Int]): Int

scala> headOrZero(List(1, 2, 3))
val res17: Int = 1

scala> headOrZero(List())
val res18: Int = 0
\end{verbatim}

\refch{immutability} will show use of pattern matching for lists in recursive
functions, and \refch{pattern-matching} will discuss pattern matching in detail.

\subsection{Tuples}

A tuple contains two or more elements and maintains the order between its
elements. We use parentheses to create a new tuple:

\begin{verbatim}
([expression], …, [expression])
\end{verbatim}

The type of a tuple whose elements have types from \code{T1} to \code{Tn}
respectively is \code{(T1, …, Tn)}. For example, the type of a tuple
whose first element is \code{Int} and second element is \code{Boolean} is
\code{(Int, Boolean)}.

\begin{verbatim}
scala> (1, true)
val res19: (Int, Boolean) = (1,true)
\end{verbatim}

To fetch the \code{i}-th element of a tuple, we can use \verb+._i+.\sidenote{The
first index is 1.}

\begin{verbatim}
scala> (1, true)._1
val res20: Int = 1
\end{verbatim}

Tuples look similar to lists but have important differences from
lists. First, a tuple's elements can have different types, while a list's
elements cannot. For example, a tuple of the type \code{(Int, Boolean)} has
one integer and one boolean, while a list of the type \code{List[Int]} can have
only integers. We say that tuples are heterogenous, while lists are homogeneous.
Second, a list allows accessing an arbitrary index of a list,
while a tuple does not. For example, \code{l(f())} is possible where \code{l} is
a list and \code{f} returns an integer, while there is no way to access the
\code{f()}-th element of a tuple since the return value of \code{f} is unknown
before execution.

We use lists and tuples for different purposes. Lists are appropriate when the
number of elements can vary and an arbitrary index should be accessible.
For instance, a list should be used to represent a collection of the heights of
students in a certain class.

\begin{verbatim}
List(189, 167, 156, 170, 183)
\end{verbatim}

It allows us to fetch the height of the \code{i}-th student.

On the other hand, tuples are appropriate when the number of elements
are fixed and each index has a specific meaning. For instance,
a tuple can represent the information of a single student, where the information
consists of one's name, one's height, and whether one has payed the school
expense or not.

\begin{verbatim}
("John Doe", 173, true)
\end{verbatim}

We can use \verb+._1+ to find the name, \verb+._2+ to find the height, and
\verb+._3+ to check whether one has payed.

We call a length-2 tuple a pair and a length-3 tuple a triple. Also, we can
consider unit as a length-0 tuple.

\subsection{Maps}

A map is a collection of pairs, where each pair consists of a key and a value.
Its provides the corresponding value when a key is given.
Maps in Scala are immutable as well. Below is the syntax to create a new map:

\begin{verbatim}
Map([expression] -> [expression], …)
\end{verbatim}

The type of a map whose keys have type \code{T} and values have type \code{S} is
\code{Map[T, S]}.

\begin{verbatim}
scala> val m = Map(1 -> "one", 2 -> "two", 3 -> "three")
val m: Map[Int,String] = Map(1 -> one, 2 -> two, 3 -> three)
\end{verbatim}

To find the value corresponding to a certain key, we use parentheses.

\begin{verbatim}
scala> m(2)
val res21: String = two
\end{verbatim}

Maps provide various
methods.\sidenote{\url{https://www.scala-lang.org/api/current/scala/collection/immutable/Map.html}}

\subsection{Classes and Objects}

An object is a value with fields and methods. Fields store values, and methods
are operations related to the object. A class is a blueprint of objects. We can
easily create multiple objects of the same structure by defining a single class.
This book uses only ``case'' classes of Scala. Case classes are similar to
classes but more convenient, e.g. automatic support for pretty printing and
pattern matching.

The syntax of a class definition is as follows:

\begin{verbatim}
case class [name]([name]: [type], …)
\end{verbatim}

The first name is the name of a new class. The names inside
the parentheses are the names of the fields of the class. A class definition
must specify the types of its fields.

\begin{verbatim}
scala> case class Student(name: String, height: Int)
class Student
\end{verbatim}

Creating new objects is similar to a function call.

\begin{verbatim}
scala> val s = Student("John Doe", 173)
val s: Student = Student(John Doe,173)
\end{verbatim}

Fields can be accessed by \code{.[name]}.

\begin{verbatim}
scala> s.name
val res22: String = John Doe
\end{verbatim}

Objects in Scala are immutable by default. If we add \code{var} to a field when
defining a class, the field becomes mutable.

\begin{verbatim}
scala> case class Student(name: String, var height: Int)
class Student

scala> val s = Student("John Doe", 173)
val s: Student = Student(John Doe,173)

scala> s.height = 180
// mutated s.height

scala> s.height
val res23: Int = 180
\end{verbatim}

\section{Interpreter}
\labsec{scala-interpreter}

An \textit{interpreter}\index{interpreter} is a program that takes source code as input and runs the code.
The Scala interpreter takes Scala source code as input. To use the interpreter,
we need to save source code into a file. Make a file with the following code, and
save it as \code{Hello.scala}.

\begin{verbatim}
println("Hello world!")
\end{verbatim}

You can excute the interpreter by typing \code{scala} with the name of a file in
your command line. Here, we need to say \code{scala Hello.scala}.

\begin{verbatim}
$ scala Hello.scala
Hello world!
\end{verbatim}

You can write multiple lines in a single file. Fix \code{Hello.scala} like
below.

\begin{verbatim}
val x = 2
println(x)
val y = x * x
println(y)
\end{verbatim}

Then, excute the interpreter again.

\begin{verbatim}
$ scala Hello.scala
2
4
\end{verbatim}

\section{Compiler}
\labsec{scala-compiler}

A \textit{compiler}\index{compiler} is a program that takes source code as input and translates it into
another language. Usually, the target language is a low-level language like
machine code or bytecode of a particular virtual machine.
The Scala compiler takes Scala source code as input and translates it into Java
bytecode. Once code is compiled, we can run the generated bytecode with the JVM.

For compilation, we need to define the main method of a program. The main method
is the entrypoint of every program running on the JVM.
Make a file with the following code, and save it as \code{Hello.scala}.

\begin{verbatim}
object Hello {
  def main(args: Array[String]): Unit = {
    println("Hello world!")
  }
}
\end{verbatim}

You can make the compiler compile the code by typing \code{scalac} with the name
of the file in your command line.

\begin{verbatim}
$ scalac Hello.scala
\end{verbatim}

After compilation, you will be able to find the \code{Hello.class} file
in the same directory. The file contains Java bytecode.

You can run the bytecode with the JVM by the \code{scala} command. In this time,
you should write only the class name.

\begin{verbatim}
$ scala Hello
Hello world!
\end{verbatim}

You can change the behavior of a program by modifying the main method.
Each time you modify, you need to re-compile the program to re-generate the
bytecode.

Running bytecode is much more efficient than interpreting Scala source code.
You can easily notice that \code{scala Hello} takes much less than \code{scala
Hello.scala} even though their results are the same.

Scala has two sorts of errors: compile-time errors and run-time errors.
Compile-time errors occur during compilation, i.e. while running \code{scalac}.
If the compiler finds things that
might go wrong at run time, it raises errors and aborts the compilation. For
example, an expression adding an integer to a boolean results in a compile-time
error because such an addition cannot succeed at run time.

\begin{verbatim}
true + 1
\end{verbatim}
\vspace{-1em}
\begin{mdframed}[hidealllines=true,backgroundcolor=red!10,innerleftmargin=3pt,innerrightmargin=3pt,leftmargin=-3pt,rightmargin=-3pt]
\begin{verbatim}
error: type mismatch;
 found   : Int(1)
 required: String
true + 1
     ^
\end{verbatim}
\vspace{-2em}
\begin{flushright}
\scriptsize\textsf{Compile-time error}
\end{flushright}
\end{mdframed}

Unfortunately, some bad behaviors cannot be detected by the compiler.
The compiler does not generate any errors for those behaviors.
Such problems will incur run-time errors during execution, i.e. while running
\code{scala},
and terminate the execution abnormally. Division by zero is one
example of run time errors.

\begin{verbatim}
1 / 0
\end{verbatim}
\vspace{-1em}
\begin{mdframed}[hidealllines=true,backgroundcolor=red!10,innerleftmargin=3pt,innerrightmargin=3pt,leftmargin=-3pt,rightmargin=-3pt]
\begin{verbatim}
java.lang.ArithmeticException: / by zero
\end{verbatim}
\vspace{-2em}
\begin{flushright}
\scriptsize\textsf{Run-time error}
\end{flushright}
\end{mdframed}

\section{SBT}
\labsec{sbt}

SBT is a build tool for Scala. Build tools help programmers work on large
projects with many files and libraries by tracking dependencies between files
and managing libraries. There are various build tools in the world, and SBT is
the most popular one for Scala.

You can create a new Scala project by the \code{sbt new} command.

\begin{verbatim}
$ sbt new scala/scala-seed.g8
[info] welcome to sbt 1.4.7
[info] loading global plugins from ~/.sbt/1.0/plugins
[info] set current project to ~/ (in build file:~/)
[info] set current project to ~/ (in build file:~/)


A minimal Scala project.

name [Scala Seed Project]: hello

Template applied in ~/hello
\end{verbatim}

After the creation, the directory structure is as follows:

\dirtree{%
  .1 hello.
  .2 build.sbt.
  .2 project.
  .3 Dependencies.scala.
  .3 build.properties.
  .2 src.
  .3 main.
  .4 scala.
  .5 example.
  .6 Hello.scala.
  .3 test.
  .4 scala.
  .5 example.
  .6 HelloSpec.scala.
}

The \code{build.sbt} file configures the project. It manages the version of Scala
used for the project, third-party libraries used in the project, and many other
things.
Source files are in the \code{src} directory. Files in \code{main} are
main source files, while files in \code{test} are only for testing.
You can add files into the \code{src/main/scala} directory and edit them to write code.

An SBT console can be started by the \code{sbt} command. The current working
directory of your shell should be the base directory of the project.

\begin{verbatim}
$ sbt
[info] welcome to sbt 1.4.7
[info] loading global plugins from ~/.sbt/1.0/plugins
[info] loading project definition from ~/hello/project
[info] loading settings for project root from build.sbt ...
[info] set current project to hello (in build file:~/hello/)

[info] sbt server started at
local:///~/.sbt/1.0/server/d4cd702f998423203dfe/sock
[info] started sbt server
sbt:hello>
\end{verbatim}

You can compile, run, and test the project by executing SBT commands in the
console.

\begin{itemize}
  \item \code{compile}: compile the project.
  \item \code{run}: run the project (re-compile if necessary).
  \item \code{test}: test the project (re-compile if necessary).
  \item \code{exit}: terminate the console.
\end{itemize}

\begin{verbatim}
sbt:hello> compile
[info] compiling 1 Scala source to ~/hello/target/scala-2.13

| => root / Compile / compileIncremental 0s
[success] Total time: 4 s
sbt:hello> test
[info] compiling 1 Scala source to ~/hello/target/scala-2.13
[info] HelloSpec:
[info] The Hello object
[info] - should say hello
[info] Run completed in 455 milliseconds.
[info] Total number of tests run: 1
[info] Suites: completed 1, aborted 0
[info] Tests: succeeded 1, failed 0, canceled 0, ignored 0
[info] All tests passed.
[success] Total time: 2 s
sbt:hello> run
[info] running example.Hello
hello
[success] Total time: 0 s
sbt:hello> exit
[info] shutting down sbt server
\end{verbatim}

To learn SBT more, refer to the SBT website.\sidenote{\url{https://www.scala-sbt.org/learn.html}}

\setchapterpreamble[u]{\margintoc}
\chapter{Immutability}
\labch{immutability}

\textit{Immutability}\index{immutability} means not changing.
Immutable variables never change their values
after initialization; immutable data structures never change their elements
once created. The opposite of immutability is mutability. While imperative
programming uses lots of mutable variables, data structures, and objects,
functional programming leverages the power of immutable varibles, data
structures, and objects. This chapter explains why immutability is important and
valuable. Also, we will see how to program without mutation.

\section{Advantages}

The book Programming in Scala~\cite{programming-in-scala}
discusses four strengths of immutability:

\begin{quote}
First, immutable objects are often easier to reason about than mutable ones,
because they do not have complex state spaces that change over time. Second, you
can pass immutable objects around quite freely, whereas you may need to make
defensive copies of mutable objects before passing them to other code. Third,
there is no way for two threads concurrently accessing an immutable to corrupt
its state once it has been properly constructed, because no thread can change the
state of an immutable. Fourth, immutable objects make safe hash table keys. If a
mutable object is mutated after it is placed into a \code{HashSet}, for example,
that object may not be found the next time you look into the \code{HashSet}.
\end{quote}

We will focus on the first two advantages:
easier reasoning and no need for defensive copies.

First, let us see why immutability makes things easy to reason about.

\begin{verbatim}
val x = 1
...
f(x)
\end{verbatim}

At the first line of the code, \code{x} is \code{1}. Since \code{x} is immutable,
there is no doubt that \code{x} is still \code{1} when \code{x} is passed as an
argument for \code{f} at the last line of the code.

\begin{verbatim}
var x = 1
...
f(x)
\end{verbatim}

On the other hand, if \code{x} is a mutable variable, one should read every line
of code in the middle to find the value of \code{x} at the time when the function
call happens.

When \code{x} is mutable, without tracking every
modification of \code{x} throughout the code, the value of \code{x} at the last
line is unknown. It hampers programmers from understanding the code
and possibly leads to more bugs.
The program with immutable \code{x} does not suffer from such problems.
Remembering only one line of the code is enough to track the value of \code{x}.

Mutable data structures cause similar problems.

\begin{verbatim}
val x = List(1, 2)
...
f(x)
...
x
\end{verbatim}

As \code{List} is immutable,
\code{x} is a list always containing \code{1} and \code{2}.

\begin{verbatim}
import scala.collection.mutable.ListBuffer
val x = ListBuffer(1, 2)
...
f(x)
...
x
\end{verbatim}

On the other hand, \code{ListBuffer} is a mutable data structure in the Scala
standard library. It is possible to add an item to or remove an item from the
list referred by \code{x}. Programmers cannot be certain about the content of \code{x}
unless they read all the lines in between. Besides, a function \code{f} also is
able to change the content of \code{x}. If one writes a program with a wrong
assumption that \code{f} does not modify \code{x}, then the program might be
buggy.

Mutable global variables make code much harder to understand than mutable local
variables.

\begin{verbatim}
def f(x: Int) = g(x, y)
\end{verbatim}

The return value of function \code{f} depends on the value of a global variable
\code{y}. If \code{y} is mutable, \code{f} is not a pure function and expecting
the behavior of \code{f} is nontrivial. \code{y} can be declared in any arbitrary
file and all files are able to change the value of \code{y}.
In the worst case, an external library defines \code{y} and source code
modifying \code{y} is not available for reading.

The examples are small and seem artificial, but immutability greatly improves
maintainability and readability of code in practice, especially for large
projects.

Now, let us see why immutability free us from making defensive copies.

\begin{verbatim}
val x = ListBuffer(1, 2)
...
f(x)
...
x
\end{verbatim}

Since \code{ListBuffer} creates mutable lists, there is no guarantee that the
content of \code{x} does not be changed by \code{f}. If it is necessary to prevent
modification, copying \code{x} is essential.

\begin{verbatim}
val x = ListBuffer(1, 2)
val y = x.clone
...
f(y)
...
x
\end{verbatim}

In cases that \code{x} has many elements and the code is executed multiple times,
copying \code{x} increases the execution time significantly.

In the code, using the \code{clone} method is enough to copy the list because the
list contains only integers. However, to pass lists containing mutable
objects safely to functions, defining additional methods for deep copy is
inevitable.

Immutability has several clear advantages. Immutability is an important concept in
functional programming. Functional programs use immutable variables and data
structures in most cases. If you write a large program whose logic is complex
and correctness is important, you should adopt the functional paradigm.
However, mind that immutability is not the silver bullet for every
program. For example, implementing algorithms in a functional style is usually
inefficient. It would be better to use mutable data structures like arrays,
mutable variables, and loops to implement algorithms. They make
programs much more efficient and faster. Choosing a programming proper paradigm to
the purpose of a program is the key to write good code.

\section{Recursion}

Repeating the same computation multiple times is a common pattern in programming.
Loops allow concise code expressing such cases. However, if everything is
immutable, going back to the beginnings of loops does not change any states.
Therefore, it is impossible to apply the same operation on different values for
each iteration or to terminate the loops. As a consequence, loops are useless in
functional programming. Functional programs use recursive functions instead of
loops to rerun computation. A \textit{recursive}\index{recursion}
function is a function that calls itself.\sidenote{In general, a definition that
refers to itself is a recursive definition. There can be recursive variables,
recursive types, and so on.}
To do more computation, the function calls itself with proper arguments.
Otherwise, it terminates the computation by returning some value.

The below \code{factorial} function calculates the factorial of a given integer.
For simplicity, we do not consider when the input is negative.
The following implementation uses an imperative style:

\begin{verbatim}
def factorial(n: Int) = {
  var i = 1, res = 1
  while (i <= n) {
    res *= i
    i += 1
  }
  res
}
\end{verbatim}

We can implement the same function in a functional style with recursion.

\begin{verbatim}
def factorial(n: Int): Int =
  if (n <= 0)
    1
  else
    n * factorial(n - 1)
\end{verbatim}

Note that recursive functions always require explicit return types in Scala,
unlike non-recursive functions, whose return types can be omitted.

The recursive version is preferred over the imperative version since its
correctness is easy to be verified.

To check the correctness of the imperative
\code{factorial} function, one should find a \textit{loop invariant}\index{loop invariant},
which is a proposition that is always true at the loop head.
The loop invariant of this case is
$((\code{i}-1)!=\code{res})\land(\code{i}\le\code{n}+1)$.
By using this invariant, we can conclude that $\code{i}=\code{n}+1$ and,
therefore, $\code{res}=(\code{i}-1)!=\code{n}!$ at the last line of the
function, which implies that it correctly implements factorial.
It is nontrival to find a proper loop invariant and show that the loop invariant
holds at the beginning of each iteration.

On the other hand,
recursive functions usually reveal their mathematical definitions more clearly
than functions using loops. Consider the following mathematical definition of
factorial:

\[n!=\begin{cases}1 & \text{if } n=0\\n \times (n-1)! &
\text{otherwise}\end{cases}\]

You can see that the implementation of the \code{factorial} function using recursion
is identical to the mathematical definition of factorial. It is almost trivial
to show that the recursive \code{factorial} function is correct.
Recursion allows concise and intuitive descriptions of mathematical functions.
In many cases, functions with recursion is much easier to be verified formally
or informally than functions with loops.

Recursive functions are also good at treating recursive data structures like
lists. A list is recursive since a nonempty list consists of the head element
and the tail list, which means that a nonempty list has another list as its component.
Writing some functions regarding lists help understanding and practicing
recursion.

The following function takes a list as an argument and returns a list whose
elements are one larger than the elements of the given list.

\begin{verbatim}
def inc1(l: List[Int]): List[Int] = l match {
  case Nil => Nil
  case h :: t => h + 1 :: inc1(t)
}
\end{verbatim}

When a given list is empty, the function returns the empty list. Otherwise, the
return value is a list whose head is one larger than the head of the given list
and tail has elements that are one larger than the elements of the tail of the
given list.

Similarly, \code{square} takes a list of integers as an argument and returns a
list whose elements are the squares of the elements of the given list.

\begin{verbatim}
def square(l: List[Int]): List[Int] = l match {
  case Nil => Nil
  case h :: t => h * h :: square(t)
}
\end{verbatim}

The following function takes a list of integers as an argument and returns a list whose
elements are odd integers.

\begin{verbatim}
def odd(l: List[Int]): List[Int] = l match {
  case Nil => Nil
  case h :: t =>
    if (h % 2 != 0)
      h :: odd(t)
    else
      odd(t)
}
\end{verbatim}

For a nonempty list, the function checks whether the head is odd or not. If the
head is odd, the resulting list contains the head, and its tail  has
only odd integers. Otherwise, the head is removed.

Similarly, \code{positive} takes a list of integers as an argument and returns a
list whose elements are positive.

\begin{verbatim}
def positive(l: List[Int]): List[Int] = l match {
  case Nil => Nil
  case h :: t =>
    if (h > 0)
      h :: positive(t)
    else
      positive(t)
}
\end{verbatim}

The following function calculates the sum of the elements of a given list.

\begin{verbatim}
def sum(l: List[Int]): Int = l match {
  case Nil => 0
  case h :: t => h + sum(t)
}
\end{verbatim}

The sum of elements in the empty list is zero, as there are no elements.
When a list is nonempty, the sum of its elements can be calculated by adding the
value of the head to the sum of its tail's elements.

Similarly, \code{product} calculates the product of the elements of a given
list.

\begin{verbatim}
def product(l: List[Int]): Int = l match {
  case Nil => 1
  case h :: t => h * product(t)
}
\end{verbatim}

Recursion has some disadvantages: overheads of function calls and stack
overflow. Most modern CPUs have enough computing power to ignore function call
overheads. However, loops are still ideal for functions in performance-critical
programs. Stack overflow happens when a
stack lacks space due to repetitive function calls. It is a critical problem
since it causes immediate termination of execution without yielding meaningful
output. Moreover, programs like web servers do not finish their execution, and
their stacks will eventually overflow. To prevent stack overflow, many functional
languages provide tail call optimization.
The following section explains tail call optimization in detail.

\section{Tail Call Optimization}
\labsec{tco}

If the last action of a function is a function call, then the call is a tail
call. When a tail call happens, the callee does every computation, and thus the
local variables of the caller have no need to remain after the call. The stack
frame of the caller can be destroyed. Most functional languages exploit this
fact to optimize tail calls. This optimization is called
\textit{tail call optimization}.\index{tail call optimization}
At compile time, compilers check whether calls are tail calls.
If a call is a tail call, the compilers generate code that eliminates the
stack frame of the caller before the call. They do not optimize non-tail function
calls because the local variables of the callers can be used after the callees
return. If every function call in a program is a tail call, the stack never
grows so that the program is safe from stack overflow.

\begin{verbatim}
def factorial(n: Int): Int =
  if (n <= 0)
    1
  else
    n * factorial(n - 1)
\end{verbatim}

The previous \code{factorial} function multiplies \code{n} and the return value
of the recursive \code{factorial(n - 1)} call. The multiplication is the last
action. The recursive call is not a tail call. The stack frame of the caller must
remain. The following process computes \code{factorial(3)}:

\begin{itemize}
\item \code{factorial(3)}
\item \code{3 * factorial(2)}
\item \code{3 * (2 * factorial(1))}
\item \code{3 * (2 * (1 * factorial(0)))}
\item \code{3 * (2 * (1 * 1))}
\item \code{3 * (2 * 1)}
\item \code{3 * 2}
\item \code{6}
\end{itemize}

At most four stack frames coexist. For a large argument, a stack grows
again and again and finally overflows.

\begin{verbatim}
factorial(10000)
\end{verbatim}
\vspace{-1em}
\begin{mdframed}[hidealllines=true,backgroundcolor=red!10,innerleftmargin=3pt,innerrightmargin=3pt,leftmargin=-3pt,rightmargin=-3pt]
\begin{verbatim}
java.lang.StackOverflowError
  at .factorial
\end{verbatim}
\vspace{-2em}
\begin{flushright}
\scriptsize\textsf{Run-time error}
\end{flushright}
\end{mdframed}

To implement the function with a tail call, instead of multiplying \code{n} and
\code{factorial(n - 1)}, the function has to pass both \code{n} and \code{n - 1}
as arguments and make the callee multiply \code{n} and $(\code{n - 1})!$.
This strategy can be interpreted as passing an intermediate result.

\begin{itemize}
\item \code{factorial(3)}
\item \code{factorial(2, intermediate result = 3)}
\item \code{factorial(1, intermediate result = 3 * 2)}
\item \code{factorial(1, intermediate result = 6)}
\item \code{factorial(0, intermediate result = 6 * 1)}
\item \code{factorial(0, intermediate result = 6)}
\item \code{6}
\end{itemize}

There is no need to return to the caller. The below code shows the
\code{factorial} function with a tail call. The function needs one more parameter
that takes an intermediate result as an argument.
\code{factorial(n, i)} computes \(\code{n}!\times\code{i}\).

\begin{verbatim}
def factorial(n: Int, inter: Int): Int =
  if (n <= 0)
    inter
  else
    factorial(n - 1, inter * n)
\end{verbatim}

The function uses a tail call. More precisely, the function is
tail-recursive. Its last action is calling itself. Unlike most functional
languages, Scala cannot optimize general tail calls. Scala optimizes only
tail-recursive calls.
The Scala compiler generates Java bytecode, which is excuted by the JVM. The JVM does not
allow bytecode to jump to the beginning of another function. In the JVM, functions can
only either return or call functions. Therefore, the Scala compiler cannot generate
optimized code by removing the stack frame of a caller. Instead, they transform
tail-recursive calls into loops.
The \code{factorial} function is compiled to the following bytecode:

\begin{verbatim}
public int factorial(int, int);
  Code:
     0: iload_1
     1: iconst_0
     2: if_icmpgt     9
     5: iload_2
     6: goto          20
     9: iload_1
    10: iconst_1
    11: isub
    12: iload_2
    13: iload_1
    14: imul
    15: istore_2
    16: istore_1
    17: goto          0
    20: ireturn
\end{verbatim}

We can check that there is no function call at
all.\sidenote{\code{invokevirtual} is a function call instruction.}
The function just jumps to instructions inside the function.
Due to the tail call optimization, the function never incurs stack overflow.

Even with tail recursion,
the result is still incorrect because of integer overflow.

\begin{verbatim}
assert(factorial(10000, 1) == 0)  // weird result
\end{verbatim}

The \code{BigInt} type resolves integer overflow.

\begin{verbatim}
def factorial(n: BigInt, inter: BigInt): BigInt =
  if (n <= 0)
    inter
  else
    factorial(n - 1, inter * n)

assert(factorial(10000, 1) > 0)
\end{verbatim}

The optimization of the Scala compiler not only prevents stack overflow but also removes the
overheads of function calls. The downside is that mutually recursive
functions using tail calls lie beyond the scope of the optimization.
Mutual recursion is recursion involving two or more definitions.
The following functions can cause stack overflow in Scala even though they use tail
calls because they are not tail-recursive:

\begin{verbatim}
def even(n: Int): Boolean = if (n <= 0) true else odd(n - 1)
def odd(n: Int): Boolean = if (n == 1) true else even(n - 1)
\end{verbatim}

In Scala, programmers can ask the compiler to check
whether functions are tail-recursive with annotations. The annotations prevent
programmers from making functions non-tail-recursive by mistakes.

\begin{verbatim}
import scala.annotation.tailrec
@tailrec def factorial(n: BigInt, inter: BigInt): BigInt =
  if (n <= 0)
    inter
  else
    factorial(n - 1, inter * n)
\end{verbatim}

A non-tail-recursive function with the \code{tailrec} annotation results in a
compile-time error.

\begin{verbatim}
@tailrec def factorial(n: Int): Int =
  if (n <= 0)
    1
  else
    n * factorial(n - 1)
\end{verbatim}
\vspace{-1em}
\begin{mdframed}[hidealllines=true,backgroundcolor=red!10,innerleftmargin=3pt,innerrightmargin=3pt,leftmargin=-3pt,rightmargin=-3pt]
\begin{verbatim}
      ^
error:
could not optimize @tailrec annotated method factorial:
it contains a recursive call not in tail position
\end{verbatim}
\vspace{-1.5em}
\begin{flushright}
\scriptsize\textsf{Compile-time error}
\end{flushright}
\end{mdframed}

The annotation does not affect the behavior of the resulting bytecode.
Regardless of the existence of the annotation, the compiler always optimizes
tail-recursive functions. Still, using the annotations is desirable to prevent
mistakes.

Calling the tail-recursive version of \code{factorial} needs the unnecessary
second argument. The below code defines a new \code{factorial} function with one
parameter and uses the tail-recursive one as a local function inside the
function.

\begin{verbatim}
def factorial(n: BigInt): BigInt = {
  @tailrec def aux(n: BigInt, inter: BigInt): BigInt =
    if (n <= 0)
      inter
    else
      aux(n - 1, inter * n)
  aux(n, 1)
}
\end{verbatim}

Some functions treating lists also can be rewritten in a tail-recursive way.
Below is a tail-recursive version of \code{sum}.

\begin{verbatim}
def sum(l: List[Int]): Int = {
  @tailrec def aux(l: List[Int], inter: Int): Int = l match {
    case Nil => inter
    case h :: t => aux(t, inter + h)
  }
  aux(l, 0)
}
\end{verbatim}

\code{aux(l, n)} calculates \code{n} plus the sum of \code{l}'s elements.

Similarly, \code{product} can be implemented in a tail-recursive way.

\begin{verbatim}
def product(l: List[Int]): Int = l match {
  @tailrec def aux(l: List[Int], inter: Int): Int = l match {
    case Nil => inter
    case h :: t => aux(t, inter * h)
  }
  aux(l, 1)
}
\end{verbatim}

\section{Exercises}

\begin{enumerate}
  \item Consider the following definition of \code{Student}:

    \code{case class Student(name: String, height: Int)}

    Implement a function that takes a list of students as an argument
    and returns a list containing the names of the students.

  \item Consider the same definition of \code{Student}.
    Implement a function that takes a list of students as an argument
    and returns a list of students whose heights are greater than 170.

  \item
    Implement a function that takes a list of integers as an argument
    and returns the length of the list.

  \item
    Implement a function that takes a list of integers and an integer as arguments
    and returns a list obtained by appending the integer at the end of the list.
    Then, compare the time complexity of appending a new element to that of
    prepending a new element by \code{::}, which is $O(1)$.
\end{enumerate}

\setchapterpreamble[u]{\margintoc}
\chapter{Functions}
\labch{functions}

This section focuses on use of functions in functional programming.
In functional programming, functions are first-class. First-class functions
allow programmers to abstract complex computation easily.
This section explains what first-class functions are.
In addition, anonymous functions and closures, which are related to first-class
functions, will be introduced. To show the power of first-class functions, we
will re-implement the functions in \refch{immutability} (\code{inc1},
\code{square}, \ldots) with first-class functions.

\section{First-Class Functions}

An entity in a programming language is \textit{first-class}\index{first-class} if it satisfies the
following conditions:

\begin{itemize}
\item It can be an argument of a function call.
\item It can be a return value of a function.
\item A variable can refer to it.
\end{itemize}

Anything that is first-class can be used as a value. Functions are highly important and
treated as values in functional languages.
Functions that are first-class are called \textit{first-class functions}.\index{first-class function}

Some people use the term higher-order functions. \textit{Higher-order
functions}\index{higher-order function} are
functions that are not first-order, where first-order functions neither take
functions as arguments nor return functions. Therefore, higher-order functions
can take functions as arguments and return functions. Strictly speaking, they
are different from first-class functions because first-class functions are
functions that can be passed as arguments or returned from functions.
However, any languages that support first-class functions support higher-order
functions and vice versa.
The reason is obvious: to pass first-class functions as arguments, there should
be higher-order functions, and to pass functions to higher-order functions,
there should be first-class functions.
Consequently, in most contexts, people do not distinguish
first-class functions and higher-order functions, and you can consider
first-class functions and higher-order functions as exchangeable terms.

Now, let us see how we can use first-class functions in Scala with some code
examples.

\begin{verbatim}
def f(x: Int): Int = x
def g(h: Int => Int): Int = h(0)

assert(g(f) == 0)
\end{verbatim}

The function \code{g} has one parameter \code{h}. The type of \code{h} is \code{Int => Int}.
An argument passed to \code{g} is a function that receives one integer
and returns an integer. In Scala, \code{=>} expresses the types
of functions. Functions without parameters have types of the form \code{() => [return type]}.
\code{[parameter type] => [return type]} is the type of a function with a
single parameter. Parentheses are required to express the types of functions
with two or more parameters:
\code{([parameter type], … ) => [return type]}. The function \code{f}
has one integer parameter and returns an integer, i.e. its type is \code{Int =>
Int}. Thus, it can be an argument for
\code{g}. Evaluating \code{g(f)} equals evaluating \code{f(0)}, which results
in \code{0}.

\begin{verbatim}
def f(y: Int): Int => Int = {
  def g(x: Int): Int = x
  g
}

assert(f(0)(0) == 0)
\end{verbatim}

The function \code{f} returns the function \code{g}. Since the return type of \code{f}
is \code{Int => Int}, its return value must be a function that takes an integer
as an argument and returns an integer. \code{g} satisfies the condition. \code{f(0)}
is the same as \code{g} and therefore is a function. \code{f(0)(0)} equals \code{g(0)},
which returns \code{0}.

\begin{verbatim}
val h0 = f(0)

assert(h0(0) == 0)
\end{verbatim}

A variable can refer to \code{f(0)}. \code{h0} refers to the return value of
\code{f(0)} and has type \code{Int => Int}. Calling variables referring to
function values is possible. \code{h0(0)} is a valid expression and results in
\code{0}.

\begin{verbatim}
val h1 = f
\end{verbatim}
\vspace{-1em}
\begin{mdframed}[hidealllines=true,backgroundcolor=red!10,innerleftmargin=3pt,innerrightmargin=3pt,leftmargin=-3pt,rightmargin=-3pt]
\begin{verbatim}
         ^
error: missing argument list for method f

Unapplied methods are only converted to functions
when a function type is expected.

You can make this conversion explicit
by writing `f _` or `f(_)` instead of `f`.
\end{verbatim}
\vspace{-2em}
\begin{flushright}
\scriptsize\textsf{Compile-time error}
\end{flushright}
\end{mdframed}

On the other hand, defining a variable referring to \code{f} results in a compile
error. In Scala, a function defined by \code{def} is not a value per se. Since \code{f}
is the name of a function but not a variable referring to a value, \code{h1}
cannot refer to the value of \code{f}. As the above error message implies,
underscores convert function names into function values.

\begin{verbatim}
val h1 = f _

assert(h1(0)(0) == 0)
\end{verbatim}

Compiling the above code succeeds. The type of \code{h1} is \code{Int => (Int => Int)}.
\code{Int => Int => Int} denotes the same type because \code{=>} is
a right-associative type operator. \code{h1(0)(0)} is valid and yields \code{0}.

Actually, above expressions except \code{val h1 = f} use function names as values
successfully. The Scala compiler transforms function names into function values
when they occur where function types are expected. Therefore, enforcing the type of
\code{h1} to be a function type corrects the code without the underscore. The
following code works well:

\begin{verbatim}
val h1: Int => Int => Int = f
\end{verbatim}

When programmers use function names as values, they usually place the names where
function types are expected. In these cases, underscores and explicit type
annotations are unnecessary. Code rarely becomes problematic and needs
underscores or type annotations like the above to enforce the transformations.

How does the compiler create function values from function names? If the parameter
type of function \code{f} is \code{Int}, the corresponding function value is
\code{(x: Int) => f(x)}. The transformation is called \textit{eta expansion}.
\index{eta expansion}
\code{(x: Int) => f(x)} is a function value without a name and does the same thing as
\code{f}. The following section covers functions without names.

\section{Anonymous Functions}

In functional programming, functions often appear only once as an argument or a
return value. Naming functions used only once is unnecessary. The meaning of
a function value is how it behave. While the parameters and body of a function
decide its behavior, its name does not affect the behavior. Naturally, functional
languages provide syntax to define functions without giving them names. Such
functions are \textit{anonymous functions}.\index{anonymous function}

The syntax of an anonymous function in Scala is as follows:

\begin{verbatim}
([parameter name]: [parameter type], …) => [expression]
\end{verbatim}

Like functions declared by \code{def},
anonymous functions can be arguments, return values, or values referred by
variables. Directly calling them is possible as well.

\begin{verbatim}
def g(h: Int => Int): Int = h(0)
g((x: Int) => x)

def f(): Int => Int = (x: Int) => x
f()(0)

val h = (x: Int) => x
h(0)

((x: Int) => x)(0)
\end{verbatim}

The code does similar things to the previous code but uses anonymous functions.

Anonymous functions need explicit parameter types as named functions do. However,
annotating every parameter type is verbose and inconvenient. The Scala compiler
infers the types of parameters when anonymous functions occur where the compiler
expects function types.

\begin{verbatim}
def g(h: Int => Int): Int = h(0)
g(x => x)
\end{verbatim}

Since \code{g} has a parameter of type \code{Int => Int}, the compiler expects
\code{x => x} to have the type \code{Int => Int}. It infers the type of \code{x} as
\code{Int}.

\begin{verbatim}
val h: Int => Int = x => x
\end{verbatim}

\code{h} has an explicit type annotation. \code{Int => Int} is the expected type of
\code{x => x}. The compiler infers the type of \code{x} as \code{Int}.

\begin{verbatim}
val h = x => x
\end{verbatim}
\vspace{-1em}
\begin{mdframed}[hidealllines=true,backgroundcolor=red!10,innerleftmargin=3pt,innerrightmargin=3pt,leftmargin=-3pt,rightmargin=-3pt]
\begin{verbatim}
         ^
error: missing parameter type
\end{verbatim}
\vspace{-2em}
\begin{flushright}
\scriptsize\textsf{Compile-time error}
\end{flushright}
\end{mdframed}

Unlike previous one, this code is problematic. Since
there is no information to infer the type of \code{x} in \code{x => x},
the compiler generates an error.

Most cases using anonymous functions are arguments for function calls.
Those functions do not require explicit parameter types. However, beginners might not
be sure about whether parameter types can be omitted or not. Specifying
parameter types is safe when you are not sure.

Scala provides one more syntax for anonymous functions: syntax using
underscores. Underscores help programmers to create anonymous functions in a
concise and intuitive way.
Underscores can be used only when certain conditions are satisfied.
Every parameter must occur exactly once in the body of a function in the
order. Moreover, the function must not be an identity function like \code{(x: Int) => x}.
In functions satisfying the conditions, underscores can replace parameters in the body.
Otherwise, it is impossible to use underscores to create anonymous functions.

\begin{verbatim}
def g0(h: Int => Int): Int = h(0)
g0(_ + 1)

def g1(h: (Int, Int) => Int): Int = h(0, 0)
g1(_ + _)
\end{verbatim}

The compiler transforms \verb!_ + 1! into \code{x => x + 1}.
Similarly, \verb!_ + _!
becomes \code{(x, y) => x + y}. The compiler automatically creates parameters
as many as underscores and substitutes the underscores with the parameters. The
mechanism clearly shows why the aforementioned conditions exist.

\begin{verbatim}
val h0 = (_: Int) + 1
val h1 = (_: Int) + (_: Int)
\end{verbatim}

Underscores can have explicit types.
Programmers should supply parameter types to succeed compiling when
the compiler cannot infer them.

The transformation happens for the shortest expression containing underscores.
Expressing anonymous functions with underscores is sometimes tricky.

\begin{verbatim}
def f(x: Int): Int = x
def g1(h: Int => Int): Int = h(0)
g1(f(_))
\end{verbatim}

As intended, \verb!f(_)! becomes \code{x => f(x)}, whose type is \code{Int =>
Int}.\sidenote{Actually, there is no need to write
\code{g(f(\_))} because it is equal to \code{g(f)}.}

\begin{verbatim}
g1(f(_ + 1))
\end{verbatim}
\vspace{-1em}
\begin{mdframed}[hidealllines=true,backgroundcolor=red!10,innerleftmargin=3pt,innerrightmargin=3pt,leftmargin=-3pt,rightmargin=-3pt]
\begin{verbatim}
     ^
error: missing parameter type for expanded function
((<x$1: error>) => x$1.$plus(1))
\end{verbatim}
\vspace{-2em}
\begin{flushright}
\scriptsize\textsf{Compile-time error}
\end{flushright}
\end{mdframed}

On the other hand, \verb!f(_ + 1)! becomes
\code{f(x => x + 1)} but not \code{x => f(x + 1)}.
As \code{f} takes an integer, not a function, it results in a compile-time error.

\begin{verbatim}
def g2(h: (Int, Int) => Int): Int = h(0, 0)
g2(f(_) + _)
\end{verbatim}

\verb!f(_) + _! becomes \code{(x, y) => f(x) + y}, whose type is \code{(Int,
Int) => Int}, and the compilation succeeds.

\begin{verbatim}
g2(f(_ + 1) + _)
\end{verbatim}
\vspace{-1em}
\begin{mdframed}[hidealllines=true,backgroundcolor=red!10,innerleftmargin=3pt,innerrightmargin=3pt,leftmargin=-3pt,rightmargin=-3pt]
\begin{verbatim}
              ^
error: missing parameter type for expanded function
((<x$2: error>) => f(((<x$1: error>) => x$1.$plus(1)))
  .<$plus: error>(x$2))
\end{verbatim}
\vspace{-2em}
\begin{flushright}
\scriptsize\textsf{Compile-time error}
\end{flushright}
\end{mdframed}

\verb!f(_ + 1) + _! becomes
\code{y => f(x => x + 1) + y} but not \code{(x, y) => f(x + 1) + y}.

Like type inference of parameter types, novices may not be sure about how
anonymous functions with underscores are transformed. It is recommended to use normal anonymous
functions without underscores for those who are not confident about the mechanism
of underscores.

\section{Closures}

\textit{Closures}\index{closure} are function values that capture
environments, which store the values of existing variables, when they are defined.
The bodies of closures may
have variables not defined in themselves, and the environments store the values
of those variables.

\begin{verbatim}
def makeAdder(x: Int): Int => Int = {
  def adder(y: Int): Int = x + y
  adder
}
\end{verbatim}

The definition of \code{adder}, \code{def adder(y: Int): Int = x + y}, does not
define but uses \code{x}. However, the code is correct.

\begin{verbatim}
val add1 = makeAdder(1)
assert(add1(2), 3)

val add2 = makeAdder(2)
assert(add2(2), 4)
\end{verbatim}

\code{add1} and \code{add2} refer to the same \code{adder} function, but the
former returns an integer one larger than an argument, and the latter returns an
integer two larger than an argument. The results of \code{add1(2)} and
\code{add2(2)} are \code{3} and \code{4}, respectively. It is possible because the
closures capture the environments when they are created. \code{add1} refers to a
thing like \code{(adder, x = 1)} instead of just \code{adder}. Similarly,
\code{add2} is actually \code{(adder, x = 2)}. Since the environment of
\code{add1} stores the fact that \code{x} is \code{1}, \code{add1(2)} results in
\code{3}. Under the environment of \code{add2}, \code{x} denotes \code{2}, and
thus \code{x + y} is \code{4} when \code{y} is \code{2}.

\section{First-Class Functions and Lists}

This section shows how first-class functions allow generalization of the functions
defined in \refch{immutability}.

\begin{verbatim}
def inc1(l: List[Int]): List[Int] = l match {
  case Nil => Nil
  case h :: t => h + 1 :: inc1(t)
}

def square(l: List[Int]): List[Int] = l match {
  case Nil => Nil
  case h :: t => h * h :: square(t)
}
\end{verbatim}

\code{inc1} increases every element of a given list by one, and \code{square}
squares every element. The two functions are remarkably similar. To make the
similarity clearer, let us rename the functions to \code{g}.

\begin{verbatim}
def g(l: List[Int]): List[Int] = l match {
  case Nil => Nil
  case h :: t => h + 1 :: g(t)
}

def g(l: List[Int]): List[Int] = l match {
  case Nil => Nil
  case h :: t => h * h :: g(t)
}
\end{verbatim}

The only difference is the left operand of \code{::} in the third line:
\code{h + 1} versus \code{h * h}. By adding one parameter, the functions become
entirely identical.

\begin{verbatim}
def g(l: List[Int], f: Int => Int): List[Int] = l match {
  case Nil => Nil
  case h :: t => f(h) :: g(t, f)
}
g(l, h => h + 1)

def g(l: List[Int], f: Int => Int): List[Int] = l match {
  case Nil => Nil
  case h :: t => f(h) :: g(t, f)
}
g(l, h => h * h)
\end{verbatim}

This function is called \code{map}. The returned list
has elements obtained by \textbf{map}ping a given function to the elements of a
given list.

\begin{verbatim}
def map(l: List[Int], f: Int => Int): List[Int] = l match {
  case Nil => Nil
  case h :: t => f(h) :: map(t, f)
}
\end{verbatim}

\code{inc1} and \code{square} can be redefined with \code{map}.

\begin{verbatim}
def inc1(l: List[Int]): List[Int] = map(l, h => h + 1)
def square(l: List[Int]): List[Int] = map(l, h => h * h)
\end{verbatim}

An underscore makes \code{inc1} conciser.

\begin{verbatim}
def inc1(l: List[Int]): List[Int] = map(l, _ + 1)
\end{verbatim}

Let us compare \code{odd} and \code{positive}.

\begin{verbatim}
def odd(l: List[Int]): List[Int] = l match {
  case Nil => Nil
  case h :: t =>
    if (h % 2 != 0)
      h :: odd(t)
    else
      odd(t)
}

def positive(l: List[Int]): List[Int] = l match {
  case Nil => Nil
  case h :: t =>
    if (h > 0)
      h :: positive(t)
    else
      positive(t)
}
\end{verbatim}

They look similar. They can become identical by renaming and adding parameters.

\begin{verbatim}
def filter(l: List[Int], f: Int => Boolean): List[Int] = l match {
  case Nil => Nil
  case h :: t =>
    if (f(h))
      h :: filter(t, f)
    else
      filter(t, f)
}
\end{verbatim}


The function is called \code{filter} because it \textbf{filter}s
unwanted elements out from a given list.

\code{odd} and \code{positive} can be redefined with \code{filter}.

\begin{verbatim}
def odd(l: List[Int]): List[Int] =
  filter(l, h => h % 2 != 0)
def positive(l: List[Int]): List[Int] =
  filter(l, h => h > 0)
\end{verbatim}

Underscores make the functions conciser.

\begin{verbatim}
def odd(l: List[Int]): List[Int] = filter(l, _ % 2 != 0)
def positive(l: List[Int]): List[Int] = filter(l, _ > 0)
\end{verbatim}

Let us compare \code{sum} and \code{product} without tail recursion.

\begin{verbatim}
def sum(l: List[Int]): Int = l match {
  case Nil => 0
  case h :: t => h + sum(t)
}

def product(l: List[Int]): Int = l match {
  case Nil => 1
  case h :: t => h * product(t)
}
\end{verbatim}

After renaming the names to \code{g}, two differences exist: \code{0} versus
\code{1} and \code{h + g(t)} versus \code{h * g(t)}. By adding two parameters, an
initial value and a function taking \code{h} and \code{g(t)} as arguments, the
functions become identical.

\begin{verbatim}
def foldRight(
  l: List[Int],
  n: Int,
  f: (Int, Int) => Int
): Int = l match {
  case Nil => n
  case h :: t => f(h, foldRight(t, n, f))
}
\end{verbatim}

This function is called \code{foldRight} since it
appends an initial value at the right side of a list and
\textbf{fold}s the list from the \textbf{right} side with a given function.

\code{sum} and \code{product} can be redefined with \code{foldRight}.

\begin{verbatim}
def sum(l: List[Int]): Int =
  foldRight(l, 0, (h, gt) => h + gt)
def product(l: List[Int]): Int =
  foldRight(l, 1, (h, gt) => h * gt)
\end{verbatim}

They may use underscores for conciseness.

\begin{verbatim}
def sum(l: List[Int]): Int = foldRight(l, 0, _ + _)
def product(l: List[Int]): Int = foldRight(l, 1, _ * _)
\end{verbatim}

The following equations give an intuitive interpretation of \code{foldRight}:

\begin{verbatim}
  foldRight(List(a, b, .., y, z), n, f)
= f(a, f(b, .. f(y, f(z, n)) .. ))

  foldRight(List(1, 2, 3), 0, add)
= add(1, add(2, add(3, 0)))

  foldRight(List(1, 2, 3), 1, mul)
= mul(1, mul(2, mul(3, 1)))
\end{verbatim}

Let us compare tail-recursive \code{sum} and \code{product}.

\begin{verbatim}
def sum(l: List[Int]): Int = {
  def aux(l: List[Int], inter: Int): Int = l match {
    case Nil => inter
    case h :: t => aux(t, inter + h)
  }
  aux(l, 0)
}

def product(l: List[Int]): Int = {
  def aux(l: List[Int], inter: Int): Int = l match {
    case Nil => inter
    case h :: t => aux(t, inter * h)
  }
  aux(l, 1)
}
\end{verbatim}

After renaming, there are two differences: \code{inter + h} versus \code{inter * h}
and \code{0} versus \code{1}. Similarly, adding two parameters makes the
functions identical.

\begin{verbatim}
def foldLeft(
  l: List[Int],
  n: Int,
  f: (Int, Int) => Int
): Int = {
  def aux(l: List[Int], inter: Int): Int = l match {
    case Nil => inter
    case h :: t => aux(t, f(inter, h))
  }
  aux(l, n)
}
\end{verbatim}

This function is called \code{foldLeft}. Its semantics is different from
\code{foldRight}. While \code{foldRight} appends an initial value at
the right side and folds a list from the right side, \code{foldLeft}
prepends an initial value at the left side and \textbf{fold}s
a list from the \textbf{left} side. The following equations give an intuitive
interpretation:

\begin{verbatim}
  foldLeft(List(a, b, .., y, z), n, f)
= f(f( .. f(f(n, a), b), .. , y), z)

  foldLeft(List(1, 2, 3), 0, add)
= add(add(add(0, 1), 2), 3)

  foldLeft(List(1, 2, 3), 1, mul)
= mul(mul(mul(1, 1), 2), 3)
\end{verbatim}

The order traversing a list does not affect the results of \code{sum} and
\code{product}. Both \code{foldRight} and \code{foldLeft} can express the functions.

\begin{verbatim}
def sum(l: List[Int]): Int = foldLeft(l, 0, _ + _)
def product(l: List[Int]): Int = foldLeft(l, 1, _ * _)
\end{verbatim}

On the other hand, the order is important for some functions.
Consider a function that takes a list of digits as arguments and returns the
decimal number obtained by concatenating the digits.
\code{foldLeft} is the easiest way to implement this function.

\begin{verbatim}
def digitToDecimal(l: List[Int]) =
  foldLeft(l, 0, _ * 10 + _)

  foldLeft(List(1, 2, 3), 0, f)
= f(f(f(0, 1), 2), 3)
= ((0 * 10 + 1) * 10 + 2) * 10 + 3
= (1 * 10 + 2) * 10 + 3
= 12 * 10 + 3
= 123
\end{verbatim}

Using \code{foldRight} with the same arguments will yield completely different
result.

\begin{verbatim}
def digitToDecimal(l: List[Int]) =
  foldRight(l, 0, _ * 10 + _)

  foldRight(List(1, 2, 3), 0, f)
= f(1, f(2, f(3, 0)))
= 1 * 10 + (2 * 10 + (3 * 10 + 0))
= 1 * 10 + (2 * 10 + 30)
= 1 * 10 + 50
= 60
\end{verbatim}

\code{map}, \code{filter}, \code{foldRight}, and
\code{foldLeft} are powerful functions. The four functions offer concise
implementation for many procedures dealing with lists.
Since they are so useful, the Scala standard library provides \code{map},
\code{filter}, \code{foldRight}, and \code{foldLeft} as the methods of the
\code{List} class. You do not need to implement \code{map},
\code{filter}, \code{foldRight}, and \code{foldLeft} by yourself.

\code{map(l, f)} can be rewritten to \code{l.map(f)} by using the \code{map}
method instead.

\begin{verbatim}
def inc1(l: List[Int]): List[Int] = l.map(_ + 1)
def square(l: List[Int]): List[Int] = l.map(h => h * h)
\end{verbatim}

\code{filter(l, f)} can be rewritten to \code{l.filter(f)} by using the
\code{filter} method instead.

\begin{verbatim}
def odd(l: List[Int]): List[Int] = l.filter(_ % 2 != 0)
def positive(l: List[Int]): List[Int] = l.filter(_ > 0)
\end{verbatim}

\code{foldRight(l, n, f)} can be rewritten to \code{l.foldRight(n)(f)} by using the
\code{foldRight} method instead.

\begin{verbatim}
def sum(l: List[Int]): Int = l.foldRight(0)(_ + _)
def product(l: List[Int]): Int = l.foldRight(1)(_ * _)
\end{verbatim}

\code{foldLeft(l, n, f)} can be rewritten to \code{l.foldLeft(n)(f)} by using the
\code{foldLeft} method instead.

\begin{verbatim}
def sum(l: List[Int]): Int = l.foldLeft(0)(_ + _)
def product(l: List[Int]): Int = l.foldLeft(1)(_ * _)
def digitToDecimal(l: List[Int]) = l.foldLeft(0)(_ * 10 + _)
\end{verbatim}

The methods in the standard library are polymorphic, i.e. they can take
arguments of various types. For example, our \code{map} function takes only a
list of integers. To use \code{map} for a list of students, we need to define a
new version of \code{map}. However, the \code{map} method in the standard
library can take lists of any types as arguments.

\begin{verbatim}
case class Student(name: String, height: Int)

def heights(l: List[Student]): List[Int] = l.map(_.height)
\end{verbatim}

The standard library provides many other useful methods for
lists.\sidenote{\url{https://www.scala-lang.org/api/current/scala/collection/immutable/List.html}}

\section{For Loops}

Scala has for loops.
In Scala, a for loop is an expression, which evaluates to a value.
For expressions are highly expressive.
Unlike \code{while}, which work with mutable variables or objects,
\code{for} of Scala helps programmers to write code in a functional and readable way.

The syntax of a for expression is as follows:

\begin{verbatim}
for ([name] <- [expression])
  yield [expression]
\end{verbatim}

The first expression should result in a collection.
Its result is a collection containing the result of evaluating the second expression
at each iteration.
Therefore, for expressions can replace use of the \code{map} method.

\begin{verbatim}
val l = for (n <- List(0, 1, 2)) yield n * n
assert(l == List(0, 1, 4))
\end{verbatim}

For expressions can appear at any places expecting expressions.

\begin{verbatim}
def square(l: List[Int]): List[Int] =
  for (n <- l)
    yield n * n
\end{verbatim}

In Scala, \code{for} is just syntactic sugar. Instead of giving specific
semantics to \code{for}, syntactic rules transform code using \code{for} into the
code using methods of collections and anonymous functions. The above function becomes
the following function, which uses \code{map}, by the transformation:

\begin{verbatim}
def square(l: List[Int]): List[Int] =
  l.map(n => n * n)
\end{verbatim}

For this reason, for expressions are powerful. Any
user-defined types can appear in for expressions if the
types define \code{map}.

For expressions can replace use of the \code{filter} method as well.

\begin{verbatim}
def positive(l: List[Int]): List[Int] =
  for (n <- l if n > 0)
    yield n
\end{verbatim}

Elements not satisfying a given condition will be omitted during iteration.

In addition, combination of \code{map} and \code{filter} can be expressed with a
for loop concisely. Consider a function that takes a list of students and
returns a list of the names of students whose height is greater than 170.
The function can be implemented with \code{map} and \code{filter} like below.

\begin{verbatim}
def tall(l: List[Student]): List[String] =
  l.filter(_.height > 170).map(_.name)
\end{verbatim}

We can use a for expression instead.

\begin{verbatim}
def tall(l: List[Student]): List[String] =
  for (s <- l if s.height > 170)
    yield s.name
\end{verbatim}

\section{Exercises}

\begin{exercise}
\labex{functions-incby}

Implement a function \code{incBy}:
\begin{verbatim}
def incBy(l: List[Int], n: Int): List[Int] = ???
\end{verbatim}
that takes a list of integers and an integer as
arguments and increases every element of the list by the given integer. Use
the \code{map} method.

\end{exercise}

\begin{exercise}
\labex{functions-gt}

Implement a function \code{gt}:
\begin{verbatim}
def gt(l: List[Int], n: Int): List[Int] = ???
\end{verbatim}
that takes a list of integers and an integer as arguments
and filters elements less than or equal to the given integer out from the list.
Use the \code{filter} method.

\end{exercise}

\begin{exercise}
\labex{functions-append}

Implement a function \code{append}:
\begin{verbatim}
def append(l: List[Int], n: Int]): List[Int] = ???
\end{verbatim}
that takes a list of integers and an integer as arguments
and returns a list obtained by appending the integer at the end of the list.
Use the \code{foldRight} method.

\end{exercise}

\begin{exercise}
\labex{functions-reverse}

Implement a function \code{reverse}:
\begin{verbatim}
def reverse(l: List[Int]): List[Int] = ???
\end{verbatim}
that takes a list of integers
and returns a list obtained by reversing the order between the elements.
Use the \code{foldLeft} method.

\end{exercise}

\setchapterpreamble[u]{\margintoc}
\chapter{Pattern Matching}
\labch{pattern-matching}

This section explains pattern matching of Scala.
Pattern matching is one of the key features of functional programming.
It helps programmers handle complex, but structured data.
We have already used a simple form of pattern matching for lists.
This section discusses the benefits
of pattern matching and various patterns available in Scala.
In addition, it will introduce the option type, which is widely-used in
functional programming.

\section{Algebraic Data Types}

It is common to include values of various shapes in a single type.

A natural number is

\begin{itemize}
\item zero or
\item the successor of a natural number.
\end{itemize}

A list is

\begin{itemize}
\item the empty list or
\item a pair of an element and a list.
\end{itemize}

A binary tree is

\begin{itemize}
\item the empty tree or
\item a tree containing a root element and two child trees.
\end{itemize}

An arithmetic expression is

\begin{itemize}
\item a number,
\item the sum of two arithmetic expressions, or
\item the difference of two arithmetic expressions.
\end{itemize}

An expression of the lambda calculus is

\begin{itemize}
\item a variable,
\item a function, which is a pair of a variable and an expression, or
\item a function application, which is a pair of two expressions.
\end{itemize}

As the examples show, in computer science, a single type often includes values of
various shapes. \textit{Algebraic data types}\index{algebraic data type}
(\acrshort{adtLabel}) express such types. An ADT
is the sum type of product types. That is why it is called ``algebraic.''
A \textit{product type}\index{product type}
is a type whose every element is an enumeration of values of types in the
same specific order. Tuple types are typical product types. A \textit{sum
type}\index{sum type}, whose
another name is a \textit{tagged union type}\index{tagged union type},
has values of multiple types as its values. Unlike a union type,
each component of a sum type has a
tag to be distinguished from other components.
In an ADT, one form of values that can be distinguished from the other forms is
called a \textit{variant}\index{variant}.

For example, an arithmetic expression, which has three variants, is

\begin{itemize}
\item a number,
\item the sum of two arithmetic expressions, or
\item the difference of two arithmetic expressions.
\end{itemize}

Therefore, we can define the \code{AE} type, which is the type of an arithmetic
expression, as the sum type of

\begin{itemize}
\item the \code{Int} type tagged with \code{Num},
\item the \code{AE * AE} type tagged with \code{Add}, and
\item the \code{AE * AE} type tagged with \code{Sub},
\end{itemize}

where \code{AE * AE} denotes the product type of \code{AE} and \code{AE}.

ADTs are common in functional languages. Most functional
languages allow users to define new ADTs. The following OCaml
code defines arithmetic expressions:

\begin{verbatim}
type ae =
| Num of int
| Add of ae * ae
| Sub of ae * ae
\end{verbatim}

Scala does not provide a direct way to define ADTs. Instead, Scala provides
traits and classes, which are more general mechanisms to define new types,
and programmers can express ADTs with traits and classes.

A new type can be defined as a trait.
The syntax of a trait definition is as follows:

\begin{verbatim}
trait [name]
\end{verbatim}

It defines a type whose name is \code{[name]}.
The following code defines the \code{AE} type,
which is the type of arithmetic expressions:

\begin{verbatim}
sealed trait AE
\end{verbatim}

The \code{sealed} modifier prevents \code{AE} being extended outside the file
that defines \code{AE}. We will get back to this point when we discuss the
exhaustivity checking of pattern matching.

Once a type is defined as a trait, the type can be used just like any other
types. For example, we can define an identity function for arithmetic
expressions.

\begin{verbatim}
def identity(ae: AE): AE = ae
\end{verbatim}

However, traits do not have ability to construct new values. It means that there
is no way to create a value of the type \code{AE} yet. We need to define the
variants of \code{AE} as case classes by extending \code{AE}.

\begin{verbatim}
case class Num(value: Int) extends AE
case class Add(left: AE, right: AE) extends AE
case class Sub(left: AE, right: AE) extends AE
\end{verbatim}

As you have seen in \refch{introduction-to-scala}, we can easily create values of case classes.

\begin{verbatim}
val n = Num(10)
val m = Num(5)
val e1 = Add(n, m)
val e2 = Sub(e1, Num(3))
\end{verbatim}

Like traits, case classes also define types. The name of each class is the name
of the defined type. Every instance of a class belongs to the type corresponding
to the class.

\begin{verbatim}
val n: Num = Num(10)
val m: Num = Num(5)
val e1: Add = Add(n, m)
val e2: Sub = Sub(e1, Num(3))
\end{verbatim}

In addition, because of the \code{extends} keyword, \code{Num}, \code{Add}, and
\code{Sub} are subtypes of \code{AE}. It means that any value of the types
\code{Num}, \code{Add}, or \code{Sub} is also a value of the type \code{AE}.

\begin{verbatim}
val n: AE = Num(10)
val m: AE = Num(5)
val e1: AE = Add(n, m)
val e2: AE = Sub(e1, Num(3))
\end{verbatim}

We know that we can access the fields of objects with their names.

\begin{verbatim}
val n: Num = Num(10)
n.value
\end{verbatim}

However, we cannot access the fields of an object when its type becomes \code{AE}.

\begin{verbatim}
val m: AE = Num(10)
m.value
\end{verbatim}
\vspace{-1em}
\begin{mdframed}[hidealllines=true,backgroundcolor=red!10,innerleftmargin=3pt,innerrightmargin=3pt,leftmargin=-3pt,rightmargin=-3pt]
\begin{verbatim}
  ^
error: value value is not a member of AE
\end{verbatim}
\vspace{-2em}
\begin{flushright}
\scriptsize\textsf{Compile-time error}
\end{flushright}
\end{mdframed}

The reason is that \code{m} can be \code{Add} or \code{Sub}, which do not have
the field \code{value}, as \code{AE} consists of not only \code{Num} but also
\code{Add} and \code{Sub}. The compiler thinks that \code{m} may not have the
field \code{value} and considers \code{m.value} as an unsafe expression, which
should be rejected.

The best way to use ADTs is pattern matching. The following function evaluates a
given arithmetic expression and returns the number denoted by the arithmetic
expression.

\begin{verbatim}
def eval(e: AE): Int = e match {
  case Num(n) => n
  case Add(l, r) => eval(l) + eval(r)
  case Sub(l, r) => eval(l) - eval(r)
}

assert(eval(Sub(Add(Num(3), Num(7)), Num(5))) == 5)
\end{verbatim}

When \code{e} is \code{Num(n)}, \code{eval} simply returns \code{n}.
When \code{e} is \code{Add(l, r)}, \code{eval} respectively evaluates \code{l}
and \code{r}, which are arithmetic expressions. \code{eval} returns the sum of
the results of \code{l} and \code{r}.
The \code{Sub(l, r)} case is similar. \code{eval} returns the difference of
the results of \code{l} and \code{r}.

The list type is another good example of an ADT. The Scala standard library defines
lists similar to the following code:

\begin{verbatim}
sealed trait List[+A]
case object Nil extends List[Nothing]
case class ::[A](head: A, tail: List[A]) extends List[A]
\end{verbatim}

This code omits some details but clearly shows the high-level idea to define
lists.\sidenote{We will not see what \code{[+A]} and \code{Nothing} are here.
You can understand the overall ADT structure without knowing those concepts.}
A list is either the empty
list or a nonempty list, which is a pair of its head and tail. \code{Nil} is
defined as a case object, not a case class, since there is only one empty list.
Every empty list is identical. We use a case object to express this idea.
\code{Nil} is created only once during entire execution, and every \code{Nil} is
identitcal. The name \code{::} looks a bit weird, but it is for
readability of pattern matching. Scala allows writing class names as infix
operators in patterns. It means that both \code{case ::(h, t) =>} and \code{case
h :: t =>} are allowed. Due to the class name \code{::}, we can write \code{case
h :: t =>} in pattern matching.

\section{Advantages}

\subsection{Conciseness}

Without pattern matching, handling ADTs becomes a complicated job. We need to
use dynamic type tests to distinguish variants and type casting to access the
fields of values.

Below is \code{eval} without pattern matching:

\begin{verbatim}
def eval(e: AE): Int =
  if (e.isInstanceOf[Num])
    e.asInstanceOf[Num].value
  else if (e.isInstanceOf[Add]) {
    val e0 = e.asInstanceOf[Add]
    eval(e0.left) + eval(e0.right)
  } else {
    val e0 = e.asInstanceOf[Sub]
    eval(e0.left) - eval(e0.right)
  }
\end{verbatim}

\code{e.isInstanceOf[Num]} tests whether \code{e} is an instance of class
\code{Num}. If it is true, \code{eval} should return the value of the field
\code{value} of \code{e}. However, \code{value} cannot be directly accessed
since \code{e}'s type is \code{AE}. Because \code{e.isInstanceOf[Num]} is true,
we are sure that \code{e}'s actual type is \code{Num}. In this case, we can
inform this knowledge to the compiler with type casting. \code{e.asInstanceOf[Num]}
does not change the value denoted by \code{e} but lets the compiler know that
the programmer guarantees the type of \code{e} to be \code{Num}. Therefore, the
compiler considers \code{e.asInstanceOf[Num]} to belong \code{Num} and allows
accessing the field \code{value}. These type tests and casting processes should
be done for the other variants, \code{Add} and \code{Sub}, too.

The code is long and complicated despite its simple functionality. Dynamic type
tests and explicit type casting occupy most of the code, while real
computation requires short code. Besides, such code is error-prone.
For example, programmers may write code like below by mistake:

\begin{verbatim}
else if (e.isInstanceOf[Add]) {
  val e0 = e.asInstanceOf[Sub]
  eval(e0.left) + eval(e0.right)
}
\end{verbatim}

While the condition checks whether \code{e} is an instance of \code{Add},
\code{e} becomes casted to \code{Sub}. Such code will trigger an error at run
time and terminate the execution abnormally.
It is easy to check whether \code{eval} is correct because it is short.
However, complex types and computation will increase the possibility of
mistakes.

Pattern matching gives us a much better solution. Pattern matching hides type
tests and casting and makes code concise. At the same time, pattern matching
removes the possibility of mistakes.

\subsection{Exhaustivity Checking}

Pattern matching checks the exhaustivity of patterns. At run time, a match error
occurs when a given value matches none of given patterns.

\begin{verbatim}
def eval(e: AE): Int = e match {
  case Add(l, r) => eval(l) + eval(r)
  case Sub(l, r) => eval(l) - eval(r)
}
\end{verbatim}

The function lacks the \code{Num} pattern.

\begin{verbatim}
eval(Num(3))
\end{verbatim}
\vspace{-1em}
\begin{mdframed}[hidealllines=true,backgroundcolor=red!10,innerleftmargin=3pt,innerrightmargin=3pt,leftmargin=-3pt,rightmargin=-3pt]
\begin{verbatim}
scala.MatchError: Num(3) (of class Num)
\end{verbatim}
\vspace{-2em}
\begin{flushright}
\scriptsize\textsf{Run-time error}
\end{flushright}
\end{mdframed}

An argument of type \code{Num} results in a match error at run time.

Fortunately, we can easily avoid such mistakes.
The Scala compiler checks whether patterns are exhaustive and warns if they are not.

\begin{verbatim}
def eval(e: AE): Int = e match {
  case Add(l, r) => eval(l) + eval(r)
  case Sub(l, r) => eval(l) - eval(r)
}
\end{verbatim}
\vspace{-1em}
\begin{mdframed}[hidealllines=true,backgroundcolor=yellow!10,innerleftmargin=3pt,innerrightmargin=3pt,leftmargin=-3pt,rightmargin=-3pt]
\begin{verbatim}
                       ^
warning: match may not be exhaustive.
It would fail on the following input: Num(_)
\end{verbatim}
\vspace{-2em}
\begin{flushright}
\scriptsize\textsf{Compile-time warning}
\end{flushright}
\end{mdframed}

The compiler warns programmers about that the patterns are not exhaustive. Moreover, it precisely
informs which patterns are missing to help debugging.
Exhaustivity checking is beneficial for complex programs. It helps programmers make
error-free programs and thus is a crucial strength of pattern matching.

For exhaustivity checking, the \code{sealed} modifier of traits is necessary.
Without \code{sealed}, a trait can be extended outside the file that defines it.
The unit of compilation is a single file, so it is impossible to find
all the variants by scanning a single file when a trait is not sealed.
Exhaustivity checking during pattern matching will be impossible.
The \code{sealed} keyword resolves the problem. Since sealed traits cannot be
extended further, it is enough to check only the file that defines a sealed trait to find
every variant of the trait. It is why we use sealed traits to define ADTs.

\subsection{Reachability Checking}

Like \code{switch-case}, pattern
matching compares a value to patterns sequentially from top to bottom and selects
the first matching pattern. If there are duplicated patterns, the latter will
not be reachable.
The compiler warns programmers when they find unreachable patterns to prevent such code.

\begin{verbatim}
def eval(e: AE): Int = e match {
  case Num(n) => 0
  case Add(l, r) => eval(l) + eval(r)
  case Num(n) => n
  case Sub(l, r) => eval(l) - eval(r)
}
\end{verbatim}
\vspace{-1em}
\begin{mdframed}[hidealllines=true,backgroundcolor=yellow!10,innerleftmargin=3pt,innerrightmargin=3pt,leftmargin=-3pt,rightmargin=-3pt]
\begin{verbatim}
  case Num(n) => n
                 ^
warning: unreachable code
\end{verbatim}
\vspace{-2em}
\begin{flushright}
\scriptsize\textsf{Compile-time warning}
\end{flushright}
\end{mdframed}

When code is simple and short, it is easy to check whether there are unreachable
patterns. However, in complex code, programmers often insert unreachable
patterns by mistake and make critical bugs. Reachability checking of the
compiler is an important feature to prevent such bugs.

\section{Patterns in Scala}

\subsection{Constant and Wildcard Patterns}

\code{switch-case} statements divide a given value into multiple cases in
imperative languages. Pattern matching is a general form of \code{switch-case}.
The following code is an example using a \code{switch-case} statement in Java:

\begin{verbatim}
String grade(int score) {
  switch (score / 10) {
    case 10: return "A";
    case 9: return "A";
    case 8: return "B";
    case 7: return "C";
    case 6: return "D";
    default: return "F";
  }
}
\end{verbatim}

Constant and wildcard patterns exist in Scala. Constant patterns are
literals like integers and strings. A constant pattern matches a given
value if a value denoted by the pattern equals the given value. The underscore
denotes the wildcard pattern, which matches every value, and is equivalent to
\code{default} of \code{switch-case}. The following function rewrites the
previous function with pattern matching:

\begin{verbatim}
def grade(score: Int): String =
  (score / 10) match {
    case 10 => "A"
    case 9 => "A"
    case 8 => "B"
    case 7 => "C"
    case 6 => "D"
    case _ => "F"
  }

assert(grade(85) == "B")
\end{verbatim}

\subsection{Or Patterns}

\code{switch-case} statements use the fall-through semantics; if \code{break}
does not exist, after executing code corresponding to a case, the flow of the
execution moves to code corresponding to the next case. Since the results of
cases \code{10} and \code{9} are identical, the function can use fall-through.

\begin{verbatim}
String grade(int score) {
  switch (score / 10) {
    case 10:
    case 9: return "A";
    case 8: return "B";
    case 7: return "C";
    case 6: return "D";
    default: return "F";
  }
}
\end{verbatim}

In contrast, pattern matching disallows fall-through. Instead, or patterns
give a way to write the same expression only once for multiple patterns. The
syntax of an or pattern is \code{[pattern] | [pattern] | …}, which is a sequence
of multiple patterns with vertical bars in between. \code{A | B} matches values
that match \code{A} or \code{B}.

\begin{verbatim}
def grade(score: Int): String =
  (score / 10) match {
    case 10 | 9 => "A"
    case 8 => "B"
    case 7 => "C"
    case 6 => "D"
    case _ => "F"
  }

assert(grade(100) == "A")
\end{verbatim}

\subsection{Nested Patterns}

Nested patterns are patterns containing patterns.
The \code{optimizeAdd} function
optimizes a given arithmetic expression by eliminating additions of zeros.

\begin{verbatim}
def optimizeAdd(e: AE): AE = e match {
  case Num(_) => e
  case Add(Num(0), r) => optimizeAdd(r)
  case Add(l, Num(0)) => optimizeAdd(l)
  case Add(l, r) => Add(optimizeAdd(l), optimizeAdd(r))
  case Sub(l, r) => Sub(optimizeAdd(l), optimizeAdd(r))
}
\end{verbatim}

Nested patterns help programmers treat values with complex structures easily.

\subsection{Patterns with Binders}

Assume that we have one more variant of \code{AE}:

\begin{verbatim}
case class Abs(e: AE) extends AE
\end{verbatim}

It denotes the absolute value of an operand.
Optimizing an arithmetic expression decorated by two
consecutive \code{Abs} operators results in the arithmetic expression with only
one \code{Abs} operator.

\begin{verbatim}
def optimizeAbs(e: AE): AE = e match {
  case Num(_) => e
  case Add(l, r) => Add(optimizeAbs(l), optimizeAbs(r))
  case Sub(l, r) => Sub(optimizeAbs(l), optimizeAbs(r))
  case Abs(Abs(e0)) => optimizeAbs(Abs(e0))
  case Abs(e0) => Abs(optimizeAbs(e0))
}
\end{verbatim}

A flaw of the implementation is that a value matching \code{Abs(e0)}
cannot be an argument of \code{optimizeAbs} directly, and constructing a new
\code{Abs} instance containing a value matching \code{e0} is essential.
The \code{@} symbol makes code efficient by binding a value matching to a pattern to a variable.
Pattern \code{[variable] @ [pattern]} makes the variable refer to a value
matching the pattern.

\begin{verbatim}
def optimizeAbs(e: AE): AE = e match {
  case Num(_) => e
  case Add(l, r) => Add(optimizeAbs(l), optimizeAbs(r))
  case Sub(l, r) => Sub(optimizeAbs(l), optimizeAbs(r))
  case Abs(e0 @ Abs(_)) => optimizeAbs(e0)
  case Abs(e0) => Abs(optimizeAbs(e0))
}
\end{verbatim}

\subsection{Type Patterns}

In \code{optimizeAbs},
the first \verb!Num(_)! pattern does no more than checking whether a value
belongs to type \code{Num}. A type pattern helps to rewrite the function. Type
patterns are in the form of \code{[name]: [type]}. If a value belongs to the
type, it matches the pattern, and the variable refers to the value. The wildcard
pattern can substitute the name if the variable is unnecessary.

\begin{verbatim}
def optimizeAbs(e: AE): AE = e match {
  case _: Num => e
  case Add(l, r) => Add(optimizeAbs(l), optimizeAbs(r))
  case Sub(l, r) => Sub(optimizeAbs(l), optimizeAbs(r))
  case Abs(e0 @ Abs(_)) => optimizeAbs(e0)
  case Abs(e0) => Abs(optimizeAbs(e0))
}
\end{verbatim}

Type patterns are useful for dynamic type checking. The following function takes
any value as an argument and check whether it is a string or
not.\sidenote{Every type is a subtype of \code{Any}, i.e. every value belongs to
\code{Any}.}

\begin{verbatim}
def show(x: Any): String = x match {
  case s: String => s + " is a string"
  case _ => "not a string"
}

assert(show("1") == "1 is a string")
assert(show(1) == "not a string")
\end{verbatim}

Note that type patterns cannot check type arguments of polymorphic types. Using
type patterns against polymorphic types is dangerous.

\begin{verbatim}
def show(x: Any): String = x match {
  case _: List[String] => "a list of strings"
  case _ => "not a list of strings"
}
\end{verbatim}
\vspace{-1em}
\begin{mdframed}[hidealllines=true,backgroundcolor=yellow!10,innerleftmargin=3pt,innerrightmargin=3pt,leftmargin=-3pt,rightmargin=-3pt]
\begin{verbatim}
          ^
warning: non-variable type argument String in type pattern
List[String] is unchecked since it is eliminated by erasure
\end{verbatim}
\vspace{-1.5em}
\begin{flushright}
\scriptsize\textsf{Compile-time warning}
\end{flushright}
\end{mdframed}

\begin{verbatim}
val l: List[Int] = List(1, 2, 3)
assert(show(l) == "a list of strings")  // weird result
\end{verbatim}

Although the type of the argument is \code{List[Int]}, it matches the first
pattern. As the warnings imply, the JVM uses type erasure
semantics, and type arguments are unavailable at run time.

\subsection{Tuple Patterns}

The syntax of a tuple pattern is \code{([pattern], …, [pattern])}.
It matches a tuple whose elements respectively match the internal patterns.

The following function uses tuple patterns to check
whether two lists are identical:

\begin{verbatim}
def equal(l0: List[Int], l1: List[Int]): Boolean =
  (l0, l1) match {
    case (h0 :: t0, h1 :: t1) =>
      h0 == h1 && equal(t0, t1)
    case (Nil, Nil) => true
    case _ => false
  }
\end{verbatim}

\subsection{Pattern Guards}

A binary search tree is

\begin{itemize}
\item the empty tree or
\item a tree containing an integral root element and two child trees.
\end{itemize}

\begin{verbatim}
sealed trait Tree
case object Empty extends Tree
case class Node(root: Int, left: Tree, right: Tree) extends Tree
\end{verbatim}

The function \code{add} takes a tree and an integer as arguments and returns a tree
obtained by adding the integer to the tree. If the integer is an element of the
given tree, the tree itself is the return value.

\begin{verbatim}
def add(t: Tree, n: Int): Tree =
  t match {
    case Empty => Node(n, Empty, Empty)
    case Node(m, t0, t1) =>
      if (n < m)
        Node(m, add(t0, n), t1)
      else if (n > m)
        Node(m, t0, add(t1, n))
      else
        t
  }
\end{verbatim}

An expression corresponding to the second pattern uses \code{if-else}. Pattern
guards allow adding constraints to patterns. A pattern in the form of
\code{[pattern] if [expression]} matches a value if the value matches the
pattern, and the expression results in \code{true}. The following version of \code{add}
uses pattern guards:

\begin{verbatim}
def add(t: Tree, n: Int): Tree =
  t match {
    case Empty => Node(n, Empty, Empty)
    case Node(m, t0, t1) if n < m =>
      Node(m, add(t0, n), t1)
    case Node(m, t0, t1) if n > m =>
      Node(m, t0, add(t1, n))
    case _ => t
  }
\end{verbatim}

Guarded patterns may be inexhaustive and need care.

\begin{verbatim}
def add(t: Tree, n: Int): Tree =
  t match {
    case Empty => Node(n, Empty, Empty)
    case Node(m, t0, t1) if n < m =>
      Node(m, add(t0, n), t1)
    case Node(m, t0, t1) if n > m =>
      Node(m, t0, add(t1, n))
  }
\end{verbatim}

The patterns in the above code is not exhaustive, but
the compiler does not warn programmers about the inexhaustivity.

\subsection{Patterns with Backticks}

The function \code{remove} takes a tree and an integer as arguments and returns a
tree obtained by removing the integer from the tree. If the integer is not an
element of the tree, the given tree itself is the return value. \code{removeMin}
is a helper function used by \code{remove}. It returns the pair of the smallest
element of a given tree and a tree obtained by removing the element from the
tree.

\begin{verbatim}
def removeMin(t: Tree): (Int, Tree) = {
  t match {
    case Node(n, Empty, t1) =>
      (n, t1)
    case Node(n, t0: Node, t1) =>
      val (min, t2) = removeMin(t0)
      (min, Node(n, t2, t1))
  }
}

def remove(t: Tree, n: Int): Tree = {
  t match {
    case Empty =>
      Empty
    case Node(m, t0, Empty) if n == m =>
      t0
    case Node(m, t0, t1: Node) if n == m =>
      val res = removeMin(t1)
      val min = res._1
      val t2 = res._2
      Node(min, t0, t2)
    case Node(m, t0, t1) if n < m =>
      Node(m, remove(t0, n), t1)
    case Node(m, t0, t1) if n > m =>
      Node(m, t0, remove(t1, n))
  }
}
\end{verbatim}

\verb!Node(`n`, t0, Empty)! can replace
\code{case Node(m, t0, Empty) if n == m}. The pattern \code{Node(n, t0, Empty)} defines
a new variable \code{n} and makes \code{n} refer to the
root element; it does not check whether the root element equals \code{n}.
However, backticks prohibit defining a new variable and allow to compare the root
element to \code{n} in the scope.

\begin{verbatim}
def remove(t: BinTree, n: Int): BinTree = {
  t match {
    case Empty =>
      Empty
    case Node(`n`, t0, Empty) =>
      t0
    case Node(`n`, t0, t1: Node) =>
      val res = removeMin(t1)
      val min = res._1
      val t2 = res._2
      Node(min, t0, t2)
    case Node(m, t0, t1) if n < m =>
      Node(m, remove(t0, n), t1)
    case Node(m, t0, t1) if n > m =>
      Node(m, t0, remove(t1, n))
  }
}
\end{verbatim}

\section{Applications of Pattern Matching}

\subsection{Variable Definitions}

It is possible to define variables with pattern matching.

\begin{verbatim}
val (n, m) = (1, 2)
assert(n == 1 && m == 2)

val (a, b, c) = ("a", "b", "c")
assert(a == "a" && b == "b" && c == "c")

val h :: t = List(1, 2, 3, 4)
assert(h == 1 && t == List(2, 3, 4))

val Add(l, r) = Add(Num(1), Num(2))
assert(l == Num(1) && r == Num(2))
\end{verbatim}

Pattern matching helps programmers declare variables concisely, but a match error occurs
when the pattern does not match the right-hand-side value. It is desirable to use
pattern matching only when there is a guarantee that the match succeeds. Since
a tuple pattern always matches a tuple value of the same length,
tuple patterns are widely used for variable definitions.

\subsection{Anonymous Functions}

The function \code{toSum} takes a list of pairs of two integers as arguments and
returns a list whose elements are the sums of the integers in the pairs.

\begin{verbatim}
def toSum(l: List[(Int, Int)]): List[Int] =
  l.map(p => p match {
    case (n, m) => n + m
  })

val l = List((0, 1), (2, 3), (3, 4))
assert(toSum(l) == List(1, 5, 7))
\end{verbatim}

The anonymous function directly uses parameter \code{p} as the target of the
pattern matching. Scala allows simplification of \verb!x => x match { … }! to
\verb!{ … }!. Therefore, we can use an enumeration of patterns as an anonymous
function.

\begin{verbatim}
def toSum(l: List[(Int, Int)]): List[Int] =
  l.map({ case (n, m) => n + m })
\end{verbatim}

\subsection{For Loops}

\code{toSum} can use a for expression instead of \code{map}.

\begin{verbatim}
def toSum(l: List[(Int, Int)]): List[Int] =
  for (p <- l)
    yield p match { case (n, m) => n + m }
\end{verbatim}

For expressions directly support pattern matching.

\begin{verbatim}
def toSum(l: List[(Int, Int)]): List[Int] =
  for ((n, m) <- l)
    yield n + m
\end{verbatim}

The code is readable and concise.

\section{Options}
\labsec{options}

The option type is a widely-used ADT. It represents a value whose existence is
optional. This section introduces the option type and explains the usage of
options.

Consider the function \code{get}, which takes a list and integer \code{n} as
arguments and returns the \code{n}th element of the list. It is problematic
when \code{n} is negative or exceeds the length of the list. Throwing exceptions
is a widely used solution in imperative languages. In Scala, \code{throw
[expression]} throws an exception. For convenience, we define the function
\code{error}, which throws an exception, like below and use it throughout the
book.

\begin{verbatim}
def error(msg: String) = throw new Exception(msg)

def get(l: List[Int], n: Int): Int =
  if (n < 0)
    error("index out of bounds")
  else l match {
    case Nil =>
      error("index out of bounds")
    case h :: t =>
      if (n == 0)
        h
      else
        get(t, n - 1)
  }
\end{verbatim}

Throwing an exception is a simple and effective solution. However, exceptions
have two problems. First, exceptions should be handled by exception handlers.

\begin{verbatim}
try {
  get(List(1, 2), 2)
} catch {
  case e: Exception =>
    // prints "index out of bounds"
    println(e.getMessage)
}
\end{verbatim}

If an exception is not handled properly, it will eventually cause a run-time
error and terminate the execution.

\begin{verbatim}
get(List(1, 2), 2)
\end{verbatim}
\vspace{-1em}
\begin{mdframed}[hidealllines=true,backgroundcolor=red!10,innerleftmargin=3pt,innerrightmargin=3pt,leftmargin=-3pt,rightmargin=-3pt]
\begin{verbatim}
java.lang.Exception: index out of bounds
\end{verbatim}
\vspace{-2em}
\begin{flushright}
\scriptsize\textsf{Run-time error}
\end{flushright}
\end{mdframed}

The Scala compiler does not check whether exceptions are handled properly.
It means that there will not be any compile-time error even if there is a
possibility of unhandled exceptions.

Another problem of exceptions is that exception handling is not local.
When an exception is thrown, the control flow suddenly jumps to the position of
the nearest exception handler. Non-local transition of the control flow usually
hinders programmers from understanding code.
Therefore, implementing \code{get} without exceptions is desirable.

The first attempt is to use \code{null}. \code{null} is a value that denotes that
it does not refer to any existing object. We can try to make \code{get} return
\code{null} when a given index is invalid.

\begin{verbatim}
def get(l: List[Int], n: Int): Int =
  if (n < 0)
    null
  else l match {
    case Nil => null
    case h :: t =>
      if (n == 0)
        h
      else
        get(t, n - 1)
  }
\end{verbatim}
\vspace{-1em}
\begin{mdframed}[hidealllines=true,backgroundcolor=red!10,innerleftmargin=3pt,innerrightmargin=3pt,leftmargin=-3pt,rightmargin=-3pt]
\begin{verbatim}
    null
    ^
error: an expression of type Null is ineligible
for implicit conversion

    case Nil => null
                ^
error: an expression of type Null is ineligible
for implicit conversion
\end{verbatim}
\vspace{-2em}
\begin{flushright}
\scriptsize\textsf{Run-time error}
\end{flushright}
\end{mdframed}

Unfortunately, \code{null} is not an element of \code{Int} in Scala.
The compiler rejects the code.
Even with the assumption that we can treat \code{null} as \code{Int},
using \code{null} is a bad solution. Dereferencing \code{null} causes a
run-time error, which is the well-known \code{NullPointerException}.
The compiler does not check whether \code{null} is dereferenced.
Therefore, using \code{null} is nothing better than using exceptions.
Use of \code{null} has been criticized enormously because \code{null} is extremely
error-prone. Even Tony Hoare, the inventor of \code{null}, said that inventing
\code{null} was a terrible mistake:

\begin{quote}
I call it my billion-dollar mistake. It was the invention of the null reference
in 1965.\sidenote{\url{https://en.wikipedia.org/wiki/Null\_pointer\#History}}
\end{quote}

The second attempt is to use a particular error-indicating value, e.g. \code{-1}.

\begin{verbatim}
def get(l: List[Int], n: Int): Int =
  if (n < 0)
    -1
  else l match {
    case Nil =>
      -1
    case h :: t =>
      if (n == 0)
        h
      else
        get(t, n - 1)
  }
\end{verbatim}

The strategy has an obvious problem. The caller cannot distinguish two
situations:
\begin{itemize}
  \item The list contains \code{-1}.
  \item The index is invalid.
\end{itemize}
Default values can be successful solutions for certain purposes but do not fit \code{get}.

Instead of using a fixed particular value in \code{get}, the caller can specify the default value.

\begin{verbatim}
def getOrElse(l: List[Int], n: Int, default: Int): Int =
  if (n < 0)
    default
  else l match {
    case Nil =>
      default
    case h :: t =>
      if (n == 0)
        h
      else
        getOrElse(t, n - 1, default)
}
\end{verbatim}

It works well when an appropriate default value
exists. However, when checking failures is per se important, the new strategy is
as bad as the previous strategy. There is no way to distinguish an element and
the default value.

Functional languages provide the option type to handle erroneous situations
safely. As the name implies, it represents an optional existence of a value.
The Scala standard library defines the option type like below.\sidenote{
We will not see what \code{[+A]} and \code{Nothing} are here.
You can understand the overall ADT structure without knowing those concepts.}

\begin{verbatim}
sealed trait Option[+A]
case object None extends Option[Nothing]
case class Some[A](value: A) extends Option[A]
\end{verbatim}

An option that may have a value of type \code{T} has type \code{Option[T]}.
An option is either \code{None} or \code{Some}.
\code{None} is a value that does not denote any value and similar
to \code{null}. It indicates a problematic situation. Like \code{Nil}, it is
defined as a case object because every \code{None} is identical. \code{Some} constructs a value that
denotes that a value exists. It is similar to a reference to a real object and
indicates that computation has succeeded.

The following code defines \code{getOption}, which returns an option.

\begin{verbatim}
def getOption(l: List[Int], n: Int): Option[Int] =
  if (n < 0)
    None
  else l match {
    case Nil =>
      None
    case h :: t =>
      if (n == 0)
        Some(h)
      else
        getOption(t, n - 1)
  }

assert(getOption(List(1, 2), 0) == Some(1))
assert(getOption(List(1, 2), 2) == None)
\end{verbatim}

For an invalid index, the return value is \code{None}. The caller can notice
that the operation has failed by \code{None}.
Otherwise, the function packs
an element inside \code{Some} to make the return value.

The Scala standard library uses options in many places. Various methods return options.
For example, \code{headOption} of a list returns \code{None} when the list is
empty. Otherwise, \code{Some} containing the head of the list is returned.

\begin{verbatim}
assert(List().headOption == None)
assert(List(1).headOption == Some(1))
\end{verbatim}

Also, \code{get} of a map returns \code{None} when the map does not have a given key.
Otherwise, \code{Some} containing the value corresponding to the key is
returned.

\begin{verbatim}
val m = Map(1 -> "one", 2 -> "two")
assert(m.get(0) == None)
assert(m.get(1) == Some("one"))
\end{verbatim}

Pattern matching allows programmers to deal with options by
distinguishing the \code{None} and \code{Some} cases. In addition, like the
methods of lists, options also provide methods to abstract common patterns.
We are going to see two methods: \code{map} and \code{flatMap}.

\code{map} can be used when we want to apply some computation only when the
previous computation has succeeded. \code{map} takes a single argument, which
must be a function. \code{opt.map(f)} results in \code{None} when \code{opt} is
\code{None}. If \code{opt} is \code{Some(v)}, then \code{opt.map(f)} evaluates
to \code{Some(f(v))}.

As an example, let us consider a map containing students.
Names are the keys, and students are the values. We want to find a student by a name and
get one's height only when the student exists. It can be implemented with
\code{map}.

\begin{verbatim}
def getHeight(
  m: Map[String, Student],
  name: String
): Option[Int] =
  m.get(name).map(_.height)
\end{verbatim}

If \code{m.get(name)} is \code{None}, then \code{m.get(name).map(\_.height)} also
is \code{None}. Otherwise, \code{m.get(name)} should be \code{Some(student)}, and
\code{m.get(name).map(\_.height)} will result in \code{Some(student.height)}.

In summary, \code{map} is useful when the computation consists of two steps, and
the first step can fail.

\code{flatMap} is similar to \code{map} but a bit different. It is useful when
the computation consists of two steps, and both steps can fail.
\code{flatMap} takes a single argument, which must be a function that returns an option.
\code{opt.flatMap(f)} results in \code{None} when \code{opt} is
\code{None}. If \code{opt} is \code{Some(v)}, then \code{opt.flatMap(f)} evaluates
to \code{f(v)}.

Let us consider a list of names and a map like before.
When the list is nonempty, we will find a student with the first name in the
list from the map. It is a typical application of \code{flatMap}.

\begin{verbatim}
def getStudent(
  l: List[String],
  m: Map[String, Student]
): Option[Student] =
  l.headOption.flatMap(m.get)
\end{verbatim}

The standard library provides many other useful methods for
options.\sidenote{\url{https://www.scala-lang.org/api/current/scala/Option.html}}


%----------------------------------------------------------------------------------------

\backmatter % Denotes the end of the main document content
\setchapterstyle{plain} % Output plain chapters from this point onwards

%----------------------------------------------------------------------------------------
%  BIBLIOGRAPHY
%----------------------------------------------------------------------------------------

% The bibliography needs to be compiled with biber using your LaTeX editor, or on the command line with 'biber main' from the template directory

\defbibnote{bibnote}{Here are the references in citation order.\par\bigskip} % Prepend this text to the bibliography
\printbibliography[heading=bibintoc, title=Bibliography, prenote=bibnote] % Add the bibliography heading to the ToC, set the title of the bibliography and output the bibliography note

%----------------------------------------------------------------------------------------
%  NOMENCLATURE
%----------------------------------------------------------------------------------------

% The nomenclature needs to be compiled on the command line with 'makeindex main.nlo -s nomencl.ist -o main.nls' from the template directory

% \nomenclature{$c$}{Speed of light in a vacuum inertial frame}
% \nomenclature{$h$}{Planck constant}

% \renewcommand{\nomname}{Notation} % Rename the default 'Nomenclature'
% \renewcommand{\nompreamble}{The next list describes several symbols that will be later used within the body of the document.} % Prepend this text to the nomenclature

% \printnomenclature % Output the nomenclature

%----------------------------------------------------------------------------------------
%  GREEK ALPHABET
%   Originally from https://gitlab.com/jim.hefferon/linear-algebra
%----------------------------------------------------------------------------------------

% \vspace{1cm}

% {\usekomafont{chapter}Greek Letters with Pronunciations} \\[2ex]
% \begin{center}
%   \newcommand{\pronounced}[1]{\hspace*{.2em}\small\textit{#1}}
%   \begin{tabular}{l l @{\hspace*{3em}} l l}
%     \toprule
%     Character & Name & Character & Name \\
%     \midrule
%     $\alpha$ & alpha \pronounced{AL-fuh} & $\nu$ & nu \pronounced{NEW} \\
%     $\beta$ & beta \pronounced{BAY-tuh} & $\xi$, $\Xi$ & xi \pronounced{KSIGH} \\
%     $\gamma$, $\Gamma$ & gamma \pronounced{GAM-muh} & o & omicron \pronounced{OM-uh-CRON} \\
%     $\delta$, $\Delta$ & delta \pronounced{DEL-tuh} & $\pi$, $\Pi$ & pi \pronounced{PIE} \\
%     $\epsilon$ & epsilon \pronounced{EP-suh-lon} & $\rho$ & rho \pronounced{ROW} \\
%     $\zeta$ & zeta \pronounced{ZAY-tuh} & $\sigma$, $\Sigma$ & sigma \pronounced{SIG-muh} \\
%     $\eta$ & eta \pronounced{AY-tuh} & $\tau$ & tau \pronounced{TOW (as in cow)} \\
%     $\theta$, $\Theta$ & theta \pronounced{THAY-tuh} & $\upsilon$, $\Upsilon$ & upsilon \pronounced{OOP-suh-LON} \\
%     $\iota$ & iota \pronounced{eye-OH-tuh} & $\phi$, $\Phi$ & phi \pronounced{FEE, or FI (as in hi)} \\
%     $\kappa$ & kappa \pronounced{KAP-uh} & $\chi$ & chi \pronounced{KI (as in hi)} \\
%     $\lambda$, $\Lambda$ & lambda \pronounced{LAM-duh} & $\psi$, $\Psi$ & psi \pronounced{SIGH, or PSIGH} \\
%     $\mu$ & mu \pronounced{MEW} & $\omega$, $\Omega$ & omega \pronounced{oh-MAY-guh} \\
%     \bottomrule
%   \end{tabular} \\[1.5ex]
%   Capitals shown are the ones that differ from Roman capitals.
% \end{center}

%----------------------------------------------------------------------------------------
%  GLOSSARY
%----------------------------------------------------------------------------------------

% The glossary needs to be compiled on the command line with 'makeglossaries main' from the template directory

\newacronym[longplural={algebraic data types}]{adtLabel}{ADT}{algebraic data type}
\newacronym[longplural={real-eval-print loop}]{replLabel}{REPL}{read-eval-print loop}
\newacronym[longplural={Java Virtual Machines}]{jvmLabel}{JVM}{Java Virtual Machine}
\newacronym[longplural={Backus-Naur forms}]{bnfLabel}{BNF}{Backus-Naur form}
\newacronym[longplural={abstract syntax trees}]{astLabel}{AST}{abstract syntax tree}
\newacronym[longplural={call-by-value}]{cbvLabel}{CBV}{call-by-value}
\newacronym[longplural={call-by-reference}]{cbrLabel}{CBR}{call-by-reference}
\newacronym[longplural={call-by-name}]{cbnLabel}{CBN}{call-by-name}
\newacronym[longplural={continuation-passing style}]{cpsLabel}{CPS}{continuation-passing style}
\newacronym[longplural={garbage collection}]{gcLabel}{GC}{garbage collection}
\newacronym[longplural={last in, first out}]{lifoLabel}{LIFO}{last in, first out}
\newacronym[longplural={use-after-free}]{uafLabel}{UAF}{use-after-free}

\setglossarystyle{listgroup} % Set the style of the glossary (see https://en.wikibooks.org/wiki/LaTeX/Glossary for a reference)
\printglossary[title=Special Terms, toctitle=List of Terms] % Output the glossary, 'title' is the chapter heading for the glossary, toctitle is the table of contents heading

%----------------------------------------------------------------------------------------
%  INDEX
%----------------------------------------------------------------------------------------

% The index needs to be compiled on the command line with 'makeindex main' from the template directory

\printindex % Output the index

%----------------------------------------------------------------------------------------
%  BACK COVER
%----------------------------------------------------------------------------------------

% If you have a PDF/image file that you want to use as a back cover, uncomment the following lines

%\clearpage
%\thispagestyle{empty}
%\null%
%\clearpage
%\includepdf{cover-back.pdf}

%----------------------------------------------------------------------------------------

\end{document}
