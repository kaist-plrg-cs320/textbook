%%%%%%%%%%%%%%%%%%%%%%%%%%%%%%%%%%%%%%%%%
% kaobook
% LaTeX Template
% Version 1.2 (4/1/2020)
%
% This template originates from:
% https://www.LaTeXTemplates.com
%
% For the latest template development version and to make contributions:
% https://github.com/fmarotta/kaobook
%
% Authors:
% Federico Marotta (federicomarotta@mail.com)
% Based on the doctoral thesis of Ken Arroyo Ohori (https://3d.bk.tudelft.nl/ken/en)
% and on the Tufte-LaTeX class.
% Modified for LaTeX Templates by Vel (vel@latextemplates.com)
%
% License:
% CC0 1.0 Universal (see included MANIFEST.md file)
%
%%%%%%%%%%%%%%%%%%%%%%%%%%%%%%%%%%%%%%%%%

%----------------------------------------------------------------------------------------
%	PACKAGES AND OTHER DOCUMENT CONFIGURATIONS
%----------------------------------------------------------------------------------------

\documentclass[
	fontsize=10pt, % Base font size
	twoside=false, % Use different layouts for even and odd pages (in particular, if twoside=true, the margin column will be always on the outside)
	%open=any, % If twoside=true, uncomment this to force new chapters to start on any page, not only on right (odd) pages
	%chapterprefix=true, % Uncomment to use the word "Chapter" before chapter numbers everywhere they appear
	%chapterentrydots=true, % Uncomment to output dots from the chapter name to the page number in the table of contents
	numbers=noenddot, % Comment to output dots after chapter numbers; the most common values for this option are: enddot, noenddot and auto (see the KOMAScript documentation for an in-depth explanation)
	%draft=true, % If uncommented, rulers will be added in the header and footer
	%overfullrule=true, % If uncommented, overly long lines will be marked by a black box; useful for correcting spacing problems
]{kaobook}

% Set the language
\usepackage[english]{babel} % Load characters and hyphenation
\usepackage[english=british]{csquotes} % English quotes

% Load packages for testing
\usepackage{blindtext}
%\usepackage{showframe} % Uncomment to show boxes around the text area, margin, header and footer
%\usepackage{showlabels} % Uncomment to output the content of \label commands to the document where they are used

% Load the bibliography package
\usepackage[style=alphabetic]{styles/kaobiblio}
\addbibresource{main.bib} % Bibliography file

% Load mathematical packages for theorems and related environments. NOTE: choose only one between 'mdftheorems' and 'plaintheorems'.
\usepackage{styles/mdftheorems}
%\usepackage{styles/plaintheorems}

\usepackage{mathpartir}
\usepackage{dirtree}

\newcommand{\code}[1]{\texttt{#1}}
\newcommand{\term}[1]{#1}

\graphicspath{{examples/documentation/images/}{images/}} % Paths in which to look for images

\makeindex[columns=3, title=Alphabetical Index, intoc] % Make LaTeX produce the files required to compile the index

\makeglossaries % Make LaTeX produce the files required to compile the glossary

\makenomenclature % Make LaTeX produce the files required to compile the nomenclature

\newcommand{\newfrac}[2] {
        \frac
                {#1}
                {#2 \rule{0pt}{11pt}}
}
\newcommand{\newinfrule}[2] {
  \ensuremath{\newfrac
    {\displaystyle #1}
    {\displaystyle #2}}
}
\newcommand{\derive}[2]{\ensuremath{\begin{array}{c}\infer{#1}{#2}\end{array}}}
\newcommand{\embox}[1]{\mbox{\emph{#1}}}
\newcommand{\finto}{\stackrel{\mbox{\tiny fin}}{\to}}

% Reset sidenote counter at chapters
%\counterwithin*{sidenote}{chapter}

%----------------------------------------------------------------------------------------

\begin{document}

%----------------------------------------------------------------------------------------
%	BOOK INFORMATION
%----------------------------------------------------------------------------------------

% \titlehead{The \texttt{kaobook} class}
% \subject{Use this document as a template}

% \title[Example and documentation of the {\normalfont\texttt{kaobook}} class]{Example and documentation \\ of the {\normalfont\texttt{kaobook}} class}
\title{Introduction to Programming Languages}

% \author[Federico Marotta]{Federico Marotta \thanks{A \LaTeX\ lover}}
\author{Jaemin Hong}

\date{}

% \publishers{An Awesome Publisher}

%----------------------------------------------------------------------------------------

\frontmatter % Denotes the start of the pre-document content, uses roman numerals

%----------------------------------------------------------------------------------------
%	OPENING PAGE
%----------------------------------------------------------------------------------------

%\makeatletter
%\extratitle{
%	% In the title page, the title is vspaced by 9.5\baselineskip
%	\vspace*{9\baselineskip}
%	\vspace*{\parskip}
%	\begin{center}
%		% In the title page, \huge is set after the komafont for title
%		\usekomafont{title}\huge\@title
%	\end{center}
%}
%\makeatother

%----------------------------------------------------------------------------------------
%	COPYRIGHT PAGE
%----------------------------------------------------------------------------------------

\makeatletter
\uppertitleback{\@titlehead} % Header

\lowertitleback{
  % \textbf{License}\\
	\copyright2021 Jaemin Hong

  \medskip

  All rights reserved. No part of this book may be reproduced in any form by any
  electronic of mechanical means (including photocopying, recording, or
  information storage and retrieval) without permission in writing from the
  author.

  % \medskip

  % \textbf{Published Date}\\
  % \today
}
\makeatother

%----------------------------------------------------------------------------------------
%	DEDICATION
%----------------------------------------------------------------------------------------

% \dedication{
% 	The harmony of the world is made manifest in Form and Number, and the heart and soul and all the poetry of Natural Philosophy are embodied in the concept of mathematical beauty.\\
% 	\flushright -- D'Arcy Wentworth Thompson
% }

%----------------------------------------------------------------------------------------
%	OUTPUT TITLE PAGE AND PREVIOUS
%----------------------------------------------------------------------------------------

% Note that \maketitle outputs the pages before here

% If twoside=false, \uppertitleback and \lowertitleback are not printed
% To overcome this issue, we set twoside=semi just before printing the title pages, and set it back to false just after the title pages
\KOMAoptions{twoside=semi}
\maketitle
\KOMAoptions{twoside=false}

%----------------------------------------------------------------------------------------
%	PREFACE
%----------------------------------------------------------------------------------------

%\input{chapters/preface.tex}

%----------------------------------------------------------------------------------------
%	TABLE OF CONTENTS & LIST OF FIGURES/TABLES
%----------------------------------------------------------------------------------------

\begingroup % Local scope for the following commands

% Define the style for the TOC, LOF, and LOT
%\setstretch{1} % Uncomment to modify line spacing in the ToC
%\hypersetup{linkcolor=blue} % Uncomment to set the colour of links in the ToC
\setlength{\textheight}{23cm} % Manually adjust the height of the ToC pages

% Turn on compatibility mode for the etoc package
\etocstandarddisplaystyle % "toc display" as if etoc was not loaded
\etocstandardlines % toc lines as if etoc was not loaded

\tableofcontents % Output the table of contents

% \listoffigures % Output the list of figures

% Comment both of the following lines to have the LOF and the LOT on different pages
% \let\cleardoublepage\bigskip
% \let\clearpage\bigskip

% \listoftables % Output the list of tables

\endgroup

%----------------------------------------------------------------------------------------
%	MAIN BODY
%----------------------------------------------------------------------------------------

\mainmatter % Denotes the start of the main document content, resets page numbering and uses arabic numbers
\setchapterstyle{kao} % Choose the default chapter heading style

\setchapterpreamble[u]{\margintoc}
\chapter{Introduction}
\labch{introduction}

What is a programming language?

The simplest answer is ``it is a language used for programming.'' However, this
answer does not help us understand programming languages. We need a better
question to get a better answer.

What does a programming language consist of?

There is a good answer for this question: ``in a narrow sense, a programming
language consists of syntax and semantics, and in a broad sense, it additionally
has a standard library and an ecosystem.''

Syntax and semantics are principal concepts to understand programming languages.
Syntax determines how a language looks like, and semantics fills the inside. If
we consider a programming language as a human, we can say that syntax is one’s
appearance, and semantics is one’s thoughts. Programmers write programs
according to syntax. Syntax decides characters used in source code. Once programs
are written, semantics decides what each program does. Without semantics, all
the programs are useless. Programs can work as being expected only after
semantics determines the meaning of them. A programming language with syntax and
semantics is complete. Programmers using that language can write programs with
the syntax and execute the programs with the semantics. From a theoretical
perspective, syntax and semantics are all of a programming language.

For programmers, syntax and semantics are not the only elements of a programming
language. First, the standard library of a language is another element. The
standard library provides various utilities required by applications: data
structures like lists and maps, functions handling file and network IO, and so
on. The standard library is like clothes for humans. A human without clothes is
a human; a programming language without a standard library is a programming
language. At the same time, clothes are important to humans as they make bodies
warm and protect bodies from dangerous objects. Similarly, a standard library is
important to a programming language as it supplies diverse functionalities for
applications. Each person wears clothes different from others, and each language
puts different things from other languages in its standard library. Some
languages include lots of utilities in their standard libraries, while others
include much less. Some languages treat lists and maps as built-in concepts in
their semantics, while others define them with other primitives in their standard libraries.
Programmers avoid using a language without a standard library because such a
language increases the effort to write programs.

Another important element to programmers is the ecosystem of a programming
language. The ecosystem includes everything related to the language: developers
and companies using the language, third-party libraries written in the language,
and so on. It is like a society for humans. If many programmers and companies
use a programming language, one can easily get help and find complementary
materials by using the same language. There will be more chances of cooperative
work and employment, too. Third-party libraries also take important roles in
software development. The standard library offers only general facilities and
often lacks domain-specific features. When a required functionality cannot be
found in the standard library, a third-party library can provide the exact
functionality. For these reasons, the ecosystem of a programming language is
important to programmers.

Practically, the standard library and the ecosystem of a language are important
elements. Unlike syntax and semantics, they are not essential. A programming
language can exist even without its standard library and ecosystem. However,
developers take standard libraries and ecosystems into account as well as syntax and
semantics to choose languages they use. From a practical perspective, a
programming language consists of syntax, semantics, a standard library, and an
ecosystem.

This book is not for helping readers use a specific programming language. It
does not recommend a specific programming language, either. This book helps
readers learn new programming languages easily. You can acquaint any programming
languages once you completely read and understand this book. Obviously, this
goal cannot be achieved if the book discusses various languages separately. It
is possible only by discussing the underlying principles of every programming
language.

The principles of programming languages can be found from their semantics. Each
language seems very different from the others, but it is actually not the case.
Precisely speaking, their insides are quite the same, while their appearances
look different. They look different because their syntax and standard libraries,
which determine the appearances, are different. However, their insides, the
semantics, fundamentally share the same mathematical principles. If you
understand essential concepts residing in the semantics of multiple languages,
it is easy to understand and learn new languages.

People who know the key principles and can separate the elements of a language
can easily learn programming languages. As an analogy, consider a man learning
how to use a computer. It is a big problem if he cannot distinguish a keyboard
from a computer. For example, he thinks ``to say hello, my right index finger
presses the keyboard, my left middle finger presses the keyboard, my right ring
finger presses the keyboard three times.'' If the layout of the keyboard changes,
he should learn the whole computer again. On the other hand, if he knows that a
keyboard is just a tool to input text, he will less suffer from the change of
the keyboard layout. As he thinks ``to say hello, I press H, E, L, L, and O,'' he
does not need to learn the whole computer again. Of course, he should learn the
new keyboard layout, but it will be much easier. In addition, it is
straightforward to apply his knowledge to do new things. For example, he will
easily figure out ``to say lol, I press L, O, and L.'' If he does not distinguish
a keyboard from a computer, he cannot find any common principles between saying
hello and saying lol. Learning programming languages is the same. People who
cannot distinguish syntax and semantics believe that they should learn the whole
language again when the syntax changes. On the other hand, people who can
distinguish syntax and semantics know that semantics remains the same even if
syntax may vary. They know that understanding the principles of semantics is
important to learn languages. Becoming familiar with the
new syntax is all they need to use a new language fluently.

This book explains the semantics of principal concepts in programming languages.
\refch{introduction-to-scala}, \refch{immutability},
\refch{functions}, and \refch{pattern-matching}
introduce the Scala programming language. This book
uses Scala to implement interpreters of languages introduced in the book.
\refch{syntax-and-semantics} explains syntax and
semantics. Then, the book finally introduces various features of programming languages.
\begin{itemize}
    \item \refch{identifiers} introduces identifiers.
    \item \refch{first-order-functions},
      \refch{first-class-functions}, and \refch{lambda-calculus} introduce functions.
    \item \refch{recursion} introduces recursion.
    \item \refch{mutable-boxes} and \refch{mutable-variables} introduce mutation.
    \item \refch{lazy-evaluation} introduces lazy evaluation.
\end{itemize}

\section{Exercises}

\begin{enumerate}
\item Write the name of a programming language that you have used.
  What are the pros and cons of the language?
\item Write the names of two programming languages you know and compare them.
\end{enumerate}


\pagelayout{wide} % No margins
\addpart{Scala}
\pagelayout{margin} % Restore margins

\setchapterpreamble[u]{\margintoc}
\chapter{Introduction to Scala}
\labch{introduction-to-scala}

This book uses Scala as an implementation language, and
this chapter thus introduces the Scala programming language. Scala stands for a
\textbf{sca}lable \textbf{la}nguage~\cite{programming-in-scala}. It is a
multi-paradigm language that allows both functional and object-oriented styles.
This book focuses on the functional nature of Scala. In this chapter, we will
see what functional programming is and why this book uses functional
programming. In addition, we will install Scala and write simple programs in
Scala.

\section{Functional Programming}

What is \textit{functional programming}?\index{functional programming}
According to Wikipedia,

\begin{quote}
Functional programming is a programming paradigm that treats computation as the
evaluation of mathematical functions and avoids changing-state and mutable data.
\end{quote}

According to the book Functional Programming in Scala~\cite{fp-in-scala},

\begin{quote}
Functional programming (FP) is based on a simple premise with far-reaching
implications: we construct our programs using only pure functions---in other words,
functions that have no side effects.
\end{quote}

The above two sentences are enough to describe functional programming.

First, consider the phrase ``the evaluation of mathematical functions.''
From the perspective of functional programming, a
program is a single mathematical expression and the execution of the program is
finding a value denoted by the expression.
Each expression consists of zero or more subexpressions and evaluates to a value.

Let us discuss how
functional programming is different from imperative programming with
code examples.

\begin{verbatim}
int x = 1;
int y = 2;
if (y < 3)
    x = x + 4;
else
    x = x - 5;
\end{verbatim}

The above code is written in C, which represents imperative languages. Imperative
programming mimics a way in which computers operate. During the execution of a
program, a state, which can be interpreted as the memory of a computer,
exists and the execution modifies the state. The execution of the above C program has
the following steps:

\begin{enumerate}
\item A state that both \code{x} and \code{y} are uninitialized
\item A state that \code{x} is \code{1} and \code{y} is uninitialized
\item A state that \code{x} is \code{1} and \code{y} is \code{2}
\item Since \code{y < 3} is \code{true} under the state of the third step, go to
the next line.
\item A state that \code{x} is \code{5} and \code{y} is \code{2}
\end{enumerate}

The state keeps changes throughout the execution of the program. Each line
modifies the current state rather than resulting in some value.

\begin{verbatim}
let x = 1 in
let y = 2 in
if y < 3 then x + 4 else x - 5
\end{verbatim}

The above code is written in OCaml, which represents functional languages. A
program is an expression and the result of the execution is the result of
evaluating the expression. The execution does not require the notion of a state.
The execution of the above OCaml program has the following steps:

\begin{enumerate}
\item Given the fact that \code{x} equals \code{1}, evaluate
\code{let y = 2 in if y < 3 then x + 4 else x - 5}.
\item Given the fact that \code{x} equals \code{1} and \code{y} equals \code{2},
evaluate \code{if y < 3 then x + 4 else x - 5}.
\item Given the fact that \code{x} equals \code{1} and \code{y} equals \code{2},
evaluating \code{y < 3} yields \code{true}, and the next step is to evaluate
\code{x + 4}.
\item Given the fact that \code{x} equals \code{1} and \code{y} equals \code{2},
evaluate \code{x + 4}.
\item The result is \code{5}.
\end{enumerate}

There is no state. Each expression consists of subexpressions. The result of
an expression is determined by the results of its subexpressions.

Since the programs are simple, two programs look similar, but it is important to
understand two different perspectives of what a program is.

Now, look at the phrases ``avoids changing-state and mutable data'' and ``using
only pure functions.'' Functional programming avoids mutable variables, mutable data
structures, and mutable objects. The term \textit{mutable}\index{mutable}
means being able to change. Its opposite is \textit{immutable}\index{immutable},
which means not being able to change. States change throughout the execution of programs.
In functional programming, states do not exist since things never change.
Due to the lack of states, a function always does the same stuff
and always returns the same value for the same arguments. Such functions
are called pure functions.

In practice, especially for large-scale projects, using only immutable things in
the whole code is often inefficient. Most real-world functional languages
provide mutation via language constructs like \code{var} of Scala, \code{ref} of
OCaml, and \code{set!} and \code{box} of Racket. However,
functional programming uses immutable things in most cases. Even without
mutation, we can still express most programs without difficulties.

As we have seen so far,
immutability is the most important concept of functional programming.
Immutability allows modular programming and eases the reasoning of programs.
Because of immutability, programs that have to be trustworthy or require parallel computing
are good applications of functional programming.
\refch{immutability} will discuss the advantages of immutability in detail and
how to write interesting programs without mutation.

There are other important characteristics of functional programming as well as
immutability. Use of first-class functions and pattern matching also take the
key roles in functional programming. Both first-class functions and pattern matching
are valuable as they help abstraction. First-class functions allow programmers
to abstact computation; pattern matching allows programmers to abstract
data. Because of the ability of abstraction,
programs whose input has complex and abstract structures like source code
are typically written in functional languages.
\refch{functions} and \refch{pattern-matching} will respectively discuss first-class
functions and pattern matching in Scala.

This book implements interpreters and type checkers. They take source code as
input and process the input according to the mathematical semantics of programming
languages. It is important
to reason about the correctness of interpreters and type checkers. These
properties exactly match the strengths of functional programming. It is why
this book uses functional programming and Scala.

Before moving on to the next section,
let us see how people use functional programming in industry.

Akka\sidenote{\url{https://akka.io/}} is a concurrent,
distributed computing library written in Scala. Many companies have been using Akka.
Apache Spark,\sidenote{\url{https://spark.apache.org/}} a well-known library for data
processing, also is written in Scala.
Play\sidenote{\url{https://www.playframework.com/}}
is a widely-used web framework based on Akka.

Facebook has developed Infer,\sidenote{\url{https://fbinfer.com/}} a static analyzer for Java,
C, C++, and Objective-C, in OCaml. Facebook and other companies including Amazon
and Mozila use Infer to find bugs statically in their programs. Facebook has
developed also Flow,\sidenote{\url{https://flow.org/}} a static type checker for JavaScript.
Jane Street\sidenote{\url{https://www.janestreet.com/}} is a financial company well-known in
the programming language community and has developed its own software in OCaml.
According to the OCaml website,\sidenote{\url{http://ocaml.org/learn/companies.html}}
various companies including Docker use OCaml.

Haskell Wiki\sidenote{\url{http://wiki.haskell.org/Haskell_in_industry}} describes that Google,
Facebook, Microsoft, Nvidia, and many other companies use Haskell.

Erlang is a functional language for concurrent and parallel computing. Elixir
operates on the Erlang virtual machine and is used for the same purpose as Erlang.
An article from Code
Sync\sidenote{\url{https://codesync.global/media/successful-companies-using-elixir-and-erlang/}}
said that various companies including WhatsApp, Pinterest, and Goldman Sachs
use Erlang and Elixir.

\section{Installation}

As Scala programs are compiled to Java bytecode, which runs on the Java Virtual
Machine (\acrshort{jvmLabel}), you must
install Java before installing Scala. Java has various versions. Scala 2.13,
which is used in this book, needs JDK 8 or higher. JDK 8 is the most recommended
one. The Scala website\sidenote{\url{https://docs.scala-lang.org/overviews/jdk-compatibility/overview.html}}
discusses compatibility issues regarding the other versions.

The Oracle website\sidenote{\url{https://www.oracle.com/java/technologies/javase/javase-jdk8-downloads.html}}
provides an installation file for JDK 8.

You can donwload an installation file for Scala 2.13 from the Scala
website.\sidenote{\url{https://www.scala-lang.org/download/}} Note that you need a
file in the ``Other resources'' section at the bottom of the page.
On macOS, you may use Homebrew instead. By installing Scala, you can use the
Scala REPL, interpreter, and compiler. \refsec{scala-repl},
\refsec{scala-interpreter}, and \refsec{scala-compiler} will discuss
their usages respectively.

Another thing to install is SBT. SBT is a build tool for Scala. An installation
file for SBT is available at the SBT
website.\sidenote{\url{https://www.scala-sbt.org/download.html}}
\refsec{sbt} will discuss the usage of SBT.

\section{REPL}
\labsec{scala-repl}

Once you install Scala, you can launch Scala REPL by typing \code{scala} in
your command line.

\begin{verbatim}
$ scala
Welcome to Scala 2.13.5.
Type in expressions for evaluation. Or try :help.

scala>
\end{verbatim}

The term \acrshort{replLabel} stands for \textbf{r}ead, \textbf{e}val, \textbf{p}rint, and
\textbf{l}oop.
It is a program that iterativley reads code from a user, evaluates the code,
and prints the result. REPL is not a place to write a program but is a good
place to write short code and see how it works.

If you input an integer to REPL, it will evaluate the integer and show the
result.

\begin{verbatim}
scala> 0
val res0: Int = 0
\end{verbatim}

It means that the expression \code{0} evaluates to the value \code{0} and the type of
\code{0} is \code{Int}.
You can try some arithmetic expressions as well.

\begin{verbatim}
scala> 1 + 2
val res1: Int = 3
\end{verbatim}

A boolean is \code{true} or \code{false} in Scala.

\begin{verbatim}
scala> true
val res2: Boolean = true
\end{verbatim}

You can also use basic logical operators.

\begin{verbatim}
scala> true && false
val res3: Boolean = false
\end{verbatim}

String literals require double quotation marks.

\begin{verbatim}
scala> "hello"
val res4: String = hello
\end{verbatim}

Operations regarding strings can be done by calling methods.

\begin{verbatim}
scala> "hello".length
val res5: Int = 5

scala> "hello".substring(0, 4)
val res6: String = hell
\end{verbatim}

Strings in Scala provide the same methods as those in
Java.\sidenote{\url{https://docs.oracle.com/javase/8/docs/api/java/lang/String.html}}

The \code{println} function prints a given message into the console.

\begin{verbatim}
scala> println("Hello world!")
Hello world!
\end{verbatim}

Note that there is no result of \code{println("Hello world!")}. Actually,
\code{println("Hello world!")} evaluates to \code{()}, which is called unit.
Unit implies that the result does not have any meaningful information. It is
similar to \code{None} in Python and \code{undefined} in JavaScript. At the
same time, functions returning unit are similar to functions whose return types
are \code{void} in C or Java. Since unit does not have meaningful information,
REPL does not show the result when it is unit.

The remainder of this section introduces basic features of Scala, such as
variables and functions, with REPL.

\subsection{Variables}

The syntax of a variable definition is as follows:

\begin{verbatim}
val [name]: [type] = [expression]
\end{verbatim}

It defines a variable whose name is \code{[name]}.
The result of the expression becomes the value denoted by the variable and
must belong to the type.

\begin{verbatim}
scala> val x: Int = 1
val x: Int = 1
\end{verbatim}

If the type of the result does not match a given type, the variable will not be
defined due to a type mismatch.

\begin{verbatim}
scala> val y: Boolean = 2
                        ^
       error: type mismatch;
        found   : Int(2)
        required: Boolean
\end{verbatim}

You can omit the \code{: [type]} part and use the following syntax instead:

\begin{verbatim}
val [name] = [expression]
\end{verbatim}

In this case, a type mimatch never happens, and the
type of the variable becomes the same as the type of its value.
People usually omit the type annotations of local variables.

\begin{verbatim}
scala> val x = 3
val x: Int = 3
\end{verbatim}

Variables defined by \code{val} cannot be mutated, i.e. their values never
change. Reassignment will incur an error. We call such variables immutable
variables.

\begin{verbatim}
scala> x = 4
         ^
       error: reassignment to val
\end{verbatim}

Sometimes, mutable variables, i.e. variables whose values can change, are useful.
Scala provides mutable variables as well as immutable variables. You need to use
\code{var} instead of \code{val} to define mutable variables. You may or may not write
the type of a variable.

\begin{verbatim}
scala> var z = 5
var z: Int = 5

scala> z = 6
// mutated z

scala> z
val res8: Int = 6
\end{verbatim}

To assign a new value to a mutable variable, the value must conform to the
type of the variable. Otherwise, a type mismatch will happen.

\begin{verbatim}
scala> z = true
           ^
       error: type mismatch;
        found   : Boolean(true)
        required: Int
\end{verbatim}

\subsection{Functions}

The syntax of a function definition is as follows:

\begin{verbatim}
def [name]([name]: [type], …): [type] = [expression]
\end{verbatim}

Many programming languages require \code{return} to specify the return value of a function.
On the other hand, functions in Scala are like functions in mathematics: \code{return} is unnecessary.
The return value of a function is the result of the body expression, which is the expression
at the right side of \code{=} in the definition. The type annotation after each parameter specifies
the type of the parameter. The type after the parentheses is the return type, which must be the
same as the type of the return value.

\begin{verbatim}
scala> def add(x: Int, y: Int): Int = x + y
def add(x: Int, y: Int): Int

scala> add(3, 7)
val res9: Int = 10
\end{verbatim}

The return types of functions can be omitted.

\begin{verbatim}
scala> def add(x: Int, y: Int) = x + y
def add(x: Int, y: Int): Int
\end{verbatim}

However, parameter types cannot be omitted.

\begin{verbatim}
scala> def add(x, y) = x + y
                ^
       error: ':' expected but ',' found.
\end{verbatim}

To write multiple expressions including variable and functions definitions
in the body of a function, we put expressions separated by line breaks
inside curly braces.
Each line will be evaluated in the order, and the result of the last line will
be the return value.

\begin{verbatim}
scala> def quadruple(x: Int): Int = {
     |   val y = x + x
     |   y + y
     | }
def quadruple(x: Int): Int
\end{verbatim}

Inside \code{quadruple}, the variable \code{y} is defined and used for the
computation of
the return value.\sidenote{Vertical bars (\code{|}) at the beginning of lines are
not part of code. They have been automatically inserted by REPL.}

Multiple expressions inside curly braces are collectively treated as a single
expression. We call such an expression a sequenced expression. Like any other
expressions, a sequenced expression can occur anywhere an expression is needed.
For example, it can be used to define a variable.

\begin{verbatim}
scala> val a = {
     |   val x = 1 + 1
     |   x + x
     | }
val a: Int = 4
\end{verbatim}

There are many other things related to functions: recursion, first-class
functions, closures, and anonymous functions. \refch{immutability} will
discuss recursion, and \refch{functions} will discuss the other topics.

\subsection{Conditionals}

A conditional expression performs computation depending on a certain
condition, i.e. a boolean value. The syntax of a conditional expression is as
follows:

\begin{verbatim}
if ([expression]) [expression] else [expression]
\end{verbatim}

The first expression is the condition; the second expression is the true branch;
the last expression is the false branch.

\begin{verbatim}
scala> if (true) 1 else 2
val res10: Int = 1
\end{verbatim}

A conditional expression evaluates to a value. It is more similar to the ternary
operator \code{? :} in C than a if statement.
We do not need to make a variable
mutable to initialize the variable with a conditional value.

\begin{verbatim}
scala> val x = if (true) 1 else 2
val x: Int = 1
\end{verbatim}

On the other hand, people write code like below in languages like C.

\begin{verbatim}
int x;
if (true)
    x = 1;
else
    x = 2;
\end{verbatim}

Conditional expressions in Scala are more expressive than the ternary operator
in C because we can make complex computation a single expression with expression
sequencing, which is impossible in C.

\begin{verbatim}
scala> if (true) {
     |   val x = 2
     |   x + x
     | } else {
     |   val x = 3
     |   x * x
     | }
val res11: Int = 4
\end{verbatim}

\subsection{Lists}

A list is a collection of zero or more elements. A list maintains the order between
its elements. Lists in Scala are immutable. Once a list is created, its elements
never change. There are two ways to create a new list in Scala:

\begin{itemize}
  \item \code{List([expression], …, [expression])}
  \item \code{[expression] :: … :: [expression] :: Nil}
\end{itemize}

The type of a list whose elements have type \code{T} is \code{List[T]}.

\begin{verbatim}
scala> List(1, 2, 3)
val res12: List[Int] = List(1, 2, 3)

scala> 1 :: 2 :: 3 :: Nil
val res13: List[Int] = List(1, 2, 3)
\end{verbatim}

\code{List(…)} is more convenient than \code{::} for creating a new list from
scratch. However, \code{::} is more flexible since it can prepend a new element
in front of an existing list.\sidenote{It does not mutate the existing list to
prepend the new element. It creates a new list with the element and the list.}

\begin{verbatim}
scala> val l = List(1, 2, 3)
val l: List[Int] = List(1, 2, 3)

scala> 0 :: l
val res14: List[Int] = List(0, 1, 2, 3)
\end{verbatim}

The \code{length} method computes the length of a list; parentheses are used
to fetch the element at a specific index.\sidenote{The first index is \code{0}.}

\begin{verbatim}
scala> l.length
val res15: Int = 3

scala> l(0)
val res16: Int = 1
\end{verbatim}

In functional programming, accessing an arbitrary element of a list by an index
is rare. We use pattern matching in most cases. The syntax of pattern matching
for a list is as follows:\sidenote{The order between the cases can vary, which
means that the \code{::} case may come first.}

\begin{verbatim}
[expression] match {
  case Nil => [expression]
  case [name] :: [name] => [expression]
}
\end{verbatim}

The expression in front of \code{match} is the target of pattern matching.
If it is an empty list, it matches \code{case Nil}. The expression
of the \code{Nil} case will be evaluated.
Otherwise, it is a nonempty list and matches \code{case [name] :: [name]}. The first
name denotes the head\sidenote{the first element} of the list, and the second
name denotes the tail\sidenote{a list consisting of all the elements except the
head} of the list. The expression of the \code{::} case will be evaluated.

The following function takes a list of integers as an argument and returns the
head. The return value is zero when the list is empty.

\begin{verbatim}
scala> def headOrZero(l: List[Int]): Int = l match {
     |   case Nil => 0
     |   case h :: t => h
     | }
def headOrZero(l: List[Int]): Int

scala> headOrZero(List(1, 2, 3))
val res17: Int = 1

scala> headOrZero(List())
val res18: Int = 0
\end{verbatim}

\refch{immutability} will show use of pattern matching for lists in recursive
functions, and \refch{pattern-matching} will discuss pattern matching in detail.

\subsection{Tuples}

A tuple contains two or more elements and maintains the order between its
elements. We use parentheses to create a new tuple:

\begin{verbatim}
([expression], …, [expression])
\end{verbatim}

The type of a tuple whose elements have types from \code{T1} to \code{Tn}
respectively is \code{(T1, …, Tn)}. For example, the type of a tuple
whose first element is \code{Int} and second element is \code{Boolean} is
\code{(Int, Boolean)}.

\begin{verbatim}
scala> (1, true)
val res19: (Int, Boolean) = (1,true)
\end{verbatim}

To fetch the \code{i}-th element of a tuple, we can use \verb+._i+.\sidenote{The
first index is 1.}

\begin{verbatim}
scala> (1, true)._1
val res20: Int = 1
\end{verbatim}

Tuples look similar to lists but have important differences from
lists. First, a tuple's elements can have different types, while a list's
elements cannot. For example, a tuple of the type \code{(Int, Boolean)} has
one integer and one boolean, while a list of the type \code{List[Int]} can have
only integers. We say that tuples are heterogenous, while lists are homogeneous.
Second, a list allows accessing an arbitrary index of a list,
while a tuple does not. For example, \code{l(f())} is possible where \code{l} is
a list and \code{f} returns an integer, while there is no way to access the
\code{f()}-th element of a tuple since the return value of \code{f} is unknown
before execution.

We use lists and tuples for different purposes. Lists are appropriate when the
number of elements can vary and an arbitrary index should be accessible.
For instance, a list should be used to represent a collection of the heights of
students in a certain class.

\begin{verbatim}
List(189, 167, 156, 170, 183)
\end{verbatim}

It allows us to fetch the height of the \code{i}-th student.

On the other hand, tuples are appropriate when the number of elements
are fixed and each index has a specific meaning. For instance,
a tuple can represent the information of a single student, where the information
consists of one's name, one's height, and whether one has payed the school
expense or not.

\begin{verbatim}
("John Doe", 173, true)
\end{verbatim}

We can use \verb+._1+ to find the name, \verb+._2+ to find the height, and
\verb+._3+ to check whether one has payed.

We call a length-2 tuple a pair and a length-3 tuple a triple. Also, we can
consider unit as a length-0 tuple.

\subsection{Maps}

A map is a collection of pairs, where each pair consists of a key and a value.
Its provides the corresponding value when a key is given.
Maps in Scala are immutable as well. Below is the syntax to create a new map:

\begin{verbatim}
Map([expression] -> [expression], …)
\end{verbatim}

The type of a map whose keys have type \code{T} and values have type \code{S} is
\code{Map[T, S]}.

\begin{verbatim}
scala> val m = Map(1 -> "one", 2 -> "two", 3 -> "three")
val m: Map[Int,String] = Map(1 -> one, 2 -> two, 3 -> three)
\end{verbatim}

To find the value corresponding to a certain key, we use parentheses.

\begin{verbatim}
scala> m(2)
val res21: String = two
\end{verbatim}

Maps provide various
methods.\sidenote{\url{https://www.scala-lang.org/api/current/scala/collection/immutable/Map.html}}

\subsection{Classes and Objects}

An object is a value with fields and methods. Fields store values, and methods
are operations related to the object. A class is a blueprint of objects. We can
easily create multiple objects of the same structure by defining a single class.
This book uses only ``case'' classes of Scala. Case classes are similar to
classes but more convenient, e.g. automatic support for pretty printing and
pattern matching.

The syntax of a class definition is as follows:

\begin{verbatim}
case class [name]([name]: [type], …)
\end{verbatim}

The first name is the name of a new class. The names inside
the parentheses are the names of the fields of the class. A class definition
must specify the types of its fields.

\begin{verbatim}
scala> case class Student(name: String, height: Int)
class Student
\end{verbatim}

Creating new objects is similar to a function call.

\begin{verbatim}
scala> val s = Student("John Doe", 173)
val s: Student = Student(John Doe,173)
\end{verbatim}

Fields can be accessed by \code{.[name]}.

\begin{verbatim}
scala> s.name
val res22: String = John Doe
\end{verbatim}

Objects in Scala are immutable by default. If we add \code{var} to a field when
defining a class, the field becomes mutable.

\begin{verbatim}
scala> case class Student(name: String, var height: Int)
class Student

scala> val s = Student("John Doe", 173)
val s: Student = Student(John Doe,173)

scala> s.height = 180
// mutated s.height

scala> s.height
val res23: Int = 180
\end{verbatim}

\section{Interpreter}
\labsec{scala-interpreter}

An \textit{interpreter}\index{interpreter} is a program that takes source code as input and runs the code.
The Scala interpreter takes Scala source code as input. To use the interpreter,
we need to save source code into a file. Make a file with the following code, and
save it as \code{Hello.scala}.

\begin{verbatim}
println("Hello world!")
\end{verbatim}

You can excute the interpreter by typing \code{scala} with the name of a file in
your command line. Here, we need to say \code{scala Hello.scala}.

\begin{verbatim}
$ scala Hello.scala
Hello world!
\end{verbatim}

You can write multiple lines in a single file. Fix \code{Hello.scala} like
below.

\begin{verbatim}
val x = 2
println(x)
val y = x * x
println(y)
\end{verbatim}

Then, excute the interpreter again.

\begin{verbatim}
$ scala Hello.scala
2
4
\end{verbatim}

\section{Compiler}
\labsec{scala-compiler}

A \textit{compiler}\index{compiler} is a program that takes source code as input and translates it into
another language. Usually, the target language is a low-level language like
machine code or bytecode of a particular virtual machine.
The Scala compiler takes Scala source code as input and translates it into Java
bytecode. Once code is compiled, we can run the generated bytecode with the JVM.

For compilation, we need to define the main method of a program. The main method
is the entrypoint of every program running on the JVM.
Make a file with the following code, and save it as \code{Hello.scala}.

\begin{verbatim}
object Hello {
  def main(args: Array[String]): Unit = {
    println("Hello world!")
  }
}
\end{verbatim}

You can make the compiler compile the code by typing \code{scalac} with the name
of the file in your command line.

\begin{verbatim}
$ scalac Hello.scala
\end{verbatim}

After compilation, you will be able to find the \code{Hello.class} file
in the same directory. The file contains Java bytecode.

You can run the bytecode with the JVM by the \code{scala} command. In this time,
you should write only the class name.

\begin{verbatim}
$ scala Hello
Hello world!
\end{verbatim}

You can change the behavior of a program by modifying the main method.
Each time you modify, you need to re-compile the program to re-generate the
bytecode.

Running bytecode is much more efficient than interpreting Scala source code.
You can easily notice that \code{scala Hello} takes much less than \code{scala
Hello.scala} even though their results are the same.

Scala has two sorts of errors: compile-time errors and run-time errors.
Compile-time errors occur during compilation, i.e. while running \code{scalac}.
If the compiler finds things that
might go wrong at run time, it raises errors and aborts the compilation. For
example, an expression adding an integer to a boolean results in a compile-time
error because such an addition cannot succeed at run time.

\begin{verbatim}
true + 1
\end{verbatim}
\vspace{-1em}
\begin{mdframed}[hidealllines=true,backgroundcolor=red!10,innerleftmargin=3pt,innerrightmargin=3pt,leftmargin=-3pt,rightmargin=-3pt]
\begin{verbatim}
error: type mismatch;
 found   : Int(1)
 required: String
true + 1
     ^
\end{verbatim}
\vspace{-2em}
\begin{flushright}
\scriptsize\textsf{Compile-time error}
\end{flushright}
\end{mdframed}

Unfortunately, some bad behaviors cannot be detected by the compiler.
The compiler does not generate any errors for those behaviors.
Such problems will incur run-time errors during execution, i.e. while running
\code{scala},
and terminate the execution abnormally. Division by zero is one
example of run time errors.

\begin{verbatim}
1 / 0
\end{verbatim}
\vspace{-1em}
\begin{mdframed}[hidealllines=true,backgroundcolor=red!10,innerleftmargin=3pt,innerrightmargin=3pt,leftmargin=-3pt,rightmargin=-3pt]
\begin{verbatim}
java.lang.ArithmeticException: / by zero
\end{verbatim}
\vspace{-2em}
\begin{flushright}
\scriptsize\textsf{Run-time error}
\end{flushright}
\end{mdframed}

\section{SBT}
\labsec{sbt}

SBT is a build tool for Scala. Build tools help programmers work on large
projects with many files and libraries by tracking dependencies between files
and managing libraries. There are various build tools in the world, and SBT is
the most popular one for Scala.

You can create a new Scala project by the \code{sbt new} command.

\begin{verbatim}
$ sbt new scala/scala-seed.g8
[info] welcome to sbt 1.4.7
[info] loading global plugins from ~/.sbt/1.0/plugins
[info] set current project to ~/ (in build file:~/)
[info] set current project to ~/ (in build file:~/)


A minimal Scala project.

name [Scala Seed Project]: hello

Template applied in ~/hello
\end{verbatim}

After the creation, the directory structure is as follows:

\dirtree{%
  .1 hello.
  .2 build.sbt.
  .2 project.
  .3 Dependencies.scala.
  .3 build.properties.
  .2 src.
  .3 main.
  .4 scala.
  .5 example.
  .6 Hello.scala.
  .3 test.
  .4 scala.
  .5 example.
  .6 HelloSpec.scala.
}

The \code{build.sbt} file configures the project. It manages the version of Scala
used for the project, third-party libraries used in the project, and many other
things.
Source files are in the \code{src} directory. Files in \code{main} are
main source files, while files in \code{test} are only for testing.
You can add files into the \code{src/main/scala} directory and edit them to write code.

An SBT console can be started by the \code{sbt} command. The current working
directory of your shell should be the base directory of the project.

\begin{verbatim}
$ sbt
[info] welcome to sbt 1.4.7
[info] loading global plugins from ~/.sbt/1.0/plugins
[info] loading project definition from ~/hello/project
[info] loading settings for project root from build.sbt ...
[info] set current project to hello (in build file:~/hello/)

[info] sbt server started at
local:///~/.sbt/1.0/server/d4cd702f998423203dfe/sock
[info] started sbt server
sbt:hello>
\end{verbatim}

You can compile, run, and test the project by executing SBT commands in the
console.

\begin{itemize}
  \item \code{compile}: compile the project.
  \item \code{run}: run the project (re-compile if necessary).
  \item \code{test}: test the project (re-compile if necessary).
  \item \code{exit}: terminate the console.
\end{itemize}

\begin{verbatim}
sbt:hello> compile
[info] compiling 1 Scala source to ~/hello/target/scala-2.13

| => root / Compile / compileIncremental 0s
[success] Total time: 4 s
sbt:hello> test
[info] compiling 1 Scala source to ~/hello/target/scala-2.13
[info] HelloSpec:
[info] The Hello object
[info] - should say hello
[info] Run completed in 455 milliseconds.
[info] Total number of tests run: 1
[info] Suites: completed 1, aborted 0
[info] Tests: succeeded 1, failed 0, canceled 0, ignored 0
[info] All tests passed.
[success] Total time: 2 s
sbt:hello> run
[info] running example.Hello
hello
[success] Total time: 0 s
sbt:hello> exit
[info] shutting down sbt server
\end{verbatim}

To learn SBT more, refer to the SBT website.\sidenote{\url{https://www.scala-sbt.org/learn.html}}

\setchapterpreamble[u]{\margintoc}
\chapter{Immutability}
\labch{immutability}

\textit{Immutability}\index{immutability} means not changing.
Immutable variables never change their values
after initialization; immutable data structures never change their elements
once created. The opposite of immutability is mutability. While imperative
programming uses lots of mutable variables, data structures, and objects,
functional programming leverages the power of immutable varibles, data
structures, and objects. This chapter explains why immutability is important and
valuable. Also, we will see how to program without mutation.

\section{Advantages}

The book Programming in Scala~\cite{programming-in-scala}
discusses four strengths of immutability:

\begin{quote}
First, immutable objects are often easier to reason about than mutable ones,
because they do not have complex state spaces that change over time. Second, you
can pass immutable objects around quite freely, whereas you may need to make
defensive copies of mutable objects before passing them to other code. Third,
there is no way for two threads concurrently accessing an immutable to corrupt
its state once it has been properly constructed, because no thread can change the
state of an immutable. Fourth, immutable objects make safe hash table keys. If a
mutable object is mutated after it is placed into a \code{HashSet}, for example,
that object may not be found the next time you look into the \code{HashSet}.
\end{quote}

We will focus on the first two advantages:
easier reasoning and no need for defensive copies.

First, let us see why immutability makes things easy to reason about.

\begin{verbatim}
val x = 1
...
f(x)
\end{verbatim}

At the first line of the code, \code{x} is \code{1}. Since \code{x} is immutable,
there is no doubt that \code{x} is still \code{1} when \code{x} is passed as an
argument for \code{f} at the last line of the code.

\begin{verbatim}
var x = 1
...
f(x)
\end{verbatim}

On the other hand, if \code{x} is a mutable variable, one should read every line
of code in the middle to find the value of \code{x} at the time when the function
call happens.

When \code{x} is mutable, without tracking every
modification of \code{x} throughout the code, the value of \code{x} at the last
line is unknown. It hampers programmers from understanding the code
and possibly leads to more bugs.
The program with immutable \code{x} does not suffer from such problems.
Remembering only one line of the code is enough to track the value of \code{x}.

Mutable data structures cause similar problems.

\begin{verbatim}
val x = List(1, 2)
...
f(x)
...
x
\end{verbatim}

As \code{List} is immutable,
\code{x} is a list always containing \code{1} and \code{2}.

\begin{verbatim}
import scala.collection.mutable.ListBuffer
val x = ListBuffer(1, 2)
...
f(x)
...
x
\end{verbatim}

On the other hand, \code{ListBuffer} is a mutable data structure in the Scala
standard library. It is possible to add an item to or remove an item from the
list referred by \code{x}. Programmers cannot be certain about the content of \code{x}
unless they read all the lines in between. Besides, a function \code{f} also is
able to change the content of \code{x}. If one writes a program with a wrong
assumption that \code{f} does not modify \code{x}, then the program might be
buggy.

Mutable global variables make code much harder to understand than mutable local
variables.

\begin{verbatim}
def f(x: Int) = g(x, y)
\end{verbatim}

The return value of function \code{f} depends on the value of a global variable
\code{y}. If \code{y} is mutable, \code{f} is not a pure function and expecting
the behavior of \code{f} is nontrivial. \code{y} can be declared in any arbitrary
file and all files are able to change the value of \code{y}.
In the worst case, an external library defines \code{y} and source code
modifying \code{y} is not available for reading.

The examples are small and seem artificial, but immutability greatly improves
maintainability and readability of code in practice, especially for large
projects.

Now, let us see why immutability free us from making defensive copies.

\begin{verbatim}
val x = ListBuffer(1, 2)
...
f(x)
...
x
\end{verbatim}

Since \code{ListBuffer} creates mutable lists, there is no guarantee that the
content of \code{x} does not be changed by \code{f}. If it is necessary to prevent
modification, copying \code{x} is essential.

\begin{verbatim}
val x = ListBuffer(1, 2)
val y = x.clone
...
f(y)
...
x
\end{verbatim}

In cases that \code{x} has many elements and the code is executed multiple times,
copying \code{x} increases the execution time significantly.

In the code, using the \code{clone} method is enough to copy the list because the
list contains only integers. However, to pass lists containing mutable
objects safely to functions, defining additional methods for deep copy is
inevitable.

Immutability has several clear advantages. Immutability is an important concept in
functional programming. Functional programs use immutable variables and data
structures in most cases. If you write a large program whose logic is complex
and correctness is important, you should adopt the functional paradigm.
However, mind that immutability is not the silver bullet for every
program. For example, implementing algorithms in a functional style is usually
inefficient. It would be better to use mutable data structures like arrays,
mutable variables, and loops to implement algorithms. They make
programs much more efficient and faster. Choosing a programming proper paradigm to
the purpose of a program is the key to write good code.

\section{Recursion}

Repeating the same computation multiple times is a common pattern in programming.
Loops allow concise code expressing such cases. However, if everything is
immutable, going back to the beginnings of loops does not change any states.
Therefore, it is impossible to apply the same operation on different values for
each iteration or to terminate the loops. As a consequence, loops are useless in
functional programming. Functional programs use recursive functions instead of
loops to rerun computation. A \textit{recursive}\index{recursion}
function is a function that calls itself.\sidenote{In general, a definition that
refers to itself is a recursive definition. There can be recursive variables,
recursive types, and so on.}
To do more computation, the function calls itself with proper arguments.
Otherwise, it terminates the computation by returning some value.

The below \code{factorial} function calculates the factorial of a given integer.
For simplicity, we do not consider when the input is negative.
The following implementation uses an imperative style:

\begin{verbatim}
def factorial(n: Int) = {
  var i = 1, res = 1
  while (i <= n) {
    res *= i
    i += 1
  }
  res
}
\end{verbatim}

We can implement the same function in a functional style with recursion.

\begin{verbatim}
def factorial(n: Int): Int =
  if (n <= 0)
    1
  else
    n * factorial(n - 1)
\end{verbatim}

Note that recursive functions always require explicit return types in Scala,
unlike non-recursive functions, whose return types can be omitted.

The recursive version is preferred over the imperative version since its
correctness is easy to be verified.

To check the correctness of the imperative
\code{factorial} function, one should find a \textit{loop invariant}\index{loop invariant},
which is a proposition that is always true at the loop head.
The loop invariant of this case is
$((\code{i}-1)!=\code{res})\land(\code{i}\le\code{n}+1)$.
By using this invariant, we can conclude that $\code{i}=\code{n}+1$ and,
therefore, $\code{res}=(\code{i}-1)!=\code{n}!$ at the last line of the
function, which implies that it correctly implements factorial.
It is nontrival to find a proper loop invariant and show that the loop invariant
holds at the beginning of each iteration.

On the other hand,
recursive functions usually reveal their mathematical definitions more clearly
than functions using loops. Consider the following mathematical definition of
factorial:

\[n!=\begin{cases}1 & \text{if } n=0\\n \times (n-1)! &
\text{otherwise}\end{cases}\]

You can see that the implementation of the \code{factorial} function using recursion
is identical to the mathematical definition of factorial. It is almost trivial
to show that the recursive \code{factorial} function is correct.
Recursion allows concise and intuitive descriptions of mathematical functions.
In many cases, functions with recursion is much easier to be verified formally
or informally than functions with loops.

Recursive functions are also good at treating recursive data structures like
lists. A list is recursive since a nonempty list consists of the head element
and the tail list, which means that a nonempty list has another list as its component.
Writing some functions regarding lists help understanding and practicing
recursion.

The following function takes a list as an argument and returns a list whose
elements are one larger than the elements of the given list.

\begin{verbatim}
def inc1(l: List[Int]): List[Int] = l match {
  case Nil => Nil
  case h :: t => h + 1 :: inc1(t)
}
\end{verbatim}

When a given list is empty, the function returns the empty list. Otherwise, the
return value is a list whose head is one larger than the head of the given list
and tail has elements that are one larger than the elements of the tail of the
given list.

Similarly, \code{square} takes a list of integers as an argument and returns a
list whose elements are the squares of the elements of the given list.

\begin{verbatim}
def square(l: List[Int]): List[Int] = l match {
  case Nil => Nil
  case h :: t => h * h :: square(t)
}
\end{verbatim}

The following function takes a list of integers as an argument and returns a list whose
elements are odd integers.

\begin{verbatim}
def odd(l: List[Int]): List[Int] = l match {
  case Nil => Nil
  case h :: t =>
    if (h % 2 != 0)
      h :: odd(t)
    else
      odd(t)
}
\end{verbatim}

For a nonempty list, the function checks whether the head is odd or not. If the
head is odd, the resulting list contains the head, and its tail  has
only odd integers. Otherwise, the head is removed.

Similarly, \code{positive} takes a list of integers as an argument and returns a
list whose elements are positive.

\begin{verbatim}
def positive(l: List[Int]): List[Int] = l match {
  case Nil => Nil
  case h :: t =>
    if (h > 0)
      h :: positive(t)
    else
      positive(t)
}
\end{verbatim}

The following function calculates the sum of the elements of a given list.

\begin{verbatim}
def sum(l: List[Int]): Int = l match {
  case Nil => 0
  case h :: t => h + sum(t)
}
\end{verbatim}

The sum of elements in the empty list is zero, as there are no elements.
When a list is nonempty, the sum of its elements can be calculated by adding the
value of the head to the sum of its tail's elements.

Similarly, \code{product} calculates the product of the elements of a given
list.

\begin{verbatim}
def product(l: List[Int]): Int = l match {
  case Nil => 1
  case h :: t => h * product(t)
}
\end{verbatim}

Recursion has some disadvantages: overheads of function calls and stack
overflow. Most modern CPUs have enough computing power to ignore function call
overheads. However, loops are still ideal for functions in performance-critical
programs. Stack overflow happens when a
stack lacks space due to repetitive function calls. It is a critical problem
since it causes immediate termination of execution without yielding meaningful
output. Moreover, programs like web servers do not finish their execution, and
their stacks will eventually overflow. To prevent stack overflow, many functional
languages provide tail call optimization.
The following section explains tail call optimization in detail.

\section{Tail Call Optimization}
\labsec{tco}

If the last action of a function is a function call, then the call is a tail
call. When a tail call happens, the callee does every computation, and thus the
local variables of the caller have no need to remain after the call. The stack
frame of the caller can be destroyed. Most functional languages exploit this
fact to optimize tail calls. This optimization is called
\textit{tail call optimization}.\index{tail call optimization}
At compile time, compilers check whether calls are tail calls.
If a call is a tail call, the compilers generate code that eliminates the
stack frame of the caller before the call. They do not optimize non-tail function
calls because the local variables of the callers can be used after the callees
return. If every function call in a program is a tail call, the stack never
grows so that the program is safe from stack overflow.

\begin{verbatim}
def factorial(n: Int): Int =
  if (n <= 0)
    1
  else
    n * factorial(n - 1)
\end{verbatim}

The previous \code{factorial} function multiplies \code{n} and the return value
of the recursive \code{factorial(n - 1)} call. The multiplication is the last
action. The recursive call is not a tail call. The stack frame of the caller must
remain. The following process computes \code{factorial(3)}:

\begin{itemize}
\item \code{factorial(3)}
\item \code{3 * factorial(2)}
\item \code{3 * (2 * factorial(1))}
\item \code{3 * (2 * (1 * factorial(0)))}
\item \code{3 * (2 * (1 * 1))}
\item \code{3 * (2 * 1)}
\item \code{3 * 2}
\item \code{6}
\end{itemize}

At most four stack frames coexist. For a large argument, a stack grows
again and again and finally overflows.

\begin{verbatim}
factorial(10000)
\end{verbatim}
\vspace{-1em}
\begin{mdframed}[hidealllines=true,backgroundcolor=red!10,innerleftmargin=3pt,innerrightmargin=3pt,leftmargin=-3pt,rightmargin=-3pt]
\begin{verbatim}
java.lang.StackOverflowError
  at .factorial
\end{verbatim}
\vspace{-2em}
\begin{flushright}
\scriptsize\textsf{Run-time error}
\end{flushright}
\end{mdframed}

To implement the function with a tail call, instead of multiplying \code{n} and
\code{factorial(n - 1)}, the function has to pass both \code{n} and \code{n - 1}
as arguments and make the callee multiply \code{n} and $(\code{n - 1})!$.
This strategy can be interpreted as passing an intermediate result.

\begin{itemize}
\item \code{factorial(3)}
\item \code{factorial(2, intermediate result = 3)}
\item \code{factorial(1, intermediate result = 3 * 2)}
\item \code{factorial(1, intermediate result = 6)}
\item \code{factorial(0, intermediate result = 6 * 1)}
\item \code{factorial(0, intermediate result = 6)}
\item \code{6}
\end{itemize}

There is no need to return to the caller. The below code shows the
\code{factorial} function with a tail call. The function needs one more parameter
that takes an intermediate result as an argument.
\code{factorial(n, i)} computes \(\code{n}!\times\code{i}\).

\begin{verbatim}
def factorial(n: Int, inter: Int): Int =
  if (n <= 0)
    inter
  else
    factorial(n - 1, inter * n)
\end{verbatim}

The function uses a tail call. More precisely, the function is
tail-recursive. Its last action is calling itself. Unlike most functional
languages, Scala cannot optimize general tail calls. Scala optimizes only
tail-recursive calls.
The Scala compiler generates Java bytecode, which is excuted by the JVM. The JVM does not
allow bytecode to jump to the beginning of another function. In the JVM, functions can
only either return or call functions. Therefore, the Scala compiler cannot generate
optimized code by removing the stack frame of a caller. Instead, they transform
tail-recursive calls into loops.
The \code{factorial} function is compiled to the following bytecode:

\begin{verbatim}
public int factorial(int, int);
  Code:
     0: iload_1
     1: iconst_0
     2: if_icmpgt     9
     5: iload_2
     6: goto          20
     9: iload_1
    10: iconst_1
    11: isub
    12: iload_2
    13: iload_1
    14: imul
    15: istore_2
    16: istore_1
    17: goto          0
    20: ireturn
\end{verbatim}

We can check that there is no function call at
all.\sidenote{\code{invokevirtual} is a function call instruction.}
The function just jumps to instructions inside the function.
Due to the tail call optimization, the function never incurs stack overflow.

Even with tail recursion,
the result is still incorrect because of integer overflow.

\begin{verbatim}
assert(factorial(10000, 1) == 0)  // weird result
\end{verbatim}

The \code{BigInt} type resolves integer overflow.

\begin{verbatim}
def factorial(n: BigInt, inter: BigInt): BigInt =
  if (n <= 0)
    inter
  else
    factorial(n - 1, inter * n)

assert(factorial(10000, 1) > 0)
\end{verbatim}

The optimization of the Scala compiler not only prevents stack overflow but also removes the
overheads of function calls. The downside is that mutually recursive
functions using tail calls lie beyond the scope of the optimization.
Mutual recursion is recursion involving two or more definitions.
The following functions can cause stack overflow in Scala even though they use tail
calls because they are not tail-recursive:

\begin{verbatim}
def even(n: Int): Boolean = if (n <= 0) true else odd(n - 1)
def odd(n: Int): Boolean = if (n == 1) true else even(n - 1)
\end{verbatim}

In Scala, programmers can ask the compiler to check
whether functions are tail-recursive with annotations. The annotations prevent
programmers from making functions non-tail-recursive by mistakes.

\begin{verbatim}
import scala.annotation.tailrec
@tailrec def factorial(n: BigInt, inter: BigInt): BigInt =
  if (n <= 0)
    inter
  else
    factorial(n - 1, inter * n)
\end{verbatim}

A non-tail-recursive function with the \code{tailrec} annotation results in a
compile-time error.

\begin{verbatim}
@tailrec def factorial(n: Int): Int =
  if (n <= 0)
    1
  else
    n * factorial(n - 1)
\end{verbatim}
\vspace{-1em}
\begin{mdframed}[hidealllines=true,backgroundcolor=red!10,innerleftmargin=3pt,innerrightmargin=3pt,leftmargin=-3pt,rightmargin=-3pt]
\begin{verbatim}
      ^
error:
could not optimize @tailrec annotated method factorial:
it contains a recursive call not in tail position
\end{verbatim}
\vspace{-1.5em}
\begin{flushright}
\scriptsize\textsf{Compile-time error}
\end{flushright}
\end{mdframed}

The annotation does not affect the behavior of the resulting bytecode.
Regardless of the existence of the annotation, the compiler always optimizes
tail-recursive functions. Still, using the annotations is desirable to prevent
mistakes.

Calling the tail-recursive version of \code{factorial} needs the unnecessary
second argument. The below code defines a new \code{factorial} function with one
parameter and uses the tail-recursive one as a local function inside the
function.

\begin{verbatim}
def factorial(n: BigInt): BigInt = {
  @tailrec def aux(n: BigInt, inter: BigInt): BigInt =
    if (n <= 0)
      inter
    else
      aux(n - 1, inter * n)
  aux(n, 1)
}
\end{verbatim}

Some functions treating lists also can be rewritten in a tail-recursive way.
Below is a tail-recursive version of \code{sum}.

\begin{verbatim}
def sum(l: List[Int]): Int = {
  @tailrec def aux(l: List[Int], inter: Int): Int = l match {
    case Nil => inter
    case h :: t => aux(t, inter + h)
  }
  aux(l, 0)
}
\end{verbatim}

\code{aux(l, n)} calculates \code{n} plus the sum of \code{l}'s elements.

Similarly, \code{product} can be implemented in a tail-recursive way.

\begin{verbatim}
def product(l: List[Int]): Int = l match {
  @tailrec def aux(l: List[Int], inter: Int): Int = l match {
    case Nil => inter
    case h :: t => aux(t, inter * h)
  }
  aux(l, 1)
}
\end{verbatim}

\section{Exercises}

\begin{enumerate}
  \item Consider the following definition of \code{Student}:

    \code{case class Student(name: String, height: Int)}

    Implement a function that takes a list of students as an argument
    and returns a list containing the names of the students.

  \item Consider the same definition of \code{Student}.
    Implement a function that takes a list of students as an argument
    and returns a list of students whose heights are greater than 170.

  \item
    Implement a function that takes a list of integers as an argument
    and returns the length of the list.

  \item
    Implement a function that takes a list of integers and an integer as arguments
    and returns a list obtained by appending the integer at the end of the list.
    Then, compare the time complexity of appending a new element to that of
    prepending a new element by \code{::}, which is $O(1)$.
\end{enumerate}

\setchapterpreamble[u]{\margintoc}
\chapter{Functions}
\labch{functions}

This section focuses on use of functions in functional programming.
In functional programming, functions are first-class. First-class functions
allow programmers to abstract complex computation easily.
This section explains what first-class functions are.
In addition, anonymous functions and closures, which are related to first-class
functions, will be introduced. To show the power of first-class functions, we
will re-implement the functions in \refch{immutability} (\code{inc1},
\code{square}, \ldots) with first-class functions.

\section{First-Class Functions}

An entity in a programming language is \textit{first-class}\index{first-class} if it satisfies the
following conditions:

\begin{itemize}
\item It can be an argument of a function call.
\item It can be a return value of a function.
\item A variable can refer to it.
\end{itemize}

Anything that is first-class can be used as a value. Functions are highly important and
treated as values in functional languages.
Functions that are first-class are called \textit{first-class functions}.\index{first-class function}

Some people use the term higher-order functions. \textit{Higher-order
functions}\index{higher-order function} are
functions that are not first-order, where first-order functions neither take
functions as arguments nor return functions. Therefore, higher-order functions
can take functions as arguments and return functions. Strictly speaking, they
are different from first-class functions because first-class functions are
functions that can be passed as arguments or returned from functions.
However, any languages that support first-class functions support higher-order
functions and vice versa.
The reason is obvious: to pass first-class functions as arguments, there should
be higher-order functions, and to pass functions to higher-order functions,
there should be first-class functions.
Consequently, in most contexts, people do not distinguish
first-class functions and higher-order functions, and you can consider
first-class functions and higher-order functions as exchangeable terms.

Now, let us see how we can use first-class functions in Scala with some code
examples.

\begin{verbatim}
def f(x: Int): Int = x
def g(h: Int => Int): Int = h(0)

assert(g(f) == 0)
\end{verbatim}

The function \code{g} has one parameter \code{h}. The type of \code{h} is \code{Int => Int}.
An argument passed to \code{g} is a function that receives one integer
and returns an integer. In Scala, \code{=>} expresses the types
of functions. Functions without parameters have types of the form \code{() => [return type]}.
\code{[parameter type] => [return type]} is the type of a function with a
single parameter. Parentheses are required to express the types of functions
with two or more parameters:
\code{([parameter type], … ) => [return type]}. The function \code{f}
has one integer parameter and returns an integer, i.e. its type is \code{Int =>
Int}. Thus, it can be an argument for
\code{g}. Evaluating \code{g(f)} equals evaluating \code{f(0)}, which results
in \code{0}.

\begin{verbatim}
def f(y: Int): Int => Int = {
  def g(x: Int): Int = x
  g
}

assert(f(0)(0) == 0)
\end{verbatim}

The function \code{f} returns the function \code{g}. Since the return type of \code{f}
is \code{Int => Int}, its return value must be a function that takes an integer
as an argument and returns an integer. \code{g} satisfies the condition. \code{f(0)}
is the same as \code{g} and therefore is a function. \code{f(0)(0)} equals \code{g(0)},
which returns \code{0}.

\begin{verbatim}
val h0 = f(0)

assert(h0(0) == 0)
\end{verbatim}

A variable can refer to \code{f(0)}. \code{h0} refers to the return value of
\code{f(0)} and has type \code{Int => Int}. Calling variables referring to
function values is possible. \code{h0(0)} is a valid expression and results in
\code{0}.

\begin{verbatim}
val h1 = f
\end{verbatim}
\vspace{-1em}
\begin{mdframed}[hidealllines=true,backgroundcolor=red!10,innerleftmargin=3pt,innerrightmargin=3pt,leftmargin=-3pt,rightmargin=-3pt]
\begin{verbatim}
         ^
error: missing argument list for method f

Unapplied methods are only converted to functions
when a function type is expected.

You can make this conversion explicit
by writing `f _` or `f(_)` instead of `f`.
\end{verbatim}
\vspace{-2em}
\begin{flushright}
\scriptsize\textsf{Compile-time error}
\end{flushright}
\end{mdframed}

On the other hand, defining a variable referring to \code{f} results in a compile
error. In Scala, a function defined by \code{def} is not a value per se. Since \code{f}
is the name of a function but not a variable referring to a value, \code{h1}
cannot refer to the value of \code{f}. As the above error message implies,
underscores convert function names into function values.

\begin{verbatim}
val h1 = f _

assert(h1(0)(0) == 0)
\end{verbatim}

Compiling the above code succeeds. The type of \code{h1} is \code{Int => (Int => Int)}.
\code{Int => Int => Int} denotes the same type because \code{=>} is
a right-associative type operator. \code{h1(0)(0)} is valid and yields \code{0}.

Actually, above expressions except \code{val h1 = f} use function names as values
successfully. The Scala compiler transforms function names into function values
when they occur where function types are expected. Therefore, enforcing the type of
\code{h1} to be a function type corrects the code without the underscore. The
following code works well:

\begin{verbatim}
val h1: Int => Int => Int = f
\end{verbatim}

When programmers use function names as values, they usually place the names where
function types are expected. In these cases, underscores and explicit type
annotations are unnecessary. Code rarely becomes problematic and needs
underscores or type annotations like the above to enforce the transformations.

How does the compiler create function values from function names? If the parameter
type of function \code{f} is \code{Int}, the corresponding function value is
\code{(x: Int) => f(x)}. The transformation is called \textit{eta expansion}.
\index{eta expansion}
\code{(x: Int) => f(x)} is a function value without a name and does the same thing as
\code{f}. The following section covers functions without names.

\section{Anonymous Functions}

In functional programming, functions often appear only once as an argument or a
return value. Naming functions used only once is unnecessary. The meaning of
a function value is how it behave. While the parameters and body of a function
decide its behavior, its name does not affect the behavior. Naturally, functional
languages provide syntax to define functions without giving them names. Such
functions are \textit{anonymous functions}.\index{anonymous function}

The syntax of an anonymous function in Scala is as follows:

\begin{verbatim}
([parameter name]: [parameter type], …) => [expression]
\end{verbatim}

Like functions declared by \code{def},
anonymous functions can be arguments, return values, or values referred by
variables. Directly calling them is possible as well.

\begin{verbatim}
def g(h: Int => Int): Int = h(0)
g((x: Int) => x)

def f(): Int => Int = (x: Int) => x
f()(0)

val h = (x: Int) => x
h(0)

((x: Int) => x)(0)
\end{verbatim}

The code does similar things to the previous code but uses anonymous functions.

Anonymous functions need explicit parameter types as named functions do. However,
annotating every parameter type is verbose and inconvenient. The Scala compiler
infers the types of parameters when anonymous functions occur where the compiler
expects function types.

\begin{verbatim}
def g(h: Int => Int): Int = h(0)
g(x => x)
\end{verbatim}

Since \code{g} has a parameter of type \code{Int => Int}, the compiler expects
\code{x => x} to have the type \code{Int => Int}. It infers the type of \code{x} as
\code{Int}.

\begin{verbatim}
val h: Int => Int = x => x
\end{verbatim}

\code{h} has an explicit type annotation. \code{Int => Int} is the expected type of
\code{x => x}. The compiler infers the type of \code{x} as \code{Int}.

\begin{verbatim}
val h = x => x
\end{verbatim}
\vspace{-1em}
\begin{mdframed}[hidealllines=true,backgroundcolor=red!10,innerleftmargin=3pt,innerrightmargin=3pt,leftmargin=-3pt,rightmargin=-3pt]
\begin{verbatim}
         ^
error: missing parameter type
\end{verbatim}
\vspace{-2em}
\begin{flushright}
\scriptsize\textsf{Compile-time error}
\end{flushright}
\end{mdframed}

Unlike previous one, this code is problematic. Since
there is no information to infer the type of \code{x} in \code{x => x},
the compiler generates an error.

Most cases using anonymous functions are arguments for function calls.
Those functions do not require explicit parameter types. However, beginners might not
be sure about whether parameter types can be omitted or not. Specifying
parameter types is safe when you are not sure.

Scala provides one more syntax for anonymous functions: syntax using
underscores. Underscores help programmers to create anonymous functions in a
concise and intuitive way.
Underscores can be used only when certain conditions are satisfied.
Every parameter must occur exactly once in the body of a function in the
order. Moreover, the function must not be an identity function like \code{(x: Int) => x}.
In functions satisfying the conditions, underscores can replace parameters in the body.
Otherwise, it is impossible to use underscores to create anonymous functions.

\begin{verbatim}
def g0(h: Int => Int): Int = h(0)
g0(_ + 1)

def g1(h: (Int, Int) => Int): Int = h(0, 0)
g1(_ + _)
\end{verbatim}

The compiler transforms \verb!_ + 1! into \code{x => x + 1}.
Similarly, \verb!_ + _!
becomes \code{(x, y) => x + y}. The compiler automatically creates parameters
as many as underscores and substitutes the underscores with the parameters. The
mechanism clearly shows why the aforementioned conditions exist.

\begin{verbatim}
val h0 = (_: Int) + 1
val h1 = (_: Int) + (_: Int)
\end{verbatim}

Underscores can have explicit types.
Programmers should supply parameter types to succeed compiling when
the compiler cannot infer them.

The transformation happens for the shortest expression containing underscores.
Expressing anonymous functions with underscores is sometimes tricky.

\begin{verbatim}
def f(x: Int): Int = x
def g1(h: Int => Int): Int = h(0)
g1(f(_))
\end{verbatim}

As intended, \verb!f(_)! becomes \code{x => f(x)}, whose type is \code{Int =>
Int}.\sidenote{Actually, there is no need to write
\code{g(f(\_))} because it is equal to \code{g(f)}.}

\begin{verbatim}
g1(f(_ + 1))
\end{verbatim}
\vspace{-1em}
\begin{mdframed}[hidealllines=true,backgroundcolor=red!10,innerleftmargin=3pt,innerrightmargin=3pt,leftmargin=-3pt,rightmargin=-3pt]
\begin{verbatim}
     ^
error: missing parameter type for expanded function
((<x$1: error>) => x$1.$plus(1))
\end{verbatim}
\vspace{-2em}
\begin{flushright}
\scriptsize\textsf{Compile-time error}
\end{flushright}
\end{mdframed}

On the other hand, \verb!f(_ + 1)! becomes
\code{f(x => x + 1)} but not \code{x => f(x + 1)}.
As \code{f} takes an integer, not a function, it results in a compile-time error.

\begin{verbatim}
def g2(h: (Int, Int) => Int): Int = h(0, 0)
g2(f(_) + _)
\end{verbatim}

\verb!f(_) + _! becomes \code{(x, y) => f(x) + y}, whose type is \code{(Int,
Int) => Int}, and the compilation succeeds.

\begin{verbatim}
g2(f(_ + 1) + _)
\end{verbatim}
\vspace{-1em}
\begin{mdframed}[hidealllines=true,backgroundcolor=red!10,innerleftmargin=3pt,innerrightmargin=3pt,leftmargin=-3pt,rightmargin=-3pt]
\begin{verbatim}
              ^
error: missing parameter type for expanded function
((<x$2: error>) => f(((<x$1: error>) => x$1.$plus(1)))
  .<$plus: error>(x$2))
\end{verbatim}
\vspace{-2em}
\begin{flushright}
\scriptsize\textsf{Compile-time error}
\end{flushright}
\end{mdframed}

\verb!f(_ + 1) + _! becomes
\code{y => f(x => x + 1) + y} but not \code{(x, y) => f(x + 1) + y}.

Like type inference of parameter types, novices may not be sure about how
anonymous functions with underscores are transformed. It is recommended to use normal anonymous
functions without underscores for those who are not confident about the mechanism
of underscores.

\section{Closures}

\textit{Closures}\index{closure} are function values that capture
environments, which store the values of existing variables, when they are defined.
The bodies of closures may
have variables not defined in themselves, and the environments store the values
of those variables.

\begin{verbatim}
def makeAdder(x: Int): Int => Int = {
  def adder(y: Int): Int = x + y
  adder
}
\end{verbatim}

The definition of \code{adder}, \code{def adder(y: Int): Int = x + y}, does not
define but uses \code{x}. However, the code is correct.

\begin{verbatim}
val add1 = makeAdder(1)
assert(add1(2), 3)

val add2 = makeAdder(2)
assert(add2(2), 4)
\end{verbatim}

\code{add1} and \code{add2} refer to the same \code{adder} function, but the
former returns an integer one larger than an argument, and the latter returns an
integer two larger than an argument. The results of \code{add1(2)} and
\code{add2(2)} are \code{3} and \code{4}, respectively. It is possible because the
closures capture the environments when they are created. \code{add1} refers to a
thing like \code{(adder, x = 1)} instead of just \code{adder}. Similarly,
\code{add2} is actually \code{(adder, x = 2)}. Since the environment of
\code{add1} stores the fact that \code{x} is \code{1}, \code{add1(2)} results in
\code{3}. Under the environment of \code{add2}, \code{x} denotes \code{2}, and
thus \code{x + y} is \code{4} when \code{y} is \code{2}.

\section{First-Class Functions and Lists}

This section shows how first-class functions allow generalization of the functions
defined in \refch{immutability}.

\begin{verbatim}
def inc1(l: List[Int]): List[Int] = l match {
  case Nil => Nil
  case h :: t => h + 1 :: inc1(t)
}

def square(l: List[Int]): List[Int] = l match {
  case Nil => Nil
  case h :: t => h * h :: square(t)
}
\end{verbatim}

\code{inc1} increases every element of a given list by one, and \code{square}
squares every element. The two functions are remarkably similar. To make the
similarity clearer, let us rename the functions to \code{g}.

\begin{verbatim}
def g(l: List[Int]): List[Int] = l match {
  case Nil => Nil
  case h :: t => h + 1 :: g(t)
}

def g(l: List[Int]): List[Int] = l match {
  case Nil => Nil
  case h :: t => h * h :: g(t)
}
\end{verbatim}

The only difference is the left operand of \code{::} in the third line:
\code{h + 1} versus \code{h * h}. By adding one parameter, the functions become
entirely identical.

\begin{verbatim}
def g(l: List[Int], f: Int => Int): List[Int] = l match {
  case Nil => Nil
  case h :: t => f(h) :: g(t, f)
}
g(l, h => h + 1)

def g(l: List[Int], f: Int => Int): List[Int] = l match {
  case Nil => Nil
  case h :: t => f(h) :: g(t, f)
}
g(l, h => h * h)
\end{verbatim}

This function is called \code{map}. The returned list
has elements obtained by \textbf{map}ping a given function to the elements of a
given list.

\begin{verbatim}
def map(l: List[Int], f: Int => Int): List[Int] = l match {
  case Nil => Nil
  case h :: t => f(h) :: map(t, f)
}
\end{verbatim}

\code{inc1} and \code{square} can be redefined with \code{map}.

\begin{verbatim}
def inc1(l: List[Int]): List[Int] = map(l, h => h + 1)
def square(l: List[Int]): List[Int] = map(l, h => h * h)
\end{verbatim}

An underscore makes \code{inc1} conciser.

\begin{verbatim}
def inc1(l: List[Int]): List[Int] = map(l, _ + 1)
\end{verbatim}

Let us compare \code{odd} and \code{positive}.

\begin{verbatim}
def odd(l: List[Int]): List[Int] = l match {
  case Nil => Nil
  case h :: t =>
    if (h % 2 != 0)
      h :: odd(t)
    else
      odd(t)
}

def positive(l: List[Int]): List[Int] = l match {
  case Nil => Nil
  case h :: t =>
    if (h > 0)
      h :: positive(t)
    else
      positive(t)
}
\end{verbatim}

They look similar. They can become identical by renaming and adding parameters.

\begin{verbatim}
def filter(l: List[Int], f: Int => Boolean): List[Int] = l match {
  case Nil => Nil
  case h :: t =>
    if (f(h))
      h :: filter(t, f)
    else
      filter(t, f)
}
\end{verbatim}


The function is called \code{filter} because it \textbf{filter}s
unwanted elements out from a given list.

\code{odd} and \code{positive} can be redefined with \code{filter}.

\begin{verbatim}
def odd(l: List[Int]): List[Int] =
  filter(l, h => h % 2 != 0)
def positive(l: List[Int]): List[Int] =
  filter(l, h => h > 0)
\end{verbatim}

Underscores make the functions conciser.

\begin{verbatim}
def odd(l: List[Int]): List[Int] = filter(l, _ % 2 != 0)
def positive(l: List[Int]): List[Int] = filter(l, _ > 0)
\end{verbatim}

Let us compare \code{sum} and \code{product} without tail recursion.

\begin{verbatim}
def sum(l: List[Int]): Int = l match {
  case Nil => 0
  case h :: t => h + sum(t)
}

def product(l: List[Int]): Int = l match {
  case Nil => 1
  case h :: t => h * product(t)
}
\end{verbatim}

After renaming the names to \code{g}, two differences exist: \code{0} versus
\code{1} and \code{h + g(t)} versus \code{h * g(t)}. By adding two parameters, an
initial value and a function taking \code{h} and \code{g(t)} as arguments, the
functions become identical.

\begin{verbatim}
def foldRight(
  l: List[Int],
  n: Int,
  f: (Int, Int) => Int
): Int = l match {
  case Nil => n
  case h :: t => f(h, foldRight(t, n, f))
}
\end{verbatim}

This function is called \code{foldRight} since it
appends an initial value at the right side of a list and
\textbf{fold}s the list from the \textbf{right} side with a given function.

\code{sum} and \code{product} can be redefined with \code{foldRight}.

\begin{verbatim}
def sum(l: List[Int]): Int =
  foldRight(l, 0, (h, gt) => h + gt)
def product(l: List[Int]): Int =
  foldRight(l, 1, (h, gt) => h * gt)
\end{verbatim}

They may use underscores for conciseness.

\begin{verbatim}
def sum(l: List[Int]): Int = foldRight(l, 0, _ + _)
def product(l: List[Int]): Int = foldRight(l, 1, _ * _)
\end{verbatim}

The following equations give an intuitive interpretation of \code{foldRight}:

\begin{verbatim}
  foldRight(List(a, b, .., y, z), n, f)
= f(a, f(b, .. f(y, f(z, n)) .. ))

  foldRight(List(1, 2, 3), 0, add)
= add(1, add(2, add(3, 0)))

  foldRight(List(1, 2, 3), 1, mul)
= mul(1, mul(2, mul(3, 1)))
\end{verbatim}

Let us compare tail-recursive \code{sum} and \code{product}.

\begin{verbatim}
def sum(l: List[Int]): Int = {
  def aux(l: List[Int], inter: Int): Int = l match {
    case Nil => inter
    case h :: t => aux(t, inter + h)
  }
  aux(l, 0)
}

def product(l: List[Int]): Int = {
  def aux(l: List[Int], inter: Int): Int = l match {
    case Nil => inter
    case h :: t => aux(t, inter * h)
  }
  aux(l, 1)
}
\end{verbatim}

After renaming, there are two differences: \code{inter + h} versus \code{inter * h}
and \code{0} versus \code{1}. Similarly, adding two parameters makes the
functions identical.

\begin{verbatim}
def foldLeft(
  l: List[Int],
  n: Int,
  f: (Int, Int) => Int
): Int = {
  def aux(l: List[Int], inter: Int): Int = l match {
    case Nil => inter
    case h :: t => aux(t, f(inter, h))
  }
  aux(l, n)
}
\end{verbatim}

This function is called \code{foldLeft}. Its semantics is different from
\code{foldRight}. While \code{foldRight} appends an initial value at
the right side and folds a list from the right side, \code{foldLeft}
prepends an initial value at the left side and \textbf{fold}s
a list from the \textbf{left} side. The following equations give an intuitive
interpretation:

\begin{verbatim}
  foldLeft(List(a, b, .., y, z), n, f)
= f(f( .. f(f(n, a), b), .. , y), z)

  foldLeft(List(1, 2, 3), 0, add)
= add(add(add(0, 1), 2), 3)

  foldLeft(List(1, 2, 3), 1, mul)
= mul(mul(mul(1, 1), 2), 3)
\end{verbatim}

The order traversing a list does not affect the results of \code{sum} and
\code{product}. Both \code{foldRight} and \code{foldLeft} can express the functions.

\begin{verbatim}
def sum(l: List[Int]): Int = foldLeft(l, 0, _ + _)
def product(l: List[Int]): Int = foldLeft(l, 1, _ * _)
\end{verbatim}

On the other hand, the order is important for some functions.
Consider a function that takes a list of digits as arguments and returns the
decimal number obtained by concatenating the digits.
\code{foldLeft} is the easiest way to implement this function.

\begin{verbatim}
def digitToDecimal(l: List[Int]) =
  foldLeft(l, 0, _ * 10 + _)

  foldLeft(List(1, 2, 3), 0, f)
= f(f(f(0, 1), 2), 3)
= ((0 * 10 + 1) * 10 + 2) * 10 + 3
= (1 * 10 + 2) * 10 + 3
= 12 * 10 + 3
= 123
\end{verbatim}

Using \code{foldRight} with the same arguments will yield completely different
result.

\begin{verbatim}
def digitToDecimal(l: List[Int]) =
  foldRight(l, 0, _ * 10 + _)

  foldRight(List(1, 2, 3), 0, f)
= f(1, f(2, f(3, 0)))
= 1 * 10 + (2 * 10 + (3 * 10 + 0))
= 1 * 10 + (2 * 10 + 30)
= 1 * 10 + 50
= 60
\end{verbatim}

\code{map}, \code{filter}, \code{foldRight}, and
\code{foldLeft} are powerful functions. The four functions offer concise
implementation for many procedures dealing with lists.
Since they are so useful, the Scala standard library provides \code{map},
\code{filter}, \code{foldRight}, and \code{foldLeft} as the methods of the
\code{List} class. You do not need to implement \code{map},
\code{filter}, \code{foldRight}, and \code{foldLeft} by yourself.

\code{map(l, f)} can be rewritten to \code{l.map(f)} by using the \code{map}
method instead.

\begin{verbatim}
def inc1(l: List[Int]): List[Int] = l.map(_ + 1)
def square(l: List[Int]): List[Int] = l.map(h => h * h)
\end{verbatim}

\code{filter(l, f)} can be rewritten to \code{l.filter(f)} by using the
\code{filter} method instead.

\begin{verbatim}
def odd(l: List[Int]): List[Int] = l.filter(_ % 2 != 0)
def positive(l: List[Int]): List[Int] = l.filter(_ > 0)
\end{verbatim}

\code{foldRight(l, n, f)} can be rewritten to \code{l.foldRight(n)(f)} by using the
\code{foldRight} method instead.

\begin{verbatim}
def sum(l: List[Int]): Int = l.foldRight(0)(_ + _)
def product(l: List[Int]): Int = l.foldRight(1)(_ * _)
\end{verbatim}

\code{foldLeft(l, n, f)} can be rewritten to \code{l.foldLeft(n)(f)} by using the
\code{foldLeft} method instead.

\begin{verbatim}
def sum(l: List[Int]): Int = l.foldLeft(0)(_ + _)
def product(l: List[Int]): Int = l.foldLeft(1)(_ * _)
def digitToDecimal(l: List[Int]) = l.foldLeft(0)(_ * 10 + _)
\end{verbatim}

The methods in the standard library are polymorphic, i.e. they can take
arguments of various types. For example, our \code{map} function takes only a
list of integers. To use \code{map} for a list of students, we need to define a
new version of \code{map}. However, the \code{map} method in the standard
library can take lists of any types as arguments.

\begin{verbatim}
case class Student(name: String, height: Int)

def heights(l: List[Student]): List[Int] = l.map(_.height)
\end{verbatim}

The standard library provides many other useful methods for
lists.\sidenote{\url{https://www.scala-lang.org/api/current/scala/collection/immutable/List.html}}

\section{For Loops}

Scala has for loops.
In Scala, a for loop is an expression, which evaluates to a value.
For expressions are highly expressive.
Unlike \code{while}, which work with mutable variables or objects,
\code{for} of Scala helps programmers to write code in a functional and readable way.

The syntax of a for expression is as follows:

\begin{verbatim}
for ([name] <- [expression])
  yield [expression]
\end{verbatim}

The first expression should result in a collection.
Its result is a collection containing the result of evaluating the second expression
at each iteration.
Therefore, for expressions can replace use of the \code{map} method.

\begin{verbatim}
val l = for (n <- List(0, 1, 2)) yield n * n
assert(l == List(0, 1, 4))
\end{verbatim}

For expressions can appear at any places expecting expressions.

\begin{verbatim}
def square(l: List[Int]): List[Int] =
  for (n <- l)
    yield n * n
\end{verbatim}

In Scala, \code{for} is just syntactic sugar. Instead of giving specific
semantics to \code{for}, syntactic rules transform code using \code{for} into the
code using methods of collections and anonymous functions. The above function becomes
the following function, which uses \code{map}, by the transformation:

\begin{verbatim}
def square(l: List[Int]): List[Int] =
  l.map(n => n * n)
\end{verbatim}

For this reason, for expressions are powerful. Any
user-defined types can appear in for expressions if the
types define \code{map}.

For expressions can replace use of the \code{filter} method as well.

\begin{verbatim}
def positive(l: List[Int]): List[Int] =
  for (n <- l if n > 0)
    yield n
\end{verbatim}

Elements not satisfying a given condition will be omitted during iteration.

In addition, combination of \code{map} and \code{filter} can be expressed with a
for loop concisely. Consider a function that takes a list of students and
returns a list of the names of students whose height is greater than 170.
The function can be implemented with \code{map} and \code{filter} like below.

\begin{verbatim}
def tall(l: List[Student]): List[String] =
  l.filter(_.height > 170).map(_.name)
\end{verbatim}

We can use a for expression instead.

\begin{verbatim}
def tall(l: List[Student]): List[String] =
  for (s <- l if s.height > 170)
    yield s.name
\end{verbatim}

\section{Exercises}

\begin{exercise}
\labex{functions-incby}

Implement a function \code{incBy}:
\begin{verbatim}
def incBy(l: List[Int], n: Int): List[Int] = ???
\end{verbatim}
that takes a list of integers and an integer as
arguments and increases every element of the list by the given integer. Use
the \code{map} method.

\end{exercise}

\begin{exercise}
\labex{functions-gt}

Implement a function \code{gt}:
\begin{verbatim}
def gt(l: List[Int], n: Int): List[Int] = ???
\end{verbatim}
that takes a list of integers and an integer as arguments
and filters elements less than or equal to the given integer out from the list.
Use the \code{filter} method.

\end{exercise}

\begin{exercise}
\labex{functions-append}

Implement a function \code{append}:
\begin{verbatim}
def append(l: List[Int], n: Int]): List[Int] = ???
\end{verbatim}
that takes a list of integers and an integer as arguments
and returns a list obtained by appending the integer at the end of the list.
Use the \code{foldRight} method.

\end{exercise}

\begin{exercise}
\labex{functions-reverse}

Implement a function \code{reverse}:
\begin{verbatim}
def reverse(l: List[Int]): List[Int] = ???
\end{verbatim}
that takes a list of integers
and returns a list obtained by reversing the order between the elements.
Use the \code{foldLeft} method.

\end{exercise}

\setchapterpreamble[u]{\margintoc}
\chapter{Pattern Matching}
\labch{pattern-matching}

This section explains pattern matching of Scala.
Pattern matching is one of the key features of functional programming.
It helps programmers handle complex, but structured data.
We have already used a simple form of pattern matching for lists.
This section discusses the benefits
of pattern matching and various patterns available in Scala.
In addition, it will introduce the option type, which is widely-used in
functional programming.

\section{Algebraic Data Types}

It is common to include values of various shapes in a single type.

A natural number is

\begin{itemize}
\item zero or
\item the successor of a natural number.
\end{itemize}

A list is

\begin{itemize}
\item the empty list or
\item a pair of an element and a list.
\end{itemize}

A binary tree is

\begin{itemize}
\item the empty tree or
\item a tree containing a root element and two child trees.
\end{itemize}

An arithmetic expression is

\begin{itemize}
\item a number,
\item the sum of two arithmetic expressions, or
\item the difference of two arithmetic expressions.
\end{itemize}

An expression of the lambda calculus is

\begin{itemize}
\item a variable,
\item a function, which is a pair of a variable and an expression, or
\item a function application, which is a pair of two expressions.
\end{itemize}

As the examples show, in computer science, a single type often includes values of
various shapes. \textit{Algebraic data types}\index{algebraic data type}
(\acrshort{adtLabel}) express such types. An ADT
is the sum type of product types. That is why it is called ``algebraic.''
A \textit{product type}\index{product type}
is a type whose every element is an enumeration of values of types in the
same specific order. Tuple types are typical product types. A \textit{sum
type}\index{sum type}, whose
another name is a \textit{tagged union type}\index{tagged union type},
has values of multiple types as its values. Unlike a union type,
each component of a sum type has a
tag to be distinguished from other components.
In an ADT, one form of values that can be distinguished from the other forms is
called a \textit{variant}\index{variant}.

For example, an arithmetic expression, which has three variants, is

\begin{itemize}
\item a number,
\item the sum of two arithmetic expressions, or
\item the difference of two arithmetic expressions.
\end{itemize}

Therefore, we can define the \code{AE} type, which is the type of an arithmetic
expression, as the sum type of

\begin{itemize}
\item the \code{Int} type tagged with \code{Num},
\item the \code{AE * AE} type tagged with \code{Add}, and
\item the \code{AE * AE} type tagged with \code{Sub},
\end{itemize}

where \code{AE * AE} denotes the product type of \code{AE} and \code{AE}.

ADTs are common in functional languages. Most functional
languages allow users to define new ADTs. The following OCaml
code defines arithmetic expressions:

\begin{verbatim}
type ae =
| Num of int
| Add of ae * ae
| Sub of ae * ae
\end{verbatim}

Scala does not provide a direct way to define ADTs. Instead, Scala provides
traits and classes, which are more general mechanisms to define new types,
and programmers can express ADTs with traits and classes.

A new type can be defined as a trait.
The syntax of a trait definition is as follows:

\begin{verbatim}
trait [name]
\end{verbatim}

It defines a type whose name is \code{[name]}.
The following code defines the \code{AE} type,
which is the type of arithmetic expressions:

\begin{verbatim}
sealed trait AE
\end{verbatim}

The \code{sealed} modifier prevents \code{AE} being extended outside the file
that defines \code{AE}. We will get back to this point when we discuss the
exhaustivity checking of pattern matching.

Once a type is defined as a trait, the type can be used just like any other
types. For example, we can define an identity function for arithmetic
expressions.

\begin{verbatim}
def identity(ae: AE): AE = ae
\end{verbatim}

However, traits do not have ability to construct new values. It means that there
is no way to create a value of the type \code{AE} yet. We need to define the
variants of \code{AE} as case classes by extending \code{AE}.

\begin{verbatim}
case class Num(value: Int) extends AE
case class Add(left: AE, right: AE) extends AE
case class Sub(left: AE, right: AE) extends AE
\end{verbatim}

As you have seen in \refch{introduction-to-scala}, we can easily create values of case classes.

\begin{verbatim}
val n = Num(10)
val m = Num(5)
val e1 = Add(n, m)
val e2 = Sub(e1, Num(3))
\end{verbatim}

Like traits, case classes also define types. The name of each class is the name
of the defined type. Every instance of a class belongs to the type corresponding
to the class.

\begin{verbatim}
val n: Num = Num(10)
val m: Num = Num(5)
val e1: Add = Add(n, m)
val e2: Sub = Sub(e1, Num(3))
\end{verbatim}

In addition, because of the \code{extends} keyword, \code{Num}, \code{Add}, and
\code{Sub} are subtypes of \code{AE}. It means that any value of the types
\code{Num}, \code{Add}, or \code{Sub} is also a value of the type \code{AE}.

\begin{verbatim}
val n: AE = Num(10)
val m: AE = Num(5)
val e1: AE = Add(n, m)
val e2: AE = Sub(e1, Num(3))
\end{verbatim}

We know that we can access the fields of objects with their names.

\begin{verbatim}
val n: Num = Num(10)
n.value
\end{verbatim}

However, we cannot access the fields of an object when its type becomes \code{AE}.

\begin{verbatim}
val m: AE = Num(10)
m.value
\end{verbatim}
\vspace{-1em}
\begin{mdframed}[hidealllines=true,backgroundcolor=red!10,innerleftmargin=3pt,innerrightmargin=3pt,leftmargin=-3pt,rightmargin=-3pt]
\begin{verbatim}
  ^
error: value value is not a member of AE
\end{verbatim}
\vspace{-2em}
\begin{flushright}
\scriptsize\textsf{Compile-time error}
\end{flushright}
\end{mdframed}

The reason is that \code{m} can be \code{Add} or \code{Sub}, which do not have
the field \code{value}, as \code{AE} consists of not only \code{Num} but also
\code{Add} and \code{Sub}. The compiler thinks that \code{m} may not have the
field \code{value} and considers \code{m.value} as an unsafe expression, which
should be rejected.

The best way to use ADTs is pattern matching. The following function evaluates a
given arithmetic expression and returns the number denoted by the arithmetic
expression.

\begin{verbatim}
def eval(e: AE): Int = e match {
  case Num(n) => n
  case Add(l, r) => eval(l) + eval(r)
  case Sub(l, r) => eval(l) - eval(r)
}

assert(eval(Sub(Add(Num(3), Num(7)), Num(5))) == 5)
\end{verbatim}

When \code{e} is \code{Num(n)}, \code{eval} simply returns \code{n}.
When \code{e} is \code{Add(l, r)}, \code{eval} respectively evaluates \code{l}
and \code{r}, which are arithmetic expressions. \code{eval} returns the sum of
the results of \code{l} and \code{r}.
The \code{Sub(l, r)} case is similar. \code{eval} returns the difference of
the results of \code{l} and \code{r}.

The list type is another good example of an ADT. The Scala standard library defines
lists similar to the following code:

\begin{verbatim}
sealed trait List[+A]
case object Nil extends List[Nothing]
case class ::[A](head: A, tail: List[A]) extends List[A]
\end{verbatim}

This code omits some details but clearly shows the high-level idea to define
lists.\sidenote{We will not see what \code{[+A]} and \code{Nothing} are here.
You can understand the overall ADT structure without knowing those concepts.}
A list is either the empty
list or a nonempty list, which is a pair of its head and tail. \code{Nil} is
defined as a case object, not a case class, since there is only one empty list.
Every empty list is identical. We use a case object to express this idea.
\code{Nil} is created only once during entire execution, and every \code{Nil} is
identitcal. The name \code{::} looks a bit weird, but it is for
readability of pattern matching. Scala allows writing class names as infix
operators in patterns. It means that both \code{case ::(h, t) =>} and \code{case
h :: t =>} are allowed. Due to the class name \code{::}, we can write \code{case
h :: t =>} in pattern matching.

\section{Advantages}

\subsection{Conciseness}

Without pattern matching, handling ADTs becomes a complicated job. We need to
use dynamic type tests to distinguish variants and type casting to access the
fields of values.

Below is \code{eval} without pattern matching:

\begin{verbatim}
def eval(e: AE): Int =
  if (e.isInstanceOf[Num])
    e.asInstanceOf[Num].value
  else if (e.isInstanceOf[Add]) {
    val e0 = e.asInstanceOf[Add]
    eval(e0.left) + eval(e0.right)
  } else {
    val e0 = e.asInstanceOf[Sub]
    eval(e0.left) - eval(e0.right)
  }
\end{verbatim}

\code{e.isInstanceOf[Num]} tests whether \code{e} is an instance of class
\code{Num}. If it is true, \code{eval} should return the value of the field
\code{value} of \code{e}. However, \code{value} cannot be directly accessed
since \code{e}'s type is \code{AE}. Because \code{e.isInstanceOf[Num]} is true,
we are sure that \code{e}'s actual type is \code{Num}. In this case, we can
inform this knowledge to the compiler with type casting. \code{e.asInstanceOf[Num]}
does not change the value denoted by \code{e} but lets the compiler know that
the programmer guarantees the type of \code{e} to be \code{Num}. Therefore, the
compiler considers \code{e.asInstanceOf[Num]} to belong \code{Num} and allows
accessing the field \code{value}. These type tests and casting processes should
be done for the other variants, \code{Add} and \code{Sub}, too.

The code is long and complicated despite its simple functionality. Dynamic type
tests and explicit type casting occupy most of the code, while real
computation requires short code. Besides, such code is error-prone.
For example, programmers may write code like below by mistake:

\begin{verbatim}
else if (e.isInstanceOf[Add]) {
  val e0 = e.asInstanceOf[Sub]
  eval(e0.left) + eval(e0.right)
}
\end{verbatim}

While the condition checks whether \code{e} is an instance of \code{Add},
\code{e} becomes casted to \code{Sub}. Such code will trigger an error at run
time and terminate the execution abnormally.
It is easy to check whether \code{eval} is correct because it is short.
However, complex types and computation will increase the possibility of
mistakes.

Pattern matching gives us a much better solution. Pattern matching hides type
tests and casting and makes code concise. At the same time, pattern matching
removes the possibility of mistakes.

\subsection{Exhaustivity Checking}

Pattern matching checks the exhaustivity of patterns. At run time, a match error
occurs when a given value matches none of given patterns.

\begin{verbatim}
def eval(e: AE): Int = e match {
  case Add(l, r) => eval(l) + eval(r)
  case Sub(l, r) => eval(l) - eval(r)
}
\end{verbatim}

The function lacks the \code{Num} pattern.

\begin{verbatim}
eval(Num(3))
\end{verbatim}
\vspace{-1em}
\begin{mdframed}[hidealllines=true,backgroundcolor=red!10,innerleftmargin=3pt,innerrightmargin=3pt,leftmargin=-3pt,rightmargin=-3pt]
\begin{verbatim}
scala.MatchError: Num(3) (of class Num)
\end{verbatim}
\vspace{-2em}
\begin{flushright}
\scriptsize\textsf{Run-time error}
\end{flushright}
\end{mdframed}

An argument of type \code{Num} results in a match error at run time.

Fortunately, we can easily avoid such mistakes.
The Scala compiler checks whether patterns are exhaustive and warns if they are not.

\begin{verbatim}
def eval(e: AE): Int = e match {
  case Add(l, r) => eval(l) + eval(r)
  case Sub(l, r) => eval(l) - eval(r)
}
\end{verbatim}
\vspace{-1em}
\begin{mdframed}[hidealllines=true,backgroundcolor=yellow!10,innerleftmargin=3pt,innerrightmargin=3pt,leftmargin=-3pt,rightmargin=-3pt]
\begin{verbatim}
                       ^
warning: match may not be exhaustive.
It would fail on the following input: Num(_)
\end{verbatim}
\vspace{-2em}
\begin{flushright}
\scriptsize\textsf{Compile-time warning}
\end{flushright}
\end{mdframed}

The compiler warns programmers about that the patterns are not exhaustive. Moreover, it precisely
informs which patterns are missing to help debugging.
Exhaustivity checking is beneficial for complex programs. It helps programmers make
error-free programs and thus is a crucial strength of pattern matching.

For exhaustivity checking, the \code{sealed} modifier of traits is necessary.
Without \code{sealed}, a trait can be extended outside the file that defines it.
The unit of compilation is a single file, so it is impossible to find
all the variants by scanning a single file when a trait is not sealed.
Exhaustivity checking during pattern matching will be impossible.
The \code{sealed} keyword resolves the problem. Since sealed traits cannot be
extended further, it is enough to check only the file that defines a sealed trait to find
every variant of the trait. It is why we use sealed traits to define ADTs.

\subsection{Reachability Checking}

Like \code{switch-case}, pattern
matching compares a value to patterns sequentially from top to bottom and selects
the first matching pattern. If there are duplicated patterns, the latter will
not be reachable.
The compiler warns programmers when they find unreachable patterns to prevent such code.

\begin{verbatim}
def eval(e: AE): Int = e match {
  case Num(n) => 0
  case Add(l, r) => eval(l) + eval(r)
  case Num(n) => n
  case Sub(l, r) => eval(l) - eval(r)
}
\end{verbatim}
\vspace{-1em}
\begin{mdframed}[hidealllines=true,backgroundcolor=yellow!10,innerleftmargin=3pt,innerrightmargin=3pt,leftmargin=-3pt,rightmargin=-3pt]
\begin{verbatim}
  case Num(n) => n
                 ^
warning: unreachable code
\end{verbatim}
\vspace{-2em}
\begin{flushright}
\scriptsize\textsf{Compile-time warning}
\end{flushright}
\end{mdframed}

When code is simple and short, it is easy to check whether there are unreachable
patterns. However, in complex code, programmers often insert unreachable
patterns by mistake and make critical bugs. Reachability checking of the
compiler is an important feature to prevent such bugs.

\section{Patterns in Scala}

\subsection{Constant and Wildcard Patterns}

\code{switch-case} statements divide a given value into multiple cases in
imperative languages. Pattern matching is a general form of \code{switch-case}.
The following code is an example using a \code{switch-case} statement in Java:

\begin{verbatim}
String grade(int score) {
  switch (score / 10) {
    case 10: return "A";
    case 9: return "A";
    case 8: return "B";
    case 7: return "C";
    case 6: return "D";
    default: return "F";
  }
}
\end{verbatim}

Constant and wildcard patterns exist in Scala. Constant patterns are
literals like integers and strings. A constant pattern matches a given
value if a value denoted by the pattern equals the given value. The underscore
denotes the wildcard pattern, which matches every value, and is equivalent to
\code{default} of \code{switch-case}. The following function rewrites the
previous function with pattern matching:

\begin{verbatim}
def grade(score: Int): String =
  (score / 10) match {
    case 10 => "A"
    case 9 => "A"
    case 8 => "B"
    case 7 => "C"
    case 6 => "D"
    case _ => "F"
  }

assert(grade(85) == "B")
\end{verbatim}

\subsection{Or Patterns}

\code{switch-case} statements use the fall-through semantics; if \code{break}
does not exist, after executing code corresponding to a case, the flow of the
execution moves to code corresponding to the next case. Since the results of
cases \code{10} and \code{9} are identical, the function can use fall-through.

\begin{verbatim}
String grade(int score) {
  switch (score / 10) {
    case 10:
    case 9: return "A";
    case 8: return "B";
    case 7: return "C";
    case 6: return "D";
    default: return "F";
  }
}
\end{verbatim}

In contrast, pattern matching disallows fall-through. Instead, or patterns
give a way to write the same expression only once for multiple patterns. The
syntax of an or pattern is \code{[pattern] | [pattern] | …}, which is a sequence
of multiple patterns with vertical bars in between. \code{A | B} matches values
that match \code{A} or \code{B}.

\begin{verbatim}
def grade(score: Int): String =
  (score / 10) match {
    case 10 | 9 => "A"
    case 8 => "B"
    case 7 => "C"
    case 6 => "D"
    case _ => "F"
  }

assert(grade(100) == "A")
\end{verbatim}

\subsection{Nested Patterns}

Nested patterns are patterns containing patterns.
The \code{optimizeAdd} function
optimizes a given arithmetic expression by eliminating additions of zeros.

\begin{verbatim}
def optimizeAdd(e: AE): AE = e match {
  case Num(_) => e
  case Add(Num(0), r) => optimizeAdd(r)
  case Add(l, Num(0)) => optimizeAdd(l)
  case Add(l, r) => Add(optimizeAdd(l), optimizeAdd(r))
  case Sub(l, r) => Sub(optimizeAdd(l), optimizeAdd(r))
}
\end{verbatim}

Nested patterns help programmers treat values with complex structures easily.

\subsection{Patterns with Binders}

Assume that we have one more variant of \code{AE}:

\begin{verbatim}
case class Abs(e: AE) extends AE
\end{verbatim}

It denotes the absolute value of an operand.
Optimizing an arithmetic expression decorated by two
consecutive \code{Abs} operators results in the arithmetic expression with only
one \code{Abs} operator.

\begin{verbatim}
def optimizeAbs(e: AE): AE = e match {
  case Num(_) => e
  case Add(l, r) => Add(optimizeAbs(l), optimizeAbs(r))
  case Sub(l, r) => Sub(optimizeAbs(l), optimizeAbs(r))
  case Abs(Abs(e0)) => optimizeAbs(Abs(e0))
  case Abs(e0) => Abs(optimizeAbs(e0))
}
\end{verbatim}

A flaw of the implementation is that a value matching \code{Abs(e0)}
cannot be an argument of \code{optimizeAbs} directly, and constructing a new
\code{Abs} instance containing a value matching \code{e0} is essential.
The \code{@} symbol makes code efficient by binding a value matching to a pattern to a variable.
Pattern \code{[variable] @ [pattern]} makes the variable refer to a value
matching the pattern.

\begin{verbatim}
def optimizeAbs(e: AE): AE = e match {
  case Num(_) => e
  case Add(l, r) => Add(optimizeAbs(l), optimizeAbs(r))
  case Sub(l, r) => Sub(optimizeAbs(l), optimizeAbs(r))
  case Abs(e0 @ Abs(_)) => optimizeAbs(e0)
  case Abs(e0) => Abs(optimizeAbs(e0))
}
\end{verbatim}

\subsection{Type Patterns}

In \code{optimizeAbs},
the first \verb!Num(_)! pattern does no more than checking whether a value
belongs to type \code{Num}. A type pattern helps to rewrite the function. Type
patterns are in the form of \code{[name]: [type]}. If a value belongs to the
type, it matches the pattern, and the variable refers to the value. The wildcard
pattern can substitute the name if the variable is unnecessary.

\begin{verbatim}
def optimizeAbs(e: AE): AE = e match {
  case _: Num => e
  case Add(l, r) => Add(optimizeAbs(l), optimizeAbs(r))
  case Sub(l, r) => Sub(optimizeAbs(l), optimizeAbs(r))
  case Abs(e0 @ Abs(_)) => optimizeAbs(e0)
  case Abs(e0) => Abs(optimizeAbs(e0))
}
\end{verbatim}

Type patterns are useful for dynamic type checking. The following function takes
any value as an argument and check whether it is a string or
not.\sidenote{Every type is a subtype of \code{Any}, i.e. every value belongs to
\code{Any}.}

\begin{verbatim}
def show(x: Any): String = x match {
  case s: String => s + " is a string"
  case _ => "not a string"
}

assert(show("1") == "1 is a string")
assert(show(1) == "not a string")
\end{verbatim}

Note that type patterns cannot check type arguments of polymorphic types. Using
type patterns against polymorphic types is dangerous.

\begin{verbatim}
def show(x: Any): String = x match {
  case _: List[String] => "a list of strings"
  case _ => "not a list of strings"
}
\end{verbatim}
\vspace{-1em}
\begin{mdframed}[hidealllines=true,backgroundcolor=yellow!10,innerleftmargin=3pt,innerrightmargin=3pt,leftmargin=-3pt,rightmargin=-3pt]
\begin{verbatim}
          ^
warning: non-variable type argument String in type pattern
List[String] is unchecked since it is eliminated by erasure
\end{verbatim}
\vspace{-1.5em}
\begin{flushright}
\scriptsize\textsf{Compile-time warning}
\end{flushright}
\end{mdframed}

\begin{verbatim}
val l: List[Int] = List(1, 2, 3)
assert(show(l) == "a list of strings")  // weird result
\end{verbatim}

Although the type of the argument is \code{List[Int]}, it matches the first
pattern. As the warnings imply, the JVM uses type erasure
semantics, and type arguments are unavailable at run time.

\subsection{Tuple Patterns}

The syntax of a tuple pattern is \code{([pattern], …, [pattern])}.
It matches a tuple whose elements respectively match the internal patterns.

The following function uses tuple patterns to check
whether two lists are identical:

\begin{verbatim}
def equal(l0: List[Int], l1: List[Int]): Boolean =
  (l0, l1) match {
    case (h0 :: t0, h1 :: t1) =>
      h0 == h1 && equal(t0, t1)
    case (Nil, Nil) => true
    case _ => false
  }
\end{verbatim}

\subsection{Pattern Guards}

A binary search tree is

\begin{itemize}
\item the empty tree or
\item a tree containing an integral root element and two child trees.
\end{itemize}

\begin{verbatim}
sealed trait Tree
case object Empty extends Tree
case class Node(root: Int, left: Tree, right: Tree) extends Tree
\end{verbatim}

The function \code{add} takes a tree and an integer as arguments and returns a tree
obtained by adding the integer to the tree. If the integer is an element of the
given tree, the tree itself is the return value.

\begin{verbatim}
def add(t: Tree, n: Int): Tree =
  t match {
    case Empty => Node(n, Empty, Empty)
    case Node(m, t0, t1) =>
      if (n < m)
        Node(m, add(t0, n), t1)
      else if (n > m)
        Node(m, t0, add(t1, n))
      else
        t
  }
\end{verbatim}

An expression corresponding to the second pattern uses \code{if-else}. Pattern
guards allow adding constraints to patterns. A pattern in the form of
\code{[pattern] if [expression]} matches a value if the value matches the
pattern, and the expression results in \code{true}. The following version of \code{add}
uses pattern guards:

\begin{verbatim}
def add(t: Tree, n: Int): Tree =
  t match {
    case Empty => Node(n, Empty, Empty)
    case Node(m, t0, t1) if n < m =>
      Node(m, add(t0, n), t1)
    case Node(m, t0, t1) if n > m =>
      Node(m, t0, add(t1, n))
    case _ => t
  }
\end{verbatim}

Guarded patterns may be inexhaustive and need care.

\begin{verbatim}
def add(t: Tree, n: Int): Tree =
  t match {
    case Empty => Node(n, Empty, Empty)
    case Node(m, t0, t1) if n < m =>
      Node(m, add(t0, n), t1)
    case Node(m, t0, t1) if n > m =>
      Node(m, t0, add(t1, n))
  }
\end{verbatim}

The patterns in the above code is not exhaustive, but
the compiler does not warn programmers about the inexhaustivity.

\subsection{Patterns with Backticks}

The function \code{remove} takes a tree and an integer as arguments and returns a
tree obtained by removing the integer from the tree. If the integer is not an
element of the tree, the given tree itself is the return value. \code{removeMin}
is a helper function used by \code{remove}. It returns the pair of the smallest
element of a given tree and a tree obtained by removing the element from the
tree.

\begin{verbatim}
def removeMin(t: Tree): (Int, Tree) = {
  t match {
    case Node(n, Empty, t1) =>
      (n, t1)
    case Node(n, t0: Node, t1) =>
      val (min, t2) = removeMin(t0)
      (min, Node(n, t2, t1))
  }
}

def remove(t: Tree, n: Int): Tree = {
  t match {
    case Empty =>
      Empty
    case Node(m, t0, Empty) if n == m =>
      t0
    case Node(m, t0, t1: Node) if n == m =>
      val res = removeMin(t1)
      val min = res._1
      val t2 = res._2
      Node(min, t0, t2)
    case Node(m, t0, t1) if n < m =>
      Node(m, remove(t0, n), t1)
    case Node(m, t0, t1) if n > m =>
      Node(m, t0, remove(t1, n))
  }
}
\end{verbatim}

\verb!Node(`n`, t0, Empty)! can replace
\code{case Node(m, t0, Empty) if n == m}. The pattern \code{Node(n, t0, Empty)} defines
a new variable \code{n} and makes \code{n} refer to the
root element; it does not check whether the root element equals \code{n}.
However, backticks prohibit defining a new variable and allow to compare the root
element to \code{n} in the scope.

\begin{verbatim}
def remove(t: BinTree, n: Int): BinTree = {
  t match {
    case Empty =>
      Empty
    case Node(`n`, t0, Empty) =>
      t0
    case Node(`n`, t0, t1: Node) =>
      val res = removeMin(t1)
      val min = res._1
      val t2 = res._2
      Node(min, t0, t2)
    case Node(m, t0, t1) if n < m =>
      Node(m, remove(t0, n), t1)
    case Node(m, t0, t1) if n > m =>
      Node(m, t0, remove(t1, n))
  }
}
\end{verbatim}

\section{Applications of Pattern Matching}

\subsection{Variable Definitions}

It is possible to define variables with pattern matching.

\begin{verbatim}
val (n, m) = (1, 2)
assert(n == 1 && m == 2)

val (a, b, c) = ("a", "b", "c")
assert(a == "a" && b == "b" && c == "c")

val h :: t = List(1, 2, 3, 4)
assert(h == 1 && t == List(2, 3, 4))

val Add(l, r) = Add(Num(1), Num(2))
assert(l == Num(1) && r == Num(2))
\end{verbatim}

Pattern matching helps programmers declare variables concisely, but a match error occurs
when the pattern does not match the right-hand-side value. It is desirable to use
pattern matching only when there is a guarantee that the match succeeds. Since
a tuple pattern always matches a tuple value of the same length,
tuple patterns are widely used for variable definitions.

\subsection{Anonymous Functions}

The function \code{toSum} takes a list of pairs of two integers as arguments and
returns a list whose elements are the sums of the integers in the pairs.

\begin{verbatim}
def toSum(l: List[(Int, Int)]): List[Int] =
  l.map(p => p match {
    case (n, m) => n + m
  })

val l = List((0, 1), (2, 3), (3, 4))
assert(toSum(l) == List(1, 5, 7))
\end{verbatim}

The anonymous function directly uses parameter \code{p} as the target of the
pattern matching. Scala allows simplification of \verb!x => x match { … }! to
\verb!{ … }!. Therefore, we can use an enumeration of patterns as an anonymous
function.

\begin{verbatim}
def toSum(l: List[(Int, Int)]): List[Int] =
  l.map({ case (n, m) => n + m })
\end{verbatim}

\subsection{For Loops}

\code{toSum} can use a for expression instead of \code{map}.

\begin{verbatim}
def toSum(l: List[(Int, Int)]): List[Int] =
  for (p <- l)
    yield p match { case (n, m) => n + m }
\end{verbatim}

For expressions directly support pattern matching.

\begin{verbatim}
def toSum(l: List[(Int, Int)]): List[Int] =
  for ((n, m) <- l)
    yield n + m
\end{verbatim}

The code is readable and concise.

\section{Options}
\labsec{options}

The option type is a widely-used ADT. It represents a value whose existence is
optional. This section introduces the option type and explains the usage of
options.

Consider the function \code{get}, which takes a list and integer \code{n} as
arguments and returns the \code{n}th element of the list. It is problematic
when \code{n} is negative or exceeds the length of the list. Throwing exceptions
is a widely used solution in imperative languages. In Scala, \code{throw
[expression]} throws an exception. For convenience, we define the function
\code{error}, which throws an exception, like below and use it throughout the
book.

\begin{verbatim}
def error(msg: String) = throw new Exception(msg)

def get(l: List[Int], n: Int): Int =
  if (n < 0)
    error("index out of bounds")
  else l match {
    case Nil =>
      error("index out of bounds")
    case h :: t =>
      if (n == 0)
        h
      else
        get(t, n - 1)
  }
\end{verbatim}

Throwing an exception is a simple and effective solution. However, exceptions
have two problems. First, exceptions should be handled by exception handlers.

\begin{verbatim}
try {
  get(List(1, 2), 2)
} catch {
  case e: Exception =>
    // prints "index out of bounds"
    println(e.getMessage)
}
\end{verbatim}

If an exception is not handled properly, it will eventually cause a run-time
error and terminate the execution.

\begin{verbatim}
get(List(1, 2), 2)
\end{verbatim}
\vspace{-1em}
\begin{mdframed}[hidealllines=true,backgroundcolor=red!10,innerleftmargin=3pt,innerrightmargin=3pt,leftmargin=-3pt,rightmargin=-3pt]
\begin{verbatim}
java.lang.Exception: index out of bounds
\end{verbatim}
\vspace{-2em}
\begin{flushright}
\scriptsize\textsf{Run-time error}
\end{flushright}
\end{mdframed}

The Scala compiler does not check whether exceptions are handled properly.
It means that there will not be any compile-time error even if there is a
possibility of unhandled exceptions.

Another problem of exceptions is that exception handling is not local.
When an exception is thrown, the control flow suddenly jumps to the position of
the nearest exception handler. Non-local transition of the control flow usually
hinders programmers from understanding code.
Therefore, implementing \code{get} without exceptions is desirable.

The first attempt is to use \code{null}. \code{null} is a value that denotes that
it does not refer to any existing object. We can try to make \code{get} return
\code{null} when a given index is invalid.

\begin{verbatim}
def get(l: List[Int], n: Int): Int =
  if (n < 0)
    null
  else l match {
    case Nil => null
    case h :: t =>
      if (n == 0)
        h
      else
        get(t, n - 1)
  }
\end{verbatim}
\vspace{-1em}
\begin{mdframed}[hidealllines=true,backgroundcolor=red!10,innerleftmargin=3pt,innerrightmargin=3pt,leftmargin=-3pt,rightmargin=-3pt]
\begin{verbatim}
    null
    ^
error: an expression of type Null is ineligible
for implicit conversion

    case Nil => null
                ^
error: an expression of type Null is ineligible
for implicit conversion
\end{verbatim}
\vspace{-2em}
\begin{flushright}
\scriptsize\textsf{Run-time error}
\end{flushright}
\end{mdframed}

Unfortunately, \code{null} is not an element of \code{Int} in Scala.
The compiler rejects the code.
Even with the assumption that we can treat \code{null} as \code{Int},
using \code{null} is a bad solution. Dereferencing \code{null} causes a
run-time error, which is the well-known \code{NullPointerException}.
The compiler does not check whether \code{null} is dereferenced.
Therefore, using \code{null} is nothing better than using exceptions.
Use of \code{null} has been criticized enormously because \code{null} is extremely
error-prone. Even Tony Hoare, the inventor of \code{null}, said that inventing
\code{null} was a terrible mistake:

\begin{quote}
I call it my billion-dollar mistake. It was the invention of the null reference
in 1965.\sidenote{\url{https://en.wikipedia.org/wiki/Null\_pointer\#History}}
\end{quote}

The second attempt is to use a particular error-indicating value, e.g. \code{-1}.

\begin{verbatim}
def get(l: List[Int], n: Int): Int =
  if (n < 0)
    -1
  else l match {
    case Nil =>
      -1
    case h :: t =>
      if (n == 0)
        h
      else
        get(t, n - 1)
  }
\end{verbatim}

The strategy has an obvious problem. The caller cannot distinguish two
situations:
\begin{itemize}
  \item The list contains \code{-1}.
  \item The index is invalid.
\end{itemize}
Default values can be successful solutions for certain purposes but do not fit \code{get}.

Instead of using a fixed particular value in \code{get}, the caller can specify the default value.

\begin{verbatim}
def getOrElse(l: List[Int], n: Int, default: Int): Int =
  if (n < 0)
    default
  else l match {
    case Nil =>
      default
    case h :: t =>
      if (n == 0)
        h
      else
        getOrElse(t, n - 1, default)
}
\end{verbatim}

It works well when an appropriate default value
exists. However, when checking failures is per se important, the new strategy is
as bad as the previous strategy. There is no way to distinguish an element and
the default value.

Functional languages provide the option type to handle erroneous situations
safely. As the name implies, it represents an optional existence of a value.
The Scala standard library defines the option type like below.\sidenote{
We will not see what \code{[+A]} and \code{Nothing} are here.
You can understand the overall ADT structure without knowing those concepts.}

\begin{verbatim}
sealed trait Option[+A]
case object None extends Option[Nothing]
case class Some[A](value: A) extends Option[A]
\end{verbatim}

An option that may have a value of type \code{T} has type \code{Option[T]}.
An option is either \code{None} or \code{Some}.
\code{None} is a value that does not denote any value and similar
to \code{null}. It indicates a problematic situation. Like \code{Nil}, it is
defined as a case object because every \code{None} is identical. \code{Some} constructs a value that
denotes that a value exists. It is similar to a reference to a real object and
indicates that computation has succeeded.

The following code defines \code{getOption}, which returns an option.

\begin{verbatim}
def getOption(l: List[Int], n: Int): Option[Int] =
  if (n < 0)
    None
  else l match {
    case Nil =>
      None
    case h :: t =>
      if (n == 0)
        Some(h)
      else
        getOption(t, n - 1)
  }

assert(getOption(List(1, 2), 0) == Some(1))
assert(getOption(List(1, 2), 2) == None)
\end{verbatim}

For an invalid index, the return value is \code{None}. The caller can notice
that the operation has failed by \code{None}.
Otherwise, the function packs
an element inside \code{Some} to make the return value.

The Scala standard library uses options in many places. Various methods return options.
For example, \code{headOption} of a list returns \code{None} when the list is
empty. Otherwise, \code{Some} containing the head of the list is returned.

\begin{verbatim}
assert(List().headOption == None)
assert(List(1).headOption == Some(1))
\end{verbatim}

Also, \code{get} of a map returns \code{None} when the map does not have a given key.
Otherwise, \code{Some} containing the value corresponding to the key is
returned.

\begin{verbatim}
val m = Map(1 -> "one", 2 -> "two")
assert(m.get(0) == None)
assert(m.get(1) == Some("one"))
\end{verbatim}

Pattern matching allows programmers to deal with options by
distinguishing the \code{None} and \code{Some} cases. In addition, like the
methods of lists, options also provide methods to abstract common patterns.
We are going to see two methods: \code{map} and \code{flatMap}.

\code{map} can be used when we want to apply some computation only when the
previous computation has succeeded. \code{map} takes a single argument, which
must be a function. \code{opt.map(f)} results in \code{None} when \code{opt} is
\code{None}. If \code{opt} is \code{Some(v)}, then \code{opt.map(f)} evaluates
to \code{Some(f(v))}.

As an example, let us consider a map containing students.
Names are the keys, and students are the values. We want to find a student by a name and
get one's height only when the student exists. It can be implemented with
\code{map}.

\begin{verbatim}
def getHeight(
  m: Map[String, Student],
  name: String
): Option[Int] =
  m.get(name).map(_.height)
\end{verbatim}

If \code{m.get(name)} is \code{None}, then \code{m.get(name).map(\_.height)} also
is \code{None}. Otherwise, \code{m.get(name)} should be \code{Some(student)}, and
\code{m.get(name).map(\_.height)} will result in \code{Some(student.height)}.

In summary, \code{map} is useful when the computation consists of two steps, and
the first step can fail.

\code{flatMap} is similar to \code{map} but a bit different. It is useful when
the computation consists of two steps, and both steps can fail.
\code{flatMap} takes a single argument, which must be a function that returns an option.
\code{opt.flatMap(f)} results in \code{None} when \code{opt} is
\code{None}. If \code{opt} is \code{Some(v)}, then \code{opt.flatMap(f)} evaluates
to \code{f(v)}.

Let us consider a list of names and a map like before.
When the list is nonempty, we will find a student with the first name in the
list from the map. It is a typical application of \code{flatMap}.

\begin{verbatim}
def getStudent(
  l: List[String],
  m: Map[String, Student]
): Option[Student] =
  l.headOption.flatMap(m.get)
\end{verbatim}

The standard library provides many other useful methods for
options.\sidenote{\url{https://www.scala-lang.org/api/current/scala/Option.html}}


\pagelayout{wide} % No margins
\addpart{Untyped Languages}
\pagelayout{margin} % Restore margins

\setchapterpreamble[u]{\margintoc}
\chapter{Syntax}
\labch{syntax}

The course defines programming languages. Defining a language is defining the
\term{syntax} and the \term{semantics} of the language. The article is about
syntax. Before going into detail about syntax, it firstly explains why defining a
language is essential.

Languages defined by the course are tiny and whom people do not use in practice.
For example, they cannot get input from users or print results; they do not have
typical types, including a string type and a floating-point number type. It seems
meaningless to define languages that do not have any usages. Defining such tiny
languages does not aim to make the course easy for undergraduate students.
Surprisingly, many kinds of PL research deal with small unused languages.

PL research often aims to prove that a language satisfies a specific property.
The language might be a language used by plenty of people at the moment or an
improved, unimplemented version of an existing language with new features. The
sentence uses the term 'property' in a broad sense: it refers to a property
derived from the definition of the language; it refers to the characteristics of
results obtained by applying a specific algorithm to code written in the
language.

Researchers define small languages because real-world languages are complicated
to be the subjects of research. The real-world languages have many features
helping programmers, such as syntactic sugar. Verifying a property of a language
containing all such features takes a long time. If all the features affected the
property, they sadly would have to deal with a language with all the features.
However, most features are not related to the property, whom the researchers want
to show. It is efficient to work on a small language containing only important
characteristics by identifying features influencing the property.

Besides, it is hard to apply research on a specific existing language to other
languages. If researchers proved a property while reflecting all the features of
the language, they would not be able to conclude that other languages with a
portion of the features satisfy the property. In contrast, if they research a
small language containing features affecting the property, they will be able to
apply the result to such other languages without considerable cost.

Research on Scala is a concrete example. Scala features objects with \term{type
members}---ignore what it is. Popular languages preceding Scala had not featured
them. It had not been sure whether type systems with objects with type members
are \term{type-sound}---ignore what type soundness is. Researchers had defined
DOT (dependent object type), which is a small language with objects with type
members, and proved the type-soundness of DOT. If they had tried to prove the
type-soundness of Scala, they would have spent decades and exerted themselves for
features orthogonal to objects with type members. However, not spending much
time, they had proved the type-soundness of DOT and could apply the result to
languages sharing the feature, such as Wyvern. Alas, even though DOT models the
feature precisely, the type-soundness of DOT does not imply the type-soundness of
Scala. Nonetheless, proving the safety of the core of Scala is crucial for those
who want to trust Scala. As features other than objects with type members of
Scala have been already verified with other researches, verifying only DOT is
quite enough.

In summary, the following is a typical flow of PL research.

1. Want to prove that feature \verb!B! of language \verb!A! satisfies property
\verb!C!.
2. Define small language \verb!a! representing \verb!B!.
3. Define property \verb!c! for \verb!a! as \verb!C! for \verb!A!.
4. Prove that \verb!a! satisfies \verb!c!.
5. \verb!A! probably satisfies \verb!C!, and other languages which feature
\verb!B! may satisfy \verb!C!.

Mind that numerous sorts of PL research do not follow the flow. PL researchers
make, prove, and verify real-world languages, programs, and systems. They invent
tools for practical usages. For instance, Infer of Facebook is a static analyzer
developed by PL researchers. Companies including Facebook and Amazon have been
using Infer.

Such practical research cannot exist without a theoretical background for core
properties and algorithms produced by research dealing with small languages. [The
foundations of
Infer](https://fbinfer.com/docs/separation-logic-and-bi-abduction.html) are
theories suggested by a few papers that are not on real-world languages. Since
the objects of the papers are small but general, Infer can analyze Java, C, C++,
and Objective-C rather than a single language.

The most crucial thing of PL research is to define and solve a small precise
problem expressing a problem of interest. So does the course. The course focuses
on essential features provided by most languages and defines tiny languages
representing the features. The course is a starting point of PL research. At the
same time, the course gives students who are not interested in PL basic knowledge
to understand and to use new languages.

\section{Syntax}

The syntax of a language determines whether code is correct code written in the
language.

\begin{verbatim}
class A
\end{verbatim}

\begin{verbatim}
class A {
\end{verbatim}

The former is code written in Scala, but the latter is not. The syntax of Scala
determines it.

In a mathematical sense, assume that the set of all possible code exists; the set
of all correct code written in language A is a subset of the former set. The
syntax of A defines the subset.

Syntax is either concrete or abstract. Despite the lack of strict definitions of
concrete and abstract syntax, they have distinct properties and are thus easily
distinguished. The course explains them briefly: concrete syntax is for people;
abstract syntax is for computers. The explanation intuitively shows what they
are.

\subsection{Concrete Syntax}

Existing for humans, \term{concrete syntax} deals with code written by people. It
defines a rule for strings and cares about all the characters including
whitespaces and newlines; it specifies rules like "two quotation marks are at the
start and the end of a string," "two consecutive backslashes indicate the start
of a comment," and "every operator is an infix operator." The specifications of
most languages describe the concrete syntax of the languages since programmers
write code according to the specifications.

\term{Backus-Naur form} (BNF) is the most popular way to describe syntax. A form
includes one or more rules. Each rule is in the form of
\verb!<symbol> ::= expression | expression …!. A symbol between angle brackets is a
\term{metavariable}, which denotes a set of strings. An expression is an
enumeration of metavariables and strings. A set denoted by the metavariable
includes strings obtained by substituting metavariables with elements of the
metavariables in one of the expressions. Every string starts and ends with
quotation marks.

The article defines the syntax of AE, a language for arithmetic expressions.

An expression of AE is

\begin{itemize}
\item an integer,
\item the sum of two expressions, or
\item the difference of two expressions.
\end{itemize}

The following is the concrete syntax of AE in the BNF:

\[
\begin{array}{l}
\texttt{digit ::= "0" | "1" | "2" | "3" | "4"} \\
\texttt{\ \ \ \ \ \ \ \ }\texttt{| "5" | "6" | "7" | "8" | "9"} \\
\texttt{nat}\texttt{\ \ \ }\texttt{::= digit | digit nat} \\
\texttt{num}\texttt{\ \ \ }\texttt{::= nat | "-" nat} \\
\texttt{expr}\texttt{\ \ }\texttt{::= num} \\
\texttt{\ \ \ \ \ \ \ \ }\texttt{| "(" expr "+" expr ")"} \\
\texttt{\ \ \ \ \ \ \ \ }\texttt{| "(" expr "-" expr ")"} \\
\end{array}
\]

The remaining part of the section shows how to interpret syntax in the BNF.
\(Digit\) is a set denoted by \(\texttt{digit}\); \(Nat\) is a set denoted by \(\texttt{
nat}\); \(Num\) is a set denoted by \(\texttt{num}\); \(Expr\) is a set denoted by
\(\texttt{expr}\).

\(Digit\) equals , a set of the digits of decimals.

\(Nat\) is the smallest set satisfying the following two conditions; it denotes
the set of every natural number. The \(\cdot\) operator denotes string
concatenation.

\begin{enumerate}
\item \(\forall d\in Digit.d\in Nat\)
\item \(\forall d\in Digit.\forall n\in Nat.d \cdot n\in Nat\)
\end{enumerate}

\(Num\) is the smallest set satisfying the following two conditions; it denotes
the set of every integer.

\begin{enumerate}
\item \(\forall n\in Nat.n\in Num\)
\item \(\forall n\in Nat.\texttt{"-"}\cdot n\in Num\)
\end{enumerate}

\(Expr\) is the smallest set satisfying the following three conditions; it
denotes the set of every arithmetic expression.

\begin{enumerate}
\item \(\forall n\in Num.n\in Expr\)
\item \(\forall e_1\in Expr.\forall e_2\in Expr.{\texttt{"("}}\cdot e_1\cdot{\texttt{"+"}}\cdot
e_2\cdot{\texttt{")"}}\in Expr\)
\item \(\forall e_1\in Expr.\forall e_2\in Expr.{\texttt{"("}}\cdot
e_1\cdot\texttt{"-"}\cdot e_2\cdot{\texttt{")"}}\in Expr\)
\end{enumerate}

\(\texttt{"(1+2)"}\) is an element of \(Expr\), but \(\texttt{"1+2"}\) is not an element
of \(Expr\) due to the lack of parentheses.

\subsection{Abstract Syntax}

Most kinds of PL research define languages with \term{abstract syntax} instead of
concrete syntax, which is unnecessarily precise. As the previous section shows,
concrete syntax cares unimportant details.

Abstract syntax is an abstract data structure describing code. Unlike concrete
syntax, which deals with strings, it deals with abstract objects. Since people
mostly use strings to represent information, strings often describe abstract
syntax. However, the essence of abstract syntax is not about strings. For
example, strings "1" and "one" represent the number one, but the essence of the
number one is that it is the successor of zero but not how people write it on
papers. In the same manner, regardless of a way of describing abstract syntax,
abstract deals with abstract objects but not strings.

The following is the abstract syntax of AE in the BNF:

\[
\begin{array}{rcl}
n & \in & \mathbb{Z} \\
e & ::= & n \\
& | & e+e \\
& | & e-e \\
\end{array}
\]

Metavariable \(n\) ranges over integers; metavariable \(e\) ranges over
expressions.

Like concrete syntax, the abstract syntax in the BNF defines a set. Let
\(\mathcal{A}\) is a set denoted by \(e\). \(\mathcal{A}\) is the smallest set
satisfying the following three conditions.

\begin{enumerate}
\item \(\forall n\in\mathbb{Z}.n\in \mathcal{A}\)
\item \(\forall e_1\in\mathcal{A}.\forall e_2\in\mathcal{A}.e_1+e_2\in\mathcal{A}\)
\item \(\forall e_1\in\mathcal{A}.\forall e_2\in\mathcal{A}.e_1-e_2\in\mathcal{A}\)
\end{enumerate}

\term{Inference rules} can define abstract syntax as well. Inference rules
typically define the semantics of languages, but the article defines abstract
syntax with inference rules to make readers familiar with inference rules. It is
possible to define concrete syntax with inference rules, but I think that it is
redundant and unnecessary.

Inference rules derive a \term{proposition} from propositions. An inference rule
is composed of a horizontal line, zero or more propositions above the line, and a
proposition below the line. If no proposition exists above the line, then the
line can be omitted. The propositions above the line are premises; the
proposition below the line is a conclusion. Every proposition in the rule may
have metavariables.

For instance, an inference rule can encode \term{modus ponens}, which implies
that for any propositions \(P\) and \(Q\), if \(P\rightarrow Q\) and \(P\), then
\(Q\). Let metavariables \(p\) and \(q\) range over propositions.

\[
\inferrule
{ p\rightarrow q \\ p }
{ q }
\]

If substituting every metavariable with an element of the metavariable in a rule
makes every premise of the rule true, then the conclusion of the rule also is
true. Assume that \(P\) and \(Q\) are propositions, and both \(Q\rightarrow P\)
and \(Q\) are true. Substituting \(p\) and \(q\) with \(Q\) and \(P\) results in
two true premises and conclusion \(P\). The following \term{proof tree} is a
proof of \(P\):

\[
\inferrule
{ Q\rightarrow P \\ Q }
{ P }
\]

One can use inference rules multiple times to prove a proposition. Assume that
\(P\), \(Q\), and \(R\) are propositions, and \(P\rightarrow(Q\rightarrow R)\),
\(P\), and \(Q\) are true. Substituting \(p\) and \(q\) with \(P\) and
\(Q\rightarrow R\) yields that \(Q\rightarrow R\) is true. Substituting \(p\) and
\(q\) with \(Q\) and \(R\) finally proves \(R\). The following proof tree gives a
proof:

\[
\inferrule
{
{\inferrule
  { P\rightarrow(Q\rightarrow R) \\ P }
  { Q\rightarrow R } }\\
  Q }
{ R }
\]

The following inference rules define the abstract syntax of AE:

\[
\inferrule
{ n\in\mathbb{Z} }
{ n\in\mathcal{A} }
\\
\inferrule
{ e_1\in\mathcal{A} \\ e_2\in\mathcal{A} }
{ e_1+e_2\in\mathcal{A} }
\\
\inferrule
{ e_1\in\mathcal{A} \\ e_2\in\mathcal{A} }
{ e_1-e_2\in\mathcal{A} }
\]

The following proof tree proves that \(4+(2-1)\) is an element of
\(\mathcal{A}\).
Note that we can use parentheses to resolve ambiguity in abstract syntax since
it defines mathematical notation.

\[
\inferrule
{
\inferrule
  { 4\in\mathbb{Z} }
  { 4\in\mathcal{A} } \\
  \inferrule
  { \inferrule
    { 2\in\mathbb{Z} }
    { 2\in\mathcal{A} } \\
    \inferrule
    { 1\in\mathbb{Z} }
    { 1\in\mathcal{A} }
  }
  { (2-1)\in\mathcal{A} }
}
{ 4+(2-1)\in\mathcal{A} }
\]

Scala code also can represent the abstract syntax of AE. It is a typical ADT; a
sealed trait and case classes define it:

\begin{verbatim}
sealed trait Expr
case class Num(n: Int) extends Expr
case class Add(l: Expr, r: Expr) extends Expr
case class Sub(l: Expr, r: Expr) extends Expr
\end{verbatim}

The following Scala code represents \(4+(2-1)\):

\begin{verbatim}
Add(Num(4), Sub(Num(2), Num(1)))
\end{verbatim}

Most sorts of abstract syntax define tree shapes. Trees following abstract syntax
are \term{abstract syntax trees} (ASTs). The below tree visualizes \(4+(2-1)\).
The structure of an object defined by the above Scala code equals the tree.

\subsection{Parsing}

\term{Parsing} is a process that transforms strings following concrete syntax
into ASTs and rejects strings not following the concrete syntax. A \term{parser}
is a parsing program. Parsing is out of the scope of the course and thus is out
of the scope of the article.

The Scala standard library provides \term{parser combinators}. Programmers can
implement parsers without detailed knowledge about parsing. The below code
implements a parser of AE. The parser takes a string as input and produces an AST
of AE; it throws an exception if the string does not follow the concrete syntax
of AE. Note that strings may contain whitespaces freely, while concrete syntax
defined by the article is tight with whitespaces.

\begin{verbatim}
import scala.util.parsing.combinator._

object Expr extends RegexParsers {
  def wrap[T](e: Parser[T]): Parser[T] = "(" ~> e <~ ")"
  lazy val n: Parser[Int] = "-?\\d+".r ^^ (_.toInt)
  lazy val e: Parser[Expr] =
    n                    ^^ Num                         |
    wrap((e <~ "+") ~ e) ^^ { case l ~ r => Add(l, r) } |
    wrap((e <~ "-") ~ e) ^^ { case l ~ r => Sub(l, r) }

  def parse(s: String): Expr =
    parseAll(e, s).getOrElse(throw new Exception)
}

Expr.parse("1")
// Num(1)

Expr.parse("(4 + (2 - 1))")
// Add(Num(4),Sub(Num(2),Num(1)))

Expr.parse("1 + 2")
// java.lang.Exception
\end{verbatim}

\newpage
\section{Exercises}

\begin{enumerate}
\item Given the following grammar:
%
\begin{verbatim}
    <WAE> ::= <num>
            | {+ <WAE> <WAE>}
            | {* <WAE> <WAE>}
            | {let {<id> <WAE>} <WAE>}
            | <id>
\end{verbatim}
%
Describe whether each of the following is \verb+<WAE>+ and why:

\begin{itemize}
\item[a)]
\begin{verbatim}
{let {x 5} {+ 8 {* x 2 3}}}
\end{verbatim}

\item[b)]
\begin{verbatim}
{with {x 0} {with {x 7}}}
\end{verbatim}

\item[c)]
\begin{verbatim}
{let {3 5} {+ 8 {- x 2}}}
\end{verbatim}

\item[d)]
\begin{verbatim}
{let {3 y} {+ 8 {* x 2}}}
\end{verbatim}

\item[e)]
\begin{verbatim}
{let {x y} {+ 8 {* x 2}}}
\end{verbatim}
\end{itemize}

\item Given the following grammar:
%
\newcommand{\BNF}[1]{$\langle$#1$\rangle$}
\newcommand{\coffee}{\mbox{\BNF{coffee}}}
\newcommand{\milk}{\mbox{\BNF{milk}}}
\newcommand{\flavor}{\mbox{\BNF{flavor}}}

\[
\begin{array}{ccc}
{\texttt{espresso} \in \coffee}
&&
\newinfrule
{e_1 \in \milk\qquad
e_2 \in \coffee
}
{\ e_1\ \texttt{on}\ e_2 \in \coffee}
\\[2em]
\newinfrule
{e_1 \in \coffee\qquad
e_2 \in \milk
}
{\ e_1\ \texttt{on}\ e_2 \in \coffee}
&&
\newinfrule
{e_1 \in \flavor \qquad
e_2 \in \coffee
}
{\ e_1\ \texttt{on}\ e_2 \in \coffee}
\\[2em]
{\texttt{milk-foam} \in \milk}
&&
{\texttt{steamed-milk} \in \milk}
\\[2em]
{\texttt{caramel} \in \flavor}
&&
{\texttt{cinnamon} \in \flavor}
\\[2em]
{\texttt{cocoa-powder} \in \flavor}
&&
{\texttt{chocolate-syrup} \in \flavor}
\end{array}
\]
where \texttt{on} is right-associative.

\begin{itemize}
\item[a)] Which of the following are examples of \BNF{coffee}?
%
\begin{enumerate}

  \item[1)] \texttt{caramel latte macchiato} % No

  \item[2)] \texttt{espresso} % Yes

  \item[3)] \texttt{steamed-milk on caramel on milk-foam on espresso} % Yes

  \item[4)] \texttt{chocolate-syrup on cocoa-powder on cinnamon on milk-foam on steamed-milk on espresso} % Yes

  \item[5)] \texttt{steamed-milk on espresso on chocolate-syrup} % No

\end{enumerate}

\item[b)] Draw a proof of why the following is or is not \BNF{coffee}:

\begin{center}
\texttt{cocoa-powder on milk-foam on steamed-milk on espresso}
\end{center}
\end{itemize}

\item Given the following grammar:
%
\begin{center}
\begin{tabular}{lll}
 \BNF{ice-cream} & $::=$ & \texttt{sprinkles on \BNF{ice-cream}} \\
  & $|$ & \texttt{cherry on \BNF{ice-cream}} \\
  & $|$ & \texttt{scoop of \BNF{flavor} on \BNF{ice-cream}} \\
  & $|$ & \texttt{sugar-cone} \\
  & $|$ & \texttt{waffle-cone} \\
 \BNF{flavor} & $::=$ & \texttt{vanilla} \\
  & $|$ & \texttt{lettuce}
\end{tabular}
\end{center}
%
\begin{itemize}
  \item[a)] Which of the following are examples of \BNF{ice-cream}?
%
\begin{itemize}
  \item[1)] \texttt{sprinkles}
  \item[2)] \texttt{sugar-cone}
  \item[3)] \texttt{vanilla}
  \item[4)] \texttt{scoop of vanilla on waffle-cone}
  \item[5)] \texttt{sprinkles on lettuce on waffle-cone}
  \item[6)] \texttt{scoop of vanilla on sprinkles on waffle-cone}
\end{itemize}

  \item[b)] Explain why the following is an \BNF{ice-cream}:

\begin{center}
\texttt{cherry on scoop of lettuce on scoop of vanilla on sugar-cone}
\end{center}
\end{itemize}

\end{enumerate}

\setchapterpreamble[u]{\margintoc}
\chapter{Semantics}
\labch{semantics}

Syntax and semantics define a programming language. Syntax determines whether
code is code written in the language. Semantics decides what the code denotes.
Without semantics, code written by programmers is no more than a string.

The shape of semantics depends on the property of interest. If the property is
about how programs modify memories of computers, defining semantics is
defining what a memory is and how code modifies a memory. If the property is
about input and output of programs, defining semantics is defining what input
and output are and what code prints for input. The course focuses on
functional languages, and the semantics of a functional language determines a
value obtained by interpreting an expression.

\section{Defining Semantics}

There are various styles of semantics: denotational semantics and operational
semantics are famous; axiomatic semantics and others exist. Denotational
semantics define values denoted by programs with mathematical methods. For
example, denotational semantics of an imperative language views a program as a
function from states to states. On the other hand, operational semantics
expresses executions of programs with logical statements, such as inference
rules. Operational semantics is more similar to the implementation of an
interpreter than denotational semantics but does differ from an implementation.

There are multiple forms of operational semantics: natural semantics,
structural operational semantics (SOS), reduction semantics, abstract machine
semantics, and others. The course mainly deals with natural semantics, as
known as big-step semantics. Natural semantics is composed of one or more
inference rules. A rule defines a value denoted by an expression. In contrast,
small-step semantics, including SOS and reduction semantics, uses inference
rules that transform an expression into an expression instead of a value. Big
step semantics produces a value at one big step, while small-step semantics
requires multiple small steps to attain a value.

Each kind of semantics has its characteristic. Different types of research
need different types of semantics. For example, defining both concrete and
abstract syntax in a denotational style allows expressing a relationship
between them mathematically and showing the soundness of abstract
interpretations. Axiomatic semantics fits verifying the correctness of program,
and proving the type-soundness of languages is a typical usage of reduction
semantics. Natural semantics intuitively defines languages and is closest to
the implementation of an interpreter.

The above explanation contains words not introduced by the course. It is
enough to understand that various ways to define semantics exist and choosing
a proper style for a subject is crucial.

\section{Natural Semantics}

The last article defined the abstract syntax of AE.

\[
\begin{array}{lrcl}
\text{Integer} & n & \in & \mathbb{Z} \\
\text{Expression} & e & ::= & n \\
&& | & e+e \\
&& | & e-e \\
\end{array}
\]

Metavariable \(n\) ranges over integers; metavariable \(e\) ranges over
expressions; \(\text{Expression}\) is the set of every expression.

This article defines the natural semantics of AE.

The semantics of AE decides values denoted by expressions of AE. The first
thing to do is defining what values are. Every arithmetic expression denotes
an integer so that every value of AE is an integer.

\[
\begin{array}{lrcl}
\text{Value} & v & ::= & n
\end{array}
\]

Metavariable \(v\) ranges over values; \(\text{Value}\) is the set of every
value; it equals \(\mathbb{Z}\).

Every expression of AE denotes a value of AE. The semantics of AE seems to be
a function from expressions to values. Let \(\Rightarrow\) be the function.

\[\Rightarrow:\ \text{Expression}\rightarrow\text{Value}\]

However, in general, not every expression of a language denotes a value. As an
execution might terminate without yielding a result due to an error,
expressions not denoting any values exist. Besides, if some expressions
produce random values, a single expression may denote multiple values.
Therefore, it is desirable to define semantics as a binary relation over
\(\text{Expression}\) and \(\text{Value}\).

\[\Rightarrow\subseteq\text{Expression}\times\text{Value}\]

For any expression \(e\) and any value \(v\), \((e,v)\in\Rightarrow\) implies
that \(e\) denotes \(v\), or \(v\) is the result of evaluating \(e\). In PL
research, notation \(\vdash e\Rightarrow v\) replaces \((e,v)\in\Rightarrow\).
\(\Rightarrow\) is a relation and thus does not imply input and output, but,
intuitively, expressions are input and values are output.

Inference rules define the semantics of AE.

\[
\vdash n\Rightarrow n
\]

If an expression is an integer, the expression denotes the integer. The rule
does not have any premises. It has the following mathematical meaning:

\[ \forall n\in\mathbb{Z}.\vdash n\Rightarrow n \]

Intuitively, \(n\) does not require any computation, and the result is \(n\).

\[
\inferrule
{ \vdash e_1\Rightarrow n_1\\\vdash e_2\Rightarrow n_2 }
{ \vdash e_1+e_2\Rightarrow n_1+n_2 }
\]

If an expression is the sum of two expressions, the expression denotes the sum
of two integers denoted by the two expressions. The rule has the following
mathematical meaning:

\[
\begin{array}{l}
\forall e_1\in\text{Expression}.
\forall e_2\in\text{Expression}.
\forall n_1\in\mathbb{Z}.
\forall n_2\in\mathbb{Z}.\\
(\vdash e_1\Rightarrow n_1)\rightarrow
(\vdash e_2\Rightarrow n_2)\rightarrow
(\vdash e_1+e_2\Rightarrow n_1+n_2)
\end{array}
\]

Intuitively, computing \(e_1+e_2\) requires computing \(e_1\) and \(e_2\), and
since \(e_1\) and \(e_2\) respectively result in \(n_1\) and \(n_2\), the
result is \(n_1+n_2\). \(e_1\) and \(e_2\) are given; intermediate computation
yields \(n_1\) and \(n_2\); the final result is \(n_1+n_2\).
Note that
in \(e_1+e_2\), \(+\) is a symbol used to represent abstract syntax, while \(+\)
in \(n_1+n_2\) denotes mathematical addtion as usual.

\[
\inferrule
{ \vdash e_1\Rightarrow n_1\\\vdash e_2\Rightarrow n_2 }
{ \vdash e_1-e_2\Rightarrow n_1-n_2 }
\]

If an expression is the difference of two expressions, the expression denotes
the difference of two integers denoted by the two expressions. The rule has
the following mathematical meaning:

\[
\begin{array}{l}
\forall e_1\in\text{Expression}.
\forall e_2\in\text{Expression}.
\forall n_1\in\mathbb{Z}.
\forall n_2\in\mathbb{Z}.\\
(\vdash e_1\Rightarrow n_1)\rightarrow
(\vdash e_2\Rightarrow n_2)\rightarrow
(\vdash e_1-e_2\Rightarrow n_1-n_2)
\end{array}
\]

Intuitively, computing \(e_1-e_2\) requires computing \(e_1\) and \(e_2\), and
since \(e_1\) and \(e_2\) respectively result in \(n_1\) and \(n_2\), the
result is \(n_1-n_2\). \(e_1\) and \(e_2\) are given; intermediate computation
yields \(n_1\) and \(n_2\); the final result is \(n_1-n_2\).
Like \(+\), \(-\) represents both expressions in abstract syntax and
mathematical subtraction.

The article keeps emphasizing that both understanding mathematical definitions
and interpreting the semantics intuitively are essential. In a mathematical
sense, the natural semantics of AE is a relation over \(\text{Expression}\)
and \(\text{Value}\), and the inference rules do not care what given things
are and what obtained things are. In contrast, intuitively, the natural
semantics find a value denoted by a given expression. An expression inside the
conclusion of a rule is input; the premises of a rule represent required
computation; a value inside the conclusion is output. Not considering
mathematical definitions, one may make a mistake while strictly thinking and
hardly understand complicated semantics. Complex semantics needs rules not
interpreted with the concepts of input, computation, and output; for instance,
computation uses output. It intuitively seems odd but is natural in a
mathematical sense, which does not have such concepts. On the other hand, not
interpreting semantics intuitively, one hardly understands a language. In
conclusion, both viewpoints are crucial.

The following rules are all of the natural semantics of AE:

\[
\vdash n\Rightarrow n
\]

\[
\inferrule
{ \vdash e_1\Rightarrow n_1\\\vdash e_2\Rightarrow n_2 }
{ \vdash e_1+e_2\Rightarrow n_1+n_2 }
\]

\[
\inferrule
{ \vdash e_1\Rightarrow n_1\\\vdash e_2\Rightarrow n_2 }
{ \vdash e_1-e_2\Rightarrow n_1-n_2 }
\]

\subsection{Drawing Proof Trees}

The following proof tree proves that \(4+(2-1)\) denotes \(5\):

\[
\inferrule
{
  \vdash 4\Rightarrow 4 \\
  \inferrule
  {\vdash 2\Rightarrow 2 \\ \vdash 1\Rightarrow 1}
  {\vdash 2-1\Rightarrow 1}
}
{\vdash4+(2-1)\Rightarrow 5}
\]

Drawing proof trees is not an interesting research topic. However, it is a
good practice to understand semantics, and some students feel difficult about
it. The article thus briefly introduces a strategy to draw proof trees.

Languages defined by the course have simple semantics. Usually, only a single
inference rule fits a given expression. The meanings of propositions are
unimportant, and substituting metavariables with appropriate expressions is
enough to draw proof trees. Drawing proof trees is often mechanical.

The remaining part of the section draws a proof tree proving that \(4+(2-1)\)
denotes \(5\) step by step. Firstly, an expression inside a conclusion is \(4+(2-1)\).

\[
\color{red}{
\inferrule
{
  \color{white}{
  \vdash 4\Rightarrow 4} \\
  \color{white}{
  \inferrule
  {\vdash 2\Rightarrow 2 \\ \vdash 1\Rightarrow 1}
  {\vdash 2-1\Rightarrow 1}
  }
}
{\vdash4+(2-1)\Rightarrow \color{white}{5}}
}
\]

A single rule fits \(4+(2-1)\). Substitute \(e_1\) and \(e_2\) with \(4\) and
\(2-1\) respectively to make premises. Do not write values of the premises.

\[
\inferrule
{
  \color{red}{\vdash 4\Rightarrow {\color{white}4}} \\
  \color{red}{
  \inferrule
  {\color{white}{\vdash 2\Rightarrow 2 \\ \vdash 1\Rightarrow 1}}
  {\vdash 2-1\Rightarrow \color{white}{1}}
  }
}
{\vdash4+(2-1)\Rightarrow \color{white}{5}}
\]

A single rule fits \(4\). The rule has no premises. Substitute \(n\) with \(
\) and write the value.

\[
\inferrule
{
  \vdash 4\Rightarrow \color{red}{4} \\
  \inferrule
  {\color{white}{\vdash 2\Rightarrow 2 \\ \vdash 1\Rightarrow 1}}
  {\vdash 2-1\Rightarrow \color{white}{1}}
}
{\vdash4+(2-1)\Rightarrow \color{white}{5}}
\]

A single rule fits \(2-1\). Substitute \(e_1\) and \(e_2\) with \(2\) and \(
\) respectively to make premises. Do not write values of the premises.

\[
\inferrule
{
  \vdash 4\Rightarrow 4 \\
  \inferrule
  {\color{red}{\vdash 2\Rightarrow \color{white}{2} \\ \vdash 1\Rightarrow \color{white}{1}}}
  {\vdash 2-1\Rightarrow \color{white}{1}}
}
{\vdash4+(2-1)\Rightarrow \color{white}{5}}
\]

A single rule fits \(2\). The rule has no premises. Substitute \(n\) with \(2\) and write the value.

\[
\inferrule
{
  \vdash 4\Rightarrow 4 \\
  \inferrule
  {\vdash 2\Rightarrow \color{red}{2} \\ \vdash 1\Rightarrow \color{white}{1}}
  {\vdash 2-1\Rightarrow \color{white}{1}}
}
{\vdash4+(2-1)\Rightarrow \color{white}{5}}
\]

A single rule fits \(1\). The rule has no premises. Substitute \(n\) with \(1\) and write the value.

\[
\inferrule
{
  \vdash 4\Rightarrow 4 \\
  \inferrule
  {\vdash 2\Rightarrow 2 \\ \vdash 1\Rightarrow \color{red}{1}}
  {\vdash 2-1\Rightarrow \color{white}{1}}
}
{\vdash4+(2-1)\Rightarrow \color{white}{5}}
\]

Compute \(2-1\) and write \(1\), the result of \(2-1\).

\[
\inferrule
{
  \vdash 4\Rightarrow 4 \\
  \inferrule
  {\vdash 2\Rightarrow 2 \\ \vdash 1\Rightarrow 1}
  {\vdash 2-1\Rightarrow \color{red}{1}}
}
{\vdash4+(2-1)\Rightarrow \color{white}{5}}
\]

Compute \(4+1\) and write \(5\), the result of \(4+(2-1)\).

\[
\inferrule
{
  \vdash 4\Rightarrow 4 \\
  \inferrule
  {\vdash 2\Rightarrow 2 \\ \vdash 1\Rightarrow 1}
  {\vdash 2-1\Rightarrow 1}
}
{\vdash4+(2-1)\Rightarrow \color{red}{5}}
\]

The tree is complete.

\subsection{Implementing an Interpreter}

The following Scala code is the implementation of an interpreter following the
natural semantics of AE:

\begin{verbatim}
sealed trait Expr
case class Num(n: Int) extends Expr
case class Add(l: Expr, r: Expr) extends Expr
case class Sub(l: Expr, r: Expr) extends Expr

def interp(e: Expr): Int = e match {
  case Num(n) => n
  case Add(l, r) => interp(l) + interp(r)
  case Sub(l, r) => interp(l) - interp(r)
}

interp(Add(Num(4), Sub(Num(2), Num(1))))  // 5
\end{verbatim}


\chapter{Identifiers}
\labch{identifiers}

\renewcommand{\plang}{\textsf{AE}\xspace}
\renewcommand{\Lang}{\textsf{VAE}\xspace}

Variables are one of the basic concepts of programming languages. A
\textit{variable}\index{variable} relates a name to a value. We use the
value of a variable by writing the name of the variable. For example, the following Scala program
prints \code{3}.

\begin{verbatim}
val x = 3
println(x)
\end{verbatim}

The program defines a variable whose name is \code{x} and value is \code{3}. At
the second line, the name \code{x} denotes the value \code{3}.

We call the names of variables identifiers. An
\textit{identifier}\index{identifier} is a name related to a
certain entity in a program. Not only the names of variables are identifiers;
there are various kinds of identifiers:

\begin{itemize}
\item Function names, which are related to functions
\item Parameter names, which are related to the values of arguments
\item Field names, which are related to values of fields
\item Method names, which are related to methods
\item Class names, which are related to classes
\end{itemize}

This chapter introduces identifiers. Identifiers in programs can split into
three groups: binding occurrences, bound occurrences, and free identifiers.
We will see what they are. This chapter discusses identifiers based on the use
of variables in programs. We will define \Lang by extending \plang of
\refch{syntax-and-semantics} with variables. Variables of \Lang are immutable.
We will deal with mutable variables in \refch{mutable-variables}.
In \Lang, the names of variables are
the only identifiers. However, as you have seen already, real-world programming
languages have many kinds of identifiers.

\section{Identifiers}

Identifiers name entities like variables and functions.
Let us discuss notions related to identifiers with the following Scala program:

\begin{verbatim}
f(0)
def f(x: Int): Int = {
  val y = 2
  x + y
}
f(1)
x - z
\end{verbatim}

In this program, \code{f}, \code{x}, \code{y}, and \code{z} are identifiers. Strictly speaking,
\code{Int} also is an identifier, but we ignore it because we do not want to take
types into account here.

A single identifier can occur multiple times in a program. For instance,
\code{f} occurs three times in the program: line 1, line 2, and line 6.
We can classify occurrences of identifiers into three categories:
binding occurrences, bound occurrences, and free identifiers.

An occurrence of an identifier is called a \textit{binding occurrence}\index{binding occurrence}
if the identifier occurs to be defined. A binding occurrence relates the
identifier to a particular entity. The program has three binding occurrences:

\begin{itemize}
  \item \code{f} at line 2

    It relates \code{f} to a function.

  \item \code{x} at line 2

    It relates \code{x} to the value of an argument given to \code{f}.

  \item \code{y} at line 3

    It relates \code{y} to the value \code{2}.
\end{itemize}

Every binding occurrence has its own scope. The \textit{scope}\index{scope} of a binding
occurrence means a code region where the identifier defined by the binding
occurrence is alive, i.e. usable. The scope of each identifier in the program is as follows:

\begin{itemize}
  \item \code{f}

    A function can be used in its body (as Scala allows recursive function
    definitions) and at the lines below its definition. The scope of
    \code{f} is from line 3 to line 7.

  \item \code{x}

    A parameter of a function can be used only in the function body. The scope of
    \code{x} is line 3 and line 4.

  \item \code{y}

    A variable can be used at the lines below its definition. The scope of
    \code{y} is line 4.
\end{itemize}

An occurrence of an identifier is called a \textit{bound occurrence}\index{bound occurrence}
if the identifier occurs to use the entity related to itself. Since an
identifier becomes related to an entity by its binding occurrence, any bound
occurrences must reside in the scope of the binding occurrence.
The program has three bound occurrences:

\begin{itemize}
  \item \code{f} at line 6

    It denotes the function defined at line 2.

  \item \code{x} at line 4

    It denotes the value of an argument passed to \code{f}.

  \item \code{y} at line 4

    It denotes the value \code{2}.
\end{itemize}

An occurrence of an identifier is called a \textit{free identifier}\index{free
identifier} if it is
neither binding nor bound. A free identifier neither introduces a new name nor
uses a name defined already. It is not in the scope of any binding occurrence of
the same identifier. The program has three free identifiers:

\begin{itemize}
  \item \code{f} at line 1

    It is outside the scope of \code{f}.

  \item \code{x} at line 7

    It is outside the scope of \code{x}.

  \item \code{z} at line 7

    The program never defines \code{z}.
\end{itemize}

We call a free identifier a \textit{free variable}\index{free variable} when it
is the name of a variable. Therefore, both \code{x} and \code{z} at line 7 are
free variables.

Now, consider a binding occurrence that resides in the scope of a binding
occurrence of the same identifier. For example, the following program has two
binding occurrences of \code{x}, and the second binding occurrence is in the
scope of the first binding occurrence.

\begin{verbatim}
def f(x: Int): Int = {
  def g(x: Int): Int =
    x
  g(x)
}
\end{verbatim}

In this case, shadowing happens. \textit{Shadowing}\index{shadowing} means that
the innermost binding occurrence \textbf{shadow}s, i.e. temporarily invalidates,
the outer binding occurrences of the same name. Therefore, \code{x} at line 2
shadows \code{x} at line 1.
\code{x} at line 3 belongs to the scope of both binding occurrences simultaneously.
It denotes the value of an argument given to \code{g}, not \code{f}, because of
shadowing. On the other hand, \code{x} at line 4 denotes the value of an
argument given to \code{f} since it belongs to the scope of only \code{x} at
line 1.

\section{Syntax}

Let us define the abstract syntax of \Lang. We do not consider concrete
syntax anymore. Therefore, the term syntax will be used to mean abstract
syntax. Also, from now on, we use the term
\textit{expressions}\index{expression} rather than
programs when we discuss languages like \Lang. For example, we say that
$1+2$ is an expression of \plang, and $1$ and $2$ are the subexpressions of
$1+2$.

Recall the example at the beginning of the chapter:

\begin{verbatim}
val x = 3
println(x)
\end{verbatim}

To add variables to \plang, we need two kinds of expressions. The first kind is
expressions defining a variable, i.e. binding an identifier. In the example,
\code{val x = 3; println(x)} is such an expression. It defines the
variable \code{x} and starts the scope of \code{x} so that \code{x} can be used
in \code{println(x)}. We can conclude that an expression defining a variable
consists of three parts: the name of the variable, an expression determining the
value of the variable, and an expression that can use the variable. These parts
are \code{x}, \code{3}, and \code{println(x)}, respectively, in the example. The
second kind is expressions using a variable, i.e. a bound occurrence. In the
example, \code{x} at the second line is such an expression. It uses the variable \code{x} to denote
the value \code{3}. Based on this observation, we can define the syntax of
\Lang.

First, we need to add a new syntactic element: identifiers. The metavariable
$x$ ranges over identifiers. Let $\embox{Id}$ be the set of every
identifier.

\[x\in\embox{Id}\]

We do not care what $\embox{Id}$ really is.

The syntax of \Lang is as follows:\footnote{We omit the common part
to \plang.}

\[e\ ::=\ \cdots\ |\ \ebind{x}{e}{e}\ |\ x\]

\begin{itemize}
  \item $\ebind{x}{e_1}{e_2}$

    It defines a new variable whose name is $x$. Therefore, the occurrence of $x$ is a
    binding occurrence. $e_1$ decides the value denoted by the variable. The
    scope of the variable includes $e_2$ but excludes $e_1$.

  \item $x$

    It uses a variable; it is either a bound occurrence of $x$ or a free identifier.
    If it belongs to the scope of a binding occurrence of the same name, then it is a
    bound occurrence and denotes the value associated with the identifier.
    Otherwise, it is a free identifier, which denotes nothing.
\end{itemize}

\section{Semantics}

To define the semantics of \Lang, we need an additional semantic element that
stores the values denoted by variables. Without such an element, we cannot know the
value of each variable. We call the element an
\textit{environment}\index{environment}. An environment is a finite partial
function.\footnote{A finite partial function is a partial function whose domain
is a finite set.} The metavariable $\sigma$ ranges over environments.

\[\embox{Env}=\embox{Id}\finto\mathbb{Z}\]
\[\sigma\in\embox{Env}\]

For example, consider an environment $\sigma$.
If $\sigma(\cx)=1$, the value of a variable named \code{x} is $1$.
An environment is a partial function because it does not have the values
related to free identifiers. If a variable named \code{y} is free in
$\sigma$, then $\sigma(\cy)$ is undefined.
In addition, it is finite since every program
defines only finitely many identifiers.

Every expression in \Lang can evaluate to an integer only under some
environment. The reason is obvious: without environments, there is no way to
find the values of variables, and thus environments are essential to evaluation.

The following rule defines the semantics of $x$:

\semanticrule{Id}{
If
  $x$ is in the domain of $\sigma$,\\
then
  \evaldn{x}{\sigma(x)}.
}

If $x$ is an element of the domain of $\sigma$, $x$ is a bound occurrence. The
environment gives us the value denoted by $x$, which is $\sigma(x)$. Then, the
result is $\sigma(x)$. Otherwise, $x$ is not in the domain and is a free
identifier. In that case, we cannot evaluate $x$. The evaluation terminates
immediately. It can be interpreted as a run-time error.

Formally, the semantics of \Lang is a
ternary relation over $\embox{Env}$, $E$, and $\mathbb{Z}$ since it must take
environments into account.

\[\Rightarrow\subseteq\embox{Env}\times E\times\mathbb{Z}\]

$(\sigma,e,n)\in\Rightarrow$ is true if and only if
$e$ evaluates to $n$ under $\sigma$.
We write $\sigma\vdash e\Rightarrow n$ instead of $(\sigma,e,n)\in\Rightarrow$.
Intuitively, $\sigma$ and $e$ are inputs, and $n$ is the corresponding output.

Rule \textsc{Id} can be formulated as the following inference rule:
\footnote{$\dom{\sigma}$ denotes the domain of $\sigma$.}

\[
  \inferrule
  { x\in\dom{\sigma} }
  { \evald{x}{\sigma(x)} }
  \quad\textsc{[Id]}
\]

When a variable is defined, the value of the variable is added to the environment.
We write $\sigma[x\mapsto n]$ to denote an environment obtained by adding the
fact that $x$ denotes $n$ to $\sigma$. Then, the following property holds:

\[
\sigma [x\mapsto n](x') =
\begin{cases}
  n & \text{if}\ x=x' \\
  \sigma(x') & \text{if}\ x\neq x'
\end{cases}
\]

The following rule defines the semantics of $\ebind{x}{e_1}{e_2}$:

\semanticrule{Val}{
If
  \evaldn{e_1}{n_1}, and
  \evaln{\sigma[x\mapsto n_1]}{e_2}{n_2},\\
then
  \evaldn{\ebind{x}{e_1}{e_2}}{n_2}.
}

To evaluate $\ebind{x}{e_1}{e_2}$, we need to determine the value of $x$ first.
It can be done by evaluating $e_1$. Since the scope of $x$ excludes $e_1$,
the evaluation proceeds under $\sigma$, which is a given environment.
The result of $e_1$ is the value of $x$, and this information must be added to
the environment. By adding the fact to $\sigma$, we get $\sigma[x\mapsto n_1]$.
As $e_2$ is the scope of $x$, $e_2$ is evaluated under $\sigma[x\mapsto n_1]$.
The result of $e_2$ is the final result.

This semantics naturally explains shadowing. Let $x$ already be in the domain of
$\sigma$. Suppose that $\sigma(x)=n$. However, $e_2$ is evaluated under
$\sigma[x\mapsto n_1]$, and $\sigma[x\mapsto n_1](x)=n_1$. When $x$ is used in
$e_2$, its value is $n_1$, not $n$. Therefore, we can say that the
innermost definition of $x$ is used for the evaluation of $e_2$. This behavior
exactly matches the concept of shadowing explained before.

Rule \textsc{Val} can be expressed as the following inference rule:

\[
\inferrule
{
  \evald{e_1}{n_1} \\
  \eval{\sigma[x\mapsto n_1]}{e_2}{n_2}
}
{ \evald{\ebind{x}{e_1}{e_2}}{n_2} }
\quad\textsc{[Val]}
\]

The remaining cases are $n$, $\eadd{e_1}{e_2}$, and $\esub{e_1}{e_2}$.
Rules for those cases are basically identical to the rules of \plang.
However, we need to additionally take environments into account.

\semanticrule{Num}{
  \evaldn{n}{n}.
}

\vspace{-1em}

\semanticrule{Add}{
If
  \evaldn{e_1}{n_1}, and \evaldn{e_2}{n_2},\\
then
  \evaldn{\eadd{e_1}{e_2}}{n_1+n_2}.
}

\vspace{-1em}

\semanticrule{Sub}{
If
  \evaldn{e_1}{n_1}, and \evaldn{e_2}{n_2},\\
then
  \evaldn{\esub{e_1}{e_2}}{n_1-n_2}.
}

Integers, addition, and subtraction never update environments.
An integer evaluates to itself. Addition and subtraction evaluates their
subexpressions under the same environment.

We can express the rules in a natural language as the following inference rules:

\[
  \evald{n}{n}\quad\textsc{[Num]}
\]

\[
  \inferrule
  { \evald{e_1}{n_1} \\ \evald{e_2}{n_2} }
  { \evald{\eadd{e_1}{e_2}}{n_1+n_2} }
  \quad\textsc{[Add]}
\]

\[
  \inferrule
  { \evald{e_1}{n_1} \\ \evald{e_2}{n_2} }
  { \evald{\esub{e_1}{e_2}}{n_1-n_2} }
  \quad\textsc{[Sub]}
\]

The following proof tree proves that $\ebind{\cx}{1}{\eadd{\cx}{\cx}}$ evaluates
to $2$ under the empty environment. Note that $[x_1\mapsto n_1,\cdots,x_m\mapsto
n_m]$ denotes an environment whose domain includes from $x_1$ to $x_m$ and each
$x_i$ is mapped to $n_i$.

\[
\inferrule
{
  \emptyset\vdash 1\Rightarrow 1 \\
  \inferrule
  {
    \inferrule
    { \cx\in\dom{[ \cx\mapsto 1]} }
    { [ \cx\mapsto 1]\vdash \cx\Rightarrow 1 } \\
    \inferrule
    { \cx\in\dom{[ \cx\mapsto 1]} }
    { [ \cx\mapsto 1]\vdash \cx\Rightarrow 1 }
  }
  {[ \cx\mapsto 1]\vdash \eadd{\cx}{\cx}\Rightarrow 2 }
}
{ \emptyset\vdash \ebind{\cx}{1}{\eadd{\cx}{\cx}}\Rightarrow 2 }
\]

\section{Interpreter}

The following Scala code implements the abstract syntax of \Lang:

\begin{verbatim}
sealed trait Expr
case class Num(n: Int) extends Expr
case class Add(l: Expr, r: Expr) extends Expr
case class Sub(l: Expr, r: Expr) extends Expr
case class Val(x: String, i: Expr, b: Expr) extends Expr
case class Id(x: String) extends Expr
\end{verbatim}

An identifier is an arbitrary string.
\code{Val($x$, $e_1$, $e_2$)} corresponds to $\ebind{x}{e_1}{e_2}$;
\code{Id($x$)} corresponds to $x$.

We use a map to represent an environment. The type of an environment is
\code{Map[String, Int]}.

\begin{verbatim}
type Env = Map[String, Int]
\end{verbatim}

We can add a pair of a key and a value to a map with the \code{+} operator.
For example,
where \code{m} is \code{Map(1 -> "one")}, \code{m + (2 -> "two")} is the same as
\code{Map(1 -> "one", 2 -> "two")}.

\begin{verbatim}
def interp(e: Expr, env: Env): Int = e match {
  case Num(n) => n
  case Add(l, r) => interp(l, env) + interp(r, env)
  case Sub(l, r) => interp(l, env) - interp(r, env)
  case Val(x, i, b) => interp(b, env + (x -> interp(i, env)))
  case Id(x) => env(x)
}
\end{verbatim}

Since the structure of the code is almost identical to the semantics rules, there
is nothing much to explain. In the \code{Id} case, when \code{x} is a key in
\code{env}, the corresponding value becomes the result of \code{interp}.
Otherwise, an exception is thrown, and the execution
terminates without producing any results.

\section{Exercises}

\begin{exercise}
\labex{identifiers-arrow}

For each of the following expression:
\begin{itemize}
  \item $\ebind{\cx}{(\ebind{\cx}{3}{\esub{5}{\cx}})}{\eadd{1}{\cx}}$
  \item $\ebind{\cx}{3}{\ebind{\cy}{5}{\eadd{1}{\cx}}}$
  \item $\ebind{\cx}{3}{\ebind{\cx}{5}{\eadd{1}{\cx}}}$
\end{itemize}
\begin{enumerate}
  \item Draw arrows from each bound occurrence to its binding occurrence.
  \item Draw dotted arrows from each shadowing variable to its shadowed variable.
\end{enumerate}

\end{exercise}

\begin{exercise}
\labex{identifiers-shadowing-impl}

This exercise asks you to implement the \code{shadowing} function, which
takes a \Lang expression as an argument and returns the set of every
identifier that becomes shadowed at least once in the expression. For
example, \code{shadowing($e$)} equals \code{Set("x")} where $e$ denotes
$\ebind{\cx}{\cy}{\ebind{\cx}{1}{\cz}}$.  The \code{shadowing} function
calls the \code{helper} function, which tracks the set of every identifier
defined already to detect shadowing.

Complete the following implementation:

\begin{verbatim}
def shadowing(e: Expr): Set[String] = helper(e, Set())

def helper(e: Expr, env: Set[String]): Set[String] =
  e match {
    case Num(n) => ???
    case Add(l, r) => ???
    case Id(x) => ???
    case Val(x, e, b) => ???
  }
\end{verbatim}

\end{exercise}

\setchapterpreamble[u]{\margintoc}
\chapter{First-Order Functions}
\labch{first-order-functions}

\renewcommand{\plang}{\textsf{VAE}\xspace}
\renewcommand{\lang}{\textsf{F1VAE}\xspace}

A function is one of the most important concepts in programming languages.
It is the key feature of functional languages, as the term functional
implies. Even in imperative languages, functions are important and widely-used.
This chapter focuses on first-order functions. \textit{First-order functions}
\index{first-order function} are functions that cannot take or return functions.
They are much restrictive than first-class functions but still very useful.

Consider the following Scala program:

\begin{verbatim}
def twice(x: Int): Int = x + x

println(twice(3) + twice(5))
\end{verbatim}

It defines a function, \code{double}. The functions takes one argument and
returns the twice of the argument. The program can call the function whenever we want.
\code{twice(3)} passes \code{3} as an argument to \code{twice}. Its result is
\code{6}, which is the twice of \code{3}. Similarly, \code{twice(5)} results in
\code{10}. Therefore, the program prints \code{16}.

This chapter defines \lang by adding first-order functions to \plang.
Every function in \lang is top-level. It means that a function definition cannot
be a part of an expression. We assume that a \lang program is a single expression
that is evaluated under an environment and a list of function definitions. This
design prevents us from exploring interesting topics like closures but enables us
to focus on the semantics of function calls. The next chapter will introduce
first-class functions and closures, which make functions more expressive.

\section{Syntax}

We can figure out the components of a function definition from the above
example. If we ignore the type annotations, the definition consists of three
parts: \code{twice}, \code{x}, and \code{x + x}. \code{twice} is the name of the
function; \code{x} is the parameter of the function; \code{x + x} is the body
of the function. Therefore, we can define the syntax of a function definition as
follows:

\[ d\ ::=\ \fundef{x}{x}{e} \]

The metavariable $d$ ranges over function definitions. Let $\embox{FunDef}$
denote the set of every function definition. A function definition
$\fundef{x_1}{x_2}{e}$ defines a function whose name is $x_1$, parameter is
$x_2$, and body is $e$. Both $x_1$ and $x_2$ are binding occurrences.
The scope of $x_1$ is the entire program; the scope of $x_2$ is $e$.
In many real-world languages, a function have zero or
more parameters. However, our syntax restricts a function to have one and only
one parameter. We adopt this design to make the semantics simple. Once you
understand a function with a single parameter, you can easily extend the concept
to a function with multiple parameters.

Using a function is to call the function. If we never call a function, the
function is useless. We need to add a new kind of expressions to the language:
function call expressions. The following is the syntax of \lang:
\sidenote{We omit the common part to \plang.}

\[ e\ ::=\ \cdots\ |\ \eappfo{x}{e} \]

$x(e)$ is a function call expression. It calls a function named $x$. $e$
determines the value of the argument of the call. Here, $x$ is a bound
occurrence. A function call always has only one argument since every function in
\lang has only one parameter.

\section{Semantics}

Like that we have introduced environments to store the values of variables,
we need a new semantic element that associates functions with their names.
Let us call it a function environment, which is a finite partial function from
identifiers to function definitions.

\[\embox{FEnv}=\embox{Id}\finto\embox{FunDef}\]
\[\Lambda\in\embox{FEnv}\]

The metavariable $\Lambda$ ranges over function environments.

Evaluation of an expression requires not only an environment but also a function
environment to handle function calls properly.
Therefore, the semantics is a relation over $\embox{Env}$,
$\embox{FEnv}$, $E$, and $\mathbb{Z}$.

\[\Rightarrow\subseteq\embox{Env}\times\embox{FEnv}\times E\times\mathbb{Z}\]

$(\sigma,\Lambda,e,n)\in\Rightarrow$ is true if and only if $e$ evalautes to $n$
under $\sigma$ and $\Lambda$. We write $\eval{\sigma,\Lambda}{e}{n}$
instead of $(\sigma,\Lambda,e,n)\in\Rightarrow$.

The following rule describes the semantics of function calls:

\semanticrule{Call}{
\begin{tabular}{@{\hskip0pt}l@{\hskip0pt}l}
  If \\
  & \evaln{\sigma\tand\Lambda}{e}{n'},\\
  & $x$ is in the domain of $\Lambda$,\\
  & $\Lambda(x)$ is $\fundef{x}{x'}{e'}$, and\\
  & \evaln{[x'\mapsto n']\tand\Lambda}{e'}{n},\\
  then \\
  & \evaln{\sigma\tand\Lambda}{\eappfo{x}{e}}{n}.
\end{tabular}
}

To evaluate $\eappfo{x}{e}$, we need to evalaute $e$ first to decide the value
of the argument. Then, we search for a function from the function environment
with a given function name, $x$. $x$ must be in the domain of the function
environment. Otherwise, $x$ is a free identifier, and a run-time error happens.
When $x$ is in the domain, we can get the corresponding function definition.
The function definition gives us the name of the parameter and the body. Since
every function is top-level, the body of each function does not belong to the
scope of any local variables. It can use no more than its own parameter. Thus,
the body should be evaluated under $[x'\mapsto n']$, not $\sigma[x'\mapsto n']$.
At the same time, since function calls do not affect function environments, the
same function environment is used for the evaluation of the body. The result of
the body is the result of the function call.

We can formulate the semantics as the following inference rule:

\[
  \inferrule
  {
    \eval{\sigma,\Lambda}{e}{n'} \\
    f\in\dom{\Lambda} \\
    \Lambda(x)=\fundef{x}{x'}{e'} \\
    \eval{[x'\mapsto n'],\Lambda}{e'}{n}
  }
  { \eval{\sigma,\Lambda}{\eappfo{x}{e}}{n} }
  \quad\textsc{[Call]}
\]

The other rules should be revised to consider function environments.
No expression modifies a function environment.

\semanticrule{Num}{
  \evaln{\sigma\tand\Lambda}{n}{n}.
}

\vspace{-1em}

\semanticrule{Add}{
If
  \evaln{\sigma\tand\Lambda}{e_1}{n_1}, and
  \evaln{\sigma\tand\Lambda}{e_2}{n_2},\\
then
  \evaln{\sigma\tand\Lambda}{\eadd{e_1}{e_2}}{n_1+n_2}.
}

\vspace{-1em}

\semanticrule{Sub}{
If
  \evaln{\sigma\tand\Lambda}{e_1}{n_1}, and
  \evaln{\sigma\tand\Lambda}{e_2}{n_2},\\
then
  \evaln{\sigma\tand\Lambda}{\esub{e_1}{e_2}}{n_1-n_2}.
}

\vspace{-1em}

\semanticrule{Val}{
If
  \evaln{\sigma\tand\Lambda}{e_1}{n_1}, and
  \evaln{\sigma[x\mapsto n_1]\tand\Lambda}{e_2}{n_2},\\
then
  \evaln{\sigma\tand\Lambda}{\ebind{x}{e_1}{e_2}}{n_2}.
}

\vspace{-1em}

\semanticrule{Id}{
If
  $x$ is in the domain of $\sigma$,\\
then
  \evaln{\sigma\tand\Lambda}{x}{\sigma(x)}.
}

We can fix the inference rules in a similar way.

\[
  \eval{\sigma,\Lambda}{n}{n}
  \quad\textsc{[Num]}
\]

\[
  \inferrule
  {
    \eval{\sigma,\Lambda}{e_1}{n_1} \\
    \eval{\sigma,\Lambda}{e_2}{n_2}
  }
  { \eval{\sigma,\Lambda}{\eadd{e_1}{e_2}}{n_1+n_2} }
  \quad\textsc{[Add]}
\]

\[
  \inferrule
  {
    \eval{\sigma,\Lambda}{e_1}{n_1} \\
    \eval{\sigma,\Lambda}{e_2}{n_2}
  }
  { \eval{\sigma,\Lambda}{\esub{e_1}{e_2}}{n_1-n_2} }
  \quad\textsc{[Sub]}
\]

\[
  \inferrule
  {
    \eval{\sigma,\Lambda}{e_1}{n_1} \\
    \eval{\sigma[x\mapsto n_1],\Lambda}{e_2}{n_2}
  }
  { \eval{\sigma,\Lambda}{\ebind{x}{e_1}{e_2}}{n_2} }
  \quad\textsc{[Val]}
\]

\[
  \inferrule
  { x\in\dom{\sigma} }
  { \eval{\sigma,\Lambda}{x}{\sigma(x)} }
  \quad\textsc{[Id]}
\]

\section{Interpreter}

The following Scala code expresses the abstract syntax of \lang:
\sidenote{We omit the common part to \plang.}

\begin{verbatim}
case class FunDef(f: String, x: String, b: Expr)

sealed trait Expr
...
case class Call(f: String, a: Expr) extends Expr
\end{verbatim}

Just like environments, function environments can be expressed as maps.
The type of a function environment is \code{Map[String, FunDef]} as it maps an
identifier to a function definition.

\begin{verbatim}
type FEnv = Map[String, FunDef]
\end{verbatim}

The following function evaluates a given expression under a given environment
and a given function environment.

\begin{verbatim}
def interp(e: Expr, env: Env, fEnv: FEnv): Int = e match {
  case Num(n) => n
  case Add(l, r) =>
    interp(l, env, fEnv) + interp(r, env, fEnv)
  case Sub(l, r) =>
    interp(l, env, fEnv) - interp(r, env, fEnv)
  case Val(x, i, b) =>
    interp(b, env + (x -> interp(i, env, fEnv)), fEnv)
  case Id(x) => env(x)
  case Call(f, a) =>
    val (x, e) = fEnv(f)
    interp(e, Map(x -> interp(a, env, fEnv)), fEnv)
}
\end{verbatim}

The implementation reflects the semantics exactly. You can easily check its
correctness with the case-wise comparison.

\section{Scope}

The current semantics is called static scope. Static scope allows the scope of a
binding occurrence to be determined statically, i.e. only by looking the code,
without executing it. In other words, a function body can use only variables
that has been defined already when the function is defined.
For example, consider the following code:

\[
  \fundef{\code{f}}{\cx}{\eadd{\cx}{\cy}}
\]

Since every function is top-level, while variables are local in \lang, \code{y}
does not belong to the scope of any binding occurrence of \code{y}. No variable
can be defined before the function. Therefore,
\code{y} is a free variable, and calling the function \code{f} must incur a
run-time error. It is true under the current semantics. The current function
call semantics evaluates the body of a function under the environment that has
only the value of the parameter. The environments at function call-sites never
affect the environment used for the evaluation of the body.

The opposite of static scope is dynamic scope, which makes every information in
the environment at each call-site available to the function body. The behavior
of a function depends on not only its argument but also its call-site.
An identifier in the body of a function becomes associated with a different
entity for each function call. It implies that the scope of a binding identifier
cannot be determined statically; it is determined at run time, i.e. dynamically.

For example, the following expression evaluates to $3$ when we assume the same
definition of \code{f} as before:

\[
  \eadd{(\ebind{\cy}{1}{\eappfo{\code{f}}{0})}}{(\ebind{\cy}{2}{\eappfo{\code{f}}{0})}}
\]

During the first function call, \code{y} in \code{f} is bound to the first
\code{y} and denotes $1$. However, during the second function call, it is
bound to the second one and denotes $2$. The scope of the first \code{y}
includes not only $\eappfo{\code{f}}{0}$, which is normal, but also the body of
\code{f}. It is the same for the second \code{y}. As you can see, under dynamic
scope, the scope of a binding identifier is not fixed; it becomes extended at
run time due to function calls.

To adopt dynamic scope to \lang, we need to change the function call semantics
as follows:

\semanticrule{Call-Dyn}{
\begin{tabular}{@{\hskip0pt}l@{\hskip0pt}l}
  If \\
  & \evaln{\sigma\tand\Lambda}{e}{n'},\\
  & $x$ is in the domain of $\Lambda$,\\
  & $\Lambda(x)$ is $\fundef{x}{x'}{e'}$, and\\
  & \evaln{\sigma[x'\mapsto n']\tand\Lambda}{e'}{n},\\
  then \\
  & \evaln{\sigma\tand\Lambda}{\eappfo{x}{e}}{n}.
\end{tabular}
}

It is equivalent to the following inference rule:

\[
  \inferrule
  {
    \eval{\sigma,\Lambda}{e}{n'} \\
    f\in\dom{\Lambda} \\
    \Lambda(x)=\fundef{x}{x'}{e'} \\
    \eval{\sigma[x'\mapsto n'],\Lambda}{e'}{n}
  }
  { \eval{\sigma,\Lambda}{\eappfo{x}{e}}{n} }
  \quad\textsc{[Call-Dyn]}
\]

The interpreter can be fixed like below.

\begin{verbatim}
case Call(f, a) =>
  val (x, e) = fEnv(f)
  interp(e, env + (x -> interp(a, env, fEnv)), fEnv)
\end{verbatim}

Dynamic scope prevents programs from being modular. The environment at each call-site
affects the behavior of a function. It hinders programmers from reasoning about
the semantics of a function based on the definition. They need to additionally consider
every possible call-site. It implies that different parts of a program unexpectedly
interfere with each other. Therefore, dynamic scope makes programs error-prone.
Because of the harmfulness of dynamic scope, most modern languages adopt static
scope.

\section{Exercises}

\begin{enumerate}
\item With the following list of function definitions in \lang:

$
\begin{array}{l}
  \fundef{\code{twice}}{\cx}{\eadd{\cx}{\cx}} \\
  \fundef{\code{x}}{\code{y}}{\code{y}} \\
  \fundef{\code{f}}{\code{x}}{\eadd{\cx}{1}} \\
  \fundef{\code{g}}{\code{g}}{\code{g}} \\
\end{array}
$

Show the results of evaluating the following expressions under the empty environment.
When it is an error, describe which error it is.
\begin{enumerate}
  \item $\eappfo{\code{twice}}{\code{twice}}$
  \item $\ebind{\cx}{5}{\eappfo{\cx}{\cx}}$
  \item $\eappfo{\code{g}}{3}$
  \item $\eappfo{\code{g}}{\code{f}}$
  \item $\eappfo{\code{g}}{\code{g}}$
\end{enumerate}
\end{enumerate}

\setchapterpreamble[u]{\margintoc}
\chapter{First-Class Functions}
\labch{first-class-functions}

\renewcommand{\plang}{\textsf{VAE}\xspace}
\renewcommand{\lang}{\textsf{FVAE}\xspace}

\textit{First-class functions}\index{first-class function} are functions that
can be used as values. They are much more expressive than first-order functions,
which are the topic of the previous chapter. This chapter explains the semantics
of first-class functions. We need to introduce the notion of a closure to make
first-class functions work properly. We will see what closures are and why they
are necessary.

This chapter defines \lang by extending \plang with first-class functions.
The only way to create a function in \lang is to make an \textit{anonymous
function}\index{anonymous function}, which is a function without a name.
However, we can add named functions as syntactic sugar. In addition,
we will see that even variable definitions can be considered as syntactic sugar.

\section{Syntax}

Consider an anonymous function in Scala:

\begin{verbatim}
(x: Int) => x + x
\end{verbatim}

If we ignore its type annotation, it consists of two parts: \code{x} and \code{x
+ x}. \code{x} is the parameter of the function; \code{x + x} is the body of the
function. This observation lets us know that an anonymous function consists of
its name and parameter.

In \textsf{F1VAE}, the sytnax of a function call is $\eappfo{x}{e}$. To call a
function, the name of the function should be given. However, it is not true in
languages with first-class functions. Let us see some function calls in Scala.

\begin{verbatim}
def twice(x: Int): Int = x + x
twice(1)
\end{verbatim}

\code{twice(1)} is a function call, and it designates a function by its name.

\begin{verbatim}
def makeAdder(x: Int): Int => Int =
  (y: Int) => x + y
makeAdder(3)(5)
\end{verbatim}

\code{makeAdder} is a function that returns a function. \code{makeAdder(3)} is a
function call, and its result is a function. Therefore, we can call the
resulting function again. \code{makeAdder(3)(5)} is an expression that calls
\code{makeAdder(3)}. It designates a function by an expression, rather than just
a name. We can conclude that the syntax of a function call in \lang should be
more general than \textsf{F1VAE} because of the presence of first-class
functions. In \lang, a function call consists of two expressions: one determines
the function to be called and the other determines the value of the argument.

We have used the term function call so far. In the context of functional
programming, we use the term \textit{function application}\index{function
application} more frequently. When we see \code{f(1)}, we say ``\code{f} is
applied to \code{1}'' instead of ``\code{f} is called with the argument
\code{1}.'' Applications sound more natural than calls especially when we are
talking about first-class functions. For example, we usually say
``\code{makeAdder(3)} is applied to \code{5}'' rather than ``\code{makeAdder(3)}
is called with the argument \code{5}.''

From the above observation on anonymous functions and function applications,
we can define the syntax of \lang. The following is the syntax of \lang:
\sidenote{We omit the common part to \plang.}

\[ e\ ::=\ \cdots\ |\ \efun{x}{e}\ |\ \eapp{e}{e} \]

\begin{itemize}
  \item $\efun{x}{e}$

    It is called an anonymous function or a \textit{lambda abstraction}.
    \index{lambda abstraction}
    It denotes a function whose parameter is $x$ and body is $e$. $x$ is a
    binding occurrence, and its scope is $e$.
    A function has zero or more parameters in many real-world languages,
    but we restrict a function in \lang to have one and only one parameter for
    simplicity as before.

  \item $\eapp{e_1}{e_2}$

    It is a function application, or just an application in short.
    $e_1$ denotes the function; $e_2$ denotes the argument.
\end{itemize}

\section{Semantics}

Integers are the only values in \plang. It is not true in \lang. Since
first-class functions are values, a value of \lang is either an integer or a
function. Thus, we define a new kind of semantic element, the value. The
metavariable $v$ ranges over values. Also, let $V$ be the set of every value.

\[ v\ ::=\ n\ |\ \clov{x}{e}{\sigma} \]

A value is either an integer or a closure. A \textit{closure}\index{closure}, which is a
function as a value, has the form $\clov{x}{e}{\sigma}$.
It is a pair of a lambda abstraction and an environment.
A lambda abstraction in a closure may have free identifiers,
but the environment of the closure can store the values denoted by the
free identifiers.

To discuss the necessity of closures, consider the following expression:

\[\eapp{\eapp{(\efun{\cx}{\efun{\cy}{\eadd{\cx}{\cy}}})}{1}}{2}\]

It is equivalent to \code{((x: Int) => (y: Int) => x + y)(1)(2)} in Scala.
The function $\efun{\cx}{\efun{\cy}{\eadd{\cx}{\cy}}}$ does not have any free
identifiers. The scope of \code{x} is ${\efun{\cy}{\eadd{\cx}{\cy}}}$; the scope
of \code{y} is ${\eadd{\cx}{\cy}}$. Therefore, both \code{x} and \code{y} in
$\eadd{\cx}{\cy}$ are bound occurrences. The whole expression must result in $3$,
which equals $1+2$, without a run-time error. You can check it by running the
equivalent Scala program.

If we consider a function value as just a lambda abstraction, not a closure,
evaluation of the above expression becomes problematic.
When the expression is evaluated, $\efun{\cx}{\efun{\cy}{\eadd{\cx}{\cy}}}$ is
applied to $1$ first. The result is a function value, which is a lambda
abstraction: $\efun{\cy}{\eadd{\cx}{\cy}}$. Next, $\efun{\cy}{\eadd{\cx}{\cy}}$
is applied to $2$. The result of the application can be computed by evaluating
$\eadd{\cx}{\cy}$ under the environment containing that $\cy$ denotes $2$.
However, there is no way to find the value of $\cx$. $\cx$ has become free
identifier although it was not in the beginning.

We adopt the notion of a closure to resolve the problem. When a lambda
expression evaluates to a function value, which is a closure, it captures the
environment. Since $\efun{\cy}{\eadd{\cx}{\cy}}$ is evaluated under the
environment containing that $\cx$ denotes $1$, its result is
$\clov{\cy}{\eadd{\cx}{\cy}}{[\cx\mapsto1]}$. The captured environment of the
closure records that $\cx$ is not a free identifier and denotes $1$.
When the closure is applied to $2$, its body ${\eadd{\cx}{\cy}}$ is evaluated
under $[\cx\mapsto1,\cy\mapsto2]$, not $[\cy\mapsto2]$. The addition
successfully results in $3$.

In summary, we need closures to retain the static scope semantics. First-class
functions can appear at any places expecting expressions.
However, the environments used for the
evaluation of their bodies must be determined statically. In other words,
the denotation of identifiers in the bodies of functions must be determined when
the functions are defined, not used. Therefore, each closure captures the surrounding
environment when it is created.

Now, let us define the semantics of \lang. Most things are the same as the
semantics of \plang, but we should be aware of that values now include not only
integers but also closures.

An environment is a finite partial function from identifiers to values.

\[ \embox{Env}=\embox{Id}\finto V \]

The semantics of \lang is a ternay relation over $\embox{Env}$, $E$, and $V$.

\[\Rightarrow\subseteq\embox{Env}\times E\times V\]

$\evald{e}{v}$ is true if and only if $e$ evaluates to $v$ under $\sigma$.

A lambda abstraction creates a closure containing the current environment.

\semanticrule{Fun}{
  \evaldn{\efun{x}{e}}{\clov{x}{e}{\sigma}}.
}

\vspace{-1em}

\[
  \evald{\efun{x}{e}}{\clov{x}{e}{\sigma}}
  \quad\textsc{[Fun]}
\]

A function application evaluates its both subexpressions. Then, it evaluates the
body of the closure under the environment obtained by adding the value of the
argument to the environment of the closure.

\semanticrule{App}{
  \begin{tabular}{@{\hskip0pt}l@{\hskip0pt}l}
    If \\
    & \evaldn{e_1}{\clov{x}{e}{\sigma'}}, \\
    & \evaldn{e_2}{v'}, and \\
    & \evaln{\sigma'[x\mapsto v']}{e}{v}, \\
    then \\
    & \evaldn{\eapp{e_1}{e_2}}{v}.
  \end{tabular}
}

\vspace{-1em}

\[
  \inferrule
  {
    \evald{e_1}{\clov{x}{e}{\sigma'}} \\
    \evald{e_2}{v'} \\
    \eval{\sigma'[x\mapsto v']}{e}{v}
  }
  { \evald{\eapp{e_1}{e_2}}{v} }
  \quad\textsc{[App]}
\]

We can reuse Rule \textsc{Num}, Rule \textsc{Add}, Rule \textsc{Sub}, and Rule \textsc{Id} of
\plang. However, it is important to note that \lang has more
cases that evaluation can fail than \plang. For example, consider Rule \textsc{Add}.

\semanticrule{Add}{
If
  \evaldn{e_1}{n_1}, and
  \evaldn{e_2}{n_2},\\
then
  \evaldn{\eadd{e_1}{e_2}}{n_1+n_2}.
}

\vspace{-1em}

\[
  \inferrule
  {
    \evald{e_1}{n_1} \\
    \evald{e_2}{n_2}
  }
  { \evald{\eadd{e_1}{e_2}}{n_1+n_2} }
  \quad\textsc{[Add]}
\]

The rule assumes the results of $e_1$ and $e_2$ to be integers. If the
assumption is violated, a run-time error happens. For example,
$\eadd{(\efun{\cx}{\cx})}{1}$ incurs a run-time error becuase the left operand
is a closure, not an integer.

We need to revise Rule \textsc{Val} of \plang a bit.
Since every value is an integer in \plang, a variable of \plang can denote only
an integer. In \lang, a variable should be able to denote a general value, not
only an integer.

\semanticrule{Val}{
If
  \evaldn{e_1}{v_1}, and
  \evaln{\sigma[x\mapsto v_1]}{e_2}{v_2},\\
then
  \evaldn{\ebind{x}{e_1}{e_2}}{v_2}.
}

\vspace{-1em}

\[
  \inferrule
  {
    \evald{e_1}{v_1} \\
    \eval{\sigma[x\mapsto v_1]}{e_2}{v_2}
  }
  { \evald{\ebind{x}{e_1}{e_2}}{v_2} }
  \quad\textsc{[Val]}
\]

Now, a variable can denote a value, not only an integer.

The following proof trees prove that
$\eapp{\eapp{(\efun{\cx}{\efun{\cy}{\eadd{\cx}{\cy}}})}{1}}{2}$
evaluates to $3$ under the empty environment.
The proof splits into three trees for readability.
Suppose that $\sigma_1=[\cx\mapsto1]$ and $\sigma_2=[\cx\mapsto1,\cy\mapsto2]$.

\[
  \inferrule
  {
    \eval{\emptyset}{
      \efun{\cx}{\efun{\cy}{\eadd{\cx}{\cy}}}
    }{\clov{\cx}{\efun{\cy}{\eadd{\cx}{\cy}}}{\emptyset}}
    \\
    \eval{\emptyset}{1}{1}
    \\
    \eval{\sigma_1}{
      \efun{\cy}{\eadd{\cx}{\cy}}
    }{\clov{\cy}{\eadd{\cx}{\cy}}{\sigma_1}}
  }
  { \eval{\emptyset}{
      \eapp{(\efun{\cx}{\efun{\cy}{\eadd{\cx}{\cy}}})}{1}
    }{\clov{\cy}{\eadd{\cx}{\cy}}{\sigma_1}}
  }
\]

\[
  \inferrule
  {
    \inferrule
    { \cx\in\dom{\sigma_2} }
    { \eval{\sigma_2}{\cx}{1} }
    \\
    \inferrule
    { \cy\in\dom{\sigma_2} }
    { \eval{\sigma_2}{\cy}{2} }
  }
  { \eval{\sigma_2}{\eadd{\cx}{\cy}}{3} }
\]

\[
\inferrule
{
  \eval{\emptyset}{
      \eapp{(\efun{\cx}{\efun{\cy}{\eadd{\cx}{\cy}}})}{1}
    }{\clov{\cy}{\eadd{\cx}{\cy}}{\sigma_1}}
  \\
  \emptyset\vdash2\Rightarrow 2
  \\
  \eval{\sigma_2}{\eadd{\cx}{\cy}}{3}
}
{ \eval{\emptyset}{
    \eapp{\eapp{(\efun{\cx}{\efun{\cy}{\eadd{\cx}{\cy}}})}{1}}{2}
  }{3}
}
\]

\section{Interpreter}

The following Scala code implements the syntax of \lang:
\sidenote{We omit the common part to \plang.}

\begin{verbatim}
sealed trait Expr
...
case class Fun(x: String, b: Expr) extends Expr
case class App(f: Expr, a: Expr) extends Expr
\end{verbatim}

\code{Fun($x$, $e$)} represents $\efun{x}{e}$; \code{App($e_1$, $e_2$)}
represents $\eapp{e_1}{e_2}$.

A value of \lang is either an integer or a closure. Thus, we represent a value
as an ADT.

\begin{verbatim}
sealed trait Value
case class NumV(n: Int) extends Value
case class CloV(p: String, b: Expr, e: Env) extends Value
\end{verbatim}

\code{NumV($n$)} represents $n$; \code{CloV($x$, $e$, $\sigma$)} represents
$\clov{x}{e}{\sigma}$.

An environment is a finite partial function from identifiers to values.
Therefore, the type of an environment is \code{Map[String, Value]}.

\begin{verbatim}
type Env = Map[String, Value]
\end{verbatim}

The following function evaluates a given expression under a given environment:

\begin{verbatim}
def interp(e: Expr, env: Env): Value = e match {
  case Num(n) => NumV(n)
  case Add(l, r) =>
    val NumV(n) = interp(l, env)
    val NumV(m) = interp(r, env)
    NumV(n + m)
  case Sub(l, r) =>
    val NumV(n) = interp(l, env)
    val NumV(m) = interp(r, env)
    NumV(n - m)
  case Id(x) => env(x)
  case Fun(x, b) => CloV(x, b, env)
  case App(f, a) =>
    val CloV(x, b, fEnv) = interp(f, env)
    interp(b, fEnv + (x -> interp(a, env)))
}
\end{verbatim}

In the \code{Num} case, the return value is \code{NumV(n)}, not \code{n},
since the function must return a value of the type \code{Value}.

In the
\code{Add} and \code{Sub} cases, we cannot assume that the operands are integers
any longer. We use pattern matching to discriminate integers from closures. If
both operands are integers, addition or subtraction succeeds. Otherwise, at
least one of them is a closure, and the interpreter crashes due to a pattern
matching failure. Note that this code is equivalent to the following code:

\begin{verbatim}
case Add(l, r) =>
  interp(l, env) match {
    case NumV(n) => interp(r, env) match {
      case NumV(m) => NumV(n + m)
      case _ => error("not an integer")
    }
    case _ => error("not an integer")
  }
\end{verbatim}

Similarly, in the \code{App} case, we use pattern matching to discriminate
closures from integers. The first expression of \code{App} must yield a clsoure,
not an integer, to make the execution succeed.

\section{Syntactic Sugar}
\labsec{fae}

We can add named local functions to \lang with the following change in the
syntax:

\[
  e\ ::=\ \cdots\ |\ \erec{x}{x}{e}{e}
\]

$\erec{x_1}{x_2}{e_1}{e_2}$ defines a function whose name is $x_1$, parameter is
$x_2$, and body is $e_1$. The scope of $x_1$ is $e_2$, and thus the function
does not allow recursion.

Instead of changing the semantics, \lang can provide named local functions as
syntactic sugar. Let $s$ be a string transformed into $\erec{x_1}{x_2}{e_1}{e_2}$
by the parser of \lang with named local functions embeded in the semantics.
To treat named local functions as syntactic sugar, the parser should transform
$s$ into $\ebind{x_1}{\efun{x_2}{e_1}}{e_2}$.

Variable definitions can be considered as syntactic sugar as well.
Let $s$ be a string transformed into $\ebind{x}{e_1}{e_2}$.
To make variable definitions syntactic sugar, the parser can transform $s$ into
$\eapp{(\efun{x}{e_2})}{e_1}$. The evaluation of $\eapp{(\efun{x}{e_2})}{e_1}$
evaluates $e_1$ first. Then, $e_2$ is evaluated under the environment that $x$ denotes
the result of $e_1$. This semantics is exactly the same as that of
$\ebind{x}{e_1}{e_2}$. Therefore, we can say that variable definitions are just
syntactic sugar in \lang.

Hereafter, we remove variable definitions from \lang and call the language
\textsf{FAE}. However, we may still use variable definitions in examples. It is
completely fine because they are considered as syntactic sugar.

Furthermore, we can treat even integers, addition, and subtraction as syntactic
sugar. The only things we need are variables, lambda abstractions, and function
applications. We can write any programs with these three kinds of expressions.
The \textit{lambda calculus}\index{lambda calculus} is a language that provides
only the three features. This book does not discuss how integers, addition, and
subtraction can be desugared into the lambda calculus.

\section{Exercises}

\begin{enumerate}
\item Consider the following expression:

\[
\ebind{\cx}{5}{
    \ebind{\code{f}}{\efun{\cy}{\eadd{\cy}{\cx}}}{
        \eapp{(\efun{\code{g}}{\eapp{\code{f}}{(\eapp{\code{g}}{1})}})}
        {(\efun{\cx}{\cx})}
    }
}
\]
Describe a trace of the evalaution in terms of arguments to the \code{interp}
function for every call. (There will be 16 calls.) The \code{interp} function
takes two arguments---an expression and an environment---so show both for each call.
For \code{Num}, \code{Id}, and \code{Fun} expressions, show their result values, which
are immediate. You can use the following abbreviations and possibly more abbreviations:

\begin{center}
\begin{tabular}{lcl}
$E_0$ & = & the whole expression \\
$E_1$ & = & $\efun{\cy}{\eadd{\cy}{\cx}}$ \\
$E_2$ & = & $\efun{\code{g}}{\eapp{\code{f}}{(\eapp{\code{g}}{1})}}$ \\
$E_3$ & = & $\efun{\cx}{\cx}$ \\
$E_4$ & = & $\ebind{\code{f}}{E_1}{\eapp{E_2}{E_3}}$
\end{tabular}
\end{center}

\item This exercise examines differences between semantics by changing scope.
The following code is an implementation of an interpreter:

\begin{verbatim}
def interp(e: Expr, env: Env): Value = e match {
  case Num(n) => NumV(n)
  case Add(l, r) =>
    val NumV(n) = interp(l, env)
    val NumV(m) = interp(r, env)
    NumV(n + m)
  case Sub(l, r) =>
    val NumV(n) = interp(l, env)
    val NumV(m) = interp(r, env)
    NumV(n - m)
  case Id(x) => lookup(x, env)
  case Fun(x, b) => CloV(x, b, env)
  case App(f, a) =>
    val CloV(x, b, fEnv) = interp(f, env)
    interp(b, __________ + (x -> interp(a, env)))
}
\end{verbatim}

Describe the semantics of the \code{App} case in prose
when we use each of the following for the blank above:
\begin{itemize}
  \item \code{env}
  \item \code{Map()}
  \item \code{fEnv}
\end{itemize}

\item This exercise extends \lang to support multiple parameters.
Consider the following language:
\[
\begin{array}{lrrl}
  \text{Expression}& e & ::= & \cdots\ |\ \efun{x\cdots x}{e}\ |\
  \eappfo{e}{e,\cdots,e}\\
  \text{Value}& v & ::= & \cdots\ |\ \clov{x\cdots x}{e}{\sigma} \\
\end{array}
\]
The semantics of some constructs are as follows:
\begin{itemize}
  \item Evaluating $\lambda x_1\ \cdots\ x_n. e$ under $\sigma$
      yields a closure $\langle \lambda x_1 \cdots\ x_n.e,\sigma \rangle$.
  \item If
    \begin{itemize}
    \item evaluating $e_0$ under $\sigma$ yields a closure $\langle \lambda x_1
      \cdots\ x_n.e,\sigma' \rangle$,
    \item evaluating $e_i$ under $\sigma$ yields $v_i$ for each $1 \leq i \leq
      n$, and
    \item evaluating $e$ under $\sigma'[x_1 \mapsto v_1, \cdots, x_n \mapsto
      v_n]$ yields $v$,
    \end{itemize}
\item[] then evaluating $\eappfo{e}{e,\cdots,e}$ under $\sigma$ yields $v$.
\end{itemize}

\begin{enumerate}
  \item Write the operational semantics of the form \fbox{$\sigma\vdash e \Rightarrow v$} for the expressions.
  \item Write the evaluation derivation of the following expression:
\[
\inferrule
{ \hspace*{0.8\textwidth} }
{
  \eval{\emptyset}{
    \eappfo{(\efun{\code{f}\
    \code{m}}{\eappfo{\code{f}}{\code{m}}})}{\efun{\cx}{\cx},8}
  }{~~~~~~~~~~}
}
\]
\end{enumerate}

\item Rewrite the following \lang expression to an \textsf{FAE} expression.
  You should not ``evaluate'' it. Consider the approach of \refsec{fae}.

\[
  \ebind{\cx}{\efun{\cy}{\eadd{8}{\cy}}}{\efun{\cy}\eapp{\eapp{\cx}{(\esub{10}{\cy})}}}
\]

\item This exercise modifies \lang to check body expressions when evaluating
  function expressions.
Consider we extend the value of \lang to include a special value $\uparrow$
to represent an error during function body checking.
Write the operational semantics of the form
\fbox{$\evald{e}{v}$} for a function expression
$\efun{x}{e}$, when its semantics  changes as follows:
\begin{itemize}
\item If every free identifier of $e$ is in the domain of $\sigma$ or is $x$,
  then evaluation of $\efun{x}{e}$ under $\sigma$ yields a closure
  containing the function expression and the environment.
\item Otherwise, evaluation of $\efun{x}{e}$ under $\sigma$ yields $\uparrow$.
\end{itemize}
You may use the semantic function \embox{fv}, which takes an
expression and returns the set of every free identifier in the expression.
For example, $\embox{fv}(\efun{\cx}{\eapp{\cy}{\cx}}) = \{ \cy \}$.

\item This exercise extends \lang with records.
  Consider the following language:
\[
\begin{array}{lrrl}
  \text{Field} & f & \in & \textit{Field} \\
  \text{Record} & \rho & \in & \embox{Record} = \embox{Field} \finto \embox{Value} \\
  \text{Expression}& e & ::= & \cdots\ |\ \{f\ e,\ \ldots,\ f\ e\}\ |\ e.f\ |\
  e;e \\
  \text{Value} & v & ::= & \cdots\ |\ \rho \\
\end{array}
\]

The semantics of some constructs are as follows:
\begin{itemize}
  \item The evaluation of $\{f_1\ e_1,\ \cdots,\ f_k\ e_k\}$
    under $\sigma$ yields a finite map $\rho$,
which maps $f_i \in \{f_1\ \cdots,\ f_k\}$
to the value $v_i$ which is evaluated from the expression $e_i$ under $\sigma$.
  \item The evaluation of $e.f$ under $\sigma$ yields the value of the field $f$ in the record $\rho$,
      where evaluation $e$ under $\sigma$ yields $\rho$.
  \item If evaluation of $e_1$ yields some value under $\sigma$, and evaluation
    of $e_2$ yields $v$ under $\sigma$,
      then evaluation of $e_1; e_2$ yields $v$ under $\sigma$.
\end{itemize}

Write the operational semantics of the form
$\boxed{\sigma \vdash e \Rightarrow v}$

\item This exercise extends \lang with pairs. Consider the following language:
\[
\begin{array}{lrrl}
  \text{Expression}& e & ::= & \cdots\ |\ (e,e)\ |\ e\textsf{.1}\ |\ e\textsf{.2} \\
  \text{Value} & v & ::= & \cdots\ |\ \fbox{ (a) }
\end{array}
\]

\begin{enumerate}
  \item Write the syntax of a pair value in \fbox{ (a) } and
    the operational semantics of the form \fbox{$\evald{e}{v}$} for the expressions.
  \item Write the evaluation derivation of the following expression:
    \[
      \inferrule
      {\hspace*{0.8\textwidth}}
      { \eval{\emptyset}{(8,(320,42)\textsf{.1})\textsf{.2}}{~~~~~} }
    \]
\end{enumerate}

\item This exercise replaces the environment-based semantics of \lang with 
a substitution-based semantics.
Consider the following implementation:
\begin{verbatim}
trait Expr
trait Value extends Expr
case class Num(n: Int) extends Expr with Value
case class Add(l: Expr, r: Expr) extends Expr
case class Sub(l: Expr, r: Expr) extends Expr
case class Val(x: String, e: Expr, b: Expr) extends Expr
case class Id(x: String) extends Expr
case class Fun(x: String, b: Expr) extends Expr with Value
case class App(f: Expr, a: Expr) extends Expr

def subst(e: Expr, x: String, v: Value): Expr = e match {
  case Num(n) => e
  case Add(l, r) =>
    Add(subst(l, x, v), subst(r, x, v))
  case Sub(l, r) =>
    Sub(subst(l, x, v), subst(r, x, v))
  case Val(y, i, b) =>
    val nb = if (y == x) b else subst(b, x, v)
    Val(y, subst(i, x, v), nb)
  case Id(name) =>
    if (name == x) v else e
  case Fun(y, b) =>
    Fun(y, if (y == x) b else subst(b, x, v))
  case App(f, a) =>
    App(subst(f, x, v), subst(a, x, v))
}

def interp(e: Expr): Value = e match {
  case Num(n) => Num(n)
  case Add(l, r) =>
    val Num(n) = interp(l)
    val Num(m) = interp(r)
    Num(n + m)
  case Sub(l, r) =>
    val Num(n) = interp(l)
    val Num(m) = interp(r)
    Num(n + m)
  case Val(x, i, b) =>
    interp(subst(b, x, interp(i)))
  case Id(x) => error("free identifier")
  case Fun(x, b) => Fun(x, b)
  case App(f, a) =>
    val Fun(x, b) = interp(f)
    interp(subst(b, x, interp(a)))
}
\end{verbatim}

In this implementation, a value is either an integer or a lambda abstraction:

\[ v\ ::=\ n\ |\ \efun{x}{e} \]

\begin{enumerate}
  \item
    Write the operational semantics of the above implementation
    of the form \fbox{$e \Rightarrow v$}
    where $e[x/v]$ denotes \code{subst($e$, $x$, $v$)}.

  \item Write the definition of the substitution $e[x/v]$
    of the form \fbox{$e[x/v]=e$}:

  \item Consider the following expression:
\[
\ebind{\code{z}}{\efun{\cx}{\esub{\cx}{\cy}}}{
    \ebind{\cy}{10}{
        (\eapp{\code{z}}{32})
    }
}
\]

\begin{enumerate}
\item What is the result of evaluating the expression under the empty
  environment in substitution-based \lang?
\item What is the result of evaluating the expression under the empty
  environment in environment-based \lang?
\item Why are the results different?
  \sidenote{We can make the semantics of substitution-based \lang equivalent to
    environment-based \lang by modifying \code{subst} function.}
\end{enumerate}
\end{enumerate}

\item Consider the following language:
\[
  \begin{array}{lrrl}
  \text{Expression}& e & ::= & n\ |\ x\ |\ \efun{x\cdots x}{e}\ |\
  \eappfo{e}{e,\cdots,e}\ |\ \textsf{get}\ e \\
  \text{Value}& v & ::= & n\ |\ \textsf{undefined}\ |\ \clov{x\cdots x}{e}{\sigma} \\
  \end{array}
\]
The semantics of some constructs are as follows:
\begin{itemize}
  \item The value of a function expression $\efun{x_1\cdots x_n}{e}$
    at an environment $\sigma$ is a closure $\clov{x_1\cdots x_n}{e}{\sigma}$.
  \item A function application $\eappfo{e_0}{e_1,\cdots,e_n}$ is evaluated as follows:
    \begin{itemize}
      \item Evaluate the subexpressions in order.
        The value of $e_0$ should be a closure
        $\clov{x_1\cdots x_m}{e}{\sigma}$
        that has $m$ parameters.
      \item Create an array $\alpha$ of size $n$ and
        initialize the $i$-th value of the array with the value of $e_{i+1}$
        where $0 \le i \le n-1$.
      \item Evaluate the closure body $e$ under the environment $\sigma$
        extended as follows:
        \begin{itemize}
          \item The value of the $i$-th parameter is the value of $e_i$
            where $1 \le i \le m \le n$.
          \item The value of the $j$-th parameter is the \textsf{undefined}
            value where $n < j \le m$.
        \end{itemize}
        and the array $\alpha$.
    \end{itemize}
  \item The value of $\textsf{get}\ e$ is the $n$-th value of the array $\alpha$
    where $n$ is the value of $e$ and the array indices start from $0$.
\end{itemize}

For example,
$\eappfo{(\efun{\cx\cy}{\cy})}{4}$
evaluates to \textsf{undefined}, and
$\eappfo{(\efun{\cx}{\textsf{get}\ 0})}{5}$
evaluate to $5$.

\begin{enumerate}
  \item Write the operational semantics of the form
    \fbox{$\eval{\sigma,\alpha}{e}{v}$}.
  \item Write the evaluation derivation of the following expression:
  \[
    \inferrule
    {\hspace*{0.8\textwidth}}
    {\eval{\emptyset,\emptyset}{\eappfo{(\efun{\cx\ \cy}{\textsf{get}\ x})}{2,19,141}}{~~~~~}}
  \]
\end{enumerate}

\item The following quote describes the JavaScript sequencing semantics:
\sidenote{\url{https://tc39.es/ecma262/\#sec-block-runtime-semantics-evaluation}}

\begin{quote}
The value of a \embox{StatementList} is the value of the last
value-producing item in the \embox{StatementList}.  For example, the
following calls to the \verb!eval! function all return the value 1:
\begin{verbatim}
eval("1;;;;;")
eval("1;()")
eval("1;var a;")
\end{verbatim}
\end{quote}
Consider the following language:
\[
  \begin{array}{lrrl}
    \text{Expression} & e & ::= & \textsf{()}\ |\ x\ |\ \efun{x}{e}\ |\ \eapp{e}{e}\ |\
    e;\cdots;e \\
    \text{Value} & v & ::= & \textsf{()}\ |\ \clov{x}{e}{\sigma} \\
  \end{array}
\]
The value of the sequence expression $e_1;\cdots;e_n$
is the value of the last expression whose value is not $\textsf{()}$.
If the values of all the expressions $e_1,\cdots,e_n$ are $\textsf{()}$,
the value of the sequence expression is $\textsf{()}$.
Write the operational semantics of each expression of the form
\fbox{$\evald{e}{v}$}.

\item Consider the following language:
\[
\begin{array}{rll}
e ::= & a& \text{atomic expression}\\
\mid& e\ a& \text{function application}\\
\mid& \textsf{fn}\ m& \text{function expression}\\
a ::= & n & \text{number}\\
\mid&x & \text{identifier}\\
m ::= & p\ \leadsto\ e & \text{pattern matching}\\
\mid& p\ \leadsto\ e\ \verb!|!\ m & \text{pattern matching sequence}\\
p ::= & \verb!_! & \text{wildcard pattern}\\
\mid&n& \text{number pattern}\\
\mid&x& \text{identifier pattern}
\end{array}
\]
where a value of the language $v$ is either a number $n$ or a closure $\langle m, \sigma\rangle$,
a result of evaluation $r$ is either a value $v$ or a failure in pattern matching $\uparrow$,
which is different from run-time errors,
and an environment $\sigma$ maps identifiers to their values.

The operational semantics rules for expressions and atomic expressions are as follows:

\fbox{$\sigma\vdash e \Rightarrow r$}
\[
\begin{array}{c}
  \inferrule
  {\sigma \vdash a \hookrightarrow v}
  {\sigma \vdash a \Rightarrow v}
  \qquad
  \inferrule
  {\sigma \vdash e \Rightarrow \uparrow}
  {\sigma \vdash e\ a \Rightarrow \uparrow}
  \qquad
  \sigma \vdash \textsf{fn}\ m \Rightarrow \langle m, \sigma \rangle
  \\[1.5em]
  \inferrule
  {
    \sigma \vdash e \Rightarrow \langle m, \sigma' \rangle \\
    \sigma \vdash a \Rightarrow v \\
    (\sigma', v) \vdash m \Rightarrow v'
  }
  {\sigma \vdash e\ a \Rightarrow v'}
  \\[1.5em]
  \inferrule
  {
    \sigma \vdash e \Rightarrow \langle m, \sigma' \rangle \\
    \sigma \vdash a \Rightarrow v \\
    (\sigma', v) \vdash m \Rightarrow \uparrow
  }
  {\sigma \vdash e\ a \Rightarrow \uparrow}
\end{array}
\]

\fbox{$\sigma\vdash a \hookrightarrow v$}
\[
\begin{array}{c}
  \sigma \vdash n \hookrightarrow n
  \qquad\qquad
  \inferrule
  {x \in \dom{\sigma}}
  {\sigma \vdash x \hookrightarrow \sigma(x)}
\end{array}
\]

The semantics of pattern matching $m$ and pattern $p$ are as follows:
\begin{itemize}
\item Evaluation of $p\ \leadsto\ e$ under $(\sigma, v)$ has two possibilities.
First, when evaluation of $p$ results in a new environment $\sigma'$,
the result of this pattern matching is the result of evaluation of $e$ under $\sigma+\sigma'$,
where $\sigma+\sigma'$ is a disjoint union of $\sigma$ and $\sigma'$.
Second, when evaluation of $p$ produces $\uparrow$,
the evaluation of this pattern matching produces $\uparrow$ as well.
\item Evaluation of ``$p\ \leadsto\ e\ \verb!|!\ m$'' under $(\sigma, v)$
also has two possibilities.
First, when evaluation of $p\ \leadsto\ e$ succeeds with a value $v'$,
the value of this pattern matching sequence is $v'$.
Second, when evaluation of $p\ \leadsto\ e$ fails,
the result of evaluation of this pattern matching sequence is
the result of evaluation of $m$.
\item Evaluation of the wildcard pattern \verb!_! under $(\sigma, v)$
produces the empty environment.
\item Evaluation of the number pattern $n$ under $(\sigma, v)$ has two possibilities.
If $v=n$,
it produces the empty environment.  Otherwise, it produces $\uparrow$.
\item Evaluation of the identifier pattern $x$ under $(\sigma, v)$
produces a singleton environment $\{x \mapsto v\}$
if $x$ is not in the domain of $\sigma$.
\end{itemize}
Write the operational semantics for $m$ and $p$
of the forms \fbox{$(\sigma, v)\vdash m \Rightarrow r$} and
\fbox{$(\sigma, v)\vdash p \Rightarrow \sigma/\uparrow$}, respectively,
where \fbox{$(\sigma, v)\vdash p \Rightarrow \sigma/\uparrow$} denotes
\fbox{$(\sigma, v)\vdash p \Rightarrow \sigma$} or
\fbox{$(\sigma, v)\vdash p \Rightarrow \uparrow$}.
Remember that the operational semantics do not specify run-time errors.
\end{enumerate}

\setchapterpreamble[u]{\margintoc}
\chapter{Lambda Calculus}
\labch{lambda-calculus}

\term{Lambda calculus} is a language featuring only variables, lambda
abstractions, and function applications. The article discusses how much lambda
calculus can express.

\section{Syntax}

The following is the abstract syntax of lambda calculus:

\[
\begin{array}{lrcl}
\text{Variable} & x & \in & \textit{Id} \\
\text{Expression} & e & ::= & x \\
&& | & \lambda x.e \\
&& | & e\ e \\
\text{Value} & v & ::= & \langle \lambda x.e,\sigma \rangle \\
\text{Environment} & \sigma & \in & \textit{Id}\hookrightarrow\text{Value}
\end{array}
\]

Unlike FAE, lambda calculus lacks integers, sums, and differences. An expression
is a variable, a lambda abstraction, or a function application. A value is a
closure but not an integer.

\section{Semantics}

The semantics of lambda calculus follows that of FAE.

\[\Rightarrow\subseteq\text{Environment}\times\text{Expression}\times\text{Value}\]

\[
\inferrule
{ x\in\mathit{Domain}(\sigma) }
{ \sigma\vdash x\Rightarrow \sigma(x)}
\]

\[
\sigma\vdash \lambda x.e\Rightarrow \langle\lambda x.e,\sigma\rangle
\]

\[
\inferrule
{ \sigma\vdash e_1\Rightarrow\langle\lambda x.e,\sigma'\rangle \\
  \sigma\vdash e_2\Rightarrow v' \\
  \sigma'\lbrack x\mapsto v'\rbrack\vdash e\Rightarrow v }
{ \sigma\vdash e_1\ e_2\Rightarrow v }
\]

\section{Church Numerals}

\term{Church numerals} encode numbers and operations treating number, such as
addition and multiplication, with lambda calculus. The following encode natural
numbers with lambda calculus:

\[
\begin{array}{rcl}
\mathit{encode}(0)&=&\lambda f.\lambda x.x \\
\mathit{encode}(1)&=&\lambda f.\lambda x.f\ x \\
\mathit{encode}(2)&=&\lambda f.\lambda x.f\ (f\ x) \\
\mathit{encode}(3)&=&\lambda f.\lambda x.f\ (f\ (f\ x)) \\
&\cdots&
\end{array}
\]

Intuitively, natural number \(n\) is a function that takes a function as an
argument and returns a function applying the argument \(n\) times:
\(\mathit{encode}(n)=f\mapsto f^n\). For any natural number \(n\) and any
function \(f\), \(n\ f\) equals \(f^n\).

The encoding of sums shows why the above encoding makes sense.

\[
\begin{array}{rcl}
+&\equiv&\lambda n.\lambda m.\lambda f.\lambda x.n\ f\ (m\ f\ x) \\
\mathit{encode}(e_1+e_2)&=&\lambda f.\lambda x.\mathit{encode}(e_1)\ f\
(\mathit{encode}(e_2)\ f\ x)
\end{array}
\]

For any natural number \(n\) and \(m\) and any function \(f\), \( (n+m)\ f\)
equals \(\lambda x.n\ f\ (m\ f\ x)\). Since \(n\ f\) equals \(f^n\), and \(m\ f\)
equals \(f^m\), it is \(\lambda x.f^n\ (f^m\ x)\) and equals \(\lambda x.f^{n+m}\
x\), or \(f^{n+m}\). Therefore, \(n+m\) equals \(f\mapsto f^{n+m}\), and the
\(\mathit{encode}\) function correctly encodes sums.

The following encodes products:

\[
\begin{array}{rcl}
\times&\equiv&\lambda n.\lambda m.\lambda f.\lambda x.n\ (m\ f)\ x \\
\mathit{encode}(e_1\times e_2)&=&\lambda f.\lambda x.\mathit{encode}(e_1)\
(\mathit{encode}(e_2)\ f)\ x
\end{array}
\]

For any natural number \(n\) and \(m\) and any function \(f\), \( (n\times m)\
f\) equals \(\lambda x.n\ (m\ f)\ x)\). Since \(m\ f\) equals \(f^m\), it is
\(\lambda x.n\ f^m\ x\). As \(n\ f\) is \(f^n\), \(n\ f^m\) is \( (f^m )^n\).
Hence, \(n\times m\) equals \(f\mapsto f^{n\times m}\), and the
\(\mathit{encode}\) function correctly encodes products.

Lambda calculus can encode differences and ratios as well. Moreover, it can
encode integers, rational numbers, and operations for them. They are beyond the
scope of the article.

\section{Church Booleans}

\term{Church Booleans} encode true, false, conditional expressions, logical sums,
logical products, and other logical language features. The hitherto defined
languages lack Boolean values. The section defines BAE by adding true, false, and
conditional expressions to AE.

\[
\begin{array}{lrcl}
\text{Boolean} & b & ::= & true \\
&& | & false \\
\text{Expression} & e & ::= & \cdots \\
&& | & b \\
&& | & \textsf{if}\ e\ e\ e \\
\text{Value} & v & ::= & \cdots \\
&& | & b
\end{array}
\]

It shows only features missed by AE. Metavariable \(b\) ranges over Boolean
values.

A conditional expression contains three subexpressions: the first one is its
condition; the second one is its true branch; the last one is its false branch.

\[
\vdash b\Rightarrow b
\]

\[
\inferrule
{ \vdash e_1\Rightarrow true \\ \vdash e_2\Rightarrow v }
{ \vdash \textsf{if}\ e_1\ e_2\ e_3\Rightarrow v}
\]

\[
\inferrule
{ \vdash e_1\Rightarrow false \\ \vdash e_3\Rightarrow v }
{ \vdash \textsf{if}\ e_1\ e_2\ e_3\Rightarrow v}
\]

If the condition of an expression is true, the true branch is evaluated, but the
false branch is not. Otherwise, the false branch is evaluated, but the true
branch is not.

The following encode true and false:

\[
\begin{array}{rcl}
\mathit{encode}(true)&=&\lambda a.\lambda b.a\ \_ \\
\mathit{encode}(false)&=&\lambda a.\lambda b.b\ \_ \\
\end{array}
\]

The underscore denotes an arbitrary expression. It implies that the value of an
argument is needless. It is possible to consider the underscore as any
expression, such as \(\lambda x.x\).

As the encoding of sums and products helps to understand the encoding of natural
number, the encoding of conditional expressions helps to understand the encoding
of true and false.

\[
\begin{array}{rcl}
\mathit{encode}(\textsf{if}\ e_1\ e_2\ e_3)&=&
\mathit{encode}(e_1)\ (\lambda\_ .\mathit{encode}(e_2))\ (\lambda\_
.\mathit{encode}(e_3))
\end{array}
\]

The underscore implies that a body does not refer to a parameter. Any parameter
name, such as \(x\), can replace the underscore. Assume that \(e_1\) denotes
true. The whole expression equals \( (\lambda a.\lambda b.a \_ )\ (\lambda\_
.\mathit{encode}(e_2))\ (\lambda\_ .\mathit{encode}(e_3))\), which is \(
(\lambda\_ .\mathit{encode}(e_2))\ \_ \). It denotes the same value as
\(\mathit{encode}(e_2)\) does. The evaluation of \(\mathit{encode}(e_3)\) is
unnecessary. If \(e_1\) denotes false, the expression denotes the same value as
\(\mathit{encode}(e_3)\) does, and the evaluation of \(\mathit{encode}(e_2)\) is
unnecessary.

Consider the following encoding:

\[
\begin{array}{rcl}
\mathit{encode}(true)&=&\lambda a.\lambda b.a \\
\mathit{encode}(false)&=&\lambda a.\lambda b.b \\
\textsf{if}&\equiv&\lambda c.\lambda a.\lambda b.c\ a\ b \\
\mathit{encode}(\textsf{if}\ e_1\ e_2\ e_3)&=&\mathit{encode}(e_1)\
\mathit{encode}(e_2)\ \mathit{encode}(e_3)
\end{array}
\]

It is simpler than the previous encoding but always evaluates both true and false
branches. Lambda calculus defined by the article uses eager evaluation, but
lambda calculus with lazy evaluation allows using the latter encoding without
computing needlessly. Later articles discuss lazy evaluation.

\section{Expressivity}

How expressive is lambda calculus? \term{Lambda computable} functions are
functions encodable with lambda calculus. Similarly, \term{Turing computable}
functions are functions implementable with \term{Turing machines}. Lambda
computable functions are Turing computable; Turing computable functions are
lambda computable. The set of every lambda computable function equals the set of
every Turing computable function; lambda calculus is \term{Turing complete}.
Computation doable with Turing machines almost equals that with real computers.
The only difference is that the tapes of Turing machines are infinite, while the
memories of computers are finite. Therefore, lambda calculus expresses everything
whom computers compute. Lambda calculus is 'the only' programming language.

\section{Exercises}

\setchapterpreamble[u]{\margintoc}
\chapter{Recursion}
\labch{recursion}

The article defines RFAE featuring recursive functions.

\section{CFAE}

CFAE adds conditional expressions to FAE.

\subsection{Syntax}

The below is the abstract syntax of CFAE. It shows a conditional expression,
which is the only new feature.

\[
\begin{array}{lrcl}
\text{Expression} & e & ::= & \cdots \\
&& | & \textsf{if0}\ e\ e\ e
\end{array}
\]

\(\textsf{if0}\) is similar to \(\textsf{if}\) of BAE, defined by the last
article, but its condition can be any value since CFAE lacks Boolean values. If a
condition is zero, the true branch is evaluated; otherwise, the false branch is
evaluated.

\subsection{Semantics}

The following define the semantics of conditional expressions:

\[
\inferrule
{ \sigma\vdash e_1\Rightarrow 0 \\ \sigma\vdash e_2\Rightarrow v }
{ \sigma\vdash \textsf{if0}\ e_1\ e_2\ e_3\Rightarrow v}
\]

\[
\inferrule
{ \sigma\vdash e_1\Rightarrow v' \\ v'\not=0 \\ \sigma\vdash e_3\Rightarrow v }
{ \sigma\vdash \textsf{if0}\ e_1\ e_2\ e_3\Rightarrow v}
\]

The semantics is similar to that of \(\textsf{if}\). Always one of the true and
false branches are evaluated, but not both.

\section{Recursion}

Is it possible to implement a factorial function with CFAE? Assume that CFAE
features multiplications. Firstly, consider a factorial function written
functionally. The following Scala function calculates factorials:

\begin{verbatim}
def factorial(n: Int): Int =
  if (n == 0) 1
  else n * factorial(n – 1)
\end{verbatim}

It seems the following CFAE expression is equivalent to the above code:

\[\textsf{val}\ factorial=\lambda n.\textsf{if0}\ n\ 1\ (n\times(factorial\
(n-1)))\ \textsf{in}\ \cdots\]

However, it is wrong since the scope of the binding occurrence of \(factorial\)
includes \(\cdots\) but excludes the lambda abstraction. Identifier \(factorial\)
in the lambda expression is free. CFAE disallows defining recursive functions.

\section{RFAE}

RFAE adds recursive functions to CFAE.

\subsection{Syntax}

The below is the abstract syntax of RFAE. It omits features common to CFAE.

\[
\begin{array}{lrcl}
\text{Expression} & e & ::= & \cdots \\
&& | & \textsf{def}\ x(x)=e\ \textsf{in}\ e
\end{array}
\]

\(\textsf{def}\ x_1(x_2)=e_1\ \textsf{in}\ e_2\) defines a recursive function. \(x_1\)
is the name of a function, and both \(e_1\) and \(e_2\) can refer to \(x_1\). For
example, the following is a factorial function:

\[\textsf{def}\ factorial(n)=\textsf{if0}\ n\ 1\ (n\times(factorial\ (n-1)))
\ \textsf{in}\ factorial\ 10\]

\subsection{Semantics}

The following defines the semantics of recursive functions:

\[
\inferrule
{ \sigma'=\sigma\lbrack x_1\mapsto\langle\lambda x_2.e_1,\sigma'\rangle\rbrack \\
  \sigma'\vdash e_2\Rightarrow v
}
{ \sigma\vdash \textsf{def}\ x_1(x_2)=e_1\ \textsf{in}\ e_2\Rightarrow v}
\]

The closure of a recursive function is similar to that of a lambda abstraction
but does not store the environment of the moment. Instead, it stores an
environment obtained by adding that the name of the function denotes the closure
to the environment. Calling a closure evaluates the body of the closure under the
environment of the closure so that recursive calls are valid.

The below proof trees prove that the factorial of one is one. The proof splits
into three trees for readability.

\[
\begin{array}{rcl}
\sigma_1&=&\lbrack f\mapsto\langle\lambda n.\textsf{if0}\ n\ 1\ (n\times(f\
(n-1))),\sigma_1\rangle\rbrack \\
\sigma_2&=&\sigma_1\lbrack n\mapsto 1\rbrack \\
&=&\lbrack f\mapsto\langle\lambda n.\textsf{if0}\ n\ 1\ (n\times(f\
(n-1))),\sigma_1\rangle,n\mapsto 1\rbrack \\
\sigma_3&=&\sigma_1\lbrack n\mapsto 0\rbrack \\
&=&\lbrack f\mapsto\langle\lambda n.\textsf{if0}\ n\ 1\ (n\times(f\
(n-1))),\sigma_1\rangle,n\mapsto 0\rbrack \\
\end{array}
\]

Assume the above.

\[
\inferrule
{
  \inferrule
  { f\in\mathit{Domain}(\sigma_2) }
  { \sigma_2\vdash f\Rightarrow \langle\lambda n.\textsf{if0}\ n\ 1\ (n\times(f\
(n-1))),\sigma_1\rangle }
  \\
  \inferrule
  {
    \inferrule
    { n\in\mathit{Domain}(\sigma_2) }
   { \sigma_2\vdash n\Rightarrow 1 } \\
    \sigma_2\vdash 1\Rightarrow 1
  }
  { \sigma_2\vdash n-1\Rightarrow 0 } \\
  \inferrule
  {
    \inferrule
    { n\in\mathit{Domain}(\sigma_3) }
    { \sigma_3\vdash n\Rightarrow 0 } \\
    \sigma_3\vdash 1\Rightarrow 1
  }
  { \sigma_3\vdash \textsf{if0}\ n\ 1\ (n\times(f\ (n-1))) \Rightarrow 1 }
}
{ \sigma_2\vdash f\ (n-1)\Rightarrow 1 }
\]

\[
\inferrule
{
  \inferrule
  { f\in\mathit{Domain}(\sigma_1)}
  { \sigma_1\vdash f\Rightarrow\langle\lambda n.\textsf{if0}\ n\ 1\ (n\times(f\
(n-1))),\sigma_1\rangle }
  \\
  {\sigma_1\vdash 1\Rightarrow 1}
  \\
  \inferrule
  {
    \inferrule
    { n\in\mathit{Domain}(\sigma_2) }
    { \sigma_2\vdash n\Rightarrow 1 } \\
    \inferrule
    {
      \inferrule
      { n\in\mathit{Domain}(\sigma_2) }
      { \sigma_2\vdash n\Rightarrow 1 } \\
      \sigma_2\vdash f\ (n-1)\Rightarrow 1
    }
    { \sigma_2\vdash (n\times(f\ (n-1)))\Rightarrow 1 }
  }
  {\sigma_2\vdash\textsf{if0}\ n\ 1\ (n\times(f\ (n-1)))\Rightarrow 1 }
}
{ \sigma_1\vdash f\ 1\Rightarrow 1 }
\]

\[
\inferrule
{
  \sigma_1=\lbrack f\mapsto\langle\lambda n.\textsf{if0}\ n\ 1\ (n\times(f\
(n-1))),\sigma_1\rangle\rbrack
  \\
  \sigma_1\vdash f\ 1\Rightarrow 1
}
{\emptyset\vdash
\textsf{def}\ f(n)=\textsf{if0}\ n\ 1\ (n\times(f\ (n-1)))\ \textsf{in}\ f\ 1
\Rightarrow 1
}
\]

\subsection{Implementing an Interpreter}

The following Scala code implements the abstract syntax and environments of RFAE:

\begin{verbatim}
sealed trait Expr
case class Num(n: Int) extends Expr
case class Add(l: Expr, r: Expr) extends Expr
case class Sub(l: Expr, r: Expr) extends Expr
case class Mul(l: Expr, r: Expr) extends Expr
case class Id(x: String) extends Expr
case class Fun(x: String, b: Expr) extends Expr
case class App(f: Expr, a: Expr) extends Expr
case class If0(c: Expr, t: Expr, f: Expr) extends Expr
case class Rec(f: String, x: String, b: Expr, e: Expr) extends Expr

sealed trait Value
case class NumV(n: Int) extends Value
case class CloV(p: String, b: Expr, var e: Env) extends Value

type Env = Map[String, Value]
def lookup(x: String, env: Env): Value =
  env.getOrElse(x, throw new Exception)
\end{verbatim}

\verb!If0! instnaces corresponds to conditional expressions; \verb!Rec! instances
corresponds to recursive functions. \verb!CloV! instances, which are closures,
have mutable environments because adding themselves to the environments requires
the environments mutable.

\begin{verbatim}
def interp(e: Expr, env: Env): Value = e match {
  case Num(n) => NumV(n)
  case Add(l, r) =>
    val NumV(n) = interp(l, env)
    val NumV(m) = interp(r, env)
    NumV(n + m)
  case Sub(l, r) =>
    val NumV(n) = interp(l, env)
    val NumV(m) = interp(r, env)
    NumV(n - m)
  case Mul(l, r) =>
    val NumV(n) = interp(l, env)
    val NumV(m) = interp(r, env)
    NumV(n * m)
  case Id(x) => lookup(x, env)
  case Fun(x, b) => CloV(x, b, env)
  case App(f, a) =>
    val CloV(x, b, fEnv) = interp(f, env)
    interp(b, fEnv + (x -> interp(a, env)))
  case If0(c, t, f) =>
    interp(
      if (interp(c, env) == NumV(0)) t else f,
      env
    )
  case Rec(f, x, b, e) =>
    val cloV = CloV(x, b, env)
    val nenv = env + (f -> cloV)
    cloV.e = nenv
    interp(e, nenv)
}
\end{verbatim}

The \verb!If0! case evaluates the true branch if the condition equals
\verb!NumV(0)! and the false branch otherwise. The \verb!Rec! case constructs a
closure and adds the closure to the environment of the closure.

The following calculates the factorial of three by calling the \verb!interp!
function:

\begin{verbatim}
// def f(n) = if0 n 1 (n * (f (n-1))) in f(3)
interp(
  Rec(
    "f", "n",
    If0(Id("n"),
        Num(1),
        Mul(
          Id("n"),
          App(Id("f"), Sub(Id("n"), Num(1)))
        )
    ),
    App(Id("f"), Num(3))
  ),
  Map.empty
)
// NumV(6)
\end{verbatim}

\section{Encoding Recursive Functions
}

As lambda calculus is Turing complete, recursive functions are encodable with
lambda calculus. Since both FAE and CFAE subsume the features of lambda calculus,
recursive functions are encodable with them as well.

\[
\begin{array}{rcl}
Z&\equiv&\lambda f.(\lambda x.f\ \lambda v.x\ x\ v)\ (\lambda x.f\ \lambda v.x\
x\ v)\\
\mathit{encode}(\textsf{def}\ x_1(x_2)=e_1\ \textsf{in}\ e_2)&=&
(\lambda x_1.e_2)\ (Z\ \lambda x_1.\lambda x_2.e_1)
\end{array}
\]

\(Z\) is a \term{fixed point combinator}; it calculates a fixed point of a given
function. A fixed point of a function is a value that makes the function yield
itself: a fixed point of function \(f\) is any \(x\) satisfying \(f(x)=x\). If an
argument given to \(Z\) is a function whose fixed point is a particular recursive
function, the result of applying \(Z\) to the function is the recursive function.
Consider \(\lambda f.\lambda x.\textsf{if0}\ x\ 1\ (x\times(f\ (x-1)))\). If
\(f\) is a factorial function, then \(\lambda x.\textsf{if0}\ x\ 1\ (x\times(f\
(x-1)))\) also is. Thus, the factorial function is a fixed point of \(\lambda
f.\lambda x.\textsf{if0}\ x\ 1\ (x\times(f\ (x-1)))\), and \(Z\ \lambda f.\lambda
x.\textsf{if0}\ x\ 1\ (x\times(f\ (x-1)))\) also is a factorial function.

How does the fixed point combinator work? \(Z\ \lambda f.\lambda x.\textsf{if0}\
x\ 1\ (x\times(f\ (x-1)))\) equals \( (\lambda x.f\ \lambda v.x\ x\ v)\ (\lambda
x.f\ \lambda v.x\ x\ v)\) if \(f\) denotes \(\lambda f.\lambda x.\textsf{if0}\ x\
1\ (x\times(f\ (x-1)))\). It equals \(f\ \lambda v.(\lambda x.f\ \lambda v.x\ x\
v)\ (\lambda x.f\ \lambda v.x\ x\ v)\ v\). Applying \(f\) to the argument results
in \(\lambda x.\textsf{if0}\ x\ 1\ (x\times(f\ (\lambda v.(\lambda x.f\ \lambda
v.x\ x\ v)\ (\lambda x.f\ \lambda v.x\ x\ v)\ v)\ (x-1)))\). Applying the
function to \(0\) yields \(1\) since \(x\) is \(0\). On the other hand, applying
the function to a nonzero value leads to \(x\times((\lambda v.(\lambda x.f\
\lambda v.x\ x\ v)\ (\lambda x.f\ \lambda v.x\ x\ v)\ v)\ (x-1))\). Then,
\(\lambda v.(\lambda x.f\ \lambda v.x\ x\ v)\ (\lambda x.f\ \lambda v.x\ x\ v)\
v\) has reappeared, but its argument has decreased by one. It successfully
simulates a recursive call and calculates factorials.

The following shows how to get the factorial of one:

\[
\begin{array}{rll}
& Z\ (\lambda f.\lambda x.\textsf{if0}\ x\ 1\ (x\times(f\ (x-1))))\ 1 \\
=&
(\lambda f.(\lambda x.f\ \lambda v.x\ x\ v)\ (\lambda x.f\ \lambda v.x\ x\ v))\
(\lambda f.\lambda x.\textsf{if0}\ x\ 1\ (x\times(f\ (x-1))))\ 1 \\
&& (f\leftarrow\lambda f.\lambda x.\textsf{if0}\ x\ 1\ (x\times(f\ (x-1)))) \\
\rightarrow &
(\lambda x.f\ \lambda v.x\ x\ v)\ (\lambda x.f\ \lambda v.x\ x\ v)\ 1 &
(f=\lambda f.\lambda x.\textsf{if0}\ x\ 1\ (x\times(f\ (x-1)))) \\
&& (x\leftarrow\lambda x.f\ \lambda v.x\ x\ v) \\
\rightarrow &
f\ (\lambda v.(\lambda x.f\ \lambda v.x\ x\ v)\ (\lambda x.f\ \lambda v.x\ x\ v)\
v)\ 1 \\
= &
(\lambda f.\lambda x.\textsf{if0}\ x\ 1\ (x\times(f\ (x-1))))\ (\lambda
v.(\lambda x.f\ \lambda v.x\ x\ v)\ (\lambda x.f\ \lambda v.x\ x\ v)\ v)\ 1\\
&& (f\leftarrow\lambda v.(\lambda x.f\ \lambda v.x\ x\ v)\ (\lambda x.f\ \lambda
v.x\ x\ v)\ v) \\
\rightarrow &
(\lambda x.\textsf{if0}\ x\ 1\ (x\times((\lambda v.(\lambda x.f\ \lambda v.x\ x\
v)\ (\lambda x.f\ \lambda v.x\ x\ v)\ v)\ (x-1))))\ 1 \\
&& (x\leftarrow 1) \\
\rightarrow &
1\times((\lambda v.(\lambda x.f\ \lambda v.x\ x\ v)\ (\lambda x.f\ \lambda v.x\
x\ v)\ v)\ 0)
\end{array}
\]

Via the fixed point combinator, a factorial function is implementable without
using a recursive function of RFAE.

\begin{verbatim}
// lambda f.(lambda x.f lambda v.x x v) (lambda x.f lambda v.x x v)
val Z =
  Fun("f",
    App(
      Fun("x",
        App(
          Id("f"),
          Fun("v", App(App(Id("x"), Id("x")), Id("v")))
        )
      ),
      Fun("x",
        App(
          Id("f"),
          Fun("v", App(App(Id("x"), Id("x")), Id("v")))
        )
      )
    )
  )

// (Z lambda f.lambda n.if0 n 1 (n * (f (n-1)))) 3
interp(
  App(
    App(
      Z,
      Fun("f", Fun("n",
        If0(Id("n"),
            Num(1),
            Mul(
              Id("n"),
              App(Id("f"), Sub(Id("n"), Num(1)))
            )
        )
      ))
    ),
    Num(3)
  ),
  Map.empty
)
// NumV(6)
\end{verbatim}

\setchapterpreamble[u]{\margintoc}
\chapter{Mutable Boxes}
\labch{mutable-boxes}

The article defines BFAE, a language with \term{boxes}, which are mutable spaces
storing data.

\section{Syntax
}

The below is the abstract syntax of BFAE. It shows only parts not in FAE.

\[
\begin{array}{lrcl}
\text{Expression} & e & ::= & \cdots \\
&& | & \textsf{ref}\ e \\
&& | & e:=e \\
&& | & !e \\
&& | & e;e
\end{array}
\]

\(\textsf{ref}\) creates a new box. \(\textsf{ref}\ e\) evaluates \(e\), creates
a new box, and stores the result of \(e\) in the box. The whole expression
denotes the box. A \term{low-level} interpretation of a box is a memory address
where its value exists. \(\textsf{ref}\) is similar to \verb!malloc! of C or
\verb!new! of object-oriented languages because it allocates spaces whom
programmers can use in a memory. For example, rewriting \(\textsf{ref}\ 1\) in C
yields \verb!int \term{p = (int }) malloc(sizeof(int)); *p = 1; return p;!.

\(:=\) modifies the content of a box. \(e_1:=e_2\) evaluates \(e_1\) and then
\(e_2\) and puts a value denoted by \(e_2\) in a box denoted by \(e_1\). The
result of the whole expression equals the value. If \(e_1\) does not denote a
box, a run-time error occurs. The expression directly modifies the memory. It is
similar to an assignment statement whose left-hand side is a pointer
\term{dereference} in C. For example, if \(x\) denotes a box or a pointer,
\(x:=2\) equals \(*x=2\).

\(!\) opens a box. If \(e\) results in a box, \(!e\) results in the content of
the box. It is similar to a pointer dereference not at the left-hand side of an
assignment statement in C. \(!x\) equals \(*x\).

\(e_1;e_2\) is an expression obtained by \term{sequencing} \(e_1\) and \(e_2\).
It evaluates \(e_1\) and then \(e_2\). The whole expression denotes a value
denoted by \(e_2\). Languages covered by the previous articles lack expression
sequencing since it is meaningless. The result of the former expression is always
discarded without being used by the latter expression. On the other hand, in the
case of BFAE, the former expression can create a box or modify the content of a
box and is therefore meaningful although its result is ignored. Most languages
allow programmers to write sequences of multiple statements or expressions
separated by commas or line breaks. Expression sequencing of BFAE equals them.

\section{Semantics
}

Unlike the previous languages, BFAE is not a purely functional language but
provides mutable boxes. Like imperative languages, execution of a BFAE program
mutates a state. Despite mutability, the semantics of BFAE retains the functional
viewpoint, which interprets an expression as a program and executes the program
by finding a value denoted by the expression.

Defining a mutable memory is crucial to define the semantics of BFAE. The article
calls such memories \term{stores}. A store saves values in boxes existing during
execution of a program. Boxes are distinguishable as they have distinct names.
The article calls the names \term{addresses}. Let \(Addr\) be the set of every
possible address. A store is a partial function from an address to a value. A
value mapped from the address of a box by a store is the content of the box.

\[
\begin{array}{lrcl}
\text{Address} & a & \in & \mathit{Addr} \\
\text{Store} & M & \in & \text{Address}\hookrightarrow\text{Value}
\end{array}
\]

Metavariable \(a\) ranges over addresses; \(M\) ranges over stores.

The semantics does not require a concrete definition of a box. Since the address
of a box determines the meaning of the box, addresses are enough for the
semantics. The result of evaluating an expression denoting a box is an address.
For example, \(\textsf{ref}\ e\) yields an address. The previous languages allow
only integers and closures to be values, but BFAE additionally allows addresses
to be values.

\[
\begin{array}{lrcl}
\text{Value} & e & ::= & \cdots \\
&& | & a
\end{array}
\]

Note that the remaining part of the article keeps using the concept of a box.
Even though the semantics abstracts boxes with addresses, in the programmers'
perspective, boxes do exist. Besides, boxes are more intuitive than addresses for
explanations.

Evaluating \(!e\) needs not only an environment but also a store. If \(e\)
denotes a box, the store has a value stored in the box. Hence, evaluating an
expression requires a store, and \(\Rightarrow\) must be a relation over
environments, stores, expressions, and values.

Evaluating \(\textsf{ref}\ e\) creates a new box; evaluating \(e_1:=e_2\) changes
the content of a box. Both modify stores. Modifying a store differs from
extending an environment with a new variable. Evaluating \(\textsf{val}\ x=e_1\
\textsf{in}\ e_2\) adds \(x\) to the environment, but only \(e_2\) uses the
extended environment because the scope of the binding occurrence of \(x\) is
\(e_2\) but nowhere else. For instance, let \(e\) denote the expression, then
both \(e'\) and addition of \(e+e'\) do not require the extended one. On the
other hand, the modified store is unnecessary for the subexpressions of an
expression that creates or modifies a box while other parts of the program need
the modified one. Consider \(x:=2;!x\) as an example. \(!x\) must know that
\(x:=2\) has changed the content of a box denoted by \(x\) into \(2\). Therefore,
how stores change due to expressions is important. If an expression contains two
subexpressions, a store obtained by evaluating the first subexpression has to be
passed to the evaluation of the second expressions. Since the semantics needs to
yield both of the resulting value and a new store, the final correct definition
of \(\Rightarrow\) is a relation over environments, stores, expressions, values,
and stores. Evaluation reads the former store and creates the second store.


\[\Rightarrow\subseteq\text{Environment}\times\text{Store}\times\text{Expression}\times\text{Value}\times\text{Store}\]

\(\sigma,M\vdash e\Rightarrow v,M'\) implies that evaluating \(e\) under
\(\sigma\) and \(M\) results in \(v\) and creates \(M'\). The semantics uses the
\term{store passing style}. The style allows defining BFAE, featuring mutable
boxes, without any mutable concepts.

The order among subexpressions matters as the subexpressions can modify stores.
Suppose that \(x\) denotes a box, and the box contains \(1\). \( (x:=2)+(!x) \)
yields \(4\) if \(x:=2\) comes first so that \(!x\) equals \(2\). In contrast, \(
(x:=2)+(!x) \) yields \(3\) if \(!x\) comes first so that \(!x\) equals \(1\).
Inference rules naturally set orders by passing stores.

The inference rules for integers, variables, lambda abstractions equal those of
FAE but additionally require stores. They maintain the contents of given stores.

\[
\sigma,M\vdash n\Rightarrow n,M
\]

\[
\inferrule
{ x\in\mathit{Domain}(\sigma) }
{ \sigma,M\vdash x\Rightarrow \sigma(x),M }
\]

\[
\sigma,M\vdash \lambda x.e\Rightarrow \langle\lambda x.e,\sigma\rangle,M
\]

A sequenced expression per se cannot modify a given store, but its subexpressions
can. The left subexpression comes before the right one. The evaluation of the
right considers any modifications of the store made by the left.

\[
\inferrule
{ \sigma,M\vdash e_1\Rightarrow v_1,M_1 \\
  \sigma,M_1\vdash e_2\Rightarrow v_2,M_2 }
{ \sigma,M\vdash e_1;e_2\Rightarrow v_2,M_2 }
\]

The rule passes \(M_1\), obtained by evaluating the left, to the evaluation of
the right and discards the result of the left. The final result equals the result
of the right.

Sums, differences, and function applications are similar to sequenced
expressions. They cannot modify stores, but their subexpressions can. The
evaluation orders are the same as those of sequenced expressions. BFAE chooses
the left-to-right order for every expression, but orders vary with languages.

\[
\inferrule
{ \sigma,M\vdash e_1\Rightarrow n_1,M_1 \\
  \sigma,M_1\vdash e_2\Rightarrow n_2,M_2 }
{ \sigma,M\vdash e_1+e_2\Rightarrow n_1+n_2,M_2 }
\]

\[
\inferrule
{ \sigma,M\vdash e_1\Rightarrow n_1,M_1 \\
  \sigma,M_1\vdash e_2\Rightarrow n_2,M_2 }
{ \sigma,M\vdash e_1-e_2\Rightarrow n_1-n_2,M_2 }
\]

\[
\inferrule
{ \sigma,M\vdash e_1\Rightarrow \langle\lambda x.e,\sigma'\rangle,M_1 \\
  \sigma,M_1\vdash e_2\Rightarrow v_1,M_2 \\
  \sigma'\lbrack x\mapsto v_1\rbrack,M_2\vdash e\Rightarrow v_2,M_3 }
{ \sigma,M\vdash e_1\ e_2\Rightarrow v_2,M_3 }
\]

Note that the evaluation of the body of a closure can modify the store as well.

The remaining, creating a box, modifying a box, and opening a box, change or read
stores directly.

\[
\inferrule
{ \sigma,M\vdash e\Rightarrow v,M' \\
  a\not\in \mathit{Domain}(M') }
{ \sigma,M\vdash \textsf{ref}\ e\Rightarrow a,M'\lbrack a\mapsto v\rbrack }
\]

When a box is created, the address of the box must not belong to the domain of a
store attained by evaluating the subexpression. The result contains the address
and the store plus a mapping from the address to the value of the subexpression.

\[
\inferrule
{ \sigma,M\vdash e_1\Rightarrow a,M_1 \\
  \sigma,M_1\vdash e_2\Rightarrow v,M_2 }
{ \sigma,M\vdash e_1:=e_2\Rightarrow v,M_2\lbrack a\mapsto v\rbrack }
\]

Like other expressions, an expression modifying a box uses the left-to-right
order. If the left subexpression results in an address, a value associated with
the address in the store changes into the value of the right subexpression. The
whole expression denotes the value.

\[
\inferrule
{ \sigma,M\vdash e\Rightarrow a,M' \\
  a\in \mathit{Domain}(M') }
{ \sigma,M\vdash !e\Rightarrow M'(a),M' }
\]

The subexpression of an expression opening a box has to yield an address. A value
denoted by the expression is a value at the address in the store. The store
differs from a given store but is the outcome of evaluating the subexpression.
For example, \(!(\textsf{ref}\ 1)\) correctly results in \(1\) only if the
opening uses the store given by \(\textsf{ref}\ 1\).

\section{Implementing an Interpreter
}

The following Scala code implements the abstract syntax, environments, and stores
of BFAE:

\begin{verbatim}
sealed trait Expr
case class Num(n: Int) extends Expr
case class Add(l: Expr, r: Expr) extends Expr
case class Sub(l: Expr, r: Expr) extends Expr
case class Id(x: String) extends Expr
case class Fun(x: String, b: Expr) extends Expr
case class App(f: Expr, a: Expr) extends Expr
case class NewBox(e: Expr) extends Expr
case class SetBox(b: Expr, e: Expr) extends Expr
case class OpenBox(b: Expr) extends Expr
case class Seqn(l: Expr, r: Expr) extends Expr

sealed trait Value
case class NumV(n: Int) extends Value
case class CloV(p: String, b: Expr, e: Env) extends Value
case class BoxV(a: Addr) extends Value

type Env = Map[String, Value]
def lookup(x: String, env: Env): Value =
  env.getOrElse(x, throw new Exception)

type Addr = Int
type Sto = Map[Addr, Value]
def storeLookup(a: Addr, sto: Sto): Value =
  sto.getOrElse(a, throw new Exception)
def malloc(sto: Sto): Addr =
  sto.keys.maxOption.getOrElse(0) + 1
\end{verbatim}

The \verb!NewBox! class corresponds to creating a box; the \verb!SetBox! class
corresponds to modifying a box; the \verb!OpenBox! class corresponds to opening a
box; the \verb!Seqn! class corresponds to sequencing expressions. \verb!BoxV!
instances model values that are addresses. \verb!Addr! denotes the type of an
address and is \verb!Int!. \verb!Sto! is the type of a store and is a map from
\verb!Addr! to \verb!Value!. The \verb!lookup! function finds a value from a
given environment; the \verb!storeLookup! function finds from a given store. The
\verb!malloc! function computes an address unused by a given store.

\verb!interp! takes an expression, an environment, and a store as arguments and
returns the pair of a value and a store.

\begin{verbatim}
def interp(e: Expr, env: Env, sto: Sto): (Value, Sto) = e match { ... }
\end{verbatim}

I discuss the cases of the pattern matching in the same order as the inference
rules.

\begin{verbatim}
case Num(n) => (NumV(n), sto)
case Id(x) => (lookup(x, env), sto)
case Fun(x, b) => (CloV(x, b, env), sto)
\end{verbatim}

The \verb!Num!, \verb!Id!, and \verb!Fun! cases use given stores as the results.

\begin{verbatim}
case Seqn(l, r) =>
  val (_, ls) = interp(l, env, sto)
  interp(r, env, ls)
case Add(l, r) =>
  val (NumV(n), ls) = interp(l, env, sto)
  val (NumV(m), rs) = interp(r, env, ls)
  (NumV(n + m), rs)
case Sub(l, r) =>
  val (NumV(n), ls) = interp(l, env, sto)
  val (NumV(m), rs) = interp(r, env, ls)
  (NumV(n - m), rs)
case App(f, a) =>
  val (CloV(x, b, fEnv), ls) = interp(f, env, sto)
  val (v, rs) = interp(a, env, ls)
  interp(b, fEnv + (x -> v), rs)
\end{verbatim}

The \verb!Seqn!, \verb!Add!, \verb!Sub!, and \verb!App! cases do not directly
modify or read stores, but pass stores returned from the recursive calls to the
other recursive calls or use them as the results.

\begin{verbatim}
case NewBox(e) =>
  val (v, s) = interp(e, env, sto)
  val a = malloc(s)
  (BoxV(a), s + (a -> v))
\end{verbatim}

The \verb!NewBox! case calls \verb!malloc! to compute an unused address after
evaluating the subexpression. The result contains the address and the extended
store.

\begin{verbatim}
case SetBox(b, e) =>
  val (BoxV(a), bs) = interp(b, env, sto)
  val (v, es) = interp(e, env, bs)
  (v, es + (a -> v))
\end{verbatim}

The \verb!SetBox! case evaluates the two subexpressions and modifies a given
store. The result contains the result of the second subexpression and the
modified store.

\begin{verbatim}
case OpenBox(e) =>
  val (BoxV(a), s) = interp(e, env, sto)
  (storeLookup(a, s), s)
\end{verbatim}

The \verb!OpenBox! case finds a value corresponding to an address denoted by the
subexpression in a given store. The store remains unchanged.

I have covered all the cases. The following shows the whole code at once:

\begin{verbatim}
sealed trait Expr
case class Num(n: Int) extends Expr
case class Add(l: Expr, r: Expr) extends Expr
case class Sub(l: Expr, r: Expr) extends Expr
case class Id(x: String) extends Expr
case class Fun(x: String, b: Expr) extends Expr
case class App(f: Expr, a: Expr) extends Expr
case class NewBox(e: Expr) extends Expr
case class SetBox(b: Expr, e: Expr) extends Expr
case class OpenBox(b: Expr) extends Expr
case class Seqn(l: Expr, r: Expr) extends Expr

sealed trait Value
case class NumV(n: Int) extends Value
case class CloV(p: String, b: Expr, e: Env) extends Value
case class BoxV(a: Addr) extends Value

type Env = Map[String, Value]
def lookup(x: String, env: Env): Value =
  env.getOrElse(x, throw new Exception)

type Addr = Int
type Sto = Map[Addr, Value]
def storeLookup(a: Addr, sto: Sto): Value =
  sto.getOrElse(a, throw new Exception)
def malloc(sto: Sto): Addr =
  sto.keys.maxOption.getOrElse(0) + 1

def interp(e: Expr, env: Env, sto: Sto): (Value, Sto) = e match {
  case Num(n) => (NumV(n), sto)
  case Id(x) => (lookup(x, env), sto)
  case Fun(x, b) => (CloV(x, b, env), sto)
  case Seqn(l, r) =>
    val (_, ls) = interp(l, env, sto)
    interp(r, env, ls)
  case Add(l, r) =>
    val (NumV(n), ls) = interp(l, env, sto)
    val (NumV(m), rs) = interp(r, env, ls)
    (NumV(n + m), rs)
  case Sub(l, r) =>
    val (NumV(n), ls) = interp(l, env, sto)
    val (NumV(m), rs) = interp(r, env, ls)
    (NumV(n - m), rs)
  case App(f, a) =>
    val (CloV(x, b, fEnv), ls) = interp(f, env, sto)
    val (v, rs) = interp(a, env, ls)
    interp(b, fEnv + (x -> v), rs)
  case NewBox(e) =>
    val (v, s) = interp(e, env, sto)
    val a = malloc(s)
    (BoxV(a), s + (a -> v))
  case SetBox(b, e) =>
    val (BoxV(a), bs) = interp(b, env, sto)
    val (v, es) = interp(e, env, bs)
    (v, es + (a -> v))
  case OpenBox(e) =>
    val (BoxV(a), s) = interp(e, env, sto)
    (storeLookup(a, s), s)
}
\end{verbatim}

The below code evaluates \( (\lambda x.(x:=1);!x)\ (\textsf{ref}\ 2) \). The
result is \(1\), and the final store has a single box whose content is \(1\).

\begin{verbatim}
// (lambda x.(x:=1); !x) (ref 2)
interp(
  App(
    Fun("x",
      Seqn(
        SetBox(Id("x"), Num(1)),
        OpenBox(Id("x"))
      )
    ),
    NewBox(Num(2))
  ),
  Map.empty,
  Map.empty
)
// (NumV(1), Map(1 -> NumV(1)))
\end{verbatim}

\section{Exercises}

\begin{enumerate}
\item Given the following expression:

\begin{verbatim}
(lambda b1. ((lambda b2. b1 := 8; !b2) b1)) (ref 7)
\end{verbatim}

write out the arguments to and results of \texttt{interp} each time it is called during
the evaluation of a program.  Write them out in the order in which the calls to \texttt{interp}
occur during evaluation.  Use the \texttt{interp} code at the end.

Write down the environments and stores using the \verb+{}+ notation,
not using the more verbose data constructors.
Also, use an arrow notation for both the store and the environment.

An example, if the first call to \texttt{interp} were:

\begin{verbatim}
interp(Add(Id("x"), Id("y")),
       Map("x" -> NumV(3), "y" -> NumV(4)),
       Map.empty)
\end{verbatim}

then a model solution would be:

\begin{verbatim}
exp: x + y
env: {x -> NumV(3), y -> NumV(4)}
sto: {}
ans: NumV(7) {}

exp: x
env: {x -> NumV(3), y -> NumV(4)}
sto: {}
ans: NumV(3) {}

exp: y
env: {x -> NumV(3), y -> NumV(4)}
sto: {}
ans: NumV(4) {}
\end{verbatim}

Note that the environment and store read in order from left to right.

\end{enumerate}

\setchapterpreamble[u]{\margintoc}
\chapter{Mutable Variables}
\labch{mutable-variables}

\renewcommand{\plang}{\textsf{FAE}\xspace}
\newcommand{\bfae}{\textsf{BFAE}\xspace}
\renewcommand{\lang}{\textsf{MFAE}\xspace}

\bfae of the previous chatper provides boxes. Boxes are good abstraction
of mutable objects and data structures but do not explain mutable variables
well. Boxes, mutable objects, mutable data structures are values, while mutable
variables are names. Mutable variables allow the values associated with
names to change. We can find the notion of a mutable variable in many
real-world languages except a few functional languages including OCaml and
Haskell.

The semantics of mutable variables seem trivial. We can change the values of
mutable variables. However, if we use mutable variables with closures, we can do
many interesting things. Consider the following Scala program:

\begin{verbatim}
def makeCounter(): () => Int = {
  var x = 0
  def counter(): Int = {
    x += 1
    x
  }
  counter
}

val counter1 = makeCounter()
val counter2 = makeCounter()

println(counter1())
println(counter2())
println(counter1())
println(counter2())
\end{verbatim}

The program defines the function \code{makeCounter}. The function has a mutable
variable \code{x} whose initial value is \code{0}. Also, it defines and returns the function
\code{counter}. \code{counter} increases the value of \code{x} by
one every time it is called. We make two counters by calling \code{makeCounter}
twice. Then, we call each counter in turn and print the return value.
What does the program print? The first value will be \code{1} since
\code{counter1} will increase \code{x} by one from zero and return \code{x}.
However, predicting the other ones is difficult. We need the exact semantics of
mutable variables to answer the question.

This chapter defines \lang by extending \plang with mutable variables.
We will see the semantics of mutable variables. Addition of mutable
variables gives us a chance to explore a different design of the function
application semantics. We will see what is the call-by-reference semantics and how
it differs from the call-by-value semantics.

\section{Syntax}

As variables are mutable in \lang, we need to add expressions that change the
values of variables. The following is the abstract syntax of \lang:
\sidenote{We omit the common part to \plang.}

\[ e\ ::=\ \cdots\ |\ \eset{x}{e} \]

$\eset{x}{e}$ is an expression changing the value of a variable. $x$ is the
variable to be updated; $e$ determines the new value of the variable. Unlike
$\eset{e_1}{e_2}$ in \bfae, the left-hand-side of an assignment is restricted to
a variable. The reason is that variables are not values. We cannot get a
variable by evaluating an expression. The only way to designate a variable is
to write the name of the variable, and the syntax reflects this point.

Note that \lang lacks sequencing expressions, which exist in \bfae. Actually, it
is not problematic at all. We can desugar sequencing expressions into
lambda abstractions and function applications: transform $\eseq{e_1}{e_2}$ into
$\eapp{(\efun{x}{e_2})}{e_1}$, where $x$ is not free in $e_2$. The semantics of
$\eseq{e_1}{e_2}$ is that evaluating $e_1$ first and then $e_2$. The evaluation
of $\eapp{(\efun{x}{e_2})}{e_1}$ is the same. First, $\efun{x}{e_2}$ evaluates
to a closure, which means that $e_2$ is not evaluated. Then, the argument,
$e_1$, is evaluated. Finally, the body of the closure, $e_2$, is evaluated. Therefore,
$e_1$ is evaluated before $e_2$. Also, since $x$ is not a free identifier in
$e_2$, the result of $e_1$ is never used even though it is passed to the
function as an argument. Thus, we can conclude that the desugaring is correct.
We may use sequencing expressions in examples as they can be easily desugared.

\section{Semantics}

Since \lang provides mutation, its semantics uses store-passing just like \bfae.
Therefore, a store is a finite partial function from addresses to values.

\[ \embox{Sto} = \embox{Addr}\finto V \]
\[ M \in \embox{Sto} \]

The semantics is a
relation over $\embox{Env}$, $\embox{Sto}$, $E$, $V$, and $\embox{Sto}$.

\[\Rightarrow\subseteq\embox{Env}\times\embox{Sto}\times E\times V\times\embox{Sto}\]

Like \plang, a value of \lang is either an integer or a closure. It is different
from \bfae, which allows addresses to be values. \bfae treats addresses as
values because expressions creating boxes evaluate to addresses. However, there
are mutable variables instead of boxes in \lang. \lang has addresses to support
mutation, but they are used only for tracking the value of each variable. Addresses
are not exposed to programmers as values.

An environment of \lang is a finite partial function from identifiers to addresses,
but not values.

\[ \embox{Env}=\embox{Id}\finto\embox{Addr} \]
\[ \sigma \in \embox{Env} \]

The semantics needs environments to find the value denoted by
a variable. \plang, whose variables are immutable, is
satisfied with environments that take identifiers as input and return values.
However, variables of \lang are mutable. Evaluation outputs a value and a store.
Environments are not the output of evaluation. Therefore, we cannot use
environments to record changes in the values of variables. On the other hand, we
can use stores to record the changes as stores are output of evaluation.
It implies that stores must contain the values of variables to make variables
mutable. Since the value of a certain variable is stored at a particular address
of a store, an environment must know the address of each variable.

One may ask if we can remove environments from the semantics and consider a
store as a partial function from identifiers to values. However, in fact,
removing environments from the semantics prevents use of static scope.
Assume that the semantics lacks environments, and a store is a partial
function from an identifier to a value.
Consider $\eseq{(\eapp{(\efun{\cx}{\eset{\cx}{1}})}{0})}{\cx}$.
To evaluate the function application, the value of the argument should be
recorded in the store. After evaluating the function body, the store will
be passed to the evaluation of $\cx$. Then, $\cx$ evaluates to $1$
without a run-time error since $\cx$ is in the store.
On the contrary, under static scope, the scope of $\cx$ includes only $x:=1$.
The expression should result in a run-time error because $\cx$ outside the
function is a free identifier. Environments are essential for resolving this
problem. Environments enable static scope, and stores make variables mutable.
The semantics must have both.

Because of the change in the definition of an environment, the semantics of
identifiers need to be revised. An environment has the address of a given
identifier, and a store has the value at a given address. Therefore,
we need two steps to find the value of a variable: find the address of a variable
from the environment and find the value at the address from the store.

\semanticrule{Id}{
  \begin{tabular}{@{\hskip0pt}l@{\hskip-10pt}l}
    If \\
    & x is in the domain of $\sigma$, and \\
    & $\sigma(x)$ is in the domain of $M$,\\
    then \\
    & \sevaldn{}{x}{M(\sigma(x))}{}.
  \end{tabular}
}

\vspace{-1em}

\[
  \inferrule
  {
    x\in\dom{\sigma} \\
    \sigma(x)\in\dom{M}
  }
  { \sevald{}{x}{M(\sigma(x))}{} }
  \quad\textsc{[Id]}
\]

Like boxes in \bfae, each variable of \lang has its own address. New variables
can be defined only by function applications. Hence,
function applications are the only expressions that create new addresses.
Let us see the semantics of function applications.

\semanticrule{App}{
  \begin{tabular}{@{\hskip0pt}l@{\hskip-10pt}l}
    If \\
    & \sevaldn{}{e_1}{\clov{x}{e}{\sigma'}}{1}, \\
    & \sevaldn{1}{e_2}{v'}{2}, \\
    & $a$ is not in the domain of $M_2$, and \\
    & \sevaln{\sigma'[x\mapsto a]}{M_2[a\mapsto v']}{e}{v}{M_3}, \\
    then \\
    & \sevaldn{}{\eapp{e_1}{e_2}}{v}{3}.
  \end{tabular}
}

\vspace{-1em}

\[
  \inferrule
  {
    \sevald{}{e_1}{\clov{x}{e}{\sigma'}}{1} \\
    \sevald{1}{e_2}{v'}{2} \\
    a\not\in \dom{M_2} \\
    \seval{\sigma'[x\mapsto a]}{M_2[a\mapsto v']}{e}{v}{M_3}
  }
  { \sevald{}{\eapp{e_1}{e_2}}{v}{3} }
  \quad\textsc{[App]}
\]

The evaluation of the subexpressions is the same as \bfae. However, the remaining
procedure is different. We cannot store the value of the argument in the
environment. It should go into the store. To put the value into the store, we
need a fresh address. The name of the parameter becomes associated with the
address in the environment; the address becomes associated with the value of the
argument in the store. Finally, the function body is evaluated.

Changing the value of a variable is similar to changing the value of a box of
\bfae. However, we have to evaluate an expression to find the address of the box
to be updated in \bfae. On the other hand, we can find the address of the
variable to be updated from the environment by using its name in \lang.

\semanticrule{Set}{
  \begin{tabular}{@{\hskip0pt}l@{\hskip-10pt}l}
    If \\
    & $x$ is in the domain of $\sigma$, and \\
    & \sevaldn{}{e}{v}{1}, \\
    then \\
    & \sevaln{\sigma}{M}{\eset{x}{e}}{v}{M_1[\sigma(x)\mapsto v]}.
  \end{tabular}
}

\vspace{-1em}

\[
  \inferrule
  {
    x\in\dom{\sigma} \\
    \sevald{}{e}{v}{1}
  }
  { \seval{\sigma}{M}{\eset{x}{e}}{v}{M_1[\sigma(x)\mapsto v]} }
  \quad\textsc{[Set]}
\]

We can reuse the rules of \bfae for the other expressions.

Now, we can answer the question in the beginning of the chapter. At each call to
\code{makeCounter}, a new address is allocated to store the value of \code{x}.
Therefore, \code{x} of \code{counter1} uses a different address from \code{x} of
\code{counter2}. Both of the first two lines of \code{println} print \code{1}.
Also, each address is permanent throughout the execution. When
a call to \code{counter1} updates the value of \code{x}, the change remains
until the next call to \code{counter1}. Thus, both of the last two lines of
\code{println} print \code{2}.

\section{Interpreter}

The following Scala code implements the syntax of \lang:
\sidenote{We omit the common part to \plang.}

\begin{verbatim}
sealed trait Expr
...
case class Set(x: String, e: Expr) extends Expr
\end{verbatim}

\code{Set($x$, $e$)} represents $\eset{x}{e}$.

The types of an address and a store can be defined as in \bfae.

\begin{verbatim}
type Addr = Int
type Sto = Map[Addr, Value]
\end{verbatim}

We need to change the type of an environment.

\begin{verbatim}
type Env = Map[String, Addr]
\end{verbatim}

As in \bfae, \code{interp} takes an expression, an environment, and a store as
arguments and returns a pair of a value and a store.
\sidenote{We omit the common part to \bfae.}

\begin{verbatim}
def interp(e: Expr, env: Env, sto: Sto): (Value, Sto) =
  e match {
    ...
    case Id(x) => (sto(env(x)), sto)
    case App(f, a) =>
      val (CloV(x, b, fEnv), ls) = interp(f, env, sto)
      val (v, rs) = interp(a, env, ls)
      val addr = rs.keys.maxOption.getOrElse(0) + 1
      interp(b, fEnv + (x -> addr), rs + (addr -> v))
    case Set(x, e) =>
      val (v, s) = interp(e, env, sto)
      (v, s + (env(x) -> v))
  }
\end{verbatim}

In the \code{Id} case, the function finds the address of the variable first and
then the value at the address.

In the \code{App} case, we use the same strategy to
the interpreter of \bfae to compute a new address. The body of the function is
evaluated under the extended environment and the extended store.

The \code{Set} case uses the environment to find the address of the variable.
Then, it updates the store to change the value of the variable.

\section{Call-by-Reference}

Novices in programming often implement a swap function incorrectly. For example,
consider the following C++ program:

\begin{verbatim}
void swap(int x, int y) {
    int tmp = x;
    x = y;
    y = tmp;
}

int a = 1, b = 2;
swap(a, b);
std::cout << a << " " << b << std::endl;
\end{verbatim}

They expect the program to print \code{2 1} as \code{swap} has been called.
On the contrary, their expectation is wrong. The result is \code{1 2}.
We can explain the reason based on
the content of this chapter. When \code{swap} is called, two new fresh addresses
are allocated for \code{x} and \code{y}. The values of \code{a} and \code{b} are
copied and stored in the addresses, respectively. The function affects only the values in
the addresses of \code{x} and \code{y}. It never touches the addresses of
\code{a} and \code{b}. As a consequence, while the values of \code{x} and
\code{y} are swapped, the values of \code{a} and \code{b} are not.

This is the usual semantics of function applications. The values of arguments are
copied and saved at fresh addresses. This semantics is called
\textit{call-by-value}\index{call-by-value} (\acrshort{cbvLabel})
as function calls pass the values of arguments.

People have explored another semantics for function applications to implement
functions like \code{swap} easily. The semantics is called
\textit{call-by-reference}\index{call-by-reference} (\acrshort{cbrLabel}).
In this semantics, function calls pass the references, i.e. addresses,
when variables are used as arguments.

The following rule defines the semantics of a function application using CBR when its
argument is a variable:

\semanticrule{App-Cbr}{
  \begin{tabular}{@{\hskip0pt}l@{\hskip-10pt}l}
    If \\
    & \sevaldn{}{e_1}{\clov{x'}{e'}{\sigma'}}{1}, \\
    & $x$ is in the domain of $\sigma$, and \\
    & \sevaln{\sigma'[x'\mapsto\sigma(x)]}{M_1}{e'}{v}{M_2}, \\
    then \\
    & \sevaldn{}{\eapp{e}{x}}{v}{2}.
  \end{tabular}
}

\vspace{-1em}

\[
  \inferrule
  {
    \sevald{}{e_1}{\clov{x'}{e'}{\sigma'}}{1} \\
    x\in \dom{\sigma} \\
    \seval{\sigma'[x'\mapsto\sigma(x)]}{M_1}{e'}{v}{M_2}
  }
  { \sevald{}{\eapp{e}{x}}{v}{2} }
  \quad\textsc{[App-Cbr]}
\]

The rule does not evaluate the argument to get a value. It simply uses the
address of the variable. Then, the parameter of the function has the exactly
same address to the argument. Any change in the parameter that happens in the
function body affects the variable outside the function. We say that the
parameter is an \textit{alias}\index{alias} of the argument as they share the
same address.

Even if we want to adopt the CBR semantics in \lang, we cannot use it when the argument is not
a variable. We cannot get an address from an expression that is not a variable.
In such cases, we fall back to the CBV semantics. The following rule specifies
such cases:

\semanticrule{App-Cbv}{
  \begin{tabular}{@{\hskip0pt}l@{\hskip-10pt}l}
    If \\
    & \sevaldn{}{e_1}{\clov{x}{e}{\sigma'}}{1}, \\
    & $e_2$ is not an identifier, \\
    & \sevaldn{1}{e_2}{v'}{2}, \\
    & $a$ is not in the domain of $M_2$, and \\
    & \sevaln{\sigma'[x\mapsto a]}{M_2[a\mapsto v']}{e}{v}{M_3}, \\
    then \\
    & \sevaldn{}{\eapp{e_1}{e_2}}{v}{3}.
  \end{tabular}
}

\vspace{-1em}

\[
  \inferrule
  {
    \sevald{}{e_1}{\clov{x}{e}{\sigma'}}{1} \\
    e_2\not\in\embox{Id} \\
    \sevald{1}{e_2}{v'}{2} \\
    a\not\in \dom{M_2} \\
    \seval{\sigma'[x\mapsto a]}{M_2[a\mapsto v']}{e}{v}{M_3}
  }
  { \sevald{}{\eapp{e_1}{e_2}}{v}{3} }
  \quad\textsc{[App-Cbv]}
\]

It is the same as Rule \textsc{App} except that it has one more premise to
ensure that the argument is not a variable.

The interpreter needs the following change:

\begin{verbatim}
case App(f, a) =>
  val (CloV(x, b, fEnv), ls) = interp(f, env, sto)
  a match {
    case Id(y) =>
      interp(b, fEnv + (x -> env(y)), ls)
    case _ =>
      val (v, rs) = interp(a, env, ls)
      val addr = rs.keys.maxOption.getOrElse(0) + 1
      interp(b, fEnv + (x -> addr), rs + (addr -> v))
  }
\end{verbatim}

It uses pattern matching on the argument expression of a function application.
When it is an identifier, the CBR semantics can be used. Otherwise, it falls
back to the CBV semantics.

We can find a few languages that support CBR in real-world. One example is C++.
In C++, if there is an ampersand in front of the name of a parameter,
the parameter uses the CBR semantics. We can fix the function \code{swap} with
this feature.

\begin{verbatim}
void swap(int &x, int &y) {
    int tmp = x;
    x = y;
    y = tmp;
}

int a = 1, b = 2;
swap(a, b);
std::cout << a << " " << b << std::endl;
\end{verbatim}

It is enough to fix only the first line to make the parameters use CBR. When
\code{swap} is applied to \code{a} and \code{b}, the addresses of \code{a} and
\code{b} are passed. The address of \code{x} is the same as that of
\code{a}, and the address of \code{y} is the same as that of \code{b}.
Therefore, the function swaps not only the values of \code{x} and \code{y} but
also the values of \code{a} and \code{b}. The program prints \code{2 1} as
intended.

\section{Exercises}

\begin{enumerate}
\item
This exercise extends \lang with pointers.
Consider the following language:
\[
  \begin{array}{@{}r@{~}r@{~}l@{}}
    e & ::= & \cdots\ |\ \ast e \ |\ \& x\ |\ \ast\eset{e}{e} \\
    v & ::= & \cdots\ |\ a \\
  \end{array}
\]
The semantics of some constructs are as follows:
\begin{itemize}
  \item The value of $\ast e$ is the value in the store at the address denoted by
    the expression.
  \item The value of $\& x$ is the address denoted by the identifier in the environment.
  \item The evaluation of $\ast\eset{e_1}{e_2}$ evaluates $e_2$ first, which is
    the value of the whole expression. Then, it evaluates $e_1$, and it maps the
    address denoted $e_1$ to the value of $e_2$.
\end{itemize}
Write the operational semantics of the form \fbox{$\sevald{}{e}{v}{}$} for the expressions.

\item The following code is an excerpt from the implementation of the interpreter for
\lang:
\begin{verbatim}
def interp(e:Expr, env:Env, sto:Sto): (Value, Sto) =
  e match {
    ...
    case App(f, a) =>
      val (CloV(x, b, fEnv), ls) = interp(f, env, sto)
      a match {
        case Id(y) =>
          interp(b, fEnv + (x -> env(y)), ls)
        case _ =>
          val (v, rs) = interp(a, env, ls)
          val addr = rs.keys.maxOption.getOrElse(0) + 1
          interp(b, fEnv + (x -> addr), rs + (addr -> v))
      }
  }
\end{verbatim}

\begin{enumerate}
  \item What is this semantics? CBV or CBR?

    \vspace{1em}
    \hspace{-2.5em}
Consider the following expression in \lang:
\[
\ebind{\code{n}}{42}{
    \ebind{\code{f}}{\efun{\code{g}}{(\eapp{\code{g}}{\code{n}})}}{
        (\eapp{\code{f}}{(\efun{\cx}{\eadd{\cx}{8}})})
    }
}
\]
\item Show the environment and store just before evaluating addition in the CBV semantics.
\item Show the environment and store just before evaluating addition in the CBR semantics.
\end{enumerate}

\item Consider the following language:

\vspace{-1em}

\[
\begin{array}{rcl}
  e & ::= & n\ |\ x\ |\ \eadd{e}{e} \\
  c & ::= & \eskip\ |\ \eset{x}{e}\ |\ \eifz{e}{c}{c}
  \ |\ \ewhilez{e}{c}\ |\ \eseq{c}{c} \\
  v & ::= & n \\
\end{array}
\]

where $c$ ranges over commands.

Under a given environment, $e$ evaluates to a value without modifying the
environment:

\[
  \inferrule
  {}
  { \evald{n}{n} }
  \quad
  \inferrule
  { x\in\dom{\sigma} }
  { \evald{x}{\sigma(x)} }
  \quad
  \inferrule
  { \evald{e_1}{n_1} \\ \evald{e_2}{n_2} }
  { \evald{\eadd{e_1}{e_2}}{n_1+n_2} }
\]

\begin{enumerate}
  \item
    A command takes an environment as input and produces a new environment as
    output. We write $\evald{c}{\sigma'}$ if $c$ produces $\sigma'$ when
    $\sigma$ is given.
    The following sentences describe the semantics of commands:
    \begin{itemize}
      \item $\eskip$ does not change a given environment.
      \item $\eset{x}{e}$ evaluates $e$ to get a value $v$ and
        updates the value of $x$ to $v$. $x$ does not need to be in a given
        environment.
      \item $\eifz{e}{c_1}{c_2}$ evaluates $e$ to get a value $v$.
        If $v$ is $0$, $c_1$ is evaluated. Otherwise, $c_2$ is evaluated.
      \item $\ewhilez{e}{c}$ evaluates $e$ to get a value $v$.
        If $v$ is not $0$, the command terminates without changing a given
        environment. Otherwise, it evaluates $c$ and then checks the result of $e$ again
        under the new environment.
        This process repeats until the result of $e$ becomes nonzero.
      \item $\eseq{c_1}{c_2}$ evaluates $c_1$ first to get a new environment.
        Then, it evaluates $c_2$ under the new environment.
    \end{itemize}
    Write the operational semantics of commands of the form
    \fbox{$\evald{c}{\sigma}$}.

  \item
    Draw the evaluation derivation of
    $\eseq{\eset{\cx}{0}}{\eifz{\cx}{\eskip}{\eset{\cx}{1}}}$
    under the empty environment.
\end{enumerate}

\end{enumerate}

\setchapterpreamble[u]{\margintoc}
\chapter{Lazy Evaluation}
\labch{lazy-evaluation}

\renewcommand{\plang}{\textsf{FAE}\xspace}
\renewcommand{\lang}{\textsf{LFAE}\xspace}

This chapter is about lazy evaluation. \textit{Lazy evaluation}\index{lazy
evaluation} means delaying the evaluation of an expression until the result
is required. The opposite of lazy evaluation is \textit{eager
evaluation}\index{eager evaluation}, which evaluates an expression even if in
the case that it is unknown whether the result of the expression is necessary
for future computation. There are multiple features that lazy evaluation can be
applied to in programming languages. For example, arguments for function
applications can be evaluated lazily; each argument of a function application is
evaluated when the argument is used in the function body, not before the
evaluation of the function body starts. Another example is a variable
definition. The initialization of variable happens when the variable is
used for the first time, not when it is defined. Actually, as you have seen
already, local variables can be considered as syntactic sugar of function
parameters, it is enough to focus on laziness in function applications.
Therefore, this book considers lazy evaluation only on arguments of function
applications.

All the previously defined languages in the book are eager languages. They use
the CBV semantics for function applications. CBV can be considered equivalent to
eager evaluation. CBV means that every argument is passed as a value. The value
of an argument can be acquired only by evaluating the argument expression. It
implies that every argument is evaluated before the evaluation of the function
body regardless of whether the argument is used during the body evaluation.
Thus, the CBV semantics is equal to the eager evaluation semantics.\sidenote{One
may wonder whether CBR is eager or not. The best answer is that CBR is
irrelevant to distinction between eagerness and laziness. As shown in the previous chapter, CBR is
possible only when an argument is a variable, whose address is known. In that
case, there is nothing to evaluate. We have to make a choice between passing the
address (CBR) and passing the value at the address (CBV). On the other hand,
choosing one of eagerness and laziness is about a choice between evaluating the
argument and not evaluating the argument. CBR can be used in both eager and lazy
languages.}

This chapter mainly discusses \textit{call-by-name}\index{call-by-name}
(\acrshort{cbnLabel}) semantics, which is one form of lazy evaluation.
In the CBN semantics, each argument is passed as its name, i.e. the expression
itself, rather than the value denoted by the expression. Since it is passed as an
expression, there is no need to evaluate the expression to obtain its value. The
expression will be evaluated when the value is required by the function body.
As you can see, CBN delays the computation of arguments by passing them as
expressions and can be considered as lazy evaluation. We will define \lang, which
is a lazy version of \plang, in this chapter. \lang adopts the CBN semantics. In
addition, we will introduce call-by-need,\sidenote{Some people use the term lazy
evaluation to denote call-by-need only. In that sense, CBN is not considered as
lazy evaluation. However, this book views lazy evaluation as a term that can be
used broadly to mean any form of delayed computation. In this sense, both CBN
and call-by-need belong to lazy evaluation.} which is another form of lazy
evaluation, as an optimized version of CBN.

Before we discuss the CBN semantics and \lang, let us see why lazy evaluation is
valuable in practice. We can find lazy evaluation in a few real-world languages.
Haskell is well-known as treating every computation lazily by
default.\sidenote{Programmers can force evaluation if they really want.} On the
other hand, Scala is an eager language but allows programmers to selectively
apply lazy evaluation to their programs. We will see code examples in Scala as
it is the language used in this book, though Haskell is the most famous lazy
language. Consider the following Scala code:

\begin{verbatim}
def square(x: Int): Int = {
  Thread.sleep(5000)  // models some expensive computation
  x * x
}

def foo(x: Int, b: Boolean): Int =
  if (b) x else 0

val file = new java.io.File("a.txt")
foo(square(10), file.exists)
\end{verbatim}

The function \code{square} takes an integer as an argument and returns its
square. It always takes five seconds to return due to \code{Thread.sleep(5000)},
which makes the thread sleep for five seconds. Of course, no one will write
such code in practice, but it is an analogy of highly expensive computation that
takes a long time.

The function \code{foo} takes one integer and one boolean. It
returns the integer when the boolean is true and zero otherwise. Therefore, the
integer value is required only when the boolean is true.

The last line of the program applies \code{foo} to \code{square(10)} and
\code{file.exists}. The second argument is true if and only
if there exists a file named \code{a.txt}. If the file does not exist, \code{foo}
returns zero, and thus the value of \code{square(10)} is unnecessary.
However, as Scala uses eager evaluation by default, \code{square(10)} is
evaluated and spends five seconds regardless of the existence of the file.
If we modify the program not to evaluate \code{square(10)} when the file is
absent, we can save time in many cases without changing the behavior of the
program.

Lazy evaluation gives us an easy solution to this issue. If the first argument
for \code{foo} is evaluated lazily, the program will evaluate
\code{square(10)} only when the file exists. In Scala, we can make a certain
parameter use the CBN semantics by adding \code{=>} in front of the type of the
parameter. Thus, the following fix completely resolves the problem:

\begin{verbatim}
def foo(x: => Int, b: Boolean): Int =
  if (b) x else 0
\end{verbatim}

We call \code{x} a by-name parameter in Scala. Since \code{x} is a by-name
parameter, the first argument for \code{foo} is evaluated only when the value of
\code{x} is needed during the evaluation of the body.

\section{Semantics}

We do not explain the syntax of \lang as it is the same as \plang.
We can move on to the semantics immediately. The definition of an environment is
the same as \plang. Also, as in \plang, $\evald{e}{v}$ is true if and only if
\evaldn{e}{v}. We can use Rule \textsc{Num} of \lang since evaluation of
integers are not affected by lazy evaluation. Similarly, Rule \textsc{Fun} can
be reused as well. Rule \textsc{Id} also remains the same.
The value of an identifier can be found in an environment.

One that certainly requires a change is the semantics of function applications.
It is the most distinctive feature of lazy languages compared to eager
languages. In the CBV semantics, we store the values of arguments in
environments and use the environments to evaluate function bodies. We still need
to store arguments in environments in the CBN semantics. However, arguments are
passed as expressions, and expressions are not values. We need a way to put
expressions in environments. The simplest solution is to define a new kind of
values as follows:
\sidenote{We omit the common part to \plang.}

\[ v\ ::=\ \cdots\ |\ \exprv{e}{\sigma} \]

$\exprv{e}{\sigma}$ is an expression as a value; we call it an expression-value.
It denotes that the computation
of $e$ under $\sigma$ has been delayed. We must keep the environment together
with the expression since the expression can have free identifiers whose values
are available in the environment. The reason is the same as why we need the
notion of a closure, which is a function with an environment. The structure of
$\exprv{e}{\sigma}$ is quite similar to the structure of a closure,
$\clov{x}{e}{\sigma}$, except that an expression-value lacks the name of a
parameter. The similarity is not just a coincidence. Both kinds of values denote
delay of computation. The evaluation of $e$ in $\exprv{e}{\sigma}$ is postponed
until the value of the argument becomes required, and the evaluation of $e$ in
$\clov{x}{e}{\sigma}$ is postponed until the closure is applied to a value.

Because of the addition of expression-values, we need to define another form of
evaluation. We call it strict evaluation. The purpose of strict evaluation is to
force an expression-value to be a ``normal'' value, which is an integer or a
closure. Strict evaluation is required because the ability of an
expression-value is limited. It can be passed as an argument or stored in an
environment like normal values but cannot be used as an operand of an arithmetic
expression or applied to a value as a function. There must be a way to convert
an expression-value to a normal value, and strict evaluation takes this role.

Strict evaluation is defined as a binary relation over $V$ and $V$. We use
$\Downarrow$ to denote strict evaluation.

\[ \Downarrow\subseteq V\times V \]

$\stricte{v_1}{v_2}$ is true if and only if $v_1$ strictly evaluates to $v_2$.
Here, $v_1$ can be any value. However, $v_2$ cannot be an expression-value; it
must be a normal value. The reason obviously comes from the purpose of strict
evaluation: converting an expression-value to a normal value.

The following rules define strict evaluation of normal values:

\semanticrule{Strict-Num}{
  \stricten{n}{n}
}

\vspace{-1em}

\[
  \stricte{n}{n}
  \quad\textsc{[Strict-Num]}
\]

\vspace{-1em}

\semanticrule{Strict-Clo}{
  \stricten{\clov{x}{e}{\sigma}}{\clov{x}{e}{\sigma}}
}

\vspace{-1em}

\[
  \stricte{\clov{x}{e}{\sigma}}{\clov{x}{e}{\sigma}}
  \quad\textsc{[Strict-Clo]}
\]

A normal value strictly evaluates to itself since it is already a normal value.

The following rule defines strict evaluation of expression-values:

\semanticrule{Strict-Expr}{
  If \evaldn{e}{v_1}, and \stricten{v_1}{v_2},
  then \stricten{\exprv{e}{\sigma}}{v_2}.
}

\vspace{-1em}

\[
  \inferrule
  { \evald{e}{v_1} \\ \stricte{v_1}{v_2} }
  { \stricte{\exprv{e}{\sigma}}{v_2} }
  \quad\textsc{[Strict-Expr]}
\]

An expression-value $\exprv{e}{\sigma}$ is strictly evaluated by evaluating $e$
under $\sigma$. The result of $e$ can be an expression-value again, and thus we
need repeated strict evaluation until reaching a normal value.

Now, we can define the semantics of function applications by using the notions of
expression-values and strict evaluation.

\semanticrule{App}{
  \begin{tabular}{@{\hskip0pt}l@{\hskip-10pt}l}
    If \\
    & \evaldn{e_1}{v_1}, \\
    & \stricten{v_1}{\clov{x}{e}{\sigma'}}, and \\
    & \evaln{\sigma'[x\mapsto\exprv{e_2}{\sigma}]}{e}{v}, \\
    then \\
    & \evaldn{\eapp{e_1}{e_2}}{v}.
  \end{tabular}
}

\vspace{-1em}

\[
  \inferrule
  {
    \evald{e_1}{v_1} \\
    \stricte{v_1}{\clov{x}{e}{\sigma'}} \\
    \eval{\sigma'[x\mapsto\exprv{e_2}{\sigma}]}{e}{v}
  }
  { \evald{\eapp{e_1}{e_2}}{v} }
  \quad\textsc{[App]}
\]

$e_1$ evaluates to $v_1$ first. $v_1$ may be an expression-value, while we need a
closure. Therefore, we strictly evaluate $v_1$ to get a closure. On the other
hand, in the CBN semantics, the argument must not be evaluated before the
evaluation of the function body. Instead of
evaluating $e_2$, we make an expression-value with $e_2$ and $\sigma$ and then
put the value into the environment.

Let us see the semantics of addition and subtraction.

\semanticrule{Add}{
  \begin{tabular}{@{\hskip0pt}l@{\hskip-10pt}l}
    If \\
    & \evaldn{e_1}{v_1}, \\
    & \stricten{v_1}{n_1}, \\
    & \evaldn{e_2}{v_2}, and\\
    & \stricten{v_2}{n_2}, \\
    then \\
    & \evaldn{\eadd{e_1}{e_2}}{n_1+n_2}.
  \end{tabular}
}

\vspace{-1em}

\[
  \inferrule
  {
    \evald{e_1}{v_1} \\
    \stricte{v_1}{n_1} \\
    \evald{e_2}{v_2} \\
    \stricte{v_2}{n_2} \\
  }
  { \evald{\eadd{e_1}{e_2}}{n_1+n_2} }
  \quad\textsc{[Add]}
\]

\vspace{-1em}

\semanticrule{Sub}{
  \begin{tabular}{@{\hskip0pt}l@{\hskip-10pt}l}
    If \\
    & \evaldn{e_1}{v_1}, \\
    & \stricten{v_1}{n_1}, \\
    & \evaldn{e_2}{v_2}, and\\
    & \stricten{v_2}{n_2}, \\
    then \\
    & \evaldn{\esub{e_1}{e_2}}{n_1-n_2}.
  \end{tabular}
}

\vspace{-1em}

\[
  \inferrule
  {
    \evald{e_1}{v_1} \\
    \stricte{v_1}{n_1} \\
    \evald{e_2}{v_2} \\
    \stricte{v_2}{n_2} \\
  }
  { \evald{\esub{e_1}{e_2}}{n_1-n_2} }
  \quad\textsc{[Sub]}
\]

There is nothing difficult. They are similar to the rules of \plang but
additionally require strict evaluation since addition and subtraction are
possible only by using integers, not expression-values.

The semantics is a correct instance of CBN but has a flaw from a practical
perspective. Consider $\eapp{(\efun{\cx}{\cx})}{(\eadd{1}{1})}$. It results in
$\exprv{\eadd{1}{1}}{\emptyset}$, not $2$. Most programmers are likely to prefer
$2$ as a result. We need to apply one last strict evaluation at the end of the
evaluation to resolve the problem. It is to say that ``the result of a program
$e$ is $v$ when $\evale{e}{v'}$ and $\stricte{v'}{v}$.'' Note that it is
different from applying strict evaluation to the evaluation of every expression
in the program. Strict evaluation is applied to only the result of the whole
expression, which is the program. In this way, we can make the result of the
above expression $2$ and eliminate the flaw.\sidenote{It is not a flaw in real-world
programming languages like Haskell. A program shows its result by output operations
(e.g. to files) rather than the value of a single expression. Each output
operation applies strict evaluation to its argument (like Rule \textsc{Add},
Rule \textsc{Sub}, and Rule \textsc{App} in \lang), and the value of
each expression does not need to be a normal value.}

If evaluating an expression in the CBV semantics results in a value,
then the CBN semantics yields the same value. It is known as a corollary of the
standardization theorem of lambda calculus~\cite{theories-of-pl}.
Note that it is true only in languages without side effects.
The result of an expression with side effects varies in the order of
the evaluation. For example, if an argument is an expression changing the value
of a box, and the body of the function reads the value of the box without using
the argument, the program can behave differently in CBV and CBN. In CBV, the
read value will be the value after the update. On the other hand, in CBN, the
update never happens, and the read value will be the original value of the box.

While the CBN semantics preserves the results of the CBV semantics,
the converse is false even without mutation, i.e. there are expressions that
yield results only in CBN.
For instance, consider a function application whose argument is a nonterminating
expression. If the function returns zero without using the argument, evaluation
with CBN results in zero, while evaluation with CBV does not terminate.

\section{Interpreter}

We need to add a new variant to \code{Value} to represent expression-values.
\sidenote{We omit the common part to \plang.}

\begin{verbatim}
sealed trait Value
...
case class ExprV(e: Expr, env: Env) extends Value
\end{verbatim}

\code{ExprV($e$, $\sigma$)} represents $\exprv{e}{\sigma}$.

The following function implements strict evaluation:

\begin{verbatim}
def strict(v: Value): Value = v match {
  case ExprV(e, env) => strict(interp(e, env))
  case _ => v
}
\end{verbatim}

We can implement \code{interp} as follows:
\sidenote{We omit the common part to \plang.}

\begin{verbatim}
def interp(e: Expr, env: Env): Value = e match {
  ...
  case Add(l, r) =>
    val NumV(n) = strict(interp(l, env))
    val NumV(m) = strict(interp(r, env))
    NumV(n + m)
  case Sub(l, r) =>
    val NumV(n) = strict(interp(l, env))
    val NumV(m) = strict(interp(r, env))
    NumV(n - m)
  case App(f, a) =>
    val CloV(x, b, fEnv) = strict(interp(f, env))
    interp(b, fEnv + (x -> ExprV(a, env)))
}
\end{verbatim}

Each case matches the corresponding rule, so there is nothing difficult.

\section{Call-by-Need}

The current implementation is efficient when a parameter appears once or less in
the function body. However, using a parameter twice or more leads to redundant
calculation. Consider the following Scala program:

\begin{verbatim}
def square(x: Int): Int = {
  Thread.sleep(5000)  // models some expensive computation
  x * x
}

def bar(x: => Int, b: Boolean): Int =
  if (b) x + x else 0

val file = new java.io.File("a.txt")
bar(square(10), file.exists)
\end{verbatim}

\code{x} appears twice in the body of \code{bar}. If \code{a.txt} exists, \code{square(10)} is
evaluated twice. Actually, we do not need to evaluate \code{square(10)} twice
since its result is always the same. Since \code{square} is an expensive
function, it is desirable to reduce the number of function calls as much as possible.
If we use CBV instead of CBN, it is possible to evaluate \code{square(10)} only
once when the file exists.
However, going back to CBV is not a good choice. It will make the program
evaluate \code{square(10)} even when the file does not exist.

The way to solve this problem is to store the value of an argument and use the
value again. This strategy is as optimal as CBV when a parameter appears
multiple times; it is as optimal as CBN when a parameter is not used at all.
For programmers, it is tedious to implement such logic in their programs by
themselves. Instead, programming languages can provide the optimization. This
optimization is called \textit{call-by-need}\index{call-by-need} as each
argument is evaluated based on need for its value. It is evaluated once if
needed and is not otherwise.

Call-by-need is not different semantics from CBN in purely functional languages.
The behaviors of a program in call-by-need and CBN are completely equal.
Call-by-need is just an optimization strategy of interpreters and compilers. On
the other hand, call-by-need is different semantics from CBN in languages with
side effects. In such languages, the number of computation of a certain
expression can affect the result. For example, consider an argument that is an expression
that increases the value of the box by one. Suppose that its value is used twice
in the function body. Then, the value of the box increases by two in CBN, while
it increases by one in call-by-need.

Since \lang lacks side effects, we can adopt call-by-need to the language as
optimization of the interpreter. There is no need to newly define the call-by-need
version of the semantics.

To store the strict value of an expression-value, we add a new field to
the class \code{ExprV}.

\begin{verbatim}
case class ExprV(
  e: Expr, env: Env, var v: Option[Value]
) extends Value
\end{verbatim}

The field is declared as mutable. Initially,
the value of the expression is unknown, and \code{v} equals \code{None}. When
the value is calculated for the first time, the value is stored in \code{v}.
The fact that \code{v} equals \code{Some(a)} for some \code{a} implies that the
value of the expression is \code{a}. In the next time we need the value again,
\code{a} can be used without any redundant computation.

The function \code{strict} requires the following change:

\begin{verbatim}
def strict(v: Value): Value = v match {
  case ExprV(_, _, Some(cache)) => cache
  case ev @ ExprV(e, env, None) =>
    val cache = strict(interp(e, env))
    ev.v = Some(cache)
    cache
  case _ => v
}
\end{verbatim}

It checks whether there exists a cached value. If it is the case, the function
simply returns the cached value. Otherwise, \code{e} is evaluated under
\code{env} like before. In addition, the function stores the value in \code{v}.

The function \code{interp} needs only one fix. When a new \code{ExprV} instance
is created in the \code{App} case, one additional argument is required to
initialize the field \code{v}.

\begin{verbatim}
case App(f, a) =>
  val CloV(x, b, fEnv) = strict(interp(f, env))
  interp(b, fEnv + (x -> ExprV(a, env, None)))
\end{verbatim}

Since we do not know the value of \code{a}, the initial value of \code{v} is \code{None}.

Purely functional languages with lazy evaluation usually adopt call-by-need
because it is just optimization but not a change in their semantics. On the
other hand, impure languages cannot consider call-by-need as optimization and
often allow programmers to choose one of them at each place. For example, Scala
uses CBN for by-name parameters and call-by-need for lazy variables. We can
define lazy variables with the \code{lazy} modifier.

\begin{verbatim}
lazy val x = {
  println(1)
  1
}
val y = x + x
\end{verbatim}

The program prints \code{1} only once. By using both by-name parameter and lazy
variable, we can simulate the call-by-need semantics in Scala.

\begin{verbatim}
def bar(_x: => Int, b: Boolean): Int = {
  lazy val x = _x
  if (b) x + x else 0
}
\end{verbatim}

If \code{b} is true, the first argument is evaluated only once. Otherwise, it is
not evaluated at all.

\section{Exercises}

\begin{enumerate}
\item Which of the following produce different results in a
 CBV language and a CBN language? Both produce the
 same result if they both produce the same number or they both produce
 closures (even if they do not behave exactly the same
 when applied).

\begin{enumerate}
  \item $\eapp{(\efun{\cy}{\eapp{\cy}{3}})}{(\efun{\cx}{\eapp{1}{2}})}$
  \item
    $\eapp{(\efun{\cy}{\eapp{\cy}{\efun{\cx}{10}}})}{(\efun{\cx}{\eapp{\cx}{(\eapp{1}{2})}})}$
  \item
    $\eapp{(\efun{\cy}{\eapp{\cy}{\efun{\cx}{10}}})}{(\efun{\cx}{\eapp{1}{2}})}$
  \item
    $\eapp{(\efun{\cy}{\cy})}{(\eadd{1}{\efun{\cx}{\cx}})}$
  \item
    $\eapp{(\efun{\cy}{\eadd{1}{2}})}{(\eadd{1}{\efun{\cx}{\cx}})}$
\end{enumerate}

\item Show the results of each expression in a CBV language and a CBN language.

\begin{enumerate}
  \item $\eadd{(\efun{\cx}{8})}{10}$
  \item $\eapp{(\efun{\cx}{8})}{(\eapp{1}{2})}$
  \item $\efun{\cx}{(\eapp{(\efun{\cy}{42})}{(\eapp{9}{2})})}$
  \item
    $\eadd{1}{(\eapp{(\efun{\cx}{\eadd{\cx}{13}})}{(\eadd{1}{\efun{\cy}{7}})})}$
  \item
    $\eadd{1}{(\eapp{(\efun{\cx}{\eadd{1}{13}})}{(\eadd{1}{\efun{\cy}{7}})})}$
\end{enumerate}

\item Note that there is a recursive call in the following function:

\begin{verbatim}
def strict(v: Value): Value = v match {
  case ev @ ExprV(e, env, None) =>
    val cache = strict(interp(e, env))
    ev.v = Some(cache)
    cache
  case ExprV(_, _, Some(cache)) => cache
  case _ => v
}
\end{verbatim}

Write an example \lang expression showing the need for the recursive \code{strict} call.

\item Consider the following expression:
\[
\begin{array}{l}
  \ebind{\cf}{\efun{\cx}{\eadd{\cy}{7}}}{\\
  \ebind{\cy}{5}{\\
  \eapp{\cf}{(\eadd{42}{\efun{\cy}{3}})}
  }
  }
\end{array}
\]

Explain the result of evaluating it under the following semantics:
\begin{enumerate}
  \item Lazy evaluation with static scope
  \item Lazy evaluation with dynamic scope
  \item Eager evaluation with static scope
  \item Eager evaluation with dynamic scope
\end{enumerate}

\item For each of the following \lang expressions, show how it is evaluated and its result.

\begin{enumerate}
  \item
    $\eapp{(\efun{\cx}{\eapp{\cx}{8}})}{(\eadd{\eapp{42}}{\efun{\cy}{\esub{\cy}{\cx}}})}$
  \item
    $\eapp{(\efun{\cy}{\eapp{(\efun{\cx}{\eadd{3}{\cx}})}{\cy})}}{5}$
\end{enumerate}

\item This exercise extends \lang with \textsf{val} and \textsf{if0}.  Consider
  the following interpreter implementation:

\begin{verbatim}
sealed trait Expr
...
case class Val(x: String, e: Expr, b: Expr) extends Expr
case class If0(c: Expr, t: Expr, f: Expr) extends Expr

def strict(v: Value): Value = ...

def interp(e: Expr, env: Env): Value = e match {
  ...
  case Val(x, e, b) => ???
  case If0(c, t, f) => ???
}
\end{verbatim}

Like in \plang, the semantics of $\ebind{x}{e_1}{e_2}$ must be the same as
$\eapp{(\efun{x}{e_2})}{e_1}$.  Like in \textsf{RFAE}, $\eifz{e_1}{e_2}{e_3}$
evaluates $e_2$ when $e_1$ evaluates to $0$ and evaluates $e_3$ when $e_1$
evaluates to a nonzero integer or a closure.

Fill every \code{???} to complete the interpreter.

\item This exercise extends \lang with pairs. Consider the following interpreter
  implementation:

\begin{verbatim}
sealed trait Expr
...
case class Pair(f: Expr, s: Expr) extends Expr
case class Fst(e: Expr) extends Expr
case class Snd(e: Expr) extends Expr

sealed trait Value
...
case class PairV(???) extends Value

def strict(v: Value): Value = ...

def interp(e: Expr, env: Env): Value = e match {
  ...
  case Pair(f, s) => ???
  case Fst(e) => ???
  case Snd(e) => ???
}
\end{verbatim}

The semantics of pairs is as follows:
\begin{itemize}
  \item \code{Pair($e_1$, $e_2$)} corresponds to $(e_1, e_2)$, which is an
    expression that creates a new pair.
  \item \code{Fst($e$)} corresponds to $e\textsf{.1}$, which is an expression
    that gives the first value of a given pair.
  \item \code{Snd($e$)} corresponds to $e\textsf{.2}$, which is an expression
    that gives the second value of a given pair.
\end{itemize}

Pairs in this language are lazy, which means that $e_1$ and $e_2$ in
$(e_1, e_2)$ are not evaluated when the pair is created.
Each of them is evaluated only when its value is needed. For example,
\begin{itemize}
  \item $(3, \eadd{(\efun{\cx}{\cx})}{4})$ does not incur a run-time error.
  \item $\eadd{(3, \eadd{(\efun{\cx}{\cx})}{4})\textsf{.1}}{5}$ does not incur a run-time error.
  \item $\eadd{(3, \eadd{(\efun{\cx}{\cx})}{4})\textsf{.2}}{5}$ incurs a run-time error.
\end{itemize}

Fill every \code{???} to complete the interpreter.

\item This exercise extends \lang with lists. Consider the following interpreter
  implementation:

\begin{verbatim}
sealed trait Expr
...
case object Nil extends Expr
case class Cons(h: Expr, t: Expr) extends Expr
case class Head(e: Expr) extends Expr
case class Tail(e: Expr) extends Expr

sealed trait Value
...
case object NilV extends Value
case class ConsV(???) extends Value

def strict(v: Value): Value = ...

def interp(e: Expr, env: Env): Value = e match {
  ...
  case Nil => ???
  case Cons(h, t) => ???
  case Head(e) => ???
  case Tail(e) => ???
}
\end{verbatim}

The semantics of lists is as follows:
\begin{itemize}
  \item \code{Nil} creates the empty list.
  \item \code{Cons($e_1$, $e_2$)} corresponds to $e_1::e_2$, which is an
    expression that creates a nonempty list. Since lists are lazy, $e_1$ and $e_2$ in
    $e_1::e_2$ are not evaluated when the list is created. Each of
    them is evaluated only when its value is needed.
  \item \code{Head($e$)} corresponds to $e\textsf{.head}$, which is an expression
    that gives the head of a list.
  \item \code{Tail($e$)} corresponds to $e\textsf{.tail}$, which is an expression
    that gives the tail of a list. The tail must be a list.
\end{itemize}
For example,
\begin{itemize}
  \item $((\eadd{0}{\textsf{Nil}})::(2::\textsf{Nil}))\textsf{.tail.head}$ does not incur a run-time error.
  \item $((\eadd{0}{\textsf{Nil}})::(2::\textsf{Nil}))\textsf{.head}$ incurs a run-time error.
  \item $(0::1)\textsf{.head}$ does not incur a run-time error.
  \item $(0::1)\textsf{.tail}$ incurs a run-time error.
\end{itemize}

Fill every \code{???} to complete the interpreter.  You may use the helper
function \verb!def isList(v: Value): Boolean!, which returns \code{true} if a
given value is \code{ConsV} or \code{NilV} and \code{false} otherwise, without
defining it.

\end{enumerate}

\setchapterpreamble[u]{\margintoc}
\chapter{Exercises for Untyped Languages}
\labch{untyped-exercises}

\newcommand{\mtt}[1]{\texttt{#1}}
\newcommand{\valnp}{\ensuremath{\embox{val}}}
\newcommand{\Space}[1]{\vspace{#1ex}}

\section{Pattern Matching}
Consider the following language $e$:
\[
\begin{array}{rll}
e ::= & a& \mbox{atomic expression}\\
\mid& e\ a& \mbox{function application}\\
\mid& \verb!fn!\ m& \mbox{function expression}\\
a ::= & n & \mbox{number}\\
\mid&x & \mbox{identifier}\\
m ::= & p\ \leadsto\ e & \mbox{pattern matching}\\
\mid& p\ \leadsto\ e\ \verb!|!\ m & \mbox{pattern matching sequence}\\
p ::= & \verb!_! & \mbox{wildcard pattern}\\
\mid&n& \mbox{number pattern}\\
\mid&x& \mbox{identifier pattern}
\end{array}
\]
where a value of the language $v$ is either a number $n$ or a closure $\langle m, \sigma\rangle$,
a result of evaluation $r$ is either a value $v$ or a failure in pattern matching $\uparrow$,
which is different from run-time errors,
and an environment $\sigma$ maps identifiers to their values.

The operational semantics rules for expressions and atomic expressions are as follows:

\fbox{$\sigma\vdash e \Rightarrow r$}
\[
\begin{array}{c}
\derive{\sigma \vdash a \Rightarrow v}{
\sigma \vdash a \hookrightarrow v
}
\qquad
\derive{\sigma \vdash e\ a \Rightarrow \uparrow}
{\begin{array}{c}
\sigma \vdash e \Rightarrow \uparrow
\end{array}
}
\qquad
\derive{\sigma \vdash e\ a \Rightarrow v'}
{\begin{array}{c}
\sigma \vdash e \Rightarrow \langle m, \sigma' \rangle \quad
\sigma \vdash a \Rightarrow v \quad
(\sigma', v) \vdash m \Rightarrow v'
\end{array}
}
\\[1.5em]
\derive{\sigma \vdash e\ a \Rightarrow \uparrow}
{\begin{array}{c}
\sigma \vdash e \Rightarrow \langle m, \sigma' \rangle \quad
\sigma \vdash a \Rightarrow v \quad
(\sigma', v) \vdash m \Rightarrow \uparrow
\end{array}
}
\qquad
\sigma \vdash \mtt{fn}\ m \Rightarrow \langle m, \sigma \rangle
\end{array}
\]

\fbox{$\sigma\vdash a \hookrightarrow v$}
\[
\begin{array}{c}
\sigma \vdash n \hookrightarrow n
\hspace*{10em}
\derive{\sigma \vdash x \hookrightarrow \sigma(x)}{
x \in \emph{Domain}(\sigma)
}
\end{array}
\]

The semantics of pattern matching $m$ and pattern $p$ are as follows:
\begin{itemize}
\item Evaluation of $p\ \leadsto\ e$ under $(\sigma, v)$ has two possibilities.
First, when evaluation of $p$ results in a new environment $\sigma'$,
the result of this pattern matching is the result of evaluation of $e$ under $\sigma+\sigma'$,
where $\sigma+\sigma'$ is a disjoint union of $\sigma$ and $\sigma'$.
Second, when evaluation of $p$ produces $\uparrow$,
the evaluation of this pattern matching produces $\uparrow$ as well.
\item Evaluation of ``$p\ \leadsto\ e\ \verb!|!\ m$'' under $(\sigma, v)$
also has two possibilities.
First, when evaluation of $p\ \leadsto\ e$ succeeds with a value $v'$,
the value of this pattern matching sequence is $v'$.
Second, when evaluation of $p\ \leadsto\ e$ fails,
the result of evaluation of this pattern matching sequence is
the result of evaluation of $m$.
\item Evaluation of the wildcard pattern \verb!_! under $(\sigma, v)$
produces the empty environment.
\item Evaluation of the number pattern $n$ under $(\sigma, v)$ has two possibilities.
If $\valnp(n)=v$ where $\valnp(n)$ returns a number $n$ for a given number pattern $n$, 
it produces the empty environment.  Otherwise, it produces $\uparrow$.
\item Evaluation of the identifier pattern $x$ under $(\sigma, v)$
produces a singleton environment $\{x \mapsto v\}$
if $x$ is not in the domain of $\sigma$.
\end{itemize}
Write the operational semantics for $m$ and $p$
of the forms \fbox{$(\sigma, v)\vdash m \Rightarrow r$} and
\fbox{$(\sigma, v)\vdash p \Rightarrow \sigma/\uparrow$}, respectively,
where \fbox{$(\sigma, v)\vdash p \Rightarrow \sigma/\uparrow$} denotes
\fbox{$(\sigma, v)\vdash p \Rightarrow \sigma$} or
\fbox{$(\sigma, v)\vdash p \Rightarrow \uparrow$}.
Remember that the operational semantics do not specify run-time errors.

\section{JavaScript Function Calls}

\newcommand{\cmd}[1]{\texttt{#1}}
\newcommand{\expr}{e}
\newcommand{\vname}{x}
\newcommand{\pname}{f}
\newcommand{\addr}{a}
\newcommand{\val}{v}
\newcommand{\sto}{M}
\newcommand{\seq}[1]{\overline{#1}}
\newcommand{\semrule}[5]{\ensuremath{#1, #2 \vdash #3 \Rightarrow #4, #5}}
%\newcommand{\derive}[2]{\ensuremath{\begin{array}{c}\infer{#1}{#2}\end{array}}}
\newcommand{\rulesep}{\qquad}
\newcommand{\seqn}{\texttt{seqn}}
\newcommand{\fun}{\texttt{fun}}
\newcommand{\app}{\texttt{app}}
\newcommand{\rec}{\texttt{rec}}
\newcommand{\get}{\texttt{get}}
\newcommand{\set}{\texttt{set}}
\newcommand{\fin}{\texttt{fin}}
\newcommand{\env}{\sigma}
\newcommand{\ccmd}[1]{\texttt{[}#1\texttt{]}}
\newcommand{\cenc}[1]{\texttt{\{}#1\texttt{\}}}
\newcommand{\args}{\ensuremath{\Lambda}}

Consider the following language $\expr$:
\begin{equation*}
\begin{array}{llll}
\expr & ::= 
& n \\
& \mid& \vname \\
& \mid& \verb!{fun {! x^* \verb+} +e\verb+}+\\
& \mid& \verb!{! e\ e^*\verb+}+\\
& \mid & \verb!{get !e\verb!}!
\end{array}
\end{equation*}
where $n$ denotes a number and
a value of the language is one of the \verb!undefined! value, a number,
or a closure $\langle\lambda \vname_1\ \cdots\ \vname_n. \expr,\env\rangle$, and
an environment maps names to values:
\begin{equation*}
\begin{gathered}
\begin{array}{r@{~}lr@{~}l@{~}l}
x \in & \embox{Name} &
\qquad\qquad
  \val \in & \embox{Value} & = \{\verb!undefined!\} + \embox{Number} + \embox{Closure}\\
n \in & \embox{Number} &
  \sigma \in & \embox{Env} & = \embox{Name} \finto \embox{Value}\\
\langle\lambda x_1\ \cdots\ x_n.\expr, \sigma\rangle \in & \embox{Closure} &
  \alpha \in & \embox{Array} & = \embox{Number} \finto \embox{Value} \\
\end{array}
\end{gathered}
\end{equation*}

The semantics of some constructs are as follows:
\begin{itemize}
\item The value of a function expression $\verb!{fun {! x_1 \cdots x_n\verb+} +e\verb+}+$
at an environment $\sigma$ is a closure $\langle\lambda \vname_1\ \cdots\ \vname_n. \expr,\env\rangle$.
\item A function application $\verb!{! e_0\ \cdots\ e_n\verb+}+$ is evaluated as follows:
\begin{itemize}
\item Evaluate the subexpressions in order.
The value of $e_0$ should be a closure 
$\langle\lambda \vname_1\ \cdots\ \vname_m. \expr,\env\rangle$
that has $m$ parameters.
\item Create an array $\alpha$ of size $n$ and
initialize the $i$-th value of the array with the value of $e_{i+1}$ where $0 \le i \le n-1$.
\item Evaluate the closure body $e$ under the environment $\sigma$
extended as follows:
\begin{itemize}
\item The value of the $i$-th parameter is the value of $e_i$ where $1 \le i \le m \le n$.
\item The value of the $j$-th parameter is the \verb!undefined! value
where $n < j \le m$.
\end{itemize}
and the array $\alpha$.
\end{itemize}

%   \item The value of an array $\alpha$ is a finite map,
% which maps $\pname_i \in \ccmd{\pname_1\ \cdots,\ \pname_k}$
% to the value $v_i$ evaluated from the expression $e_i$.

  \item The value of $\verb!{get !e\verb!}!$ is the $n$-th value of the array $\alpha$
where $n$ is the value of $e$ and the array indices start from $0$.
\end{itemize}

For example,
$\cenc{\cenc{\texttt{fun}\ \cenc{x\ y}\ y}\ 4}$
evaluates to \verb!undefined!, and
$\cenc{\cenc{\texttt{fun}\ \cenc{x}\ \cenc{\texttt{get}\ 0}}\ 5}$
evaluate to $5$.

\begin{itemize}
  \item[a)]
Write the operational semantics of the form
$\boxed{\sigma, \alpha \vdash \expr \Rightarrow \val}$


  \item[b)] Write the evaluation derivation of the following expressions:

\hspace*{-5em}
\derive
{\hspace*{\textwidth}}
{\emptyset, \emptyset\vdash \cenc{\cenc{\texttt{fun}\ \cenc{x\ y}\ \cenc{\texttt{get}\ x}}\ 2\ 19\ 141}
\Rightarrow~~~~~~~~}
\end{itemize}

\section{JavaScript Sequencing}

The following quote describes the JavaScript sequencing semantics:
\begin{quote}
The value of a \embox{StatementList} is the value of the last
value-producing them in the \embox{StatementList}.  For example, the
following calls to the \verb!eval! function all return the value 1:
\begin{verbatim}
eval("1;;;;;")
eval("1;()")
eval("1;var a;")
\end{verbatim}
\end{quote}
Consider the following language $e$:
\[
\begin{array}{rlllr@{~}c@{~}ll r@{~}c@{~}l}
e ::= & \verb!()! & \texttt{Void}
&\quad& e &\in& \texttt{Exp}
&\quad& v \in \texttt{Val} &=& \texttt{Closure} \cup \lbrace \texttt{()} \rbrace\\
\mid& x & \texttt{Id}
&&  x &\in& \texttt{Var}
&&\sigma \in \texttt{Env} &=& \texttt{Var} \xrightarrow{\texttt{fin}} \texttt{Val}\\
\mid& \verb!{fun {! x\verb+} +e\verb+}+ & \texttt{Fun}
&&&&
&&\langle \lambda x. e, \sigma\rangle \in \texttt{Closure} &=& \texttt{Exp} \times \texttt{Env}\\
\mid& \verb!{!e\ e\verb+}+ & \texttt{App}\\
\mid& \verb!{!e\verb!; !\cdots\ \verb!; ! e\verb+}+ & \texttt{Seq}\\
\end{array}
\]
where evaluation of $e$ results in either a closure
$\langle \lambda x.e, \sigma\rangle$ or the void \verb!()!.
The value of the sequence expression $\verb!{!e_1\verb!; !\cdots\ \verb!; ! e_n\verb+}+$
is the value of the last expression whose value is not \verb!()!.
If the values of all the expressions $e_1, \cdots, e_n$ are \verb!()!,
the value of the sequence expression is \verb!()!.
Write the operational semantics of each expression of the form
$\sigma \vdash e \Rightarrow v$. 

\begin{itemize}
  \item \underline{$\texttt{()}$}:
  \item \underline{$x$}:
  \item \underline{\texttt{\string{fun \string{$x$\string} $e$\string}}}:
  \item \underline{\texttt{\string{$e$ $e$\string}}}:
  \item \underline{\texttt{\string{$e$; $\cdots$ ; $e$\string}}}:
\end{itemize}


% \pagelayout{wide} % No margins
% \addpart{Typed Languages}
% \pagelayout{margin} % Restore margins

\appendix % From here onwards, chapters are numbered with letters, as is the appendix convention

\pagelayout{wide} % No margins
\addpart{Appendix}
\pagelayout{margin} % Restore margins

\chapter{Solutions to Exercises}
\labch{solutions}

Many exercises are meant to have multiple solutions. Each provided solution may
not be the only solution.

\textbf{\refex{immutability-student}}
\vspace{-1em}
\begin{verbatim}
def names(l: List[Student]): List[String] = l match {
  case Nil => Nil
  case h :: t => h.name :: names(t)
}
\end{verbatim}

\textbf{\refex{immutability-tall}}
\vspace{-1em}
\begin{verbatim}
def tall(l: List[Student]): List[Student] = l match {
  case Nil => Nil
  case h :: t =>
    if (h.height > 170)
      h :: tall(t)
    else
      tall(t)
}
\end{verbatim}

\textbf{\refex{immutability-length}}
\vspace{-1em}
\begin{verbatim}
def length(l: List[Int]): Int = l match {
  case Nil => 0
  case h :: t => 1 + length(t)
}
\end{verbatim}

\textbf{\refex{immutability-append}}
\vspace{-1em}
\begin{verbatim}
def append(l: List[Int], n: Int): List[Int] = l match {
  case Nil => n :: Nil
  case h :: t => h :: append(t, n)
}
\end{verbatim}
\vspace{-1em}
$O(n)$

\textbf{\refex{functions-incby}}
\vspace{-1em}
\begin{verbatim}
def incBy(l: List[Int], n: Int): List[Int] = l.map(_ + n)
\end{verbatim}

\textbf{\refex{functions-gt}}
\vspace{-1em}
\begin{verbatim}
def gt(l: List[Int], n: Int): List[Int] = l.filter(_ > n)
\end{verbatim}

\textbf{\refex{functions-append}}
\vspace{-1em}
\begin{verbatim}
def append(l: List[Int], n: Int): List[Int] =
  l.foldRight(n :: Nil)(_ :: _)
\end{verbatim}

\textbf{\refex{functions-reverse}}
\vspace{-1em}
\begin{verbatim}
def reverse(l: List[Int]): List[Int] =
  l.foldLeft(Nil: List[Int])((l, e) => e :: l)
\end{verbatim}

\textbf{\refex{syntax-and-semantics-expr}}
\begin{enumerate}
  \item No
  \item No
  \item No
  \item No
  \item Yes
\end{enumerate}

\textbf{\refex{syntax-and-semantics-icecream}}
\begin{enumerate}
  \item No
  \item Yes
  \item No
  \item Yes
  \item No
  \item Yes
  \item Yes
\end{enumerate}

\textbf{\refex{syntax-and-semantics-coffee}}
\begin{enumerate}
  \item No
  \item Yes
  \item Yes
  \item Yes
  \item No
  \item Yes
\end{enumerate}

\textbf{\refex{syntax-and-semantics-pair}}
\begin{enumerate}
  \item
    \[
      \inferrule
      { {e_1}\Rightarrow{v_1}\\{e_2}\Rightarrow{v_2} }
      { {(e_1,e_2)}\Rightarrow{(v_1,v_2)} }
    \]

    \[
      \inferrule
      { {e}\Rightarrow{(v_1,v_2)} }
      { {e\textsf{.1}}\Rightarrow{v_1} }
    \]

    \[
      \inferrule
      { {e}\Rightarrow{(v_1,v_2)} }
      { {e\textsf{.2}}\Rightarrow{v_2} }
    \]
  \item
    \[
      \inferrule
      {
        \inferrule
        {
          {8}\Rightarrow{8} \\
          \inferrule
          {
            \inferrule
            { {320}\Rightarrow{320} \\ {42}\Rightarrow{42} }
            { {(320,42)}\Rightarrow{(320,42)} }
          }
          { {(320,42)\textsf{.1}}\Rightarrow{320} }
        }
        { {(8,(320,42)\textsf{.1})}\Rightarrow{(8,320)} }
      }
      { {(8,(320,42)\textsf{.1})\textsf{.2}}\Rightarrow{320} }
    \]
\end{enumerate}

\textbf{\refex{syntax-and-semantics-record}}
\[
  \inferrule
  { {e_1}\Rightarrow{v_1} \\\cdots\\ {e_n}\Rightarrow{v_n} }
  { {\{l_1:e_1,\cdots,l_n:e_n\}}\Rightarrow{\langle l_1:v_1,\cdots,l_n:v_n\rangle} }
\]

\[
  \inferrule
  { {e}\Rightarrow{\langle\cdots,l:v,\cdots\rangle} }
  { {e.l}\Rightarrow{v} }
\]

\textbf{\refex{syntax-and-semantics-stlist}}
\[
  {\textsf{()}}\Rightarrow{\textsf{()}}
\]

\[
  \inferrule
  { {e_1}\Rightarrow{\textsf{()}}\\ \cdots\\ {e_n}\Rightarrow{\textsf{()}} }
  { {e_1;\cdots;e_n}\Rightarrow{\textsf{()}} }
\]

\[
  \inferrule
  {
    {e_1}\Rightarrow{v_1}\quad\cdots\quad{e_m}\Rightarrow{v_m}\quad
    v_m\not=\textsf{()}\quad
    {e_{m+1}}\Rightarrow{\textsf{()}}\quad\cdots\quad{e_n}\Rightarrow{\textsf{()}}
  }
  { {e_1;\cdots;e_n}\Rightarrow{v_m} }
\]

\textbf{\refex{identifiers-arrow}}

\begin{itemize}
  \item $\ebind{\cx_A}{(\ebind{\cx_B}{3}{\esub{5}{\cx_C}})}{\eadd{1}{\cx_D}}$
  \begin{enumerate}
    \item $C\rightarrow B,D\rightarrow A$
    \item No shadowing
  \end{enumerate}
  \item $\ebind{\cx_A}{3}{\ebind{\cy_B}{5}{\eadd{1}{\cx_C}}}$
  \begin{enumerate}
    \item $C\rightarrow A$
    \item No shadowing
  \end{enumerate}
  \item $\ebind{\cx_A}{3}{\ebind{\cx_B}{5}{\eadd{1}{\cx_C}}}$
  \begin{enumerate}
    \item $C\rightarrow B$
    \item $B\rightarrow A$
  \end{enumerate}
\end{itemize}

\textbf{\refex{identifiers-shadowing-impl}}
\vspace{-1em}
\begin{verbatim}
case Num(n) => Set()
case Add(l, r) => helper(l, env) ++ helper(r, env)
case Id(x) => Set()
case Val(x, e, b) =>
  val s = if (env(x)) Set(x) else Set()
  s ++ helper(e, env) ++ helper(b, env + x)
\end{verbatim}

\textbf{\refex{first-order-functions-eval}}
\begin{enumerate}
  \item Free identifier error
  \item $5$
  \item $3$
  \item Free identifier error
  \item Free identifier error
\end{enumerate}

\textbf{\refex{first-class-functions-trace}}

$\sigma_1=[\cx\mapsto5]$\\
$\sigma_2=\sigma_1[\cf\mapsto v_1]$\\
$\sigma_3=\sigma_2[\code{g}\mapsto v_2]$\\
$\sigma_4=\sigma_2[\cx\mapsto1]$\\
$\sigma_5=\sigma_1[\cy\mapsto1]$\\
$v_1=\clov{\cy}{\eadd{\cy}{\cx}}{\sigma_1}$\\
$v_2=\clov{\cx}{\cx}{\sigma_2}$\\
$
\begin{array}{@{}c|c|c@{}}
  \text{expr} & \text{env} & \text{res} \\\hline
\ebind{\cx}{5}{
   \ebind{\code{f}}{\efun{\cy}{\eadd{\cy}{\cx}}}{
     \eapp{(\efun{\code{g}}{\eapp{\code{f}}{(\eapp{\code{g}}{1})}})}
       {(\efun{\cx}{\cx})}}} & \emptyset & \\
5 & \emptyset & 5 \\
\ebind{\code{f}}{\efun{\cy}{\eadd{\cy}{\cx}}}{
  \eapp{(\efun{\code{g}}{\eapp{\code{f}}{(\eapp{\code{g}}{1})}})}
    {(\efun{\cx}{\cx})}} & \sigma_1 & \\
\efun{\cy}{\eadd{\cy}{\cx}} & \sigma_1 & v_1 \\
\eapp{(\efun{\code{g}}{\eapp{\code{f}}{(\eapp{\code{g}}{1})}})}{(\efun{\cx}{\cx})} & \sigma_2 & \\
\efun{\code{g}}{\eapp{\code{f}}{(\eapp{\code{g}}{1})}} & \sigma_2 &
\clov{\code{g}}{\eapp{\code{f}}{(\eapp{\code{g}}{1})}}{\sigma_2} \\
\efun{\cx}{\cx} & \sigma_2 & v_2 \\
\eapp{\code{f}}{(\eapp{\code{g}}{1})} & \sigma_3 & \\
\cf & \sigma_3 & v_1 \\
\eapp{\code{g}}{1} & \sigma_3 \\
\code{g} & \sigma_3 & v_2 \\
1 & \sigma_3 & 1 \\
\cx & \sigma_4 & 1 \\
\eadd{\cy}{\cx} & \sigma_5 \\
\cy & \sigma_5 & 1 \\
\cx & \sigma_5 & 5 \\
\end{array}
$
\\

\textbf{\refex{first-class-functions-scope}}
\begin{enumerate}
  \item Dynamic scoping
  \item The argument is the only identifier the body can use unless the body defines a new identifier by itself.
  \item Static scoping
\end{enumerate}

\textbf{\refex{first-class-functions-rewrite}}

$\eapp{(\efun{\cx}{\efun{\cy}\eapp{\eapp{\cx}{(\esub{10}{\cy})}}})}{\efun{\cy}{\eadd{8}{\cy}}}$
\\

\textbf{\refex{first-class-functions-rewrite-impl}}
\vspace{-1em}
\begin{verbatim}
case Num(n) => Num(n)
case Id(x) => Id(x)
case Val(x, e, b) => App(Fun(x, desugar(b)), desugar(e))
case Fun(x, b) => Fun(x, desugar(b))
case App(f, a) => App(desugar(f), desugar(a))
\end{verbatim}

\textbf{\refex{first-class-functions-pair}}
\vspace{-1em}
\begin{verbatim}
App(
  App(
    Fun("x", Fun("y", Fun("z",
      App(App(Id("z"), Id("x")), Id("y"))
    ))),
    desugar(f)
  ),
  desugar(s)
)
\end{verbatim}

\textbf{\refex{first-class-functions-closure}}

$\eapp{\eapp{(\efun{\cx}{\efun{\cy}{\cx}})}{0}}{0}$
\\

\textbf{\refex{first-class-functions-terminate}}

$\eapp{1}{1}$
\\

\textbf{\refex{first-class-functions-free}}

$\efun{\cx}{\cy}$
\\

\textbf{\refex{first-class-functions-fv}}
\[
  \inferrule
  { \embox{fv}(e)\subseteq\dom{\sigma}\cup\{x\} }
  { \evald{\efun{x}{e}}{\clov{x}{e}{\sigma}} }
\]

\[
  \inferrule
  { \embox{fv}(e)\not\subseteq\dom{\sigma}\cup\{x\} }
  { \evald{\efun{x}{e}}{\uparrow} }
\]

\textbf{\refex{first-class-functions-mult}}
\begin{enumerate}
  \item
    \[
      \evald{\lambda x_1\cdots x_n. e}{\langle \lambda x_1\cdots x_n.e,\sigma \rangle}
    \]

    \[
      \inferrule
      {
        \evald{e_0}{\langle \lambda x_1\cdots x_n.e,\sigma' \rangle}\\
        \evald{e_1}{v_1}\quad\cdots\quad\evald{e_n}{v_n}\\
        \eval{\sigma'[x_1 \mapsto v_1,\cdots,x_n \mapsto v_n]}{e}{v}
      }
      { \evald{\eappfo{e_0}{e_1,\cdots,e_n}}{v} }
    \]
  \item
    $\sigma_1=[\cf\mapsto{\clov{\cx}{\cx}{\emptyset}},\code{m}\mapsto8],
    \sigma_2=[\cx\mapsto8]$
    \[
      \small
      \inferrule
      {
        \evale{\efun{\code{f}\ \code{m}}{\eappfo{\code{f}}{\code{m}}}}
        {\clov{\code{f}\ \code{m}}{\eappfo{\code{f}}{\code{m}}}{\emptyset}} \\
        \evale{\efun{\cx}{\cx}}{\clov{\cx}{\cx}{\emptyset}} \\
        \evale{8}{8} \\
        \inferrule
        {
          \inferrule
          { \cf\in\dom{\sigma_1} }
          { \eval{\sigma_1}{\code{f}}{\clov{\cx}{\cx}{\emptyset}} }
          \\
          \inferrule
          { \code{m}\in\dom{\sigma_1} }
          { \eval{\sigma_1}{\code{m}}{8} }
          \\
          \inferrule
          { \cx\in\dom{\sigma_2} }
          { \eval{\sigma_2}{\cx}{8} }
        }
        { \eval{\sigma_1}{\eappfo{\code{f}}{\code{m}} }{8} }
      }
      {
        \evale{
          \eappfo{(\efun{\code{f}\
          \code{m}}{\eappfo{\code{f}}{\code{m}}})}{\efun{\cx}{\cx},8}
        }{8}
      }
    \]
\end{enumerate}

\textbf{\refex{first-class-functions-js-app}}
\begin{enumerate}
  \item
    \[
      \eval{\sigma,\alpha}{\efun{x_1\cdots x_n}{e}}{\clov{x_1\cdots x_n}{e}{\sigma}}
    \]
    \[
      \tiny
      \inferrule
      {
        \eval{\sigma,\alpha}{e_0}{\clov{x_1\cdots x_m}{e}{\sigma}}\\
        \eval{\sigma,\alpha}{e_1}{v_1}\quad\cdots\quad\eval{\sigma,\alpha}{e_n}{v_n}\\
        \eval{\sigma'[x_1\mapsto v_1,\cdots,x_{\min(m,n)}\mapsto v_{\min(m,n)},
        x_{\min(m,n)+1}\mapsto\textsf{undefined},\cdots,x_m\mapsto\textsf{undefined}],
        [v_1,\cdots,v_n]}{e}{v}
      }
      { \eval{\sigma,\alpha}{\eappfo{e_0}{e_1,\cdots,e_n}}{v} }
    \]
    \[
      \inferrule
      { \eval{\sigma,\alpha}{e}{n} \\ 0\le n<\embox{Length}(\alpha) }
      { \eval{\sigma,\alpha}{\textsf{get}\ e}{\alpha[n]} }
    \]
  \item
    \[
      \tiny
      \inferrule
      {
        \eval
        {\emptyset,[]}
        {\efun{\cx\ \cy}{\textsf{get}\ \cx}}
        {\clov{\cx\ \cy}{\textsf{get}\ \cx}{\emptyset}}
        \\
        \eval{\emptyset,[]}{2}{2} \\
        \eval{\emptyset,[]}{19}{19} \\
        \eval{\emptyset,[]}{141}{141} \\
        \inferrule
        {
          \inferrule
          { \cx\in\dom{[\cx\mapsto2,\cy\mapsto19]} }
          { \eval{[\cx\mapsto2,\cy\mapsto19],[2,19,141]}{\cx}{2} }
          \\
          0\le2<\embox{Length}([2,19,141])
        }
        {
          \eval
          {[\cx\mapsto2,\cy\mapsto19],[2,19,141]}
          {\textsf{get}\ \cx}
          {141}
        }
      }
      {
        \eval
        {\emptyset,[]}
        {\eappfo{(\efun{\cx\ \cy}{\textsf{get}\ \cx})}{2,19,141}}
        {141}
      }
    \]
\end{enumerate}

\textbf{\refex{first-class-functions-subst}}
\newcommand{\evals}[2]{#1\Rightarrow#2}
\begin{enumerate}
  \item
    \[
      \evals{n}{n}
    \]

    \[
      \evals{\efun{x}{e}}{\efun{x}{e}}
    \]

    \[
      \inferrule
      { \evals{e_1}{\efun{x}{e}} \\
        \evals{e_2}{v_2} \\
        \evals{e[x/v_2]}{v} }
      { \evals{\eapp{e_1}{e_2}}{v} }
    \]
  \item
      \[
        \begin{array}{rcl}
          n[x/v] &=& n \\
          x'[x/v] &=& v \quad \text{if}\ x=x' \\
          x'[x/v] &=& x' \quad \text{if}\ x\not=x' \\
          (\efun{x'}{e})[x/v] &=& \efun{x'}{e} \quad \text{if}\ x=x' \\
          (\efun{x'}{e})[x/v] &=& \efun{x'}{e[x/v]} \quad \text{if}\ x\not=x' \\
          (\eapp{e_1}{e_2})[x/v] &=& \eapp{e_1[x/v]}{e_2[x/v]} \\
        \end{array}
      \]
  \item
    \begin{enumerate}
        \item 0
        \item error
    \end{enumerate}
  \item
\begin{verbatim}
case Fun(y, b) =>
  if (y == x)
    Fun(y, b)
  else {
    val ny = fresh(binding(b) ++ free(b) ++ free(v) + x)
    Fun(ny, subst(subst(b, y, Id(ny)), x, v))
  }
\end{verbatim}
\end{enumerate}

\textbf{\refex{first-class-functions-js-record}}
\begin{enumerate}
  \item
      \[
        \inferrule
        { \evald{e_1}{v_1} \\\cdots\\ \evald{e_n}{v_n} }
        { \evald{\{l_1:e_1,\cdots,l_n:e_n\}}{\langle l_1:v_1,\cdots,l_n:v_n\rangle} }
      \]

      \[
        \inferrule
        { \evald{e}{\langle\cdots,l:v,\cdots\rangle} }
        { \evald{e.l}{v} }
      \]

      \[
        \inferrule
        {
          \evald{e}{\langle l_1:v_1,\cdots,l_n:v_n\rangle} \\
          l\not\in\{l_1,\cdots,l_n\} \\
          l_i=\code{proto} \\
          \evald{v_i.l}{v}
        }
        { \evald{e.l}{v} }
      \]

      \[
        \inferrule
        {
          \evald{e}{\langle l_1:v_1,\cdots,l_n:v_n\rangle} \\
          l\not\in\{l_1,\cdots,l_n\} \\
          \code{proto}\not\in\{l_1,\cdots,l_n\}
        }
        { \evald{e.l}{\textsf{undefined}} }
      \]

      \[
        \inferrule
        {
          \evald{e_1}{v_1} \\
          \evald{v_1.l}{\clov{x}{e}{\sigma'}} \\
          \evald{e_2}{v_2} \\
          \eval{\sigma'[\code{this}\mapsto v_1,x\mapsto v_2]}{e}{v}
        }
        { \evald{e_1.l(e_2)}{v} }
      \]
  \item
    \[
      \tiny
      \inferrule
      {
        \inferrule
        {
          \inferrule
          { \eval{\emptyset}{1}{1} }
          { \eval{\emptyset}{\{\code{x}:1\}}{\langle\code{x}:1\rangle} }
        }
        { \eval
          {\emptyset}
          {\{\code{proto}:\{\code{x}:1\}\}}
          {\langle\code{proto}:\langle\code{x}:1\rangle\rangle}
        } \\
        \code{x}\not\in\{\code{proto}\} \\
        \code{proto}=\code{proto} \\
        \inferrule
        { \evald{\langle\code{x}:1\rangle}{\langle\code{x}:1\rangle} }
        { \evald{\langle\code{x}:1\rangle.\code{x}}{1} }
      }
      { \eval{\emptyset}{\{\code{proto}:\{\code{x}:1\}\}.\code{x}}{1} }
    \]
  \item
    $\efun{\code{x}}{\code{this}.\code{a}}$
\end{enumerate}

\textbf{\refex{first-class-functions-exc}}
\begin{enumerate}
    \item
    \[
      \evald{n}{n}
    \]

    \[
      \inferrule
      { x\in\dom{\sigma} }
      { \evald{x}{\sigma(x)} }
    \]

    \[
      \evald{\efun{x}{e}}{\clov{x}{e}{\sigma}}
    \]

    \[
      \inferrule
      { \evald{e_1}{\textsf{exc}} }
      { \evald{e_1+e_2}{\textsf{exc}} }
    \]

    \[
      \inferrule
      { \evald{e_1}{v} \\
        \evald{e_2}{\textsf{exc}} }
      { \evald{e_1+e_2}{\textsf{exc}} }
    \]

    \[
      \inferrule
      { \evald{e_1}{n_1} \\
        \evald{e_2}{n_2} }
      { \evald{e_1+e_2}{n_1+n_2} }
    \]

    \[
      \inferrule
      { \evald{e_1}{\textsf{exc}} }
      { \evald{e_1\ e_2}{\textsf{exc}} }
    \]

    \[
      \inferrule
      { \evald{e_1}{v} \\
        \evald{e_2}{\textsf{exc}} }
      { \evald{e_1\ e_2}{\textsf{exc}} }
    \]

    \[
      \inferrule
      { \evald{e_1}{\clov{x}{e}{\sigma'}} \\
        \evald{e_2}{v_2} \\
        \eval{\sigma'[x\mapsto v_2]}{e}{r}
      }
      { \evald{\eapp{e_1}{e_2}}{r} }
    \]

    \[
      \evald{\textsf{throw}}{\textsf{exc}}
    \]

    \[
      \inferrule
      { \evald{e_1}{v} }
      { \evald{\textsf{try}\ e_1\ \textsf{catch}\ e_2}{v} }
    \]

    \[
      \inferrule
      { \evald{e_1}{\textsf{exc}} \\ \evald{e_2}{r} }
      { \evald{\textsf{try}\ e_1\ \textsf{catch}\ e_2}{r} }
    \]
    \item
  \[
    \inferrule
    {
      \inferrule
      {
        \evale{1}{1} \\
        \evale{\textsf{throw}}{\textsf{exc}}
      }
      { \evale{1 + \textsf{throw}}{\textsf{exc}} } \\
      \inferrule
      { \evale{\textsf{throw}}{\textsf{exc}} }
      { \evale{\textsf{throw} + 2}{\textsf{exc}} } \\
    }
    { \evale{\textsf{try}\ (1 + \textsf{throw})\ \textsf{catch}\ (\textsf{throw} + 2)}{\textsf{exc}} }
  \]
\end{enumerate}

\textbf{\refex{recursion-arrow}}

$
\begin{array}{@{}l@{}}
  \ebind{\cf_A}{\efun{\cx_B}{\eifz{\cx_C}{0}{(\eapp{\cf_D}{(\esub{\cx_E}{1})})}}}{ \\
  \erec{\cf_F}{\cx_G}{\eifz{\cx_H}{0}{(\eapp{\cf_I}{(\esub{\cx_J}{1})})}}{ \\
  \eapp{\cf_K}{\cy_L}
  }
  }
\end{array}
$
\begin{enumerate}
  \item $C\rightarrow B,E\rightarrow B,H\rightarrow G,I\rightarrow
    F,J\rightarrow G,K\rightarrow F$
  \item $D,L$
\end{enumerate}

\textbf{\refex{recursion-bindings}}
\vspace{-1em}
\begin{verbatim}
case Num(n) => Set()
case Id(x) => Set()
case Val(x, e, b) => (bindings(e) ++ bindings(b)) + x
case App(f, a) => bindings(f) ++ bindings(a)
case Fun(x, b) => bindings(b) + x
case Rec(f, x, b, e) => bindings(b) ++ bindings(e) + f + x

case Num(n) => Set()
case Id(x) => Set(x)
case Val(x, e, b) => frees(e) ++ (frees(b) - x)
case App(f, a) => frees(f) ++ frees(a)
case Fun(x, b) => frees(b) - x
case Rec(f, x, b, e) => (frees(b) - f - x) ++ (frees(e) - f)
\end{verbatim}

\textbf{\refex{recursion-scope}}
\begin{enumerate}
    \item Does not terminate
    \item 49
\end{enumerate}

\textbf{\refex{recursion-eager-if}}

$\eifz{0}{0}{(\eapp{0}{0})}$
\\

\textbf{\refex{recursion-bool}}
\[ \evald{b}{b} \]

\[
  \inferrule
  { \evald{e_1}{v_1} \\ \evald{e_2}{v_2} }
  { \evald{e_1\wedge e_2}{v_1\wedge v_2} }
\]

\[
  \inferrule
  { \evald{e}{v} }
  { \evald{\neg e}{\neg v} }
\]

\[
  \inferrule
  { \evald{e_1}{\true} \\ \evald{e_2}{v_2} }
  { \evald{\eif{e_1}{e_2}{e_3}}{v_2} }
\]

\[
  \inferrule
  { \evald{e_1}{\false} \\ \evald{e_3}{v_3} }
  { \evald{\eif{e_1}{e_2}{e_3}}{v_3} }
\]

\textbf{\refex{recursion-racket-if}}
\newcommand{\eand}[2]{\textsf{and}\ #1\ #2}
\newcommand{\eor}[2]{\textsf{or}\ #1\ #2}
\begin{enumerate}
  \item
    \[
      { \evald{b}{b} }
    \]

    \[
      \inferrule
      { \evald{e_1}{v_1} \\ v_1\not={\false} \\ \evald{e_2}{v_2} }
      { \evald{\eif{e_1}{e_2}{e_3}}{v_2} }
    \]

    \[
      \inferrule
      { \evald{e_1}{\false} \\ \evald{e_3}{v_3} }
      { \evald{\eif{e_1}{e_2}{e_3}}{v_3} }
    \]

    \[
      \inferrule
      { \evald{e_1}{\false} }
      { \evald{\eand{e_1}{e_2}}{\false} }
    \]

    \[
      \inferrule
      { \evald{e_1}{v_1} \\ v_1\not={\false} \\ \evald{e_2}{v_2} }
      { \evald{\eand{e_1}{e_2}}{v_2} }
    \]

    \[
      \inferrule
      { \evald{e_1}{v_1} \\ v_1\not={\false} }
      { \evald{\eor{e_1}{e_2}}{v_1} }
    \]

    \[
      \inferrule
      { \evald{e_1}{\false} \\ \evald{e_2}{v_2}}
      { \evald{\eor{e_1}{e_2}}{v_2} }
    \]
  \item
    \[
      \scriptsize
      \inferrule
      {
        \inferrule
        {
          \eval{\emptyset}{\false}{\false} \\
          \eval{\emptyset}{2}{2}
        }
        { \eval{\emptyset}{\eor{\false}{2}}{2} } \\
        2\not={\false} \\
        \inferrule
        { \eval{\emptyset}{\false}{\false} }
        { \eval{\emptyset}{\eand{\false}{2}}{\false} } \\
      }
      { \eval{\emptyset}{\eif{(\eor{\false}{2})}{(\eand{\false}{2})}{1}}{\false} }
    \]
\end{enumerate}

\textbf{\refex{recursion-fix-nont}}

Change $\ebind{\code{g}}{\eapp{\cy}{\cy}}{}$to
$\ebind{\code{g}}{\efun{\cx}{\eapp{\eapp{\cy}{\cy}}{\cx}}}{}$\!.
\\

\textbf{\refex{recursion-fix-free}}

Change $\ebind{\cf}{\eapp{\cz}{({\efun{\cv}{\sumbodyfv}})}}{}$to
$\ebind{\cf}{\eapp{\cz}{({\efun{\cf}{\efun{\cv}{\sumbodyfv}}})}}{}$\!.
\\

\textbf{\refex{recursion-sum-arrow}}

$\begin{array}{l}
  \ebind{\cf_A}{(\\
  \ \ \ \ \ebind{\cx_B}{\efun{\cy_C}{(\\
  \ \ \ \ \ \ \ \ \ebind{\cf_D}{\efun{\cv_E}{\eapp{\eapp{\cy_F}{\cy_G}}{\cv_H}}}{\\
  \ \ \ \ \ \ \ \ \efun{\cv_I}{\eifz{\cv_J}{0}{(\eadd{\cv_K}{\eapp{\cf_L}{(\esub{\cv_M}{1})}})}}}\\
  \ \ \ \ )}}{\\
  \ \ \ \ \eapp{\cx_N}{\cx_O}\\
  )}}{\\
  \eapp{\cf_P}{10}
  }
\end{array}$
\begin{enumerate}
  \item $F\rightarrow C,G\rightarrow C,H\rightarrow E,J\rightarrow I,
    K\rightarrow I,L\rightarrow D,M\rightarrow I,N\rightarrow B,
    O\rightarrow B,P\rightarrow A$
  \item No shadowing
  \item
    $e={\ebind{\cf}{\efun{\cv}{\eapp{\eapp{\cy}{\cy}}{\cv}}}{\efun{\cv}{\sumbodyfv}}}$
    \\
    $\clov{\cv}{\sumbodyfv}{[
      \cy\mapsto\clov{\cy}{e}{\emptyset},
      \cf\mapsto\clov{\cv}{\eapp{\eapp{\cy}{\cy}}{\cv}}{[\cy\mapsto\clov{\cy}{e}{\emptyset}]}
    ]}$

\end{enumerate}

\textbf{\refex{recursion-describe}}

An argument to $\cz$ must be a function whose fixed point is a recursive function
to be constructed. For example, to make the $\embox{sum}$ function with $\cz$,
its argument must be a function that returns $\embox{sum}$ when $\embox{sum}$ is
given.
\\

\textbf{\refex{recursion-mutual}}

$\begin{array}{@{}l@{}}
  \ebind{\cz}{(\efun{\cb}{\\
  \ \ \ \ \ebind{\code{fx}}{(\efun{\code{fy}}{\\
  \ \ \ \ \ \ \ \ \ebind{\cf}{\efun{\cx}{(\eapp{\code{fy}}{\code{fy}})\textsf{.1}\ \cx}}{\\
  \ \ \ \ \ \ \ \ \ebind{\code{g}}{\efun{\cx}{(\eapp{\code{fy}}{\code{fy}})\textsf{.2}\ \cx}}{\\
  \ \ \ \ \ \ \ \ \eapp{\cb}{(\cf,\code{g})}}}}\\\ \ \ \ )}{\\
  \ \ \ \ \eapp{\code{fx}}{\code{fx}}}}\\)}{\\
  \ebind{\cf}{\eapp{\cz}{\efun{\cf}{(
    \efun{\code{n}}{\eifz{\code{n}}{\true}{(\eapp{\cf\textsf{.2}}{(\esub{\code{n}}{1})})}},
    \efun{\code{n}}{\eifz{\code{n}}{\false}{(\eapp{\cf\textsf{.1}}{(\esub{\code{n}}{1})})}}
  )}}}{\\
  \ebind{\code{even}}{\cf\textsf{.1}}{\\
  \ebind{\code{odd}}{\cf\textsf{.2}}{\\
  (\eapp{\code{even}}{10}, \eapp{\code{odd}}{10})}}}}
\end{array}$
\\

\textbf{\refex{mutable-boxes-trace}}

$
\begin{array}{@{}c|c|c|c@{}}
  \text{expr} & \text{env} & \text{sto} & \text{res} \\\hline
  \eapp{(\efun{\cx}{\eapp{(\efun{\cy}{\eseq{\eset{\cx}{8}}{\ederef{\cy}}})}{\cx}})}{\eref{7}}
  & \emptyset & \emptyset & (8,[a_0\mapsto8]) \\
  \efun{\cx}{\eapp{(\efun{\cy}{\eseq{\eset{\cx}{8}}{\ederef{\cy}}})}{\cx}}
  & \emptyset & \emptyset &
  (\clov{\cx}{\eapp{(\efun{\cy}{\eseq{\eset{\cx}{8}}{\ederef{\cy}}})}{\cx}}{\emptyset},\emptyset) \\
  \eref{7} & \emptyset & \emptyset & (a_0,[a_0\mapsto7]) \\
  \eapp{(\efun{\cy}{\eseq{\eset{\cx}{8}}{\ederef{\cy}}})}{\cx} &
  [\cx\mapsto a_0] & [a_0\mapsto7] & (8,[a_0\mapsto8]) \\
  \efun{\cy}{\eseq{\eset{\cx}{8}}{\ederef{\cy}}} &
  [\cx\mapsto a_0] & [a_0\mapsto7] &
  (\clov{\cy}{\eseq{\eset{\cx}{8}}{\ederef{\cy}}}{[\cx\mapsto a_0]},[a_0\mapsto7]) \\
  \cx & [\cx\mapsto a_0] & [a_0\mapsto7] & (a_0,[a_0\mapsto7]) \\
  \eseq{\eset{\cx}{8}}{\ederef{\cy}} & [\cx\mapsto a_0,\cy\mapsto a_0]
  & [a_0\mapsto7] & (8,[a_0\mapsto8]) \\
  \eset{\cx}{8} & [\cx\mapsto a_0,\cy\mapsto a_0] & [a_0\mapsto7]
  & (8,[a_0\mapsto8]) \\
  8 & [\cx\mapsto a_0,\cy\mapsto a_0] & [a_0\mapsto7] & (8,[a_0\mapsto7]) \\
  \ederef{\cy} & [\cx\mapsto a_0,\cy\mapsto a_0] & [a_0\mapsto8]
  & (8,[a_0\mapsto8]) \\
  \cy & [\cx\mapsto a_0,\cy\mapsto a_0] & [a_0\mapsto8] & (a_0,[a_0\mapsto8]) \\
\end{array}
$
\\

\textbf{\refex{mutable-boxes-desugar}}
\vspace{-1em}
\begin{verbatim}
App(Fun(fresh(free(r)), desugar(r)), desugar(l))
\end{verbatim}

\textbf{\refex{mutable-boxes-landin}}

$\begin{array}{l}
  \ebind{\cz}{\efun{\cb}{(\\
  \ \ \ \ \ebind{\ca}{\eref{\efun{\cx}{\cy}}}{\\
  \ \ \ \ \ebind{\cf}{(\eapp{\cb}{\efun{\cx}{\eapp{\ederef{\ca}}{\cx}}})}{\\
  \ \ \ \ \eseq{\eset{\ca}{\cf}}{\\
  \ \ \ \ \cf\\
  }}})}}{\\
  \ebind{\cf}{\eapp{\cz}{(\efun{\cf}{\efun{\cv}{\sumbodyfv}})}}{\\
  \eapp{\cf}{10}}}
\end{array}$
\\

\textbf{\refex{mutable-variables-trace}}

$
\begin{array}{@{}c|c|c@{}}
  \text{expr} & \text{env} & \text{sto} \\\hline
  \eapp{(\efun{\cx}{\cx})}{(\eapp{(\efun{\cx}{\cx})}{1})} & \emptyset & \emptyset \\
  \efun{\cx}{\cx} & \emptyset & \emptyset \\
  \eapp{(\efun{\cx}{\cx})}{1} & \emptyset & \emptyset \\
  \efun{\cx}{\cx} & \emptyset & \emptyset \\
  1 & \emptyset & \emptyset \\
  \cx & [\cx\mapsto a_1] & [a_1\mapsto1] \\
  \cx & [\cx\mapsto a_2] & [a_1\mapsto1,a_2\mapsto1] \\
\end{array}
$
\\

\textbf{\refex{mutable-variables-cbr}}
\begin{enumerate}
  \item CBR
  \item
    $[\code{n} \mapsto a_1, \cf \mapsto a_2, \cx \mapsto a_4]$,\\
$[
    a_1 \mapsto 42,
    a_2 \mapsto \clov{\code{g}}{\eapp{\code{g}}{\code{n}}}{[\code{n} \mapsto a_1]},
    a_3 \mapsto \clov{\cx}{\eadd{\cx}{8}}{[\code{n} \mapsto a_1, \code{f} \mapsto a_2]},
    a_4 \mapsto 42
]$
  \item
    $[\code{n} \mapsto a_1, \cf \mapsto a_2, \cx \mapsto a_1]$,\\
$[
    a_1 \mapsto 42,
    a_2 \mapsto \clov{\code{g}}{\eapp{\code{g}}{\code{n}}}{[\code{n} \mapsto a_1]},
    a_3 \mapsto \clov{\cx}{\eadd{\cx}{8}}{[\code{n} \mapsto a_1, \code{f} \mapsto a_2]}
]$
\end{enumerate}

\textbf{\refex{mutable-variables-record}}
\begin{enumerate}
  \item
\begin{verbatim}
val (RecV(fields), rs) = interp(r, env, fs)
interp(b, fenv + (x -> fields(f)), rs)
\end{verbatim}
  \item 2,
    $[a_1 \mapsto 2, a_2 \mapsto \{\cz{:}a_1\}, a_3 \mapsto \clov{\cy}{\eset{\cy}{2}}{[\cx \mapsto a_2]}]$
  \item 1,
    $[a_1 \mapsto 1, a_2 \mapsto \{\cz{:}a_1\}, a_3 \mapsto \clov{\cy}{\eset{\cy}{2}}{[\cx \mapsto a_2]}, a_4 \mapsto 2]$
\end{enumerate}

\textbf{\refex{mutable-variables-ptr}}
\begin{enumerate}
    \item
\[
  \inferrule
  { \sevald{}{e}{a}{1} \\ a\in\dom{M_1} }
  { \sevald{}{\ast e}{M_1(a)}{1} }
\]

\[
  \inferrule
  { x\in\dom{\sigma} }
  { \sevald{}{\&x}{\sigma(x)}{} }
\]

\[
  \inferrule
  { \sevald{}{e_2}{v}{1} \\ \sevald{1}{e_1}{a}{2} }
  { \seval{\sigma}{M}{\ast\eset{e_1}{e_2}}{v}{M_2[a\mapsto v]} }
\]
\item
\begin{verbatim}
case Deref(p) =>
  val (pv, ps) = interp(p, env, sto)
  val PtrV(a) = pv
  (ps(a), ps)
case Ref(x) => (PtrV(env(x)), sto)
case Assign(p, e) =>
  val (ev, es) = interp(e, env, sto)
  val (pv, ps) = interp(p, env, es)
  val PtrV(a) = pv
  (ev, ps + (a -> ev))
\end{verbatim}
\end{enumerate}

\textbf{\refex{mutable-variables-imp}}
\begin{enumerate}
  \item
    \[
      { \evald{\eskip}{\sigma} }
    \]

    \[
      \inferrule
      { \evald{e}{v} }
      { \evald{\eset{x}{e}}{\sigma[x\mapsto v]} }
    \]

    \[
      \inferrule
      { \evald{e}{0} \\ \evald{c_1}{\sigma_1} }
      { \evald{\eifz{e}{c_1}{c_2}}{\sigma_1} }
    \]

    \[
      \inferrule
      { \evald{e}{v} \\ v\not=0 \\ \evald{c_2}{\sigma_2} }
      { \evald{\eifz{e}{c_1}{c_2}}{\sigma_2} }
    \]

    \[
      \inferrule
      { \evald{e}{0} \\ \evald{c}{\sigma_1} \\ \eval{\sigma_1}{\ewhilez{e}{c}}{\sigma_2} }
      { \evald{\ewhilez{e}{c}}{\sigma_2} }
    \]

    \[
      \inferrule
      { \evald{e}{v} \\ v\not=0 }
      { \evald{\ewhilez{e}{c}}{\sigma} }
    \]

    \[
      \inferrule
      { \evald{c_1}{\sigma_1} \\ \eval{\sigma_1}{c_2}{\sigma_2} }
      { \evald{\eseq{c_1}{c_2}}{\sigma_2} }
    \]
  \item
    \[
      \scriptsize
      \inferrule
      {
        \inferrule
        { \eval{\emptyset}{0}{0} }
        { \eval{\emptyset}{\eset{{\cx}}{0}}{[{\cx}\mapsto0]} }
        \\
        \inferrule
        {
          \inferrule
          { {\cx}\in\dom{[{\cx}\mapsto0]} }
          { \eval{[{\cx}\mapsto0]}{{\cx}}{0} }
          \quad
          \inferrule
          {}
          { \eval{[{\cx}\mapsto0]}{{\eskip}}{[{\cx}\mapsto0]} }
        }
        { \eval{[{\cx}\mapsto0]}
          {\eseq{\eifz{{\cx}}{{\eskip}}{\eset{{\cx}}{1}}}}
          {[{\cx}\mapsto0]} }
      }
      { \eval{\emptyset}
        {\eseq{\eset{{\cx}}{0}}{\eifz{{\cx}}{{\eskip}}{\eset{{\cx}}{1}}}}
        {[{\cx}\mapsto0]}
      }
    \]
\end{enumerate}

\textbf{\refex{garbage-collection-copying}}

\begin{itemize}
  \item Stack: \code{13}
\item From space:
\begin{tabular}{|c|c|c|c|c|c|c|c|c|c|c|c|@{\hskip0pt}c@{\hskip0pt}|}
 \hline
 {1}&{2}&{2}&{3}&{7}&{99}&{19}&{99}&{16}&{10}&{99}&{13}&{5}\\
 \hline \multicolumn{1}{c}{}\\[-16pt]
  \multicolumn{1}{c}{\tiny \code{0}}&
  \multicolumn{1}{c}{\tiny \code{1}}&
  \multicolumn{1}{c}{\tiny \code{2}}&
  \multicolumn{1}{c}{\tiny \code{3}}&
  \multicolumn{1}{c}{\tiny \code{4}}&
  \multicolumn{1}{c}{\tiny \code{5}}&
  \multicolumn{1}{c}{\tiny \code{6}}&
  \multicolumn{1}{c}{\tiny \code{7}}&
  \multicolumn{1}{c}{\tiny \code{8}}&
  \multicolumn{1}{c}{\tiny \code{9}}&
  \multicolumn{1}{c}{\tiny \code{10}}&
  \multicolumn{1}{c}{\tiny \code{11}}&
  \multicolumn{1}{c}{\tiny \code{12}}
 \\
\end{tabular}
\item To space:
\begin{tabular}{|c|c|c|c|c|c|c|c|c|c|c|c|@{\hskip0pt}c@{\hskip0pt}|}
 \hline
 {3}&{16}&{5}&{4}&{19}&{13}&{1}&{4}&{0}&{0}&{0}&{0}&{0}\\
 \hline \multicolumn{1}{c}{}\\[-16pt]
  \multicolumn{1}{c}{\tiny \code{13}}&
  \multicolumn{1}{c}{\tiny \code{14}}&
  \multicolumn{1}{c}{\tiny \code{15}}&
  \multicolumn{1}{c}{\tiny \code{16}}&
  \multicolumn{1}{c}{\tiny \code{17}}&
  \multicolumn{1}{c}{\tiny \code{18}}&
  \multicolumn{1}{c}{\tiny \code{19}}&
  \multicolumn{1}{c}{\tiny \code{20}}&
  \multicolumn{1}{c}{\tiny \code{21}}&
  \multicolumn{1}{c}{\tiny \code{22}}&
  \multicolumn{1}{c}{\tiny \code{23}}&
  \multicolumn{1}{c}{\tiny \code{24}}&
  \multicolumn{1}{c}{\tiny \code{25}}
 \\
\end{tabular}
\end{itemize}

\textbf{\refex{lazy-evaluation-trace}}
\begin{itemize}
  \item Call-by-name\\
$
\begin{array}{@{}c|c@{}}
  \text{expr} & \text{env} \\\hline
  \eapp{(\efun{\cx}{\eadd{\cx}{\cx}})}{(\eadd{1}{2})} & \emptyset \\
  \efun{\cx}{\eadd{\cx}{\cx}} & \emptyset \\
  \eadd{\cx}{\cx} & [\cx\mapsto\exprv{\eadd{1}{2}}{\emptyset}] \\
  \cx & [\cx\mapsto\exprv{\eadd{1}{2}}{\emptyset}] \\
  \eadd{1}{2} & \emptyset \\
  1 & \emptyset \\
  2 & \emptyset \\
  \cx & [\cx\mapsto\exprv{\eadd{1}{2}}{\emptyset}] \\
  \eadd{1}{2} & \emptyset \\
  1 & \emptyset \\
  2 & \emptyset \\
\end{array}
$
  \item Call-by-need\\
$
\begin{array}{@{}c|c@{}}
  \text{expr} & \text{env} \\\hline
  \eapp{(\efun{\cx}{\eadd{\cx}{\cx}})}{(\eadd{1}{2})} & \emptyset \\
  \efun{\cx}{\eadd{\cx}{\cx}} & \emptyset \\
  \eadd{\cx}{\cx} & [\cx\mapsto(\eadd{1}{2},\emptyset,\cdot)] \\
  \cx & [\cx\mapsto(\eadd{1}{2},\emptyset,\cdot)] \\
  \eadd{1}{2} & \emptyset \\
  1 & \emptyset \\
  2 & \emptyset \\
  \cx & [\cx\mapsto(\eadd{1}{2},\emptyset,3)] \\
\end{array}
$
\end{itemize}

\textbf{\refex{lazy-evaluation-eval}}
\begin{enumerate}
  \item CBV: error, CBN: error
  \item CBV: error, CBN: 10
  \item CBV: error, CBN: error
  \item CBV: error, CBN: error
  \item CBV: error, CBN: 3
  \item CBV: error, CBN: error
  \item CBV: error, CBN: 8
  \item CBV: $\clov{\cx}{(\eapp{(\efun{\cy}{42})}{(\eapp{9}{2})})}{\emptyset}$,
    CBN: $\clov{\cx}{(\eapp{(\efun{\cy}{42})}{(\eapp{9}{2})})}{\emptyset}$
  \item CBV: error, CBN: error
  \item CBV: error, CBN: 15
\end{enumerate}

\textbf{\refex{lazy-evaluation-scope}}
\begin{enumerate}
  \item error
  \item 12
  \item error
  \item error
\end{enumerate}

\textbf{\refex{lazy-evaluation-strict}}

$\eapp{(\efun{\cy}{\eapp{(\efun{\cx}{\eadd{\cx}{0}})}{\cy}})}{0}$
\\

\textbf{\refex{lazy-evaluation-val}}
\vspace{-1em}
\begin{verbatim}
case Val(x, e, b) =>
  interp(b, env + (x -> ExprV(e, env)))
case If0(c, t, f) =>
  interp(if (strict(interp(c, env)) == NumV(0)) t else f, env)
\end{verbatim}

\textbf{\refex{lazy-evaluation-pair}}
\vspace{-1em}
\begin{verbatim}
case class PairV(f: Value, s: Value) extends Value

case Pair(f, s) => PairV(ExprV(f, env), ExprV(s, env))
case Fst(e) =>
  val PairV(v, _) = strict(interp(e, env))
  v
case Snd(e) =>
  val PairV(_, v) = strict(interp(e, env))
  v
\end{verbatim}

\textbf{\refex{lazy-evaluation-list}}
\vspace{-1em}
\begin{verbatim}
case class ConsV(h: Value, t: Value) extends Value

case Nil => NilV
case Cons(h, t) => ConsV(ExprV(h, env), ExprV(t, env))
case Head(e) =>
  val ConsV(h, _) = strict(interp(e, env))
  strict(h)
case Tail(e) =>
  val ConsV(_, t) = strict(interp(e, env))
  val v = strict(t)
  if (isList(v)) v else error()
\end{verbatim}

\textbf{\refex{lazy-evaluation-racket}}
\begin{enumerate}
  \item
    \[
      \stricte{n}{n}
    \]
    \[
      \inferrule
      {}
      { \stricte{\clov{x}{e}{\sigma}}{\clov{x}{e}{\sigma}} }
    \]
    \[
      \inferrule
      { \evald{e}{v} }
      { \stricte{\delay(e,\sigma)}{v} }
    \]
    \[
      \inferrule
      { \evald{e}{v'} \\ \stricte{v'}{v} }
      { \stricte{\lazy(e,\sigma)}{v} }
    \]
  \item
    \[
      \evald{\delay\ e}{\delay(e,\sigma)}
    \]
    \[
      \inferrule
      {}
      { \evald{\lazy\ e}{\lazy(e,\sigma)} }
    \]
    \[
      \inferrule
      { \evald{e}{v'} \\ \stricte{v'}{v} }
      { \evald{\force\ e}{v} }
    \]
\end{enumerate}

\textbf{\refex{continuations-lfae}}
\vspace{-1em}
\begin{verbatim}
case ExprV(e, env) =>
  interp(e, env, v => strict(v, k))

case Add(l, r) =>
  interp(l, env, v1 =>
    interp(r, env, v2 =>
      strict(v1, n =>
        strict(v2, m => {
          val NumV(l) = n
          val NumV(r) = m
          k(NumV(l + r))
        }))))
case App(f, a) =>
  interp(f, env, v =>
    strict(v, fv => {
      val CloV(x, b, fenv) = fv
      interp(b, fenv + (x -> ExprV(a, env)), k)
    }))
\end{verbatim}

\textbf{\refex{first-class-continuations-result}}
\begin{enumerate}
  \item 3
  \item 6
  \item Error
  \item 5
  \item Does not terminate
\end{enumerate}

\textbf{\refex{first-class-continuations-print}}

$\begin{array}{@{}l}
  \eapp{(\evcc{\cx}{\cx})}{\efun{\cy}{\cy}} \\
  \evcc{\cx}{\cx} \\
  \cx \\
  \efun{\cy}{\cy} \\
  \efun{\cy}{\cy} \\
  \cy \\
\end{array}$
\\

\textbf{\refex{first-class-continuations-reduction}}

$\sigma=[\cx\mapsto\langle\emptyset\vdash8::(+)::\square\ ||\ \blacksquare\rangle]$
\\
$
\begin{array}{lrcr}
& \emptyset\vdash (\evcc{\cx}{42+(\cx\ 2)})+8::\square &||& \blacksquare \\
\rightarrow & \emptyset\vdash\evcc{\cx}{42+(\cx\ 2)}
  ::\emptyset\vdash8::(+)::\square &||& \blacksquare \\
\rightarrow & \sigma\vdash42+(\cx\ 2)
  ::\emptyset\vdash8::(+)::\square &||& \blacksquare \\
\rightarrow & \sigma\vdash42::\sigma\vdash(\cx\ 2)::(+)
  ::\emptyset\vdash8::(+)::\square &||& \blacksquare \\
\rightarrow & \sigma\vdash(\cx\ 2)::(+)
  ::\emptyset\vdash8::(+)::\square &||& 42::\blacksquare \\
\rightarrow & \sigma\vdash\cx::\sigma\vdash2::(@)::(+)
  ::\emptyset\vdash8::(+)::\square &||& 42::\blacksquare \\
\rightarrow & \sigma\vdash2::(@)::(+)::\emptyset\vdash8::(+)::\square
  &||& \langle\emptyset\vdash8::(+)::\square\ ||\ \blacksquare\rangle::42::\blacksquare \\
\rightarrow & (@)::(+)::\emptyset\vdash8::(+)::\square
  &||& 2::\langle\emptyset\vdash8::(+)::\square\ ||\ \blacksquare\rangle::42::\blacksquare \\
\rightarrow & \emptyset\vdash8::(+)::\square &||& 2::\blacksquare \\
\rightarrow & (+)::\square &||& 8::2::\blacksquare \\
\rightarrow & \square &||& 10::\blacksquare \\
\end{array}
$
\\

\textbf{\refex{first-class-continuations-var}}
\vspace{-1em}
\begin{verbatim}
case Num(n) => k(NumV(n), sto)
case Id(x) => k(sto(env(x)), sto)
case Fun(x, b) => k(CloV(x, b, env), sto)
case App(f, a) =>
  interp(f, env, sto, (fv, fs) =>
    interp(a, env, fs, (av, as) => fv match {
      case CloV(x, b, fenv) =>
        val addr = malloc(as)
        interp(b, fenv + (x -> addr), as + (addr -> av), k)
      case ContV(k) => k(av, as)
    })
  )
case Set(x, e) =>
  interp(e, env, sto, (v, s) =>
    k(v, s + (env(x) -> v))
  )
case Vcc(x, b) =>
  val addr = malloc(sto)
  interp(b, env + (x -> addr), sto + (addr -> ContV(k)), k)
\end{verbatim}

\textbf{\refex{first-order-representation-of-continuations-val}}
\vspace{-1em}
\begin{verbatim}
case class ValSecondK(x: String, b: Expr, env: Env, k: Cont) extends Cont
case class If0SecondK(t: Expr, f: Expr, env: Env, k: Cont) extends Cont

case ValSecondK(x, b, env, k) => interp(b, env + (x -> v), k)
case If0SecondK(t, f, env, k) => interp(if (v == NumV(0)) t else f, env, k)

case Val(x, e, b) => interp(e, env, ValSecondK(x, b, env, k))
case If0(c, t, f) => interp(c, env, If0SecondK(t, f, env, k))
\end{verbatim}

\textbf{\refex{first-order-representation-of-continuations-pair}}
\vspace{-1em}
\begin{verbatim}
case class PairSecondK(s: Expr, env: Env, k: Cont) extends Cont
case class DoPairK(fv: Value, k: Cont) extends Cont
case class DoFstK(k: Cont) extends Cont
case class DoSndK(k: Cont) extends Cont

case PairSecondK(s, env, k) => interp(s, env, DoPairK(v, k))
case DoPairK(fv, k) => continue(k, PairV(fv, v))
case DoFstK(k) =>
  val PairV(fv, _) = v
  continue(k, fv)
case DoSndK(k) =>
  val PairV(_, sv) = v
  continue(k, sv)

case Pair(f, s) => interp(f, env, PairSecondK(s, env, k))
case Fst(p) => interp(p, env, DoFstK(k))
case Snd(p) => interp(p, env, DoSndK(k))
\end{verbatim}

\textbf{\refex{nameless-representation-of-expressions-trans}}

$\eapp{\eapp{\eapp{(\efun{}{\efun{}{\efun{}{\eadd{(\esub{\underline{0}}{\underline{2}})}{\underline{1}}}}})}{42}}{0}}{10}$
\\

\textbf{\refex{nameless-representation-of-expressions-detrans}}

$\efun{\cx}{\efun{\cy}{\efun{\cz}{\eapp{\eapp{\cz}{\cy}}{\cz}}}}$
\\

\textbf{\refex{type-systems-unsound}}

$\eadd{1}{\cx}$
\\

\textbf{\refex{type-systems-list}}

\[
  \typeofd{\textsf{nil}[\tau]}{\textsf{list}\ \tau}
\]

\[
  \inferrule
  { \typeofd{e_1}{\tau} \\ \typeofd{e_2}{\textsf{list}\ \tau} }
  { \typeofd{\textsf{cons}\ e_1\ e_2}{\textsf{list}\ \tau} }
\]

\[
  \inferrule
  { \typeofd{e}{\textsf{list}\ \tau} }
  { \typeofd{\textsf{head}\ e}{\tau} }
\]

\[
  \inferrule
  { \typeofd{e}{\textsf{list}\ \tau} }
  { \typeofd{\textsf{tail}\ e}{\textsf{list}\ \tau} }
\]

\textbf{\refex{type-systems-box}}
\begin{enumerate}
  \item
    \[
      \inferrule
      { \typeofd{e}{\tau} }
      { \typeofd{\eref{e}}{\eref{\tau}} }
    \]

    \[
      \inferrule
      { \typeofd{e}{\eref{\tau}} }
      { \typeofd{\ederef{e}}{\tau} }
    \]

    \[
      \inferrule
      { \typeofd{e_1}{\eref{\tau}} \\ \typeofd{e_2}{\tau} }
      { \typeofd{\eset{e_1}{e_2}}{\tau} }
    \]

    \[
      \inferrule
      { \typeofd{e_1}{\tau_1} \\ \typeofd{e_2}{\tau_2} }
      { \typeofd{\eseq{e_1}{e_2}}{\tau_2} }
    \]
  \item
    $\Gamma_1=[\cx:\eref{\tnum}],\Gamma_2=\Gamma_1[\cy:\tnum]$
    \[
      \tiny
      \inferrule
      {
        \inferrule
        { \typeofe{3}{\tnum} }
        { \typeofe{\eref{3}}{\eref{\tnum}} }
        \\
        \inferrule
        {
          \inferrule
          {
            \inferrule
            {
              \inferrule
              { \cx\in\dom{\Gamma_1} }
              { \typeof{\Gamma_1}{\cx}{\eref{\tnum}} }
            }
            { \typeof{\Gamma_1}{\ederef{\cx}}{\tnum} }
            \\
            \typeof{\Gamma_1}{7}{\tnum}
          }
          { \typeof{\Gamma_1}{\ederef{\cx}+7}{\tnum} }
          \\
          \inferrule
          {
            \inferrule
            {
              \inferrule
              { \cx\in\Gamma_2}
              { \typeof{\Gamma_2}{\cx}{\eref{\tnum}} }
              \\
              \typeof{\Gamma_2}{8}{\tnum}
            }
            { \typeof{\Gamma_2}{\eset{\cx}{8}}{\tnum} }
            \\
            \inferrule
            {
              \inferrule
              { \cy\in\dom{\Gamma_2}}
              { \typeof{\Gamma_2}{\cy}{\tnum} }
              \\
              \inferrule
              {
                \inferrule
                { \cx\in\dom{\Gamma_2} }
                { \typeof{\Gamma_2}{\cx}{\eref{\tnum}} }
              }
              { \typeof{\Gamma_2}{\ederef{\cx}}{\tnum} }
            }
            { \typeof{\Gamma_2}{\cy+\ederef{\cx}}{\tnum} }
          }
          { \typeof{\Gamma_2}{\eseq{\eset{\cx}{8}}{\cy+\ederef{\cx}}}{\tnum} }
        }
        { \typeof{\Gamma_1}{\ebind{\cy}{\ederef{\cx}+7}{\eseq{\eset{\cx}{8}}{\cy+\ederef{\cx}}}}{\tnum} }
      }
      { \typeofe{\ebind{\cx}{\eref{3}}{\ebind{\cy}{\ederef{\cx}+7}{\eseq{\eset{\cx}{8}}{\cy+\ederef{\cx}}}}}{\tnum} }
    \]

\end{enumerate}

\textbf{\refex{type-systems-var}}
\begin{enumerate}
  \item
    \[
      \inferrule
      { x\in\dom{\Gamma} \\ \typeofd{e}{\Gamma(x)} }
      { \typeofd{\eset{x}{e}}{\Gamma(x)} }
    \]
  \item
    \[
      \inferrule
      { \typeofd{e}{\tau\ast} }
      { \typeofd{\ast e}{\tau} }
    \]

    \[
      \inferrule
      { x\in\dom{\Gamma} }
      { \typeofd{\&x}{\Gamma(x)\ast} }
    \]

    \[
      \inferrule
      { \typeofd{e_1}{\tau\ast} \\ \typeofd{e_2}{\tau} }
      { \typeofd{\ast\eset{e_1}{e_2}}{\tau} }
    \]
\end{enumerate}

\textbf{\refex{typing-recursive-functions-terminate}}

$\erect{\cf}{\cx}{\tnum}{\tarrow{\tnum}{\tnum}}{\eapp{\cf}{\cx}}{\eapp{\cf}{0}}$
\\

\textbf{\refex{typing-recursive-functions-imp}}
\begin{enumerate}
  \item
    \[
      \typeofd{n}{\tnum}
    \]

    \[
      \typeofd{b}{\textsf{bool}}
    \]

    \[
      \inferrule
      { x\in\dom{\Gamma} }
      { \typeofd{x}{\Gamma(x)} }
    \]

    \[
      \inferrule
      { \typeofd{e_1}{\tnum} \\ \typeofd{e_2}{\tnum} }
      { \typeofd{e_1+e_2}{\tnum} }
    \]

    \[
      \inferrule
      { \typeofd{e_1}{\tnum} \\ \typeofd{e_2}{\tnum} }
      { \typeofd{e_1<e_2}{\textsf{bool}} }
    \]
  \item
    \[
      \typeofd{\textsf{skip}}{\Gamma}
    \]

    \[
      \inferrule
      { x\not\in\dom{\Gamma} \\ \typeofd{e}{\tau} }
      { \typeofd{\eset{x}{e}}{\Gamma[x:\tau]} }
    \]

    \[
      \inferrule
      { x\in\dom{\Gamma} \\ \typeofd{e}{\Gamma(x)} }
      { \typeofd{\eset{x}{e}}{\Gamma} }
    \]

    \[
      \inferrule
      { \typeofd{e}{\textsf{bool}} \\
        \typeofd{c_1}{\Gamma_1} \\
        \typeofd{c_2}{\Gamma_1} }
      { \typeofd{\eif{e}{c_1}{c_2}}{\Gamma_1} }
    \]

    \[
      \inferrule
      { \typeofd{e}{\textsf{bool}} \\ \typeofd{c}{\Gamma_1} }
      { \typeofd{\textsf{while}\ e\ c}{\Gamma} }
    \]

    \[
      \inferrule
      { \typeofd{c_1}{\Gamma_1} \\ \typeof{\Gamma_1}{c_2}{\Gamma_2} }
      { \typeofd{\eseq{c_1}{c_2}}{\Gamma_2} }
    \]
\end{enumerate}

\textbf{\refex{algebraic-data-types-eval}}
\begin{enumerate}
  \item 10
  \item 10
\end{enumerate}

\textbf{\refex{algebraic-data-types-norec}}

$\etdef{\code{A}}{\cX}{\code{A}}{\cY}{\tnum}{0}$
\\

\textbf{\refex{algebraic-data-types-nowf}}

$\begin{array}{@{}l}
  \etdef{\cX}{\code{A}}{\code{Y}}{\code{B}}{\tnum}{ \\
  \eapp{( \\
  \ \ \ \ \etdef{\cY}{\code{C}}{\tnum}{\code{D}}{\tnum}{ \\
  \ \ \ \ \efunt{\cx}{\cX}{\ematch{\cx}{ \\
  \ \ \ \ \ \ \ \ \code{A}}{\cy}{\ematch{\cy}{ \\
  \ \ \ \ \ \ \ \ \ \ \ \ \code{C}}{\cz}{0}{ \\
  \ \ \ \ \ \ \ \ \ \ \ \ \code{D}}{\cz}{0}}{ \\
  \ \ \ \ \ \ \ \ \code{B}}{\cy}{0}} \\
  })}{( \\
  \ \ \ \ \etdef{\cY}{\code{E}}{\tnum}{\code{F}}{\tnum}{ \\
  \ \ \ \ \eapp{\code{A}}{(\eapp{\code{E}}{0})} \\
  })}}
\end{array}$

\textbf{\refex{algebraic-data-types-nested}}

$\textsf{Color}$
\[
  \inferrule
  { \typeofd{e}{t} \\
    t\in\dom{\Gamma} \\
    \Gamma(t)=x_1@\tau_1+x_2@\tau_2 \\
    \Gamma'=\Gamma[x_1:\tarrow{\tau_1}{t},x_2:\tarrow{\tau_2}{t}] \\
    \typeof{\Gamma'[x_3:\tau_1]}{e_1}{\tau} \\
    \typeof{\Gamma'[x_4:\tau_2]}{e_2}{\tau} }
  { \typeofd{\ematch{e}{x_1}{x_3}{e_1}{x_2}{x_4}{e_2}}{\tau} }
\]

\textbf{\refex{algebraic-data-types-mult}}
\begin{enumerate}
  \item
    \[
      \inferrule
      { \wftd{\tau_1}\\\cdots\\\wftd{\tau_n}\\\wftd{\tau} }
      { \wftd{\tarrow{(\tau_1,\cdots,\tau_n)}{\tau}} }
    \]
  \item
    \[
      \inferrule
      { \wftd{\tau_1}\\\cdots\\\wftd{\tau_n} \\
        \typeof{\Gamma[x_1:\tau_1,\cdots,x_n:\tau_n]}{e}{\tau} }
      { \typeofd{\efun{(x_1{:}\tau_1,\cdots,x_n{:}\tau_n)}{e}}{\tarrow{(\tau_1,\cdots,\tau_n)}{\tau}} }
    \]

    \[
      \inferrule
      { \typeofd{e}{\tarrow{(\tau_1,\cdots,\tau_n)}{\tau}} \\
        \typeofd{e_1}{\tau_1}\\\cdots\\\typeofd{e_n}{\tau_n} }
      { \typeofd{e(e_1,\cdots,e_n)}{\tau} }
    \]

    \[
      \inferrule
      {
        \Gamma'=\Gamma[t=x_1@(\tau_{11},\cdots,\tau_{1m_1})+\cdots+x_n@(\tau_{n1},\cdots,\tau_{nm_n}),
        x_1:\tarrow{(\tau_{11},\cdots,\tau_{1m_1})}{t},\cdots,x_n:\tarrow{(\tau_{n1},\cdots,\tau_{nm_n})}{t}] \\
        \wft{\Gamma'}{\tau_{11}} \\\cdots\\ \wft{\Gamma'}{\tau_{nm_n}} \\
        \typeof{\Gamma'}{e}{\tau}
      }
      { \typeofd{\textsf{type}\
      t=x_1@(\tau_{11},\cdots,\tau_{1m_1})+\cdots+x_n@(\tau_{n1},\cdots,\tau_{nm_n})\
      \textsf{in}\ e}{\tau} }
    \]

    \[
      \inferrule
      {
        \typeofd{e}{t} \\
        t\in\dom{\Gamma} \\
        \Gamma(t)=x_1@(\tau_{11},\cdots,\tau_{1m_1})+\cdots+x_n@(\tau_{n1},\cdots,\tau_{nm_n}) \\
        \typeof{\Gamma[x_{11}:\tau_{11},\cdots,x_{1m_1}:\tau_{1m_1}]}{e_1}{\tau}
        \quad\cdots\quad
        \typeof{\Gamma[x_{n1}:\tau_{n1},\cdots,x_{nm_n}:\tau_{nm_n}]}{e_n}{\tau}
      }
      { \typeofd{e\ \textsf{match}\ x_1(x_{11},\cdots,x_{1m_1})\rightarrow
      e_1,\cdots,x_n(x_{n1},\cdots,x_{nm_n})\rightarrow e_n}{\tau} }
    \]
  \item
    $\Gamma=[\code{Fruit}=\code{Apple}@()+\code{Banana}@(\tarrow{(\code{Fruit})}{\tnum},\code{Fruit})+\code{Cherry}@(\tnum),
        \code{Apple}:\tarrow{()}{\code{Fruit}},
        \code{Banana}:\tarrow{(\tarrow{(\code{Fruit})}{\tnum},\code{Fruit})}{\code{Fruit}},
        \code{Cherry}:\tarrow{(\tnum)}{\code{Fruit}}
        ]$\\
    $\Gamma'=\Gamma[\cf:\tarrow{(\code{Fruit})}{\tnum},\cx:\tnum]$\\
    $e'={\code{Apple}()\ \textsf{match}\ \code{Apple}()\rightarrow
          42,\code{Banana}(\cf,\cx)\rightarrow\cf(\cx),\code{Cherry}(\cx)\rightarrow\cx}$\\
    $e={\textsf{type}\
    \code{Fruit}=\code{Apple}@()+\code{Banana}@(\tarrow{(\code{Fruit})}{\tnum},\code{Fruit})+
          \code{Cherry}@(\tnum)\ \textsf{in}\ e'
          }$\\
    \[
      \scriptsize
      T:
      \inferrule
      {
        \inferrule
        {
          \inferrule
          { \code{Apple}\in\dom{\Gamma} }
          { \typeofd{\code{Apple}}{\tarrow{()}{\code{Fruit}}} }
        }
        { \typeofd{\code{Apple}()}{\code{Fruit}} }
        \\
        \code{Fruit}\in\dom{\Gamma}
        \\
        \Gamma(\code{Fruit})=\code{Apple}@()+\code{Banana}@(\tarrow{(\code{Fruit})}{\tnum},\code{Fruit})+\code{Cherry}@(\tnum)
        \\
        \typeofd{42}{\tnum}
        \\
        \inferrule
        {
          \inferrule
          { \cf\in\dom{\Gamma'} }
          { \typeof{\Gamma'}{\cf}{\tarrow{(\code{Fruit})}{\tnum}} }
          \\
          \inferrule
          { \cx\in\dom{\Gamma'} }
          { \typeof{\Gamma'}{\cx}{\code{Fruit}} }
        }
        { \typeof{\Gamma'}{\cf(\cx)}{\tnum} }
        \\
        \inferrule
        { \cx\in\dom{\Gamma[\cx:\tnum]} }
        { \typeof{\Gamma[\cx:\tnum]}{\cx}{\tnum} }
      }
      {
        \typeofd{e'}{\tnum}
      }
    \]
    \[
      \scriptsize
      \inferrule
      {
        \Gamma=\Gamma
        \quad
        \inferrule
        {
          \inferrule
          { \code{Fruit}\in\dom{\Gamma} }
          { \wftd{\code{Fruit}} }
          \\
          \wftd{\tnum}
        }
        { \wftd{\tarrow{(\code{Fruit})}{\tnum}} }
        \quad
        \inferrule
        { \code{Fruit}\in\dom{\Gamma} }
        { \wftd{\code{Fruit}} }
        \quad
        \inferrule
        { \code{Fruit}\in\dom{\Gamma} }
        { \wftd{\code{Fruit}} }
        \quad
        T
      }
      { \typeofe{e}{\tnum} }
    \]
\end{enumerate}

\textbf{\refex{algebraic-data-types-nonterminate}}

$\begin{array}{@{}l@{}}
  \textsf{type}\ \cX=\code{toX}@(\tarrow{\cX}{\cX}); \\
  \textsf{val}\ \code{fromX}:\tarrow{\cX}{(\tarrow{\cX}{\cX})}=\efunt{\cx}{\cX}{\cx\ \textsf{match}\ \code{toX}(\cf)\rightarrow\cf};\\
  \textsf{val}\ \cf:\cX=\eapp{\code{toX}}{(\efunt{\cx}{\cX}{(\eapp{(\eapp{\code{fromX}}{\cx})}{\cx})})};\\
  \eapp{(\eapp{\code{fromX}}{\cf})}{\cf}
\end{array}$
\\

\textbf{\refex{parametric-polymorphism-typeck}}
\begin{enumerate}
  \item
    $\Gamma_1=[\cf:{\tforall{\alpha}{\tarrow{\alpha}{\alpha}}}]$\\
    $\Gamma_2=[\alpha,\cx:\alpha]$\\
    \[
      \tiny
      \inferrule
      {
        \inferrule
        {
          \inferrule
          {
            \inferrule
            {
              \inferrule
              { \alpha\in\dom{[\alpha]} }
              { \wft{[\alpha]}{\alpha} }
              \\
              \inferrule
              { \alpha\in\dom{[\alpha]} }
              { \wft{[\alpha]}{\alpha} }
            }
            { \wft{[\alpha]}{\tarrow{\alpha}{\alpha}} }
          }
          { \wft{\emptyset}{\tforall{\alpha}{\tarrow{\alpha}{\alpha}}} }
          \\
          \inferrule
          {
            \inferrule
            {
              \inferrule
              { \cf\in\dom{\Gamma_1} }
              { \typeof{\Gamma_1}{\cf}{\tforall{\alpha}{\tarrow{\alpha}{\alpha}}} }
            }
            { \typeof{\Gamma_1}{\etapp{\cf}{\tnum}}{\tarrow{\tnum}{\tnum}} }
            \\
            \typeof{\Gamma_1}{10}{\tnum}
          }
          { \typeof{\Gamma_1}{\etapp{\cf}{\tnum}\ 10}{\tnum} }
        }
        {
          \typeofe{\efunt{\cf}{\tforall{\alpha}{\tarrow{\alpha}{\alpha}}}{
          \etapp{\cf}{\tnum}\
          10}}{\tarrow{(\tforall{\alpha}{\tarrow{\alpha}{\alpha}})}{\tnum}}
        }
        \\
        \inferrule
        {
          \inferrule
          {
            \inferrule
            { \alpha\in\dom{[\alpha]} }
            { \wft{[\alpha]}{\alpha} }
            \\
            \inferrule
            { \cx\in\dom{\Gamma_2} }
            { \typeof{\Gamma_2}{\cx}{\alpha} }
          }
          {
            \typeof{[\alpha]}{\efunt{\cx}{\alpha}{\cx}}{\tarrow{\alpha}{\alpha}}
          }
        }
        {
          \typeofe{\etfun{\alpha}{\efunt{\cx}{\alpha}{\cx}}}{\tforall{\alpha}{\tarrow{\alpha}{\alpha}}}
        }
      }
      { \typeofe{(\efunt{\cf}{\tforall{\alpha}{\tarrow{\alpha}{\alpha}}}{
        \etapp{\cf}{\tnum}\ 10})\
        (\etfun{\alpha}{\efunt{\cx}{\alpha}{\cx}})}{\tnum} }
    \]
  \item
    $\Gamma_1=[\alpha,\beta]$\\
    $\Gamma_2=\Gamma_1[\cf:\tarrow{\alpha}{\beta}]$\\
    $\Gamma_3=\Gamma_2[\cx:\alpha]$\\
    $\Gamma_4=[\cy:\tnum]$\\
    \[
      T:
      \tiny
      \inferrule
      {
        \inferrule
        {
          \inferrule
          {
            \inferrule
            {
              \inferrule
              {
                \inferrule
                {
                  \inferrule
                  { \alpha\in\dom{\Gamma_1} }
                  { \wft{\Gamma_1}{\alpha} }
                  \\
                  \inferrule
                  { \beta\in\dom{\Gamma_1} }
                  { \wft{\Gamma_1}{\beta} }
                }
                { \wft{\Gamma_1}{\tarrow{\alpha}{\beta}} }
                \\
                \inferrule
                {
                  \inferrule
                  { \alpha\in\dom{\Gamma_2} }
                  { \wft{\Gamma_2}{\alpha} }
                  \\
                  \inferrule
                  {
                    \inferrule
                    { \cf\in\dom{\Gamma_3} }
                    { \typeof{\Gamma_3}{\cf}{\tarrow{\alpha}{\beta}} }
                    \\
                    \inferrule
                    { \cx\in\dom{\Gamma_3} }
                    { \typeof{\Gamma_3}{\cx}{\alpha} }
                  }
                  { \typeof{\Gamma_3}{\cf\ \cx}{\beta} }
                }
                {
                  \typeof{\Gamma_2}
                  {\efunt{\cx}{\alpha}{\cf\ \cx}}{\tarrow{\alpha}{\beta}}
                }
              }
              { \typeof{\Gamma_1}{
                  \efunt{\cf}{\tarrow{\alpha}{\beta}}{
                    \efunt{\cx}{\alpha}{\cf\ \cx}}}
                {\tarrow{(\tarrow{\alpha}{\beta})}{(\tarrow{\alpha}{\beta})}}
              }
            }
            { \typeof{[\alpha]}{\etfun{\beta}{
                \efunt{\cf}{\tarrow{\alpha}{\beta}}{
                  \efunt{\cx}{\alpha}{\cf\ \cx}}}}
              {\tforall{\beta}{\tarrow{(\tarrow{\alpha}{\beta})}{(\tarrow{\alpha}{\beta})}}}
            }
          }
          { \typeofe
            {\etfun{\alpha}{\etfun{\beta}{
              \efunt{\cf}{\tarrow{\alpha}{\beta}}{
                \efunt{\cx}{\alpha}{\cf\ \cx}}}}}
            {\tforall{\alpha}{\tforall{\beta}{\tarrow{(\tarrow{\alpha}{\beta})}{(\tarrow{\alpha}{\beta})}}}}
          }
        }
        {
          \typeofe
          {\etapp{(\etfun{\alpha}{\etfun{\beta}{
            \efunt{\cf}{\tarrow{\alpha}{\beta}}{
              \efunt{\cx}{\alpha}{\cf\ \cx}
            }
          }})}{\tnum}}
          {\tforall{\beta}{\tarrow{(\tarrow{\tnum}{\beta})}{(\tarrow{\tnum}{\beta})}}}
        }
      }
      { \typeofe
        {\etapp{\etapp{(\etfun{\alpha}{\etfun{\beta}{
          \efunt{\cf}{\tarrow{\alpha}{\beta}}{
            \efunt{\cx}{\alpha}{\cf\ \cx}
          }
        }})}{\tnum}}{\tnum}
        }
        {\tarrow{(\tarrow{\tnum}{\tnum})}{(\tarrow{\tnum}{\tnum})}}
      }
    \]
    \[
      \tiny
      \inferrule
      {
        \inferrule
        {
          T
          \\
          \inferrule
          {
            \wft{\emptyset}{\tnum} \\
            \inferrule
            {
              \typeof{\Gamma_4}{17}{\tnum}
              \\
              \inferrule
              { \cy\in\dom{\Gamma_4} }
              { \typeof{\Gamma_4}{\cy}{\tnum} }
            }
            { \typeof{\Gamma_4}{17-\cy}{\tnum} }
          }
          { \typeofe{\efunt{\cy}{\tnum}{17-\cy}}{\tarrow{\tnum}{\tnum}} }
        }
        { \typeofe{\etapp{\etapp{(\etfun{\alpha}{\etfun{\beta}{
            \efunt{\cf}{\tarrow{\alpha}{\beta}}{
              \efunt{\cx}{\alpha}{\cf\ \cx}
            }
          }})}{\tnum}}{\tnum}\ (\efunt{\cy}{\tnum}{17-\cy})}
          {\tarrow{\tnum}{\tnum}} }
        \\
        \typeofe{9}{\tnum}
      }
      {
        \typeofe{
          \etapp{\etapp{(\etfun{\alpha}{\etfun{\beta}{
            \efunt{\cf}{\tarrow{\alpha}{\beta}}{
              \efunt{\cx}{\alpha}{\cf\ \cx}
            }
          }})}{\tnum}}{\tnum}\ (\efunt{\cy}{\tnum}{17-\cy})\ 9
        }{\tnum}
      }
    \]
\end{enumerate}

\textbf{\refex{parametric-polymorphism-nameless}}
\vspace{-1em}
\begin{verbatim}
case class ArrowT(p: Type, r: Type) extends Type
case class ForallT(t: Type) extends Type
case class VarT(i: Int) extends Type

case NumT => Nameless.NumT
case ArrowT(p, r) =>
  Nameless.ArrowT(transform(p, ctx), transform(r, ctx))
case ForallT(a, t) =>
  Nameless.ForallT(transform(t, a :: ctx))
case VarT(a) =>
  Nameless.VarT(locate(a, ctx))
\end{verbatim}

\textbf{\refex{parametric-polymorphism-adt}}
\begin{enumerate}
  \item
    \[
      \inferrule
      {
        \Gamma'=\Gamma[t=[\alpha]x_1@\tau_1+x_2@\tau_2,x_1:\tforall{\alpha}{\tarrow{\tau_1}{t[\alpha]}},x_2:\tforall{\alpha}{\tarrow{\tau_2}{t[\alpha]}}] \\
        t\not\in\dom{\Gamma} \\
        \Gamma'[\alpha]\vdash\tau_1 \\
        \Gamma'[\alpha]\vdash\tau_2 \\
        \typeof{\Gamma'}{e}{\tau} \\
        \wftd{\tau}
      }
      { \typeofd{\etdef{t[\alpha]}{x_1}{\tau_1}{x_2}{\tau_2}{e}}{\tau} }
    \]

    \[
      \inferrule
      {
        t\in\dom{\Gamma} \\
        \Gamma(t)=[\alpha]x_1@\tau_1+x_2@\tau_2 \\
        \typeofd{e}{t[\tau]} \\
        \typeof{\Gamma[x_3:\tau_1[\alpha\leftarrow\tau]]}{e_1}{\tau'} \\
        \typeof{\Gamma[x_4:\tau_2[\alpha\leftarrow\tau]]}{e_2}{\tau'}
      }
      { \typeofd{\ematch{e}{x_1}{x_3}{e_1}{x_2}{x_4}{e_2}}{\tau'} }
    \]
  \item
    \[
      \inferrule
      { t\in\dom{\Gamma} \\ \wftd{\tau} }
      { \Gamma\vdash t[\tau] }
    \]
  \item
    $\tforall{\alpha}{\tarrow{\code{option}[\alpha]}{\tarrow{\alpha}{\alpha}}}$
\end{enumerate}

\textbf{\refex{subtype-polymorphism-welltyped}}
\begin{enumerate}
  \item Not well-typed
  \item
  \[
    \scriptsize
    \inferrule
    {
      \typeofe{1}{\tnum} \\
      \typeofe{\{\}}{\{\}} \\
      \inferrule
      {
        \inferrule
        { \typeof{\emptyset}{2}{\tnum} }
        { \typeof{\emptyset}{\{\ca=2\}}{\{\ca:\tnum\}} }
        \\
        \subt{\{\ca:\tnum\}}{\{\}}
      }
      { \typeofe{\{\ca=2\}}{\{\}} }
    }
    { \typeofe{\eifz{1}{\{\}}{\{\ca=2\}}}{\{\}} }
  \]
\end{enumerate}

\textbf{\refex{subtype-polymorphism-arrow}}

$\eapp{(\efunt{\cx}{\tarrow{\tnum}{\tnum}}{\eapp{\cx}{1}})}{1}$
\\

\textbf{\refex{subtype-polymorphism-subtyper}}
\vspace{-1em}
\begin{verbatim}
case (_, TopT) => true
case (BottomT, _) => true
case (NumT, NumT) => true
case (ArrowT(p1, r1), ArrowT(p2, r2)) =>
  subtype(p2, p1) && subtype(r1, r2)
case (RecordT(fs1), RecordT(fs2)) =>
  fs2.forall{
    case (x, t2) => fs1.get(x) match {
      case None => false
      case Some(t1) => subtype(t1, t2)
    }
  }
\end{verbatim}

\textbf{\refex{subtype-polymorphism-typer}}
\vspace{-1em}
\begin{verbatim}
case Add(l, r) =>
  val lt = typeCheck(l, tenv)
  val rt = typeCheck(r, tenv)
  if (!subtype(lt, NumT)) error()
  if (!subtype(rt, NumT)) error()
  NumT
case App(f, a) =>
  val ft = typeCheck(f, tenv)
  val at = typeCheck(a, tenv)
  ft match {
    case NumT => error()
    case ArrowT(pt, rt) =>
      if (!subtype(at, pt)) error()
      rt
    case TopT => error()
    case BottomT => BottomT
  }
\end{verbatim}

\textbf{\refex{subtype-polymorphism-lattice}}
\vspace{-1em}
\begin{verbatim}
case (BottomT, t) => t
case (t, BottomT) => t
case (NumT, NumT) => NumT
case (ArrowT(p1, r1), ArrowT(p2, r2)) =>
  ArrowT(glb(p1, p2), lub(r1, r2))
case (RecordT(fs1), RecordT(fs2)) =>
  val fs = fs1.keySet & fs2.keySet
  RecordT(fs.map(x => x -> lub(fs1(x), fs2(x))).toMap)

case (TopT, t) => t
case (t, TopT) => t
case (NumT, NumT) => NumT
case (ArrowT(p1, r1), ArrowT(p2, r2)) =>
  ArrowT(lub(p1, p2), glb(r1, r2))
case (RecordT(fs1), RecordT(fs2)) =>
  val fs = fs1.keySet | fs2.keySet
  RecordT(fs.map(x =>
    x -> glb(fs1.getOrElse(x, TopT), fs2.getOrElse(x, TopT))
  ).toMap)

case If0(c, t, f) =>
  val ct = typeCheck(c, tenv)
  val tt = typeCheck(t, tenv)
  val ft = typeCheck(f, tenv)
  if (subtype(ct, NumT)) lub(tt, ft) else error()
\end{verbatim}

\textbf{\refex{subtype-polymorphism-box}}
\[
\inferrule
{ \subt{\tau_1}{\tau_2} \\ \subt{\tau_2}{\tau_1} }
{ \subt{\textsf{box}\ \tau_1}{\textsf{box}\ \tau_2} }
\]

\textbf{\refex{subtype-polymorphism-list}}
\begin{enumerate}
  \item
    \[
      \inferrule
      { \subt{\tau_1}{\tau_2} }
      { \subt{\textsf{list}\ \tau_1}{\textsf{list}\ \tau_2} }
    \]
  \item
    \[
      \inferrule
      { \subt{\tau_1}{\tau_2} \\ \subt{\tau_2}{\tau_1} }
      { \subt{\textsf{list}\ \tau_1}{\textsf{list}\ \tau_2} }
    \]
\end{enumerate}

\textbf{\refex{subtype-polymorphism-vcc}}
\begin{enumerate}
  \item
    \[
      \inferrule
      { \typeof{\Gamma[x:\tarrow{\tau}{\tbot}]}{e}{\tau} }
      { \typeofd{(\evcc{x}{e}){:}\tau}{\tau} }
    \]
  \item
    \[
      \scriptsize
      \inferrule
      {
        \inferrule
        {
          \inferrule
          {
            \inferrule
            {
              \inferrule
              { \cx\in\dom{\Gamma} }
              { \typeofd{\cx}{\tarrow{\tnum}{\tbot}} }
              \\
              \typeofd{1}{\tnum}
            }
            { \typeofd{\eapp{\cx}{1}}{\tbot} }
            \\
            \subt{\tbot}{\tarrow{\tnum}{\tnum}}
          }
          { \typeofd{\eapp{\cx}{1}}{\tarrow{\tnum}{\tnum}} }
          \\
          \typeofd{42}{\tnum}
        }
        { \typeofd{\eapp{(\eapp{\cx}{1})}{42}}{\tnum} }
      }
      { \typeofe{(\evcc{\cx}{\eapp{(\eapp{\cx}{1})}{42}}){:}\tnum}{\tnum} }
    \]

    $\Gamma=[\cx:\tarrow{\tnum}{\tbot}]$
\end{enumerate}

\textbf{\refex{subtype-polymorphism-intersection}}
\begin{enumerate}
  \item
    \[
      \inferrule{\subt{\tau_1}{\tau_3}}{\subt{\tau_1\land\tau_2}{\tau_3}}
    \]

    \[
      \inferrule{\subt{\tau_2}{\tau_3}}{\subt{\tau_1\land\tau_2}{\tau_3}}
    \]
  \item
    \[
      \inferrule{\subt{\tau_3}{\tau_1}\\\subt{\tau_3}{\tau_2}}{\subt{\tau_3}{\tau_1\land\tau_2}}
    \]
  \item
    \[
      \inferrule
      {
        \inferrule
        { \subt{\tau_2}{\tau_2} }
        { \subt{\tau_1\land\tau_2}{\tau_2} } \\
        \inferrule
        { \subt{\tau_1}{\tau_1} }
        { \subt{\tau_1\land\tau_2}{\tau_1} }
      }
      { \subt{\tau_1\land\tau_2}{\tau_2\land\tau_1} }
    \]
\end{enumerate}

\textbf{\refex{subtype-polymorphism-union}}
\begin{enumerate}
  \item
    \[
      \inferrule
      { \subt{\tau_1}{\tau_3} \\ \subt{\tau_2}{\tau_3} }
      { \subt{(\tau_1\lor\tau_2)}{\tau_3} }
    \]
  \item
    \[
      \inferrule
      { \subt{\tau_3}{\tau_1} }
      { \subt{\tau_3}{(\tau_1\lor\tau_2)} }
    \]

    \[
      \inferrule
      { \subt{\tau_3}{\tau_2} }
      { \subt{\tau_3}{(\tau_1\lor\tau_2)} }
    \]
  \item
    \[
      \scriptsize
      \inferrule
      {
        \inferrule
        {
          \inferrule
          { \subt{\tau_1}{\tau_1} }
          { \subt{\tau_1}{(\tau_1\lor(\tau_2\lor\tau_3))} }
          \\
          \inferrule
          {
            \inferrule
            { \subt{\tau_2}{\tau_2} }
            { \subt{\tau_2}{(\tau_2\lor\tau_3)} }
          }
          { \subt{\tau_2}{(\tau_1\lor(\tau_2\lor\tau_3))} }
        }
        { \subt{(\tau_1\lor\tau_2)}{(\tau_1\lor(\tau_2\lor\tau_3))} }
        \\
        \inferrule
        {
          \inferrule
          { \subt{\tau_3}{\tau_3} }
          { \subt{\tau_3}{(\tau_2\lor\tau_3)} }
        }
        { \subt{\tau_3}{(\tau_1\lor(\tau_2\lor\tau_3))} }
      }
      { \subt{((\tau_1\lor\tau_2)\lor\tau_3)}{(\tau_1\lor(\tau_2\lor\tau_3))} }
    \]
\end{enumerate}

\textbf{\refex{subtype-polymorphism-adt}}
\[
  \inferrule
  {
    \Gamma(t_1)=x_1@\tau_1+\cdots+x_n@\tau_n \\
    \Gamma(t_2)=x_1@\tau_1+\cdots+x_n@\tau_n+x_{n+1}@\tau_{n+1}+\cdots+x_{n+m}@\tau_{n+m} \\
    \Gamma\vdash\subt{\tau_1}{\tau_1'} \\
    \cdots \\
    \Gamma\vdash\subt{\tau_n}{\tau_n'}
  }
  { \Gamma\vdash\subt{t_1}{t_2} }
\]

\textbf{\refex{subtype-polymorphism-exception}}
\[
  { \typeofd{n}{\tnum\uparrow\tbot} }
\]
\[
  \inferrule
  { \typeofd{e_1}{\tnum\uparrow\tau} \\ \typeofd{e_2}{\tnum\uparrow\tau} }
  { \typeofd{e_1+e_2}{\tnum\uparrow\tau} }
\]
\[
  \inferrule
  { x\in\dom{\Gamma} }
  { \typeofd{x}{\Gamma(x)\uparrow\tbot} }
\]
\[
  \inferrule
  { \typeof{\Gamma[x:\tau_1]}{e}{\tau_2\uparrow\tau_3} }
  { \typeofd{\efunt{x}{\tau_1}{e}}{(\tarrow{\tau_1}{\tau_2/\uparrow\tau_3})\uparrow\tbot} }
\]
\[
  \inferrule
  { \typeofd{e_1}{(\tarrow{\tau_1}{\tau_2/\uparrow\tau_3})\uparrow\tau_3} \\
    \typeofd{e_2}{\tau_1\uparrow\tau_3} }
  { \typeofd{e_1\ e_2}{\tau_2\uparrow\tau_3} }
\]
\[
  \inferrule
  { \typeofd{e}{\tau\uparrow\tau} }
  { \typeofd{\textsf{throw}\ e}{\tbot\uparrow\tau} }
\]
\[
  \inferrule
  { \typeofd{e_1}{\tau_1\uparrow\tau_2} \\
    \typeofd{e_2}{(\tarrow{\tau_2}{\tau_1/\uparrow\tau_3})\uparrow\tau_3}}
  { \typeofd{\textsf{try}\ e_1\ \textsf{catch}\ e_2}{\tau_1\uparrow\tau_3} }
\]

\textbf{\refex{type-inference-pair}}
\begin{enumerate}
  \item
\begin{verbatim}
case (PairT(l1, r1), PairT(l2, r2)) =>
  unify(l1, l2)
  unify(r1, r2)

case PairT(l, r) =>
  occurs(t1, l) || occurs(t1, r)

case Pair(l, r) =>
  val lt = typeCheck(l, tenv)
  val rt = typeCheck(r, tenv)
  PairT(lt, rt)
case Fst(e) =>
  val et = typeCheck(e, tenv)
  val ft = VarT(None)
  val st = VarT(None)
  unify(PairT(ft, st), et)
  ft
case Snd(e) =>
  val et = typeCheck(e, tenv)
  val ft = VarT(None)
  val st = VarT(None)
  unify(PairT(ft, st), et)
  st
\end{verbatim}
  \item
    \verb+ArrowT(VarT(Some(PairT(VarT(Some(NumT)), VarT(None)))), NumT)+
\end{enumerate}

\textbf{\refex{type-inference-box}}
\vspace{-1em}
\begin{verbatim}
case (BoxT(t3), BoxT(t4)) =>
  unify(t3, t4)

case BoxT(t) =>
  occurs(t1, t)

case NewBox(e) =>
  val et = typeCheck(e, tenv)
  BoxT(et)
case OpenBox(b) =>
  val bt = typeCheck(b, tenv)
  val t = VarT(None)
  unify(bt, BoxT(t))
  t
case SetBox(b, e) =>
  val bt = typeCheck(b, tenv)
  val et = typeCheck(e, tenv)
  unify(bt, BoxT(et))
  et
\end{verbatim}

\textbf{\refex{type-inference-list}}
\vspace{-1em}
\begin{verbatim}
case (ListT(t3), ListT(t4)) =>
  unify(t3, t4)

case ListT(t) =>
  occurs(t1, t)

case Nil =>
  val t = VarT(None)
  ListT(t)
case Cons(h, t) =>
  val ht = typeCheck(h, tenv)
  val tt = typeCheck(t, tenv)
  unify(ListT(ht), tt)
  tt
case Head(e) =>
  val et = typeCheck(e, tenv)
  val t = VarT(None)
  unify(et, ListT(t))
  t
case Tail(e) =>
  val et = typeCheck(e, tenv)
  val t = VarT(None)
  unify(et, ListT(t))
  et
\end{verbatim}

\textbf{\refex{type-inference-option}}
\vspace{-1em}
\begin{verbatim}
case (OptionT(t3), OptionT(t4)) =>
  unify(t3, t4)

case OptionT(t) =>
  occurs(t1, t)

case NoneE =>
  OptionT(VarT(None))
case SomeE(e) =>
  OptionT(typeCheck(e, tenv))
case Match(e, e1, x, e2) =>
  val et = typeCheck(e, tenv)
  val t = VarT(None)
  unify(OptionT(t), et)
  val nt = typeCheck(e1, tenv)
  val st = typeCheck(e2, tenv + (x -> t))
  unify(nt, st)
  nt
\end{verbatim}

\textbf{\refex{type-inference-sysf}}
\begin{enumerate}
  \item
    \[
      \begin{array}{rcl}
        \ersr{n}&=&n \\
        \ersr{\eadd{e_1}{e_2}}&=&\eadd{\ersr{e_1}}{\ersr{e_2}} \\
        \ersr{\esub{e_1}{e_2}}&=&\esub{\ersr{e_1}}{\ersr{e_2}} \\
        \ersr{x}&=&x \\
        \ersr{\efunt{x}{\tau}{e}}&=&\efun{x}{\ersr{e}} \\
        \ersr{\eapp{e_1}{e_2}}&=&\eapp{\ersr{e_1}}{\ersr{e_2}} \\
        \ersr{\etfun{\alpha}{e}}&=&\ersr{e} \\
        \ersr{\etapp{e}{\tau}}&=&\ersr{e} \\
      \end{array}
    \]
  \item
    \begin{enumerate}
      \item $\eapp{(\efun{\cx}{\cx})}{1}$
      \item $\eapp{\eapp{(\efun{\cx}{\efun{\cy}{\cy}})}{1}}{2}$
    \end{enumerate}
  \item
    \begin{enumerate}
      \item
        $\efunt{\cx}{\tforall{\alpha}{\tarrow{\alpha}{\alpha}}}
        {\eapp{\eapp{(\etapp{\cx}{\tarrow{\tnum}{\tnum}})}{(\efunt{\cx}{\tnum}{\cx})}}{(\eapp{\etapp{\cx}{\tnum}}{1})}}$
      \item
        $\eapp{
          \eapp{(\efunt{\cx}{\tforall{\alpha}{\tarrow{\alpha}{\alpha}}}{\efunt{\cy}{\tnum}{\eapp{\eapp{(\etapp{\cx}{\tarrow{\tnum}{\tnum}})}{(\etapp{\cx}{\tnum})}}{\cy}}})}
          {(\etfun{\alpha}{\efunt{\cx}{\alpha}{\cx}})}
        }{1}$
      \item
        $\eapp{(\efunt{\cx}{\tforall{\alpha}{\tarrow{\alpha}{\alpha}}}{\eapp{\eapp{(\etapp{\cx}{\tarrow{\tnum}{\tnum}})}{(\etapp{\cx}{\tnum})}}{1}})}
        {\etfun{\alpha}{\efunt{\cy}{\alpha}{\cy}}}$
    \end{enumerate}
\end{enumerate}

\textbf{\refex{type-inference-hm}}
\begin{enumerate}
  \item
    \[
      \tiny
      \inferrule
      {
        \inferrule
        {
          \inferrule
          {
            \inferrule
            {
              \cx\in\dom{\Gamma_1} \\
              \tarrow{\tnum}{\tnum}\succ\tarrow{\tnum}{\tnum}
            }
            { \typeof{\Gamma_1}{\cx}{\tarrow{\tnum}{\tnum}} }
            \\
            \typeof{\Gamma_1}{42}{\tnum}
          }
          { \typeof{\Gamma_1}{\eapp{\cx}{42}}{\tnum} }
        }
        { \typeofe{\efun{\cx}{\eapp{\cx}{42}}}{\tarrow{(\tarrow{\tnum}{\tnum})}{\tnum}} } \\
        \inferrule
        {
          \inferrule
          { \cy\in\dom{\Gamma_2} \\ \tnum\succ\tnum }
          { \typeof{\Gamma_2}{\cy}{\tnum} }
        }
        { \typeofe{\efun{\cy}{\cy}}{\tarrow{\tnum}{\tnum}} }
      }
      { \typeofe{\eapp{(\efun{\cx}{\eapp{\cx}{42}})}{\efun{\cy}{\cy}}}{\tnum} }
    \]
    $\Gamma_1=[\cx:\tarrow{\tnum}{\tnum}]$, $\Gamma_2=[\cy:\tnum]$
  \item Not well-typed
  \item
    \[
      \tiny
      \inferrule
      {
        \inferrule
        {
          \inferrule
          { \cy\in\dom{\Gamma_1} \\ \alpha\succ\alpha }
          { \typeof{\Gamma_1}{\cy}{\alpha} }
        }
        { \typeofe{\efun{\cy}{\cy}}{\tarrow{\alpha}{\alpha}} }
        \\
        \inferrule
        { \embox{FTV}(\tarrow{\alpha}{\alpha})\setminus\embox{FTV}(\emptyset)=\{\alpha\} }
        { \tarrow{\alpha}{\alpha}\prec_\emptyset\tforall{\alpha}{\tarrow{\alpha}{\alpha}} }
        \\
        \inferrule
        {
          \inferrule
          {
            \inferrule
            {
              x\in\dom{\Gamma_2} \\
              \tforall{\alpha}{\tarrow{\alpha}{\alpha}}\succ\tarrow{\tnum}{\tnum}
            }
            { \typeof{\Gamma_2}{\cx}{\tarrow{\tnum}{\tnum}} }
            \\
            \typeof{\Gamma_2}{42}{\tnum}
          }
          { \typeof{\Gamma_2}{\eapp{\cx}{42}}{\tnum} }
          \inferrule
          {
            \inferrule
            {
              x\in\dom{\Gamma_2} \\
              \tforall{\alpha}{\tarrow{\alpha}{\alpha}}\succ\tarrow{\tbool}{\tbool}
            }
            { \typeof{\Gamma_2}{\cx}{\tarrow{\tbool}{\tbool}} }
            \\
            \typeof{\Gamma_2}{\true}{\tbool}
          }
          { \typeof{\Gamma_2}{\eapp{\cx}{\true}}{\tbool} }
        }
        { \typeof{\Gamma_2}{\eseq{\eapp{\cx}{42}}{\eapp{\cx}{\true}}}{\tbool} }
      }
      { \typeofe{\ebind{\cx}{\efun{\cy}{\cy}}{\eseq{\eapp{\cx}{42}}{\eapp{\cx}{\true}}}}{\tbool} }
    \]

    $\Gamma_1=[\cy:\alpha]$,
    $\Gamma_2=[\cx:\forall\alpha.\alpha\rightarrow\alpha]$
\end{enumerate}


%----------------------------------------------------------------------------------------

\backmatter % Denotes the end of the main document content
\setchapterstyle{plain} % Output plain chapters from this point onwards

%----------------------------------------------------------------------------------------
%	BIBLIOGRAPHY
%----------------------------------------------------------------------------------------

% The bibliography needs to be compiled with biber using your LaTeX editor, or on the command line with 'biber main' from the template directory

\defbibnote{bibnote}{Here are the references in citation order.\par\bigskip} % Prepend this text to the bibliography
\printbibliography[heading=bibintoc, title=Bibliography, prenote=bibnote] % Add the bibliography heading to the ToC, set the title of the bibliography and output the bibliography note

%----------------------------------------------------------------------------------------
%	NOMENCLATURE
%----------------------------------------------------------------------------------------

% The nomenclature needs to be compiled on the command line with 'makeindex main.nlo -s nomencl.ist -o main.nls' from the template directory

% \nomenclature{$c$}{Speed of light in a vacuum inertial frame}
% \nomenclature{$h$}{Planck constant}

% \renewcommand{\nomname}{Notation} % Rename the default 'Nomenclature'
% \renewcommand{\nompreamble}{The next list describes several symbols that will be later used within the body of the document.} % Prepend this text to the nomenclature

% \printnomenclature % Output the nomenclature

%----------------------------------------------------------------------------------------
%	GREEK ALPHABET
% 	Originally from https://gitlab.com/jim.hefferon/linear-algebra
%----------------------------------------------------------------------------------------

% \vspace{1cm}

% {\usekomafont{chapter}Greek Letters with Pronounciation} \\[2ex]
% \begin{center}
% 	\newcommand{\pronounced}[1]{\hspace*{.2em}\small\textit{#1}}
% 	\begin{tabular}{l l @{\hspace*{3em}} l l}
% 		\toprule
% 		Character & Name & Character & Name \\
% 		\midrule
% 		$\alpha$ & alpha \pronounced{AL-fuh} & $\nu$ & nu \pronounced{NEW} \\
% 		$\beta$ & beta \pronounced{BAY-tuh} & $\xi$, $\Xi$ & xi \pronounced{KSIGH} \\
% 		$\gamma$, $\Gamma$ & gamma \pronounced{GAM-muh} & o & omicron \pronounced{OM-uh-CRON} \\
% 		$\delta$, $\Delta$ & delta \pronounced{DEL-tuh} & $\pi$, $\Pi$ & pi \pronounced{PIE} \\
% 		$\epsilon$ & epsilon \pronounced{EP-suh-lon} & $\rho$ & rho \pronounced{ROW} \\
% 		$\zeta$ & zeta \pronounced{ZAY-tuh} & $\sigma$, $\Sigma$ & sigma \pronounced{SIG-muh} \\
% 		$\eta$ & eta \pronounced{AY-tuh} & $\tau$ & tau \pronounced{TOW (as in cow)} \\
% 		$\theta$, $\Theta$ & theta \pronounced{THAY-tuh} & $\upsilon$, $\Upsilon$ & upsilon \pronounced{OOP-suh-LON} \\
% 		$\iota$ & iota \pronounced{eye-OH-tuh} & $\phi$, $\Phi$ & phi \pronounced{FEE, or FI (as in hi)} \\
% 		$\kappa$ & kappa \pronounced{KAP-uh} & $\chi$ & chi \pronounced{KI (as in hi)} \\
% 		$\lambda$, $\Lambda$ & lambda \pronounced{LAM-duh} & $\psi$, $\Psi$ & psi \pronounced{SIGH, or PSIGH} \\
% 		$\mu$ & mu \pronounced{MEW} & $\omega$, $\Omega$ & omega \pronounced{oh-MAY-guh} \\
% 		\bottomrule
% 	\end{tabular} \\[1.5ex]
% 	Capitals shown are the ones that differ from Roman capitals.
% \end{center}

%----------------------------------------------------------------------------------------
%	GLOSSARY
%----------------------------------------------------------------------------------------

% The glossary needs to be compiled on the command line with 'makeglossaries main' from the template directory

% \newglossaryentry{computer}{
% 	name=computer,
% 	description={is a programmable machine that receives input, stores and manipulates data, and provides output in a useful format}
% }

% % Glossary entries (used in text with e.g. \acrfull{fpsLabel} or \acrshort{fpsLabel})
% \newacronym[longplural={Frames per Second}]{fpsLabel}{FPS}{Frame per Second}
% \newacronym[longplural={Tables of Contents}]{tocLabel}{TOC}{Table of Contents}
\newacronym[longplural={algebraic data types}]{adtLabel}{ADT}{algebraic data type}
\newacronym[longplural={real-eval-print-loop}]{replLabel}{REPL}{read-eval-print-loop}
\newacronym[longplural={Java Virtual Machines}]{jvmLabel}{JVM}{Java Virtual Machine}

\setglossarystyle{listgroup} % Set the style of the glossary (see https://en.wikibooks.org/wiki/LaTeX/Glossary for a reference)
\printglossary[title=Special Terms, toctitle=List of Terms] % Output the glossary, 'title' is the chapter heading for the glossary, toctitle is the table of contents heading

%----------------------------------------------------------------------------------------
%	INDEX
%----------------------------------------------------------------------------------------

% The index needs to be compiled on the command line with 'makeindex main' from the template directory

\printindex % Output the index

%----------------------------------------------------------------------------------------
%	BACK COVER
%----------------------------------------------------------------------------------------

% If you have a PDF/image file that you want to use as a back cover, uncomment the following lines

%\clearpage
%\thispagestyle{empty}
%\null%
%\clearpage
%\includepdf{cover-back.pdf}

%----------------------------------------------------------------------------------------

\end{document}
