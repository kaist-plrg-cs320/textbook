%%%%%%%%%%%%%%%%%%%%%%%%%%%%%%%%%%%%%%%%%
% kaobook
% LaTeX Template
% Version 1.3 (December 9, 2021)
%
% This template originates from:
% https://www.LaTeXTemplates.com
%
% For the latest template development version and to make contributions:
% https://github.com/fmarotta/kaobook
%
% Authors:
% Federico Marotta (federicomarotta@mail.com)
% Based on the doctoral thesis of Ken Arroyo Ohori (https://3d.bk.tudelft.nl/ken/en)
% and on the Tufte-LaTeX class.
% Modified for LaTeX Templates by Vel (vel@latextemplates.com)
%
% License:
% CC0 1.0 Universal (see included MANIFEST.md file)
%
%%%%%%%%%%%%%%%%%%%%%%%%%%%%%%%%%%%%%%%%%

%----------------------------------------------------------------------------------------
%  PACKAGES AND OTHER DOCUMENT CONFIGURATIONS
%----------------------------------------------------------------------------------------

\documentclass[
  b5paper, % Page size
  fontsize=10pt, % Base font size
  twoside=true, % Use different layouts for even and odd pages (in particular, if twoside=true, the margin column will be always on the outside)
  %open=any, % If twoside=true, uncomment this to force new chapters to start on any page, not only on right (odd) pages
  %chapterentrydots=true, % Uncomment to output dots from the chapter name to the page number in the table of contents
  numbers=noenddot, % Comment to output dots after chapter numbers; the most common values for this option are: enddot, noenddot and auto (see the KOMAScript documentation for an in-depth explanation)
]{kaobook}

\renewcommand{\marginlayout}{%
  \newgeometry{
    top=27.4\vscale,
    bottom=27.4\vscale,
    inner=24.8\hscale,
    textwidth=147\hscale,
    marginparsep=6.2\hscale,
    marginparwidth=0\hscale,
  }%
  \recalchead
}

% Choose the language
\ifxetexorluatex
  \usepackage{polyglossia}
  \setmainlanguage{english}
\else
  \usepackage[english]{babel} % Load characters and hyphenation
\fi
\usepackage[english=british]{csquotes}  % English quotes

% Load packages for testing
\usepackage{blindtext}
%\usepackage{showframe} % Uncomment to show boxes around the text area, margin, header and footer
%\usepackage{showlabels} % Uncomment to output the content of \label commands to the document where they are used

% Load the bibliography package
\usepackage{kaobiblio}
\addbibresource{main.bib} % Bibliography file

% Load mathematical packages for theorems and related environments
\usepackage[framed=true]{kaotheorems}

% Load the package for hyperreferences
\usepackage{kaorefs}

\usepackage{mathpartir}
\usepackage{dirtree}
\usepackage{xspace}
\usepackage{tikz}
\usepackage{fancyvrb}
\usepackage{bussproofs}
\usepackage{proof}
\usepackage{upquote}
\usepackage{arydshln}
\usepackage{graphicx}
\usepackage{amsthm}

\usetikzlibrary{arrows}
\tikzset{
  treenode/.style = {align=center, inner sep=0pt, text centered},
  arn_u/.style = {treenode, circle, black, draw=black, text width=0.8cm, line width=0.25mm},
  arn_w/.style = {treenode, circle, white, draw=white, text width=0.8cm, line width=0.25mm},
  arn_nc/.style = {treenode, circle, black, draw=white, text width=0.8cm, line width=0.25mm},
}

\newcommand{\shorteq}{\code{=}}

\newcommand{\Lang}{}
\newcommand{\plang}{}
\newcommand{\code}[1]{\texttt{#1}}
\newcommand{\cx}{\code{x}}
\newcommand{\cy}{\code{y}}
\newcommand{\cz}{\code{z}}
\newcommand{\cf}{\code{f}}
\newcommand{\cv}{\code{v}}
\newcommand{\ca}{\code{a}}
\newcommand{\cb}{\code{b}}
\newcommand{\cX}{\code{X}}
\newcommand{\cY}{\code{Y}}
\newcommand{\cZ}{\code{Z}}
\newcommand{\inred}{\color{red}}

\newcommand{\chnum}[1]{\hyperref[ch:#1]{\ref{ch:#1}}}
\newcommand{\refchs}[2]{Chapters \chnum{#1} and \chnum{#2}}
\newcommand{\refchss}[3]{Chapters \chnum{#1}, \chnum{#2}, and \chnum{#3}}
\newcommand{\refex}[1]{Exercise~\hyperref[thm:#1]{\ref{thm:#1}}}
\newcommand{\labex}[1]{\labthm{#1}}
\newcommand{\pto}{\mathrel{\ooalign{\hfil$\mapstochar$\hfil\cr$\to$\cr}}}
\newcommand{\finto}{\stackrel{\mbox{\tiny fin}}{\pto}}
\newcommand{\embox}[1]{\mbox{\emph{#1}}}
\newcommand{\dom}[1]{\embox{Domain}(#1)}
\newcommand{\iadd}{+_{{}_{\mathbb{Z}}}}
\newcommand{\isub}{-_{{}_{\mathbb{Z}}}}

\newcommand{\fundef}[3]{\textsf{def}\ #1(#2)\shorteq#3}
\newcommand{\clov}[3]{\langle\efun{#1}{#2},#3\rangle}
\newcommand{\tclov}[3]{\langle\etfun{#1}{#2},#3\rangle}
\newcommand{\contv}[2]{\langle#1,#2\rangle}
\newcommand{\exprv}[2]{(#1,#2)}
\newcommand{\eadd}[2]{#1+#2}
\newcommand{\esub}[2]{#1-#2}
\newcommand{\ebind}[3]{\textsf{val}\ #1\shorteq#2\ \textsf{in}\ #3}
\newcommand{\evcc}[2]{\textsf{vcc}\ #1\ \textsf{in}\ #2}
\newcommand{\efun}[2]{\lambda#1.#2}
\newcommand{\efunt}[3]{\lambda#1{:}#2.#3}
\newcommand{\eapp}[2]{#1\ #2}
\newcommand{\eappfo}[2]{#1(#2)}
\newcommand{\erec}[4]{\textsf{def}\ #1(#2)\shorteq#3\ \textsf{in}\ #4}
\newcommand{\eifz}[3]{\textsf{if0}\ #1\ #2\ #3}
\newcommand{\eif}[3]{\textsf{if}\ #1\ #2\ #3}
\newcommand{\eskip}{\textsf{skip}}
\newcommand{\ewhilez}[2]{\textsf{while0}\ #1\ #2}
\newcommand{\ewhile}[2]{\textsf{while}\ #1\ #2}
\newcommand{\eref}[1]{\textsf{box}\ #1}
\newcommand{\ederef}[1]{!#1}
\newcommand{\eset}[2]{#1{:}\shorteq#2}
\newcommand{\eseq}[2]{#1;#2}
\newcommand{\erect}[6]{\textsf{def}\ #1(#2{:}#3){:}#4\shorteq#5\ \textsf{in}\ #6}
\newcommand{\etdef}[6]{\textsf{type}\ #1=#2@#3+#4@#5\ \textsf{in}\ #6}
\newcommand{\ematch}[7]{#1\ \textsf{match}\ #2(#3)\rightarrow #4,#5(#6)\rightarrow #7}
\newcommand{\etfun}[2]{\Lambda#1.#2}
\newcommand{\etapp}[2]{#1[#2]}
\newcommand{\true}{\textsf{true}}
\newcommand{\false}{\textsf{false}}

\newcommand{\tnum}{\textsf{num}}
\newcommand{\tbool}{\textsf{bool}}
\newcommand{\tfun}{\textsf{fun}}
\newcommand{\tarrow}[2]{#1\rightarrow#2}
\newcommand{\tforall}[2]{\forall#1.#2}
\newcommand{\ttop}{\textsf{top}}
\newcommand{\tbot}{\textsf{bottom}}

\newcommand{\mtk}{\square}
\newcommand{\evalk}[2]{#1\vdash#2::}
\newcommand{\evalkd}[1]{\evalk{\sigma}{#1}}
\newcommand{\evalke}[1]{\evalk{\emptyset}{#1}}
\newcommand{\addk}{(+)::}
\newcommand{\subk}{(-)::}
\newcommand{\appk}{(@)::}

\newcommand{\mts}{\blacksquare}
\newcommand{\conss}[1]{#1::}

\newcommand{\eval}[3]{#1\vdash#2\Rightarrow#3}
\newcommand{\evaln}[3]{\ensuremath{#2} evaluates to \ensuremath{#3} under
\ensuremath{#1}\xspace}
\newcommand{\evald}[2]{\eval{\sigma}{#1}{#2}}
\newcommand{\evaldn}[2]{\evaln{\sigma}{#1}{#2}}
\newcommand{\evale}[2]{\eval{\emptyset}{#1}{#2}}

\newcommand{\typeof}[3]{#1\vdash#2:#3}
\newcommand{\typeofd}[2]{\typeof{\Gamma}{#1}{#2}}
\newcommand{\typeofe}[2]{\typeof{\emptyset}{#1}{#2}}
\newcommand{\typeofn}[3]{the type of \ensuremath{#2} is
\ensuremath{#3} under \ensuremath{#1}\xspace}
\newcommand{\typeofdn}[2]{\typeofn{\Gamma}{#1}{#2}}
\newcommand{\typeofnc}[3]{The type of \ensuremath{#2} is
\ensuremath{#3} under \ensuremath{#1}\xspace}
\newcommand{\typeofdnc}[2]{\typeofnc{\Gamma}{#1}{#2}}

\newcommand{\wft}[2]{#1\vdash#2}
\newcommand{\wftd}[1]{\wft{\Gamma}{#1}}
\newcommand{\wftn}[2]{\ensuremath{#2} is well-formed under \ensuremath{#1}}
\newcommand{\wftdn}[1]{\wftn{\Gamma}{#1}}

\newcommand{\subt}[2]{#1<:#2}
\newcommand{\subtn}[2]{\ensuremath{#1} is a subtype of \ensuremath{#2}}

\newcommand{\stricte}[2]{#1\Downarrow#2}
\newcommand{\stricten}[2]{\ensuremath{#1} strictly evaluates to \ensuremath{#2}}

\newcommand{\seval}[5]{#1,#2\vdash#3\Rightarrow#4,#5}
\newcommand{\sevaln}[5]{
  \ensuremath{#3} evaluates to \ensuremath{#4} and changes the store from
  \ensuremath{#2} to \ensuremath{#5} under \ensuremath{#1}\xspace
}
\newcommand{\sevald}[4]{\seval{\sigma}{M_{#1}}{#2}{#3}{M_{#4}}}
\newcommand{\sevaldn}[4]{\sevaln{\sigma}{M_{#1}}{#2}{#3}{M_{#4}}}
\newcommand{\sevale}[3]{\seval{\emptyset}{\emptyset}{#1}{#2}{#3}}

\newcommand{\semanticrule}[2]{
\vspace{1em}
\textbf{Rule \textsc{#1}}\\
#2
\vspace{1em}
}
\newcommand{\typerule}[2]{
\vspace{1em}
\textbf{Rule \textsc{#1}}\\
#2
\vspace{1em}
}
\newcommand{\tand}{\text{ and }}

\graphicspath{{images/}} % Paths in which to look for images

\makeindex[columns=3, title=Alphabetical Index, intoc] % Make LaTeX produce the files required to compile the index

\makeglossaries % Make LaTeX produce the files required to compile the glossary
% \input{glossary.tex} % Include the glossary definitions

% \makenomenclature % Make LaTeX produce the files required to compile the nomenclature

% Reset sidenote counter at chapters
%\counterwithin*{sidenote}{chapter}

%----------------------------------------------------------------------------------------

\begin{document}

%----------------------------------------------------------------------------------------
%  BOOK INFORMATION
%----------------------------------------------------------------------------------------

% \titlehead{The \texttt{kaobook} class}
% \subject{Use this document as a template}

% \title[Example and documentation of the {\normalfont\texttt{kaobook}} class]{Example and documentation \\ of the {\normalfont\texttt{kaobook}} class}
% \subtitle{Customise this page according to your needs}
\title{Introduction to Programming Languages}

% \author[Federico Marotta]{Federico Marotta\thanks{A \LaTeX\ lover}}
\author{Jaemin Hong and Sukyoung Ryu}

% \date{\today}
\date{}

% \publishers{An Awesome Publisher}

%----------------------------------------------------------------------------------------

\frontmatter % Denotes the start of the pre-document content, uses roman numerals

%----------------------------------------------------------------------------------------
%  OPENING PAGE
%----------------------------------------------------------------------------------------

%\makeatletter
%\extratitle{
%  % In the title page, the title is vspaced by 9.5\baselineskip
%  \vspace*{9\baselineskip}
%  \vspace*{\parskip}
%  \begin{center}
%    % In the title page, \huge is set after the komafont for title
%    \usekomafont{title}\huge\@title
%  \end{center}
%}
%\makeatother

%----------------------------------------------------------------------------------------
%  COPYRIGHT PAGE
%----------------------------------------------------------------------------------------

\makeatletter
\uppertitleback{\@titlehead} % Header

% \lowertitleback{
%   \textbf{Disclaimer}\\
%   You can edit this page to suit your needs. For instance, here we have a no copyright statement, a colophon and some other information. This page is based on the corresponding page of Ken Arroyo Ohori's thesis, with minimal changes.

%   \medskip

%   \textbf{No copyright}\\
%   \cczero\ This book is released into the public domain using the CC0 code. To the extent possible under law, I waive all copyright and related or neighbouring rights to this work.

%   To view a copy of the CC0 code, visit: \\\url{http://creativecommons.org/publicdomain/zero/1.0/}

%   \medskip

%   \textbf{Colophon} \\
%   This document was typeset with the help of \href{https://sourceforge.net/projects/koma-script/}{\KOMAScript} and \href{https://www.latex-project.org/}{\LaTeX} using the \href{https://github.com/fmarotta/kaobook/}{kaobook} class.

%   The source code of this book is available at:\\\url{https://github.com/fmarotta/kaobook}

%   (You are welcome to contribute!)

%   \medskip

%   \textbf{Publisher} \\
%   First printed in May 2019 by \@publishers
% }
\lowertitleback{
  \copyright2022 Jaemin Hong and Sukyoung Ryu

  \medskip

  All rights reserved. No part of this book may be reproduced in any form by any
  electronic of mechanical means (including photocopying, recording, or
  information storage and retrieval) without permission in writing from the
  authors.
}
\makeatother

%----------------------------------------------------------------------------------------
%  DEDICATION
%----------------------------------------------------------------------------------------

% \dedication{
%   The harmony of the world is made manifest in Form and Number, and the heart and soul and all the poetry of Natural Philosophy are embodied in the concept of mathematical beauty.\\
%   \flushright -- D'Arcy Wentworth Thompson
% }

%----------------------------------------------------------------------------------------
%  OUTPUT TITLE PAGE AND PREVIOUS
%----------------------------------------------------------------------------------------

% Note that \maketitle outputs the pages before here

\maketitle

%----------------------------------------------------------------------------------------
%  PREFACE
%----------------------------------------------------------------------------------------

\setchapterpreamble[u]{\margintoc}
\chapter{Acknowledgement}
\labch{acknowledgement}

The contents of this book are based on the KAIST \textit{Programming Languages}
course. We thank PLT\sidenote{\url{https://racket-lang.org/people.html}} since
the course referred to many materials from PLT in the beginning.
We also thank every student who took the
course before. We have learned many things from the interaction with the
students, and those lessons have affected various parts of the book. In
addition, we thank all the previous and current teaching assistants of the
course. They gave opinions to the course and wrote some of the exercises in the
book. Especially, Jihyeok Park highly contributed to the course, and Jihee Park
helped us edit the exercises.

We would be delighted to receive comments and corrections, which may be sent to
\code{jaemin.hong@kaist.ac.kr}. We thank in advance everyone who will contribute
to the book in the future.


%----------------------------------------------------------------------------------------
%  TABLE OF CONTENTS & LIST OF FIGURES/TABLES
%----------------------------------------------------------------------------------------

\begingroup % Local scope for the following commands

% Define the style for the TOC, LOF, and LOT
%\setstretch{1} % Uncomment to modify line spacing in the ToC
%\hypersetup{linkcolor=blue} % Uncomment to set the colour of links in the ToC
\setlength{\textheight}{230\hscale} % Manually adjust the height of the ToC pages

% Turn on compatibility mode for the etoc package
\etocstandarddisplaystyle % "toc display" as if etoc was not loaded
\etocstandardlines % "toc lines" as if etoc was not loaded

\tableofcontents % Output the table of contents

% \listoffigures % Output the list of figures

% Comment both of the following lines to have the LOF and the LOT on different pages
% \let\cleardoublepage\bigskip
% \let\clearpage\bigskip

% \listoftables % Output the list of tables

\endgroup

%----------------------------------------------------------------------------------------
%  MAIN BODY
%----------------------------------------------------------------------------------------

\mainmatter % Denotes the start of the main document content, resets page numbering and uses arabic numbers
\setchapterstyle{kao} % Choose the default chapter heading style

\setchapterpreamble[u]{\margintoc}
\chapter{Introduction}
\labch{introduction}

What is a programming language?

The simplest answer is ``it is a language used for programming.'' However, this
answer does not help us understand programming languages. We need a better
question to get a better answer.

What does a programming language consist of?

There is a good answer for this question: ``in a narrow sense, a programming
language consists of syntax and semantics, and in a broad sense, it additionally
has a standard library and an ecosystem.''

Syntax and semantics are principal concepts to understand programming languages.
Syntax determines how a language looks like, and semantics fills the inside. If
we consider a programming language as a human, we can say that syntax is one’s
appearance, and semantics is one’s thoughts. Programmers write programs
according to syntax. Syntax decides characters used in source code. Once programs
are written, semantics decides what each program does. Without semantics, all
the programs are useless. Programs can work as being expected only after
semantics determines the meaning of them. A programming language with syntax and
semantics is complete. Programmers using that language can write programs with
the syntax and execute the programs with the semantics. From a theoretical
perspective, syntax and semantics are all of a programming language.

For programmers, syntax and semantics are not the only elements of a programming
language. First, the standard library of a language is another element. The
standard library provides various utilities required by applications: data
structures like lists and maps, functions handling file and network IO, and so
on. The standard library is like clothes for humans. A human without clothes is
a human; a programming language without a standard library is a programming
language. At the same time, clothes are important to humans as they make bodies
warm and protect bodies from dangerous objects. Similarly, a standard library is
important to a programming language as it supplies diverse functionalities for
applications. Each person wears clothes different from others, and each language
puts different things from other languages in its standard library. Some
languages include lots of utilities in their standard libraries, while others
include much less. Some languages treat lists and maps as built-in concepts in
their semantics, while others define them with other primitives in their standard libraries.
Programmers avoid using a language without a standard library because such a
language increases the effort to write programs.

Another important element to programmers is the ecosystem of a programming
language. The ecosystem includes everything related to the language: developers
and companies using the language, third-party libraries written in the language,
and so on. It is like a society for humans. If many programmers and companies
use a programming language, one can easily get help and find complementary
materials by using the same language. There will be more chances of cooperative
work and employment, too. Third-party libraries also take important roles in
software development. The standard library offers only general facilities and
often lacks domain-specific features. When a required functionality cannot be
found in the standard library, a third-party library can provide the exact
functionality. For these reasons, the ecosystem of a programming language is
important to programmers.

Practically, the standard library and the ecosystem of a language are important
elements. Unlike syntax and semantics, they are not essential. A programming
language can exist even without its standard library and ecosystem. However,
developers take standard libraries and ecosystems into account as well as syntax and
semantics to choose languages they use. From a practical perspective, a
programming language consists of syntax, semantics, a standard library, and an
ecosystem.

This book is not for helping readers use a specific programming language. It
does not recommend a specific programming language, either. This book helps
readers learn new programming languages easily. You can acquaint any programming
languages once you completely read and understand this book. Obviously, this
goal cannot be achieved if the book discusses various languages separately. It
is possible only by discussing the underlying principles of every programming
language.

The principles of programming languages can be found from their semantics. Each
language seems very different from the others, but it is actually not the case.
Precisely speaking, their insides are quite the same, while their appearances
look different. They look different because their syntax and standard libraries,
which determine the appearances, are different. However, their insides, the
semantics, fundamentally share the same mathematical principles. If you
understand essential concepts residing in the semantics of multiple languages,
it is easy to understand and learn new languages.

People who know the key principles and can separate the elements of a language
can easily learn programming languages. As an analogy, consider a man learning
how to use a computer. It is a big problem if he cannot distinguish a keyboard
from a computer. For example, he thinks ``to say hello, my right index finger
presses the keyboard, my left middle finger presses the keyboard, my right ring
finger presses the keyboard three times.'' If the layout of the keyboard changes,
he should learn the whole computer again. On the other hand, if he knows that a
keyboard is just a tool to input text, he will less suffer from the change of
the keyboard layout. As he thinks ``to say hello, I press H, E, L, L, and O,'' he
does not need to learn the whole computer again. Of course, he should learn the
new keyboard layout, but it will be much easier. In addition, it is
straightforward to apply his knowledge to do new things. For example, he will
easily figure out ``to say lol, I press L, O, and L.'' If he does not distinguish
a keyboard from a computer, he cannot find any common principles between saying
hello and saying lol. Learning programming languages is the same. People who
cannot distinguish syntax and semantics believe that they should learn the whole
language again when the syntax changes. On the other hand, people who can
distinguish syntax and semantics know that semantics remains the same even if
syntax may vary. They know that understanding the principles of semantics is
important to learn languages. Becoming familiar with the
new syntax is all they need to use a new language fluently.

This book explains the semantics of principal concepts in programming languages.
\refch{introduction-to-scala}, \refch{immutability},
\refch{functions}, and \refch{pattern-matching}
introduce the Scala programming language. This book
uses Scala to implement interpreters of languages introduced in the book.
\refch{syntax-and-semantics} explains syntax and
semantics. Then, the book finally introduces various features of programming languages.
\begin{itemize}
    \item \refch{identifiers} introduces identifiers.
    \item \refch{first-order-functions},
      \refch{first-class-functions}, and \refch{lambda-calculus} introduce functions.
    \item \refch{recursion} introduces recursion.
    \item \refch{mutable-boxes} and \refch{mutable-variables} introduce mutation.
    \item \refch{lazy-evaluation} introduces lazy evaluation.
\end{itemize}

\section{Exercises}

\begin{enumerate}
\item Write the name of a programming language that you have used.
  What are the pros and cons of the language?
\item Write the names of two programming languages you know and compare them.
\end{enumerate}


% \pagelayout{wide} % No margins
% \addpart{Class Options, Commands and Environments}
% \pagelayout{margin} % Restore margins

% \input{chapters/options.tex}
% \input{chapters/textnotes.tex}
% \input{chapters/figsntabs.tex}
% \input{chapters/references.tex}

% \pagelayout{wide} % No margins
% \addpart{Design and Additional Features}
% \pagelayout{margin} % Restore margins

% \input{chapters/layout.tex}
% \input{chapters/mathematics.tex}

% \appendix % From here onwards, chapters are numbered with letters, as is the appendix convention

% \pagelayout{wide} % No margins
% \addpart{Appendix}
% \pagelayout{margin} % Restore margins

% \input{chapters/appendix.tex}

%----------------------------------------------------------------------------------------

\backmatter % Denotes the end of the main document content
\setchapterstyle{plain} % Output plain chapters from this point onwards

%----------------------------------------------------------------------------------------
%  BIBLIOGRAPHY
%----------------------------------------------------------------------------------------

% The bibliography needs to be compiled with biber using your LaTeX editor, or on the command line with 'biber main' from the template directory

\defbibnote{bibnote}{Here are the references in citation order.\par\bigskip} % Prepend this text to the bibliography
\printbibliography[heading=bibintoc, title=Bibliography, prenote=bibnote] % Add the bibliography heading to the ToC, set the title of the bibliography and output the bibliography note

%----------------------------------------------------------------------------------------
%  NOMENCLATURE
%----------------------------------------------------------------------------------------

% The nomenclature needs to be compiled on the command line with 'makeindex main.nlo -s nomencl.ist -o main.nls' from the template directory

% \nomenclature{$c$}{Speed of light in a vacuum inertial frame}
% \nomenclature{$h$}{Planck constant}

% \renewcommand{\nomname}{Notation} % Rename the default 'Nomenclature'
% \renewcommand{\nompreamble}{The next list describes several symbols that will be later used within the body of the document.} % Prepend this text to the nomenclature

% \printnomenclature % Output the nomenclature

%----------------------------------------------------------------------------------------
%  GREEK ALPHABET
%   Originally from https://gitlab.com/jim.hefferon/linear-algebra
%----------------------------------------------------------------------------------------

% \vspace{1cm}

% {\usekomafont{chapter}Greek Letters with Pronunciations} \\[2ex]
% \begin{center}
%   \newcommand{\pronounced}[1]{\hspace*{.2em}\small\textit{#1}}
%   \begin{tabular}{l l @{\hspace*{3em}} l l}
%     \toprule
%     Character & Name & Character & Name \\
%     \midrule
%     $\alpha$ & alpha \pronounced{AL-fuh} & $\nu$ & nu \pronounced{NEW} \\
%     $\beta$ & beta \pronounced{BAY-tuh} & $\xi$, $\Xi$ & xi \pronounced{KSIGH} \\
%     $\gamma$, $\Gamma$ & gamma \pronounced{GAM-muh} & o & omicron \pronounced{OM-uh-CRON} \\
%     $\delta$, $\Delta$ & delta \pronounced{DEL-tuh} & $\pi$, $\Pi$ & pi \pronounced{PIE} \\
%     $\epsilon$ & epsilon \pronounced{EP-suh-lon} & $\rho$ & rho \pronounced{ROW} \\
%     $\zeta$ & zeta \pronounced{ZAY-tuh} & $\sigma$, $\Sigma$ & sigma \pronounced{SIG-muh} \\
%     $\eta$ & eta \pronounced{AY-tuh} & $\tau$ & tau \pronounced{TOW (as in cow)} \\
%     $\theta$, $\Theta$ & theta \pronounced{THAY-tuh} & $\upsilon$, $\Upsilon$ & upsilon \pronounced{OOP-suh-LON} \\
%     $\iota$ & iota \pronounced{eye-OH-tuh} & $\phi$, $\Phi$ & phi \pronounced{FEE, or FI (as in hi)} \\
%     $\kappa$ & kappa \pronounced{KAP-uh} & $\chi$ & chi \pronounced{KI (as in hi)} \\
%     $\lambda$, $\Lambda$ & lambda \pronounced{LAM-duh} & $\psi$, $\Psi$ & psi \pronounced{SIGH, or PSIGH} \\
%     $\mu$ & mu \pronounced{MEW} & $\omega$, $\Omega$ & omega \pronounced{oh-MAY-guh} \\
%     \bottomrule
%   \end{tabular} \\[1.5ex]
%   Capitals shown are the ones that differ from Roman capitals.
% \end{center}

%----------------------------------------------------------------------------------------
%  GLOSSARY
%----------------------------------------------------------------------------------------

% The glossary needs to be compiled on the command line with 'makeglossaries main' from the template directory

\setglossarystyle{listgroup} % Set the style of the glossary (see https://en.wikibooks.org/wiki/LaTeX/Glossary for a reference)
\printglossary[title=Special Terms, toctitle=List of Terms] % Output the glossary, 'title' is the chapter heading for the glossary, toctitle is the table of contents heading

%----------------------------------------------------------------------------------------
%  INDEX
%----------------------------------------------------------------------------------------

% The index needs to be compiled on the command line with 'makeindex main' from the template directory

\printindex % Output the index

%----------------------------------------------------------------------------------------
%  BACK COVER
%----------------------------------------------------------------------------------------

% If you have a PDF/image file that you want to use as a back cover, uncomment the following lines

%\clearpage
%\thispagestyle{empty}
%\null%
%\clearpage
%\includepdf{cover-back.pdf}

%----------------------------------------------------------------------------------------

\end{document}
