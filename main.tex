%%%%%%%%%%%%%%%%%%%%%%%%%%%%%%%%%%%%%%%%%
% kaobook
% LaTeX Template
% Version 1.2 (4/1/2020)
%
% This template originates from:
% https://www.LaTeXTemplates.com
%
% For the latest template development version and to make contributions:
% https://github.com/fmarotta/kaobook
%
% Authors:
% Federico Marotta (federicomarotta@mail.com)
% Based on the doctoral thesis of Ken Arroyo Ohori (https://3d.bk.tudelft.nl/ken/en)
% and on the Tufte-LaTeX class.
% Modified for LaTeX Templates by Vel (vel@latextemplates.com)
%
% License:
% CC0 1.0 Universal (see included MANIFEST.md file)
%
%%%%%%%%%%%%%%%%%%%%%%%%%%%%%%%%%%%%%%%%%

%----------------------------------------------------------------------------------------
%	PACKAGES AND OTHER DOCUMENT CONFIGURATIONS
%----------------------------------------------------------------------------------------

\documentclass[
	fontsize=10pt, % Base font size
	twoside=false, % Use different layouts for even and odd pages (in particular, if twoside=true, the margin column will be always on the outside)
	%open=any, % If twoside=true, uncomment this to force new chapters to start on any page, not only on right (odd) pages
	%chapterprefix=true, % Uncomment to use the word "Chapter" before chapter numbers everywhere they appear
	%chapterentrydots=true, % Uncomment to output dots from the chapter name to the page number in the table of contents
	numbers=noenddot, % Comment to output dots after chapter numbers; the most common values for this option are: enddot, noenddot and auto (see the KOMAScript documentation for an in-depth explanation)
	%draft=true, % If uncommented, rulers will be added in the header and footer
	%overfullrule=true, % If uncommented, overly long lines will be marked by a black box; useful for correcting spacing problems
]{kaobook}

% Set the language
\usepackage[english]{babel} % Load characters and hyphenation
\usepackage[english=british]{csquotes} % English quotes

% Load packages for testing
\usepackage{blindtext}
%\usepackage{showframe} % Uncomment to show boxes around the text area, margin, header and footer
%\usepackage{showlabels} % Uncomment to output the content of \label commands to the document where they are used

% Load the bibliography package
\usepackage{styles/kaobiblio}
\addbibresource{main.bib} % Bibliography file

% Load mathematical packages for theorems and related environments. NOTE: choose only one between 'mdftheorems' and 'plaintheorems'.
\usepackage{styles/mdftheorems}
%\usepackage{styles/plaintheorems}

\usepackage{mathpartir}
\newcommand{\term}[1]{\textit{#1}\index{\MakeLowercase{#1}}}

\graphicspath{{examples/documentation/images/}{images/}} % Paths in which to look for images

\makeindex[columns=3, title=Alphabetical Index, intoc] % Make LaTeX produce the files required to compile the index

\makeglossaries % Make LaTeX produce the files required to compile the glossary

\makenomenclature % Make LaTeX produce the files required to compile the nomenclature

% Reset sidenote counter at chapters
%\counterwithin*{sidenote}{chapter}

%----------------------------------------------------------------------------------------

\begin{document}

%----------------------------------------------------------------------------------------
%	BOOK INFORMATION
%----------------------------------------------------------------------------------------

% \titlehead{The \texttt{kaobook} class}
% \subject{Use this document as a template}

% \title[Example and documentation of the {\normalfont\texttt{kaobook}} class]{Example and documentation \\ of the {\normalfont\texttt{kaobook}} class}
\title{Introduction to Programming Languages}
% \subtitle{Customise this page according to your needs}

% \author[Federico Marotta]{Federico Marotta \thanks{A \LaTeX\ lover}}
\author{Jaemin Hong}

\date{}

% \publishers{An Awesome Publisher}

%----------------------------------------------------------------------------------------

\frontmatter % Denotes the start of the pre-document content, uses roman numerals

%----------------------------------------------------------------------------------------
%	OPENING PAGE
%----------------------------------------------------------------------------------------

%\makeatletter
%\extratitle{
%	% In the title page, the title is vspaced by 9.5\baselineskip
%	\vspace*{9\baselineskip}
%	\vspace*{\parskip}
%	\begin{center}
%		% In the title page, \huge is set after the komafont for title
%		\usekomafont{title}\huge\@title
%	\end{center}
%}
%\makeatother

%----------------------------------------------------------------------------------------
%	COPYRIGHT PAGE
%----------------------------------------------------------------------------------------

\makeatletter
\uppertitleback{\@titlehead} % Header

\lowertitleback{
	\textbf{Disclaimer}\\
	You can edit this page to suit your needs. For instance, here we have a no copyright statement, a colophon and some other information. This page is based on the corresponding page of Ken Arroyo Ohori's thesis, with minimal changes.

	\medskip

	\textbf{No copyright}\\
	\cczero\ This book is released into the public domain using the CC0 code. To the extent possible under law, I waive all copyright and related or neighbouring rights to this work.

	To view a copy of the CC0 code, visit: \\\url{http://creativecommons.org/publicdomain/zero/1.0/}

	\medskip

	\textbf{Colophon} \\
	This document was typeset with the help of \href{https://sourceforge.net/projects/koma-script/}{\KOMAScript} and \href{https://www.latex-project.org/}{\LaTeX} using the \href{https://github.com/fmarotta/kaobook/}{kaobook} class.

	The source code of this book is available at:\\\url{https://github.com/fmarotta/kaobook}

	(You are welcome to contribute!)

	\medskip

	\textbf{Publisher} \\
	First printed in May 2019 by \@publishers
}
\makeatother

%----------------------------------------------------------------------------------------
%	DEDICATION
%----------------------------------------------------------------------------------------

% \dedication{
% 	The harmony of the world is made manifest in Form and Number, and the heart and soul and all the poetry of Natural Philosophy are embodied in the concept of mathematical beauty.\\
% 	\flushright -- D'Arcy Wentworth Thompson
% }

%----------------------------------------------------------------------------------------
%	OUTPUT TITLE PAGE AND PREVIOUS
%----------------------------------------------------------------------------------------

% Note that \maketitle outputs the pages before here

% If twoside=false, \uppertitleback and \lowertitleback are not printed
% To overcome this issue, we set twoside=semi just before printing the title pages, and set it back to false just after the title pages
\KOMAoptions{twoside=semi}
\maketitle
\KOMAoptions{twoside=false}

%----------------------------------------------------------------------------------------
%	PREFACE
%----------------------------------------------------------------------------------------

%\input{chapters/preface.tex}

%----------------------------------------------------------------------------------------
%	TABLE OF CONTENTS & LIST OF FIGURES/TABLES
%----------------------------------------------------------------------------------------

\begingroup % Local scope for the following commands

% Define the style for the TOC, LOF, and LOT
%\setstretch{1} % Uncomment to modify line spacing in the ToC
%\hypersetup{linkcolor=blue} % Uncomment to set the colour of links in the ToC
\setlength{\textheight}{23cm} % Manually adjust the height of the ToC pages

% Turn on compatibility mode for the etoc package
\etocstandarddisplaystyle % "toc display" as if etoc was not loaded
\etocstandardlines % toc lines as if etoc was not loaded

\tableofcontents % Output the table of contents

\listoffigures % Output the list of figures

% Comment both of the following lines to have the LOF and the LOT on different pages
\let\cleardoublepage\bigskip
\let\clearpage\bigskip

\listoftables % Output the list of tables

\endgroup

%----------------------------------------------------------------------------------------
%	MAIN BODY
%----------------------------------------------------------------------------------------

\mainmatter % Denotes the start of the main document content, resets page numbering and uses arabic numbers
\setchapterstyle{kao} % Choose the default chapter heading style

\setchapterpreamble[u]{\margintoc}
\chapter{Introduction}
\labch{introduction}

What is a programming language?

The simplest answer is ``it is a language used for programming.'' However, this
answer does not help us understand programming languages. We need a better
question to get a better answer.

What does a programming language consist of?

There is a good answer for this question: ``in a narrow sense, a programming
language consists of syntax and semantics, and in a broad sense, it additionally
has a standard library and an ecosystem.''

Syntax and semantics are principal concepts to understand programming languages.
Syntax determines how a language looks like, and semantics fills the inside. If
we consider a programming language as a human, we can say that syntax is one’s
appearance, and semantics is one’s thoughts. Programmers write programs
according to syntax. Syntax decides characters used in source code. Once programs
are written, semantics decides what each program does. Without semantics, all
the programs are useless. Programs can work as being expected only after
semantics determines the meaning of them. A programming language with syntax and
semantics is complete. Programmers using that language can write programs with
the syntax and execute the programs with the semantics. From a theoretical
perspective, syntax and semantics are all of a programming language.

For programmers, syntax and semantics are not the only elements of a programming
language. First, the standard library of a language is another element. The
standard library provides various utilities required by applications: data
structures like lists and maps, functions handling file and network IO, and so
on. The standard library is like clothes for humans. A human without clothes is
a human; a programming language without a standard library is a programming
language. At the same time, clothes are important to humans as they make bodies
warm and protect bodies from dangerous objects. Similarly, a standard library is
important to a programming language as it supplies diverse functionalities for
applications. Each person wears clothes different from others, and each language
puts different things from other languages in its standard library. Some
languages include lots of utilities in their standard libraries, while others
include much less. Some languages treat lists and maps as built-in concepts in
their semantics, while others define them with other primitives in their standard libraries.
Programmers avoid using a language without a standard library because such a
language increases the effort to write programs.

Another important element to programmers is the ecosystem of a programming
language. The ecosystem includes everything related to the language: developers
and companies using the language, third-party libraries written in the language,
and so on. It is like a society for humans. If many programmers and companies
use a programming language, one can easily get help and find complementary
materials by using the same language. There will be more chances of cooperative
work and employment, too. Third-party libraries also take important roles in
software development. The standard library offers only general facilities and
often lacks domain-specific features. When a required functionality cannot be
found in the standard library, a third-party library can provide the exact
functionality. For these reasons, the ecosystem of a programming language is
important to programmers.

Practically, the standard library and the ecosystem of a language are important
elements. Unlike syntax and semantics, they are not essential. A programming
language can exist even without its standard library and ecosystem. However,
developers take standard libraries and ecosystems into account as well as syntax and
semantics to choose languages they use. From a practical perspective, a
programming language consists of syntax, semantics, a standard library, and an
ecosystem.

This book is not for helping readers use a specific programming language. It
does not recommend a specific programming language, either. This book helps
readers learn new programming languages easily. You can acquaint any programming
languages once you completely read and understand this book. Obviously, this
goal cannot be achieved if the book discusses various languages separately. It
is possible only by discussing the underlying principles of every programming
language.

The principles of programming languages can be found from their semantics. Each
language seems very different from the others, but it is actually not the case.
Precisely speaking, their insides are quite the same, while their appearances
look different. They look different because their syntax and standard libraries,
which determine the appearances, are different. However, their insides, the
semantics, fundamentally share the same mathematical principles. If you
understand essential concepts residing in the semantics of multiple languages,
it is easy to understand and learn new languages.

People who know the key principles and can separate the elements of a language
can easily learn programming languages. As an analogy, consider a man learning
how to use a computer. It is a big problem if he cannot distinguish a keyboard
from a computer. For example, he thinks ``to say hello, my right index finger
presses the keyboard, my left middle finger presses the keyboard, my right ring
finger presses the keyboard three times.'' If the layout of the keyboard changes,
he should learn the whole computer again. On the other hand, if he knows that a
keyboard is just a tool to input text, he will less suffer from the change of
the keyboard layout. As he thinks ``to say hello, I press H, E, L, L, and O,'' he
does not need to learn the whole computer again. Of course, he should learn the
new keyboard layout, but it will be much easier. In addition, it is
straightforward to apply his knowledge to do new things. For example, he will
easily figure out ``to say lol, I press L, O, and L.'' If he does not distinguish
a keyboard from a computer, he cannot find any common principles between saying
hello and saying lol. Learning programming languages is the same. People who
cannot distinguish syntax and semantics believe that they should learn the whole
language again when the syntax changes. On the other hand, people who can
distinguish syntax and semantics know that semantics remains the same even if
syntax may vary. They know that understanding the principles of semantics is
important to learn languages. Becoming familiar with the
new syntax is all they need to use a new language fluently.

This book explains the semantics of principal concepts in programming languages.
\refch{introduction-to-scala}, \refch{immutability},
\refch{functions}, and \refch{pattern-matching}
introduce the Scala programming language. This book
uses Scala to implement interpreters of languages introduced in the book.
\refch{syntax-and-semantics} explains syntax and
semantics. Then, the book finally introduces various features of programming languages.
\begin{itemize}
    \item \refch{identifiers} introduces identifiers.
    \item \refch{first-order-functions},
      \refch{first-class-functions}, and \refch{lambda-calculus} introduce functions.
    \item \refch{recursion} introduces recursion.
    \item \refch{mutable-boxes} and \refch{mutable-variables} introduce mutation.
    \item \refch{lazy-evaluation} introduces lazy evaluation.
\end{itemize}

\section{Exercises}

\begin{enumerate}
\item Write the name of a programming language that you have used.
  What are the pros and cons of the language?
\item Write the names of two programming languages you know and compare them.
\end{enumerate}


\pagelayout{wide} % No margins
\addpart{Scala}
\pagelayout{margin} % Restore margins

\setchapterpreamble[u]{\margintoc}
\chapter{Introduction to Scala}
\labch{intro-to-scala}

\section{Scala}

Professor Odersky and his students in LAMP at EPFL have developed Scala. Scala
stands for ``\textbf{Sca}lable \textbf{La}nguage.'' The
``Growing a Language''\sidenote{\url{https://www.youtube.com/watch?v=_ahvzDzKdB0}}
talk given by Guy Steele has influenced Scala.
Scala supports both \term{functional} and \term{object-oriented} styles and aims a
growable language. Later articles will discuss functional programming more.

\begin{kaobox}[frametitle=Compilers and interpreters]
\term{Compilers} and \term{interpreters} are out of the topic but I believe that it is
the best point to discuss them.

Compilers are programs that translate one programming language into another
language. Usually, the term ``compilers'' is used for a narrower range of
programs: compilers get code written in high-level languages, whom people can
easily understand, as input and generate code written in low-level languages
like \term{machine languages} and \term{bytecode}. Physical machines can directly
interpret machine languages and \term{virtual machines} can directly interpret
bytecode. For example, GCC translates C and C++ into machine languages and
\verb+javac+ translates Java into Java bytecode. Since machines can directly
interpret the low-level languages, compiled forms of programs guarantee fast
execution. However, when programmers modify their code, they must re-compile
the code to execute programs so that checking results immediately after
changing the code takes a long time.

On the other hand, interpreters directly interpret code given as input and
show results instead of generating new code. Python and JavaScript are typical
languages using interpreters. Since given code is a \term{string}, interpreters do
complex procedures including \term{parsing} at run time and therefore execution
takes a long time. However, because changing code does not require compiling,
checking results after changing the code takes a short time.
\end{kaobox}

Scala compilers compile Scala code into Java bytecode executable by Java
virtual machines (JVM). Every Scala project can use the Java standard library
and any external Java libraries. Not only using compiled Java code but also
mixing Java code and Scala code in a single project is possible. In addition,
the Scala standard library supports
conversion\sidenote{\url{https://www.scala-lang.org/api/current/scala/jdk/CollectionConverters\$.html}}
between Java \term{collections} and
Scala collections. In practice, interoperability with Java is useful a lot.
Lack of Scala libraries is not problematic since libraries from the Java
ecosystem are available.

The most recent Scala version in July 2019 is 2.13.0. At the same time, Scala
, as known as Dotty, is coming. The most recent Dotty version in July 2019 is
.16.0. The official GitHub repositories are
\url{https://github.com/scala/scala}
and \url{https://github.com/lampepfl/dotty}, respectively.

\section{Scala in Industry}

Unfortunately, Scala is not one of the most popular languages. According to
GitHub Octoverse 2017\sidenote{\url{https://octoverse.github.com/2017}},
Scala was ranked the thirteenth place. The criterion was
the number of PRs in one year.
GitHub Octoverse 2018\sidenote{\url{https://octoverse.github.com/2018}}
and after showed until the tenth place and Scala did not appear on the list.
According to Stackoverflow Developer Survey 2019\sidenote{\url{https://insights.stackoverflow.com/survey/2019}},
Scala was ranked the twentieth place.

However, we can find several places using Scala. PLRG at KAIST has developed
JavaScript static analyzer
SAFE 2.0\sidenote{\url{https://github.com/sukyoung/safe}}
in Scala. Trivially, the official Scala and Dotty compilers have been
developed in Scala. In industry, Scala is popular for concurrent programming
and distributed computing.
Akka\sidenote{\url{https://akka.io/}} is a concurrent,
distributed computing library written in Scala. Many companies have been using Akka.
Apache Spark\sidenote{\url{https://spark.apache.org/}}, a well-known library for data
processing, also is written in Scala.
Play\sidenote{\url{https://www.playframework.com/}}
is a widely-used web framework based on Akka.

In another point of view, Scala is valuable since it is a good introduction to
functional programming. Nowadays, functional programming is popular in
industry as well as academia. Using one of functional languages like OCaml and
Haskell in future is highly possible for undergraduate students studying
computer science of these days. Therefore, if you started programming with
imperative* languages like C, Java, and Python, experiencing Scala will
increase the choices of your future job.

\section{Programming in Scala}

The official Scala web site\sidenote{\url{https://www.scala-lang.org/}} provides
instructions for installation. The site discusses also possible IDEs for
Scala.

\subsection{Scala REPL}

After installing Scala, type \verb+scala+ in a command line to execute the Scala
REPL. The term ``REPL'' stands for ``\textbf{r}ead, \textbf{e}val, \textbf{p}rint, and
\textbf{l}oop''.
It is a program that reads a line from a user, evaluates the code, prints
the result, and goes back to the first. Languages with interpreters used to
support REPLs but, nowadays, languages with compilers also supports REPLs. In
the Scala REPL, for every line of input, the Scala compiler compiles the code
and the REPL shows the result of the execution.

\begin{verbatim}
scala> 0
res0: Int = 0
\end{verbatim}

For input \term{expression} \verb+0+, the resulting \term{value} is \verb+0+ and the type of \verb+0+
is \verb+Int+.

\begin{verbatim}
scala> 1 * 1
res1: Int = 1
\end{verbatim}

It works well for more complex expressions.

\subsection{Variable Declarations}

\begin{verbatim}
scala> val x = 2
x: Int = 2
\end{verbatim}

To declare a variable, \verb+val+ can be used. \verb+val [name] = [expression]+ implies
that the result of evaluating the expression becomes the thing referred by the
name of the variable. The type of \verb+x+ is inferred as \verb+Int+.

\begin{verbatim}
scala> x = 3
<console>:12: error: reassignment to val
       x = 3
\end{verbatim}

Since \verb+val+ declares \term{immutable} variables, assigning an expression to an
already defined variable is impossible. Immutability is the most important
property of functional programming.

\begin{verbatim}
scala> val x = 4
x: Int = 4
\end{verbatim}

On the other hand, we can declare \verb+x+ again because, in the REPL, two \verb+x+'s
belong to different scopes. It is impossible to define two local variables of
the same name in a single scope.

\begin{verbatim}
scala> val y: Int = x + 1
y: Int = 5
\end{verbatim}

Mentioning explicitly the type of a variable is possible. Usually, due to
type inference*, type annotations are unnecessary but they are required in rare cases.

\begin{verbatim}
scala> var z = 6
z: Int = 6

scala> z = 7
z: Int = 7
\end{verbatim}

\verb+var+ allows to declare \term{mutable} variables.

\begin{verbatim}
scala> z = "8"
<console>:12: error: type mismatch;
 found   : String("8")
 required: Int
       z = "8"
           ^
\end{verbatim}

To assign an expression to a mutable variable, the expression must follow the
type of the variable. Functional programming avoids using mutable variables.
Actually, and surprisingly, prohibiting mutable variables does not decrease
expressivity of programs at all. Therefore, \verb+val+ should be used in most cases
but, for efficiency, \verb+var+ might be used. In the course, if there is no
special instruction, students cannot use \verb+var+ in their homework.

\subsection{Function Declarations}

\begin{verbatim}
scala> def add(x: Int, y: Int) = x + y
add: (x: Int, y: Int)Int
\end{verbatim}

\verb+def+ declares functions. \verb+def [name]([name]: [type], …) = [expression]+ is
the form. Like functions in mathematics, \verb+return+ is unnecessary. The return
value of a function call is the result of evaluating the expression at the
right-hand side. Since \verb+x + y+ has type \verb+Int+, the return type of \verb+add+ is
also \verb+Int+ due to type inference.

\begin{verbatim}
scala> add(4, 5)
res2: Int = 9
\end{verbatim}

The return value is correct.

\begin{verbatim}
scala> def add(x: Int, y: Int): Int = x + y
add: (x: Int, y: Int)Int
\end{verbatim}

Like variables, it is possible to annotate return types of functions.

\begin{verbatim}
scala> def add(x, y): Int = x + y
<console>:1: error: ':' expected but ',' found.
       def add(x, y): Int = x + y
                ^
\end{verbatim}

Unlike some languages including OCaml, types of function \term{parameters} are essential.

\begin{verbatim}
scala> def f(x: Int) = {
     |   val y = x + 1
     |   val z = x + 2
     |   y * y + z * z
     | }
f: (x: Int)Int
\end{verbatim}

If a function needs to use local variables, the body of the function is a
sequence of multiple expressions bundled in a pair of curly brackets. Every
expression inside the brackets is sequentially evaluated from top to bottom.
The result is equal to the result of the last expression.

\begin{verbatim}
scala> def g(x: Int) = {
     |   println(x)
     |   x * x
     |   x
     | }
g: (x: Int)Int

scala> g(10)
10
res3: Int = 10
\end{verbatim}

\verb+g+ shows the evaluation of bundled multiple expressions more clearly. \verb+g+
prints \verb+10+, calculates \verb+x * x+, whose result is discarded, and returns \verb+10+,
the result of evaluating \verb+x+.

\begin{verbatim}
scala> val w = {
     |   val a = 5
     |   a + a + 1
     | }
w: Int = 11

scala> add({
     |   val x = 6
     |   x + x
     | }, 0)
res4: Int = 12
\end{verbatim}

Since a sequence of bundled expressions is an expression, it can be used
anywhere expecting expressions.

\subsection{Conditionals}

\begin{verbatim}
scala> if (true) 13 else 14
res5: Int = 13
\end{verbatim}

\term{Conditional} expressions use \verb+if+ and \verb+else+ like many other languages.
Evaluating a conditional expression results in a value.

\begin{verbatim}
scala> val a = if (false) 13 else 14
a: Int = 14

scala> var b = 0
b: Int = 0

scala> if (false) b = 14 else b = 15

scala> b
res6: Int = 15
\end{verbatim}

In Scala, immutable variables are enough to assign conditional values. On the
other hand, imperative languages require mutable variables and reassignment.
\verb+if+ and \verb+else+ in Scala look similar to the ternary operator \verb+? :+ in
  imperative languages but are more general.

\begin{verbatim}
scala> if (true) {
     |   val x = 8
     |   x + x
     | } else {
     |   val x = 9
     |   x * x
     | }
res7: Int = 16
\end{verbatim}

Since a sequence of bundled expressions is an expression, conditional
expressions can use curly brackets to express complex computation while it is
impossible in imperative languages.

\subsection{Compiling Scala Code}

The Scala compiler \verb+scalac+ compiles Scala code into Java bytecode.

\begin{verbatim}
object App {
  def main(args: Array[String]) = println("Hello world!")
}
\end{verbatim}

Save the above code into \verb+App.scala+ file and type \verb+scala App.scala+ and
\verb+scala App+ in a command line. The console will print \verb+"Hello world!"+. \verb+scalac+
compiles \verb+App.scala+ and generates \verb+App.class+. \verb+scala+ uses JVM to execute
\verb+App.class+.

Like Java, the \verb+main+ \term{method} is the entry point of every Scala program. The
main` method must contain code has to be executed.

In practice, like other languages, programmers use build tools for their Scala
projects. Like CMake for C and C++ and Ant, Maven, and Gradle for Java, Scala
projects usually use SBT.

\setchapterpreamble[u]{\margintoc}
\chapter{Introduction to Functional Programming}
\labch{intro-to-fp}

\section{The Definition of Functional Programming}

What is functional programming? English Wikipedia says the following:

> Functional programming is a programming paradigm that treats computation as the
evaluation of mathematical functions and avoids changing-state and mutable data.

According to the book "Functional Programming in Scala":

> Functional programming (FP) is based on a simple premise with far-reaching
implications: we construct our programs using only pure functions—in other words,
functions that have no side effects.

It is the first sentence of the first chapter.

The above two sentences are enough to describe functional programming. Firstly,
consider the phrase 'the evaluation of mathematical functions.' In the last
article, I mentioned that everything is an expression in Scala and an expression
is evaluated into a value. In the perspective of functional programming, a
program is a single (mathematical) expression and the execution of the program is
finding a value denoted by the expression. The following code shows how
functional programming is different from imperative programming.

\begin{verbatim}
int x = 1;
int y = 2;
if (y < 3)
    x = x + 4;
else
    x = x - 5;
\end{verbatim}

\begin{verbatim}
let x = 1 in
let y = 2 in
if y < 3 then x + 4 else x - 5
\end{verbatim}

The first code is written in C, which represents imperative languages. Imperative
programming mimics a way that computers operate in. During the execution of a
program, a \term{state}, which can be interpreted as a memory of a computer,
exists and the execution modifies the state. The execution of the C program has
the following steps:

\begin{enumerate}
\item A state that both \verb+x+ and \verb+y+ are uninitialized
\item A state that \verb+x+ is \verb+1+ and \verb+y+ is uninitialized
\item A state that \verb+x+ is \verb+1+ and \verb+y+ is 2
\item Since \verb+y < 3+ is \verb+true+ under the state of the third step, go to
the next line.
\item A state that \verb+x+ is \verb+5+ and \verb+y+ is 2
\end{enumerate}

The second code is written in OCaml, which represents functional languages. A
program is an expression and the result of the execution is the result of
evaluating the expression. The execution does not require a notion of a state.
The execution of the OCaml program has the following steps:

\begin{enumerate}
\item Given a fact that \verb+x+ equals \verb+1+, evaluate
\verb!let y = 2 in if y < 3 then x + 4 else x – 5!.
\item Given a fact that \verb!x! equals \verb!1! and \verb!y! equals \verb!2!,
evaluate \verb!if y < 3 then x + 4 else x – 5!.
\item Given a fact that \verb!x! equals \verb!1! and \verb!y! equals \verb!2!,
evaluating \verb!y < 3! yields \verb!true! and thus evaluate \verb!x + 4!.
\item Given a fact that \verb!x! equals \verb!1! and \verb!y! equals \verb!2!,
evaluate \verb!x + 4!.
\item The result is \verb!5!.
\end{enumerate}

Since the programs are simple, two programs look similar but it is important to
understand two different perspectives of what a program is.

Secondly, look at the phrases 'avoids changing-state and mutable data' and 'using
only pure functions.' The last article said that functional programming avoids
mutable variables. Functional programming does not modify data including
variables and objects. States, which change throughout the execution of programs,
do not exist. Due to the lack of states, a function always does the same stuff
and for given the same argument, always returns the same value. Such functions
are called \term{pure functions}.

In practice, especially for large-scale projects, using only immutable things in
the whole code is usually inefficient. Most real-world functional languages
provide mutable variables or structures like \verb!var! of Scala, \verb!ref! of
OCaml, and \verb!set!! and \verb!box! of Racket. However, during the course, you
will see that functional programming uses immutable things in most cases but can
still express a lot of programs without difficulties. I discuss the advantages of
immutability in a later section of the article.

\section{Functional Programming in Industry}

Before going deeply into functional programming, let us find how people use
functional programming in industry.

\subsection{OCaml}

Facebook has developed Infer\sidenote{\url{https://fbinfer.com/}}, a static analyzer for Java,
C, C++, and Objective-C, in OCaml. Facebook and other companies including Amazon
and Mozila use Infer to find bugs statically in their programs. Facebook has
developed also Flow\sidenote{\url{https://flow.org/}}, a static type checker for JavaScript.

Jane Street\sidenote{\url{https://www.janestreet.com/}} is a financial company well-known in
the PL community and has developed its own software in OCaml.

According to the official OCaml web
site\sidenote{\url{http://ocaml.org/learn/companies.html}}, various companies including Docker
use OCaml.

\subsection{Haskell}

Haskell Wiki\sidenote{\url{http://wiki.haskell.org/Haskell_in_industry}} mentions that Google,
Facebook, Microsoft, Nvidia, and many others use Haskell.

\subsection{Erlang and Elixir}

Erlang is a functional language for concurrent and parallel computing. Elixir
operates on Erlang virtual machines and is used for the same purpose as Erlang.
An article from Code
Sync\sidenote{\url{https://codesync.global/media/successful-companies-using-elixir-and-erlang/}}
introduced companies including WhatsApp, Pinterest, and Goldman Sachs using
Erlang and Elixir.

Programs whose input has complicated and abstract structures like source code and
programs that have to be trustworthy or targets concurrent and parallel computing
are typically written in functional languages.

\section{The Advantages of Immutability}

The advantages of immutability per se become the advantages of functional
programming. The book "Programming in Scala" discusses four strength of
immutability:

> First, immutable objects are often easier to reason about than mutable ones,
because they do not have complex state spaces that change over time. Second, you
can pass immutable objects around quite freely, whereas you may need to make
defensive copies of mutable objects before passing them to other code. Third,
there is no way for two threads concurrently accessing an immutable to corrupt
its state once it has been properly constructed, because no thread can change the
state of an immutable. Fourth, immutable objects make safe hash table keys. If a
mutable object is mutated after it is placed into a \verb!HashSet!, for example,
that object may not be found the next time you look into the \verb!HashSet!.

\subsection{Is Easy to Reason about}

\begin{verbatim}
val x = 1
...
f(x)
\end{verbatim}

At the first line of the code, \verb!x! is \verb!1!. Since \verb!x! is immutable,
there is no doubt that \verb!x! is still \verb!1! when \verb!x! is passed as an
argument for \verb!f! at the last line of the code.

\begin{verbatim}
var x = 1
...
f(x)
\end{verbatim}

On the other hand, if \verb!x! is a mutable variable, one should read every line
of code in the middle to find the value of \verb!x! at the time when the function
call happens.

For small programs written by one person, whether \verb!x! is immutable or
mutable is unimportant. If the program does not expect any future usages, mutable
\verb!x! is fine. However, suppose the situation reading the code without any
prior knowledge about the code. A big difference between immutable and mutable
\verb!x! exists even though there is no person who reads only the first and the
last line of course. When \verb!x! is mutable, without tracking every
modification of \verb!x! throughout the code, the value of \verb!x! at the last
line is unknown. It makes programmers feel difficulties to understand the code so
possibly leads to more bugs. The program with immutable \verb!x! does not suffer
from such a problem. Remembering only one line of the code is enough to track the
value of \verb!x!.

Mutable data structures cause similar problems.

\begin{verbatim}
val x = List(1, 2)
...
f(x)
...
x
\end{verbatim}

\verb!List! is an immutable data structure came from the Scala standard library.
\verb!x! is always a list containing \verb!1! and \verb!2!.

\begin{verbatim}
import scala.collection.mutable.ListBuffer
val x = ListBuffer(1, 2)
...
f(x)
...
x
\end{verbatim}

On the other hand, \verb!ListBuffer! is a mutable data structure in the Scala
standard library. It is possible to add an item to or to remove an item from a
list referred by \verb!x!. Programmers are not sure about the content of \verb!x!
unless they read all the lines in between. Besides, function \verb!f! also is
able to change the content of \verb!x!. If one writes a program with a wrong
assumption that \verb!f! does not modify \verb!x!, then the program might be
buggy.

Mutable global variables make code much harder to understand than mutable local
variables.

\begin{verbatim}
def f(x: Int) = g(x, y)
\end{verbatim}

The return value of function \verb!f! depends on the value of global variable
\verb!y!. If \verb!y! is mutable, \verb!f! is not a pure function and expecting
the behavior of \verb!f! is nontrivial. \verb!y! can be declared in any arbitrary
file and all files are able to access \verb!y! and to change the value of
\verb!y!. In the worst case, an external library defines \verb!y! and source code
modifying \verb!y! is not available for reading.

The examples are small and seem artificial but immutability greatly improves
maintainability and readability of code in practice, especially for large
projects.

\subsection{Does not Need Defensive Copies}

\begin{verbatim}
val x = ListBuffer(1, 2)
...
f(x)
...
x
\end{verbatim}

Since \verb!ListBuffer! creates mutable lists, there is no guarantee that the
content of \verb!x! does not change by \verb!f!. If it is necessary to prevent
modification, copying \verb!x! is essential.

\begin{verbatim}
val x = ListBuffer(1, 2)
val y = x.clone
...
f(y)
...
x
\end{verbatim}

In cases that \verb!x! has many elements and the code is executed multiple times,
copying \verb!x! increases the execution time significantly.

In the code, using the \verb!clone! method is enough to copy the list because the
list contains only integral values. However, to pass lists containing mutable
objects safely to functions, defining additional methods for deep copy is
inevitable.

\subsection{Guarantees Safe Accesses from Multiple Threads}

\begin{verbatim}
val x = ListBuffer(1, 2)
def a() = f(x)
def b() = g(x)
\end{verbatim}

Consider two different threads calling functions \verb!a! and \verb!b!,
respectively. Functions \verb!f! and \verb!g! are arbitrarily overlapped so that
each execution can yield a different result from the result of each other. Some
results might be unwanted. Even worse, the \verb!ListBuffer! collection is unsafe
for concurrent accesses from multiple threads and therefore the program perhaps
crashes during execution.

By using \verb!List! instead of \verb!ListBuffer!, the program becomes safe.
Functions \verb!f! and \verb!g! can only read the content of \verb!x! and cannot
change the content. For any possible execution orders, the program behaves
equally and is safe to call functions \verb!a! and \verb!b! from two different
threads.

\subsection{Creates Safe Hash Values}

The hash value of an object is a key to find the object in hash tables. For the
purpose, the same object must have the same hash value all the time. However, the
hash value of a mutable object changes when the object is modified. It is
problematic when using data structures using hash tables like sets and maps.

\begin{verbatim}
val x = ListBuffer(0)
val y = Set(x, ...)
val z = Map(x -> 0, ...)

y.contains(x)
z(x)

x += 1

y.contains(x)
z(x)
\end{verbatim}

The former \verb!y.contains(x)! and \verb!z(x)! yield \verb!true! and \verb!0!,
respectively. After appending \verb!1! to \verb!x!, the hash value of \verb!x!
changes and therefore \verb!y.contains! results in \verb!false! and \verb!z(x)!
throws an exception named \verb!NoSuchElementException!. (Note that \verb!Set!
and \verb!Map! of the Scala standard library use hash tables only when they
contain more than four elements so that the explanation holds only for \verb!y!
and \verb!z! with more than four elements.) Using mutable objects as hash keys is
dangerous while immutable objects maintain the same hash values throughout
execution.

Immutability has several clear advantages. Immutability is important in
functional programming. Functional programs use immutable variables and objects
in most cases. However, mind that immutability is not the silver bullet for every
program. For example, implementing algorithms like sorts in a functional style is
highly inefficient. Use mutable data structures like arrays, mutable variables,
and loops like \verb!for! and \verb!while! to implement algorithms. It makes
programs much more readable and faster. Choosing a proper programming paradigm to
the purpose of a program is the key to write good code.

\section{Recursion}

Repeating the same computation multiple times is a common pattern in programming.
Loops allow concise code expressing such cases. However, if everything is
immutable, going back to the beginnings of loops does not change any states and
therefore it is impossible to apply the same operation on different values for
each iteration or to terminate the loops. As a consequence, loops are useless in
functional programming. Functional programs use recursive functions instead of
loops to rerun computation. A recursive function is a function that calls itself.
To do more computation, the function calls itself with proper arguments.
Otherwise, it terminates with some return value.

The below \verb!factorial! functions calculate the factorial of a given integral
argument. For simplicity, assume that the functions return one for negative
integers. The former uses a loop and the latter uses \term{recursion}.

\begin{verbatim}
def factorial(n: Int) = {
  var i = 1, res = 1
  while (i <= n) {
    res *= i
    i += 1
  }
  res
}

def factorial(n: Int): Int = if (n <= 0) 1 else n * factorial(n - 1)
\end{verbatim}

In Scala, recursive functions require explicit return types.

Recursive functions usually reveal their mathematical definitions more clearly
then functions using loops.

\[n!=\begin{cases}1 & \text{if } n=0\\n \times (n-1)! &
\text{otherwise}\end{cases}\]

The implementation of the \verb!factorial! function using recursion is identical
to the mathematical definition of factorial. Recursion allows not only repetitive
computation but also concise and intuitive descriptions of mathematical
functions. Recursive functions are easier to be verified that they are correct
than imperative versions of the functions. Even if mathematical verification is
unnecessary, recursion is better for intuitive reasoning of functions than loops.
However, some functions become efficient when their implementation uses loops
rather than recursion. Selecting an appropriate implementation strategy is
crucial.

Recursion has disadvantages: overheads of function calls and \term{stack
overflow}. Most modern CPUs have enough computing power to ignore function call
overheads but loops are ideal for functions with short computation time in
programs whose performance is extremely important. Stack overflow happens when a
stack lacks space due to repetitive function calls. It is a critical problem
since it causes immediate termination of execution without yielding meaningful
output. Moreover, programs like web servers do not finish their execution so that
stack overflow must happen. To resolve the problem, many functional languages try
\term{tail call optimization} to prevent stack overflow. The last part of the
article deals with tail call optimization in detail.

\section{Lists}

Almost all programs use lists. Understanding a way to define and to treat lists
in functional programming is good practice. This section defines lists and
functions dealing with lists. The Scala standard library contains \verb!List! but
the section defines its own version of \verb!List!.

\subsection{The Definition of a List}

A list contains finite elements. An order exists among the elements and
duplicates might exist. Simply, a list is an enumeration of a finite number of
values. However, defining lists in functional programming requires a recursive
definition of a list. The following is the definition of a list:

A list is

\begin{itemize}
\item \verb!Nil!: the empty list or
\item \verb!Cons!: a pair of a value and a list.
\end{itemize}

(Lisp, its variants, and some other languages use name \verb!Nil! and
\verb!Cons!. To describe functional lists, the names are typically chosen.)

Any nonempty list has at least one element. The list is divided into the first
element and the remaining elements. The first element is \term{head} and the list
of the remaining elements is \term{tail}. The definition describes every list and
a thing satisfying the definition is a list. The following expresses a list
containing \verb!1!, \verb!2!, and \verb!3! using \verb!Nil! and \verb!Cons!:

\begin{verbatim}
Cons(1, Cons(2, Cons(3, Nil)))
\end{verbatim}

For simple implementation, assume lists contain only integral values as elements.
The following Scala code implements lists:

\begin{verbatim}
trait List
case object Nil extends List
case class Cons(head: Int, tail: List) extends List
\end{verbatim}

The \verb!List! type is a typical \term{algebraic data type}.

\verb!trait List! defines type \verb!List!.

\verb!case object Nil extends List! defines value \verb!Nil! and declares that
\verb!Nil! is a value of type \verb!List!. Keyword \verb!object! implies that
\verb!Nil! is a \term{singleton object}. The code is \term{syntactic sugar} of
the following:

\begin{verbatim}
class Nil$ extends List
val Nil = new Nil$
\end{verbatim}

A singleton object is a unique instance of a class. The class cannot have any
instance other than the singleton object. Since all empty lists are identical to
the single empty list, rather than defining \verb!Nil! as a class and creating
instances whenever the empty list appears, it is more efficient to define the
single \verb!Nil! instance and to refer to the instance for any appearance of the
empty list. Using single \verb!Nil! is safe because it is immutable.

\verb!case class Cons(head: Int, tail: List) extends List! defines a \verb!Cons!
class. Every instance of \verb!Cons! belongs to \verb!List!. An instance has two
fields \verb!head!, whose type is \verb!Int!, and \verb!tail!, whose type is
\verb!List!.

The following code constructs list values:

\begin{verbatim}
Nil
Cons(1, Nil)
Cons(1, Cons(2, Nil))
\end{verbatim}

Functional lists are \term{singly linked lists} in the perspective of data
structures. Accessing the next element is possible while accessing the previous
element is impossible.

\subsection{Functions for Lists}

An arbitrary list is either the empty list or a nonempty list. Programmers use
\term{pattern matching} to split into the two cases, \verb!Nil! and \verb!Cons!.

\begin{verbatim}
def f(l: List) = l match {
  case Nil => "The empty list"
  case Cons(h, t) => "A pair of an integer and a list"
}
\end{verbatim}

Scala requires form \verb![expression] match { case [pattern] => [expression] ... }!
for pattern matching. Firstly, the first expression is evaluated. The result
is compared to the patterns. The result of the whole expression is the result of
an expression corresponding to the first matching pattern. If \verb!l! is
\verb!Nil!, the return value is \verb!"The empty list"!. Otherwise, the return
value is \verb!"A pair of an integer and a list"!. \verb!h! and \verb!t!
respectively refer to the head and tail of \verb!l!. The expression corresponding
to the \verb!Cons! pattern may use \verb!h! and \verb!t!.

The \verb!inc1! function takes a list as an argument and returns a list whose
elements are one larger than the elements of the given list.

\begin{verbatim}
def inc1(l: List): List = l match {
  case Nil => Nil
  case Cons(h, t) => Cons(h + 1, inc1(t))
}
\end{verbatim}

For the given empty list, the function returns the empty list. Otherwise, the
return value is a list whose head is one larger than the head of the given list
and tail has elements that are one larger than the elements of the tail of the
given list.

Define function \verb!square!, which takes a list as an argument and returns a
list whose elements are the squares of the elements of the given list. Check the
below code after trying to find an answer.

\begin{verbatim}
def square(l: List): List = l match {
  case Nil => Nil
  case Cons(h, t) => Cons(h * h, square(t))
}
\end{verbatim}

The \verb!odd! function takes a list as an argument and returns a list whose
every element is an odd integer.

\begin{verbatim}
def odd(l: List): List = l match {
  case Nil => Nil
  case Cons(h, t) =>
    if (h \% 2 != 0) Cons(h, odd(t))
    else odd(t)
}
\end{verbatim}

For a nonempty list, the function checks whether the head is odd or not. If the
head is odd, the resulting list contains the head and its tail is the tail with
only odd integers. Otherwise, the head is removed.

Define function \verb!positive!, which takes a list as an argument and returns a
list whose every element is a positive integer.

\begin{verbatim}
def positive(l: List): List = l match {
  case Nil => Nil
  case Cons(h, t) =>
    if (h > 0) Cons(h, positive(t))
    else positive(t)
}
\end{verbatim}

The \verb!length! function calculates the length of a given list.

\begin{verbatim}
def length(l: List): Int = l match {
  case Nil => 0
  case Cons(h, t) => 1 + length(t)
}
\end{verbatim}

The length of the empty list is zero. The length of a nonempty list is one larger
than the length of its tail.

Define functions \verb!sum! and \verb!product!, which respectively calculate the
sum and the product of the elements of a given list.

\begin{verbatim}
def sum(l: List): Int = l match {
  case Nil => 0
  case Cons(h, t) => h + sum(t)
}
\end{verbatim}

\begin{verbatim}
def product(l: List): Int = l match {
  case Nil => 1
  case Cons(h, t) => h * product(t)
}
\end{verbatim}

Define function \verb!addBack!, which takes a list and an integer as arguments
and return a list obtained by appending the integer at the end of the list.

\begin{verbatim}
def addBack(l: List, n: Int): List = l match {
  case Nil => Cons(n, Nil)
  case Cons(h, t) => Cons(h, addBack(t, n))
}
\end{verbatim}

Adding an element at the end of a list requires \term{time complexity} of
\(O(n)\). \term{Space complexity} also is \(O(n)\) since the function creates a
new entire list. In contrast, prepending an element at the beginning of a list
requires both time and space complexity of \(O(1)\). Therefore, adding an element
at the beginning instead of the end of a list is desirable in functional
programming.

\section{Tail Call Optimization}

If the last action of a function is calling a function, then the call is a tail
call. When a tail call happens, after the call, the \term{callee} does every
computation and thus the local variables of the \term{caller} have no need to
remain. The stack frame of the caller can be destroyed. Most functional languages
optimize tail calls. At compile time, compilers check whether calls are tail
calls. If a call is a tail call, the compilers generate code that eliminates the
stack frame of the caller before the call. They do not optimize non-tail function
calls because the local variables of the callers can be used after returning from
the callees. If every function call in a program is a tail call, a stack never
grows so that the program is safe from stack overflow.

\begin{verbatim}
def factorial(n: Int): Int = if (n <= 0) 1 else n * factorial(n - 1)
\end{verbatim}

The previous \verb!factorial! function multiplies \verb!n! and the return value
of the recursive \verb!factorial(n -1)! call. The multiplication is the last
action. The recursive call is not a tail call. The stack frame of the caller must
remain. The following process computes \verb!factorial(3)!:

\begin{itemize}
\item \verb!factorial(3)!
\item \verb!3 * factorial(2)!
\item \verb!3 * (2 * factorial(1))!
\item \verb!3 * (2 * (1 * factorial(0)))!
\item \verb!3 * (2 * (1 * 1))!
\item \verb!3 * (2 * 1)!
\item \verb!3 * 2!
\item \verb!6!
\end{itemize}

At most four stack frames coexist. For a large enough argument, a stack grows
again and again and finally overflows.

\begin{verbatim}
scala> def factorial(n: Int): Int = if (n <= 0) 1 else n * factorial(n - 1)
factorial: (n: Int)Int

scala> factorial(10000)
java.lang.StackOverflowError
  at .factorial(<console>:12)
\end{verbatim}

To implement the function using a tail call, instead of multiplying \verb!n! and
\verb!factorial(n - 1)!, the function has to pass both \verb!n! and \verb!n - 1!
as arguments and to make the callee multiply them. One can interpret the strategy
as passing an intermediate result.

\begin{itemize}
\item \verb!factorial(3)!
\item \verb!factorial(2, intermediate result = 3)!
\item \verb!factorial(1, intermediate result = 3 * 2)!
\item \verb!factorial(1, intermediate result = 6)!
\item \verb!factorial(0, intermediate result = 6 * 1)!
\item \verb!factorial(0, intermediate result = 6)!
\item \verb!6!
\end{itemize}

There is no need to return to the caller. The below code shows the
\verb!factorial! function with a tail call. The function needs one more parameter
that takes an intermediate result. \verb!factorial(n, i)! computes \(n!\times
i\).

\begin{verbatim}
def factorial(n: Int, inter: Int): Int =
  if (n <= 0) inter else factorial(n - 1, inter * n)
\end{verbatim}

The function uses the tail call. More precisely, the function is
\term{tail-recursive}. Its last action is calling itself. Unlike most functional
languages, Scala cannot optimize general tail calls. Scala optimizes only
tail-recursive calls.

Scala compilers generate Java bytecode, whom JVMs execute. The JVMs does not
allow to jump to the beginning of another function. In the JVMs, functions can
only either return or call functions. The Scala compilers cannot generate
optimized code by removing the stack frame of the caller. Instead, they transform
tail-recursive calls into loops. \verb!javap! disassembles Java bytecode files.
The compiled and disassembled \verb!factorial! function using tail recursion does
not call any function but jumps to instructions inside the function.

\begin{verbatim}
public int factorial(int, int);
  Code:
     0: iload_1
     1: iconst_0
     2: if_icmpgt     9
     5: iload_2
     6: goto          20
     9: iload_1
    10: iconst_1
    11: isub
    12: iload_2
    13: iload_1
    14: imul
    15: istore_2
    16: istore_1
    17: goto          0
    20: ireturn
\end{verbatim}

The function does not result in stack overflow.

(The result is still abnormal due to integer overflow. The \verb!BigInt! type
resolves it.)

The transformation not only prevents stack overflow but also removes the
overheads of function calls. The downside is that \term{mutually recursive
functions} using tail calls lie beyond the scope of the transformation. The below
functions may cause stack overflow.

\begin{verbatim}
def even(n: Int): Boolean = if (n <= 0) true else odd(n - 1)
def odd(n: Int): Boolean = if (n == 1) true else even(n - 1)
\end{verbatim}

In Scala, by using \term{annotations}, programmers ask the compilers to check
whether functions are tail-recursive. The annotations prevent non-tail-recursive
function made by mistakes.

\begin{verbatim}
import scala.annotation.tailrec
@tailrec def factorial(n: Int, inter: Int): Int =
  if (n <= 0) inter else factorial(n - 1, inter * n)
\end{verbatim}

A non-tail-recursive function with the \verb!tailrec! annotation results in a
compile error. The annotation does not affect the behaviors of the compilers.
Regardless of the existence of the annotation, the compilers always optimize
tail-recursive functions. Still, using the annotations is desirable to prevent
mistakes.

Calling the tail-recursive version of \verb!factorial! needs the unnecessary
second argument. The below code defines a new \verb!factorial! function with one
parameter and uses the tail-recursive one as a local function inside the
function.

\begin{verbatim}
def factorial(n: Int): Int = {
  @tailrec def aux(n: Int, inter: Int): Int =
    if (n <= 0) inter else aux(n - 1, inter * n)
  aux(n, 1)
}
\end{verbatim}

By revising \verb!length!, it becomes tail-recursive:

\begin{verbatim}
def length(l: List): Int = {
  @tailrec def aux(l: List, inter: Int): Int = l match {
    case Nil => inter
    case Cons(h, t) => aux(t, inter + 1)
  }
  aux(l, 0)
}
\end{verbatim}

The \verb!sum! and \verb!product! functions can be modified in the same manner.

\begin{verbatim}
def sum(l: List): Int = {
  @tailrec def aux(l: List, inter: Int): Int = l match {
    case Nil => inter
    case Cons(h, t) => aux(t, inter + h)
  }
  aux(l, 0)
}
\end{verbatim}

\begin{verbatim}
def product(l: List): Int = l match {
  @tailrec def aux(l: List, inter: Int): Int = l match {
    case Nil => inter
    case Cons(h, t) => aux(t, inter * h)
  }
  aux(l, 1)
}
\end{verbatim}

The following is tail-recursive \verb!inc1!.

\begin{verbatim}
def inc1(l: List): List = {
  @tailrec def aux(l: List, inter: List): List = l match {
    case Nil => inter
    case Cons(h, t) => aux(t, addBack(inter, h + 1))
  }
  aux(l, Nil)
}
\end{verbatim}

Its result is correct but takes \(O(n^2)\) time while the original version
requires only \(O(n)\).

It is possible to implement tail-recursive \verb!inc1! using \(O(n)\) time.

\begin{verbatim}
def reverse(l: List): List = {
  @tailrec def aux(l: List, inter: List): List = l match {
    case Nil => inter
    case Cons(h, t) => aux(t, Cons(h, inter))
  }
  aux(l, Nil)
}

def inc1(l: List): List = {
  @tailrec def aux(l: List, inter: List): List = l match {
    case Nil => inter
    case Cons(h, t) => aux(t, Cons(h + 1, inter))
  }
  reverse(aux(l, Nil))
}
\end{verbatim}

\setchapterpreamble[u]{\margintoc}
\chapter{First-Class Functions}
\labch{first-class-functions}

\renewcommand{\plang}{\textsf{VAE}\xspace}
\renewcommand{\lang}{\textsf{FVAE}\xspace}

\textit{First-class functions}\index{first-class function} are functions that
can be used as values. They are much more expressive than first-order functions,
which are the topic of the previous chapter. This chapter explains the semantics
of first-class functions. We need to introduce the notion of a closure to make
first-class functions work properly. We will see what closures are and why they
are necessary.

This chapter defines \lang by extending \plang with first-class functions.
The only way to create a function in \lang is to make an \textit{anonymous
function}\index{anonymous function}, which is a function without a name.
However, we can add named functions as syntactic sugar. In addition,
we will see that even variable definitions can be considered as syntactic sugar.

\section{Syntax}

Consider an anonymous function in Scala:

\begin{verbatim}
(x: Int) => x + x
\end{verbatim}

If we ignore its type annotation, it consists of two parts: \code{x} and \code{x
+ x}. \code{x} is the parameter of the function; \code{x + x} is the body of the
function. This observation lets us know that an anonymous function consists of
its name and parameter.

In \textsf{F1VAE}, the sytnax of a function call is $\eappfo{x}{e}$. To call a
function, the name of the function should be given. However, it is not true in
languages with first-class functions. Let us see some function calls in Scala.

\begin{verbatim}
def twice(x: Int): Int = x + x
twice(1)
\end{verbatim}

\code{twice(1)} is a function call, and it designates a function by its name.

\begin{verbatim}
def makeAdder(x: Int): Int => Int =
  (y: Int) => x + y
makeAdder(3)(5)
\end{verbatim}

\code{makeAdder} is a function that returns a function. \code{makeAdder(3)} is a
function call, and its result is a function. Therefore, we can call the
resulting function again. \code{makeAdder(3)(5)} is an expression that calls
\code{makeAdder(3)}. It designates a function by an expression, rather than just
a name. We can conclude that the syntax of a function call in \lang should be
more general than \textsf{F1VAE} because of the presence of first-class
functions. In \lang, a function call consists of two expressions: one determines
the function to be called and the other determines the value of the argument.

We have used the term function call so far. In the context of functional
programming, we use the term \textit{function application}\index{function
application} more frequently. When we see \code{f(1)}, we say ``\code{f} is
applied to \code{1}'' instead of ``\code{f} is called with the argument
\code{1}.'' Applications sound more natural than calls especially when we are
talking about first-class functions. For example, we usually say
``\code{makeAdder(3)} is applied to \code{5}'' rather than ``\code{makeAdder(3)}
is called with the argument \code{5}.''

From the above observation on anonymous functions and function applications,
we can define the syntax of \lang. The following is the syntax of \lang:
\sidenote{We omit the common part to \plang.}

\[ e\ ::=\ \cdots\ |\ \efun{x}{e}\ |\ \eapp{e}{e} \]

\begin{itemize}
  \item $\efun{x}{e}$

    It is called an anonymous function or a \textit{lambda abstraction}.
    \index{lambda abstraction}
    It denotes a function whose parameter is $x$ and body is $e$. $x$ is a
    binding occurrence, and its scope is $e$.
    A function has zero or more parameters in many real-world languages,
    but we restrict a function in \lang to have one and only one parameter for
    simplicity as before.

  \item $\eapp{e_1}{e_2}$

    It is a function application, or just an application in short.
    $e_1$ denotes the function; $e_2$ denotes the argument.
\end{itemize}

\section{Semantics}

Integers are the only values in \plang. It is not true in \lang. Since
first-class functions are values, a value of \lang is either an integer or a
function. Thus, we define a new kind of semantic element, the value. The
metavariable $v$ ranges over values. Also, let $V$ be the set of every value.

\[ v\ ::=\ n\ |\ \clov{x}{e}{\sigma} \]

A value is either an integer or a closure. A \textit{closure}\index{closure}, which is a
function as a value, has the form $\clov{x}{e}{\sigma}$.
It is a pair of a lambda abstraction and an environment.
A lambda abstraction in a closure may have free identifiers,
but the environment of the closure can store the values denoted by the
free identifiers.

To discuss the necessity of closures, consider the following expression:

\[\eapp{\eapp{(\efun{\cx}{\efun{\cy}{\eadd{\cx}{\cy}}})}{1}}{2}\]

It is equivalent to \code{((x: Int) => (y: Int) => x + y)(1)(2)} in Scala.
The function $\efun{\cx}{\efun{\cy}{\eadd{\cx}{\cy}}}$ does not have any free
identifiers. The scope of \code{x} is ${\efun{\cy}{\eadd{\cx}{\cy}}}$; the scope
of \code{y} is ${\eadd{\cx}{\cy}}$. Therefore, both \code{x} and \code{y} in
$\eadd{\cx}{\cy}$ are bound occurrences. The whole expression must result in $3$,
which equals $1+2$, without a run-time error. You can check it by running the
equivalent Scala program.

If we consider a function value as just a lambda abstraction, not a closure,
evaluation of the above expression becomes problematic.
When the expression is evaluated, $\efun{\cx}{\efun{\cy}{\eadd{\cx}{\cy}}}$ is
applied to $1$ first. The result is a function value, which is a lambda
abstraction: $\efun{\cy}{\eadd{\cx}{\cy}}$. Next, $\efun{\cy}{\eadd{\cx}{\cy}}$
is applied to $2$. The result of the application can be computed by evaluating
$\eadd{\cx}{\cy}$ under the environment containing that $\cy$ denotes $2$.
However, there is no way to find the value of $\cx$. $\cx$ has become free
identifier although it was not in the beginning.

We adopt the notion of a closure to resolve the problem. When a lambda
expression evaluates to a function value, which is a closure, it captures the
environment. Since $\efun{\cy}{\eadd{\cx}{\cy}}$ is evaluated under the
environment containing that $\cx$ denotes $1$, its result is
$\clov{\cy}{\eadd{\cx}{\cy}}{[\cx\mapsto1]}$. The captured environment of the
closure records that $\cx$ is not a free identifier and denotes $1$.
When the closure is applied to $2$, its body ${\eadd{\cx}{\cy}}$ is evaluated
under $[\cx\mapsto1,\cy\mapsto2]$, not $[\cy\mapsto2]$. The addition
successfully results in $3$.

In summary, we need closures to retain the static scope semantics. First-class
functions can appear at any places expecting expressions.
However, the environments used for the
evaluation of their bodies must be determined statically. In other words,
the denotation of identifiers in the bodies of functions must be determined when
the functions are defined, not used. Therefore, each closure captures the surrounding
environment when it is created.

Now, let us define the semantics of \lang. Most things are the same as the
semantics of \plang, but we should be aware of that values now include not only
integers but also closures.

An environment is a finite partial function from identifiers to values.

\[ \embox{Env}=\embox{Id}\finto V \]

The semantics of \lang is a ternay relation over $\embox{Env}$, $E$, and $V$.

\[\Rightarrow\subseteq\embox{Env}\times E\times V\]

$\evald{e}{v}$ is true if and only if $e$ evaluates to $v$ under $\sigma$.

A lambda abstraction creates a closure containing the current environment.

\semanticrule{Fun}{
  \evaldn{\efun{x}{e}}{\clov{x}{e}{\sigma}}.
}

\vspace{-1em}

\[
  \evald{\efun{x}{e}}{\clov{x}{e}{\sigma}}
  \quad\textsc{[Fun]}
\]

A function application evaluates its both subexpressions. Then, it evaluates the
body of the closure under the environment obtained by adding the value of the
argument to the environment of the closure.

\semanticrule{App}{
  \begin{tabular}{@{\hskip0pt}l@{\hskip0pt}l}
    If \\
    & \evaldn{e_1}{\clov{x}{e}{\sigma'}}, \\
    & \evaldn{e_2}{v'}, and \\
    & \evaln{\sigma'[x\mapsto v']}{e}{v}, \\
    then \\
    & \evaldn{\eapp{e_1}{e_2}}{v}.
  \end{tabular}
}

\vspace{-1em}

\[
  \inferrule
  {
    \evald{e_1}{\clov{x}{e}{\sigma'}} \\
    \evald{e_2}{v'} \\
    \eval{\sigma'[x\mapsto v']}{e}{v}
  }
  { \evald{\eapp{e_1}{e_2}}{v} }
  \quad\textsc{[App]}
\]

We can reuse Rule \textsc{Num}, Rule \textsc{Add}, Rule \textsc{Sub}, and Rule \textsc{Id} of
\plang. However, it is important to note that \lang has more
cases that evaluation can fail than \plang. For example, consider Rule \textsc{Add}.

\semanticrule{Add}{
If
  \evaldn{e_1}{n_1}, and
  \evaldn{e_2}{n_2},\\
then
  \evaldn{\eadd{e_1}{e_2}}{n_1+n_2}.
}

\vspace{-1em}

\[
  \inferrule
  {
    \evald{e_1}{n_1} \\
    \evald{e_2}{n_2}
  }
  { \evald{\eadd{e_1}{e_2}}{n_1+n_2} }
  \quad\textsc{[Add]}
\]

The rule assumes the results of $e_1$ and $e_2$ to be integers. If the
assumption is violated, a run-time error happens. For example,
$\eadd{(\efun{\cx}{\cx})}{1}$ incurs a run-time error becuase the left operand
is a closure, not an integer.

We need to revise Rule \textsc{Val} of \plang a bit.
Since every value is an integer in \plang, a variable of \plang can denote only
an integer. In \lang, a variable should be able to denote a general value, not
only an integer.

\semanticrule{Val}{
If
  \evaldn{e_1}{v_1}, and
  \evaln{\sigma[x\mapsto v_1]}{e_2}{v_2},\\
then
  \evaldn{\ebind{x}{e_1}{e_2}}{v_2}.
}

\vspace{-1em}

\[
  \inferrule
  {
    \evald{e_1}{v_1} \\
    \eval{\sigma[x\mapsto v_1]}{e_2}{v_2}
  }
  { \evald{\ebind{x}{e_1}{e_2}}{v_2} }
  \quad\textsc{[Val]}
\]

Now, a variable can denote a value, not only an integer.

The following proof trees prove that
$\eapp{\eapp{(\efun{\cx}{\efun{\cy}{\eadd{\cx}{\cy}}})}{1}}{2}$
evaluates to $3$ under the empty environment.
The proof splits into three trees for readability.
Suppose that $\sigma_1=[\cx\mapsto1]$ and $\sigma_2=[\cx\mapsto1,\cy\mapsto2]$.

\[
  \inferrule
  {
    \eval{\emptyset}{
      \efun{\cx}{\efun{\cy}{\eadd{\cx}{\cy}}}
    }{\clov{\cx}{\efun{\cy}{\eadd{\cx}{\cy}}}{\emptyset}}
    \\
    \eval{\emptyset}{1}{1}
    \\
    \eval{\sigma_1}{
      \efun{\cy}{\eadd{\cx}{\cy}}
    }{\clov{\cy}{\eadd{\cx}{\cy}}{\sigma_1}}
  }
  { \eval{\emptyset}{
      \eapp{(\efun{\cx}{\efun{\cy}{\eadd{\cx}{\cy}}})}{1}
    }{\clov{\cy}{\eadd{\cx}{\cy}}{\sigma_1}}
  }
\]

\[
  \inferrule
  {
    \inferrule
    { \cx\in\dom{\sigma_2} }
    { \eval{\sigma_2}{\cx}{1} }
    \\
    \inferrule
    { \cy\in\dom{\sigma_2} }
    { \eval{\sigma_2}{\cy}{2} }
  }
  { \eval{\sigma_2}{\eadd{\cx}{\cy}}{3} }
\]

\[
\inferrule
{
  \eval{\emptyset}{
      \eapp{(\efun{\cx}{\efun{\cy}{\eadd{\cx}{\cy}}})}{1}
    }{\clov{\cy}{\eadd{\cx}{\cy}}{\sigma_1}}
  \\
  \emptyset\vdash2\Rightarrow 2
  \\
  \eval{\sigma_2}{\eadd{\cx}{\cy}}{3}
}
{ \eval{\emptyset}{
    \eapp{\eapp{(\efun{\cx}{\efun{\cy}{\eadd{\cx}{\cy}}})}{1}}{2}
  }{3}
}
\]

\section{Interpreter}

The following Scala code implements the syntax of \lang:
\sidenote{We omit the common part to \plang.}

\begin{verbatim}
sealed trait Expr
...
case class Fun(x: String, b: Expr) extends Expr
case class App(f: Expr, a: Expr) extends Expr
\end{verbatim}

\code{Fun($x$, $e$)} represents $\efun{x}{e}$; \code{App($e_1$, $e_2$)}
represents $\eapp{e_1}{e_2}$.

A value of \lang is either an integer or a closure. Thus, we represent a value
as an ADT.

\begin{verbatim}
sealed trait Value
case class NumV(n: Int) extends Value
case class CloV(p: String, b: Expr, e: Env) extends Value
\end{verbatim}

\code{NumV($n$)} represents $n$; \code{CloV($x$, $e$, $\sigma$)} represents
$\clov{x}{e}{\sigma}$.

An environment is a finite partial function from identifiers to values.
Therefore, the type of an environment is \code{Map[String, Value]}.

\begin{verbatim}
type Env = Map[String, Value]
\end{verbatim}

The following function evaluates a given expression under a given environment:

\begin{verbatim}
def interp(e: Expr, env: Env): Value = e match {
  case Num(n) => NumV(n)
  case Add(l, r) =>
    val NumV(n) = interp(l, env)
    val NumV(m) = interp(r, env)
    NumV(n + m)
  case Sub(l, r) =>
    val NumV(n) = interp(l, env)
    val NumV(m) = interp(r, env)
    NumV(n - m)
  case Id(x) => env(x)
  case Fun(x, b) => CloV(x, b, env)
  case App(f, a) =>
    val CloV(x, b, fEnv) = interp(f, env)
    interp(b, fEnv + (x -> interp(a, env)))
}
\end{verbatim}

In the \code{Num} case, the return value is \code{NumV(n)}, not \code{n},
since the function must return a value of the type \code{Value}.

In the
\code{Add} and \code{Sub} cases, we cannot assume that the operands are integers
any longer. We use pattern matching to discriminate integers from closures. If
both operands are integers, addition or subtraction succeeds. Otherwise, at
least one of them is a closure, and the interpreter crashes due to a pattern
matching failure. Note that this code is equivalent to the following code:

\begin{verbatim}
case Add(l, r) =>
  interp(l, env) match {
    case NumV(n) => interp(r, env) match {
      case NumV(m) => NumV(n + m)
      case _ => error("not an integer")
    }
    case _ => error("not an integer")
  }
\end{verbatim}

Similarly, in the \code{App} case, we use pattern matching to discriminate
closures from integers. The first expression of \code{App} must yield a clsoure,
not an integer, to make the execution succeed.

\section{Syntactic Sugar}
\labsec{fae}

We can add named local functions to \lang with the following change in the
syntax:

\[
  e\ ::=\ \cdots\ |\ \erec{x}{x}{e}{e}
\]

$\erec{x_1}{x_2}{e_1}{e_2}$ defines a function whose name is $x_1$, parameter is
$x_2$, and body is $e_1$. The scope of $x_1$ is $e_2$, and thus the function
does not allow recursion.

Instead of changing the semantics, \lang can provide named local functions as
syntactic sugar. Let $s$ be a string transformed into $\erec{x_1}{x_2}{e_1}{e_2}$
by the parser of \lang with named local functions embeded in the semantics.
To treat named local functions as syntactic sugar, the parser should transform
$s$ into $\ebind{x_1}{\efun{x_2}{e_1}}{e_2}$.

Variable definitions can be considered as syntactic sugar as well.
Let $s$ be a string transformed into $\ebind{x}{e_1}{e_2}$.
To make variable definitions syntactic sugar, the parser can transform $s$ into
$\eapp{(\efun{x}{e_2})}{e_1}$. The evaluation of $\eapp{(\efun{x}{e_2})}{e_1}$
evaluates $e_1$ first. Then, $e_2$ is evaluated under the environment that $x$ denotes
the result of $e_1$. This semantics is exactly the same as that of
$\ebind{x}{e_1}{e_2}$. Therefore, we can say that variable definitions are just
syntactic sugar in \lang.

Hereafter, we remove variable definitions from \lang and call the language
\textsf{FAE}. However, we may still use variable definitions in examples. It is
completely fine because they are considered as syntactic sugar.

Furthermore, we can treat even integers, addition, and subtraction as syntactic
sugar. The only things we need are variables, lambda abstractions, and function
applications. We can write any programs with these three kinds of expressions.
The \textit{lambda calculus}\index{lambda calculus} is a language that provides
only the three features. This book does not discuss how integers, addition, and
subtraction can be desugared into the lambda calculus.

\section{Exercises}

\begin{enumerate}
\item Consider the following expression:

\[
\ebind{\cx}{5}{
    \ebind{\code{f}}{\efun{\cy}{\eadd{\cy}{\cx}}}{
        \eapp{(\efun{\code{g}}{\eapp{\code{f}}{(\eapp{\code{g}}{1})}})}
        {(\efun{\cx}{\cx})}
    }
}
\]
Describe a trace of the evalaution in terms of arguments to the \code{interp}
function for every call. (There will be 16 calls.) The \code{interp} function
takes two arguments---an expression and an environment---so show both for each call.
For \code{Num}, \code{Id}, and \code{Fun} expressions, show their result values, which
are immediate. You can use the following abbreviations and possibly more abbreviations:

\begin{center}
\begin{tabular}{lcl}
$E_0$ & = & the whole expression \\
$E_1$ & = & $\efun{\cy}{\eadd{\cy}{\cx}}$ \\
$E_2$ & = & $\efun{\code{g}}{\eapp{\code{f}}{(\eapp{\code{g}}{1})}}$ \\
$E_3$ & = & $\efun{\cx}{\cx}$ \\
$E_4$ & = & $\ebind{\code{f}}{E_1}{\eapp{E_2}{E_3}}$
\end{tabular}
\end{center}

\item This exercise examines differences between semantics by changing scope.
The following code is an implementation of an interpreter:

\begin{verbatim}
def interp(e: Expr, env: Env): Value = e match {
  case Num(n) => NumV(n)
  case Add(l, r) =>
    val NumV(n) = interp(l, env)
    val NumV(m) = interp(r, env)
    NumV(n + m)
  case Sub(l, r) =>
    val NumV(n) = interp(l, env)
    val NumV(m) = interp(r, env)
    NumV(n - m)
  case Id(x) => lookup(x, env)
  case Fun(x, b) => CloV(x, b, env)
  case App(f, a) =>
    val CloV(x, b, fEnv) = interp(f, env)
    interp(b, __________ + (x -> interp(a, env)))
}
\end{verbatim}

Describe the semantics of the \code{App} case in prose
when we use each of the following for the blank above:
\begin{itemize}
  \item \code{env}
  \item \code{Map()}
  \item \code{fEnv}
\end{itemize}

\item This exercise extends \lang to support multiple parameters.
Consider the following language:
\[
\begin{array}{lrrl}
  \text{Expression}& e & ::= & \cdots\ |\ \efun{x\cdots x}{e}\ |\
  \eappfo{e}{e,\cdots,e}\\
  \text{Value}& v & ::= & \cdots\ |\ \clov{x\cdots x}{e}{\sigma} \\
\end{array}
\]
The semantics of some constructs are as follows:
\begin{itemize}
  \item Evaluating $\lambda x_1\ \cdots\ x_n. e$ under $\sigma$
      yields a closure $\langle \lambda x_1 \cdots\ x_n.e,\sigma \rangle$.
  \item If
    \begin{itemize}
    \item evaluating $e_0$ under $\sigma$ yields a closure $\langle \lambda x_1
      \cdots\ x_n.e,\sigma' \rangle$,
    \item evaluating $e_i$ under $\sigma$ yields $v_i$ for each $1 \leq i \leq
      n$, and
    \item evaluating $e$ under $\sigma'[x_1 \mapsto v_1, \cdots, x_n \mapsto
      v_n]$ yields $v$,
    \end{itemize}
\item[] then evaluating $\eappfo{e}{e,\cdots,e}$ under $\sigma$ yields $v$.
\end{itemize}

\begin{enumerate}
  \item Write the operational semantics of the form \fbox{$\sigma\vdash e \Rightarrow v$} for the expressions.
  \item Write the evaluation derivation of the following expression:
\[
\inferrule
{ \hspace*{0.8\textwidth} }
{
  \eval{\emptyset}{
    \eappfo{(\efun{\code{f}\
    \code{m}}{\eappfo{\code{f}}{\code{m}}})}{\efun{\cx}{\cx},8}
  }{~~~~~~~~~~}
}
\]
\end{enumerate}

\item Rewrite the following \lang expression to an \textsf{FAE} expression.
  You should not ``evaluate'' it. Consider the approach of \refsec{fae}.

\[
  \ebind{\cx}{\efun{\cy}{\eadd{8}{\cy}}}{\efun{\cy}\eapp{\eapp{\cx}{(\esub{10}{\cy})}}}
\]

\item This exercise modifies \lang to check body expressions when evaluating
  function expressions.
Consider we extend the value of \lang to include a special value $\uparrow$
to represent an error during function body checking.
Write the operational semantics of the form
\fbox{$\evald{e}{v}$} for a function expression
$\efun{x}{e}$, when its semantics  changes as follows:
\begin{itemize}
\item If every free identifier of $e$ is in the domain of $\sigma$ or is $x$,
  then evaluation of $\efun{x}{e}$ under $\sigma$ yields a closure
  containing the function expression and the environment.
\item Otherwise, evaluation of $\efun{x}{e}$ under $\sigma$ yields $\uparrow$.
\end{itemize}
You may use the semantic function \embox{fv}, which takes an
expression and returns the set of every free identifier in the expression.
For example, $\embox{fv}(\efun{\cx}{\eapp{\cy}{\cx}}) = \{ \cy \}$.

\item This exercise extends \lang with records.
  Consider the following language:
\[
\begin{array}{lrrl}
  \text{Field} & f & \in & \textit{Field} \\
  \text{Record} & \rho & \in & \embox{Record} = \embox{Field} \finto \embox{Value} \\
  \text{Expression}& e & ::= & \cdots\ |\ \{f\ e,\ \ldots,\ f\ e\}\ |\ e.f\ |\
  e;e \\
  \text{Value} & v & ::= & \cdots\ |\ \rho \\
\end{array}
\]

The semantics of some constructs are as follows:
\begin{itemize}
  \item The evaluation of $\{f_1\ e_1,\ \cdots,\ f_k\ e_k\}$
    under $\sigma$ yields a finite map $\rho$,
which maps $f_i \in \{f_1\ \cdots,\ f_k\}$
to the value $v_i$ which is evaluated from the expression $e_i$ under $\sigma$.
  \item The evaluation of $e.f$ under $\sigma$ yields the value of the field $f$ in the record $\rho$,
      where evaluation $e$ under $\sigma$ yields $\rho$.
  \item If evaluation of $e_1$ yields some value under $\sigma$, and evaluation
    of $e_2$ yields $v$ under $\sigma$,
      then evaluation of $e_1; e_2$ yields $v$ under $\sigma$.
\end{itemize}

Write the operational semantics of the form
$\boxed{\sigma \vdash e \Rightarrow v}$

\item This exercise extends \lang with pairs. Consider the following language:
\[
\begin{array}{lrrl}
  \text{Expression}& e & ::= & \cdots\ |\ (e,e)\ |\ e\textsf{.1}\ |\ e\textsf{.2} \\
  \text{Value} & v & ::= & \cdots\ |\ \fbox{ (a) }
\end{array}
\]

\begin{enumerate}
  \item Write the syntax of a pair value in \fbox{ (a) } and
    the operational semantics of the form \fbox{$\evald{e}{v}$} for the expressions.
  \item Write the evaluation derivation of the following expression:
    \[
      \inferrule
      {\hspace*{0.8\textwidth}}
      { \eval{\emptyset}{(8,(320,42)\textsf{.1})\textsf{.2}}{~~~~~} }
    \]
\end{enumerate}

\item This exercise replaces the environment-based semantics of \lang with 
a substitution-based semantics.
Consider the following implementation:
\begin{verbatim}
trait Expr
trait Value extends Expr
case class Num(n: Int) extends Expr with Value
case class Add(l: Expr, r: Expr) extends Expr
case class Sub(l: Expr, r: Expr) extends Expr
case class Val(x: String, e: Expr, b: Expr) extends Expr
case class Id(x: String) extends Expr
case class Fun(x: String, b: Expr) extends Expr with Value
case class App(f: Expr, a: Expr) extends Expr

def subst(e: Expr, x: String, v: Value): Expr = e match {
  case Num(n) => e
  case Add(l, r) =>
    Add(subst(l, x, v), subst(r, x, v))
  case Sub(l, r) =>
    Sub(subst(l, x, v), subst(r, x, v))
  case Val(y, i, b) =>
    val nb = if (y == x) b else subst(b, x, v)
    Val(y, subst(i, x, v), nb)
  case Id(name) =>
    if (name == x) v else e
  case Fun(y, b) =>
    Fun(y, if (y == x) b else subst(b, x, v))
  case App(f, a) =>
    App(subst(f, x, v), subst(a, x, v))
}

def interp(e: Expr): Value = e match {
  case Num(n) => Num(n)
  case Add(l, r) =>
    val Num(n) = interp(l)
    val Num(m) = interp(r)
    Num(n + m)
  case Sub(l, r) =>
    val Num(n) = interp(l)
    val Num(m) = interp(r)
    Num(n + m)
  case Val(x, i, b) =>
    interp(subst(b, x, interp(i)))
  case Id(x) => error("free identifier")
  case Fun(x, b) => Fun(x, b)
  case App(f, a) =>
    val Fun(x, b) = interp(f)
    interp(subst(b, x, interp(a)))
}
\end{verbatim}

In this implementation, a value is either an integer or a lambda abstraction:

\[ v\ ::=\ n\ |\ \efun{x}{e} \]

\begin{enumerate}
  \item
    Write the operational semantics of the above implementation
    of the form \fbox{$e \Rightarrow v$}
    where $e[x/v]$ denotes \code{subst($e$, $x$, $v$)}.

  \item Write the definition of the substitution $e[x/v]$
    of the form \fbox{$e[x/v]=e$}:

  \item Consider the following expression:
\[
\ebind{\code{z}}{\efun{\cx}{\esub{\cx}{\cy}}}{
    \ebind{\cy}{10}{
        (\eapp{\code{z}}{32})
    }
}
\]

\begin{enumerate}
\item What is the result of evaluating the expression under the empty
  environment in substitution-based \lang?
\item What is the result of evaluating the expression under the empty
  environment in environment-based \lang?
\item Why are the results different?
  \sidenote{We can make the semantics of substitution-based \lang equivalent to
    environment-based \lang by modifying \code{subst} function.}
\end{enumerate}
\end{enumerate}

\item Consider the following language:
\[
  \begin{array}{lrrl}
  \text{Expression}& e & ::= & n\ |\ x\ |\ \efun{x\cdots x}{e}\ |\
  \eappfo{e}{e,\cdots,e}\ |\ \textsf{get}\ e \\
  \text{Value}& v & ::= & n\ |\ \textsf{undefined}\ |\ \clov{x\cdots x}{e}{\sigma} \\
  \end{array}
\]
The semantics of some constructs are as follows:
\begin{itemize}
  \item The value of a function expression $\efun{x_1\cdots x_n}{e}$
    at an environment $\sigma$ is a closure $\clov{x_1\cdots x_n}{e}{\sigma}$.
  \item A function application $\eappfo{e_0}{e_1,\cdots,e_n}$ is evaluated as follows:
    \begin{itemize}
      \item Evaluate the subexpressions in order.
        The value of $e_0$ should be a closure
        $\clov{x_1\cdots x_m}{e}{\sigma}$
        that has $m$ parameters.
      \item Create an array $\alpha$ of size $n$ and
        initialize the $i$-th value of the array with the value of $e_{i+1}$
        where $0 \le i \le n-1$.
      \item Evaluate the closure body $e$ under the environment $\sigma$
        extended as follows:
        \begin{itemize}
          \item The value of the $i$-th parameter is the value of $e_i$
            where $1 \le i \le m \le n$.
          \item The value of the $j$-th parameter is the \textsf{undefined}
            value where $n < j \le m$.
        \end{itemize}
        and the array $\alpha$.
    \end{itemize}
  \item The value of $\textsf{get}\ e$ is the $n$-th value of the array $\alpha$
    where $n$ is the value of $e$ and the array indices start from $0$.
\end{itemize}

For example,
$\eappfo{(\efun{\cx\cy}{\cy})}{4}$
evaluates to \textsf{undefined}, and
$\eappfo{(\efun{\cx}{\textsf{get}\ 0})}{5}$
evaluate to $5$.

\begin{enumerate}
  \item Write the operational semantics of the form
    \fbox{$\eval{\sigma,\alpha}{e}{v}$}.
  \item Write the evaluation derivation of the following expression:
  \[
    \inferrule
    {\hspace*{0.8\textwidth}}
    {\eval{\emptyset,\emptyset}{\eappfo{(\efun{\cx\ \cy}{\textsf{get}\ x})}{2,19,141}}{~~~~~}}
  \]
\end{enumerate}

\item The following quote describes the JavaScript sequencing semantics:
\sidenote{\url{https://tc39.es/ecma262/\#sec-block-runtime-semantics-evaluation}}

\begin{quote}
The value of a \embox{StatementList} is the value of the last
value-producing item in the \embox{StatementList}.  For example, the
following calls to the \verb!eval! function all return the value 1:
\begin{verbatim}
eval("1;;;;;")
eval("1;()")
eval("1;var a;")
\end{verbatim}
\end{quote}
Consider the following language:
\[
  \begin{array}{lrrl}
    \text{Expression} & e & ::= & \textsf{()}\ |\ x\ |\ \efun{x}{e}\ |\ \eapp{e}{e}\ |\
    e;\cdots;e \\
    \text{Value} & v & ::= & \textsf{()}\ |\ \clov{x}{e}{\sigma} \\
  \end{array}
\]
The value of the sequence expression $e_1;\cdots;e_n$
is the value of the last expression whose value is not $\textsf{()}$.
If the values of all the expressions $e_1,\cdots,e_n$ are $\textsf{()}$,
the value of the sequence expression is $\textsf{()}$.
Write the operational semantics of each expression of the form
\fbox{$\evald{e}{v}$}.

\item Consider the following language:
\[
\begin{array}{rll}
e ::= & a& \text{atomic expression}\\
\mid& e\ a& \text{function application}\\
\mid& \textsf{fn}\ m& \text{function expression}\\
a ::= & n & \text{number}\\
\mid&x & \text{identifier}\\
m ::= & p\ \leadsto\ e & \text{pattern matching}\\
\mid& p\ \leadsto\ e\ \verb!|!\ m & \text{pattern matching sequence}\\
p ::= & \verb!_! & \text{wildcard pattern}\\
\mid&n& \text{number pattern}\\
\mid&x& \text{identifier pattern}
\end{array}
\]
where a value of the language $v$ is either a number $n$ or a closure $\langle m, \sigma\rangle$,
a result of evaluation $r$ is either a value $v$ or a failure in pattern matching $\uparrow$,
which is different from run-time errors,
and an environment $\sigma$ maps identifiers to their values.

The operational semantics rules for expressions and atomic expressions are as follows:

\fbox{$\sigma\vdash e \Rightarrow r$}
\[
\begin{array}{c}
  \inferrule
  {\sigma \vdash a \hookrightarrow v}
  {\sigma \vdash a \Rightarrow v}
  \qquad
  \inferrule
  {\sigma \vdash e \Rightarrow \uparrow}
  {\sigma \vdash e\ a \Rightarrow \uparrow}
  \qquad
  \sigma \vdash \textsf{fn}\ m \Rightarrow \langle m, \sigma \rangle
  \\[1.5em]
  \inferrule
  {
    \sigma \vdash e \Rightarrow \langle m, \sigma' \rangle \\
    \sigma \vdash a \Rightarrow v \\
    (\sigma', v) \vdash m \Rightarrow v'
  }
  {\sigma \vdash e\ a \Rightarrow v'}
  \\[1.5em]
  \inferrule
  {
    \sigma \vdash e \Rightarrow \langle m, \sigma' \rangle \\
    \sigma \vdash a \Rightarrow v \\
    (\sigma', v) \vdash m \Rightarrow \uparrow
  }
  {\sigma \vdash e\ a \Rightarrow \uparrow}
\end{array}
\]

\fbox{$\sigma\vdash a \hookrightarrow v$}
\[
\begin{array}{c}
  \sigma \vdash n \hookrightarrow n
  \qquad\qquad
  \inferrule
  {x \in \dom{\sigma}}
  {\sigma \vdash x \hookrightarrow \sigma(x)}
\end{array}
\]

The semantics of pattern matching $m$ and pattern $p$ are as follows:
\begin{itemize}
\item Evaluation of $p\ \leadsto\ e$ under $(\sigma, v)$ has two possibilities.
First, when evaluation of $p$ results in a new environment $\sigma'$,
the result of this pattern matching is the result of evaluation of $e$ under $\sigma+\sigma'$,
where $\sigma+\sigma'$ is a disjoint union of $\sigma$ and $\sigma'$.
Second, when evaluation of $p$ produces $\uparrow$,
the evaluation of this pattern matching produces $\uparrow$ as well.
\item Evaluation of ``$p\ \leadsto\ e\ \verb!|!\ m$'' under $(\sigma, v)$
also has two possibilities.
First, when evaluation of $p\ \leadsto\ e$ succeeds with a value $v'$,
the value of this pattern matching sequence is $v'$.
Second, when evaluation of $p\ \leadsto\ e$ fails,
the result of evaluation of this pattern matching sequence is
the result of evaluation of $m$.
\item Evaluation of the wildcard pattern \verb!_! under $(\sigma, v)$
produces the empty environment.
\item Evaluation of the number pattern $n$ under $(\sigma, v)$ has two possibilities.
If $v=n$,
it produces the empty environment.  Otherwise, it produces $\uparrow$.
\item Evaluation of the identifier pattern $x$ under $(\sigma, v)$
produces a singleton environment $\{x \mapsto v\}$
if $x$ is not in the domain of $\sigma$.
\end{itemize}
Write the operational semantics for $m$ and $p$
of the forms \fbox{$(\sigma, v)\vdash m \Rightarrow r$} and
\fbox{$(\sigma, v)\vdash p \Rightarrow \sigma/\uparrow$}, respectively,
where \fbox{$(\sigma, v)\vdash p \Rightarrow \sigma/\uparrow$} denotes
\fbox{$(\sigma, v)\vdash p \Rightarrow \sigma$} or
\fbox{$(\sigma, v)\vdash p \Rightarrow \uparrow$}.
Remember that the operational semantics do not specify run-time errors.
\end{enumerate}

\setchapterpreamble[u]{\margintoc}
\chapter{Scala Collections}
\labch{scala-collections}

It is the third article about functional programming. The last article dealt with
first-class functions, anonymous functions, and option types. It generalized
functions treating lists with first-class functions. This article introduces the
\verb!List! and \verb!Option! types defined in the Scala standard library. It
focuses on syntactic differences between the custom types defined in the previous
articles and the types came from the library rather than emphasizing new
concepts. Besides, it deals with \verb!for!, which is extremely expressive.

\section{List}

The library defines the \verb!List! class. The \verb!scala.collection.immutable!
package includes the class. Programmers can use lists without any import
statements because \term{aliases} of types and values required to use lists exist
in the \verb!scala!
package\sidenote{\url{https://www.scala-lang.org/api/current/scala/index.html}}, whom compilers
always automatically import.

Like the custom lists, the empty list is \verb!Nil!.

On the other hand, \verb!::! replaces \verb!Cons!. Expressions like
\verb!::(0, Nil)! construct lists.

The \verb!::! method of the list class provides more intuitive representations of
lists than the previous code. An expression of form \verb![expression].[method name]([expression], …)!
invokes the method referred by the name. \verb!Nil.::(0)!
creates a list with single element \verb!0!. Scala allows using methods as infix
operators. Operators whose names end with the colon are right-associative. Their
owners are the operands at the right sides. Therefore, \verb!0 :: Nil! and
\verb!Nil.::(0)! are the same expression. Both denote the same value as
\verb!::(0, Nil)! does. Similarly, \verb!0 :: 1 :: Nil! and
\verb!Nil.::(0).::(1)! are identical. They and \verb!::(0, ::(1, Nil))! result in
the same list, which contains \verb!0! and \verb!1!. \verb!0 :: 1 :: Nil! is the
most intuitive representation among them.

\verb!List! also can create lists. \verb!List([expression], …)! evaluates every
expression between the round brackets and constructs a list containing the
results. \verb!List(0, 1)! and \verb!0 :: 1 :: Nil! produce the same value. Both
representations are popular, but the \verb!List(…)! form fits enumerating all
elements well while expressions like \verb!0 :: 1 :: l! are typical when
prepending elements in front of existing lists.

\begin{verbatim}
// custom lists
Cons(0, Cons(1, Cons(2, Nil)))

// standard library lists
::(0, ::(1, ::(2, Nil)))
Nil.::(2).::(1).::(0)
0 :: 1 :: 2 :: Nil
List(0, 1, 2)
\end{verbatim}

\verb!::! replaces \verb!Cons! during pattern matching as well. Scala allows
using class names as infix operators in pattern expressions. \verb!case h :: t =>!
is valid.

\begin{verbatim}
// custom lists
Cons(0, Cons(1, Cons(2, Nil))) match {
  case Nil => "foo"
  case Cons(h, t) => "bar"
}

// standard library lists
List(0, 1, 2) match {
  case Nil => "foo"
  case h :: t => "bar"
}
\end{verbatim}

The \verb!List! type in the library is \term{polymorphic}. The custom lists
contain only integral values. However, the lists of the library can be lists
containing values of any types. The types of lists are not \verb!List!'s.
Instead, like \verb!List[Int]!, the types represent the types of the elements of
the lists. \verb!List[Int]! is the type of a list containing only integers.

\begin{verbatim}
// custom lists
Cons(0, Cons(1, Cons(2, Nil))): List

// standard library lists
List(): List[Nothing]
List(true): List[Boolean]
List(0, 1, 2): List[Int]
\end{verbatim}

\verb!Nothing! is the \term{bottom} type, which is a \term{subtype} of every
type. Values never belong to the \verb!Nothing! type, but the empty list has type
\verb!List[Nothing]! since it does not contain any elements. Both
\term{parametric polymorphism} and \term{subtype polymorphism} are essential
concepts, and thus later articles deal with them. For those who are unfamiliar
with the notions, understanding that a list containing values of type \verb!A!
has type \verb!List[A]! is sufficient to use lists.

Lists have the \verb!map!, \verb!filter!, \verb!foldRight!, and \verb!foldLeft!
methods. \verb!foldRight! and \verb!foldLeft! are curried. \term{Currying}
transforms functions with multiple parameters into sequences of functions with a
single parameter. The \verb!:\! and \verb!/:! methods respectively give the same
results as \verb!foldRight! and \verb!foldLeft! give.

\begin{verbatim}
// custom lists
def inc1(l: List): List = list_map(l, _ + 1)
def odd(l: List): List = list_filter(l, _ % 2 != 0)
def sum(l: List): Int = list_foldRight(l, 0, _ + _)
def product(l: List): Int = list_foldLeft(l, 1, _ * _)

// standard library lists
def inc1(l: List[Int]): List[Int] = l.map(_ + 1)
def odd(l: List[Int]): List[Int] = l.filter(_ % 2 != 0)
def sum(l: List[Int]): Int = l.foldRight(0)(_ + _)
def sum(l: List[Int]): Int = (l :\ 0)(_ + _)
def product(l: List[Int]): Int = l.foldLeft(1)(_ * _)
def product(l: List[Int]): Int = (1 /: l)(_ * _)
\end{verbatim}

The methods can take functions of various types as arguments.

\begin{verbatim}
List(1, 2, 3).map(_ == 1)  // List(true, false, false)
List("", "a", "ab").filter(_.length == 1)  // List("a")

def addBack(l: List[Int], n: Int): List[Int] =
  l.foldRight(List(n))(_ :: _)
def addBack(l: List[Int], n: Int): List[Int] =
  (l :\ List(n))(_ :: _)

def reverse(l: List[Int]): List[Int] =
  l.foldLeft(Nil: List[Int])((t, h) => h :: t)
def reverse(l: List[Int]): List[Int] =
  ((Nil: List[Int]) /: l)((t, h) => h :: t)
\end{verbatim}

The \verb!list_get! and \verb!list_getOption! methods find an element at an
arbitrary index. The library provides the same functionality. Lists are per se
functions. They have single integral parameter \verb!n! and return the \verb!n!th
elements of them. Like \verb!list_get!, exceptions occur when \verb!n! is wrong.
The \verb!lift! method converts unsafe functions into safe functions returning
\verb!None! instead of raising exceptions. The lifted functions do the thing
\verb!list_getOption! does.

\begin{verbatim}
// custom lists
list_get(Cons(0, Cons(1, Cons(2, Nil))), 0)
list_getOption(Cons(0, Cons(1, Cons(2, Nil))), 0)

// standard library lists
List(0, 1, 2)(0)
List(0, 1, 2).lift(0)
\end{verbatim}

Other methods substitute the roles of other previously defined functions.

\begin{verbatim}
// custom lists
addBack(Cons(0, Cons(1, Nil)), 2)

length(Cons(0, Cons(1, Cons(2, Nil))))
reverse(Cons(0, Cons(1, Cons(2, Nil))))

// standard library lists
List(0, 1) :+ 2
List(0, 1) ++ List(2)
List(0, 1, 2).length
List(0, 1, 2).reverse
\end{verbatim}

The web site of the
library\sidenote{\url{https://www.scala-lang.org/api/current/scala/collection/immutable/List.html}}
gives the full list of the methods of the \verb!List! class. This article
introduces only one additional important method. The \verb!flatMap! method is
similar to the \verb!map! method, but its parameter is a function returning a
collection. Here, the term 'collection' is broad in its meaning. Even options are
collections. (Precisely, the function returns a value whose type is
\verb!IterableOnce[T]!.) After applying a given function to the elements of a
given list, while \verb!map! puts the results in a list, \verb!flatMap! puts the
elements of the results in a list. Its name implies that it *\term{flat}*tens the
return value of \verb!map!.

\begin{verbatim}
List(0, 1, 2).flatMap(List(_))  // List(0, 1, 2)
List(0, 1, 2).flatMap(0 to _)  // List(0, 0, 1, 0, 1, 2)

def div100(n: Int): Option[Int] =
  if (n == 0) None else Some(100 / n)
List(0, 1, 2).flatMap(div100)  // List(100, 50)
\end{verbatim}

\verb!list_foldLeft! is tail-recursive, but \verb!list_map!, \verb!list_filter!,
and \verb!list_foldRight! are not. Stacks overflow when lists given as arguments
are long. Fortunately, the methods of the library are free from such a problem.
The library defines lists to become mutable only when being accessed inside the
library. The methods use \verb!while! loops, require time complexity of \(O(n)\),
and do not make stacks overflow. Lists are immutable for library users, and
therefore using lists maintains immutability and does not harm the functional
paradigm.

\section{Option}

The library defines the \verb!Option! class as well. The \verb!scala! package
includes the class so that import statements are unnecessary.

The names, \verb!None! and \verb!Some!, are identical to the custom options.

\begin{verbatim}
// custom options
None
Some(0)

// standard library options
None
Some(0)
\end{verbatim}

Therefore, pattern matching is the same.

\begin{verbatim}
// custom options
Some(0) match {
  case None => "foo"
  case Some(n) => "bar"
}

// standard library options
Some(0) match {
  case None => "foo"
  case Some(n) => "bar"
}
\end{verbatim}

Like lists, options are polymorphic. \verb!Some! wraps not only integers but also
any values. Values of type \verb!Option[T]! contain values of type \verb!T! or
nothing.

\begin{verbatim}
// custom options
Some(0): Option

// standard library options
None: Option[Nothing]
Some(true): Option[Boolean]
Some(0): Option[Int]
\end{verbatim}

The class defines the \verb!map! and \verb!flatMap! methods. The two methods can
take functions of various types as arguments.

\begin{verbatim}
// custom options
option_map(Some(0), n => n * n)
option_flatMap(Some(0), div100)

// standard library options
Some(0).map(n => n * n)
Some(0).flatMap(div100)
\end{verbatim}

The web site\sidenote{\url{https://www.scala-lang.org/api/current/scala/Option.html}} shows
the full list of the methods of the \verb!Option! class.

\section{for}

Scala has \verb!for! statements. In fact, \verb!for! expressions, which denote
values, exist in Scala. \verb!for! expressions are highly expressive. Unlike
\verb!while! loops, which work with mutable variables or objects, \verb!for! of
Scala helps programmers to write code in a functional and readable way.

Firstly, let us see \verb!for! statements, which do not produce values. The
syntax is similar to the syntax of Java ('foreach' loops) or Python but different
from the syntax of C.

\begin{verbatim}
for (n <- List(0, 1, 2))
  println(n * n)
\end{verbatim}

The code prints \verb!0!, \verb!1!, and \verb!4!. \verb!n! refers to \verb!0! at
the first iteration, \verb!1! at the second iteration, and \verb!2! at the third
iteration.

\verb!for! expressions use the \verb!yield! keyword.

\begin{verbatim}
for (n <- List(0, 1, 2))
  yield n * n
\end{verbatim}

It results in \verb!List(0, 1, 4)!. The result of \verb!for (…) yield [expressions]!
is a collection containing the result of evaluating the expression
at each iteration. \verb!for! expressions can appear at any places expecting
expressions.

\begin{verbatim}
val x = for (n <- List(0, 1, 2)) yield n * n
\end{verbatim}

In Scala, \verb!for! is just syntactic sugar. Instead of giving specific
semantics to \verb!for!, syntactic rules transform code using \verb!for! into the
code using methods of collections and anonymous functions. The above code becomes
code using \verb!foreach! and \verb!map!.

\begin{verbatim}
List(0, 1, 2).foreach(n => println(n * n))
List(0, 1, 2).map(n => n * n)
\end{verbatim}

For this reason, \verb!for! statements and expressions are powerful. Any
user-defined types can appear in \verb!for! statements or expressions if the
types define methods like \verb!foreach! and \verb!map!.

Programs use \verb!for! expressions for other purposes than accessing each
element in a collection once. \verb!;! allows nesting iterations. A nested
iteration yields values and stores them in a single collection.

\begin{verbatim}
for (n <- List(0, 1, 2);
     m <- 0 to n)
  yield m * m
\end{verbatim}

The result is \verb!List(0, 0, 1, 0, 1, 4)!. Nesting \verb!for! expressions does
not produce the same result.

\begin{verbatim}
for (n <- List(0, 1, 2)) yield
  for (m <- 0 to n)
    yield m * m
// List(Vector(0), Vector(0, 1), Vector(0, 1, 4))
\end{verbatim}

The code using a semicolon is more concise and more readable than the code using
the nested expressions.

\verb!if! prevents accessing elements not satisfying a given condition.

\begin{verbatim}
for (n <- List(0, 1, 2) if n % 2 == 0;
     m <- 0 to n)
  yield m * m
\end{verbatim}

It results in \verb!List(0, 0, 1, 4)!. \verb!1! is skipped.

The code using \verb!;! and \verb!if! changes into the code using \verb!flatMap!
and \verb!filter!.

\begin{verbatim}
List(0, 1, 2)
  .filter(n => n % 2 == 0)
  .flatMap(n => 0 to n)
  .map(m => m * m)
\end{verbatim}

Since options have \verb!foreach!, \verb!map!, \verb!filter!, and \verb!flatMap!,
options can appear in \verb!for! expressions. The remaining of the article shows
how \verb!for! expressions make code easy to understand.

\begin{verbatim}
def f(x: Int): Option[Int] = ...
def g(x: Int, y: Int, z: Int, w: Int): Int = ...
\end{verbatim}

Function \verb!f! may fail and thus returns an option. Function \verb!g! has four
integral parameters. Consider a program that calculates \verb!f(0)!, \verb!f(1)!,
\verb!f(2)!, and \verb!f(3)! and uses the results as arguments of \verb!g! only
if every function call has succeeded. Pattern matching allows simple code.

\begin{verbatim}
(f(0), f(1), f(2), f(3)) match {
  case (Some(x), Some(y), Some(z), Some(w)) =>
    Some(g(x, y, z, w))
  case _ =>
    None
}
\end{verbatim}

It works well because pattern matching can be done on tuples as well. However,
assume that \verb!f! takes a long time to produce the result. Since the program
calls \verb!f! four times regardless of whether the previous call has succeeded
or not, the program is inefficient. It is desirable to call next \verb!f! only if
the last call has returned a value belongs to \verb!Some!. Let us use sequential
pattern matching.

\begin{verbatim}
f(0) match {
  case Some(x) => f(1) match {
    case Some(y) => f(2) match {
      case Some(z) => f(3) match {
        case Some(w) =>
          Some(g(x, y, z, w))
        case None => None }
      case None => None }
    case None => None }
  case None => None }
\end{verbatim}

It is efficient but verbose and complicated. \verb!flatMap! and \verb!map!
improve the code.

\begin{verbatim}
f(0).flatMap(x =>
f(1).flatMap(y =>
f(2).flatMap(z =>
f(3).map(w =>
  g(x, y, z, w)
))))
\end{verbatim}

A \verb!for! expression makes code clear and concise.

\begin{verbatim}
for (x <- f(0);
     y <- f(1);
     z <- f(2);
     w <- f(3))
  yield g(x, y, z, w)
\end{verbatim}

\setchapterpreamble[u]{\margintoc}
\chapter{Pattern Matching}
\labch{pattern-matching}

This section explains pattern matching of Scala.
Pattern matching is one of the key features of functional programming.
It helps programmers handle complex, but structured data.
We have already used a simple form of pattern matching for lists.
This section discusses the benefits
of pattern matching and various patterns available in Scala.
In addition, it will introduce the option type, which is widely-used in
functional programming.

\section{Algebraic Data Types}

It is common to include values of various shapes in a single type.

A natural number is

\begin{itemize}
\item zero or
\item the successor of a natural number.
\end{itemize}

A list is

\begin{itemize}
\item the empty list or
\item a pair of an element and a list.
\end{itemize}

A binary tree is

\begin{itemize}
\item the empty tree or
\item a tree containing a root element and two child trees.
\end{itemize}

An arithmetic expression is

\begin{itemize}
\item a number,
\item the sum of two arithmetic expressions, or
\item the difference of two arithmetic expressions.
\end{itemize}

An expression of the lambda calculus is

\begin{itemize}
\item a variable,
\item a function, which is a pair of a variable and an expression, or
\item a function application, which is a pair of two expressions.
\end{itemize}

As the examples show, in computer science, a single type often includes values of
various shapes. \textit{Algebraic data types}\index{algebraic data type}
(\acrshort{adtLabel}) express such types. An ADT
is the sum type of product types. That is why it is called ``algebraic.''
A \textit{product type}\index{product type}
is a type whose every element is an enumeration of values of types in the
same specific order. Tuple types are typical product types. A \textit{sum
type}\index{sum type}, whose
another name is a \textit{tagged union type}\index{tagged union type},
has values of multiple types as its values. Unlike a union type,
each component of a sum type has a
tag to be distinguished from other components.
In an ADT, one form of values that can be distinguished from the other forms is
called a \textit{variant}\index{variant}.

For example, an arithmetic expression, which has three variants, is

\begin{itemize}
\item a number,
\item the sum of two arithmetic expressions, or
\item the difference of two arithmetic expressions.
\end{itemize}

Therefore, we can define the \code{AE} type, which is the type of an arithmetic
expression, as the sum type of

\begin{itemize}
\item the \code{Int} type tagged with \code{Num},
\item the \code{AE * AE} type tagged with \code{Add}, and
\item the \code{AE * AE} type tagged with \code{Sub},
\end{itemize}

where \code{AE * AE} denotes the product type of \code{AE} and \code{AE}.

ADTs are common in functional languages. Most functional
languages allow users to define new ADTs. The following OCaml
code defines arithmetic expressions:

\begin{verbatim}
type ae =
| Num of int
| Add of ae * ae
| Sub of ae * ae
\end{verbatim}

Scala does not provide a direct way to define ADTs. Instead, Scala provides
traits and classes, which are more general mechanisms to define new types,
and programmers can express ADTs with traits and classes.

A new type can be defined as a trait.
The syntax of a trait definition is as follows:

\begin{verbatim}
trait [name]
\end{verbatim}

It defines a type whose name is \code{[name]}.
The following code defines the \code{AE} type,
which is the type of arithmetic expressions:

\begin{verbatim}
sealed trait AE
\end{verbatim}

The \code{sealed} modifier prevents \code{AE} being extended outside the file
that defines \code{AE}. We will get back to this point when we discuss the
exhaustivity checking of pattern matching.

Once a type is defined as a trait, the type can be used just like any other
types. For example, we can define an identity function for arithmetic
expressions.

\begin{verbatim}
def identity(ae: AE): AE = ae
\end{verbatim}

However, traits do not have ability to construct new values. It means that there
is no way to create a value of the type \code{AE} yet. We need to define the
variants of \code{AE} as case classes by extending \code{AE}.

\begin{verbatim}
case class Num(value: Int) extends AE
case class Add(left: AE, right: AE) extends AE
case class Sub(left: AE, right: AE) extends AE
\end{verbatim}

As you have seen in \refch{introduction-to-scala}, we can easily create values of case classes.

\begin{verbatim}
val n = Num(10)
val m = Num(5)
val e1 = Add(n, m)
val e2 = Sub(e1, Num(3))
\end{verbatim}

Like traits, case classes also define types. The name of each class is the name
of the defined type. Every instance of a class belongs to the type corresponding
to the class.

\begin{verbatim}
val n: Num = Num(10)
val m: Num = Num(5)
val e1: Add = Add(n, m)
val e2: Sub = Sub(e1, Num(3))
\end{verbatim}

In addition, because of the \code{extends} keyword, \code{Num}, \code{Add}, and
\code{Sub} are subtypes of \code{AE}. It means that any value of the types
\code{Num}, \code{Add}, or \code{Sub} is also a value of the type \code{AE}.

\begin{verbatim}
val n: AE = Num(10)
val m: AE = Num(5)
val e1: AE = Add(n, m)
val e2: AE = Sub(e1, Num(3))
\end{verbatim}

We know that we can access the fields of objects with their names.

\begin{verbatim}
val n: Num = Num(10)
n.value
\end{verbatim}

However, we cannot access the fields of an object when its type becomes \code{AE}.

\begin{verbatim}
val m: AE = Num(10)
m.value
\end{verbatim}
\vspace{-1em}
\begin{mdframed}[hidealllines=true,backgroundcolor=red!10,innerleftmargin=3pt,innerrightmargin=3pt,leftmargin=-3pt,rightmargin=-3pt]
\begin{verbatim}
  ^
error: value value is not a member of AE
\end{verbatim}
\vspace{-2em}
\begin{flushright}
\scriptsize\textsf{Compile-time error}
\end{flushright}
\end{mdframed}

The reason is that \code{m} can be \code{Add} or \code{Sub}, which do not have
the field \code{value}, as \code{AE} consists of not only \code{Num} but also
\code{Add} and \code{Sub}. The compiler thinks that \code{m} may not have the
field \code{value} and considers \code{m.value} as an unsafe expression, which
should be rejected.

The best way to use ADTs is pattern matching. The following function evaluates a
given arithmetic expression and returns the number denoted by the arithmetic
expression.

\begin{verbatim}
def eval(e: AE): Int = e match {
  case Num(n) => n
  case Add(l, r) => eval(l) + eval(r)
  case Sub(l, r) => eval(l) - eval(r)
}

assert(eval(Sub(Add(Num(3), Num(7)), Num(5))) == 5)
\end{verbatim}

When \code{e} is \code{Num(n)}, \code{eval} simply returns \code{n}.
When \code{e} is \code{Add(l, r)}, \code{eval} respectively evaluates \code{l}
and \code{r}, which are arithmetic expressions. \code{eval} returns the sum of
the results of \code{l} and \code{r}.
The \code{Sub(l, r)} case is similar. \code{eval} returns the difference of
the results of \code{l} and \code{r}.

The list type is another good example of an ADT. The Scala standard library defines
lists similar to the following code:

\begin{verbatim}
sealed trait List[+A]
case object Nil extends List[Nothing]
case class ::[A](head: A, tail: List[A]) extends List[A]
\end{verbatim}

This code omits some details but clearly shows the high-level idea to define
lists.\sidenote{We will not see what \code{[+A]} and \code{Nothing} are here.
You can understand the overall ADT structure without knowing those concepts.}
A list is either the empty
list or a nonempty list, which is a pair of its head and tail. \code{Nil} is
defined as a case object, not a case class, since there is only one empty list.
Every empty list is identical. We use a case object to express this idea.
\code{Nil} is created only once during entire execution, and every \code{Nil} is
identitcal. The name \code{::} looks a bit weird, but it is for
readability of pattern matching. Scala allows writing class names as infix
operators in patterns. It means that both \code{case ::(h, t) =>} and \code{case
h :: t =>} are allowed. Due to the class name \code{::}, we can write \code{case
h :: t =>} in pattern matching.

\section{Advantages}

\subsection{Conciseness}

Without pattern matching, handling ADTs becomes a complicated job. We need to
use dynamic type tests to distinguish variants and type casting to access the
fields of values.

Below is \code{eval} without pattern matching:

\begin{verbatim}
def eval(e: AE): Int =
  if (e.isInstanceOf[Num])
    e.asInstanceOf[Num].value
  else if (e.isInstanceOf[Add]) {
    val e0 = e.asInstanceOf[Add]
    eval(e0.left) + eval(e0.right)
  } else {
    val e0 = e.asInstanceOf[Sub]
    eval(e0.left) - eval(e0.right)
  }
\end{verbatim}

\code{e.isInstanceOf[Num]} tests whether \code{e} is an instance of class
\code{Num}. If it is true, \code{eval} should return the value of the field
\code{value} of \code{e}. However, \code{value} cannot be directly accessed
since \code{e}'s type is \code{AE}. Because \code{e.isInstanceOf[Num]} is true,
we are sure that \code{e}'s actual type is \code{Num}. In this case, we can
inform this knowledge to the compiler with type casting. \code{e.asInstanceOf[Num]}
does not change the value denoted by \code{e} but lets the compiler know that
the programmer guarantees the type of \code{e} to be \code{Num}. Therefore, the
compiler considers \code{e.asInstanceOf[Num]} to belong \code{Num} and allows
accessing the field \code{value}. These type tests and casting processes should
be done for the other variants, \code{Add} and \code{Sub}, too.

The code is long and complicated despite its simple functionality. Dynamic type
tests and explicit type casting occupy most of the code, while real
computation requires short code. Besides, such code is error-prone.
For example, programmers may write code like below by mistake:

\begin{verbatim}
else if (e.isInstanceOf[Add]) {
  val e0 = e.asInstanceOf[Sub]
  eval(e0.left) + eval(e0.right)
}
\end{verbatim}

While the condition checks whether \code{e} is an instance of \code{Add},
\code{e} becomes casted to \code{Sub}. Such code will trigger an error at run
time and terminate the execution abnormally.
It is easy to check whether \code{eval} is correct because it is short.
However, complex types and computation will increase the possibility of
mistakes.

Pattern matching gives us a much better solution. Pattern matching hides type
tests and casting and makes code concise. At the same time, pattern matching
removes the possibility of mistakes.

\subsection{Exhaustivity Checking}

Pattern matching checks the exhaustivity of patterns. At run time, a match error
occurs when a given value matches none of given patterns.

\begin{verbatim}
def eval(e: AE): Int = e match {
  case Add(l, r) => eval(l) + eval(r)
  case Sub(l, r) => eval(l) - eval(r)
}
\end{verbatim}

The function lacks the \code{Num} pattern.

\begin{verbatim}
eval(Num(3))
\end{verbatim}
\vspace{-1em}
\begin{mdframed}[hidealllines=true,backgroundcolor=red!10,innerleftmargin=3pt,innerrightmargin=3pt,leftmargin=-3pt,rightmargin=-3pt]
\begin{verbatim}
scala.MatchError: Num(3) (of class Num)
\end{verbatim}
\vspace{-2em}
\begin{flushright}
\scriptsize\textsf{Run-time error}
\end{flushright}
\end{mdframed}

An argument of type \code{Num} results in a match error at run time.

Fortunately, we can easily avoid such mistakes.
The Scala compiler checks whether patterns are exhaustive and warns if they are not.

\begin{verbatim}
def eval(e: AE): Int = e match {
  case Add(l, r) => eval(l) + eval(r)
  case Sub(l, r) => eval(l) - eval(r)
}
\end{verbatim}
\vspace{-1em}
\begin{mdframed}[hidealllines=true,backgroundcolor=yellow!10,innerleftmargin=3pt,innerrightmargin=3pt,leftmargin=-3pt,rightmargin=-3pt]
\begin{verbatim}
                       ^
warning: match may not be exhaustive.
It would fail on the following input: Num(_)
\end{verbatim}
\vspace{-2em}
\begin{flushright}
\scriptsize\textsf{Compile-time warning}
\end{flushright}
\end{mdframed}

The compiler warns programmers about that the patterns are not exhaustive. Moreover, it precisely
informs which patterns are missing to help debugging.
Exhaustivity checking is beneficial for complex programs. It helps programmers make
error-free programs and thus is a crucial strength of pattern matching.

For exhaustivity checking, the \code{sealed} modifier of traits is necessary.
Without \code{sealed}, a trait can be extended outside the file that defines it.
The unit of compilation is a single file, so it is impossible to find
all the variants by scanning a single file when a trait is not sealed.
Exhaustivity checking during pattern matching will be impossible.
The \code{sealed} keyword resolves the problem. Since sealed traits cannot be
extended further, it is enough to check only the file that defines a sealed trait to find
every variant of the trait. It is why we use sealed traits to define ADTs.

\subsection{Reachability Checking}

Like \code{switch-case}, pattern
matching compares a value to patterns sequentially from top to bottom and selects
the first matching pattern. If there are duplicated patterns, the latter will
not be reachable.
The compiler warns programmers when they find unreachable patterns to prevent such code.

\begin{verbatim}
def eval(e: AE): Int = e match {
  case Num(n) => 0
  case Add(l, r) => eval(l) + eval(r)
  case Num(n) => n
  case Sub(l, r) => eval(l) - eval(r)
}
\end{verbatim}
\vspace{-1em}
\begin{mdframed}[hidealllines=true,backgroundcolor=yellow!10,innerleftmargin=3pt,innerrightmargin=3pt,leftmargin=-3pt,rightmargin=-3pt]
\begin{verbatim}
  case Num(n) => n
                 ^
warning: unreachable code
\end{verbatim}
\vspace{-2em}
\begin{flushright}
\scriptsize\textsf{Compile-time warning}
\end{flushright}
\end{mdframed}

When code is simple and short, it is easy to check whether there are unreachable
patterns. However, in complex code, programmers often insert unreachable
patterns by mistake and make critical bugs. Reachability checking of the
compiler is an important feature to prevent such bugs.

\section{Patterns in Scala}

\subsection{Constant and Wildcard Patterns}

\code{switch-case} statements divide a given value into multiple cases in
imperative languages. Pattern matching is a general form of \code{switch-case}.
The following code is an example using a \code{switch-case} statement in Java:

\begin{verbatim}
String grade(int score) {
  switch (score / 10) {
    case 10: return "A";
    case 9: return "A";
    case 8: return "B";
    case 7: return "C";
    case 6: return "D";
    default: return "F";
  }
}
\end{verbatim}

Constant and wildcard patterns exist in Scala. Constant patterns are
literals like integers and strings. A constant pattern matches a given
value if a value denoted by the pattern equals the given value. The underscore
denotes the wildcard pattern, which matches every value, and is equivalent to
\code{default} of \code{switch-case}. The following function rewrites the
previous function with pattern matching:

\begin{verbatim}
def grade(score: Int): String =
  (score / 10) match {
    case 10 => "A"
    case 9 => "A"
    case 8 => "B"
    case 7 => "C"
    case 6 => "D"
    case _ => "F"
  }

assert(grade(85) == "B")
\end{verbatim}

\subsection{Or Patterns}

\code{switch-case} statements use the fall-through semantics; if \code{break}
does not exist, after executing code corresponding to a case, the flow of the
execution moves to code corresponding to the next case. Since the results of
cases \code{10} and \code{9} are identical, the function can use fall-through.

\begin{verbatim}
String grade(int score) {
  switch (score / 10) {
    case 10:
    case 9: return "A";
    case 8: return "B";
    case 7: return "C";
    case 6: return "D";
    default: return "F";
  }
}
\end{verbatim}

In contrast, pattern matching disallows fall-through. Instead, or patterns
give a way to write the same expression only once for multiple patterns. The
syntax of an or pattern is \code{[pattern] | [pattern] | …}, which is a sequence
of multiple patterns with vertical bars in between. \code{A | B} matches values
that match \code{A} or \code{B}.

\begin{verbatim}
def grade(score: Int): String =
  (score / 10) match {
    case 10 | 9 => "A"
    case 8 => "B"
    case 7 => "C"
    case 6 => "D"
    case _ => "F"
  }

assert(grade(100) == "A")
\end{verbatim}

\subsection{Nested Patterns}

Nested patterns are patterns containing patterns.
The \code{optimizeAdd} function
optimizes a given arithmetic expression by eliminating additions of zeros.

\begin{verbatim}
def optimizeAdd(e: AE): AE = e match {
  case Num(_) => e
  case Add(Num(0), r) => optimizeAdd(r)
  case Add(l, Num(0)) => optimizeAdd(l)
  case Add(l, r) => Add(optimizeAdd(l), optimizeAdd(r))
  case Sub(l, r) => Sub(optimizeAdd(l), optimizeAdd(r))
}
\end{verbatim}

Nested patterns help programmers treat values with complex structures easily.

\subsection{Patterns with Binders}

Assume that we have one more variant of \code{AE}:

\begin{verbatim}
case class Abs(e: AE) extends AE
\end{verbatim}

It denotes the absolute value of an operand.
Optimizing an arithmetic expression decorated by two
consecutive \code{Abs} operators results in the arithmetic expression with only
one \code{Abs} operator.

\begin{verbatim}
def optimizeAbs(e: AE): AE = e match {
  case Num(_) => e
  case Add(l, r) => Add(optimizeAbs(l), optimizeAbs(r))
  case Sub(l, r) => Sub(optimizeAbs(l), optimizeAbs(r))
  case Abs(Abs(e0)) => optimizeAbs(Abs(e0))
  case Abs(e0) => Abs(optimizeAbs(e0))
}
\end{verbatim}

A flaw of the implementation is that a value matching \code{Abs(e0)}
cannot be an argument of \code{optimizeAbs} directly, and constructing a new
\code{Abs} instance containing a value matching \code{e0} is essential.
The \code{@} symbol makes code efficient by binding a value matching to a pattern to a variable.
Pattern \code{[variable] @ [pattern]} makes the variable refer to a value
matching the pattern.

\begin{verbatim}
def optimizeAbs(e: AE): AE = e match {
  case Num(_) => e
  case Add(l, r) => Add(optimizeAbs(l), optimizeAbs(r))
  case Sub(l, r) => Sub(optimizeAbs(l), optimizeAbs(r))
  case Abs(e0 @ Abs(_)) => optimizeAbs(e0)
  case Abs(e0) => Abs(optimizeAbs(e0))
}
\end{verbatim}

\subsection{Type Patterns}

In \code{optimizeAbs},
the first \verb!Num(_)! pattern does no more than checking whether a value
belongs to type \code{Num}. A type pattern helps to rewrite the function. Type
patterns are in the form of \code{[name]: [type]}. If a value belongs to the
type, it matches the pattern, and the variable refers to the value. The wildcard
pattern can substitute the name if the variable is unnecessary.

\begin{verbatim}
def optimizeAbs(e: AE): AE = e match {
  case _: Num => e
  case Add(l, r) => Add(optimizeAbs(l), optimizeAbs(r))
  case Sub(l, r) => Sub(optimizeAbs(l), optimizeAbs(r))
  case Abs(e0 @ Abs(_)) => optimizeAbs(e0)
  case Abs(e0) => Abs(optimizeAbs(e0))
}
\end{verbatim}

Type patterns are useful for dynamic type checking. The following function takes
any value as an argument and check whether it is a string or
not.\sidenote{Every type is a subtype of \code{Any}, i.e. every value belongs to
\code{Any}.}

\begin{verbatim}
def show(x: Any): String = x match {
  case s: String => s + " is a string"
  case _ => "not a string"
}

assert(show("1") == "1 is a string")
assert(show(1) == "not a string")
\end{verbatim}

Note that type patterns cannot check type arguments of polymorphic types. Using
type patterns against polymorphic types is dangerous.

\begin{verbatim}
def show(x: Any): String = x match {
  case _: List[String] => "a list of strings"
  case _ => "not a list of strings"
}
\end{verbatim}
\vspace{-1em}
\begin{mdframed}[hidealllines=true,backgroundcolor=yellow!10,innerleftmargin=3pt,innerrightmargin=3pt,leftmargin=-3pt,rightmargin=-3pt]
\begin{verbatim}
          ^
warning: non-variable type argument String in type pattern
List[String] is unchecked since it is eliminated by erasure
\end{verbatim}
\vspace{-1.5em}
\begin{flushright}
\scriptsize\textsf{Compile-time warning}
\end{flushright}
\end{mdframed}

\begin{verbatim}
val l: List[Int] = List(1, 2, 3)
assert(show(l) == "a list of strings")  // weird result
\end{verbatim}

Although the type of the argument is \code{List[Int]}, it matches the first
pattern. As the warnings imply, the JVM uses type erasure
semantics, and type arguments are unavailable at run time.

\subsection{Tuple Patterns}

The syntax of a tuple pattern is \code{([pattern], …, [pattern])}.
It matches a tuple whose elements respectively match the internal patterns.

The following function uses tuple patterns to check
whether two lists are identical:

\begin{verbatim}
def equal(l0: List[Int], l1: List[Int]): Boolean =
  (l0, l1) match {
    case (h0 :: t0, h1 :: t1) =>
      h0 == h1 && equal(t0, t1)
    case (Nil, Nil) => true
    case _ => false
  }
\end{verbatim}

\subsection{Pattern Guards}

A binary search tree is

\begin{itemize}
\item the empty tree or
\item a tree containing an integral root element and two child trees.
\end{itemize}

\begin{verbatim}
sealed trait Tree
case object Empty extends Tree
case class Node(root: Int, left: Tree, right: Tree) extends Tree
\end{verbatim}

The function \code{add} takes a tree and an integer as arguments and returns a tree
obtained by adding the integer to the tree. If the integer is an element of the
given tree, the tree itself is the return value.

\begin{verbatim}
def add(t: Tree, n: Int): Tree =
  t match {
    case Empty => Node(n, Empty, Empty)
    case Node(m, t0, t1) =>
      if (n < m)
        Node(m, add(t0, n), t1)
      else if (n > m)
        Node(m, t0, add(t1, n))
      else
        t
  }
\end{verbatim}

An expression corresponding to the second pattern uses \code{if-else}. Pattern
guards allow adding constraints to patterns. A pattern in the form of
\code{[pattern] if [expression]} matches a value if the value matches the
pattern, and the expression results in \code{true}. The following version of \code{add}
uses pattern guards:

\begin{verbatim}
def add(t: Tree, n: Int): Tree =
  t match {
    case Empty => Node(n, Empty, Empty)
    case Node(m, t0, t1) if n < m =>
      Node(m, add(t0, n), t1)
    case Node(m, t0, t1) if n > m =>
      Node(m, t0, add(t1, n))
    case _ => t
  }
\end{verbatim}

Guarded patterns may be inexhaustive and need care.

\begin{verbatim}
def add(t: Tree, n: Int): Tree =
  t match {
    case Empty => Node(n, Empty, Empty)
    case Node(m, t0, t1) if n < m =>
      Node(m, add(t0, n), t1)
    case Node(m, t0, t1) if n > m =>
      Node(m, t0, add(t1, n))
  }
\end{verbatim}

The patterns in the above code is not exhaustive, but
the compiler does not warn programmers about the inexhaustivity.

\subsection{Patterns with Backticks}

The function \code{remove} takes a tree and an integer as arguments and returns a
tree obtained by removing the integer from the tree. If the integer is not an
element of the tree, the given tree itself is the return value. \code{removeMin}
is a helper function used by \code{remove}. It returns the pair of the smallest
element of a given tree and a tree obtained by removing the element from the
tree.

\begin{verbatim}
def removeMin(t: Tree): (Int, Tree) = {
  t match {
    case Node(n, Empty, t1) =>
      (n, t1)
    case Node(n, t0: Node, t1) =>
      val (min, t2) = removeMin(t0)
      (min, Node(n, t2, t1))
  }
}

def remove(t: Tree, n: Int): Tree = {
  t match {
    case Empty =>
      Empty
    case Node(m, t0, Empty) if n == m =>
      t0
    case Node(m, t0, t1: Node) if n == m =>
      val res = removeMin(t1)
      val min = res._1
      val t2 = res._2
      Node(min, t0, t2)
    case Node(m, t0, t1) if n < m =>
      Node(m, remove(t0, n), t1)
    case Node(m, t0, t1) if n > m =>
      Node(m, t0, remove(t1, n))
  }
}
\end{verbatim}

\verb!Node(`n`, t0, Empty)! can replace
\code{case Node(m, t0, Empty) if n == m}. The pattern \code{Node(n, t0, Empty)} defines
a new variable \code{n} and makes \code{n} refer to the
root element; it does not check whether the root element equals \code{n}.
However, backticks prohibit defining a new variable and allow to compare the root
element to \code{n} in the scope.

\begin{verbatim}
def remove(t: BinTree, n: Int): BinTree = {
  t match {
    case Empty =>
      Empty
    case Node(`n`, t0, Empty) =>
      t0
    case Node(`n`, t0, t1: Node) =>
      val res = removeMin(t1)
      val min = res._1
      val t2 = res._2
      Node(min, t0, t2)
    case Node(m, t0, t1) if n < m =>
      Node(m, remove(t0, n), t1)
    case Node(m, t0, t1) if n > m =>
      Node(m, t0, remove(t1, n))
  }
}
\end{verbatim}

\section{Applications of Pattern Matching}

\subsection{Variable Definitions}

It is possible to define variables with pattern matching.

\begin{verbatim}
val (n, m) = (1, 2)
assert(n == 1 && m == 2)

val (a, b, c) = ("a", "b", "c")
assert(a == "a" && b == "b" && c == "c")

val h :: t = List(1, 2, 3, 4)
assert(h == 1 && t == List(2, 3, 4))

val Add(l, r) = Add(Num(1), Num(2))
assert(l == Num(1) && r == Num(2))
\end{verbatim}

Pattern matching helps programmers declare variables concisely, but a match error occurs
when the pattern does not match the right-hand-side value. It is desirable to use
pattern matching only when there is a guarantee that the match succeeds. Since
a tuple pattern always matches a tuple value of the same length,
tuple patterns are widely used for variable definitions.

\subsection{Anonymous Functions}

The function \code{toSum} takes a list of pairs of two integers as arguments and
returns a list whose elements are the sums of the integers in the pairs.

\begin{verbatim}
def toSum(l: List[(Int, Int)]): List[Int] =
  l.map(p => p match {
    case (n, m) => n + m
  })

val l = List((0, 1), (2, 3), (3, 4))
assert(toSum(l) == List(1, 5, 7))
\end{verbatim}

The anonymous function directly uses parameter \code{p} as the target of the
pattern matching. Scala allows simplification of \verb!x => x match { … }! to
\verb!{ … }!. Therefore, we can use an enumeration of patterns as an anonymous
function.

\begin{verbatim}
def toSum(l: List[(Int, Int)]): List[Int] =
  l.map({ case (n, m) => n + m })
\end{verbatim}

\subsection{For Loops}

\code{toSum} can use a for expression instead of \code{map}.

\begin{verbatim}
def toSum(l: List[(Int, Int)]): List[Int] =
  for (p <- l)
    yield p match { case (n, m) => n + m }
\end{verbatim}

For expressions directly support pattern matching.

\begin{verbatim}
def toSum(l: List[(Int, Int)]): List[Int] =
  for ((n, m) <- l)
    yield n + m
\end{verbatim}

The code is readable and concise.

\section{Options}
\labsec{options}

The option type is a widely-used ADT. It represents a value whose existence is
optional. This section introduces the option type and explains the usage of
options.

Consider the function \code{get}, which takes a list and integer \code{n} as
arguments and returns the \code{n}th element of the list. It is problematic
when \code{n} is negative or exceeds the length of the list. Throwing exceptions
is a widely used solution in imperative languages. In Scala, \code{throw
[expression]} throws an exception. For convenience, we define the function
\code{error}, which throws an exception, like below and use it throughout the
book.

\begin{verbatim}
def error(msg: String) = throw new Exception(msg)

def get(l: List[Int], n: Int): Int =
  if (n < 0)
    error("index out of bounds")
  else l match {
    case Nil =>
      error("index out of bounds")
    case h :: t =>
      if (n == 0)
        h
      else
        get(t, n - 1)
  }
\end{verbatim}

Throwing an exception is a simple and effective solution. However, exceptions
have two problems. First, exceptions should be handled by exception handlers.

\begin{verbatim}
try {
  get(List(1, 2), 2)
} catch {
  case e: Exception =>
    // prints "index out of bounds"
    println(e.getMessage)
}
\end{verbatim}

If an exception is not handled properly, it will eventually cause a run-time
error and terminate the execution.

\begin{verbatim}
get(List(1, 2), 2)
\end{verbatim}
\vspace{-1em}
\begin{mdframed}[hidealllines=true,backgroundcolor=red!10,innerleftmargin=3pt,innerrightmargin=3pt,leftmargin=-3pt,rightmargin=-3pt]
\begin{verbatim}
java.lang.Exception: index out of bounds
\end{verbatim}
\vspace{-2em}
\begin{flushright}
\scriptsize\textsf{Run-time error}
\end{flushright}
\end{mdframed}

The Scala compiler does not check whether exceptions are handled properly.
It means that there will not be any compile-time error even if there is a
possibility of unhandled exceptions.

Another problem of exceptions is that exception handling is not local.
When an exception is thrown, the control flow suddenly jumps to the position of
the nearest exception handler. Non-local transition of the control flow usually
hinders programmers from understanding code.
Therefore, implementing \code{get} without exceptions is desirable.

The first attempt is to use \code{null}. \code{null} is a value that denotes that
it does not refer to any existing object. We can try to make \code{get} return
\code{null} when a given index is invalid.

\begin{verbatim}
def get(l: List[Int], n: Int): Int =
  if (n < 0)
    null
  else l match {
    case Nil => null
    case h :: t =>
      if (n == 0)
        h
      else
        get(t, n - 1)
  }
\end{verbatim}
\vspace{-1em}
\begin{mdframed}[hidealllines=true,backgroundcolor=red!10,innerleftmargin=3pt,innerrightmargin=3pt,leftmargin=-3pt,rightmargin=-3pt]
\begin{verbatim}
    null
    ^
error: an expression of type Null is ineligible
for implicit conversion

    case Nil => null
                ^
error: an expression of type Null is ineligible
for implicit conversion
\end{verbatim}
\vspace{-2em}
\begin{flushright}
\scriptsize\textsf{Run-time error}
\end{flushright}
\end{mdframed}

Unfortunately, \code{null} is not an element of \code{Int} in Scala.
The compiler rejects the code.
Even with the assumption that we can treat \code{null} as \code{Int},
using \code{null} is a bad solution. Dereferencing \code{null} causes a
run-time error, which is the well-known \code{NullPointerException}.
The compiler does not check whether \code{null} is dereferenced.
Therefore, using \code{null} is nothing better than using exceptions.
Use of \code{null} has been criticized enormously because \code{null} is extremely
error-prone. Even Tony Hoare, the inventor of \code{null}, said that inventing
\code{null} was a terrible mistake:

\begin{quote}
I call it my billion-dollar mistake. It was the invention of the null reference
in 1965.\sidenote{\url{https://en.wikipedia.org/wiki/Null\_pointer\#History}}
\end{quote}

The second attempt is to use a particular error-indicating value, e.g. \code{-1}.

\begin{verbatim}
def get(l: List[Int], n: Int): Int =
  if (n < 0)
    -1
  else l match {
    case Nil =>
      -1
    case h :: t =>
      if (n == 0)
        h
      else
        get(t, n - 1)
  }
\end{verbatim}

The strategy has an obvious problem. The caller cannot distinguish two
situations:
\begin{itemize}
  \item The list contains \code{-1}.
  \item The index is invalid.
\end{itemize}
Default values can be successful solutions for certain purposes but do not fit \code{get}.

Instead of using a fixed particular value in \code{get}, the caller can specify the default value.

\begin{verbatim}
def getOrElse(l: List[Int], n: Int, default: Int): Int =
  if (n < 0)
    default
  else l match {
    case Nil =>
      default
    case h :: t =>
      if (n == 0)
        h
      else
        getOrElse(t, n - 1, default)
}
\end{verbatim}

It works well when an appropriate default value
exists. However, when checking failures is per se important, the new strategy is
as bad as the previous strategy. There is no way to distinguish an element and
the default value.

Functional languages provide the option type to handle erroneous situations
safely. As the name implies, it represents an optional existence of a value.
The Scala standard library defines the option type like below.\sidenote{
We will not see what \code{[+A]} and \code{Nothing} are here.
You can understand the overall ADT structure without knowing those concepts.}

\begin{verbatim}
sealed trait Option[+A]
case object None extends Option[Nothing]
case class Some[A](value: A) extends Option[A]
\end{verbatim}

An option that may have a value of type \code{T} has type \code{Option[T]}.
An option is either \code{None} or \code{Some}.
\code{None} is a value that does not denote any value and similar
to \code{null}. It indicates a problematic situation. Like \code{Nil}, it is
defined as a case object because every \code{None} is identical. \code{Some} constructs a value that
denotes that a value exists. It is similar to a reference to a real object and
indicates that computation has succeeded.

The following code defines \code{getOption}, which returns an option.

\begin{verbatim}
def getOption(l: List[Int], n: Int): Option[Int] =
  if (n < 0)
    None
  else l match {
    case Nil =>
      None
    case h :: t =>
      if (n == 0)
        Some(h)
      else
        getOption(t, n - 1)
  }

assert(getOption(List(1, 2), 0) == Some(1))
assert(getOption(List(1, 2), 2) == None)
\end{verbatim}

For an invalid index, the return value is \code{None}. The caller can notice
that the operation has failed by \code{None}.
Otherwise, the function packs
an element inside \code{Some} to make the return value.

The Scala standard library uses options in many places. Various methods return options.
For example, \code{headOption} of a list returns \code{None} when the list is
empty. Otherwise, \code{Some} containing the head of the list is returned.

\begin{verbatim}
assert(List().headOption == None)
assert(List(1).headOption == Some(1))
\end{verbatim}

Also, \code{get} of a map returns \code{None} when the map does not have a given key.
Otherwise, \code{Some} containing the value corresponding to the key is
returned.

\begin{verbatim}
val m = Map(1 -> "one", 2 -> "two")
assert(m.get(0) == None)
assert(m.get(1) == Some("one"))
\end{verbatim}

Pattern matching allows programmers to deal with options by
distinguishing the \code{None} and \code{Some} cases. In addition, like the
methods of lists, options also provide methods to abstract common patterns.
We are going to see two methods: \code{map} and \code{flatMap}.

\code{map} can be used when we want to apply some computation only when the
previous computation has succeeded. \code{map} takes a single argument, which
must be a function. \code{opt.map(f)} results in \code{None} when \code{opt} is
\code{None}. If \code{opt} is \code{Some(v)}, then \code{opt.map(f)} evaluates
to \code{Some(f(v))}.

As an example, let us consider a map containing students.
Names are the keys, and students are the values. We want to find a student by a name and
get one's height only when the student exists. It can be implemented with
\code{map}.

\begin{verbatim}
def getHeight(
  m: Map[String, Student],
  name: String
): Option[Int] =
  m.get(name).map(_.height)
\end{verbatim}

If \code{m.get(name)} is \code{None}, then \code{m.get(name).map(\_.height)} also
is \code{None}. Otherwise, \code{m.get(name)} should be \code{Some(student)}, and
\code{m.get(name).map(\_.height)} will result in \code{Some(student.height)}.

In summary, \code{map} is useful when the computation consists of two steps, and
the first step can fail.

\code{flatMap} is similar to \code{map} but a bit different. It is useful when
the computation consists of two steps, and both steps can fail.
\code{flatMap} takes a single argument, which must be a function that returns an option.
\code{opt.flatMap(f)} results in \code{None} when \code{opt} is
\code{None}. If \code{opt} is \code{Some(v)}, then \code{opt.flatMap(f)} evaluates
to \code{f(v)}.

Let us consider a list of names and a map like before.
When the list is nonempty, we will find a student with the first name in the
list from the map. It is a typical application of \code{flatMap}.

\begin{verbatim}
def getStudent(
  l: List[String],
  m: Map[String, Student]
): Option[Student] =
  l.headOption.flatMap(m.get)
\end{verbatim}

The standard library provides many other useful methods for
options.\sidenote{\url{https://www.scala-lang.org/api/current/scala/Option.html}}


\pagelayout{wide} % No margins
\addpart{Untyped Languages}
\pagelayout{margin} % Restore margins

\setchapterpreamble[u]{\margintoc}
\chapter{Syntax}
\labch{syntax}

The course defines programming languages. Defining a language is defining the
\term{syntax} and the \term{semantics} of the language. The article is about
syntax. Before going into detail about syntax, it firstly explains why defining a
language is essential.

Languages defined by the course are tiny and whom people do not use in practice.
For example, they cannot get input from users or print results; they do not have
typical types, including a string type and a floating-point number type. It seems
meaningless to define languages that do not have any usages. Defining such tiny
languages does not aim to make the course easy for undergraduate students.
Surprisingly, many kinds of PL research deal with small unused languages.

PL research often aims to prove that a language satisfies a specific property.
The language might be a language used by plenty of people at the moment or an
improved, unimplemented version of an existing language with new features. The
sentence uses the term 'property' in a broad sense: it refers to a property
derived from the definition of the language; it refers to the characteristics of
results obtained by applying a specific algorithm to code written in the
language.

Researchers define small languages because real-world languages are complicated
to be the subjects of research. The real-world languages have many features
helping programmers, such as syntactic sugar. Verifying a property of a language
containing all such features takes a long time. If all the features affected the
property, they sadly would have to deal with a language with all the features.
However, most features are not related to the property, whom the researchers want
to show. It is efficient to work on a small language containing only important
characteristics by identifying features influencing the property.

Besides, it is hard to apply research on a specific existing language to other
languages. If researchers proved a property while reflecting all the features of
the language, they would not be able to conclude that other languages with a
portion of the features satisfy the property. In contrast, if they research a
small language containing features affecting the property, they will be able to
apply the result to such other languages without considerable cost.

Research on Scala is a concrete example. Scala features objects with \term{type
members}---ignore what it is. Popular languages preceding Scala had not featured
them. It had not been sure whether type systems with objects with type members
are \term{type-sound}---ignore what type soundness is. Researchers had defined
DOT (dependent object type), which is a small language with objects with type
members, and proved the type-soundness of DOT. If they had tried to prove the
type-soundness of Scala, they would have spent decades and exerted themselves for
features orthogonal to objects with type members. However, not spending much
time, they had proved the type-soundness of DOT and could apply the result to
languages sharing the feature, such as Wyvern. Alas, even though DOT models the
feature precisely, the type-soundness of DOT does not imply the type-soundness of
Scala. Nonetheless, proving the safety of the core of Scala is crucial for those
who want to trust Scala. As features other than objects with type members of
Scala have been already verified with other researches, verifying only DOT is
quite enough.

In summary, the following is a typical flow of PL research.

1. Want to prove that feature \verb!B! of language \verb!A! satisfies property
\verb!C!.
2. Define small language \verb!a! representing \verb!B!.
3. Define property \verb!c! for \verb!a! as \verb!C! for \verb!A!.
4. Prove that \verb!a! satisfies \verb!c!.
5. \verb!A! probably satisfies \verb!C!, and other languages which feature
\verb!B! may satisfy \verb!C!.

Mind that numerous sorts of PL research do not follow the flow. PL researchers
make, prove, and verify real-world languages, programs, and systems. They invent
tools for practical usages. For instance, Infer of Facebook is a static analyzer
developed by PL researchers. Companies including Facebook and Amazon have been
using Infer.

Such practical research cannot exist without a theoretical background for core
properties and algorithms produced by research dealing with small languages. [The
foundations of
Infer](https://fbinfer.com/docs/separation-logic-and-bi-abduction.html) are
theories suggested by a few papers that are not on real-world languages. Since
the objects of the papers are small but general, Infer can analyze Java, C, C++,
and Objective-C rather than a single language.

The most crucial thing of PL research is to define and solve a small precise
problem expressing a problem of interest. So does the course. The course focuses
on essential features provided by most languages and defines tiny languages
representing the features. The course is a starting point of PL research. At the
same time, the course gives students who are not interested in PL basic knowledge
to understand and to use new languages.

\section{Syntax}

The syntax of a language determines whether code is correct code written in the
language.

\begin{verbatim}
class A
\end{verbatim}

\begin{verbatim}
class A {
\end{verbatim}

The former is code written in Scala, but the latter is not. The syntax of Scala
determines it.

In a mathematical sense, assume that the set of all possible code exists; the set
of all correct code written in language A is a subset of the former set. The
syntax of A defines the subset.

Syntax is either concrete or abstract. Despite the lack of strict definitions of
concrete and abstract syntax, they have distinct properties and are thus easily
distinguished. The course explains them briefly: concrete syntax is for people;
abstract syntax is for computers. The explanation intuitively shows what they
are.

\subsection{Concrete Syntax}

Existing for humans, \term{concrete syntax} deals with code written by people. It
defines a rule for strings and cares about all the characters including
whitespaces and newlines; it specifies rules like "two quotation marks are at the
start and the end of a string," "two consecutive backslashes indicate the start
of a comment," and "every operator is an infix operator." The specifications of
most languages describe the concrete syntax of the languages since programmers
write code according to the specifications.

\term{Backus-Naur form} (BNF) is the most popular way to describe syntax. A form
includes one or more rules. Each rule is in the form of
\verb!<symbol> ::= expression | expression …!. A symbol between angle brackets is a
\term{metavariable}, which denotes a set of strings. An expression is an
enumeration of metavariables and strings. A set denoted by the metavariable
includes strings obtained by substituting metavariables with elements of the
metavariables in one of the expressions. Every string starts and ends with
quotation marks.

The article defines the syntax of AE, a language for arithmetic expressions.

An expression of AE is

\begin{itemize}
\item an integer,
\item the sum of two expressions, or
\item the difference of two expressions.
\end{itemize}

The following is the concrete syntax of AE in the BNF:

\[
\begin{array}{l}
\texttt{digit ::= "0" | "1" | "2" | "3" | "4"} \\
\texttt{\ \ \ \ \ \ \ \ }\texttt{| "5" | "6" | "7" | "8" | "9"} \\
\texttt{nat}\texttt{\ \ \ }\texttt{::= digit | digit nat} \\
\texttt{num}\texttt{\ \ \ }\texttt{::= nat | "-" nat} \\
\texttt{expr}\texttt{\ \ }\texttt{::= num} \\
\texttt{\ \ \ \ \ \ \ \ }\texttt{| "(" expr "+" expr ")"} \\
\texttt{\ \ \ \ \ \ \ \ }\texttt{| "(" expr "-" expr ")"} \\
\end{array}
\]

The remaining part of the section shows how to interpret syntax in the BNF.
\(Digit\) is a set denoted by \(\texttt{digit}\); \(Nat\) is a set denoted by \(\texttt{
nat}\); \(Num\) is a set denoted by \(\texttt{num}\); \(Expr\) is a set denoted by
\(\texttt{expr}\).

\(Digit\) equals , a set of the digits of decimals.

\(Nat\) is the smallest set satisfying the following two conditions; it denotes
the set of every natural number. The \(\cdot\) operator denotes string
concatenation.

\begin{enumerate}
\item \(\forall d\in Digit.d\in Nat\)
\item \(\forall d\in Digit.\forall n\in Nat.d \cdot n\in Nat\)
\end{enumerate}

\(Num\) is the smallest set satisfying the following two conditions; it denotes
the set of every integer.

\begin{enumerate}
\item \(\forall n\in Nat.n\in Num\)
\item \(\forall n\in Nat.\texttt{"-"}\cdot n\in Num\)
\end{enumerate}

\(Expr\) is the smallest set satisfying the following three conditions; it
denotes the set of every arithmetic expression.

\begin{enumerate}
\item \(\forall n\in Num.n\in Expr\)
\item \(\forall e_1\in Expr.\forall e_2\in Expr.{\texttt{"("}}\cdot e_1\cdot{\texttt{"+"}}\cdot
e_2\cdot{\texttt{")"}}\in Expr\)
\item \(\forall e_1\in Expr.\forall e_2\in Expr.{\texttt{"("}}\cdot
e_1\cdot\texttt{"-"}\cdot e_2\cdot{\texttt{")"}}\in Expr\)
\end{enumerate}

\(\texttt{"(1+2)"}\) is an element of \(Expr\), but \(\texttt{"1+2"}\) is not an element
of \(Expr\) due to the lack of parentheses.

\subsection{Abstract Syntax}

Most kinds of PL research define languages with \term{abstract syntax} instead of
concrete syntax, which is unnecessarily precise. As the previous section shows,
concrete syntax cares unimportant details.

Abstract syntax is an abstract data structure describing code. Unlike concrete
syntax, which deals with strings, it deals with abstract objects. Since people
mostly use strings to represent information, strings often describe abstract
syntax. However, the essence of abstract syntax is not about strings. For
example, strings "1" and "one" represent the number one, but the essence of the
number one is that it is the successor of zero but not how people write it on
papers. In the same manner, regardless of a way of describing abstract syntax,
abstract deals with abstract objects but not strings.

The following is the abstract syntax of AE in the BNF:

\[
\begin{array}{rcl}
n & \in & \mathbb{Z} \\
e & ::= & n \\
& | & e+e \\
& | & e-e \\
\end{array}
\]

Metavariable \(n\) ranges over integers; metavariable \(e\) ranges over
expressions.

Like concrete syntax, the abstract syntax in the BNF defines a set. Let
\(\mathcal{A}\) is a set denoted by \(e\). \(\mathcal{A}\) is the smallest set
satisfying the following three conditions.

\begin{enumerate}
\item \(\forall n\in\mathbb{Z}.n\in \mathcal{A}\)
\item \(\forall e_1\in\mathcal{A}.\forall e_2\in\mathcal{A}.e_1+e_2\in\mathcal{A}\)
\item \(\forall e_1\in\mathcal{A}.\forall e_2\in\mathcal{A}.e_1-e_2\in\mathcal{A}\)
\end{enumerate}

\term{Inference rules} can define abstract syntax as well. Inference rules
typically define the semantics of languages, but the article defines abstract
syntax with inference rules to make readers familiar with inference rules. It is
possible to define concrete syntax with inference rules, but I think that it is
redundant and unnecessary.

Inference rules derive a \term{proposition} from propositions. An inference rule
is composed of a horizontal line, zero or more propositions above the line, and a
proposition below the line. If no proposition exists above the line, then the
line can be omitted. The propositions above the line are premises; the
proposition below the line is a conclusion. Every proposition in the rule may
have metavariables.

For instance, an inference rule can encode \term{modus ponens}, which implies
that for any propositions \(P\) and \(Q\), if \(P\rightarrow Q\) and \(P\), then
\(Q\). Let metavariables \(p\) and \(q\) range over propositions.

\[
\inferrule
{ p\rightarrow q \\ p }
{ q }
\]

If substituting every metavariable with an element of the metavariable in a rule
makes every premise of the rule true, then the conclusion of the rule also is
true. Assume that \(P\) and \(Q\) are propositions, and both \(Q\rightarrow P\)
and \(Q\) are true. Substituting \(p\) and \(q\) with \(Q\) and \(P\) results in
two true premises and conclusion \(P\). The following \term{proof tree} is a
proof of \(P\):

\[
\inferrule
{ Q\rightarrow P \\ Q }
{ P }
\]

One can use inference rules multiple times to prove a proposition. Assume that
\(P\), \(Q\), and \(R\) are propositions, and \(P\rightarrow(Q\rightarrow R)\),
\(P\), and \(Q\) are true. Substituting \(p\) and \(q\) with \(P\) and
\(Q\rightarrow R\) yields that \(Q\rightarrow R\) is true. Substituting \(p\) and
\(q\) with \(Q\) and \(R\) finally proves \(R\). The following proof tree gives a
proof:

\[
\inferrule
{
{\inferrule
  { P\rightarrow(Q\rightarrow R) \\ P }
  { Q\rightarrow R } }\\
  Q }
{ R }
\]

The following inference rules define the abstract syntax of AE:

\[
\inferrule
{ n\in\mathbb{Z} }
{ n\in\mathcal{A} }
\\
\inferrule
{ e_1\in\mathcal{A} \\ e_2\in\mathcal{A} }
{ e_1+e_2\in\mathcal{A} }
\\
\inferrule
{ e_1\in\mathcal{A} \\ e_2\in\mathcal{A} }
{ e_1-e_2\in\mathcal{A} }
\]

The following proof tree proves that \(4+(2-1)\) is an element of
\(\mathcal{A}\).
Note that we can use parentheses to resolve ambiguity in abstract syntax since
it defines mathematical notation.

\[
\inferrule
{
\inferrule
  { 4\in\mathbb{Z} }
  { 4\in\mathcal{A} } \\
  \inferrule
  { \inferrule
    { 2\in\mathbb{Z} }
    { 2\in\mathcal{A} } \\
    \inferrule
    { 1\in\mathbb{Z} }
    { 1\in\mathcal{A} }
  }
  { (2-1)\in\mathcal{A} }
}
{ 4+(2-1)\in\mathcal{A} }
\]

Scala code also can represent the abstract syntax of AE. It is a typical ADT; a
sealed trait and case classes define it:

\begin{verbatim}
sealed trait Expr
case class Num(n: Int) extends Expr
case class Add(l: Expr, r: Expr) extends Expr
case class Sub(l: Expr, r: Expr) extends Expr
\end{verbatim}

The following Scala code represents \(4+(2-1)\):

\begin{verbatim}
Add(Num(4), Sub(Num(2), Num(1)))
\end{verbatim}

Most sorts of abstract syntax define tree shapes. Trees following abstract syntax
are \term{abstract syntax trees} (ASTs). The below tree visualizes \(4+(2-1)\).
The structure of an object defined by the above Scala code equals the tree.

\subsection{Parsing}

\term{Parsing} is a process that transforms strings following concrete syntax
into ASTs and rejects strings not following the concrete syntax. A \term{parser}
is a parsing program. Parsing is out of the scope of the course and thus is out
of the scope of the article.

The Scala standard library provides \term{parser combinators}. Programmers can
implement parsers without detailed knowledge about parsing. The below code
implements a parser of AE. The parser takes a string as input and produces an AST
of AE; it throws an exception if the string does not follow the concrete syntax
of AE. Note that strings may contain whitespaces freely, while concrete syntax
defined by the article is tight with whitespaces.

\begin{verbatim}
import scala.util.parsing.combinator._

object Expr extends RegexParsers {
  def wrap[T](e: Parser[T]): Parser[T] = "(" ~> e <~ ")"
  lazy val n: Parser[Int] = "-?\\d+".r ^^ (_.toInt)
  lazy val e: Parser[Expr] =
    n                    ^^ Num                         |
    wrap((e <~ "+") ~ e) ^^ { case l ~ r => Add(l, r) } |
    wrap((e <~ "-") ~ e) ^^ { case l ~ r => Sub(l, r) }

  def parse(s: String): Expr =
    parseAll(e, s).getOrElse(throw new Exception)
}

Expr.parse("1")
// Num(1)

Expr.parse("(4 + (2 - 1))")
// Add(Num(4),Sub(Num(2),Num(1)))

Expr.parse("1 + 2")
// java.lang.Exception
\end{verbatim}

\newpage
\section{Exercises}

\begin{enumerate}
\item Given the following grammar:
%
\begin{verbatim}
    <WAE> ::= <num>
            | {+ <WAE> <WAE>}
            | {* <WAE> <WAE>}
            | {let {<id> <WAE>} <WAE>}
            | <id>
\end{verbatim}
%
Describe whether each of the following is \verb+<WAE>+ and why:

\begin{itemize}
\item[a)]
\begin{verbatim}
{let {x 5} {+ 8 {* x 2 3}}}
\end{verbatim}

\item[b)]
\begin{verbatim}
{with {x 0} {with {x 7}}}
\end{verbatim}

\item[c)]
\begin{verbatim}
{let {3 5} {+ 8 {- x 2}}}
\end{verbatim}

\item[d)]
\begin{verbatim}
{let {3 y} {+ 8 {* x 2}}}
\end{verbatim}

\item[e)]
\begin{verbatim}
{let {x y} {+ 8 {* x 2}}}
\end{verbatim}
\end{itemize}

\item Given the following grammar:
%
\newcommand{\BNF}[1]{$\langle$#1$\rangle$}
\newcommand{\coffee}{\mbox{\BNF{coffee}}}
\newcommand{\milk}{\mbox{\BNF{milk}}}
\newcommand{\flavor}{\mbox{\BNF{flavor}}}

\[
\begin{array}{ccc}
{\texttt{espresso} \in \coffee}
&&
\newinfrule
{e_1 \in \milk\qquad
e_2 \in \coffee
}
{\ e_1\ \texttt{on}\ e_2 \in \coffee}
\\[2em]
\newinfrule
{e_1 \in \coffee\qquad
e_2 \in \milk
}
{\ e_1\ \texttt{on}\ e_2 \in \coffee}
&&
\newinfrule
{e_1 \in \flavor \qquad
e_2 \in \coffee
}
{\ e_1\ \texttt{on}\ e_2 \in \coffee}
\\[2em]
{\texttt{milk-foam} \in \milk}
&&
{\texttt{steamed-milk} \in \milk}
\\[2em]
{\texttt{caramel} \in \flavor}
&&
{\texttt{cinnamon} \in \flavor}
\\[2em]
{\texttt{cocoa-powder} \in \flavor}
&&
{\texttt{chocolate-syrup} \in \flavor}
\end{array}
\]
where \texttt{on} is right-associative.

\begin{itemize}
\item[a)] Which of the following are examples of \BNF{coffee}?
%
\begin{enumerate}

  \item[1)] \texttt{caramel latte macchiato} % No

  \item[2)] \texttt{espresso} % Yes

  \item[3)] \texttt{steamed-milk on caramel on milk-foam on espresso} % Yes

  \item[4)] \texttt{chocolate-syrup on cocoa-powder on cinnamon on milk-foam on steamed-milk on espresso} % Yes

  \item[5)] \texttt{steamed-milk on espresso on chocolate-syrup} % No

\end{enumerate}

\item[b)] Draw a proof of why the following is or is not \BNF{coffee}:

\begin{center}
\texttt{cocoa-powder on milk-foam on steamed-milk on espresso}
\end{center}
\end{itemize}

\item Given the following grammar:
%
\begin{center}
\begin{tabular}{lll}
 \BNF{ice-cream} & $::=$ & \texttt{sprinkles on \BNF{ice-cream}} \\
  & $|$ & \texttt{cherry on \BNF{ice-cream}} \\
  & $|$ & \texttt{scoop of \BNF{flavor} on \BNF{ice-cream}} \\
  & $|$ & \texttt{sugar-cone} \\
  & $|$ & \texttt{waffle-cone} \\
 \BNF{flavor} & $::=$ & \texttt{vanilla} \\
  & $|$ & \texttt{lettuce}
\end{tabular}
\end{center}
%
\begin{itemize}
  \item[a)] Which of the following are examples of \BNF{ice-cream}?
%
\begin{itemize}
  \item[1)] \texttt{sprinkles}
  \item[2)] \texttt{sugar-cone}
  \item[3)] \texttt{vanilla}
  \item[4)] \texttt{scoop of vanilla on waffle-cone}
  \item[5)] \texttt{sprinkles on lettuce on waffle-cone}
  \item[6)] \texttt{scoop of vanilla on sprinkles on waffle-cone}
\end{itemize}

  \item[b)] Explain why the following is an \BNF{ice-cream}:

\begin{center}
\texttt{cherry on scoop of lettuce on scoop of vanilla on sugar-cone}
\end{center}
\end{itemize}

\end{enumerate}

\setchapterpreamble[u]{\margintoc}
\chapter{Semantics}
\labch{semantics}

Syntax and semantics define a programming language. Syntax determines whether
code is code written in the language. Semantics decides what the code denotes.
Without semantics, code written by programmers is no more than a string.

The shape of semantics depends on the property of interest. If the property is
about how programs modify memories of computers, defining semantics is
defining what a memory is and how code modifies a memory. If the property is
about input and output of programs, defining semantics is defining what input
and output are and what code prints for input. The course focuses on
functional languages, and the semantics of a functional language determines a
value obtained by interpreting an expression.

\section{Defining Semantics}

There are various styles of semantics: denotational semantics and operational
semantics are famous; axiomatic semantics and others exist. Denotational
semantics define values denoted by programs with mathematical methods. For
example, denotational semantics of an imperative language views a program as a
function from states to states. On the other hand, operational semantics
expresses executions of programs with logical statements, such as inference
rules. Operational semantics is more similar to the implementation of an
interpreter than denotational semantics but does differ from an implementation.

There are multiple forms of operational semantics: natural semantics,
structural operational semantics (SOS), reduction semantics, abstract machine
semantics, and others. The course mainly deals with natural semantics, as
known as big-step semantics. Natural semantics is composed of one or more
inference rules. A rule defines a value denoted by an expression. In contrast,
small-step semantics, including SOS and reduction semantics, uses inference
rules that transform an expression into an expression instead of a value. Big
step semantics produces a value at one big step, while small-step semantics
requires multiple small steps to attain a value.

Each kind of semantics has its characteristic. Different types of research
need different types of semantics. For example, defining both concrete and
abstract syntax in a denotational style allows expressing a relationship
between them mathematically and showing the soundness of abstract
interpretations. Axiomatic semantics fits verifying the correctness of program,
and proving the type-soundness of languages is a typical usage of reduction
semantics. Natural semantics intuitively defines languages and is closest to
the implementation of an interpreter.

The above explanation contains words not introduced by the course. It is
enough to understand that various ways to define semantics exist and choosing
a proper style for a subject is crucial.

\section{Natural Semantics}

The last article defined the abstract syntax of AE.

\[
\begin{array}{lrcl}
\text{Integer} & n & \in & \mathbb{Z} \\
\text{Expression} & e & ::= & n \\
&& | & e+e \\
&& | & e-e \\
\end{array}
\]

Metavariable \(n\) ranges over integers; metavariable \(e\) ranges over
expressions; \(\text{Expression}\) is the set of every expression.

This article defines the natural semantics of AE.

The semantics of AE decides values denoted by expressions of AE. The first
thing to do is defining what values are. Every arithmetic expression denotes
an integer so that every value of AE is an integer.

\[
\begin{array}{lrcl}
\text{Value} & v & ::= & n
\end{array}
\]

Metavariable \(v\) ranges over values; \(\text{Value}\) is the set of every
value; it equals \(\mathbb{Z}\).

Every expression of AE denotes a value of AE. The semantics of AE seems to be
a function from expressions to values. Let \(\Rightarrow\) be the function.

\[\Rightarrow:\ \text{Expression}\rightarrow\text{Value}\]

However, in general, not every expression of a language denotes a value. As an
execution might terminate without yielding a result due to an error,
expressions not denoting any values exist. Besides, if some expressions
produce random values, a single expression may denote multiple values.
Therefore, it is desirable to define semantics as a binary relation over
\(\text{Expression}\) and \(\text{Value}\).

\[\Rightarrow\subseteq\text{Expression}\times\text{Value}\]

For any expression \(e\) and any value \(v\), \((e,v)\in\Rightarrow\) implies
that \(e\) denotes \(v\), or \(v\) is the result of evaluating \(e\). In PL
research, notation \(\vdash e\Rightarrow v\) replaces \((e,v)\in\Rightarrow\).
\(\Rightarrow\) is a relation and thus does not imply input and output, but,
intuitively, expressions are input and values are output.

Inference rules define the semantics of AE.

\[
\vdash n\Rightarrow n
\]

If an expression is an integer, the expression denotes the integer. The rule
does not have any premises. It has the following mathematical meaning:

\[ \forall n\in\mathbb{Z}.\vdash n\Rightarrow n \]

Intuitively, \(n\) does not require any computation, and the result is \(n\).

\[
\inferrule
{ \vdash e_1\Rightarrow n_1\\\vdash e_2\Rightarrow n_2 }
{ \vdash e_1+e_2\Rightarrow n_1+n_2 }
\]

If an expression is the sum of two expressions, the expression denotes the sum
of two integers denoted by the two expressions. The rule has the following
mathematical meaning:

\[
\begin{array}{l}
\forall e_1\in\text{Expression}.
\forall e_2\in\text{Expression}.
\forall n_1\in\mathbb{Z}.
\forall n_2\in\mathbb{Z}.\\
(\vdash e_1\Rightarrow n_1)\rightarrow
(\vdash e_2\Rightarrow n_2)\rightarrow
(\vdash e_1+e_2\Rightarrow n_1+n_2)
\end{array}
\]

Intuitively, computing \(e_1+e_2\) requires computing \(e_1\) and \(e_2\), and
since \(e_1\) and \(e_2\) respectively result in \(n_1\) and \(n_2\), the
result is \(n_1+n_2\). \(e_1\) and \(e_2\) are given; intermediate computation
yields \(n_1\) and \(n_2\); the final result is \(n_1+n_2\).
Note that
in \(e_1+e_2\), \(+\) is a symbol used to represent abstract syntax, while \(+\)
in \(n_1+n_2\) denotes mathematical addtion as usual.

\[
\inferrule
{ \vdash e_1\Rightarrow n_1\\\vdash e_2\Rightarrow n_2 }
{ \vdash e_1-e_2\Rightarrow n_1-n_2 }
\]

If an expression is the difference of two expressions, the expression denotes
the difference of two integers denoted by the two expressions. The rule has
the following mathematical meaning:

\[
\begin{array}{l}
\forall e_1\in\text{Expression}.
\forall e_2\in\text{Expression}.
\forall n_1\in\mathbb{Z}.
\forall n_2\in\mathbb{Z}.\\
(\vdash e_1\Rightarrow n_1)\rightarrow
(\vdash e_2\Rightarrow n_2)\rightarrow
(\vdash e_1-e_2\Rightarrow n_1-n_2)
\end{array}
\]

Intuitively, computing \(e_1-e_2\) requires computing \(e_1\) and \(e_2\), and
since \(e_1\) and \(e_2\) respectively result in \(n_1\) and \(n_2\), the
result is \(n_1-n_2\). \(e_1\) and \(e_2\) are given; intermediate computation
yields \(n_1\) and \(n_2\); the final result is \(n_1-n_2\).
Like \(+\), \(-\) represents both expressions in abstract syntax and
mathematical subtraction.

The article keeps emphasizing that both understanding mathematical definitions
and interpreting the semantics intuitively are essential. In a mathematical
sense, the natural semantics of AE is a relation over \(\text{Expression}\)
and \(\text{Value}\), and the inference rules do not care what given things
are and what obtained things are. In contrast, intuitively, the natural
semantics find a value denoted by a given expression. An expression inside the
conclusion of a rule is input; the premises of a rule represent required
computation; a value inside the conclusion is output. Not considering
mathematical definitions, one may make a mistake while strictly thinking and
hardly understand complicated semantics. Complex semantics needs rules not
interpreted with the concepts of input, computation, and output; for instance,
computation uses output. It intuitively seems odd but is natural in a
mathematical sense, which does not have such concepts. On the other hand, not
interpreting semantics intuitively, one hardly understands a language. In
conclusion, both viewpoints are crucial.

The following rules are all of the natural semantics of AE:

\[
\vdash n\Rightarrow n
\]

\[
\inferrule
{ \vdash e_1\Rightarrow n_1\\\vdash e_2\Rightarrow n_2 }
{ \vdash e_1+e_2\Rightarrow n_1+n_2 }
\]

\[
\inferrule
{ \vdash e_1\Rightarrow n_1\\\vdash e_2\Rightarrow n_2 }
{ \vdash e_1-e_2\Rightarrow n_1-n_2 }
\]

\subsection{Drawing Proof Trees}

The following proof tree proves that \(4+(2-1)\) denotes \(5\):

\[
\inferrule
{
  \vdash 4\Rightarrow 4 \\
  \inferrule
  {\vdash 2\Rightarrow 2 \\ \vdash 1\Rightarrow 1}
  {\vdash 2-1\Rightarrow 1}
}
{\vdash4+(2-1)\Rightarrow 5}
\]

Drawing proof trees is not an interesting research topic. However, it is a
good practice to understand semantics, and some students feel difficult about
it. The article thus briefly introduces a strategy to draw proof trees.

Languages defined by the course have simple semantics. Usually, only a single
inference rule fits a given expression. The meanings of propositions are
unimportant, and substituting metavariables with appropriate expressions is
enough to draw proof trees. Drawing proof trees is often mechanical.

The remaining part of the section draws a proof tree proving that \(4+(2-1)\)
denotes \(5\) step by step. Firstly, an expression inside a conclusion is \(4+(2-1)\).

\[
\color{red}{
\inferrule
{
  \color{white}{
  \vdash 4\Rightarrow 4} \\
  \color{white}{
  \inferrule
  {\vdash 2\Rightarrow 2 \\ \vdash 1\Rightarrow 1}
  {\vdash 2-1\Rightarrow 1}
  }
}
{\vdash4+(2-1)\Rightarrow \color{white}{5}}
}
\]

A single rule fits \(4+(2-1)\). Substitute \(e_1\) and \(e_2\) with \(4\) and
\(2-1\) respectively to make premises. Do not write values of the premises.

\[
\inferrule
{
  \color{red}{\vdash 4\Rightarrow {\color{white}4}} \\
  \color{red}{
  \inferrule
  {\color{white}{\vdash 2\Rightarrow 2 \\ \vdash 1\Rightarrow 1}}
  {\vdash 2-1\Rightarrow \color{white}{1}}
  }
}
{\vdash4+(2-1)\Rightarrow \color{white}{5}}
\]

A single rule fits \(4\). The rule has no premises. Substitute \(n\) with \(
\) and write the value.

\[
\inferrule
{
  \vdash 4\Rightarrow \color{red}{4} \\
  \inferrule
  {\color{white}{\vdash 2\Rightarrow 2 \\ \vdash 1\Rightarrow 1}}
  {\vdash 2-1\Rightarrow \color{white}{1}}
}
{\vdash4+(2-1)\Rightarrow \color{white}{5}}
\]

A single rule fits \(2-1\). Substitute \(e_1\) and \(e_2\) with \(2\) and \(
\) respectively to make premises. Do not write values of the premises.

\[
\inferrule
{
  \vdash 4\Rightarrow 4 \\
  \inferrule
  {\color{red}{\vdash 2\Rightarrow \color{white}{2} \\ \vdash 1\Rightarrow \color{white}{1}}}
  {\vdash 2-1\Rightarrow \color{white}{1}}
}
{\vdash4+(2-1)\Rightarrow \color{white}{5}}
\]

A single rule fits \(2\). The rule has no premises. Substitute \(n\) with \(2\) and write the value.

\[
\inferrule
{
  \vdash 4\Rightarrow 4 \\
  \inferrule
  {\vdash 2\Rightarrow \color{red}{2} \\ \vdash 1\Rightarrow \color{white}{1}}
  {\vdash 2-1\Rightarrow \color{white}{1}}
}
{\vdash4+(2-1)\Rightarrow \color{white}{5}}
\]

A single rule fits \(1\). The rule has no premises. Substitute \(n\) with \(1\) and write the value.

\[
\inferrule
{
  \vdash 4\Rightarrow 4 \\
  \inferrule
  {\vdash 2\Rightarrow 2 \\ \vdash 1\Rightarrow \color{red}{1}}
  {\vdash 2-1\Rightarrow \color{white}{1}}
}
{\vdash4+(2-1)\Rightarrow \color{white}{5}}
\]

Compute \(2-1\) and write \(1\), the result of \(2-1\).

\[
\inferrule
{
  \vdash 4\Rightarrow 4 \\
  \inferrule
  {\vdash 2\Rightarrow 2 \\ \vdash 1\Rightarrow 1}
  {\vdash 2-1\Rightarrow \color{red}{1}}
}
{\vdash4+(2-1)\Rightarrow \color{white}{5}}
\]

Compute \(4+1\) and write \(5\), the result of \(4+(2-1)\).

\[
\inferrule
{
  \vdash 4\Rightarrow 4 \\
  \inferrule
  {\vdash 2\Rightarrow 2 \\ \vdash 1\Rightarrow 1}
  {\vdash 2-1\Rightarrow 1}
}
{\vdash4+(2-1)\Rightarrow \color{red}{5}}
\]

The tree is complete.

\subsection{Implementing an Interpreter}

The following Scala code is the implementation of an interpreter following the
natural semantics of AE:

\begin{verbatim}
sealed trait Expr
case class Num(n: Int) extends Expr
case class Add(l: Expr, r: Expr) extends Expr
case class Sub(l: Expr, r: Expr) extends Expr

def interp(e: Expr): Int = e match {
  case Num(n) => n
  case Add(l, r) => interp(l) + interp(r)
  case Sub(l, r) => interp(l) - interp(r)
}

interp(Add(Num(4), Sub(Num(2), Num(1))))  // 5
\end{verbatim}


\pagelayout{wide} % No margins
\addpart{Typed Languages}
\pagelayout{margin} % Restore margins

% \appendix % From here onwards, chapters are numbered with letters, as is the appendix convention

% \pagelayout{wide} % No margins
% \addpart{Appendix}
% \pagelayout{margin} % Restore margins

% \input{chapters/appendix.tex}

%----------------------------------------------------------------------------------------

\backmatter % Denotes the end of the main document content
\setchapterstyle{plain} % Output plain chapters from this point onwards

%----------------------------------------------------------------------------------------
%	BIBLIOGRAPHY
%----------------------------------------------------------------------------------------

% The bibliography needs to be compiled with biber using your LaTeX editor, or on the command line with 'biber main' from the template directory

\defbibnote{bibnote}{Here are the references in citation order.\par\bigskip} % Prepend this text to the bibliography
\printbibliography[heading=bibintoc, title=Bibliography, prenote=bibnote] % Add the bibliography heading to the ToC, set the title of the bibliography and output the bibliography note

%----------------------------------------------------------------------------------------
%	NOMENCLATURE
%----------------------------------------------------------------------------------------

% The nomenclature needs to be compiled on the command line with 'makeindex main.nlo -s nomencl.ist -o main.nls' from the template directory

\nomenclature{$c$}{Speed of light in a vacuum inertial frame}
\nomenclature{$h$}{Planck constant}

\renewcommand{\nomname}{Notation} % Rename the default 'Nomenclature'
\renewcommand{\nompreamble}{The next list describes several symbols that will be later used within the body of the document.} % Prepend this text to the nomenclature

\printnomenclature % Output the nomenclature

%----------------------------------------------------------------------------------------
%	GREEK ALPHABET
% 	Originally from https://gitlab.com/jim.hefferon/linear-algebra
%----------------------------------------------------------------------------------------

\vspace{1cm}

% {\usekomafont{chapter}Greek Letters with Pronounciation} \\[2ex]
% \begin{center}
% 	\newcommand{\pronounced}[1]{\hspace*{.2em}\small\textit{#1}}
% 	\begin{tabular}{l l @{\hspace*{3em}} l l}
% 		\toprule
% 		Character & Name & Character & Name \\
% 		\midrule
% 		$\alpha$ & alpha \pronounced{AL-fuh} & $\nu$ & nu \pronounced{NEW} \\
% 		$\beta$ & beta \pronounced{BAY-tuh} & $\xi$, $\Xi$ & xi \pronounced{KSIGH} \\
% 		$\gamma$, $\Gamma$ & gamma \pronounced{GAM-muh} & o & omicron \pronounced{OM-uh-CRON} \\
% 		$\delta$, $\Delta$ & delta \pronounced{DEL-tuh} & $\pi$, $\Pi$ & pi \pronounced{PIE} \\
% 		$\epsilon$ & epsilon \pronounced{EP-suh-lon} & $\rho$ & rho \pronounced{ROW} \\
% 		$\zeta$ & zeta \pronounced{ZAY-tuh} & $\sigma$, $\Sigma$ & sigma \pronounced{SIG-muh} \\
% 		$\eta$ & eta \pronounced{AY-tuh} & $\tau$ & tau \pronounced{TOW (as in cow)} \\
% 		$\theta$, $\Theta$ & theta \pronounced{THAY-tuh} & $\upsilon$, $\Upsilon$ & upsilon \pronounced{OOP-suh-LON} \\
% 		$\iota$ & iota \pronounced{eye-OH-tuh} & $\phi$, $\Phi$ & phi \pronounced{FEE, or FI (as in hi)} \\
% 		$\kappa$ & kappa \pronounced{KAP-uh} & $\chi$ & chi \pronounced{KI (as in hi)} \\
% 		$\lambda$, $\Lambda$ & lambda \pronounced{LAM-duh} & $\psi$, $\Psi$ & psi \pronounced{SIGH, or PSIGH} \\
% 		$\mu$ & mu \pronounced{MEW} & $\omega$, $\Omega$ & omega \pronounced{oh-MAY-guh} \\
% 		\bottomrule
% 	\end{tabular} \\[1.5ex]
% 	Capitals shown are the ones that differ from Roman capitals.
% \end{center}

%----------------------------------------------------------------------------------------
%	GLOSSARY
%----------------------------------------------------------------------------------------

% The glossary needs to be compiled on the command line with 'makeglossaries main' from the template directory

% \newglossaryentry{computer}{
% 	name=computer,
% 	description={is a programmable machine that receives input, stores and manipulates data, and provides output in a useful format}
% }

% % Glossary entries (used in text with e.g. \acrfull{fpsLabel} or \acrshort{fpsLabel})
% \newacronym[longplural={Frames per Second}]{fpsLabel}{FPS}{Frame per Second}
% \newacronym[longplural={Tables of Contents}]{tocLabel}{TOC}{Table of Contents}

% \setglossarystyle{listgroup} % Set the style of the glossary (see https://en.wikibooks.org/wiki/LaTeX/Glossary for a reference)
% \printglossary[title=Special Terms, toctitle=List of Terms] % Output the glossary, 'title' is the chapter heading for the glossary, toctitle is the table of contents heading

%----------------------------------------------------------------------------------------
%	INDEX
%----------------------------------------------------------------------------------------

% The index needs to be compiled on the command line with 'makeindex main' from the template directory

\printindex % Output the index

%----------------------------------------------------------------------------------------
%	BACK COVER
%----------------------------------------------------------------------------------------

% If you have a PDF/image file that you want to use as a back cover, uncomment the following lines

%\clearpage
%\thispagestyle{empty}
%\null%
%\clearpage
%\includepdf{cover-back.pdf}

%----------------------------------------------------------------------------------------

\end{document}
