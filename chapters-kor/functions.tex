\chapter{함수}
\labch{functions}

이 장은 함수형 프로그래밍에서 함수를 어떻게 사용하는지 설명한다. 함수형
프로그래밍에서 함수는 \term{일급}{first class}이다. \term{일급 함수}{first-class
function}는 프로그래머가 복잡한 계산을 단순하게 요약하는 것을 돕는다. 이 장에서
일급 함수가 무엇인지 알아볼 것이다. 더 나아가, 일급 함수와 관련된 개념인
\term{익명 함수}{anonymous function}와 \term{클로저}{closure}도 소개한다. 일급
함수의 유용함을 알아보기 위해 \refch{immutability}에서 작성한 \code{inc1}이나
\code{square} 같은 함수를 일급 함수를 활용해 다시 구현할 것이다.

\section{일급 함수}

프로그래밍 언어에서 어떤 대상이 일급이라는 말은 그 대상이 다음 조건을 만족한다는
뜻이다.

\begin{itemize}
\item 함수에 인자로 전달될 수 있다.
\item 함수의 결괏값이 될 수 있다.
\item 변수가 나타내는 값이 될 수 있다.
\end{itemize}

즉, 어떤 대상이 일급이라는 말은 그 대상을 값으로서 사용할 수 있다는 의미다.
함수형 언어에서 함수는 매우 중요하며 값으로 취급된다. 이렇게 값으로 사용될 수
있는 함수를 일급 함수라 부른다.

어떤 사람들은 \term{고차 함수}{higher-order function}라는 용어를 사용한다. 고차
함수는 \term{일차 함수}{first-order function}가 아닌 함수를 말한다. 여기서 일차
함수는 함수를 인자로 받거나 함수를 결과로 낼 수 없는 함수다. 그러니 고차 함수는
함수를 인자로 받거나 함수를 결과로 낼 수 있는 함수다. 일급 함수는 인자로
전달되거나 결괏값이 될 수 있는 함수를 뜻하니, 엄밀히 따지면 고차 함수는 일급
함수와는 다른 뜻인 셈이다. 그러나 당연하게도 일급 함수를 제공하는 언어는 고차
함수를 제공하고, 고차 함수를 제공하는 언어는 일급 함수를 제공한다. 일급 함수를
인자로 전달했다면 그 인자를 받은 함수는 고차 함수일 것이고, 고차 함수가 인자로
함수를 받았다면 인자로 전달된 함수는 일급 함수일 것이기 때문이다. 그래서 많은
경우에 사람들은 두 용어를 구분하지 않고 사용하며, 두 용어를 같은 뜻으로 이해해도
크게 문제되지 않는다.

스칼라에서 일급 함수를 사용하는 코드 예시를 보자.

\begin{verbatim}
def g(h: Int => Int): Int = h(0)
def f(x: Int): Int = x
assert(g(f) == 0)
\end{verbatim}

함수 \code{g}는 매개변수 \code{h}를 가진다. \code{h}의 타입은 \code{Int =>
Int}다. \code{g}에 전달되는 인자는 정수 하나를 받아 정수를 결과로 내는 함수다.
스칼라에서 함수의 타입을 나타낼 때는 \code{=>}를 사용한다. 매개변수가 없는
함수의 타입은 \code{() => [return type]}이다. 매개변수가 하나라면 타입이
\code{[parameter type] => [return type]}이다. 매개변수가 두 개 이상이라면 괄호를
사용해 \code{([parameter type], … ) => [return type]}이라 적는다. 함수
\code{f}는 정수 하나를 인자로 받아 정수를 결과로 낸다. 그러니 \code{f}의 타입은
\code{Int => Int}이며 \code{f}를 \code{g}의 인자로 사용할 수 있다. \code{g(f)}를
계산하는 것은 \code{f(0)}을 계산하는 것과 같으니 \code{g(f)}의 결과는 0이다.

\begin{verbatim}
def f(y: Int): Int => Int = {
  def g(x: Int): Int = x
  g
}

assert(f(0)(0) == 0)
\end{verbatim}

함수 \code{f}는 함수 \code{g}를 결과로 낸다. \code{f}의 결과 타입이 \code{Int =>
Int}이므로 결괏값이 정수를 받아 정수를 내놓는 함수여야 한다. \code{g}는 이
조건을 만족한다. \code{f(0)}은 \code{g}와 같고, \code{f(0)(0)}은 \code{g(0)}과
같다. 따라서 \code{f(0)(0)}의 결과는 0이다.

\begin{verbatim}
val h0 = f(0)

assert(h0(0) == 0)
\end{verbatim}

\code{f(0)}의 결과를 변수에 저장할 수 있다. \code{h0}가 \code{f(0)}의 결과를
나타내며 \code{Int => Int} 타입을 가진다. 어떤 변수가 나타내는 값이 함수라면 그
변수를 마치 함수처럼 사용할 수 있다. 따라서 \code{h0(0)}이라 적을 수 있으며 그 결과는 0이다.

\begin{verbatim}
val h1 = f
\end{verbatim}
\vspace{-1em}
\begin{mdframed}[hidealllines=true,backgroundcolor=gray!10,innerleftmargin=3pt,innerrightmargin=3pt,leftmargin=-3pt,rightmargin=-3pt]
\begin{verbatim}
         ^
error: missing argument list for method f

Unapplied methods are only converted to functions
when a function type is expected.

You can make this conversion explicit
by writing `f _` or `f(_)` instead of `f`.
\end{verbatim}
\vspace{-2em}
\begin{flushright}
\scriptsize\textsf{컴파일 시간 오류}
\end{flushright}
\end{mdframed}

한편 \code{f}와 같은 값을 나타내는 변수를 정의하려 하면 컴파일 시간 오류가
발생한다. 스칼라에서 \code{def}를 통해 정의한 함수는 그 자체로는 값이 아니다.
\code{f}는 함수의 이름일 뿐이지 어떤 값을 나타내는 변수가 아니다. 그렇기에
\code{f}을 초깃값으로 하여 변수를 정의할 수 없다. 위의 오류 문구가 이야기하듯이,
함수 이름을 값으로 바꾸려면 언더스코어가 필요하다.

\begin{verbatim}
val h1 = f _

assert(h1(0)(0) == 0)
\end{verbatim}

위 코드는 성공적으로 컴파일된다. \code{h1}의 타입은 \code{Int => (Int =>
Int)}이다. \code{=>}가 \term{오른쪽 결합을 하므로}{right-associative} 이 타입을
\code{Int => Int => Int}라 적어도 된다. \code{h1(0)(0)}은 \code{f(0)(0)}과
동일하며 그 결과는 0이다.

사실 위에서 \code{val h1 = f}라고 작성하기 전까지는 함수 이름을 값으로 사용하는
데 문제가 없었다. 그 이유는 함수 타입의 값이 필요한 곳에 함수 이름이 등장했을 때
스칼라 컴파일러가 자동으로 함수 이름을 값으로 바꾸기 때문이다. 따라서
\code{h1}의 타입을 함수 타입으로 지정해 주면 언더스코어를 사용하지 않아도
\code{f}를 \code{h1}에 대입할 수 있다. 다음 코드는 잘 컴파일된다.

\begin{verbatim}
val h1: Int => Int => Int = f
\end{verbatim}

프로그래머가 함수 이름을 값으로 사용하려는 대부분의 경우에 함수 타입이 필요한
자리에 함수 이름을 적는다. 이런 경우에는 언더스코어를 붙이거나 의식적으로 타입
표시를 달지 않아도 문제가 없다. 바로 앞선 예시처럼 함수 이름이 값으로 바뀌도록
강제하기 위해 언더스코어나 타입 표시를 추가해야 하는 경우는 드물다.

컴파일러는 함수 이름을 어떻게 값으로 바꿀까? 만약 \code{f}의 매개변수 타입이
\code{Int}이면, \code{f}에 대응되는 값은 \code{(x: Int) => f(x)}이다. 이러한
변환을 \term{에타 확대}{eta expansion}라 부른다. 여기서 \code{(x: Int) =>
f(x)}는 \code{f}와
같은 일을 하는 이름 없는 함수다. 이제 이름 없는 함수에 관해 알아보자.

\section{익명 함수}

함수형 프로그래밍에서는 어떤 함수가 딱 한 번, 인자나 결괏값으로만 사용되는
경우가 흔하다. 이런 경우 함수에 굳이 이름을 붙이는 것이 오히려 번거롭다. 함수에
이름을 붙이는 것은 그 함수를 여러 번 사용할 때 편하게 이름으로 부르기 위함이다.
함수의 동작은 매개변수와 몸통에 의해 결정되며, 이름은 동작에 영향을 주지 않는다.
그래서 함수형 언어는 이름을 붙이지 않고 함수를 정의할 방법을 제공한다. 이름을
붙이지 않고 정의한 함수를 익명 함수라 부른다.

스칼라에서 익명 함수는 다음과 같이 만든다.

\begin{verbatim}
([parameter name]: [parameter type], …) => [expression]
\end{verbatim}

\code{def}를 통해 정의한 함수와 마찬가지로, 익명 함수는 인자도 될 수 있고,
결괏값도 될 수 있고, 변수가 나타내는 값도 될 수 있다. 익명 함수를 어딘가에
저장하지 않고 바로 호출할 수도 있다.

\begin{verbatim}
def g(h: Int => Int): Int = h(0)
g((x: Int) => x)

def f(): Int => Int = (x: Int) => x
f()(0)

val h = (x: Int) => x
h(0)

((x: Int) => x)(0)
\end{verbatim}

위 코드는 익명 함수를 사용해서 이전 절에서 본 코드와 같은 일을 수행한다.

\code{def}를 통해 정의한 함수가 그렇듯이 익명 함수도 매개변수 타입 표시를 필요로 한다.
하지만 오직 한 번 사용할 익명 함수의 매개변수 타입을 매번 명시하는 것은 번거롭고
코드를 장황하게 만든다. 이를 해결하기 위해 함수 타입이 필요한 자리에 익명 함수가
등장하면 컴파일러가 매개변수 타입을 스스로 알아낸다.

\begin{verbatim}
def g(h: Int => Int): Int = h(0)
g(x => x)
\end{verbatim}

\code{g}의 매개변수 타입은 \code{Int => Int}이다. \code{Int => Int} 타입이
필요한 자리에 \code{x => x}가 사용되었으므로, 컴파일러가 \code{x}의 타입이
\code{Int}임을 알아낼 수 있다.

\begin{verbatim}
val h: Int => Int = x => x
\end{verbatim}

\code{h}에 타입 표시가 붙어 있다. 마찬가지로 컴파일러가 \code{x}의 타입이
\code{Int}임을 알아낸다.

\begin{verbatim}
val h = x => x
\end{verbatim}
\vspace{-1em}
\begin{mdframed}[hidealllines=true,backgroundcolor=gray!10,innerleftmargin=3pt,innerrightmargin=3pt,leftmargin=-3pt,rightmargin=-3pt]
\begin{verbatim}
         ^
error: missing parameter type
\end{verbatim}
\vspace{-2em}
\begin{flushright}
\scriptsize\textsf{컴파일 시간 오류}
\end{flushright}
\end{mdframed}

반면 이 경우에는 \code{x}의 타입을 컴파일러가 알아낼 정보가 없기에 컴파일 시간
오류가 발생한다.

익명 함수는 대개 인자로 사용된다. 익명 함수를 인자로 받을 함수의 매개변수 타입이
익명 함수의 타입을 결정하기 때문에 컴파일러가 익명 함수의 매개변수 타입을 알아낼
수 있다. 즉, 많은 경우 익명 함수의 매개변수 타입을 생략해도 된다. 하지만 항상
그렇지는 않다. 매개변수 타입을 생략하도 될지 아직 감이 잘 오지 않는다면 안전하게
매개변수 타입을 붙이는 편이 좋다.

스칼라는 언더스코어를 사용해서 익명 함수를 만드는 방법도 제공한다. 언더스코어를
사용하면 익명 함수를 간결하면서도 직관적으로 만들 수 있다. 단, 모든 익명 함수를
언더스코어를 사용해 만들 수 있는 것은 아니다. 특정 조건을 만족하는 익명 함수만
가능하다. 그 조건이란, 각 매개변수가 정확히 한 번씩 순서대로 몸통에 등장하는
것이다. 또한, \code{(x: Int) => x} 같은 항등 함수도 아니어야 한다. 이 조건을
만족한다면 함수 몸통에서 매개변수가 필요한 자리에 언더스코어를 써서 익명 함수를
만들 수 있다.

\begin{verbatim}
def g0(h: Int => Int): Int = h(0)
g0(_ + 1)

def g1(h: (Int, Int) => Int): Int = h(0, 0)
g1(_ + _)
\end{verbatim}

컴파일러는 \verb!_ + 1!을 \code{x => x + 1}로 변환한다. 비슷하게, \verb!_ + _!는
\code{(x, y) => x + y}가 된다. 컴파일러가 자동으로 언더스코어 개수만큼
매개변수를 만든 뒤 각 언더스코어를 매개변수로 대체하는 것이다. 이 변환
과정으로부터 앞서 설명한 조건이 필요한 이유를 자연스럽게 이해할 수 있다.

\begin{verbatim}
val h0 = (_: Int) + 1
val h1 = (_: Int) + (_: Int)
\end{verbatim}

언더스코어에 타입 표시를 붙일 수도 있다. 이렇게 해야만 하는 경우는 드물지만,
간혹 컴파일러가 매개변수 타입을 알아내지 못한다면 타입 표시를 직접 붙여야 한다.

이 변환은 언더스코어를 포함하는 가장 짧은 식을 대상으로 한다. 그래서 때로는 내가
원하는 익명 함수를 언더스코어를 사용해 정의하지 못할 수 있다.

\begin{verbatim}
def f(x: Int): Int = x
def g1(h: Int => Int): Int = h(0)
g1(f(_))
\end{verbatim}

의도한 대로 \verb!f(_)!는 \code{x => f(x)}로 바뀐다. 사실 \verb!g(f(_))!라
적기보다는 그냥 \code{g(f)}라 적는 편이 낫다.

\begin{verbatim}
g1(f(_ + 1))
\end{verbatim}
\vspace{-1em}
\begin{mdframed}[hidealllines=true,backgroundcolor=gray!10,innerleftmargin=3pt,innerrightmargin=3pt,leftmargin=-3pt,rightmargin=-3pt]
\begin{verbatim}
     ^
error: missing parameter type for expanded function
((<x$1: error>) => x$1.$plus(1))
\end{verbatim}
\vspace{-2em}
\begin{flushright}
\scriptsize\textsf{컴파일 시간 오류}
\end{flushright}
\end{mdframed}

한편, \verb!f(_ + 1)!은 \code{x => f(x + 1)} 대신 \code{f(x => x + 1)}로 바뀐다.
\verb!_ + 1!이 언더스코어를 포함하는 가장 짧은 식이기 때문이다. \code{f}는
함수가 아니라 정수를 인자로 받는 함수이므로 위 코드는 컴파일 시간 오류를
일으킨다.

\begin{verbatim}
def g2(h: (Int, Int) => Int): Int = h(0, 0)
g2(f(_) + _)
\end{verbatim}

\verb!f(_) + _!는 \code{(x, y) => f(x) + y}가 되어 컴파일이 성공한다.

\begin{verbatim}
g2(f(_ + 1) + _)
\end{verbatim}
\vspace{-1em}
\begin{mdframed}[hidealllines=true,backgroundcolor=gray!10,innerleftmargin=3pt,innerrightmargin=3pt,leftmargin=-3pt,rightmargin=-3pt]
\begin{verbatim}
              ^
error: missing parameter type for expanded function
((<x$2: error>) => f(((<x$1: error>) => x$1.$plus(1)))
  .<$plus: error>(x$2))
\end{verbatim}
\vspace{-2em}
\begin{flushright}
\scriptsize\textsf{컴파일 시간 오류}
\end{flushright}
\end{mdframed}

하지만 \verb!f(_ + 1) + _!는 \code{(x, y) => f(x + 1) + y} 대신 \code{y => f(x
=> x + 1) + y}가 된다.

언더스코어를 사용해 익명 함수를 만드는 것에 아직 익숙하지 않다면 자신이 작성한
익명 함수가 무슨 함수를 뜻하는 것인지 정확히 알기 어려울 수 있다. 그렇다면
언더스코어를 사용하지 않고 익명 함수를 만드는 편이 안전하다.

\section{클로저}

클로저는 만들어질 때의 \term{환경}{environment}을 기억하고 있는 함수를 말한다.
여기서 환경은 존재하는 각 변수의 값을 기록한 것이다. 클로저의 몸통은 클로저
내부에서 정의하지 않은 변수를 사용할 수 있다. 그 변수의 값은 환경에 존재한다.

\begin{verbatim}
def makeAdder(x: Int): Int => Int = {
  def adder(y: Int): Int = x + y
  adder
}
\end{verbatim}

\code{adder}의 정의를 보면 \code{x}를 사용하지만 정의하지는 않는다.

\begin{verbatim}
val add1 = makeAdder(1)
assert(add1(2) == 3)

val add2 = makeAdder(2)
assert(add2(2) == 4)
\end{verbatim}

\code{add1}과 \code{add2}는 같은 \code{adder} 함수를 나타내지만, \code{add1}은
주어진 정수보다 1 큰 정수를 결과로 내는 반면, \code{add2}는 주어진 정수보다 2 큰
정수를 결과로 낸다. \code{add1(2)}와 \code{add2(2)}의 결과가 각각 3과 4로 서로
다른 것이다. 이는 클로저가 자신이 만들어질 때의 환경을 가지고 있기 때문이다.
\code{add1}은 그냥 \code{adder}가 아니라 \code{(adder, x = 1)}과 같은 값이다.
마찬가지로, \code{add2}는 \code{(adder, x = 2)}이다. \code{add1}의 환경에는
\code{x}가 1이라는 정보가 저장되어 있기에 \code{add1(2)}의 결과가 3이고,
\code{add2}의 환경에는 \code{x}가 2라는 정보가 저장되어 있기에 \code{add2(2)}의
결과가 4이다.

\section{일급 함수와 리스트}

\refch{immutability}에서 정의한 함수를 일급 함수를 통해 어떻게 일반화할 수
있는지 알아보자.

\begin{verbatim}
def inc1(l: List[Int]): List[Int] = l match {
  case Nil => Nil
  case h :: t => h + 1 :: inc1(t)
}

def square(l: List[Int]): List[Int] = l match {
  case Nil => Nil
  case h :: t => h * h :: square(t)
}
\end{verbatim}

\code{inc1}은 주어진 리스트의 각 원소를 1씩 키우고, \code{square}는 각 원소를
제곱한다. 두 함수는 굉장히 비슷하다. 두 함수의 이름을 g로 통일하면 더
비슷해진다.

\begin{verbatim}
def g(l: List[Int]): List[Int] = l match {
  case Nil => Nil
  case h :: t => h + 1 :: g(t)
}

def g(l: List[Int]): List[Int] = l match {
  case Nil => Nil
  case h :: t => h * h :: g(t)
}
\end{verbatim}

유일한 차이점은 \code{::}의 왼쪽에 \code{h + 1}과 \code{h * h} 중 무엇이 오냐는
것이다. 매개변수 하나만 추가하면 두 함수를 완전히 같게 만들 수 있다.

\begin{verbatim}
def g(l: List[Int], f: Int => Int): List[Int] = l match {
  case Nil => Nil
  case h :: t => f(h) :: g(t, f)
}
g(l, h => h + 1)

def g(l: List[Int], f: Int => Int): List[Int] = l match {
  case Nil => Nil
  case h :: t => f(h) :: g(t, f)
}
g(l, h => h * h)
\end{verbatim}

이 함수를 \code{map}이라 부른다. 각 원소와 대응(map)되는 값들로 이루어진
리스트를 만들기 때문이다.

\begin{verbatim}
def map(l: List[Int], f: Int => Int): List[Int] = l match {
  case Nil => Nil
  case h :: t => f(h) :: map(t, f)
}
\end{verbatim}

\code{inc1}과 \code{square}를 \code{map}을 사용해 다시 구현할 수 있다.

\begin{verbatim}
def inc1(l: List[Int]): List[Int] = map(l, h => h + 1)
def square(l: List[Int]): List[Int] = map(l, h => h * h)
\end{verbatim}

\code{inc1}을 구현할 때는 언더스코어를 사용할 수 있다.

\begin{verbatim}
def inc1(l: List[Int]): List[Int] = map(l, _ + 1)
\end{verbatim}

이번에는 \code{odd}와 \code{positive}를 비교해 보자.

\begin{verbatim}
def odd(l: List[Int]): List[Int] = l match {
  case Nil => Nil
  case h :: t =>
    if (h % 2 != 0)
      h :: odd(t)
    else
      odd(t)
}

def positive(l: List[Int]): List[Int] = l match {
  case Nil => Nil
  case h :: t =>
    if (h > 0)
      h :: positive(t)
    else
      positive(t)
}
\end{verbatim}

이 두 함수도 서로 비슷하다. 함수 이름을 통일하고 매개변수를 추가하면 두 함수를
같게 만들 수 있다.

\begin{verbatim}
def filter(l: List[Int], f: Int => Boolean): List[Int] = l match {
  case Nil => Nil
  case h :: t =>
    if (f(h))
      h :: filter(t, f)
    else
      filter(t, f)
}
\end{verbatim}

이 함수를 \code{filter}라 부른다. 조건을 만족하지 않는 원소를 걸러 내기(filter) 때문이다.

\code{odd}와 \code{positive}를 \code{filter}를 사용해 다시 구현할 수 있다.

\begin{verbatim}
def odd(l: List[Int]): List[Int] =
  filter(l, h => h % 2 != 0)
def positive(l: List[Int]): List[Int] =
  filter(l, h => h > 0)
\end{verbatim}

언더스코어를 사용하면 더 간결하다.

\begin{verbatim}
def odd(l: List[Int]): List[Int] = filter(l, _ % 2 != 0)
def positive(l: List[Int]): List[Int] = filter(l, _ > 0)
\end{verbatim}

이제 꼬리 재귀를 사용하지 않은 \code{sum}과 \code{product}를 보자.

\begin{verbatim}
def sum(l: List[Int]): Int = l match {
  case Nil => 0
  case h :: t => h + sum(t)
}

def product(l: List[Int]): Int = l match {
  case Nil => 1
  case h :: t => h * product(t)
}
\end{verbatim}

함수 이름을 \code{g}로 같게 만들고 나면 차이점이 둘 남는다. 하나는 \code{Nil}
케이스가 계산하는 식 0과 1이고, 다른 하나는 \code{h :: t} 케이스가 계산하는 식
\code{h + g(t)}와 \code{h * g(t)}이다. 그러니 두 함수를 완전히 같게 만들려면
매개변수를 둘 추가해야 한다. 하나는 \code{Nil} 케이스를 위한 초깃값, 다른 하나는
\code{h와 g(t)}를 인자로 받아 계산하는 함수다.

\begin{verbatim}
def foldRight(
  l: List[Int],
  n: Int,
  f: (Int, Int) => Int
): Int = l match {
  case Nil => n
  case h :: t => f(h, foldRight(t, n, f))
}
\end{verbatim}

이 함수를 \code{foldRight}라 부른다. 초깃값을 리스트의 오른쪽(right) 끝에 붙인 뒤 그쪽부터 리스트를 접기(fold) 때문이다.

\code{sum}과 \code{product}를 \code{foldRight}를 사용해 다시 구현할 수 있다.

\begin{verbatim}
def sum(l: List[Int]): Int =
  foldRight(l, 0, (h, gt) => h + gt)
def product(l: List[Int]): Int =
  foldRight(l, 1, (h, gt) => h * gt)
\end{verbatim}

언더스코어를 사용하면 더 간결하다.

\begin{verbatim}
def sum(l: List[Int]): Int = foldRight(l, 0, _ + _)
def product(l: List[Int]): Int = foldRight(l, 1, _ * _)
\end{verbatim}

\code{foldRight}가 하는 일을 직관적으로 설명하면 다음과 같다.

\begin{verbatim}
  foldRight(List(a, b, .., y, z), n, f)
= f(a, f(b, .. f(y, f(z, n)) .. ))

  foldRight(List(1, 2, 3), 0, add)
= add(1, add(2, add(3, 0)))

  foldRight(List(1, 2, 3), 1, mul)
= mul(1, mul(2, mul(3, 1)))
\end{verbatim}

이번에는 꼬리 재귀를 사용한 \code{sum}과 \code{product}를 볼 차례다.

\begin{verbatim}
def sum(l: List[Int]): Int = {
  def aux(l: List[Int], inter: Int): Int = l match {
    case Nil => inter
    case h :: t => aux(t, inter + h)
  }
  aux(l, 0)
}

def product(l: List[Int]): Int = {
  def aux(l: List[Int], inter: Int): Int = l match {
    case Nil => inter
    case h :: t => aux(t, inter * h)
  }
  aux(l, 1)
}
\end{verbatim}

이번에도 이름을 통일하고 나면 차이점이 두 개다. \code{h :: t} 케이스에
\code{aux}의 두 번째 인자로 사용되는 식과 \code{aux} 바깥에서 \code{aux}를 부를
때 두 번째 인자로 사용하는 식이다. 따라서 매개변수를 두 개 추가해야 한다.

\begin{verbatim}
def foldLeft(
  l: List[Int],
  n: Int,
  f: (Int, Int) => Int
): Int = {
  def aux(l: List[Int], inter: Int): Int = l match {
    case Nil => inter
    case h :: t => aux(t, f(inter, h))
  }
  aux(l, n)
}
\end{verbatim}

이 함수를 \code{foldLeft}라 부른다. \code{foldRight}와는 다른 일을 하는 함수다.
\code{foldRight}는 리스트를 오른쪽부터 접는다. 반면, 이 함수는 초깃값을 리스트의
왼쪽(left) 끝에 붙인 뒤 그쪽부터 리스트를 접는다(fold). \code{foldList}가 하는
일을 다음과 같이 직관적으로 이해할 수 있다.

\begin{verbatim}
  foldLeft(List(a, b, .., y, z), n, f)
= f(f( .. f(f(n, a), b), .. , y), z)

  foldLeft(List(1, 2, 3), 0, add)
= add(add(add(0, 1), 2), 3)

  foldLeft(List(1, 2, 3), 1, mul)
= mul(mul(mul(1, 1), 2), 3)
\end{verbatim}

사실 \code{sum}과 \code{product}를 구현할 때는 어느 쪽부터 접는지 별로 중요하지
않다. 다음과 같이 \code{sum}과 \code{product}를 \code{foldLeft}를 사용해서 다시
구현할 수도 있다.

\begin{verbatim}
def sum(l: List[Int]): Int = foldLeft(l, 0, _ + _)
def product(l: List[Int]): Int = foldLeft(l, 1, _ * _)
\end{verbatim}

하지만 어떤 함수의 경우에는 방향이 중요하다. 0에서 9 사이의 숫자로 구성된
리스트를 받아 리스트의 원소들을 이어 붙여 만들어지는 십진수를 계산하는 함수를
만들어 보자. \code{foldLeft}를 사용하면 된다.

\begin{verbatim}
def digitToDecimal(l: List[Int]) =
  foldLeft(l, 0, _ * 10 + _)

  foldLeft(List(1, 2, 3), 0, f)
= f(f(f(0, 1), 2), 3)
= ((0 * 10 + 1) * 10 + 2) * 10 + 3
= (1 * 10 + 2) * 10 + 3
= 12 * 10 + 3
= 123
\end{verbatim}

\code{foldRight}를 사용하면 전혀 다른 함수가 만들어진다.

\begin{verbatim}
def digitToDecimal(l: List[Int]) =
  foldRight(l, 0, _ * 10 + _)

  foldRight(List(1, 2, 3), 0, f)
= f(1, f(2, f(3, 0)))
= 1 * 10 + (2 * 10 + (3 * 10 + 0))
= 1 * 10 + (2 * 10 + 30)
= 1 * 10 + 50
= 60
\end{verbatim}

\code{map}, \code{filter}, \code{foldRight}, \code{foldLeft}는 매우 유용하다.
이들 함수를 사용하면 리스트를 처리하는 다양한 작업을 간결하게 구현할 수 있다.
사실 그 유용함 때문에 표준 라이브러리의 리스트가 이미 \code{map}, \code{filter},
\code{foldRight}, \code{foldLeft}를 메서드로 제공한다. 이들을 함수로 직접 구현할
필요가 없는 것이다.

\code{map} 메서드를 사용하려면 \code{map(l, f)} 대신 \code{l.map(f)}라 적으면 된다.

\begin{verbatim}
def inc1(l: List[Int]): List[Int] = l.map(_ + 1)
def square(l: List[Int]): List[Int] = l.map(h => h * h)
\end{verbatim}

\code{filter(l, f)} 대신 \code{l.filter(f)}라 적는다.

\begin{verbatim}
def odd(l: List[Int]): List[Int] = l.filter(_ % 2 != 0)
def positive(l: List[Int]): List[Int] = l.filter(_ > 0)
\end{verbatim}

\code{foldRight(l, n, f)} 대신 \code{l.foldRight(n)(f)}라 적는다.

\begin{verbatim}
def sum(l: List[Int]): Int = l.foldRight(0)(_ + _)
def product(l: List[Int]): Int = l.foldRight(1)(_ * _)
\end{verbatim}

\code{foldLeft(l, n, f)} 대신 \code{l.foldLeft(n)(f)}라 적는다.

\begin{verbatim}
def sum(l: List[Int]): Int = l.foldLeft(0)(_ + _)
def product(l: List[Int]): Int = l.foldLeft(1)(_ * _)
def digitToDecimal(l: List[Int]) = l.foldLeft(0)(_ * 10 + _)
\end{verbatim}

우리가 정의한 함수와 달리 표준 라이브러리의 메서드는
\term{다형}{polymorhpic}이다. 이는 메서드가 받을 수 있는 인자의 타입이 하나로
고정되지 않음을 뜻한다. 가령 우리의 \code{map} 함수는 정수의 리스트만 인자로
받을 수 있으며 결괏값도 정수의 리스트만 가능하다. 다른 타입의 리스트를 다루려면
새로운 \code{map} 함수를 만들어야 한다. 하지만 \code{map} 메서드는 무슨 타입의
리스트든 처리할 수 있다.

\begin{verbatim}
case class Student(name: String, height: Int)

def heights(l: List[Student]): List[Int] = l.map(_.height)
\end{verbatim}

이 밖에도 표준 라이브러리에는 리스트를 위한 다양한 메서드가 정의되어
있다.\footnote{\url{https://www.scala-lang.org/api/current/scala/collection/immutable/List.html}}

\section{\code{for} 반복문}

스칼라는 \code{for} 반복문을 제공한다. 스칼라의 \code{for} 반복문은 식으로서
하나의 값으로 계산된다. \code{while}은 수정할 수 있는 변수나 객체와 함께
사용해야만 의미 있지만, \code{for}는 함수형 프로그래밍의 방법으로 이해하기 쉬운
코드를 작성하는 데 큰 도움을 준다.

\code{for} 식은 다음과 같이 쓴다.

\begin{verbatim}
for ([name] <- [expression])
  yield [expression]
\end{verbatim}

첫 번째 식의 결과는 리스트 같은 컬렉션이어야 한다. \code{for} 식의 결과는 매
반복에서 두 번째 식을 계산한 결과를 모아 놓은 컬렉션이다. 즉, \code{for} 식은
\code{map} 메서드와 같은 일을 한다.

\begin{verbatim}
val l = for (n <- List(0, 1, 2)) yield n * n
assert(l == List(0, 1, 4))
\end{verbatim}

\code{for} 식은 식이 필요한 어느 곳에서든 사용될 수 있다.

\begin{verbatim}
def square(l: List[Int]): List[Int] =
  for (n <- l)
    yield n * n
\end{verbatim}

스칼라에서 \code{for}는 그저 \term{문법 설탕}{syntactic sugar}이다. \code{for}의
동작이 따로 정의되어 있는 것이 아니라, 컴파일러가 \code{for}를 사용한 코드를
컬렉션의 메서드와 익명 함수를 사용하는 코드로 변환한다. 컴파일러는 위 함수를
다음과 같이 \code{map}을 사용하는 함수로 변환한다.

\begin{verbatim}
def square(l: List[Int]): List[Int] =
  l.map(n => n * n)
\end{verbatim}

이는 \code{for} 식이 강력한 이유다. 꼭 리스트가 아니더라도 \code{map} 메서드를
정의하고 있다면 어느 타입이든 \code{for}를 사용할 수 있다.

\code{for} 식은 \code{filter}도 대신할 수 있다.

\begin{verbatim}
def positive(l: List[Int]): List[Int] =
  for (n <- l if n > 0)
    yield n
\end{verbatim}

주어진 조건을 만족하지 않는 원소는 반복 과정에서 생략된다.

더 나아가, \code{map}과 \code{filter}를 함께 사용하는 코드 역시 \code{for}를
통해 간결하게 표현할 수 있다. 학생의 리스트를 인자로 받아 키가 170보다 큰 학생의
이름을 담은 리스트를 내놓는 함수를 만들어 보자. \code{map}과 \code{filter}를
사용한다면 다음과 같이 구현할 수 있다.

\begin{verbatim}
def tall(l: List[Student]): List[String] =
  l.filter(_.height > 170).map(_.name)
\end{verbatim}

\code{for}를 사용하면 다음과 같다.

\begin{verbatim}
def tall(l: List[Student]): List[String] =
  for (s <- l if s.height > 170)
    yield s.name
\end{verbatim}

\section{예제}

\begin{exercise}
\labex{functions-incby}

정수 리스트와 정수 하나를 인자로 받아 리스트의 각 원소를 주어진 정수만큼
증가시키는 함수 \code{incBy}를 \code{map} 메서드를 사용해 구현하라.

\begin{verbatim}
def incBy(l: List[Int], n: Int): List[Int] = ???
\end{verbatim}

\end{exercise}

\begin{exercise}
\labex{functions-gt}

정수 리스트와 정수 하나를 인자로 받아 주어진 정수보다 큰 원소만 남기는 함수
\code{gt}를 \code{filter} 메서드를 사용해 구현하라.

\begin{verbatim}
def gt(l: List[Int], n: Int): List[Int] = ???
\end{verbatim}

\end{exercise}

\begin{exercise}
\labex{functions-append}

정수 리스트와 정수 하나를 인자로 받아 리스트의 끝에 정수를 붙이는 함수
\code{append}를 \code{foldRight} 메서드를 사용해 구현하라.

\begin{verbatim}
def append(l: List[Int], n: Int): List[Int] = ???
\end{verbatim}

\end{exercise}

\begin{exercise}
\labex{functions-reverse}

정수 리스트를 인자로 받아 순서가 거꾸로된 리스트를 만드는 함수 \code{reverse}를
\code{foldLeft} 메서드를 사용해 구현하라.

\begin{verbatim}
def reverse(l: List[Int]): List[Int] = ???
\end{verbatim}

\end{exercise}
