\chapter{문법과 의미}
\labch{syntax-and-semantics}

\renewcommand{\Lang}{\code{AE}}

이 장은 \term{문법}{syntax}과 \term{의미}{semantics}를 다룬다.

프로그래밍 언어의 문법은 언어의 겉모습을 결정한다. 문법은 \term{구체적
문법}{concrete syntax}과 \term{요약 문법}{abstract syntax}으로 다시 나눌 수
있다. 구체적 문법은 프로그램을 문자열로 취급하는 반면, 요약 문법은 프로그램의
구조를 \term{나무}{tree} 형태로 표현한다. 그리고 \term{파싱}{parsing}이 구체적
문법과 요약 문법 사이를 이어 준다. 파싱을 통해 문자열이 나무로 바뀌는 것이다. 이
장에서 구체적 문법, 요약 문법, 파싱에 대해 알아본다.

프로그래밍 언어의 의미는 그 언어로 작성된 각 프로그램의 동작을 결정한다. 이
장에서 언어의 의미를 정의하는 방법을 알아본다. 더 나아가 \term{문법
설탕}{syntactic sugar}이 무엇인지도 알아본다.

\section{구체적 문법}

프로그램을 작성하는 것은 곧 문자열을 쓰는 것이다. 이때 어떤 문자열은
프로그램이고, 어떤 문자열은 프로그램이 아니다. 예를 들어 다음 코드는 스칼라
프로그램이다.

\begin{verbatim}
println()
\end{verbatim}

한편 다음 코드는 스칼라 프로그램이 아니다.

\begin{verbatim}
println(
\end{verbatim}

\term{구체적 문법}{concrete syntax}은 특정 문자열이 프로그램인지 아닌지 정한다. 스칼라의 구체적 문법에
따라 \code{println()}은 프로그램이지만 닫는 괄호가 빠진 \code{println(}은
프로그램이 아닌 것이다. 구체적 문법을 모르면 특정 문자열이 프로그램인지 아닌지
알 수 없다. 그러니 구체적 문법은 프로그래밍 언어의 필수 요소다. 이 절에서 구체적
문법을 엄밀하게 정의하는 방법을 알아본다.

프로그램이 문자열로 표현되므로 가장 먼저 할 일은 문자열을 정의하는 것이다.
문자열은 유한한 개수의 \term{문자}{character}를 나열한 것이다. 문자의 정의는 언어마다
다르다. 어떤 언어에서는 \term{아스키}{ASCII} 코드로 표현되는 것들만이 문자다. 어떤
언어에서는 \term{유니코드}{Unicode}로 표현되는 모든 것이 문자다. 우리는 모든 언어에 관해
일반적으로 이야기하고 싶으므로, 문자의 집합을 특정한 집합으로 고정하지 않겠다.
대신 문자의 집합이 주어진다고 가정한다. 그 집합에 정확히 어떤 문자들이
속하는지는 중요하지 않다. $C$는 문자의 집합을 나타낸다.

\[C = \text{모든 문자의 집합} = \{ c\ |\ c\text{는 문자} \}\]

이제 문자는 $C$의 원소다.

문자를 정의했으니 이를 바탕으로 문자열을 정의할 수 있다. $S$는 문자열의
집합이다.

\[S = \text{모든 문자열의 집합} = \{ \code{"}c_1\cdots c_n\code{"}\ |\
c_1,\cdots,c_n\in C \}\]

문자열은 $S$의 원소로, 0개 이상의 문자를 나열한 것이다. 예를 들어,
$\code{'a'},\code{'b'}\in C$이면 $\code{"aba"}\in S$이다.

모든 문자열이 프로그램인 것은 아니므로 프로그램의 정의는 문자열의 정의와 달라야
한다. 앞서 본 것처럼 \code{"println()"}과 \code{"println("} 모두 문자열이지만
하나는 프로그램이고 하나는 프로그램이 아니다.

구체적 문법을 정의한다는 것은 무슨 문자열이 프로그램인지 정의하는 것이다. 다시
말해, 구체적 문법을 정의하는 것은 프로그램의 집합을 정의하는 것이다. $P$는
프로그램의 집합이다.

\[P = \text{모든 프로그램의 집합} = \{ p\ |\ p\text{는 프로그램} \}\]

프로그램은 $P$의 원소다. $P$에는 한 가지 조건이 붙는다. 모든 프로그램은
문자열이므로 $P$가 $S$의 \term{부분 집합}{subset}이어야 한다는 것이다.

\[P\subseteq S\]

각 언어의 구체적 문법은 $S$의 부분 집합인 $P$를 정의한다. 언어마다 $P$가 서로
다르다. 예를 들어, $P$가 스칼라의 구체적 문법에 의해 정의되었을 때
$\code{"println()"} \in P$이고 $\code{"println("}\not\in P$라고 말해도 된다.
$P$가 다른 언어의 구체적 문법에 의해 정의되었다면 \code{"println()"}이 $P$의
원소가 아닐 수 있다.

언어의 구체적 문법을 정의하려면 집합 $P$를 정의해야 한다. 문제는 $P$가 대개
\term{무한 집합}{inifinte set}이라는 점이다. 각 언어로 작성될 수 있는 프로그램의
개수는 무한하다. 무한 집합의 모든 원소를 나열할 수 없기에 무한 집합을 정의하기는
어렵다. 구체적 문법을 정의하려면 무한 집합을 정의할 방법이 필요하다.

구체적 문법을 정의하는 가장 일반적인 방법은 \term{배커스-나우르
표기법}{Backus-Naur form}\terme{BNF}이다. BNF를 사용하여 직관적으로 문자열로 이루어진
집합을 정의할 수 있다. BNF는 정의한 집합의 각 원소를 만드는 방법을 알려 준다.
BNF를 사용해 구체적 문법을 정의하기에 앞서 BNF를 사용하는 데 필요한 개념을 먼저
알아보자.

BNF에서는 \term{터미널}{terminal}, \term{논터미널}{nonterminal},
\term{식}{expression}, 이렇게 세 가지 개념을 사용한다. 각 터미널은 문자열이다.
가령 \code{"0"}과 \code{"ab"}는 터미널이다. 논터미널은 \code{<} \code{>} 사이에
이름을 적어 표기하며 각 논터미널은 문자열의 집합을 나타낸다. 예를 들어
\code{<digit>}은 논터미널이며 집합
$\{\code{"0"},\code{"1"},\code{"2"},\code{"3"},\code{"4"},\code{"5"},\code{"6"},\code{"7"},\code{"8"},\code{"9"}\}$를
나타내도록 만들 수 있다. 식은 한 개 이상의 터미널 혹은 논터미널을 나열한 것이다.
다음은 식의 예시다.

\begin{itemize}
  \item \code{"abc"} (터미널 하나)

  \item \code{"0" "1"} (터미널 여럿)

  \item \code{<digit>} (논터미널 하나)
  \item \code{<digit> <number>} (논터미널 여럿)
  \item \code{"-" <number>} (터미널 하나와 논터미널 하나)
\end{itemize}

각 식은 문자열의 집합을 나타내며, 그 집합은 식을 구성하는 각 부품이 나타내는
문자열을 이어 붙여 만든 문자열로 이루어진다. 가령 식 \code{"0" "1"}에서
\code{"0"}은 \code{"0"}을 나타내고 \code{"1"}은 \code{"1"}을 나타내므로 식이
나타내는 집합은 $\{\code{"01"}\}$이다. 또, 만약 \code{<digit>}이 집합
$\{\code{"0"},\code{"1"},\code{"2"},\code{"3"},\code{"4"},\code{"5"},\code{"6"},\code{"7"},\code{"8"},\code{"9"}\}$를
나타낸다면 \code{"0" <digit>}이라는 식은
$\{\code{"00"},\code{"01"},\code{"02"},\code{"03"},\code{"04"},\code{"05"},\code{"06"},\code{"07"},\code{"08"},\code{"09"}\}$를
나타낸다.

이제 어떻게 해야 각 논터미널이 의미하는 집합을 정해 줄 수 있는지 보자. BNF는
다음 형태를 통해 논터미널의 뜻을 정의할 수 있도록 한다.

\begin{verbatim}
[nonterminal] ::= [expression] | [expression] | …
\end{verbatim}

우변의 세로줄은 여러 식을 구분하는 역할을 한다. 우변의 식들이 좌변의 논터미널의
뜻을 결정한다. 식이 나타내는 문자열 집합을 모두 합한 것이 곧 논터미널이 나타내는
집합이다. 예를 들면 다음과 같이 써서 \code{<digit>}이 집합
$\{\code{"0"},\code{"1"},\code{"2"},\code{"3"},\code{"4"},\code{"5"},\code{"6"},\code{"7"},\code{"8"},\code{"9"}\}$를
나타내도록 만들 수 있다.

\begin{verbatim}
<digit> ::= "0" | "1" | "2" | "3" | "4" | "5" | "6" | "7" | "8" | "9"
\end{verbatim}

BNF의 구체적인 사용 예시를 보기 위해 작은 언어인 \Lang의 구체적 문법을 정의해
보겠다. \Lang는 \term{산술식}{arithmetic expression}을 뜻한다. 이 언어는 십진법으로
표현된 정수를 더하고 빼는 기능만 제공한다.

\Lang~프로그램은 십진수를 표현할 수 있어야 한다. 즉, 다음 문자열은 \Lang~프로그램이다.

\begin{itemize}
  \item \code{"0"}
  \item \code{"1"}
  \item \code{"-10"}
  \item \code{"42"}
\end{itemize}

더 나아가, 프로그램이 덧셈과 뺄셈을 표현할 수 있어야 한다. 그러니 다음 문자열
역시 프로그램이다.

\begin{itemize}
  \item \code{"0+1"}
  \item \code{"-2-1"}
  \item \code{"1+-3+42"}
  \item \code{"4-3+2-1"}
\end{itemize}

우선 십진수를 표현하는 모든 문자열의 집합을 BNF로 정의하자. 다음과 같이 정의할 수 있다.

\begin{verbatim}
<digit>  ::= "0" | "1" | "2" | "3" | "4"
           | "5" | "6" | "7" | "8" | "9"
<nat>    ::= <digit> | <digit> <nat>
<number> ::= <nat> | "-" <nat>
\end{verbatim}

이렇게 정의하면 \code{<digit>}이 $\{$\code{"0"}, \code{"1"}, \code{"2"},
\code{"3"}, \code{"4"}, \code{"5"}, \code{"6"}, \code{"7"}, \code{"8"},
\code{"9"}$\}$를 나타낸다는 사실은 이미 보았다. \code{<digit>}이 \code{<nat>}의
원소를 만드는 한 가지 방법이므로, \code{<digit>}이 나타내는 각 문자열은 곧
\code{<nat>}이 나타내는 문자열이기도 하다. 그러니 $\{$\code{"0"}, \code{"1"},
\code{"2"}, \code{"3"}, \code{"4"}, \code{"5"}, \code{"6"}, \code{"7"},
\code{"8"}, \code{"9"}$\}$는 \code{<nat>}이 나타내는 집합의 일부다. 동시에
\code{<digit> <nat>} 역시 \code{<nat>}의 원소를 만드는 방법이다. 즉,
\code{<digit>}과 \code{<nat>}에서 문자열을 하나씩 골라서 이어 붙임으로써
\code{<nat>}의 새 원소를 만들 수 있다. 가령 \code{<digit>}에 \code{"1"}이 속하고
\code{<nat>}에 \code{"0"}이 속하니 \code{"10"}도 \code{<nat>}에 속한다. 이
과정을 반복하면 무수히 많은 문자열이 탄생한다. \code{"1"}과 \code{"10"}을 이으면
\code{"110"}이 나오고, \code{"1"}과 \code{"110"}을 이으면 \code{"1110"}이
나온다. 최종적으로는 \code{<nat>}이 \code{'0'}부터 \code{'9'}까지의 문자로
이루어진 모든 문자열의 집합을 나타냄을 알 수 있다. 다시 말해 \code{0}을 자연수에
포함할 때 \code{<nat>}은 십진 자연수를 나타내는 모든 문자열을 포함한다.

\code{<number>}가 나타내는 집합은 쉽게 알 수 있다. \code{<nat>}이
\code{<number>}의 원소를 만드는 한 가지 방법이므로 \code{<nat>}이 나타내는
집합이 \code{<number>}가 나타내는 집합의 일부다. 식 \code{"-" <nat>}은
\code{<number>}의 원소를 만드는 다른 방법이다. 이는 \code{"-"}와 \code{<nat>}의
원소를 이어 붙여 만든 문자열 역시 \code{<number>}에 속함을 뜻한다. 그러니
\code{<number>}는 십진 정수를 나타내는 모든 문자열을 포함한다.

이제 덧셈과 뺄셈만 추가하면 구체적 문법의 정의가 끝난다.

...

전과 같은 방법으로 \code{<expr>}가 나타내는 집합이 무엇인지 알아낼 수 있다. 바로
십진 정수 및 그 덧셈과 뺄셈으로 이루어진 모든 산술식을 나타내는 문자열의
집합이다. 따라서 \code{<expr>}가 \Lang의 $P$를 정의한다고 말할 수 있으며,
\Lang의 구체적 문법이 드디어 정의되었다.

\section{요약 문법}

문법을 구체적 문법만으로 정의하는 것은 언어 사용자와 설계자 모두의 관점에서
문제를 일으킨다. 개발자는 대개 여러 언어를 배우며, 구체적 문법은 언어마다 큰
차이가 있다. 두 정수를 인자로 받아 그 합을 계산하는 함수를 생각해 보자. 네 개의
서로 다른 언어로 다음과 같이 함수를 작성할 수 있다.

\begin{itemize}
  \item 파이썬

\begin{verbatim}
def add(n, m):
    return n + m
\end{verbatim}

  \item 자바스크립트

\begin{verbatim}
function add(n, m) {
    return n + m;
}
\end{verbatim}

  \item 래킷

\begin{verbatim}
(define (add n m) (+ n m))
\end{verbatim}

  \item 오캐멀

\begin{verbatim}
let add n m = n + m
\end{verbatim}
\end{itemize}

같은 함수를 정의하는 데도 코드가 매우 다르게 생겼다. 함수를 정의할 때 사용하는
키워드가 파이썬에서는 \code{def}, 자바스크립트에서는 \code{function}, 래킷에서는
\code{define}, 오캐멀에서는 \code{let}이다. 또, 파이썬과 자바스크립트는
매개변수를 쓸 때 괄호와 쉼표를 요구하지만 래킷과 오캐멀은 그렇지 않다.
자바스크립트는 다른 세 언어와 달리 함수 몸통을 중괄호 안에 넣는다. 래킷은
\code{+}를 \term{전위 연산자}{prefix operator}로 취급하지만 나머지는 \code{+}를
중위 연산자로 취급한다. 이런 차이는 모두 구체적 문법의 차이다. 개발자들은 종종
서로 다른 구체적 문법 때문에 여러 언어를 배우는 데 어려움을 겪는다.

하지만 사실 각 언어로 구현한 함수의 구조는 완전히 같다. 어느 언어에서든 함수
정의는 함수 이름, 매개변수, 몸통 식으로 이루어진다. 위 예시에서 정의한 함수는
이름이 \code{add}이고 매개변수가 \code{n}과 \code{m}이고 몸통이 \code{n}과
\code{m}을 더하는 식이다. 어느 언어에서든 덧셈식은 두 \term{피연산자}{operand}로
구성되며, 몸통은 \code{n}과 \code{m}을 덧셈의 피연산자로 사용한다.

그러니 개발자는 여러 언어를 쉽게 배우기 위해 프로그램을 나타내는 문자열 그
자체보다는 프로그램의 구조에 집중해야 한다. 언어가 바뀔 때 문자열이 크게
달라지는 것과 달리 구조는 변하지 않는다.

한편 언어 설계자는 구체적 문법이 무시하고 싶은 정보까지 구체적으로 다루기 때문에
불편을 느낀다. 예를 들어 \code{"2+1"}과 \code{"02+001"} 모두 \Lang~프로그램이다.
두 프로그램은 문자열로서는 서로 다르지만 $2+1$이라는 같은 산술식을 표현한다.
\Lang의 설계자들이 산술식을 계산하는 과정을 구현할 때 $2+1$과 $2-1$의 차이는
중요해도 \code{"2+1"}과 \code{"02+001"}의 차이는 중요하지 않다. 설계자는
프로그램의 구조에 집중하고 싶지 문자열 그 자체에 집중하고 싶지는 않은 것이다.

이처럼 개발자와 언어 설계자 모두에게 구체적 문법은 문제를 일으킨다. 사람들은
구조에만 집중하고 싶어 하지만 구체적 문법은 문자열만 열심히 표현하고 구조는 잘
요약하여 표현하지는 못한다. 물론 구체적 문법이라는 개념을 없앨 수는 없다. 누구나
프로그램은 문자열로 작성하며, 이 과정에서 구체적 문법이 반드시 필요하다.
그러면서도 우리는 문자열의 차이에 영향받지 않고 프로그램의 구조를 기술할 방법이
필요하다. 이를 위해 탄생한 개념이 요약 문법이다. 구체적 문법과 요약 문법은
서로를 보완하여 함께 언어의 문법을 정의한다.

\term{요약 문법}{abstract syntax}은 프로그램의 구조를 나무로 표현한다. 각 프로그램은 여러 부품으로
이루어지며, 각 부품은 또 다시 여러 부품으로 이루어진다. 나무를 사용해 이런
구조를 엄밀하게 표현할 수 있다. 각 부품은 하나의 나무로, 그 뿌리는 해당 부품의
종류를 나타내고 자식 나무는 구성 부품을 나타낸다.

함수 \code{add}를 나무로 표현해 보자. 함수 정의는 이름 \code{add}, 첫 매개변수
\code{n}, 둘째 매개변수 \code{m}, 몸통 식, 이렇게 총 네 개의 부품으로 구성된다.
따라서 다음 나무를 그릴 수 있다.

\begin{center}
\begin{tikzpicture}[
    level distance=1.2cm,
    level 1/.style={sibling distance=1.2cm},
]
  \node [arn_u] {\tiny FunDef}
  child{
    node [arn_u] {\code{add}}
  }
  child{
    node [arn_u] {\code{n}}
  }
  child{
    node [arn_u] {\code{m}}
  }
  child{
    node [arn_w] {}
  };
\end{tikzpicture}
\end{center}

나무의 뿌리는 \term{심벌}{symbol} FunDef로, 이 나무가 함수 정의를 표현한다는
사실을 알려 준다. 나무는 \code{add}, \code{n}, \code{m}, 몸통 식을 자식으로
가지며, 아직 몸통 식에 해당하는 나무는 그리지 않았다.

몸통 식은 덧셈식으로, 덧셈의 피연산자를 부품으로 가진다.

\begin{center}
\begin{tikzpicture}[
    level distance=1.2cm,
    level 1/.style={sibling distance=1.2cm},
    level 2/.style={sibling distance=1.2cm}
]
  \node [arn_u] {\tiny FunDef}
  child{
    node [arn_u] {\code{add}}
  }
  child{
    node [arn_u] {\code{n}}
  }
  child{
    node [arn_u] {\code{m}}
  }
  child{
    node [arn_u] {\small Add}
    child{
      node [arn_w] {}
    }
    child{
      node [arn_w] {}
    }
  };
\end{tikzpicture}
\end{center}

덧셈식을 나타내는 나무이므로 뿌리가 Add라는 심벌이다. 피연산자가 두 개이므로
자식이 둘이다.

첫 연산자는 \code{n}이다.

\begin{center}
\begin{tikzpicture}[
    level distance=1.2cm,
    level 1/.style={sibling distance=1.2cm},
    level 2/.style={sibling distance=1.2cm}
]
  \node [arn_u] {\tiny FunDef}
  child{
    node [arn_u] {\code{add}}
  }
  child{
    node [arn_u] {\code{n}}
  }
  child{
    node [arn_u] {\code{m}}
  }
  child{
    node [arn_u] {\small Add}
    child{
      node [arn_u] {\small Name}
      child{
        node [arn_u] {\code{n}}
      }
    }
    child{
      node [arn_w] {}
    }
  };
\end{tikzpicture}
\end{center}

뿌리는 Name이라는 심벌로, 이 식이 그냥 어떤 이름이라는 사실을 나타낸다. 나무의
유일한 자식은 이름 \code{n}이다.

둘째 피연산자도 비슷하게 표현할 수 있다.

\begin{center}
\begin{tikzpicture}[
    level distance=1.2cm,
    level 1/.style={sibling distance=1.2cm},
    level 2/.style={sibling distance=1.2cm}
]
  \node [arn_u] {\tiny FunDef}
  child{
    node [arn_u] {\code{add}}
  }
  child{
    node [arn_u] {\code{n}}
  }
  child{
    node [arn_u] {\code{m}}
  }
  child{
    node [arn_u] {\small Add}
    child{
      node [arn_u] {\small Name}
      child{
        node [arn_u] {\code{n}}
      }
    }
    child{
      node [arn_u] {\small Name}
      child{
        node [arn_u] {\code{m}}
      }
    }
  };
\end{tikzpicture}
\end{center}

위 나무는 함수 정의의 구조를 표현한다. 무슨 언어로 작성된 코드인지는 상관없다.
파이썬 함수 정의의 구조일 수도 있고, 자바스크립트 함수 정의의 구조일 수도 있다.
이렇게 프로그램을 나무로 표현함으로써 문자열에 있는 중요하지 않은 정보는
무시하고 프로그램의 구조에 집중할 수 있다.

요약 문법이 프로그램을 나무로 표현하므로 언어의 요약 문법을 정의하는 것은
프로그램을 나타내는 모든 나무의 집합을 정의하는 것이다. \Lang의 요약 문법을
정의해 보자. 각 정수는 \Lang~프로그램이며, 정수 프로그램은 그 정수를 부품으로
가진다. 따라서, $E$가 모든 프로그램의 집합을 나타낸다고 할 때, 다음 문장이
참이다.

\begin{center}
\begin{tikzpicture}[
    level distance=1.2cm,
    level 1/.style={sibling distance=1.2cm},
]
  \node at (-2,-0.6) {$\text{만약 }n\in\mathbb{Z}\text{이면,}$};
  \node [arn_u] at (0,0) {\small Num}
  child{
    node [arn_u] {$n$}
  };
  \node at (1,-0.6) {$\in E$.};
\end{tikzpicture}
\end{center}

두 산술식의 합 역시 프로그램이다. 덧셈을 하는 프로그램은 피연산자 둘을 부품으로
가진다. 각 피연산자는 산술식이며, 곧 프로그램이기도 하다. 그러니 다음 문장 역시
참이다.

\begin{center}
\begin{tikzpicture}[
    level distance=1.2cm,
    level 1/.style={sibling distance=1.2cm},
]
  \node at (-2.3,-0.6) {$\text{만약 }e_1,e_2\in E\text{이면,}$};
  \node [arn_u] at (0,0) {\small Add}
  child{
    node [arn_nc] {$e_1$}
  }
  child{
    node [arn_nc] {$e_2$}
  };
  \node at (1.2,-0.6) {$\in E$.};
\end{tikzpicture}
\end{center}

뺄셈도 비슷하다.

\begin{center}
\begin{tikzpicture}[
    level distance=1.2cm,
    level 1/.style={sibling distance=1.2cm},
]
  \node at (-2.3,-0.6) {$\text{만약 }e_1,e_2\in E\text{이면,}$};
  \node [arn_u] at (0,0) {\small Sub}
  child{
    node [arn_nc] {$e_1$}
  }
  child{
    node [arn_nc] {$e_2$}
  };
  \node at (1.2,-0.6) {$\in E$.};
\end{tikzpicture}
\end{center}

위 사실들을 종합하면, \Lang의 요약 문법은 다음 조건을 모두 만족하는 가장 작은 집합 $E$라고 정의할 수 있다.

\begin{center}
\begin{tikzpicture}[
    level distance=1.2cm,
    level 1/.style={sibling distance=1.2cm},
]
  \node at (-2,-0.6) {$\text{만약 }n\in\mathbb{Z}\text{이면,}$};
  \node [arn_u] at (0,0) {\small Num}
  child{
    node [arn_u] {$n$}
  };
  \node at (1,-0.6) {$\in E$.};
\end{tikzpicture}
\end{center}

\begin{center}
\begin{tikzpicture}[
    level distance=1.2cm,
    level 1/.style={sibling distance=1.2cm},
]
  \node at (-2.3,-0.6) {$\text{만약 }e_1,e_2\in E\text{이면,}$};
  \node [arn_u] at (0,0) {\small Add}
  child{
    node [arn_nc] {$e_1$}
  }
  child{
    node [arn_nc] {$e_2$}
  };
  \node at (1.2,-0.6) {$\in E$.};
\end{tikzpicture}
\end{center}

\begin{center}
\begin{tikzpicture}[
    level distance=1.2cm,
    level 1/.style={sibling distance=1.2cm},
]
  \node at (-2.3,-0.6) {$\text{만약 }e_1,e_2\in E\text{이면,}$};
  \node [arn_u] at (0,0) {\small Sub}
  child{
    node [arn_nc] {$e_1$}
  }
  child{
    node [arn_nc] {$e_2$}
  };
  \node at (1.2,-0.6) {$\in E$.};
\end{tikzpicture}
\end{center}

가령 다음 나무는 \Lang~프로그램이다.

요약 문법에 따라 정의된 집합에 속하는 나무를 \term{요약 문법 나무}{abstract
syntax tree}\terme{AST}라 부른다.

요약 문법은 대수적 타입을 사용해 스칼라로 쉽게 구현할 수 있다. 다음 코드는
\Lang의 요약 문법을 구현한다.

\begin{verbatim}
sealed trait AE
case class Num(value: Int) extends AE
case class Add(left: AE, right: AE) extends AE
case class Sub(left: AE, right: AE) extends AE
\end{verbatim}

\code{Num($n$)}은 다음 나무를 나타낸다.

\begin{center}
\begin{tikzpicture}[
    level distance=1.2cm,
    level 1/.style={sibling distance=1.2cm},
]
  \node [arn_u] {\small Num}
  child{
    node [arn_u] {$n$}
  };
\end{tikzpicture}
\end{center}

\code{Add($e_1$, $e_2$)}는 다음 나무를 나타낸다.

\begin{center}
\begin{tikzpicture}[
    level distance=1.2cm,
    level 1/.style={sibling distance=1.2cm},
]
  \node [arn_u] {\small Add}
  child{
    node [arn_nc] {$e_1$}
  }
  child{
    node [arn_nc] {$e_2$}
  };
\end{tikzpicture}
\end{center}

\code{Sub($e_1$, $e_2$)}는 다음 나무를 나타낸다.

\begin{center}
\begin{tikzpicture}[
    level distance=1.2cm,
    level 1/.style={sibling distance=1.2cm},
]
  \node [arn_u] {\small Sub}
  child{
    node [arn_nc] {$e_1$}
  }
  child{
    node [arn_nc] {$e_2$}
  };
\end{tikzpicture}
\end{center}

예를 들어 앞서 본 AST는 스칼라 코드로 \code{Add(Sub(Num(5), Num(1)),
Num(3))}이라 쓸 수 있다.

요약 문법이 프로그램을 나무로 표현하기는 해도, 프로그램을 표현할 때마다 나무를
그리기는 귀찮다. 그래서 대개의 경우 코드와 비슷한 표기를 사용해 나무를 나타낸다.
가령 맥락상 AST를 뜻하는 것이 분명하다면 뿌리가 Num이고 자식이 $n$인 나무를
그리는 대신 그냥 $n$이라고 쓸 수 있다. 비슷하게, 덧셈과 뺄셈을 나타내는 나무를
그리는 대신 그냥 $\eadd{e_1}{e_2}$와 $\esub{e_1}{e_2}$라고 쓸 수 있다. 여기서
$+$와 $-$는 수학적인 덧셈과 뺄셈을 나타내는 기호가 아니라는 사실에 주의해야
한다. 그저 표기의 일부일 뿐으로 그 자체로는 아무런 뜻도 없다.

위 표기법을 사용해 \Lang의 요약 문법을 다시 정의할 수 있다. $E$는 다음 조건을
모두 만족하는 가장 작은 집합이다.

\begin{itemize}
  \item 만약 $n\in\mathbb{Z}$이면, $n\in E$.
  \item 만약 $e_1,e_2\in E$이면, $e_1+e_2\in E$.
  \item 만약 $e_1,e_2\in E$이면, $e_1-e_2\in E$.
\end{itemize}

비록 표기법은 나무처럼 생기지 않지만, 여전히 나무를 나타내고 있다.

표기법은 그저 임의로 정한 약속일 뿐이므로 다른 표기법을 정의하고 사용하는 것도
가능하다. 예를 들면 덧셈을 나타내는 AST를 $\eadd{e_1}{e_2}$ 대신 $\embox{ADD}\
e_1\ e_2$라고 쓰기로 약속할 수도 있다. 이처럼 표기법은 마음대로 정할 수 있다.
하지만 일단 한번 표기법을 정했으면 다른 사람들이 헷갈리지 않도록 그 표기법만
사용해야 한다.

요약 문법의 정의를 더 간결하게 만들기 위해 BNF를 사용할 수도 있다. 다음은
\Lang의 요약 문법을 BNF로 정의한 것이다.

\[e\ ::=\ n\ |\ e+e\ |\ e-e\]

요약 문법의 특정 원소를 나타내는 심벌을 \term{메타변수}{metavariable}라 부른다.
각 심벌이 일종의 변수지만, 정의하려는 언어에서 개발자들이 사용하는 변수가 아니라
한 단계 위, 즉 언어 설계자가 언어를 정의하는 단계에서 사용하는 변수라는 뜻으로
메타변수라 부르는 것이다. 위 요약 문법 정의에서 $e$는 프로그램을 나타내는
메타변수고 $n$은 정수를 나타내는 메타변수다.

AST를 표기법을 사용해 간단히 적을 때 필요에 따라 괄호를 사용할 수 있다. 만약
그냥 $3-1+2$라고만 쓰면 다음 두 나무 중 무엇을 뜻하는 것인지 모호할 수 있다.

\begin{center}
\begin{tikzpicture}[
    level distance=1.2cm,
    level 1/.style={sibling distance=1.8cm},
    level 2/.style={sibling distance=1.2cm},
]
  \node [arn_u] at (-3,0) {\small Add}
  child{
    node [arn_u] {\small Sub}
    child{
      node [arn_u] {\small Num}
      child{
        node [arn_u] {$3$}
      }
    }
    child{
      node [arn_u] {\small Num}
      child{
        node [arn_u] {$1$}
      }
    }
  }
  child{
    node [arn_u] {\small Num}
    child{
      node [arn_u] {$2$}
    }
  };
  \node at (0,-2) {or};
  \node [arn_u] at (3,0) {\small Sub}
  child{
    node [arn_u] {\small Num}
    child{
      node [arn_u] {$3$}
    }
  }
  child{
    node [arn_u] {\small Add}
    child{
      node [arn_u] {\small Num}
      child{
        node [arn_u] {$1$}
      }
    }
    child{
      node [arn_u] {\small Num}
      child{
        node [arn_u] {$2$}
      }
    }
  };
\end{tikzpicture}
\end{center}

왼쪽 나무를 나타내고 싶다면 $(3-1)+2$라 쓰는 편이 명확하고, 오른쪽 나무를
나타내고 싶다면 $3-(1+2)$라고 쓰는 편이 명확하다.

\section{파싱}

구체적 문법이 프로그램을 문자열로 간주하는 반면, 요약 문법은 프로그램을 나무로
간주한다. 이 차이를 해소하는 것이 파싱이다. \term{파싱}{parsing}은 구체적 문법을 따르는 문자열을
AST로 바꾸는 과정이다. 파싱을 수행하는 프로그램을 \term{파서}{parser}라 부른다. 파서를
$S$(모든 문자열의 집합)에서 $E$(모든 AST의 집합)로 가는 \term{부분 함수}{partial
function}로 생각할 수 있다.

\[\embox{parse}: S\pto E\]

\begin{kaobox}[frametitle=부분 함수]
  집합 $A$에서 집합 $B$로 가는 부분 함수는 $A$의 부분 집합 $S$에서 $B$로 가는
  함수다. $S$를 부분 함수의 정의역domain이라 부른다. $A\rightarrow B$는 $A$에서
  $B$로 가는 함수의 집합이고, $A\mapsto B$는 $A$에서 $B$로 가는 부분 함수의
  집합이다.

  $f$가 $A$에서 $B$로 가는 부분 함수라 하자. 그러면 $f(a)$가 정의되지 않은 $A$의
  원소 $a$가 존재할 수 있다. 개발자의 시각에서 $f$는 $A$에서
  \code{Option[$B$]}로 가는 함수로, 결괏값이 정의되지 않으면 \code{None}이
  나오고, 결괏값이 $b$이면 \code{Some($b$)}가 나온다.
\end{kaobox}

주어진 입력이 $P$(모든 프로그램의 집합)에 속하지 않으면 $\embox{parse}$의 결과는
정의되지 않는다. 그래서 $\embox{parse}$가 부분 함수다. 만약 입력이 $P$에 속하면
$\embox{parse}$는 그 문자열에 대응되는 AST를 만든다.

\Lang의 파서를 생각해 보자. 다음 문자열은 \Lang~프로그램이 아니므로
$\embox{parse}$의 결과가 정의되지 않는다.

\begin{itemize}
  \item \code{1+}
  \item \code{2*4}
  \item \code{0++3}
\end{itemize}

반면 다음은 $\embox{parse}$가 성공한 예시다.

\begin{center}
\begin{tikzpicture}[
    level distance=1.2cm,
    level 1/.style={sibling distance=1.2cm},
]
  \node at (-2.5,-1.2) {$\embox{parse}(\code{"-1+002"})=$};
  \node [arn_u] at (0,0) {\small Add}
  child{
    node [arn_u] {\small Num}
    child{
      node [arn_u] {$-1$}
    }
  }
  child{
    node [arn_u] {\small Num}
    child{
      node [arn_u] {$2$}
    }
  };
  \node at (1.8,-1.2) {$=-1+2$};
\end{tikzpicture}
\end{center}

이 책은 파서를 구현하는 방법을 다루지 않는다.

\section{의미}

문법은 언어의 필수 요소로 어떤 문자열이 프로그램이고 프로그램이 어떤 구조를
가지는지 알려 준다. 하지만 프로그램이 어떻게 실행되는지는 알려 주지 않는다.
개발자가 프로그램을 작성하는 이유는 실행하기 위함이다. 작성한 프로그램을
실행했을 때 무슨 일이 일어날지 알아야 한다. \term{의미}{semantics}는 언어의 또 다른 필수 요소로
프로그램의 동작을 정의한다.

\Lang의 의미를 정의해 보자. 프로그램의 구조가 그 동작을 결정하는 것이
자연스러우니, 의미는 요약 문법에 기반하여 정의한다. 어떤 프로그램의 의미란 그
프로그램을 실행했을 때 무슨 일이 일어나는지를 말하며, \Lang의 의미는 각
\Lang~프로그램의 의미를 결정한다. \Lang~프로그램은 실행했을 때 한 가지 일만
한다. 바로 해당 산술식을 계산한 결과를 알려 주는 것이다. 예를 들면, 의미를
올바르게 정의한 경우 $0+1$은 $1$을 결과로 낸다. 이때 $0+1$의 $+$는 그 자체로는
덧셈을 뜻하지 않으며 언어의 의미를 정의하기 전까지는 $0+1$의 결과가 무엇인지
말할 수 없다. \Lang을 합리적인 언어로 만들기 위해서는 의미를 잘 정의하여 $0+1$의
결과가 $1$이 되게끔 해 줘야 한다.

무수히 많은 프로그램이 존재하기 때문에 각 프로그램의 의미를 따로따로 정의할 수는
없다. 그렇기 때문에 요약 문법이 정의한 프로그램의 구조를 활용하여 의미를
정의한다. 요약 문법에 따라 프로그램을 $n$, $\eadd{e_1}{e_2}$,
$\esub{e_1}{e_2}$의 세 종류로 나눌 수 있다. 특정 종류에 속한 모든 프로그램의
의미를 한 번에 정의함으로써 무한히 많은 프로그램의 의미를 모두 정의할 것이다.

가장 간단한 경우는 $n$이다. $n$은 정수 하나로 구성된 식이다. 정수를 계산하면 그
정수가 그대로 나온다. 이러한 의미를 다음 규칙으로 표현할 수 있다.

\semanticrule{Num}{
  $n$의 결과는 $n$이다.
}

규칙 Num을 사용해 다음 사실을 알아낼 수 있다.

\begin{itemize}
  \item $1$의 결과는 $1$이다.
  \item $5$의 결과는 $5$이다.
\end{itemize}

다음은 $\eadd{e_1}{e_2}$ 차례다. $e_1$이 산술식이므로 계산하면 어떤 정수가
나온다. 그 정수를 $n_1$이라 하자. 마찬가지로 $e_2$ 역시 계산하면 어떤 정수가
나온다. 그 정수는 $n_2$라 하자. 이 장에서 수학적 덧셈을 나타낼 때는 $+$ 대신
$\iadd$라 쓰겠다. 요약 문법에서 사용한 아무 뜻도 없는 기호인 $+$와 수학적 덧셈을
분명하게 구분하려는 것이다. 문법과 의미 두 개념 모두에 익숙해지고 나면 둘 다
$+$라고만 써도 맥락상 무엇을 뜻하는 것인지 쉽게 알 수 있다. 그러니 다음 장부터는
수학적 덧셈도 $+$라 쓸 것이다. 다음 규칙은 $\eadd{e_1}{e_2}$의 의미를 정의한다.

\semanticrule{Add}{
만약 $e_1$의 결과는 $n_1$이고, $e_2$의 결과는 $n_2$이면,\\
$e_1+e_2$의 결과는 $n_1\iadd n_2$이다.
}

$\esub{e_1}{e_2}$의 의미도 비슷하게 정의할 수 있다. 앞서 $\iadd$를 사용한
것처럼, 이 장에서 수학적 뺄셈을 나타낼 때는 $\isub$이라 쓰겠다. 다음 규칙은
$\esub{e_1}{e_2}$의 의미를 정의한다.

\semanticrule{Sub}{
만약 $e_1$의 결과는 $n_1$이고, $e_2$의 결과는 $n_2$이면,\\
$e_1-e_2$의 결과는 $n_1\isub n_2$이다.
}


이 세 규칙이 모여 다 함께 \Lang의 의미를 정의한다. 이제 우리는 모든 \Lang
프로그램의 동작을 알 수 있다. $(3-1)+2$를 생각해 보자. 다음 과정에 따라
$(3-1)+2$를 계산하여 $4$를 얻는다.

\begin{enumerate}
  \item (규칙 Num에 의해) $3$의 결과는 $3$이다.
  \item (규칙 Num에 의해) $1$의 결과는 $1$이다.
  \item (규칙 Sub에 의해) 만약 $3$의 결과는 $3$이고, $1$의 결과는
    $1$이면, $3-1$의 결과는 $2$이다.
  \item (1, 2, 3에 의해) $3-1$의 결과는 $2$이다.
  \item (규칙 Num에 의해) $2$의 결과는 $2$이다.
  \item (규칙 Add에 의해) 만약 $3-1$의 결과는 $2$이고, $2$의 결과는 $2$이면,
    $(3-1)+2$의 결과는 $4$이다.
  \item (4, 5, 6에 의해) $(3-1)+2$의 결과는 $4$이다.
\end{enumerate}

이제 \Lang의 의미를 더 수학적인 방법으로 정의해 보자. 의미는 각 프로그램의 실행
결과를 정의한다. 이 경우 결과는 정수 하나다. 의미가 프로그램을 받아 정수를
내놓는다고 볼 수 있는 것이다. 따라서 의미를 프로그램에서 정수로 가는 함수로
생각하는 편이 자연스럽다.

\[\embox{eval}:E\rightarrow \mathbb{Z}\]

$E$의 원소 $e$가 주어졌을 때 유일한 정수 $\embox{eval}(e)$가 존재한다.
\Lang에서 이는 사실이다. 모든 산술식을 계산하여 정수를 얻을 수 있다.

하지만 다른 언어까지 고려하면 의미를 함수로 정의하는 것이 좋은 선택이 아니다.
어떤 프로그램은 아무 결과도 내지 않는다. \term{실행이 끝나지
않는}{nonterminating} 프로그램이 그런 예시다. 실행 시간 오류를 일으키는
프로그램도 여기에 속한다. 다음 장에서 실행 시간 오류를 일으키는 프로그램을 보게
될 것이다. 또한, 결과가 유일하지 않은 프로그램도 존재한다. 그런 프로그램은
\term{비결정적}{nondeterministic} 프로그램이라 부른다. 예를 들어 여러
\term{스레드}{thread}를 사용하는 동시성 프로그램의 동작은 실행 중에 스레드가
실행되는 순서에 따라 달라진다. 스레드의 실행 순서가 바뀌면 결과도 바뀔 수 있다.
결과가 없는 프로그램이나 비결정적 프로그램이 있기 때문에 의미를 함수로 정의할 수
없다. 함수보다는 \term{관계}{relation}로 의미를 정의해야 한다. \Lang의 의미는
함수로 정의해도 문제가 없지만, 이 장에서 논의한 내용을 다른 언어에도 쉽게 적용할
수 있도록 의미를 관계로 정의하겠다.

\Lang의 의미를 $E$와 $\mathbb{Z}$ 사이의 \term{이항관계}{binary relation}인
$\Rightarrow$로 정의하겠다.

\begin{kaobox}[frametitle=이항관계]
  집합 $A$와 $B$ 사이의 이항관계는 $A\times B$의 부분 집합이다. 이때 $A\times B$는
  $\{(a,b)\ |\ a\in A\land b\in B\}$이다.

  $R$을 $A$와 $B$ 사이의 이항관계라 하자. 그러면 $R \subseteq A\times B$이다.
  $a\in A$이고 $b\in B$일 때, $(a,b)\in R$이면 $a\ R\ b$라 쓴다. 예를 들어 $<$는
  $\mathbb{Z}$와 $\mathbb{Z}$ 사이의 이항관계이고 $(1,2)\in<$ 대신 $1<2$라 쓸 수
  있다.
\end{kaobox}

\[\Rightarrow\subseteq E\times\mathbb{Z}\]

$(e,n)\in\Rightarrow$, 즉 $e\Rightarrow n$은 $e$를 계산한 결과가 $n$임을 뜻한다.

의미를 수학적으로 다시 정의해 보자.

\semanticrule{Num}{
$n\Rightarrow n$.
}

\vspace{-1em}

\semanticrule{Add}{
만약 $e_1\Rightarrow n_1$이고, $e_2\Rightarrow n_2$이면,\\
$e_1+e_2\Rightarrow n_1\iadd n_2$.
}

\vspace{-1em}

\semanticrule{Sub}{
만약 $e_1\Rightarrow n_1$이고, $e_2\Rightarrow n_2$이면,\\
$e_1-e_2\Rightarrow n_1\isub n_2$.
}

이제 \term{추론 규칙}{inference rule}을 사용해 위 규칙을 다시 적어 보겠다.
추론 규칙은 기존 \term{명제}{proposition}로부터 새로운 명제를 증명할 수 있게 해 주는
규칙이다. 추론 규칙은 다음 형태로 작성한다.

\[
  \inferrule
  { \text{전제}_1 \\ \text{전제}_2 \\ \cdots \\ \text{전제}_n }
  { \text{결론} }
\]

추론 규칙은 가로줄, 줄 위의 명제, 줄 아래의 명제로 구성된다. 줄 위의 명제는
\term{전제}{premise}라 부르고, 줄 아래의 명제는 \term{결론}{conclusion}이라
부른다. 규칙이 뜻하는 바는 모든 전제가 참일 때 결론도 참이라는 것이다. 한 추론
규칙의 전제는 한 개일 수도, 여러 개일 수도 있다. 전제가 아예 없는 것도 가능하다.
전제가 없는 추론 규칙의 결론은 항상 참인 명제다. 전제가 없으면 추론 규칙의
가로줄을 생략할 수 있다.

\Lang의 의미를 추론 규칙으로 정의하자.

\[
  n\Rightarrow n
  \quad\textsc{[Num]}
\]

\[
  \inferrule
  { e_1\Rightarrow n_1 \\ e_2\Rightarrow n_2 }
  { e_1+e_2\Rightarrow n_1\iadd n_2 }
  \quad\textsc{[Add]}
\]

\[
  \inferrule
  { e_1\Rightarrow n_1 \\ e_2\Rightarrow n_2 }
  { e_1-e_2\Rightarrow n_1\isub n_2 }
  \quad\textsc{[Sub]}
\]

보다시피 자연어로 규칙을 적는 것보다 훨씬 명확하고 간결하게 의미를 표현할 수
있다.

추론 규칙을 사용해 $(3-1)+2\Rightarrow4$를 증명할 수 있다. 추론 규칙을 사용해
명제를 증명할 때는 \term{증명 나무}{proof tree}를 그린다. 증명 나무의 뿌리는
증명하려는 명제이고, 자식은 결론을 뒷받침하는 증거다. 전산학에서 보통 나무의
뿌리를 위쪽에 놓는 것과 달리, 증명 나무는 진짜 나무처럼 뿌리를 밑에 둔다. 자식은
그 위에 온다.

다음은 $3\Rightarrow 3$을 증명하는 증명 나무다.

\[3\Rightarrow3\]

규칙 Num에 전제가 없기 때문에 증명 나무에 뿌리밖에 없다.

비슷한 방법으로 $1\Rightarrow 1$을 증명할 수 있다.

\[1\Rightarrow1\]

규칙 Sub와 위의 나무를 바탕으로 $3-1\Rightarrow2$를 증명하는 증명 나무를 아래와
같이 그릴 수 있다.

\[
  \inferrule
  { 3\Rightarrow3 \\ 1\Rightarrow1 }
  { 3-1\Rightarrow2 }
\]

규칙 Num을 다시 사용하여 $2\Rightarrow2$를 증명할 수 있다.

\[2\Rightarrow2\]

그러고 나면 $(3-1)+2\Rightarrow4$의 증명 나무를 완성할 수 있다.

\[
  \inferrule
  {
    \inferrule
    { 3\Rightarrow3 \\ 1\Rightarrow1 }
    { 3-1\Rightarrow2 }
    \\
    2\Rightarrow2
  }
  { (3-1)+2\Rightarrow4 }
\]

위 예시에서는 증명 나무가 무엇인지 보여 주려고 증명 나무를 \term{잎}{leaf}부터
그렸다. 하지만 실제로 새로운 명제를 증명하기 위해 증명 나무를 그릴 때는 증명할
명제를 뿌리에 놓고 거기부터 시작하는 편이 자연스럽다. 증명 나무를 그리려면, 우선
계산할 프로그램을 쓰고 그 위에 가로줄을 그린다. 결과는 아직 모르니 쓰지 않는다.

\[
  \inferrule
  {
    \color{white}
    \inferrule
    { 3\Rightarrow3 \\ 1\Rightarrow1 }
    { 3-1\Rightarrow2 }
    \\
    \color{white}
    2\Rightarrow2
  }
  { (3-1)+2\Rightarrow{\color{white}4} }
\]

그러고 나서 적용 가능한 추론 규칙을 찾는다. 이 경우 프로그램이 덧셈식이므로 규칙
Add를 적용할 수 있다.

\[
  \inferrule
  {
    \inferrule
    { \color{white}3\Rightarrow3 \\ \color{white}1\Rightarrow1 }
    { 3-1\Rightarrow\color{white}2 }
    \\
    2\Rightarrow\color{white}2
  }
  { (3-1)+2\Rightarrow{\color{white}4} }
\]

이제 $3-1$과 $2$를 각각 계산해야 한다. $3-1$부터 보자. $3-1$은 뺄셈식이므로 규칙
Sub를 적용한다.

\[
  \inferrule
  {
    \inferrule
    { 3\Rightarrow\color{white}3 \\ 1\Rightarrow\color{white}1 }
    { 3-1\Rightarrow\color{white}2 }
    \\
    2\Rightarrow\color{white}2
  }
  { (3-1)+2\Rightarrow{\color{white}4} }
\]

규칙 Num을 통해 $3\Rightarrow3$을 알아낸다.

\[
  \inferrule
  {
    \inferrule
    { 3\Rightarrow3 \\ 1\Rightarrow\color{white}1 }
    { 3-1\Rightarrow\color{white}2 }
    \\
    2\Rightarrow\color{white}2
  }
  { (3-1)+2\Rightarrow{\color{white}4} }
\]

비슷하게 $1\Rightarrow1$을 알아낸다.

\[
  \inferrule
  {
    \inferrule
    { 3\Rightarrow3 \\ 1\Rightarrow1 }
    { 3-1\Rightarrow\color{white}2 }
    \\
    2\Rightarrow\color{white}2
  }
  { (3-1)+2\Rightarrow{\color{white}4} }
\]

머릿속으로 $3$에서 $1$을 빼서 $2$를 얻는다.

\[
  \inferrule
  {
    \inferrule
    { 3\Rightarrow3 \\ 1\Rightarrow1 }
    { 3-1\Rightarrow2 }
    \\
    2\Rightarrow\color{white}2
  }
  { (3-1)+2\Rightarrow{\color{white}4} }
\]

규칙 Num을 다시 사용해 $2\Rightarrow2$를 알아낸다.

\[
  \inferrule
  {
    \inferrule
    { 3\Rightarrow3 \\ 1\Rightarrow1 }
    { 3-1\Rightarrow2 }
    \\
    2\Rightarrow2
  }
  { (3-1)+2\Rightarrow{\color{white}4} }
\]

마지막으로 $2$와 $2$의 합을 계산하고 나면 증명 나무를 완성할 수 있다.

\[
  \inferrule
  {
    \inferrule
    { 3\Rightarrow3 \\ 1\Rightarrow1 }
    { 3-1\Rightarrow2 }
    \\
    2\Rightarrow2
  }
  { (3-1)+2\Rightarrow4 }
\]

이처럼 증명 나무를 그리는 과정은 차례대로 올바른 추론 규칙을 찾아 기계적으로
적용하는 것이다. 정의한 언어의 의미를 무시한 채 자신의 직관으로 증명 나무를
그려서는 안 된다. 프로그램의 결과를 증명하는 증명 나무를 때로는 \term{결과 유도
과정}{evaluation derivation}이라 부른다. 결과가 어떤 계산 과정을 통해 유도되었는지
드러내기 때문이다.

지금까지 의미를 정의한 방법을 \term{큰 보폭 과정 의미}{big-step operational
semantics}라 부른다. ``과정''은 정의한 의미가 계산 과정을 상세히 보여 준다는
뜻이고, ``큰 걸음''은 프로그램에서 실행 결과까지 큰 보폭으로 한 걸음만에
도달한다는 뜻이다. 의미를 정의하는 다른 방법도 있다. \term{결과
의미}{denotational semantics}는 계산 과정 없이 결과만을 드러내고, \term{작은
보폭 과정 의미}{small-step operational semantics}는 프로그램에서 실행 결과까지
작은 보폭으로 여러 걸음에 걸쳐 도달한다. 이 책에서는 거의 항상 큰 보폭 과정
의미를 사용하지만, 뒤에서 작은 보폭 과정 의미를 사용하는 경우도 있을 것이다.

\term{인터프리터}{interpreter}는 프로그램을 입력받아 해당 프로그램을 실행하는
프로그램이다. \Lang의 의미에 따라 인터프리터를 구현하는 것은 어렵지 않다.
인터프리터는 AST를 인자로 받아 정수를 결과로 내는 함수 하나로 이루어진다.

\begin{verbatim}
def interp(e: AE): Int = e match {
  case Num(n) => n
  case Add(l, r) => interp(l) + interp(r)
  case Sub(l, r) => interp(l) - interp(r)
}
\end{verbatim}

\section{문법 설탕}

\term{문법 설탕}{syntactic sugar}은 언어의 의미를 바꾸지 않은 채 문법적 코드
\term{변환}{transformation} 규칙만 정의해 언어에 새로운 기능을 추가한다. 문법
설탕은 언어 설계자가 적은 노력만으로 유용한 기능을 언어에 넣을 수 있게 해 주다
보니 현실 세계 언어에서 널리 사용된다.

정수의 \term{부호 바꾸기}{negation} 연산을 \Lang에 추가하고 싶다고 하자. 한 가지
방법은 문법과 의미를 모두 바꾸는 것이다. 우선 구체적 문법에 부호 바꾸기 연산을
추가한다.

\begin{verbatim}
<expr> ::= <number> | <expr> "+" <expr> | <expr> "-" <expr> | "-(" <expr> ")"
\end{verbatim}

그 다음에는 요약 문법도 고친다.

\[e\ ::=\ n\ |\ e+e\ |\ e-e\ |\ -e\]

파서도 올바르게 바꾸어야 한다. 가령 \code{"-(03+4)"}는 이제 $-(3+4)$로 파싱된다.

마지막으로 $-e$를 처리할 수 있도록 의미에 새로운 규칙을 추가한다.

\semanticrule{Neg}{
만약 $e$의 결과는 $n$이면,\\
$-e$의 결과는 $\isub n$이다.
}

여기서 $\isub$은 수학적 부호 바꾸기를 나타낸다.

같은 규칙을 추론 규칙으로도 쓸 수 있다.

\[
  \inferrule
  { e\Rightarrow n}
  { -e\Rightarrow\isub n }
  \quad\textsc{Neg}
\]

이런 방식으로 기능을 언어에 추가하면 언어의 모든 요소, 즉 구체적 문법, 요약
문법, 의미를 모두 고쳐야 하기 때문에 많은 노력이 필요하다.

\Lang에 정수 부호 바꾸기 연산을 추가하는 다른 방법은 문법 설탕을 사용하는
것이다. 이 방법을 사용하면 구체적 문법과 파서만 고쳐도 된다. 구체적 문법은 전과
똑같이 바꾼다. 한편, 파서는 이제 \code{expr}가 $e$로 파싱될 때 \code{"-(" expr
")"}를 $0-e$로 파싱한다. 예를 들면 \code{"-(03+4)"}를 $0-(3+4)$로 파싱하는 것이다.
$n$의 값에 상관없이 $0\isub n=\isub n$이 참이니 이는 올바르다. 문법 설탕을
사용하면 언어에 기능을 이렇게나 쉽게 추가할 수 있다. 언어 설계자가 새로운 기능을
문법적인 방법을 통해 기존 기능으로 변환함으로써 언어에 기능을 추가한 것이다.
문법 설탕으로 존재하는 기능을 파싱 중에 다른 기능으로 변환하는 과정을 \term{설탕
제거}{desugaring}라 부른다.

현실 세계 연어에서 여러 문법 설탕을 찾을 수 있다. 예를 들어 Scala의 \code{for}
반복문은 문법 설탕이다.

\begin{quote}
  A for comprehension \code{for ($p$ <- $e$) yield $e'$} is translated to
  \code{$e$.map \{ case $p$ =>
  $e'$ \}}.\footnote{\url{https://www.scala-lang.org/files/archive/spec/2.13/06-expressions.html\#for-comprehensions-and-for-loops}}
\end{quote}

또한, C, 스칼라, 래킷 등의 언어에 있는 \term{매크로}{macro}는 개발자가 직접
정의한 문법 설탕으로 볼 수 있다.

\section{예제}

\begin{exercise}
\labex{syntax-and-semantics-expr}

다음과 같은 구체적 문법이 있다.

\begin{verbatim}
<expr> ::= <num>
         | "{" "+" <expr> <expr> "}"
         | "{" "*" <expr> <expr> "}"
         | "{" "let" "{" <id> <expr> "}" <expr> "}"
         | <id>
\end{verbatim}

다음의 각 문자열이 \code{expr}인지 쓰고 이유를 설명하라.
\code{<id>}는 한 개 이상의 로마자(\code{a-z}, \code{A-Z})로 구성되고,
\code{<num>}은 한 개 이상의 숫자(\code{0-9})로 구성된다.
터미널 사이에 공백을 자유롭게 넣을 수 있다고 가정하자.

\begin{enumerate}
  \item \verb!{let {x 5} {+ 8 {* x 2 3}}}!
  \item \verb!{with {x 0} {with {x 7}}}!
  \item \verb!{let {3 5} {+ 8 {- x 2}}}!
  \item \verb!{let {3 y} {+ 8 {* x 2}}}!
  \item \verb!{let {x y} {+ 8 {* x 2}}}!
\end{enumerate}

\end{exercise}

\begin{exercise}
\labex{syntax-and-semantics-icecream}

다음과 같은 구체적 문법이 있다.

\begin{verbatim}
<ice-cream> ::= "sprinkles" "on" <ice-cream>
              | "cherry" "on" <ice-cream>
              | "scoop" "of" <flavor> "on" <ice-cream>
              | "sugar-cone"
              | "waffle-cone"
<flavor>    ::= "vanilla"
              | "lettuce"
\end{verbatim}

터미널 사이에 공백을 자유롭게 넣을 수 있다고 가정하자.
다음의 각 문자열이 \code{<ice-cream>}인지 쓰고 이유를 설명하라.

\begin{enumerate}
  \item \code{sprinkles}
  \item \code{sugar-cone}
  \item \code{vanilla}
  \item \code{scoop of vanilla on waffle-cone}
  \item \code{sprinkles on lettuce on waffle-cone}
  \item \code{scoop of vanilla on sprinkles on waffle-cone}
  \item \code{cherry on scoop of lettuce on scoop of vanilla on sugar-cone}
\end{enumerate}

\end{exercise}

\begin{exercise}
\labex{syntax-and-semantics-coffee}

다음과 같은 구체적 문법이 있다.

\newcommand{\BNF}[1]{\code{<#1>}}
\newcommand{\coffee}{\mbox{\BNF{coffee}}}
\newcommand{\milk}{\mbox{\BNF{milk}}}
\newcommand{\flavor}{\mbox{\BNF{flavor}}}

\[
\begin{array}{ccc}
  \code{espresso} \in \coffee
  &&
  \inferrule
  { e_1 \in \milk \\ e_2 \in \coffee }
  { e_1\ \code{"on"}\ e_2 \in \coffee }
  \\[2em]
  \inferrule
  { e_1 \in \coffee \\ e_2 \in \milk }
  { e_1\ \code{"on"}\ e_2 \in \coffee }
  &&
  \inferrule
  { e_1 \in \flavor \\ e_2 \in \coffee }
  { e_1\ \code{"on"}\ e_2 \in \coffee }
  \\[2em]
  \code{"milk-foam"} \in \milk
  &&
  \code{"steamed-milk"} \in \milk
  \\[2em]
  \code{"caramel"} \in \flavor
  &&
  \code{"cinnamon"} \in \flavor
  \\[2em]
  \code{"cocoa-powder"} \in \flavor
  &&
  \code{"chocolate-syrup"} \in \flavor
\end{array}
\]

터미널 사이에 공백을 자유롭게 넣을 수 있다고 가정하자.
다음의 각 문자열이 $\coffee$인지 쓰고 이유를 설명하라.

\begin{enumerate}
  \item \code{caramel latte macchiato}
  \item \code{espresso}
  \item \code{steamed-milk on caramel on milk-foam on espresso}
  \item \code{chocolate-syrup on cocoa-powder on cinnamon on milk-foam on steamed-milk on espresso}
  \item \code{steamed-milk on espresso on chocolate-syrup}
  \item \code{cocoa-powder on milk-foam on steamed-milk on espresso}
\end{enumerate}

\end{exercise}
