\chapter{프로그래밍 언어}
\labch{introduction}

\term{프로그래밍 언어}{programming language}란 무엇일까?

간단하게는 ``프로그래밍에 사용하는 언어다''라고 대답할 수 있다. 그렇지만 이 답은
우리가 프로그래밍 언어가 무엇인지 이해하는 데 별 도움을 주지 못한다. 더 나은
답을 원한다면 더 나은 질문을 해야 한다.

프로그래밍 언어는 무엇으로 구성될까?

이 질문에는 좀 더 좋은 답이 있다. ``좁게는 \term{문법}{syntax}과
\term{의미}{semantics}로 이루어지고, 넓게는 \term{표준 라이브러리}{standard
library}와 \term{생태계}{ecosystem}까지 추가로 가진다.''

문법과 의미는 프로그래밍 언어를 이해하는 데 필수적인 개념이다. 문법은 각 언어의
겉모습을 결정하고, 의미는 그 속을 채운다. 프로그래밍 언어가 한 명의 사람이라면,
문법은 외모이고 의미는 성격이라 할 수 있다. 프로그래머는 문법에 따라 프로그램을
작성한다. 작성한 코드에 어떤 문자가 어떤 순서로 등장할지 문법이 정하는 것이다.
프로그램을 작성했다면 이제 의미가 그 프로그램이 무슨 일을 하는지 정할 차례다.
언어의 의미가 주어지지 않으면 프로그램은 쓸모없는 셈이다. 의미가 프로그램의
동작을 규정해야 프로그래머가 원하는 대로 프로그램이 일할 수 있다. 그러므로
문법과 의미가 주어진다면 프로그래머는 그 언어로 프로그램을 작성하고 실행할 수
있다. 즉, 문법과 의미를 갖춘 프로그래밍 언어는 그 자체로 완전하다. 이론적인
맥락에서 문법과 의미가 프로그래밍 언어의 전부라고 할 수 있는 이유다.

하지만 개발자들에게는 문법과 의미가 전부가 아니다. 표준 라이브러리 역시 중요한
요소다. 표준 라이브러리는 유용한 프로그램을 작성하는 데 필요한 다양한 기능을
제공한다. \term{리스트}{list}와 \term{맵}{map} 등의 자료 구조, 파일 및 네트워크 입출력 처리
함수 같은 기능들 말이다. 표준 라이브러리는 사람으로 치면 옷과 같다. 나체의
사람도 물론 사람이다. 마찬가지로 표준 라이브러리가 없는 프로그래밍 언어도 여전히
프로그래밍 언어다. 하지만 옷이 몸을 따뜻하게 만들고 위험한 물체로부터 보호하기에
사람에게 중요하듯, 표준 라이브러리는 여러 편의 기능을 제공한다는 면에서
프로그래밍 언어에게 소중하다. 각 사람이 서로 다른 옷을 입는 것처럼, 각 언어는
조금씩 다른 기능을 표준 라이브러리에 넣는다. 어떤 언어는 최대한 많은 기능을 표준
라이브러리에 넣고, 어떤 언어는 그보다 훨씬 적은 기능만을 넣는다. 어떤 언어는
리스트와 맵을 아예 언어의 의미에 박아 넣는가 하면, 어떤 언어는 의미에는 더
단순한 \term{프리미티브}{primitive}만 정의한 뒤 이를 바탕으로 표준 라이브러리에
리스트와 맵을 정의한다. 만약 언어에 표준 라이브러리가 없다면 프로그램을 작성하는
데 드는 노력이 늘어날 테니, 개발자들은 표준 라이브러리가 없는 언어를 원하지
않는다.

개발자들에게는 언어의 생태계 또한 중요하다. 생태계는 언어와 관련된 모든 것을
포함한다. 그 언어를 사용하는 개발자와 회사, 그 언어를 위한
\term{서드 파티 라이브러리}{third-party library} 등이 해당한다. 사람으로 치면 사회인 셈이다. 많은
개발자와 회사가 사용하는 언어를 배운다면 막힐 때 도움을 받거나 관련 자료를 찾기
쉽고, 더 나아가 더 많은 고용과 협력의 기회도 생긴다. 서드 파티 라이브러리도
중요하다. 표준 라이브러리는 대개 일반적인 기능만 제공하고 특정 분야에 특화된
기능은 제공하지 않는다. 원하는 기능이 표준 라이브러리에 없을지라도, 그 기능을
제공하는 서드 파티 라이브러리가 있을 수는 있다. 이처럼 언어의 생태계는
개발자에게 중요한 요소다.

그렇기에 현실적으로는 문법과 의미뿐 아니라 표준 라이브러리와 생태계 역시 언어의
중요한 구성 요소다. 물론 표준 라이브러리와 생태계 없이도 언어는 존재할 수 있으니
문법과 의미처럼 필수 요소라고는 할 수 없다. 그러나 개발자들은 사용할 언어를 고를
때 분명 문법과 의미뿐 아니라 표준 라이브러리와 생태계까지 고려할 테다. 그래서
우리는 실용적인 맥락에서 프로그래밍 언어가 문법, 의미, 표준 라이브러리, 생태계로
구성된다고 이야기한다.

이 책은 독자들이 특정 언어를 잘 다룰 수 있도록 돕지 않는다. 특정 언어를
추천하지도 않는다. 그 대신 새로운 언어를 쉽고 빠르게 배울 수 있도록 돕는다. 이
책을 끝까지 읽고 완전히 이해했다면 어떤 언어든 금방 익힐 수 있을 것이다. 이 책이
서로 다른 각각의 언어를 따로따로 설명한다면 이런 목표는 이루기 어렵다. 그렇기에
이 책은 모든 프로그래밍 언어의 기저에 있는 기본 원리를 다룬다.

프로그래밍 언어의 원리는 언어의 의미에서 찾을 수 있다. 각 언어가 다른 언어들과는
매우 달라 보일 때가 많다. 하지만 그렇지 않다. 사실 겉모습이 다를 뿐이지, 그 속은
거의 같다. 코드의 겉모습을 결정하는 문법과 표준 라이브러리가 다르기 때문에 달라
보이지만, 그 속에 있는 의미는 같은 수학적 원리를 공유한다. 여러 언어의 의미에서
공통으로 등장하는 필수 개념들을 이해한다면, 새로운 언어를 배우고 익히는 것은
어려울 게 없다.

핵심 원리를 이해하고 언어의 구성 요소들을 분리할 수 있는 사람은 언어를 쉽게 배울
수 있다. 비유로서, 컴퓨터를 배우는 사람을 떠올려 보자. 만약 그 사람이 키보드와
컴퓨터를 구분할 수 없다면 큰 문제다. 가령 ```안녕'이라고 말하려면 왼손 중지로
키보드를 누르고, 오른손 중지로 키보드를 누르고, 왼손 약지로 키보드를 두 번
누르고, 오른손 검지로 키보드를 누르고, 왼손 중지로 키보드를 눌러야 한다''고
생각하는 식이다. 키보드의 글자 배치가 바뀐다면 컴퓨터를 처음부터 다시 배워야 할
것이다. 반면, 키보드는 그저 글자를 입력하기 위한 도구일 뿐이라는 사실을 안다면
글자 배치가 바뀌는 게 별 문제가 아니다. ```안녕'이라고 말하려면 ㅇ, ㅏ, ㄴ, ㄴ,
ㅕ, ㅇ을 눌러야 한다''고 생각하기에 글자 배치가 바뀌어도 컴퓨터를 다시 배울
필요는 없는 것이다. 물론 새로운 글자 배치를 익혀야 하겠지만, 이는 컴퓨터를 다시
배우는 것보다는 훨씬 쉽다. 더 나아가, 기존 지식을 새로운 일에 활용하기도 쉽다.
예를 들면 ```아니'라고 말하려면 ㅇ, ㅏ, ㄴ, ㅣ를 눌러야 한다''는 사실을 쉽게
알아낼 수 있다. 만약 키보드와 컴퓨터를 구분하지 못한다면, ``안녕''을 말하는 것과
``아니''를 말하는 것 사이의 어떠한 공통 원리도 찾지 못할 것이다. 프로그래밍 언어를
배울 때도 마찬가지다. 문법과 의미를 구분하지 못하는 사람은 문법만 바뀌어도
언어를 새로 배워야 한다. 하지만 문법과 의미를 구분할 줄 아는 사람은 문법이
바뀌어도 의미는 거의 그대로라는 사실을 안다. 의미의 기본 원리를 이미 이해한
이상, 새로운 문법에만 적응하면 새 언어에 금세 능숙해질 수 있다.

이 책은 프로그래밍 언어들이 공통적으로 제공하는 기본 기능의 의미를 설명한다.
...에서는 우선 \term{스칼라}{Scala} 언어를 소개한다. 스칼라는 책에서 도입하는 언어들의
\term{인터프리터}{interpreter}와 \term{타입 검사기}{type checker}를 구현하는 데 사용된다.
...에서는 문법과 의미를 알아본다. 그 뒤에는 프로그래밍 언어의 여러 기능을
살펴본다.

\begin{itemize}
\item \refch{introduction}은 \term{식별자}{identifier}를 다룬다.
\item ...은 \term{함수}{function}를 다룬다.
\item ...은 \term{수정}{mutation}을 다룬다.
\item ...은 \term{쓰레기 수집}{garbage collection}을 다룬다.
\item ...은 \term{느긋한 계산}{lazy evaluation}을 다룬다.
\item ...은 \term{컨티뉴에이션}{continuation}을 다룬다.
\item ...은 \term{드 브루인 번호}{De Bruijn index}를 다룬다.
\item ...은 기초적인 \term{타입 체계}{type system}를 다룬다.
\item ...은 \term{대수적 타입}{algebraic data type}을 다룬다.
\item ...은 \term{매개변수를 통한 다형성}{parametric polymorphism}을 다룬다.
\item ...은 \term{서브타입을 통한 다형성}{subtype polymorphism}을 다룬다.
\item ...은 \term{타입 추론}{type inference}을 다룬다.
\end{itemize}

각 장은 특정 기능을 제공하는 작은 언어를 정의함으로써 그 기능을 설명한다. 그렇게 정의한 작은 언어들은 너무 적은 기능만을 제공하기 때문에 실용적으로 사용하기에는 다소 불편하다. 그 대신 간단하기 때문에 각 장의 주제에 집중하는 데 도움이 될 것이다.

\section{예제}

\begin{exercise}

지금까지 사용해 본 프로그래밍 언어 하나의 이름을 쓰라. 그 언어의 장점과 단점은 무엇인가?

\end{exercise}

\begin{exercise}

알고 있는 프로그래밍 언어 두 개의 이름을 쓰고 두 언어를 비교하라.

\end{exercise}

% This book explains the semantics of principal concepts in programming languages.
% Chapters \chnum{introduction-to-scala}, \chnum{immutability}, \chnum{functions},
% and \chnum{pattern-matching} introduce the Scala programming language. This book
% uses Scala to implement interpreters and type checkers of languages introduced
% in the book.  \refch{syntax-and-semantics} explains syntax and semantics. Then,
% the book finally introduces various features of programming languages.
% \begin{itemize}
%     \item \refch{identifiers} introduces identifiers.
%     \item \refchss{first-order-functions}{first-class-functions}{recursion} introduce functions.
%     \item \refchs{boxes}{mutable-variables} introduce mutation.
%     \item \refch{garbage-collection} introduces garbage collection.
%     \item \refch{lazy-evaluation} introduces lazy evaluation.
%     \item \refchss{continuations}{first-class-continuations}{first-order-representation-of-continuations} introduce continuations.
%     \item \refch{nameless-representation-of-expressions} introduces De Bruijn indices.
%     \item \refchs{type-systems}{typing-recursive-functions} introduce basic type systems.
%     \item \refch{algebraic-data-types} introduces algebraic data types.
%     \item \refch{parametric-polymorphism} introduces parametric polymorphism.
%     \item \refch{subtype-polymorphism} introduces subtype polymorphism.
%     \item \refch{type-inference} introduces type inference.
% \end{itemize}
