\chapter{패턴 대조}
\labch{pattern-matching}

이 장에서는 스칼라의 \term{패턴 대조}{pattern matching}를 살펴본다. 패턴 대조는
함수형 프로그래밍의 핵심 개념 중 하나로, 복잡하지만 정해진 구조를 갖는 데이터를
다루는 것을 돕는다. 이미 앞에서 리스트를 다룰 때 간단한 형태의 패턴 대조를
사용했다. 이 장에서는 패턴 대조의 장점을 알아본 뒤 스칼라에서 사용할 수 있는
다양한 패턴을 소개한다. 또한, 함수형 프로그래밍에서 자주 사용되는 타입인
\term{옵션}{option} 타입을 살펴본다.

\section{대수적 타입}

한 타입에 다양한 형태의 값이 속하는 경우가 많다.

가령, 자연수는

\begin{itemize}
\item 0이거나
\item 어떤 자연수의 다음 수다.
\end{itemize}

리스트는

\begin{itemize}
\item 빈 리스트거나
\item 원소와 어떤 리스트의 순서쌍이다.
\end{itemize}

\term{이진 나무}{binary tree}는

\begin{itemize}
\item 빈 나무거나
\item \term{뿌리}{root} 원소와 두 자식 나무로 구성된 나무다.
\end{itemize}

\term{산술식}{arithmetic expression}은

\begin{itemize}
\item 수거나
\item 두 산술식의 합이거나
\item 두 산술식의 차다.
\end{itemize}

\term{람다 대수}{lambda calculus}의 식은

\begin{itemize}
\item 변수거나
\item 매개변수와 식으로 구성된 익명 함수거나
\item 두 식으로 구성된 함수 호출이다.
\end{itemize}

이처럼 전산학에서 하나의 타입은 대개 여러 형태의 값들을 포함한다. 이런 타입은
\term{대수적 타입}{algebraic data type}\terme{ADT}으로 표현된다. 대수적 타입이라는
이름이 붙은 이유는 한 타입이 \term{곱 타입}{product type}의 \term{합 타입}{sum
type}으로 정의되기 때문이다. 곱 타입의 각 원소는 특정한 타입들의 값을 나열한
것이다. 튜플 타입이 대표적인 곱 타입이다. 합 타입은 \term{이름 붙은 합집합
타입}{tagged union type}이라고도 불리는데, 하나의 합 타입은 마치 여러 타입의
합집합을 만든 것처럼 여러 타입의 값들을 포함한다. 단, 단순히 합집합만 만드는
것이 아니라 각 값이 어느 타입에서 온 것인지 구분할 수 있도록 이름을 붙인다.
이렇게 합 타입의 구성 요소가 되는 각 타입에 이름을 붙인 것을
\term{이형}{variant}이라고 부른다.

예를 들어 산술식은 다음과 같은 세 개의 이형으로 구성된다.

\begin{itemize}
\item 수거나
\item 두 산술식의 합이거나
\item 두 산술식의 차다.
\end{itemize}

이를 바탕으로 산술식의 타입인 \code{AE}를 합 타입으로 정의할 수 있다. \code{AE}는 다음 세 이형의 합이다.

\begin{itemize}
\item \code{Num}이라고 이름 붙인 \code{Int} 타입
\item \code{Add}라고 이름 붙인 \code{AE * AE} 타입
\item \code{Sub}라고 이름 붙인 \code{AE * AE} 타입
\end{itemize}

여기서 \code{AE * AE}는 \code{AE}와 \code{AE}의 곱 타입이다.

대수적 타입은 함수형 언어에서 쉽게 찾을 수 있다. 대부분의 함수형 언어는
프로그래머가 새로운 대수적 타입을 정의할 수 있도록 한다. 다음 오캐멀 코드는
산술식 타입을 정의한다.

\begin{verbatim}
type ae =
| Num of int
| Add of ae * ae
| Sub of ae * ae
\end{verbatim}

스칼라는 대수적 타입만을 정의하는 직접적인 방법을 제공하는 대신, 일반적인 타입을
정의하는 기능으로서 \term{트레잇}{trait}과 클래스를 제공한다. 프로그래머는
트레잇과 클래스를 사용해 대수적 타입을 정의할 수 있다.

새로운 타입은 트레잇으로 정의한다. 그 방법은 다음과 같다.

\begin{verbatim}
trait [name]
\end{verbatim}

위 코드는 이름이 \code{[name]}인 타입을 정의한다. 즉, 다음 코드는 산술식의
타입인 \code{AE}를 정의한다.

\begin{verbatim}
sealed trait AE
\end{verbatim}

여기서 \code{sealed}는 \code{AE}가 정의된 파일 안에서만 \code{AE}를
\term{상속}{inheritance}하는 클래스를 정의할 수 있도록 만든다. \code{sealed}를
붙여야 하는 이유는 뒤에서 이야기하겠다.

타입을 트레잇으로 정의하고 나면 원래 존재하던 다른 타입과 같은 방법으로 사용할 수 있다. 가령 산술식을 위한 항등 함수를 다음과 같이 정의할 수 있다.

\begin{verbatim}
def identity(ae: AE): AE = ae
\end{verbatim}

하지만 트레잇은 그 자체로는 그 타입에 속하는 값을 만들 방법을 제공하지 않는다.
다시 말해 아직 \code{AE} 타입의 값을 만들 수 없다는 것이다. 이 문제를 해결하려면
\code{AE}의 이형들을 \code{AE}를 상속하는 케이스 클래스로서 정의해야 한다.

\begin{verbatim}
case class Num(value: Int) extends AE
case class Add(left: AE, right: AE) extends AE
case class Sub(left: AE, right: AE) extends AE
\end{verbatim}

\refch{introduction-to-scala}에서 본 것처럼 케이스 클래스의 값을 쉽게 만들 수 있다.

\begin{verbatim}
val n = Num(10)
val m = Num(5)
val e1 = Add(n, m)
val e2 = Sub(e1, Num(3))
\end{verbatim}

트레잇과 마찬가지로 클래스는 새로운 타입을 정의한다. 각 클래스의 이름이 곧 정의한 타입의 이름이다. 클래스의 값은 그 클래스에 해당하는 타입에 속한다.

\begin{verbatim}
val n: Num = Num(10)
val m: Num = Num(5)
val e1: Add = Add(n, m)
val e2: Sub = Sub(e1, Num(3))
\end{verbatim}

더 나아가, 각 클래스가 \code{AE}를 상속했으므로 \code{Num}, \code{Add},
\code{Sub}이 \code{AE}의 \term{서브타입}{subtype}이다. 이는 어떤 값이
\code{Num}, \code{Add}, \code{Sub} 중 한 타입에 속하면 자동으로 \code{AE}
타입에도 속한다는 뜻이다.

\begin{verbatim}
val n: AE = Num(10)
val m: AE = Num(5)
val e1: AE = Add(n, m)
val e2: AE = Sub(e1, Num(3))
\end{verbatim}

각 객체의 필드에는 필드 이름을 통해 접근할 수 있다.

\begin{verbatim}
val n: Num = Num(10)
n.value
\end{verbatim}

그러나 객체의 타입이 \code{AE}일 때는 필드에 접근할 수 없다.

\begin{verbatim}
val m: AE = Num(10)
m.value
\end{verbatim}
\vspace{-1em}
\begin{mdframed}[hidealllines=true,backgroundcolor=gray!10,innerleftmargin=3pt,innerrightmargin=3pt,leftmargin=-3pt,rightmargin=-3pt]
\begin{verbatim}
  ^
error: value value is not a member of AE
\end{verbatim}
\vspace{-2em}
\begin{flushright}
\scriptsize\textsf{컴파일 시간 오류}
\end{flushright}
\end{mdframed}

\code{m}이 \code{AE} 타입의 값이므로, 반드시 \code{Num} 타입의 값이라는 보장이
없기 때문이다. \code{m}이 \code{Add}나 \code{Sub} 타입의 값일 수도 있으며,
\code{Add}와 \code{Sub}에는 \code{value}라는 필드가 없다. 따라서 컴파일러는
\code{m}이 \code{value}라는 필드를 가지고 있지 않을 수도 있다고 판단하여
\code{m.value}를 실행해서는 안 되는 식으로 간주해 버린다.

그렇기에 대수적 타입에 속하는 값을 다룰 때는 필드에 직접 접근하는 것보다 패턴
대조를 사용하는 편이 좋다. 다음 함수는 주어진 산술식을 계산하여 해당 산술식이
나타내는 정수를 내놓는다.

\begin{verbatim}
def eval(e: AE): Int = e match {
  case Num(n) => n
  case Add(l, r) => eval(l) + eval(r)
  case Sub(l, r) => eval(l) - eval(r)
}

assert(eval(Sub(Add(Num(3), Num(7)), Num(5))) == 5)
\end{verbatim}

만약 \code{e}가 \code{Num(n)}이면 \code{eval}은 바로 \code{n}을 결과로 낸다.
\code{e}가 \code{Add(l, r)}이면 \code{l}과 \code{r}이 어떤 산술식이므로 재귀
호출을 통해 \code{l}과 \code{r}을 각각 계산한 뒤 합을 구한다. \code{e}가
\code{Sub(l, r)}일 때는 비슷하지만 합 대신 차를 구한다.

리스트는 대수적 타입의 또 다른 좋은 예시다. 스칼라 표준 라이브러리는 다음과 같이
리스트를 대수적 타입으로 정의한다.

\begin{verbatim}
sealed trait List[A]
case object Nil extends List[Nothing]
case class ::[A](head: A, tail: List[A]) extends List[A]
\end{verbatim}

위 코드는 실제 코드에서 중요한 부분만 간추린 것으로, 리스트를 정의하는 핵심
아이디어를 잘 드러낸다. 리스트는 빈 리스트이거나 비어 있지 않은 리스트이며, 비어
있지 않은 리스트는 머리 원소와 꼬리 리스트로 구성된다. 모든 빈 리스트는
동일하므로 빈 리스트는 \code{class} 대신 \code{object} 키워드를 사용해 정의된다.
이 코드는 \code{Nil}이라는 값을 정의하고 그 값이 \code{List} 타입에 속하게
만든다. 한편, 비어 있지 않은 리스트를 나타내는 클래스의 이름인 \code{::}이 다소
이상해 보일 수 있다. 이 이름은 패턴 대조를 사용할 때 코드를 보기 좋게 만들기
위한 것이다. 스칼라에서는 패턴을 작성할 때 클래스 이름을 \term{중위
연산자}{infix operator}처럼 사용하는 것을 허용한다. 즉, \code{case ::(h, t)
=>}라고 적어도 되고 \code{case h :: t =>}라고 적어도 된다. 이처럼 클래스 이름이
\code{::}인 덕분에 \code{case h :: t =>}라는 직관적인 코드를 작성할 수 있다.

\section{장점}

\subsection{간결함}

패턴 대조가 없다면 대수적 타입을 다루는 것이 매우 복잡한 작업이 되어 버린다.
\term{동적 타입 검사}{dynamic type test}를 통해 주어진 값이 무슨 이형에 속하는지
확인하고 \term{타입 변환}{type casting}을 통해 필드에 접근해야 한다.

다음은 패턴 대조를 사용하지 않고 \code{eval}을 구현한 것이다.

\begin{verbatim}
def eval(e: AE): Int =
  if (e.isInstanceOf[Num])
    e.asInstanceOf[Num].value
  else if (e.isInstanceOf[Add]) {
    val e0 = e.asInstanceOf[Add]
    eval(e0.left) + eval(e0.right)
  } else {
    val e0 = e.asInstanceOf[Sub]
    eval(e0.left) - eval(e0.right)
  }
\end{verbatim}

\code{e.isInstanceOf[Num]}은 \code{e}가 \code{Num} 클래스에 속하는지 확인한다.
만약 그렇다면 \code{eval}은 \code{e}의 \code{value} 필드에 저장되어 있는 값을
내놓아야 한다. 하지만 \code{e}의 타입이 \code{AE}이므로 \code{value} 필드에 바로
접근할 수 없다. 물론 우리는 \code{e.isInstanceOf[Num]}의 결과가
\code{true}이므로 \code{e}의 진짜 타입은 \code{Num}이라는 사실을 안다. 그러니
타입 변환을 통해 이 사실을 컴파일러에게 알려 줘야 한다.
\code{e.asInstanceOf[Num]}은 \code{e}가 나타내는 값은 바꾸지 않은 채
컴파일러에게 \code{e}의 타입이 \code{Num}이라는 정보를 제공한다. 그러면
컴파일러는 \code{e.asInstanceOf[Num]}을 \code{Num} 타입의 식으로 간주하여
\code{value} 필드에 접근하는 것을 허용한다. \code{Add}와 \code{Sub} 이형을
위해서도 동일한 동적 타입 검사 및 타입 변환 과정이 필요하다.

실제로 하는 일은 간단한 데 비해 코드가 너무 길고 복잡하다. 실제로 의미 있는
계산을 하는 부분은 짧고 동적 타입 검사와 타입 변환을 위한 코드가 대부분이다. 또,
이런 코드는 실수의 가능성을 키운다. 가령 실수로 다음과 같은 코드를 작성할 수
있다.

\begin{verbatim}
else if (e.isInstanceOf[Add]) {
  val e0 = e.asInstanceOf[Sub]
  eval(e0.left) + eval(e0.right)
}
\end{verbatim}

\code{e}의 타입이 \code{Add}인지 확인한 것과 달리, \code{e}의 타입을
\code{Sub}으로 변환하고 있다. 이런 코드는 실행 시간 오류를 발생시킨다. 다행히도
\code{eval}은 간단한 함수기 때문에 이런 실수를 했는지 확인하는 것이 어렵지 않다.
하지만 타입이 복잡해지고 계산이 복잡해지면 실수를 저지르고도 이를 모르고
넘어가기 쉽다.

반면, 패턴 대조를 사용하면 이런 문제가 전혀 없다. 명시적으로 동적 타입 검사와
타입 변환을 위한 코드를 작성할 필요가 없기에 코드가 훨씬 간결해진다. 실수를 할
가능성조차 없애 주는 것은 덤이다.

\subsection{완전성 검사}

패턴 대조는 패턴의 \term{완전성}{exhaustivity}을 검사한다. 만약 주어진 값이 어떤
패턴에도 일치하지 않으면 실행 시간 오류가 발생한다.

\begin{verbatim}
def eval(e: AE): Int = e match {
  case Add(l, r) => eval(l) + eval(r)
  case Sub(l, r) => eval(l) - eval(r)
}
\end{verbatim}

위 코드는 \code{Num} 패턴을 빠트렸다.

\begin{verbatim}
eval(Num(3))
\end{verbatim}
\vspace{-1em}
\begin{mdframed}[hidealllines=true,backgroundcolor=gray!10,innerleftmargin=3pt,innerrightmargin=3pt,leftmargin=-3pt,rightmargin=-3pt]
\begin{verbatim}
scala.MatchError: Num(3) (of class Num)
\end{verbatim}
\vspace{-2em}
\begin{flushright}
\scriptsize\textsf{실행 시간 오류}
\end{flushright}
\end{mdframed}

따라서 \code{Num} 타입의 인자를 넣으면 실행 시간 오류가 발생한다.

다행히 패턴을 빠트리는 실수를 쉽게 막을 수 있다. 스칼라 컴파일러는 패턴이
완전한지 확인하고 그렇지 않다면 \term{경고}{warning} 문구를 띄운다.

\begin{verbatim}
def eval(e: AE): Int = e match {
  case Add(l, r) => eval(l) + eval(r)
  case Sub(l, r) => eval(l) - eval(r)
}
\end{verbatim}
\vspace{-1em}
\begin{mdframed}[hidealllines=true,backgroundcolor=gray!10,innerleftmargin=3pt,innerrightmargin=3pt,leftmargin=-3pt,rightmargin=-3pt]
\begin{verbatim}
                       ^
warning: match may not be exhaustive.
It would fail on the following input: Num(_)
\end{verbatim}
\vspace{-2em}
\begin{flushright}
\scriptsize\textsf{컴파일 시간 경고}
\end{flushright}
\end{mdframed}

컴파일러가 패턴이 완전하지 않다고 경고한 것을 볼 수 있다. 단순히 완전하지
않다고만 이야기하는 것이 아니라 무슨 패턴이 없는지 알려 줘서 디버깅을 돕기까지
한다. 프로그램이 복잡할수록 완전성 검사의 이점이 잘 드러난다. 완전성 검사 덕분에
실행 시간 오류를 일으키지 않는 프로그램을 작성하기 쉬워지므로 완전성 검사는 패턴
대조의 중요한 장점이다.

완전성 검사를 하려면 트레잇을 정의할 때 \code{sealed}를 붙이는 것이 필수다.
\code{sealed}를 붙이지 않은 트레잇은 아무 파일에서나 상속할 수 있다. 컴파일은
파일 단위로 이루어지므로 \code{sealed}가 붙지 않은 트레잇에 속하는 모든 이형을
컴파일러가 알아낼 방법이 없다. 따라서 완전성을 검사할 수 없게 된다.
\code{sealed}를 붙이는 것만으로 이 문제를 해결할 수 있다. 트레잇을 정의하는
파일에 모든 이형이 정의되어야 하므로 해당 파일만 보면 모든 이형을 찾을 수 있는
것이다. 그러므로 대수적 타입을 정의할 때는 반드시 잊지 말고 \code{sealed}를
붙이는 편이 좋다.

\subsection{도달 가능성 검사}

명령형 언어가 제공하는 \code{switch-case}와 마찬가지로 패턴 대조는 위에서 아래로
가면서 주어진 값과 패턴을 차례로 비교해 값과 일치하는 첫 번째 패턴을 찾는다.
만약 같은 패턴이 두 번 등장한다면 실행 중에 두 번째 패턴에 도달하는 일은 절대
없을 것이다. 이런 코드를 작성하는 것을 방지하기 위해 컴파일러는 \term{도달할 수
없는}{unreachable} 패턴을 발견하면 프로그래머에게 경고한다.

\begin{verbatim}
def eval(e: AE): Int = e match {
  case Num(n) => 0
  case Add(l, r) => eval(l) + eval(r)
  case Num(n) => n
  case Sub(l, r) => eval(l) - eval(r)
}
\end{verbatim}
\vspace{-1em}
\begin{mdframed}[hidealllines=true,backgroundcolor=gray!10,innerleftmargin=3pt,innerrightmargin=3pt,leftmargin=-3pt,rightmargin=-3pt]
\begin{verbatim}
  case Num(n) => n
                 ^
warning: unreachable code
\end{verbatim}
\vspace{-2em}
\begin{flushright}
\scriptsize\textsf{컴파일 시간 경고}
\end{flushright}
\end{mdframed}

코드가 짧고 간단하다면 눈으로 보는 것만으로도 도달할 수 없는 패턴이 있는지 쉽게
알 수 있다. 하지만 복잡한 코드에서는 패턴을 잘못 작성하여 도달할 수 없는 패턴을
만들 가능성이 크다. 컴파일러의 \term{도달 가능성}{reachability} 검사는 이런
문제를 예방해 준다.

\section{스칼라가 제공하는 패턴}

\subsection{상수 패턴과 와일드카드 패턴}

명령형 언어에서는 \code{switch-case}를 사용해 주어진 값을 여러 경우로 나누어
처리한다. 패턴 대조는 \code{switch-case}의 일반적인 형태다. 원래
\code{switch-case}로 하던 일을 패턴 대조로도 할 수 있다. 다음은 자바에서
\code{switch-case}를 사용한 예시 코드다.

\begin{verbatim}
String grade(int score) {
  switch (score / 10) {
    case 10: return "A";
    case 9: return "A";
    case 8: return "B";
    case 7: return "C";
    case 6: return "D";
    default: return "F";
  }
}
\end{verbatim}

같은 작업을 스칼라의 \term{상수 패턴}{constant pattern}과 \term{와일드카드
패턴}{wildcard pattern}을 사용해서 수행할 수 있다. 상수 패턴은 정수나 문자열
같은 \term{리터럴}{literal}이다. 주어진 값이 해당 상수 패턴이 나타내는 값과
같다면 값이 패턴과 일치한다. 와일드카드 패턴은 언더스코어로 적으며, 모든 값과
일치한다. 즉, \code{switch-case}의 \code{default}와 같은 역할을 한다. 다음
함수는 앞서 본 함수를 패턴 대조를 사용해 다시 구현한 것이다.

\begin{verbatim}
def grade(score: Int): String =
  (score / 10) match {
    case 10 => "A"
    case 9 => "A"
    case 8 => "B"
    case 7 => "C"
    case 6 => "D"
    case _ => "F"
  }

assert(grade(85) == "B")
\end{verbatim}

\subsection{또는 패턴}

\code{switch-case}를 사용하는 경우, 명시적으로 \code{break}를 통해
\code{switch-case}를 빠져나가지 않으면 자동으로 다음 케이스에 해당하는 코드가
이어서 실행된다. 가령 앞서 본 \code{grade} 함수는 주어진 값이 10인 경우나 9인
경우나 하는 일이 동일하므로 다음처럼 구현할 수 있다.

\begin{verbatim}
String grade(int score) {
  switch (score / 10) {
    case 10:
    case 9: return "A";
    case 8: return "B";
    case 7: return "C";
    case 6: return "D";
    default: return "F";
  }
}
\end{verbatim}

반면 패턴 대조를 사용할 때는 그런 일이 없다. 언제나 일치한 패턴에 해당하는
코드만 실행된다. 만약 같은 작업을 여러 케이스에서 수행하고 싶다면 \term{또는
패턴}{or pattern}을 사용해야 한다. 또는 패턴은 여러 패턴을 \code{|}로 구분하여
나열해 \code{[pattern] | [pattern] | ...} 형태로 적는다. \code{A | B}는
\code{A}나 \code{B} 중 적어도 하나에 일치하는 모든 값과 일치한다.

\begin{verbatim}
def grade(score: Int): String =
  (score / 10) match {
    case 10 | 9 => "A"
    case 8 => "B"
    case 7 => "C"
    case 6 => "D"
    case _ => "F"
  }

assert(grade(100) == "A")
\end{verbatim}

\subsection{중첩된 패턴}

\term{중첩된 패턴}{nested pattern}은 패턴 안에 패턴을 넣은 것을 말한다. 다음의
\code{optimizeAdd} 함수는 0을 더하는 산술식을 없앰으로써 주어진 산술식을
\term{최적화}{optimization}한다.

\begin{verbatim}
def optimizeAdd(e: AE): AE = e match {
  case Num(_) => e
  case Add(Num(0), r) => optimizeAdd(r)
  case Add(l, Num(0)) => optimizeAdd(l)
  case Add(l, r) => Add(optimizeAdd(l), optimizeAdd(r))
  case Sub(l, r) => Sub(optimizeAdd(l), optimizeAdd(r))
}
\end{verbatim}

중첩된 패턴을 사용하면 복잡한 구조를 가진 값을 쉽게 다룰 수 있다.

\subsection{변수를 정의하는 패턴}

\code{AE}에 이형 하나를 추가해 보자.

\begin{verbatim}
case class Abs(e: AE) extends AE
\end{verbatim}

\code{Abs}는 절댓값을 나타낸다. 어떤 값의 절댓값의 절댓값을 계산하는 식은 그
값의 절댓값을 계산하는 식으로 최적화할 수 있다.

\begin{verbatim}
def optimizeAbs(e: AE): AE = e match {
  case Num(_) => e
  case Add(l, r) => Add(optimizeAbs(l), optimizeAbs(r))
  case Sub(l, r) => Sub(optimizeAbs(l), optimizeAbs(r))
  case Abs(Abs(e0)) => optimizeAbs(Abs(e0))
  case Abs(e0) => Abs(optimizeAbs(e0))
}
\end{verbatim}

이 구현의 아쉬운 점은 \code{Abs(e0)}에 해당하는 값을 바로 사용하지 못하고
\code{e0}에 해당하는 값으로부터 다시 \code{Abs(e0)}를 만든다는 것이다.
\code{@}를 사용하면 특정 패턴과 일치하는 값을 변수에 저장함으로써 이 문제를
해결할 수 있다. \code{[variable] @ [pattern]}은 해당 패턴과 일치하는 값을 해당
변수에 저장한다.

\begin{verbatim}
def optimizeAbs(e: AE): AE = e match {
  case Num(_) => e
  case Add(l, r) => Add(optimizeAbs(l), optimizeAbs(r))
  case Sub(l, r) => Sub(optimizeAbs(l), optimizeAbs(r))
  case Abs(e0 @ Abs(_)) => optimizeAbs(e0)
  case Abs(e0) => Abs(optimizeAbs(e0))
}
\end{verbatim}

\subsection{타입 패턴}

\code{optimizeAbs}에서 첫 번째 패턴인 \verb!Num(_)!은 주어진 값이 \code{Num}
타입에 속하는지만 확인한다. 타입 패턴을 사용해도 같은 일을 할 수 있다. 타입
패턴은 \code{[name]: [type]} 형태로 적는다. 만약 주어진 값이 해당 타입에 속하면,
값이 패턴과 일치하며 해당 이름의 변수가 그 값을 나타낸다. 변수를 정의하고 싶지
않은 경우에는 와일드카드 패턴을 사용하여 이름 대신 언더스코어를 적으면 된다.

\begin{verbatim}
def optimizeAbs(e: AE): AE = e match {
  case _: Num => e
  case Add(l, r) => Add(optimizeAbs(l), optimizeAbs(r))
  case Sub(l, r) => Sub(optimizeAbs(l), optimizeAbs(r))
  case Abs(e0 @ Abs(_)) => optimizeAbs(e0)
  case Abs(e0) => Abs(optimizeAbs(e0))
}
\end{verbatim}

타입 패턴은 동적 타입 검사를 하는 코드를 작성할 때 유용하다. 다음 함수는 아무
값이나 인자로 받아 그 값이 문자열인지 아닌지 확인한다. 스칼라에서 \code{Any}는
모든 값이 속한 타입이다.

\begin{verbatim}
def show(x: Any): String = x match {
  case s: String => s + " is a string"
  case _ => "not a string"
}

assert(show("1") == "1 is a string")
assert(show(1) == "not a string")
\end{verbatim}

타입 패턴은 \term{다형 타입}{polymorphic type}의 \term{타입 인자}{type
argument}를 검사하지 않는다. 따라서 타입 패턴을 다형 타입과 함께 사용할 때는
특별히 주의를 기울일 필요가 있다.

\begin{verbatim}
def show(x: Any): String = x match {
  case _: List[String] => "a list of strings"
  case _ => "not a list of strings"
}
\end{verbatim}
\vspace{-1em}
\begin{mdframed}[hidealllines=true,backgroundcolor=gray!10,innerleftmargin=3pt,innerrightmargin=3pt,leftmargin=-3pt,rightmargin=-3pt]
\begin{verbatim}
          ^
warning: non-variable type argument String in type pattern
List[String] is unchecked since it is eliminated by erasure
\end{verbatim}
\vspace{-1.5em}
\begin{flushright}
\scriptsize\textsf{컴파일 시간 경고}
\end{flushright}
\end{mdframed}

\begin{verbatim}
val l: List[Int] = List(1, 2, 3)
assert(show(l) == "a list of strings")  // weird result
\end{verbatim}

인자로 주어진 값의 타입이 \code{List[Int]}이지만 \code{List[String]} 타입을
요구하는 첫 번째 패턴에 일치한다. 경고 문구를 보면 알 수 있듯이, 자바 가상
기계가 타입 인자를 실행 중에 가지고 있지 않기 때문이다.

\subsection{튜플 패턴}

튜플 패턴은 \code{([pattern], ..., [pattern])} 형태로 쓴다. 이 패턴에는 각
원소가 각 내부 패턴과 일치하는 튜플 값이 일치한다.

다음 함수는 두 리스트가 같은지 확인하기 위해 튜플 패턴을 사용한다.

\begin{verbatim}
def equal(l0: List[Int], l1: List[Int]): Boolean =
  (l0, l1) match {
    case (h0 :: t0, h1 :: t1) =>
      h0 == h1 && equal(t0, t1)
    case (Nil, Nil) => true
    case _ => false
  }
\end{verbatim}

\subsection{조건이 붙은 패턴}

\term{이진 탐색 나무}{binary search tree}는

\begin{itemize}
\item 빈 나무거나
\item 뿌리 원소인 정숫값과 두 자식 나무로 구성된 나무다.
\end{itemize}

\begin{verbatim}
sealed trait Tree
case object Empty extends Tree
case class Node(root: Int, left: Tree, right: Tree) extends Tree
\end{verbatim}

다음 함수 \code{add}는 주어진 나무에 주어진 정수를 추가한다. 만약 정수가 이미 그
나무의 원소라면 아무 일도 일어나지 않는다.

\begin{verbatim}
def add(t: Tree, n: Int): Tree =
  t match {
    case Empty => Node(n, Empty, Empty)
    case Node(m, t0, t1) =>
      if (n < m)
        Node(m, add(t0, n), t1)
      else if (n > m)
        Node(m, t0, add(t1, n))
      else
        t
  }
\end{verbatim}

두 번째 케이스에 대응되는 식은 조건식을 사용한다. 조건식을 사용하는 대신 패턴에
조건을 붙일 수 있다. 어떤 값이 \code{[pattern] if [expression]} 형태의 패턴에
일치하려면 값이 해당 패턴에 일치하면서 동시에 주어진 식을 \code{true}로 만들어야
한다. 다음처럼 패턴에 조건을 붙여서 \code{add}를 구현할 수 있다.

\begin{verbatim}
def add(t: Tree, n: Int): Tree =
  t match {
    case Empty => Node(n, Empty, Empty)
    case Node(m, t0, t1) if n < m =>
      Node(m, add(t0, n), t1)
    case Node(m, t0, t1) if n > m =>
      Node(m, t0, add(t1, n))
    case _ => t
  }
\end{verbatim}

패턴에 조건을 붙이면 패턴이 완전하지 않을 수 있기에 유의해야 한다.

\begin{verbatim}
def add(t: Tree, n: Int): Tree =
  t match {
    case Empty => Node(n, Empty, Empty)
    case Node(m, t0, t1) if n < m =>
      Node(m, add(t0, n), t1)
    case Node(m, t0, t1) if n > m =>
      Node(m, t0, add(t1, n))
  }
\end{verbatim}

위 코드의 패턴은 완전하지 않지만 컴파일러가 아무런 경고도 띄우지 않는다.

\subsection{백틱 패턴}

다음 함수 \code{remove}는 주어진 나무에서 주어진 정수를 제거한다. 만약 정수가 그
나무의 원소가 아니면 아무 일도 일어나지 않는다. \code{removeMin}은
\code{remove}를 구현하는 데 사용되는 보조 함수다. 이 함수는 주어진 나무에서 가장
작은 원소를 제거하고 제거된 원소를 알려 준다.

\begin{verbatim}
def removeMin(t: Tree): (Int, Tree) = {
  t match {
    case Node(n, Empty, t1) =>
      (n, t1)
    case Node(n, t0: Node, t1) =>
      val (min, t2) = removeMin(t0)
      (min, Node(n, t2, t1))
  }
}

def remove(t: Tree, n: Int): Tree = {
  t match {
    case Empty =>
      Empty
    case Node(m, t0, Empty) if n == m =>
      t0
    case Node(m, t0, t1: Node) if n == m =>
      val res = removeMin(t1)
      val min = res._1
      val t2 = res._2
      Node(min, t0, t2)
    case Node(m, t0, t1) if n < m =>
      Node(m, remove(t0, n), t1)
    case Node(m, t0, t1) if n > m =>
      Node(m, t0, remove(t1, n))
  }
}
\end{verbatim}

위 코드에서 \code{case Node(m, t0, Empty) if n == m} 대신 \verb!Node(`n`, t0, Empty)!
라고 쓸 수 있다. \term{백틱}{backtick} 없이 그냥 \code{Node(n, t0,
Empty)}라고 쓴 패턴은 새로운 변수 \code{n}을 정의하고 \code{n}이 뿌리 원소를
나타내도록 만든다. 뿌리 원소가 \code{n}과 같은 값인지 확인하는 패턴이 아닌
것이다. 새로운 변수를 정의하는 대신 기존의 변수 \code{n}과 뿌리 원소를
비교하려면 백틱을 사용해야 한다.

\begin{verbatim}
def remove(t: BinTree, n: Int): BinTree = {
  t match {
    case Empty =>
      Empty
    case Node(`n`, t0, Empty) =>
      t0
    case Node(`n`, t0, t1: Node) =>
      val res = removeMin(t1)
      val min = res._1
      val t2 = res._2
      Node(min, t0, t2)
    case Node(m, t0, t1) if n < m =>
      Node(m, remove(t0, n), t1)
    case Node(m, t0, t1) if n > m =>
      Node(m, t0, remove(t1, n))
  }
}
\end{verbatim}

\section{패턴 대조의 활용}

\subsection{변수 정의}

변수를 정의할 때 패턴 대조를 사용할 수 있다.

\begin{verbatim}
val (n, m) = (1, 2)
assert(n == 1 && m == 2)

val (a, b, c) = ("a", "b", "c")
assert(a == "a" && b == "b" && c == "c")

val h :: t = List(1, 2, 3, 4)
assert(h == 1 && t == List(2, 3, 4))

val Add(l, r) = Add(Num(1), Num(2))
assert(l == Num(1) && r == Num(2))
\end{verbatim}

패턴 대조는 변수를 간결하게 정의하는 것을 돕지만 만약 작성한 패턴이 우변의 식이
나타내는 값과 일치하지 않으면 실행 시간 오류가 발생한다. 따라서 패턴 대조가
성공한다는 보장이 있을 때만 패턴 대조를 사용해 변수를 정의하는 것이 좋다. 가령
튜플 패턴은 같은 길이의 튜플 값과 언제나 일치하므로 변수를 정의할 때 튜플 패턴을
사용하는 일이 흔하다.

\subsection{익명 함수}

다음 함수 \code{toSum}은 두 정수의 순서쌍들로 구성된 리스트를 받아 각 순서쌍의
두 정수의 합으로 구성된 리스트를 결과로 낸다.

\begin{verbatim}
def toSum(l: List[(Int, Int)]): List[Int] =
  l.map(p => p match {
    case (n, m) => n + m
  })

val l = List((0, 1), (2, 3), (3, 4))
assert(toSum(l) == List(1, 5, 7))
\end{verbatim}

\code{map}의 인자로 사용된 익명 함수는 매개변수 \code{p}를 곧바로 패턴 대조의
대상으로 삼는다. 스칼라에서는 \verb!x => x match { ... }! 형태의 익명 함수를
\verb!{ ... }! 형태로 짧게 적을 수 있다. 즉, 중괄호 안에 케이스를 나열한 것을
익명 함수로서 사용할 수 있는 셈이다.

\begin{verbatim}
def toSum(l: List[(Int, Int)]): List[Int] =
  l.map({ case (n, m) => n + m })
\end{verbatim}

\subsection{\code{for} 반복문}

\code{toSum}을 구현할 때 \code{map} 메서드 대신 \code{for} 식을 사용할 수 있다.

\begin{verbatim}
def toSum(l: List[(Int, Int)]): List[Int] =
  for (p <- l)
    yield p match { case (n, m) => n + m }
\end{verbatim}

\code{for} 식에서 바로 패턴 대조를 사용해도 된다.

\begin{verbatim}
def toSum(l: List[(Int, Int)]): List[Int] =
  for ((n, m) <- l)
    yield n + m
\end{verbatim}

코드가 매우 간결하면서도 이해하기 쉽다.

\section{옵션}

옵션 타입은 널리 사용되는 대수적 타입이다. 값이 존재할 수도 있고 아닐 수도
있다는 사실을 표현한다. 이 절은 옵션 타입을 소개하고 사용 방법을 설명한다.

리스트의 \code{n}번 째 원소를 내놓는 함수 \code{get}을 생각해 보자. 만약
\code{n}이 음수거나 리스트의 길이보다 크거나 같으면 문제가 생긴다. 명령형
언어에서는 대개 \term{예외}{exception}를 발생시킴으로써 이를 해결한다.
스칼라에서는 \code{throw [expression]} 형태의 식을 사용해 예외를 발생시킨다.
편의상 다음과 같이 예외를 발생시키는 \code{error} 함수를 정의하고 앞으로 계속
사용하겠다.

\begin{verbatim}
def error(msg: String) = throw new Exception(msg)

def get(l: List[Int], n: Int): Int =
  if (n < 0)
    error("index out of bounds")
  else l match {
    case Nil =>
      error("index out of bounds")
    case h :: t =>
      if (n == 0)
        h
      else
        get(t, n - 1)
  }
\end{verbatim}

예외를 발생시키는 것은 간단하고 효과적인 해결책이지만 두 가지 문제가 있다.
첫째로, 예외는 \term{예외 처리기}{exception handler}를 통해 처리해 주어야 한다.

\begin{verbatim}
try {
  get(List(1, 2), 2)
} catch {
  case e: Exception =>
    // prints "index out of bounds"
    println(e.getMessage)
}
\end{verbatim}

만약 예외가 처리되지 않으면 실행 시간 오류가 발생한다.

\begin{verbatim}
get(List(1, 2), 2)
\end{verbatim}
\vspace{-1em}
\begin{mdframed}[hidealllines=true,backgroundcolor=gray!10,innerleftmargin=3pt,innerrightmargin=3pt,leftmargin=-3pt,rightmargin=-3pt]
\begin{verbatim}
java.lang.Exception: index out of bounds
\end{verbatim}
\vspace{-2em}
\begin{flushright}
\scriptsize\textsf{실행 시간 오류}
\end{flushright}
\end{mdframed}

스칼라 컴파일러는 예외가 처리되는지 확인하지 않는다. 다시 말해 예외가 처리되지
않아 실행 시간 오류가 발생할 수 있어도 컴파일 시간 오류를 통해 미리 알 수 없다는
것이다.

두 번째 문제는 예외 처리가 지역적이지 않다는 점이다. 예외가 발생하면 실행 흐름이
갑작스럽게 가장 가까운 예외 처리기로 이동한다. 이런 비지역적인 실행 흐름의
이동은 많은 경우에 코드를 이해하기 어렵게 만든다. 그러니 \code{get}을 예외 없이
구현하는 편이 바람직하다.

한 가지 가능한 시도는 \code{null}을 사용하는 것이다. \code{null}은 어떤 객체도
가리키지 않는 값이다. 예외를 발생시키는 대신 \code{null}을 결과로 내놓는 방법을
시도해 볼 수 있다.

\begin{verbatim}
def get(l: List[Int], n: Int): Int =
  if (n < 0)
    null
  else l match {
    case Nil => null
    case h :: t =>
      if (n == 0)
        h
      else
        get(t, n - 1)
  }
\end{verbatim}
\vspace{-1em}
\begin{mdframed}[hidealllines=true,backgroundcolor=gray!10,innerleftmargin=3pt,innerrightmargin=3pt,leftmargin=-3pt,rightmargin=-3pt]
\begin{verbatim}
    null
    ^
error: an expression of type Null is ineligible
for implicit conversion

    case Nil => null
                ^
error: an expression of type Null is ineligible
for implicit conversion
\end{verbatim}
\vspace{-2em}
\begin{flushright}
\scriptsize\textsf{컴파일 시간 오류}
\end{flushright}
\end{mdframed}

안타깝게도 스칼라에서 \code{null}은 \code{Int} 타입에 속하지 않는다. 그래서
컴파일 시간 오류가 발생한다. 설사 \code{null}이 \code{Int} 타입에 속한다
하더라도 \code{null}을 사용하는 것은 사실 좋은 해결책이 아니다. \code{null}을
단순히 저장하거나 전달하는 것이 아니라, \code{null}을 의미 있는 계산에
사용하려고 하면 잘 알려진 \code{NullPointerException}이 발생한다. 컴파일러는
\code{NullPointerException}이 발생할 수 있는지 확인하지 않는다. 이런 면에서
\code{null}을 사용하는 것은 예외를 사용하는 것보다 크게 나은 점이 없다.
\code{null}을 사용하면 오류를 유발하기 매우 쉽기 때문에 \code{null}이라는 개념은
지속적으로 크게 비판받아 왔다. \code{null}의 발명자인 \term{토니 호어}{Tony
Hoare}조차 \code{null}을 만든 것이 끔찍한 실수였다고 후회했다.

\begin{quote}
I call it my billion-dollar mistake. It was the invention of the null reference
in 1965.\footnote{\url{https://en.wikipedia.org/wiki/Null\_pointer\#History}}
\end{quote}

또 다른 시도는 -1과 같이 실패를 의미하는 특별한 값을 사용하는 것이다.

\begin{verbatim}
def get(l: List[Int], n: Int): Int =
  if (n < 0)
    -1
  else l match {
    case Nil =>
      -1
    case h :: t =>
      if (n == 0)
        h
      else
        get(t, n - 1)
  }
\end{verbatim}

이 방법은 누가 봐도 문제가 있다. 함수를 호출한 쪽에서 다음 두 상황을 구별할 수 없다.

\begin{itemize}
\item 리스트가 –1을 가지고 있다.
\item 주어진 위치가 올바르지 않다.
\end{itemize}

이처럼 실패를 나타내는 특별한 값을 정해서 사용하는 것은 어떤 경우에는 통할지 몰라도 모든 경우에 통하지는 않는다.

특별한 값을 \code{get}이 미리 정해 놓는 것이 문제라면, \code{get}을 호출하는 쪽에서 기본값을 정하게 할 수도 있다.

\begin{verbatim}
def getOrElse(l: List[Int], n: Int, default: Int): Int =
  if (n < 0)
    default
  else l match {
    case Nil =>
      default
    case h :: t =>
      if (n == 0)
        h
      else
        getOrElse(t, n - 1, default)
}
\end{verbatim}

만약 \code{get}을 호출하는 목적에 부합하는 기본값이 존재한다면 이것만으로도
그럭저럭 괜찮다. 하지만 위치가 올바른지 확인하는 것 그 자체가 중요하다면 이
방법이나 이전 방법이나 비슷하게 별로다. 호출한 쪽에서 결과를 보고 리스트의
원소인지 기본값인지 구분할 수가 없다.

함수형 언어는 이처럼 계산이 실패할 수 있는 상황을 안전하게 표현하기 위해 옵션
타입을 제공한다. 옵션이라는 이름에서 느껴지듯이, 값의 존재 여부가 옵션인 상황을
표현한다. 스칼라 표준 라이브러리는 옵션 타입을 다음과 같이 정의한다.

\begin{verbatim}
sealed trait Option[A]
case object None extends Option[Nothing]
case class Some[A](value: A) extends Option[A]
\end{verbatim}

\code{T} 타입의 값을 가질 수도 있는 옵션 값은 \code{Option[T]} 타입에 속한다.
옵션 값은 \code{None}이거나 \code{Some}이다. \code{None}은 아무 값도 없는
상태로, \code{null}과 비슷하다. 계산이 실패한 상황을 나타낸다. \code{Nil}과
마찬가지로, 모든 \code{None}이 똑같기 때문에 \code{case object}로 정의되었다.
\code{Some}은 어떤 값이 존재하는 상태로, \code{null}이 아닌 객체와 비슷하며 계산이 성공한 상황을 나타낸다.

다음 코드는 결과로 옵션을 내놓는 \code{getOption} 함수를 정의한다.

\begin{verbatim}
def getOption(l: List[Int], n: Int): Option[Int] =
  if (n < 0)
    None
  else l match {
    case Nil =>
      None
    case h :: t =>
      if (n == 0)
        Some(h)
      else
        getOption(t, n - 1)
  }

assert(getOption(List(1, 2), 0) == Some(1))
assert(getOption(List(1, 2), 2) == None)
\end{verbatim}

주어진 위치가 올바르지 않으면 결괏값이 \code{None}이다. 호출한 쪽에서
\code{None}을 보면 계산이 실패했음을 바로 알 수 있다. 주어진 위치가 올바르면
원소를 \code{Some} 안에 넣어 결괏값을 만든다.

스칼라 표준 라이브러리는 옵션을 많은 곳에서 사용한다. 옵션을 내놓는 여러
메서드를 찾을 수 있다. 예를 들어 리스트의 \code{headOption} 메서드는 리스트가
비어 있으면 \code{None}을 내놓고 그렇지 않으면 머리를 \code{Some}에 넣어 내놓는다.

\begin{verbatim}
assert(List().headOption == None)
assert(List(1).headOption == Some(1))
\end{verbatim}

또, 맵의 \code{get} 메서드는 주어진 열쇠가 맵에 없으면 \code{None}을 내놓고
그렇지 않으면 열쇠에 대응되는 값을 \code{Some}에 넣어 내놓는다.

\begin{verbatim}
val m = Map(1 -> "one", 2 -> "two")
assert(m.get(0) == None)
assert(m.get(1) == Some("one"))
\end{verbatim}

패턴 대조를 사용해 주어진 옵션을 \code{None}인 경우와 \code{Some}인 경우로
나누어 처리할 수 있다. 또한, 리스트를 사용한 비슷한 작업을 쉽게 처리할 수 있도록
리스트가 여러 메서드를 제공하는 것처럼, 옵션 역시 여러 메서드를 제공한다. 그중
가장 중요한 둘은 \code{map}과 \code{flatMap}이다.

\code{map} 메서드는 이전 계산이 성공한 경우에만 계산을 이어서 하고 싶은 경우에
유용하다. \code{map}은 함수 하나를 인자로 받는다. \code{opt.map(f)}는
\code{opt}가 \code{None}이면 결과가 \code{None}이고 \code{opt}가
\code{Some(v)}이면 결과가 \code{Some(f(v))}이다.

\code{map} 메서드가 유용한 상황을 알아보기 위해 학생을 담고 있는 맵을 생각해
보자. 학생의 이름이 열쇠고 학생 객체가 연결된 값이다. 하고 싶은 일은 특정 이름을
가진 학생의 키를 알아내는 것이다. 이 작업을 \code{map} 메서드를 사용해 다음과 같이 구현할 수 있다.

\begin{verbatim}
def getHeight(
  m: Map[String, Student],
  name: String
): Option[Int] =
  m.get(name).map(_.height)
\end{verbatim}

만약 \code{m.get(name)}이 \code{None}이면 \verb!m.get(name).map(_.height)! 역시
\code{None}이다. 그렇지 않은 경우에는 \code{m.get(name)}이
\code{Some(student)}일 것이고, \verb!m.get(name).map(_.height)!는
\code{Some(student.height)}가 된다.

이처럼 계산이 두 단계로 이루어지는 데 첫 단계가 실패할 수 있을 때 \code{map}이 유용하다.

\code{flatMap}은 \code{map}과 비슷하면서도 약간 다르다. \code{flatMap}은 계산이
두 단계이면서 각 단계가 모두 실패할 수 있을 때 사용한다. \code{flatMap} 역시
함수 하나를 인자로 받는데, 그 함수는 옵션을 내놓는 함수여야 한다.
\code{opt.flatMap(f)}는 \code{opt}가 \code{None}이면 결과가 \code{None}이고
\code{opt}가 \code{Some(f)}이면 결과가 \code{f(v)}이다.

이름의 리스트와 앞 예시에서 사용한 맵이 있을 때, 리스트에서 가장 먼저 나오는
이름을 가진 학생을 찾아보자. \code{flatMap}을 사용하면 쉽게 구현할 수 있다.

\begin{verbatim}
def getStudent(
  l: List[String],
  m: Map[String, Student]
): Option[Student] =
  l.headOption.flatMap(m.get)
\end{verbatim}

표준 라이브러리는 그 밖에도 옵션을 다루기 위한 여러 메서드를
제공한다.\footnote{\url{https://www.scala-lang.org/api/current/scala/Option.html}}
