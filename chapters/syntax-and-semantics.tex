\setchapterpreamble[u]{\margintoc}
\chapter{Syntax and Semantics}
\labch{syntax-and-semantics}

\renewcommand{\lang}{\code{AE}\xspace}

This chapter is about syntax and semantics.

\textit{Syntax}\index{syntax} of a programming language decides the
appearance of the language. Syntax consists of concrete syntax and abstract
syntax. While concrete syntax describes programs as strings, abstract syntax
describes the structures of programs as trees. Parsing is the process bridging
the gap between concrete syntax and abstract syntax. A string is transformed to
a tree by parsing. This chapter explains concrete syntax, abstract syntax, and
parsing.

\textit{Semantics}\index{semantics} of a programming langauge determines the
behavior of each program. This chapter explains how we can define the semantics
of a language. In addition, we will see what syntactic sugar is.

\section{Concrete Syntax}

People write programs with strings. Some strings are valid programs, while other
strings are not. For example, consider the following code:

\begin{verbatim}
println()
\end{verbatim}

It is a valid Scala program. On the other hand, the following code is not a
valid Scala program.

\begin{verbatim}
println(
\end{verbatim}

\textit{Concrete syntax}\index{concrete syntax} determines whether a certain string is a program or not.
According to the concrete syntax of Scala, \code{println()} is a
program, but \code{println(} is not because the closing
parenthesis is missing. Without concrete syntax, programmers cannot know whether
a given string is a program or not. Concrete syntax is one of the essential
elements of a programming language. This section defines concrete syntax
formally and explains how concrete syntax can be specified.

The first thing to do is to define strings since programs are represented as
strings. A string is a finite sequence of characters. The definition of a
character varies in programming languages. In some languages, a character is
limited to those expressed by the ASCII code. In other cases, every Unicode symbol
is considered as a character. As we want to deal with general programming
languages, we do not fix the characters to be a specific set. Instead, we assume
that a set of characters is given and do not care about which characters exactly
exist in the set. From now on, $C$ is the set of every character:

\[C = \text{the set of every character} = \{ c\ |\ c\text{ is a character} \}\]

Now, a character is an element of $C$.

Once characters are defined, we can define strings with the definition of
characters. $S$ is the set of every string:

\[S = \text{the set of every string} = \{ \code{"}c_1\cdots c_n\code{"}\ |\
c_1,\ldots,c_n\in C \}\]

A string is an element of $S$, which is a
sequence of zero or more characters. For example, if $\code{'a'},\code{'b'}\in
C$, then $\code{"aba"}\in S$.

The definition of strings differs from the definition of programs because some
strings are not programs. As we have seen already, $\code{"println()"}$ is a program,
but $\code{"println("}$ is not although both are strings.

Defining concrete syntax is to define which strings are programs. Therefore,
we can say that concrete syntax determines the set of every program.
Let $P$ be the set of every program.

\[P = \text{the set of every program} = \{ p\ |\ p\text{ is a program} \}\]

A program is an element of $P$.
Then, $P$ has only one restriction: $P$ should be a subset of $S$, as
every program is a string.

\[P\subseteq S\]

The concrete syntax of each language defines its own $P$ as a subset of
$S$. Each language has different $P$ from each other. For instance,
$\code{"println()"} \in P$ and $\code{"println("}\not\in P$,
where $P$ is defined by the concrete syntax of Scala. At the same
time, there can be a language whose $P$ does not have $\code{"println()"}$ as an element.

To define the concrete syntax of a language, we need to define the set $P$. The
problem is that $P$ is usually an infinite set. There are infinitely many
programs in each language. Defining an infinite set is difficult because we
cannot enumerate every element of an infinite set. We need a way to define
infinite sets to define concrete syntax.

The most popular way to define concrete syntax is \textit{Backus-Naur
form}\index{Backus-Naur form} (\acrshort{bnfLabel}). It defines a set of strings
in an intuitive way. It provides a constructive definition, i.e. it lets us know
how we can construct an element of the set. To define concrete syntax with BNF,
we need to discuss the concepts used in BNF first.

BNF has three concepts: terminals, nonterminals, and expressions. A
\textit{terminal}\index{terminal} is a string. For example, \code{"0"} and
\code{"ab"} are terminals. A \textit{nonterminal}\index{nonterminal} is a name
between a pair of angle brackets and denotes a set of strings. For instance,
\code{<digit>} is a nonterminal and may denote the set
$\{\code{"0"},\code{"1"},\code{"2"},\code{"3"},\code{"4"},\code{"5"},\code{"6"},\code{"7"},\code{"8"},\code{"9"}\}$.
An \textit{expression}\index{expression} is an enumeration of one or more
terminals/nonterminals. Therefore, all of the following are expressions:

\begin{itemize}
  \item \code{"abc"} (a single terminal)
  \item \code{"0"} \code{"1"} (multiple terminals)
  \item \code{<digit>} (a single nonterminal)
  \item \code{<digit>} \code{<number>} (multiple nonterminals)
  \item \code{"-"} \code{<number>} (a single terminal and a single nonterminal)
\end{itemize}

An expression also denotes a set of strings. The set denoted by an expression is
the concatenation of strings denoted by its components. For example, in the
expression \code{"0"} \code{"1"}, \code{"0"} denotes \code{"0"}, and \code{"1"}
denotes \code{"1"}. Therefore, \code{"0"} \code{"1"} denotes the set $\{
  \code{"01"} \}$. If \code{<digit>} denotes the set
$\{\code{"0"},\code{"1"},\code{"2"},\code{"3"},\code{"4"},\code{"5"},\code{"6"},\code{"7"},\code{"8"},\code{"9"}\}$,
then \code{"0"} \code{<digit>} denotes
$\{\code{"00"},\code{"01"},\code{"02"},\code{"03"},\code{"04"},\code{"05"},\code{"06"},\code{"07"},\code{"08"},\code{"09"}\}$.

Now, let us see how we can define a new set from scratch. BNF allow us to define
the set denoted by a nonterminal with the following form:

\begin{verbatim}
[nonterminal] ::= [expression] | [expression] | …
\end{verbatim}

The vertical bars in the right hand side separate distinct expressions.
The expressions define the set denoted by the nonterminal. The union of the set
denoted by the expressions equals the set denoted by the nonterminal. For
example, the following definition makes \code{<digit>} denote
$\{\code{"0"},\code{"1"},\code{"2"},\code{"3"},\code{"4"},\code{"5"},\code{"6"},\code{"7"},\code{"8"},\code{"9"}\}$:

\begin{verbatim}
<digit> ::= "0" | "1" | "2" | "3" | "4"
          | "5" | "6" | "7" | "8" | "9"
\end{verbatim}

From now on, we are going to define the concrete syntax of a tiny language named
\lang to show example usage of BNF. \lang stands for arithmetic expressions. Its
features is limited to addition and subtraction of decimal integers.

\lang programs should be able to express decimal integers. Thus, the
following strings should be programs of \lang:

\begin{itemize}
  \item \code{"0"}
  \item \code{"1"}
  \item \code{"-10"}
  \item \code{"42"}
\end{itemize}

At the same time, programs should be able to express addition and subtraction.
Thus, the following strings also should be programs:

\begin{itemize}
  \item \code{"0+1"}
  \item \code{"-2-1"}
  \item \code{"1+-3+42"}
  \item \code{"4-3+2-1"}
\end{itemize}

First, we can define the set of every string that represents a decimal integer
in BNF. It can be done with the following definitions:

\begin{verbatim}
<digit>  ::= "0" | "1" | "2" | "3" | "4"
           | "5" | "6" | "7" | "8" | "9"
<nat>    ::= <digit> | <digit> <nat>
<number> ::= <nat> | "-" <nat>
\end{verbatim}

We know that \code{<digit>} denotes
$\{$\code{"0"}, \code{"1"}, \code{"2"}, \code{"3"}, \code{"4"},
\code{"5"}, \code{"6"}, \code{"7"}, \code{"8"}, \code{"9"}$\}$.
Since \code{<digit>} is one way to construct an element of \code{<nat>}, every
string denoted by \code{<digit>} is also a string of \code{<nat>}. Hence,
$\{$\code{"0"}, \code{"1"}, \code{"2"}, \code{"3"}, \code{"4"},
\code{"5"}, \code{"6"}, \code{"7"}, \code{"8"}, \code{"9"}$\}$
is a subset of the set denoted by \code{<nat>}. At the same time, \code{<digit>}
\code{nat} is the other way to construct an element. We can make a new element
of \code{<nat>} by selecting strings from \code{<digit>} and \code{<nat>}
respectively. For example, \code{<digit>} can denote \code{"1"}, and
\code{<num>} can denote \code{"0"}. Therefore, \code{"10"} is an element of
\code{<num>}. By repeating this process, we can construct infinitely many
strings, e.g. \code{"110"} by concatenating \code{"1"} and \code{"10"},
\code{"1110"} by concatenating \code{"1"} and \code{"110"}, and so on. In the
end, we can conclude that \code{<nat>} denotes the set of every string that
consists of the characters from \code{'0'} to \code{'9'}, i.e. every string that
represents a decimal natural number.\sidenote{This book considers zero as a
natural number.}

Finding the set denoted by \code{<number>} is easier. Since \code{<nat>} is one
way to construct an element of \code{<number>}, the set denoted by \code{<nat>}
is a subset of the set denoted by \code{<number>}. The expression \code{"-"} \code{<nat>} is
the other way to construct an element. It implies that if we concatenate
\code{"-"} and a string denoted by \code{<nat>}, we can get a new element of
\code{<number>}. In conclusion, \code{<number>} denotes the set of every string
that represents a decimal integer.

It is enough to add only addition and subtraction to complete the definition of
the concrete syntax.

\begin{verbatim}
<expr> ::= <number> | <expr> "+" <expr> | <expr> "-" <expr>
\end{verbatim}

In a similar way, we can figure out which set is denoted by \code{<expr>}. The
set includes every string that represents arithmetic expression consisting of
decimal integers, addition, and subtraction. We can say that \code{<expr>}
defines $P$ of \lang, and the concrete syntax of \lang has been defined now.

\section{Abstract Syntax}

Defining syntax solely with concrete syntax has problems from both language
users' and language designers' perspectives.

Programmers usually learn multiple languages. Languages vary in their concrete
syntax a lot. Consider a function that takes two integers as arguments and
returns the sum of the integers. We can implement the function in four different
languages like below.

\begin{itemize}
  \item Python

\begin{verbatim}
def add(n, m):
    return n + m
\end{verbatim}

  \item JavaScript

\begin{verbatim}
function add(n, m) {
    return n + m;
}
\end{verbatim}

  \item Racket

\begin{verbatim}
(define (add n m) (+ n m))
\end{verbatim}

  \item OCaml

\begin{verbatim}
let add n m = n + m
\end{verbatim}
\end{itemize}

They look so different even though they define the same function. The
keyword to define a function is \code{def} in Python, \code{function} in
JavaScript, \code{define} in Racket, and \code{let} in OCaml. It is not
the only difference. Python and JavaScript need parentheses and commas
for parameters, while Racket and OCaml do not. JavaScript puts function
bodies inside curly braces. Racket treats \code{+} as a prefix operator,
while the others treat \code{+} as a infix operator. These differences
are the differences between the concrete syntax of each language.
Various forms of concrete syntax hinder programmers from learning
multiple languages easily.

However, their structures are quite the same. In every language, a
function definition consists of a name, parameters, and a body
expression. The above example defines a function whose name is \code{add},
parameters are \code{n} and \code{m}, and body is an expression that
adds \code{n} and \code{m}. In every language, an addition expression
consists of two operands. The body expression uses \code{n} and \code{m}
as the operands of the addition.

Thus, programmers should focus on the structures of programs, rather
than strings per se, to learn multiple languages easily. The structures
remain the same even when the strings vary.

At the same time, concrete syntax cares about tedious details that
language designers want to ignore. For example, both \code{"2+1"} and
\code{"02+001"} are \lang programs. They are different strings but
represent the same arithmetic expression: $2+1$. When the designers of
\lang define logic to evaluate arithmetic expressions, distinction
between $2+1$ and $2-1$ is important, but distinction between
\code{"2+1"} and \code{"02+001"} is completely unnecessary. The
designers want to focus only on the structures of programs but not
strings.

For both programmers and designers, concrete syntax is problematic
because it describes only strings and does not give good abstraction of
structures even though people want to focus on the structures. Of
course, we cannot discard the notion of concrete syntax. Everyone write
programs as strings, and concrete syntax is essential for that step. At
the same time, we need a way to describe the structures of programs
without being affected by differences in strings. To meet the need, we
introduce another notion of syntax: abstract syntax. Concrete syntax and
abstract syntax are complementary. They collectively construct the
syntax of a language.

\textit{Abstract syntax}\index{abstract syntax} describes the structure of a program as a tree. A program
consists of multiple components. Each component consists of subcomponents again.
Trees formally express such recursive structures. A component can be represented
as a tree whose root describes the sort of a component and children are the
trees representing the subcomponents.

As an example, let us express the function \code{add} as a tree. The function
definition has four components: the name \code{add}, the first parameter
\code{n}, the second parameter \code{m}, and the body expression. The following
tree represents the function definition:

\begin{center}
\begin{tikzpicture}[
    level distance=1.2cm,
    level 1/.style={sibling distance=1.2cm},
]
  \node [arn_u] {\scriptsize FunDef}
  child{
    node [arn_u] {\code{add}}
  }
  child{
    node [arn_u] {\code{n}}
  }
  child{
    node [arn_u] {\code{m}}
  }
  child{
    node [arn_w] {}
  };
\end{tikzpicture}
\end{center}

The root of the tree is the symbol FunDef, which explains that this tree
represents a function definition. The tree has four children: \code{add},
\code{n}, \code{m}, and the body expression. We do not know how to draw the tree
representing the body expression yet.

The body expression is an addition expression. It has two components: the
operands of the addition.

\begin{center}
\begin{tikzpicture}[
    level distance=1.2cm,
    level 1/.style={sibling distance=1.2cm},
    level 2/.style={sibling distance=1.2cm}
]
  \node [arn_u] {\scriptsize FunDef}
  child{
    node [arn_u] {\code{add}}
  }
  child{
    node [arn_u] {\code{n}}
  }
  child{
    node [arn_u] {\code{m}}
  }
  child{
    node [arn_u] {\small Add}
    child{
      node [arn_w] {}
    }
    child{
      node [arn_w] {}
    }
  };
\end{tikzpicture}
\end{center}

The root of the tree is Add, as the expression is addition. It has two
children: the operand expressions.

The first operand expression consists of a single component: \code{n}.

\begin{center}
\begin{tikzpicture}[
    level distance=1.2cm,
    level 1/.style={sibling distance=1.2cm},
    level 2/.style={sibling distance=1.2cm}
]
  \node [arn_u] {\scriptsize FunDef}
  child{
    node [arn_u] {\code{add}}
  }
  child{
    node [arn_u] {\code{n}}
  }
  child{
    node [arn_u] {\code{m}}
  }
  child{
    node [arn_u] {\small Add}
    child{
      node [arn_u] {\small Name}
      child{
        node [arn_u] {\code{n}}
      }
    }
    child{
      node [arn_w] {}
    }
  };
\end{tikzpicture}
\end{center}

The root of the tree is Name, since the expression is just a name. The
only child is \code{n}.

The second operand expression can be similarly represented as a tree.

\begin{center}
\begin{tikzpicture}[
    level distance=1.2cm,
    level 1/.style={sibling distance=1.2cm},
    level 2/.style={sibling distance=1.2cm}
]
  \node [arn_u] {\scriptsize FunDef}
  child{
    node [arn_u] {\code{add}}
  }
  child{
    node [arn_u] {\code{n}}
  }
  child{
    node [arn_u] {\code{m}}
  }
  child{
    node [arn_u] {\small Add}
    child{
      node [arn_u] {\small Name}
      child{
        node [arn_u] {\code{n}}
      }
    }
    child{
      node [arn_u] {\small Name}
      child{
        node [arn_u] {\code{m}}
      }
    }
  };
\end{tikzpicture}
\end{center}

The above tree represents the structure of the function definition. It is
independent of its underlying programming language. The tree can be a Python
function definition and a JavaScript function definition at the same time. By
expressing programs with trees, we can ignore unnecessary details in strings and
focus on the structures of programs.

As abstract syntax treats programs as trees, defining the abstract syntax of a
language is to define the set of every tree that represents a program. Let us
define the abstract syntax of \lang. In \lang, every natural number is a
program. A natural number program has only one component: the natural number
itself. Therefore, the following fact is true, where $E$ denotes the set of
every program:

\begin{center}
\begin{tikzpicture}[
    level distance=1.2cm,
    level 1/.style={sibling distance=1.2cm},
]
  \node at (-2,-0.6) {$\text{If }n\in\mathbb{Z}\text{, then}$};
  \node [arn_u] at (0,0) {\small Num}
  child{
    node [arn_u] {$n$}
  };
  \node at (1,-0.6) {$\in E$.};
\end{tikzpicture}
\end{center}

Addition of two arithmetic expressions is also a program. Such a program has
two components: the left and right operands. Each operand is an arithmetic
expression and thus a program. Therefore, the following fact is true:

\begin{center}
\begin{tikzpicture}[
    level distance=1.2cm,
    level 1/.style={sibling distance=1.2cm},
]
  \node at (-2.3,-0.6) {$\text{If }e_1,e_2\in T\text{, then}$};
  \node [arn_u] at (0,0) {\small Add}
  child{
    node [arn_nc] {$e_1$}
  }
  child{
    node [arn_nc] {$e_2$}
  };
  \node at (1.2,-0.6) {$\in E$.};
\end{tikzpicture}
\end{center}

Subtraction is similar.

\begin{center}
\begin{tikzpicture}[
    level distance=1.2cm,
    level 1/.style={sibling distance=1.2cm},
]
  \node at (-2.3,-0.6) {$\text{If }e_1,e_2\in E\text{, then}$};
  \node [arn_u] at (0,0) {\small Sub}
  child{
    node [arn_nc] {$e_1$}
  }
  child{
    node [arn_nc] {$e_2$}
  };
  \node at (1.2,-0.6) {$\in E$.};
\end{tikzpicture}
\end{center}

By collecting all the above facts, we can define the abstract syntax of \lang as
the smallest set $E$ satisfying the following conditions.

\begin{center}
\begin{tikzpicture}[
    level distance=1.2cm,
    level 1/.style={sibling distance=1.2cm},
]
  \node at (-2,-0.6) {$\text{If }n\in\mathbb{Z}\text{, then}$};
  \node [arn_u] at (0,0) {\small Num}
  child{
    node [arn_u] {$n$}
  };
  \node at (1,-0.6) {$\in E$.};
\end{tikzpicture}
\end{center}

\begin{center}
\begin{tikzpicture}[
    level distance=1.2cm,
    level 1/.style={sibling distance=1.2cm},
]
  \node at (-2.3,-0.6) {$\text{If }e_1,e_2\in E\text{, then}$};
  \node [arn_u] at (0,0) {\small Add}
  child{
    node [arn_nc] {$e_1$}
  }
  child{
    node [arn_nc] {$e_2$}
  };
  \node at (1.2,-0.6) {$\in E$.};
\end{tikzpicture}
\end{center}

\begin{center}
\begin{tikzpicture}[
    level distance=1.2cm,
    level 1/.style={sibling distance=1.2cm},
]
  \node at (-2.3,-0.6) {$\text{If }e_1,e_2\in E\text{, then}$};
  \node [arn_u] at (0,0) {\small Sub}
  child{
    node [arn_nc] {$e_1$}
  }
  child{
    node [arn_nc] {$e_2$}
  };
  \node at (1.2,-0.6) {$\in E$.};
\end{tikzpicture}
\end{center}

For example, the following tree represents an \lang program:

\begin{center}
\begin{tikzpicture}[
    level distance=1.2cm,
    level 1/.style={sibling distance=1.8cm},
    level 2/.style={sibling distance=1.2cm},
]
  \node [arn_u] {\small Add}
  child{
    node [arn_u] {\small Sub}
    child{
      node [arn_u] {\small Num}
      child{
        node [arn_u] {$5$}
      }
    }
    child{
      node [arn_u] {\small Num}
      child{
        node [arn_u] {$1$}
      }
    }
  }
  child{
    node [arn_u] {\small Num}
    child{
      node [arn_u] {$3$}
    }
  };
\end{tikzpicture}
\end{center}

We call a tree that is an element of the set defined by abstract syntax an
\textit{abstract syntax tree}\index{abstract syntax tree} (\acrshort{astLabel}).

Abstract syntax can be easily implemented with ADTs in Scala. The following code
implements the abstract syntax of \lang:

\begin{verbatim}
sealed trait AE
case class Num(value: Int) extends AE
case class Add(left: AE, right: AE) extends AE
case class Sub(left: AE, right: AE) extends AE
\end{verbatim}

\code{Num($n$)} corresponds to

\begin{center}
\begin{tikzpicture}[
    level distance=1.2cm,
    level 1/.style={sibling distance=1.2cm},
]
  \node [arn_u] {\small Num}
  child{
    node [arn_u] {$n$}
  };
\end{tikzpicture}
\end{center}

\code{Add($e_1$, $e_2$)} corresponds to

\begin{center}
\begin{tikzpicture}[
    level distance=1.2cm,
    level 1/.style={sibling distance=1.2cm},
]
  \node [arn_u] {\small Add}
  child{
    node [arn_nc] {$e_1$}
  }
  child{
    node [arn_nc] {$e_2$}
  };
\end{tikzpicture}
\end{center}

\code{Sub($e_1$, $e_2$)} corresponds to

\begin{center}
\begin{tikzpicture}[
    level distance=1.2cm,
    level 1/.style={sibling distance=1.2cm},
]
  \node [arn_u] {\small Sub}
  child{
    node [arn_nc] {$e_1$}
  }
  child{
    node [arn_nc] {$e_2$}
  };
\end{tikzpicture}
\end{center}

The previous AST can be written as \code{Add(Sub(Num(3),
Num(1)), Num(3))} in Scala.

It is inconvenient to draw a tree every time we need to express a program. For
this reason, people usually use notations that look like code to represent
trees. For example, we can simply write $n$ instead of drawing a tree whose root
is Num and only child is Num when it is clear that $n$ denotes an AST
from the context. Similarly, we can use $e_1+e_2$ and $e_1-e_2$
instead of trees that represent addition and subtraction respectively. Note that
$+$ and $-$ in the notations are not the mathematical addition and subtraction
operators. They are just parts of the notations and do not have any meaning.

We can define the abstract syntax of \lang again by using the above notations.
$E$ is the smallest set satisfying the following conditions:

\begin{itemize}
  \item If $n\in\mathbb{Z}$, then $n\in E$.
  \item If $e_1,e_2\in E$, then $e_1+e_2\in E$.
  \item If $e_1,e_2\in E$, then $e_1-e_2\in E$.
\end{itemize}

Even though the notations themselves do not look like trees at all, they still
represent ASTs. Also, symbols like $+$ and $-$ do not have any meaning. It
is extremely important to keep these points in your mind. Otherwise, you will
mix abstract syntax using notations up with concrete syntax in the end.

Notations are just notations. You can define different notations and use them.
For example, one may use $\embox{ADD}\ e_1\ e_2$ instead of $e_1 + e_2$ to represent
addition. You can freely choose notations, but once you define them, you should
consistently use them not to make other peoples confused.

To make the definition of abstract syntax more concise, we adopt BNF to the
definition of abstract syntax. We can re-define the abstract syntax of \lang with
BNF:

\[e\ ::=\ n\ |\ e+e\ |\ e-e\]

We call each symbol that denotes a particular element in abstract syntax a
\textit{metavariable}.\index{metavariable}
It is called \textbf{meta}variable because it is a variable at the
meta-level, not the level of the defined programming language. For example, $e$
is a metavariable that ranges over programs, and $n$ is a metavariable that
ranges over integers.

We often use parentheses to express elements of abstract syntax without
ambiguity. For instance, $3-1+2$ can be interpreted in two ways:

\begin{center}
\begin{tikzpicture}[
    level distance=1.2cm,
    level 1/.style={sibling distance=1.8cm},
    level 2/.style={sibling distance=1.2cm},
]
  \node [arn_u] at (-3,0) {\small Add}
  child{
    node [arn_u] {\small Sub}
    child{
      node [arn_u] {\small Num}
      child{
        node [arn_u] {$3$}
      }
    }
    child{
      node [arn_u] {\small Num}
      child{
        node [arn_u] {$1$}
      }
    }
  }
  child{
    node [arn_u] {\small Num}
    child{
      node [arn_u] {$2$}
    }
  };
  \node at (0,-2) {or};
  \node [arn_u] at (3,0) {\small Sub}
  child{
    node [arn_u] {\small Num}
    child{
      node [arn_u] {$3$}
    }
  }
  child{
    node [arn_u] {\small Add}
    child{
      node [arn_u] {\small Num}
      child{
        node [arn_u] {$1$}
      }
    }
    child{
      node [arn_u] {\small Num}
      child{
        node [arn_u] {$2$}
      }
    }
  };
\end{tikzpicture}
\end{center}

If we write $(3-1)+2$, it is clear that it denotes the former. Otherwise, we
write $3-(1+2)$ to denote the latter.

\section{Parsing}

Concrete syntax considers programs as strings, while abstract syntax considers
programs as trees. Parsing bridges this gap. \textit{Parsing}\index{parsing} is
a process of transforming a string following concrete syntax into an AST. A
parser is a program that parses input. We can consider a parser as a partial
function from $S$ (the set of every string) to $E$ (the set of
every AST).

\[\embox{parse}: S\pto E\]

\begin{kaobox}[frametitle=Partial functions]
  A partial function from a set $A$ to a set $B$ is a function from a subset $S$ of
  $A$ to $B$. $S$ is called the domain of definition, or just domain in short, of the partial function.
  While $A\rightarrow B$ is a set of functions from $A$ to $B$, $A\pto B$ is a set
  of partial functions from $A$ to $B$.

  Let $f$ be a partial function from $A$ to $B$. Then, there can be $a\in A$ such
  that $f(a)$ is undefined. From a programmers' perspective, $f$ can be
  interpreted as a function from $A$ to \code{Option[$B$]}, where \code{None}
  means that the image is undefined and \code{Some($b$)} means that the image is
  $b$.
\end{kaobox}

The result of $\embox{parse}$
is undefined when the input does not belong to $P$ (the set of every
program). That is why $\embox{parse}$ is a partial function. When the input
belongs to $P$, $\embox{parse}$ results in the corresponding AST.

Consider the parser of \lang. The results of $\embox{parse}$ are undefined for
the following strings, as they are not \lang programs:

\begin{itemize}
  \item \code{1+}
  \item \code{2*4}
  \item \code{0++3}
\end{itemize}

On the other hand, below is an example of when $\embox{parse}$ succeeds.

\begin{center}
\begin{tikzpicture}[
    level distance=1.2cm,
    level 1/.style={sibling distance=1.2cm},
]
  \node at (-2.5,-1.2) {$\embox{parse}(\code{"-1+002"})=$};
  \node [arn_u] at (0,0) {\small Add}
  child{
    node [arn_u] {\small Num}
    child{
      node [arn_u] {$-1$}
    }
  }
  child{
    node [arn_u] {\small Num}
    child{
      node [arn_u] {$2$}
    }
  };
  \node at (1.8,-1.2) {$=-1+2$};
\end{tikzpicture}
\end{center}

This book does not discuss implementation of parsers.

\section{Semantics}

Syntax is an essential element of a programming language. It allows us to know
which strings are programs and what the structures of programs are. However,
syntax does not explain execution of programs. Programmers write programs to
execute them. They should know what will happen when their programs are
executed. Therefore, we need semantics in addition to syntax. Semantics is the
other essential element of a programming language. It defines the behaviors of
programs.

Let us define the semantics of \lang. Semantics is defined based on abstract
syntax. The structure of a program determines its behavior. Since abstract
syntax represents the structure, it is natural to use abstract syntax for
semantics. The semantics of \lang defines the semantics of each \lang program,
where the semantics of a program means things that happen when the program is
executed. When an \lang program is executed, it does one thing: outputs the
result of the evaluation of the arithmetic expression.
For example, $0+1$ should result
in $1$ if the semantics is defined correctly. Note that $+$ in $0+1$ does not
mean addition, and we cannot say anything about the result of $0+1$ until the
semantics is defined. To make \lang a reasonable language, we must define the
semantics of \lang so that $0+1$ results in $1$.

There are infinitely many programs. We cannot define the semantics of each
program separately. We need to utilize the structures of programs defined by the
abstract syntax. According to the abstract syntax, programs can split into three
groups: $n$, $e_1+e_2$, and $e_1-e_2$. By defining the semantics of each group
at once, we can complete the semantics of infinitely many programs in a finite
method.

The simplest case is $n$. $n$ is an expression consisting of an integer. An
integer evaluates to itself. We represent this semantics as the following rule:

\semanticrule{Num}{
$n$ evaulates to $n$.
}

We can conclude the following facts by using Rule \textsc{Num}.

\begin{itemize}
  \item $1$ evaulates to $1$.
  \item $5$ evaulates to $5$.
\end{itemize}

The next case is $e_1+e_2$. As $e_1$ is an arithmetic expression, it results in
some integer. Let $n_1$ be the integer. Similarly, $e_2$ also results in some
integer. Let $n_2$ be the integer. Then, the result of $e_1+e_2$ is the sum of
$n_1$ and $n_2$. In this chapter, we use $\iadd$ instead of $+$ to denote
mathematical addition. It will help you distinguish mathematical addition from
$+$ used for the abstract syntax. Once you become familiar with syntax and
semantics, you can easily distinguish them by checking the context even if both
are denoted by $+$. From the next chapter, we will use $+$ for both abstract
syntax and mathematical addition. The following rule defines the semantics of
$e_1+e_2$.

\semanticrule{Add}{
If $e_1$ evaluates to $n_1$, and $e_2$ evaluates to $n_2$,\\
then $e_1+e_2$ evaluates to $n_1\iadd n_2$.
}

We can define the semantics of $e_1-e_2$ in a similar way. Like $\iadd$,
we use $\isub$ for mathematical subtraction in this chapter. The
following rule defines the semantics of $e_1-e_2$.

\semanticrule{Sub}{
If $e_1$ evaluates to $n_1$, and $e_2$ evaluates to $n_2$,\\
then $e_1-e_2$ evaluates to $n_1\isub n_2$.
}

These three rules are all of the semantics of \lang. We now know the behavior of
every \lang program. For example, consider $(3-1)+2$. The following steps
show why $(3-1)+2$ evaluates to $4$.

\begin{enumerate}
  \item (By Rule \textsc{Num}) $3$ evaluates to $3$.
  \item (By Rule \textsc{Num}) $1$ evaluates to $1$.
  \item (By Rule \textsc{Sub}) If $3$ evaluates to $3$ and $1$ evaluates to $1$, then $3-1$
    evaluates to $2$.
  \item (By 1, 2, and 3) $3-1$ evaluates to $2$.
  \item (By Rule \textsc{Num}) $2$ evaluates to $2$.
  \item (By Rule \textsc{Add}) If $3-1$ evaluates to $2$ and $2$ evaluates to $2$, then
    $(3-1)+2$ evaluates to $4$.
  \item (By 4, 5, and 6) $(3-1)+2$ evaluates to $4$.
\end{enumerate}

Now, let us define the semantics of \lang in a more mathematical way. The
semantics defines the result of the execution of each program. Here, the result
is an integer. We can say that semantics outputs an integer when a program is
given. Thus, the semantics can be considered as a function from a program to an
integer.

\[\embox{eval}:E\rightarrow \mathbb{Z}\]

For each $e\in E$, there should exist a unique integer $\embox{eval}(e)$. It
is obviously true in \lang. Every arithmetic expression evaluates to a unique
integer.

However, defining semantics as a function is a bad choice in other languages.
Some programs do not produce any results. Nonterminating programs are such
examples. Programs that incur run-time errors also belong to this category. You
will see programs with run-time errors in the next chapter. Moreover, there is a
program whose result is not unique. We call such programs nondeterministic
programs. For example, the behavior of a concurrent program with multiple
threads depends on how the threads are interleaved during execution. If the
threads are interleaved differently, the result may change. Programs without
results and nondeterministic programs prevent us from defining semantics as a
function. We should define semantics as a relation. Even though the semantics of
\lang can be defined as a function, we define the semantics as a relation to
make the discussion of this chapter easily extendable to other languages.

We define the semantics of \lang as $\Rightarrow$, a binary relation over $E$
and $\mathbb{Z}$.

\begin{kaobox}[frametitle=Binary relations]
A binary relation over sets $A$ and $B$ is a subset of $A\times B$,
where $A\times B=\{(a,b)\ |\ a\in A\land b\in B\}$.

Let $R$ be a binary relation over $A$ and $B$. Then, $R \subseteq A\times B$. For $a\in
A$ and $b\in B$, we write $a\ R\ b$ when $(a,b)\in R$. For example, $<$ is a
binary relation over $\mathbb{Z}$ and $\mathbb{Z}$, and we can write $1<2$ instead of
$(1,2)\in<$.
\end{kaobox}

\[\Rightarrow\subseteq E\times\mathbb{Z}\]

$(e,n)\in\Rightarrow$, i.e. $e\Rightarrow n$ implies that $e$ evaluates to $n$.

Let us define the semantics again with mathematical concepts.

\semanticrule{Num}{
$n\Rightarrow n$.
}

\vspace{-1em}

\semanticrule{Add}{
If $e_1\Rightarrow n_1$ and $e_2\Rightarrow n_2$,\\
then $e_1+e_2\Rightarrow n_1\iadd n_2$.
}

\vspace{-1em}

\semanticrule{Sub}{
If $e_1\Rightarrow n_1$ and $e_2\Rightarrow n_2$,\\
then $e_1-e_2\Rightarrow n_1\isub n_2$.
}

We use one more mathematical concept: inference rules. An \textit{inference
rule}\index{inference rule} is a rule to prove a new proposition from given
propositions. An inference rule has the following form:

\[
  \inferrule
  { \embox{premise}_1 \\ \embox{premise}_2 \\ \cdots \\ \embox{premise}_n }
  { \embox{conclusion} }
\]

It consists of a horizontal line, propositions above the line, and a proposition
below the line. We call the propositions above the line the
\textit{premises}\index{premise} and the proposition below the line the
\textit{conclusion}\index{conclusion}. The rule means that if every premise is
true, then also the conclusion is true. A single inference rule can have zero or
more premises. A rule without premises implies that its conclusion is always
true. When a rule does not have any premises, we can omit the horizontal line.

Let us define the semantics of \lang with inference rules.

\[
  n\Rightarrow n
  \quad\textsc{[Num]}
\]

\[
  \inferrule
  { e_1\Rightarrow n_1 \\ e_2\Rightarrow n_2 }
  { e_1+e_2\Rightarrow n_1\iadd n_2 }
  \quad\textsc{[Add]}
\]

\[
  \inferrule
  { e_1\Rightarrow n_1 \\ e_2\Rightarrow n_2 }
  { e_1-e_2\Rightarrow n_1\isub n_2 }
  \quad\textsc{[Sub]}
\]

As you can see, the rules are much clearer and more concise than the rules
written in a natural language.

We can prove $(3-1)+2\Rightarrow4$ with the rules. We usually draw a proof tree
when we prove a proposition with inference rules. A \textit{proof
tree}\index{proof tree} is a tree whose root is the proposition to be proven.
Each node of the tree is a proposition, and the children node of a node are
evidences supporting that the proposition of the node is true. Unlike most trees in
computer science, we place the root of a proof tree at the bottom. Every node is
placed below its children.

The following proof tree proves $3\Rightarrow3$.

\[3\Rightarrow3\]

The tree has only the root node because Rule \textsc{Num} does not have any
premises.

Similarly, the following proof tree proves $1\Rightarrow1$.

\[1\Rightarrow1\]

We draw the following proof tree with Rule \textsc{Sub} and the above trees
to prove $3-1\Rightarrow2$.

\[
  \inferrule
  { 3\Rightarrow3 \\ 1\Rightarrow1 }
  { 3-1\Rightarrow2 }
\]

By using Rule \textsc{Num} again, we prove $2\Rightarrow2$.

\[2\Rightarrow2\]

Finally, we get the proof tree of $(3-1)+2\Rightarrow4$.

\[
  \inferrule
  {
    \inferrule
    { 3\Rightarrow3 \\ 1\Rightarrow1 }
    { 3-1\Rightarrow2 }
    \\
    2\Rightarrow2
  }
  { (3-1)+2\Rightarrow4 }
\]

To explain what proof trees are, we have drawn the proof tree from its leaf
nodes. However, we usually draw a proof tree from the root node.
We start by drawing a horizontal line and writing the program we want to evaluate.

\[
  \inferrule
  {
    \color{white}
    \inferrule
    { 3\Rightarrow3 \\ 1\Rightarrow1 }
    { 3-1\Rightarrow2 }
    \\
    \color{white}
    2\Rightarrow2
  }
  { (3-1)+2\Rightarrow{\color{white}4} }
\]

Then, we find which inference rule can be applied. In this case, we can use Rule
\textsc{Add} since the program is addition.

\[
  \inferrule
  {
    \inferrule
    { \color{white}3\Rightarrow3 \\ \color{white}1\Rightarrow1 }
    { 3-1\Rightarrow\color{white}2 }
    \\
    2\Rightarrow\color{white}2
  }
  { (3-1)+2\Rightarrow{\color{white}4} }
\]

We need to evalaute $3-1$ and $2$ respectively. Let us focus on $3-1$ first.
Since $3-1$ is subtraction, we use Rule \textsc{Sub}.

\[
  \inferrule
  {
    \inferrule
    { 3\Rightarrow\color{white}3 \\ 1\Rightarrow\color{white}1 }
    { 3-1\Rightarrow\color{white}2 }
    \\
    2\Rightarrow\color{white}2
  }
  { (3-1)+2\Rightarrow{\color{white}4} }
\]

We can conclude that $3\Rightarrow3$ from Rule \textsc{Num}.

\[
  \inferrule
  {
    \inferrule
    { 3\Rightarrow3 \\ 1\Rightarrow\color{white}1 }
    { 3-1\Rightarrow\color{white}2 }
    \\
    2\Rightarrow\color{white}2
  }
  { (3-1)+2\Rightarrow{\color{white}4} }
\]

Similarly, $1\Rightarrow1$.

\[
  \inferrule
  {
    \inferrule
    { 3\Rightarrow3 \\ 1\Rightarrow1 }
    { 3-1\Rightarrow\color{white}2 }
    \\
    2\Rightarrow\color{white}2
  }
  { (3-1)+2\Rightarrow{\color{white}4} }
\]

By subtracting $1$ from $3$, we get $2$.

\[
  \inferrule
  {
    \inferrule
    { 3\Rightarrow3 \\ 1\Rightarrow1 }
    { 3-1\Rightarrow2 }
    \\
    2\Rightarrow\color{white}2
  }
  { (3-1)+2\Rightarrow{\color{white}4} }
\]

We use Rule \textsc{Num} agian and get $2\Rightarrow2$.

\[
  \inferrule
  {
    \inferrule
    { 3\Rightarrow3 \\ 1\Rightarrow1 }
    { 3-1\Rightarrow2 }
    \\
    2\Rightarrow2
  }
  { (3-1)+2\Rightarrow{\color{white}4} }
\]

Finally, we can complete the proof tree and prove $(3-1)+2\Rightarrow4$.

\[
  \inferrule
  {
    \inferrule
    { 3\Rightarrow3 \\ 1\Rightarrow1 }
    { 3-1\Rightarrow2 }
    \\
    2\Rightarrow2
  }
  { (3-1)+2\Rightarrow4 }
\]

Sometimes, we call a proof tree proving the result of a program a
\textit{evaluation derivation}\index{evaluation derivation}
since the tree explains how the result of the program is derived.

The way of defining semantics we have seen so far is \textit{big-step
operational semantics}.\index{big-step operational semantics} It is called
``operational'' because it focuses on which \textbf{operation}s happen during execution
and ``big-step'' because it finds the result of a program by taking a single
\textbf{big} step. There are other ways to define semantics: denotational
semantics and small-step operational semantics. Most part of this book uses
big-step operational semantics. However, it will use small-step operational
semantics to deal with continuations later.

An \textit{interpreter}\index{interpreter} is a program that takes a program as
input and evaluates the program. We can easily implement an interpreter of
\lang according to its semantics. The interpreter consists of a single function
that takes an AST as an argument and returns an integer.

\begin{verbatim}
def interp(e: AE): Int = e match {
  case Num(n) => n
  case Add(l, r) => interp(l) + interp(r)
  case Sub(l, r) => interp(l) - interp(r)
}
\end{verbatim}

\section{Syntactic Sugar}

\textit{Syntactic sugar}\index{syntactic sugar} adds a new feature to a language
by defining syntactic transformation rules instead of changing the semantics.
Syntactic sugar is widely used in real-world programming languages because it
allows languages to provide useful features without increasing the burden of the
language designers too much.

Suppose that we want to add integer negation to \lang. It can be done by
modifying both syntax and semantics of \lang. First, we fix the concrete syntax
to add integer negation.

\begin{verbatim}
<expr> ::= <number> | <expr> "+" <expr>
         | <expr> "-" <expr> | "-" "(" expr ")"
\end{verbatim}

Similarly, we fix the abstract syntax, too.

\[e\ ::=\ n\ |\ e+e\ |\ e-e\ |\ -e\]

The parser should be changed accordingly. For example, \code{-(03+4)} is parsed
to $-(3+4)$.

Finally, we add a new rule to the semantics to handle the $-e$ case.

\semanticrule{Neg}{
If $e$ evaluates to $n$,\\
then $-e$ evaluates to $\isub n$.
}

Note that $\isub$ denotes mathematical negation.

We can express the same thing as an inference rule.

\[
  \inferrule
  { e\Rightarrow n}
  { -e\Rightarrow\isub n }
  \quad\textsc{Neg}
\]

It requires lots of work, as we need to fix every component of the language.

Another way to add integer negation to \lang is to add it as syntactic sugar. It
is enough to modify the concrete syntax and the parser. The change in the
concrete syntax is the same as before. Now, we fix the parser to parse \code{"-"
"(" expr ")"} to $0-e$ when \code{expr} is parsed to $e$. For example,
\code{-(03+4)} is parsed to $0-(3+4)$. Since $0\isub n=\isub n$ for any integer
$n$, we are done. It is the power of syntactic sugar. Language designers can
easily add new features by syntactically transforming them to existing features.
The procedure removing syntactic sugar by transformation is called
\textit{desugaring}\index{desugaring}.

We can find various examples of syntactic sugar in real-world languages. For
instance, for loops in Scala are supported as syntactic sugar.

\begin{quote}
  A for comprehension \code{for ($p$ <- $e$) yield $e'$} is translated to
  \code{$e$.map \{ case $p$ =>
  $e'$ \}}.\sidenote{\url{https://www.scala-lang.org/files/archive/spec/2.13/06-expressions.html\#for-comprehensions-and-for-loops}}
\end{quote}

In addition, macros in languages like C, Scala, LISP, and Rust can be considered
as user-defined syntactic sugar.

\section{Exercises}

\begin{enumerate}
\item Consider the following concrete syntax:

\begin{verbatim}
<expr> ::= <num>
         | "{" "+" <expr> <expr> "}"
         | "{" "*" <expr> <expr> "}"
         | "{" "let" "{" <id> <expr> "}" <expr> "}"
         | <id>
\end{verbatim}

Describe whether each of the following is \code{expr} and why.
Assume that it is allowed to add whitespaces among terminals freely.

\begin{enumerate}
  \item \verb!{let {x 5} {+ 8 {* x 2 3}}}!
  \item \verb!{with {x 0} {with {x 7}}}!
  \item \verb!{let {3 5} {+ 8 {- x 2}}}!
  \item \verb!{let {3 y} {+ 8 {* x 2}}}!
  \item \verb!{let {x y} {+ 8 {* x 2}}}!
\end{enumerate}

\item Consider the following concrete syntax:

\newcommand{\BNF}[1]{\code{<#1>}}
\newcommand{\coffee}{\mbox{\BNF{coffee}}}
\newcommand{\milk}{\mbox{\BNF{milk}}}
\newcommand{\flavor}{\mbox{\BNF{flavor}}}

\[
\begin{array}{ccc}
  \code{espresso} \in \coffee
  &&
  \inferrule
  { e_1 \in \milk \\ e_2 \in \coffee }
  { e_1\ \code{"on"}\ e_2 \in \coffee }
  \\[2em]
  \inferrule
  { e_1 \in \coffee \\ e_2 \in \milk }
  { e_1\ \code{"on"}\ e_2 \in \coffee }
  &&
  \inferrule
  { e_1 \in \flavor \\ e_2 \in \coffee }
  { e_1\ \code{"on"}\ e_2 \in \coffee }
  \\[2em]
  \code{"milk-foam"} \in \milk
  &&
  \code{"steamed-milk"} \in \milk
  \\[2em]
  \code{"caramel"} \in \flavor
  &&
  \code{"cinnamon"} \in \flavor
  \\[2em]
  \code{"cocoa-powder"} \in \flavor
  &&
  \code{"chocolate-syrup"} \in \flavor
\end{array}
\]

Assume that it is allowed to add whitespaces among terminals freely.

\begin{enumerate}
  \item Which of the following are elements of \BNF{coffee}?
  \begin{enumerate}
    \item \code{caramel latte macchiato}
    \item \code{espresso}
    \item \code{steamed-milk on caramel on milk-foam on espresso}
    \item \code{chocolate-syrup on cocoa-powder on cinnamon on milk-foam on steamed-milk on espresso}
    \item \code{steamed-milk on espresso on chocolate-syrup}
  \end{enumerate}
  \item Explain whether the following is \code{<coffee>} or not:

  \code{cocoa-powder on milk-foam on steamed-milk on espresso}
\end{enumerate}

\item Consider the following concrete syntax:

\begin{verbatim}
<ice-cream> ::= "sprinkles" "on" <ice-cream>
              | "cherry" "on" <ice-cream>
              | "scoop" "of" <flavor> "on" <ice-cream>
              | "sugar-cone"
              | "waffle-cone"
<flavor>    ::= "vanilla"
              | "lettuce"
\end{verbatim}

Assume that it is allowed to add whitespaces among terminals freely.

\begin{enumerate}
  \item Which of the following are elements of \code{<ice-cream>}?
  \begin{enumerate}
    \item \code{sprinkles}
    \item \code{sugar-cone}
    \item \code{vanilla}
    \item \code{scoop of vanilla on waffle-cone}
    \item \code{sprinkles on lettuce on waffle-cone}
    \item \code{scoop of vanilla on sprinkles on waffle-cone}
  \end{enumerate}
  \item Explain why the following is an element of \code{<ice-cream>}:

  \code{cherry on scoop of lettuce on scoop of vanilla on sugar-cone}
\end{enumerate}
\end{enumerate}
