\setchapterpreamble[u]{\margintoc}
\chapter{Parametric Polymorphism}
\labch{parametric-polymorphism}

\renewcommand{\plang}{\textsf{TFAE}\xspace}
\renewcommand{\lang}{\textsf{PTFAE}\xspace}

A function in \plang is more restrictive than a function in \textsf{FAE}. Consider
$\efun{\cx}{\cx}$ in \textsf{FAE}. It is an identity function, which takes a value as an
argument and returns the value without changing it. Any value can be an argument
for this function. Since the body of the function does nothing with the argument,
the evaluation of the body never causes a run-time error. On the other hand,
$\efunt{\cx}{\tnum}{\cx}$ in \plang is an identity function that takes only an
integer. The parameter type annotation restricts the type of an argument to be
only $\tnum$. The type system rejects a program that passes a value other than
an integer to the function. However, since the body simply returns the argument,
a run-time error never happens even when the argument is not an integer.
This example shows that a single identity function can be applied to a value of
every type in \textsf{FAE}, while a different identity function is required for
each type in \plang.

Because of this restrictiveness, programmers have to define more functions in
\plang than \textsf{FAE} to implement the same program. For example, the
following \textsf{FAE} expression does not incur run-time errors:\sidenote{Suppose
that we extends \textsf{FAE} with booleans.}

$
  \begin{array}{@{}l}
    \ebind{\cf}{\efun{\cx}{\cx}}{\\
    \ebind{\cy}{\cf\ 1}{\\
    \cf\ \textsf{true}
    }}
  \end{array}
$

On the other hand, the following \plang expression is rejected by the type
system:

$
  \begin{array}{@{}l}
    \ebind{\cf}{\efunt{\cx}{\tnum}{\cx}}{\\
    \ebind{\cy}{\cf\ 1}{\\
    \cf\ \textsf{true}
    }}
  \end{array}
$

The expression is ill-typed because $\cf\ \textsf{true}$ on the last line
applies a function of $\tarrow{\tnum}{\tnum}$ to a value that does not belong to
$\tnum$. To make an equivalent and well-typed expression, we should define
one more function like below.\sidenote{Suppose that the type of a boolean is
\tbool.}

$
  \begin{array}{@{}l}
    \ebind{\cf}{\efunt{\cx}{\tnum}{\cx}}{\\
    \ebind{\code{g}}{\efunt{\cx}{\tbool}{\cx}}{\\
    \ebind{\cy}{\cf\ 1}{\\
    \code{g}\ \textsf{true}
    }}}
  \end{array}
$

Now, the expression is well-typed since \code{g}, whose type is
$\tarrow{\code{bool}}{\code{bool}}$, is applied to a value of $\code{bool}$.

Because of this problem, writing programs in \plang is inconvenient. Programmers
should define multiple functions of the same functionality to convince the type
checker that their program does not incur any run-time errors. Such functions of
an overlapping role lead to duplicated code, which increases the code length and
harms maintainability. When a program has plenty of functions and types, the
amount of duplicated code will become huge.

Polymorphism can resolve the problem. \textit{Polymorphism}\index{polymorphism}
is to use a single entity as multiple types. For example, it may allow
$\efun{\cx}{\cx}$ to have multiple types: $\tarrow{\tnum}{\tnum}$ and
$\tarrow{\tbool}{\tbool}$. There are three widely-used ways to
realize polymorphism in a language: parametric polymorphism, subtype
polymorphism, and ad-hoc polymorphism. The topic of this chapter is parametric
polymorphism. \refch{subtype-polymorphism} introduces subtype polymorphism, and
ad-hoc polymorphism is outside the scope of this book.

To introduce parametric polymorphism, we first need to discuss what
parametrization is. Functions are well-known examples of parametrizing entities.
Each function parametrizes an expression with a value
(or an expression in the case of lazy languages). Consider $\efun{\cx}{\cx+\cx}$.
In this function, $\cx$ is the parameter. The body, $\cx+\cx$ is parametrized by $\cx$.
This function is the most general form of adding a value to the same value. By
applying the function, we can express any expresion that adds a value to the
same value. For example, $1+1$ is equivalent to $(\efun{\cx}{\cx+\cx})\ 1$, and
$42+42$ is equivalent to $(\efun{\cx}{\cx+\cx})\ 42$. A function abstracts an
expression by replacing some portion of the expression with the parameter. By
applying a function to values, multiple expressions can be
expressed without repeating the common constituents. Only different parts should
be written as an argument in each case.

\textit{Parametric polymorphism}\index{parametric polymorphism}
allows entities to be parametrized by types. It is a new form of
parametrization, which functions do not provide. Parametric polymorphism allows
parametrizing an expression with a type, instead of a value. To distinguish this
new notion of parametrization from functions, we use the term \textit{type
abstraction}\index{type abstraction}. While functions are applied to values to
replace their parameters with real values, type abstractions are applied to
types to replace their \textit{type parameters}\index{type parameter} with real
types. To differentiate application of type abstractions from application of
functions, we use the term \textit{type application}\index{type application}.
Since type abstractions parametrize expressions, the results of type application
are values, just like functions.
The following table compares functions and type abstractions:
\begin{center}
\begin{tabular}{c|c|c}
  & Function & Type abstraction \\\hline
  What is parametrized? & Expression & Expression \\
  With what? & Value & Type \\
  Applied to what? & Value & Type \\
  Result of application & Value & Value \\
\end{tabular}
\end{center}

Consider $\efunt{\cx}{\tnum}{\cx}$ and $\efunt{\cx}{\tbool}{\cx}$.
The only difference is the type annotation: $\tnum$ and $\tbool$.
We can parametrize both expressions with a type by introducing a type parameter
$\alpha$. By replacing $\tnum$ with $\alpha$ in $\efunt{\cx}{\tnum}{\cx}$, we
obtain $\efunt{\cx}{\alpha}{\cx}$. Similarly, by replacing $\tbool$
with $\alpha$ in $\efunt{\cx}{\tbool}{\cx}$, we obtain
$\efunt{\cx}{\alpha}{\cx}$. The results are exactly identical to each other.
We can make a type abstraction that takes a type $\tau$ as a \textit{type
argument}\index{type argument} and returns $\efunt{\cx}{\tau}{\cx}$ as a result.
This book uses $\Lambda$ to denote type abstractions. Thus, the type abstraction
we want is $\etfun{\alpha}{\efunt{\cx}{\alpha}{\cx}}$. The type abstraction can
be applied to types to recover the original expressions. This book uses $[]$ to
denote type application. Then,
$\etapp{(\etfun{\alpha}{\efunt{\cx}{\alpha}{\cx}})}{\tnum}$ is equivalent to
$\efunt{\cx}{\tnum}{\cx}$, and
$\etapp{(\etfun{\alpha}{\efunt{\cx}{\alpha}{\cx}})}{\tbool}$ is equivalent to
$\efunt{\cx}{\tbool}{\cx}$.

After adding parametric polymorphism, we can make the previous example well-typed
while defining a function only once.

$
  \begin{array}{@{}l}
    \ebind{\cf}{\etfun{\alpha}{\efunt{\cx}{\alpha}{\cx}}}{\\
    \ebind{\cy}{\etapp{\cf}{\tnum}\ 1}{\\
    \etapp{\cf}{\tbool}\ \textsf{true}
    }}
  \end{array}
$

It is still more complex than the \textsf{FAE} version but defines a function
only once, unlike the \plang version.

Traditionally, parametric polymorphism was supported by only functional
languages. For example, OCaml and Haskell have been well-known for their support for
parametric polymorphism. On the other hand, object-oriented languages
provided only subtype polymorphism. For instance, Java lacked parametric
polymorphism until Java 4. However, programmers in these days require languages
to provide more advanced features because their programs become more
complicated. For this reason, Java has been supporting parametric polymorphism
since Java 5. Many recent languages, such as Scala, provide both parametric and
subtype polymorphism. In the context of object-oriented programming, parametric
polymorphism is often called \textit{generics}\index{generics} since it allows
generic programming.

This chapter defines \lang by extending \plang with parametric polymorphism.
\lang is known as System F in the programming language community. System F was
first discoverd by Girard in the context of logic in
1972~\cite{girard1972interpretation}. Later, Reynolds independently discoverd
the equivalent system in the context of computer science in
1974~\cite{reynolds1974towards}. System F, or \lang, is the most foundatinoal
formulation of parametric polymorphism, and its metatheory and variants are
widely studied even in these days.

\section{Syntax}

First, we introduce type identifiers, like in \textsf{TVFAE}. Type identifiers
are used as type parameters. Let $\embox{TId}$ be the set of every type identifier.
Metavariable $\alpha$ ranges over type identifiers.

\[ \alpha \in \embox{TId} \]

We add two sorts of an expression: type abstraction and type application.

\[ e\ ::=\ \cdots \ |\ \Lambda\alpha.e \ |\ e[\tau] \]

\begin{itemize}
  \item $\etfun{\alpha}{e}$ is a type abstraction. $\alpha$ is the type
    parameter, and $e$ is the body. $\alpha$ in $\etfun{\alpha}{e}$
    is a binding occurrence whose scope is $e$.
  \item $\etapp{e}{\tau}$ is a type application expression. $e$ is denotes the type
    abstraction applied to $\tau$.
\end{itemize}

In addition, we add two sorts of a type.

\[ \tau\ ::=\ \cdots \ |\ \alpha \ |\ \tforall{\alpha}{\tau} \]

\begin{itemize}
  \item $\alpha$ is a type identifier as a type. Like in \textsf{TVFAE}, a
    type identifier can be used as a type.
    For example, in $\etfun{\alpha}{\efunt{\cx}{\alpha}{\cx}}$, the second
    $\alpha$ is the type of $\cx$.
  \item $\tforall{\alpha}{\tau}$ is a \textit{universally quantified
    type}\index{universally quantified type}. It is the type of a type
    abstraction that takes a type as a type argument and returns a value of the type
    obtained by replacing every $\alpha$ with the given type in $\tau$.
    $\alpha$ in $\tforall{\alpha}{\tau}$ is a binding occurrence whose scope is
    $\tau$.
    For instance, the type of $\etfun{\alpha}{\efunt{\cx}{\alpha}{\cx}}$ is
    $\tforall{\alpha}{\tarrow{\alpha}{\alpha}}$ since applying the type
    abstraction to $\tnum$ results in a function of $\tarrow{\tnum}{\tnum}$ and
    applying to $\tbool$ results in a function of $\tarrow{\tbool}{\tbool}$.
\end{itemize}

\section{Dynamic Semantics}

The addition of type abstractions adds a new sort of a value to the language.

\[ v\ ::=\ \cdots \ |\ \tclov{\alpha}{e}{\sigma} \]

$\tclov{\alpha}{e}{\sigma}$ is a type abstraction value. It is similar to a
closure. A closure is a function value, which contains the definition of a
function and the environment of when the function is defined. Similarly, a type
abstraction value contains the definition of a type abstraction and the
environment of when the type abstraction is defined.

Now, we define the dynamic semantics of \lang. We should define the rules for
the added expressions.

\semanticrule{TyAbs}{
  \evaldn{\etfun{\alpha}{e}}{\tclov{\alpha}{e}{\sigma}}.
}

\vspace{-1em}

\[
  \evald{\etfun{\alpha}{e}}{\tclov{\alpha}{e}{\sigma}}
  \quad\textsc{[TyAbs]}
\]

A type abstraction evaluates to a type abstraction value. It is almost the same
as Rule~\textsc{Fun}.

\semanticrule{TyApp}{
  \begin{tabular}{@{\hskip0pt}l@{\hskip-10pt}l}
    If \\
    & \evaldn{e}{\tclov{\alpha}{e'}{\sigma'}} and \\
    & \evaln{\sigma'}{e'[\alpha\leftarrow\tau]}{v}, \\
    then \\
    & \evaldn{\etapp{e}{\tau}}{v}.
  \end{tabular}
}

\vspace{-1em}

\[
  \inferrule
  { \evald{e}{\tclov{\alpha}{e'}{\sigma'}} \\
    \eval{\sigma'}{e'[\alpha\leftarrow\tau]}{v} }
  { \evald{\etapp{e}{\tau}}{v} }
  \quad\textsc{[TyApp]}
\]

To evaluate a type application expression, its only subexpression is evaluated.
The result must be a type abstraction value. $e'[\alpha\leftarrow\tau]$ denotes
an expression obtained by substituting every free occurrence of $\alpha$ with $\tau$
in $e'$.\sidenote{We do not discuss substitution in detail here. Interested readers can
refer to Exercise~8 of \refch{first-class-functions}, which introduces
substition of an identifier with a value in an expression. The same principle
applies to substition of a type identifier with a type in an expression.}
Thus, the rule states that every occurrence of the type parameter is replaced with
the given type argument in the body of the type abstraction. Finally, the
expression obtained by the substitution is evaluated under the environment held
by the type abstraction value. Its result is the result of the whole type
application expression.

One may wonder why Rule~\textsc{TyApp} needs substitution because types do not
have any roles at run time. We can omit substitution in the dynamic semantics of
\lang. However, if we extend the language to support any form of dynamic type
testing, substitution is mandatory. In addition, if we want to prove type
soundness, we should prove that every expression of a certain type evaluates to a
value of the same type when the evaluation terminates. This property is called
type preservation, and evaluation will not preserve the type of an expression if
the rule omits substitution. For these reasons, Rule~\textsc{TyApp} requires
substitution.

The following proof trees prove that
$\etapp{(\etfun{\alpha}{\efunt{\cx}{\alpha}{\cx}})}{\tnum}\ 1$
evaluates to $1$:

\[
  \inferrule
  {
    \evale
    {\etfun{\alpha}{\efunt{\cx}{\alpha}{\cx}}}
    {\tclov{\alpha}{\efunt{\cx}{\alpha}{\cx}}{\emptyset}}
    \\
    \evale{\efunt{\cx}{\tnum}{\cx}}{\clov{\cx}{\cx}{\emptyset}}
  }
  { \evale
    {\etapp{(\etfun{\alpha}{\efunt{\cx}{\alpha}{\cx}})}{\tnum}}
    {\clov{\cx}{\cx}{\emptyset}}
  }
\]

\[
\inferrule
{
  \evale
  {\etapp{(\etfun{\alpha}{\efunt{\cx}{\alpha}{\cx}})}{\tnum}}
  {\clov{\cx}{\cx}{\emptyset}}
  \\
  \evale{1}{1}
  \\
  \inferrule
  { \cx\in\dom{\sigma_1} }
  { \eval{\sigma_1}{\cx}{1} }
}
{ \evale{\etapp{(\etfun{\alpha}{\efunt{\cx}{\alpha}{\cx}})}{\tnum}\ 1}{1} }
\]

where $\sigma_1=[\cx\mapsto1]$.

% \[
% \begin{array}{rcl}
% \textsf{num} \lbrack\alpha\leftarrow\tau\rbrack &=& \textsf{num} \\
% (\tau_1\rightarrow\tau_2) \lbrack\alpha\leftarrow\tau\rbrack &=&
% (\tau_1
% \lbrack\alpha\leftarrow\tau\rbrack)\rightarrow(\tau_2\lbrack\alpha\leftarrow\tau\rbrack)
% \\
% \alpha \lbrack\alpha\leftarrow\tau\rbrack &=& \tau \\
% \alpha' \lbrack\alpha\leftarrow\tau\rbrack &=& \alpha'\\
% \textsf{(if } \alpha\not=\alpha'\textsf{)} \\
% (\forall\alpha.\tau') \lbrack\alpha\leftarrow\tau\rbrack &=& \forall\alpha.\tau'
% \\
% (\forall\alpha'.\tau') \lbrack\alpha\leftarrow\tau\rbrack &=&
% \forall\alpha'.(\tau'\lbrack\alpha\leftarrow\tau\rbrack) \\
% \textsf{(if } \alpha\not=\alpha'\textsf{)} \\
% \end{array}
% \]

% \[
% \begin{array}{rcl}
% n \lbrack\alpha\leftarrow\tau\rbrack &=& n \\
% (e_1+e_2) \lbrack\alpha\leftarrow\tau\rbrack &=&
% (e_1\lbrack\alpha\leftarrow\tau\rbrack) + (e_2\lbrack\alpha\leftarrow\tau\rbrack)
% \\
% (e_1-e_2) \lbrack\alpha\leftarrow\tau\rbrack &=&
% (e_1\lbrack\alpha\leftarrow\tau\rbrack) - (e_2\lbrack\alpha\leftarrow\tau\rbrack)
% \\
% x \lbrack\alpha\leftarrow\tau\rbrack &=& x \\
% (\lambda x:\tau'.e) \lbrack\alpha\leftarrow\tau\rbrack &=&
% \lambda
% x:(\tau'\lbrack\alpha\leftarrow\tau\rbrack).(e\lbrack\alpha\leftarrow\tau\rbrack)
% \\
% (e_1\ e_2) \lbrack\alpha\leftarrow\tau\rbrack &=&
% (e_1\lbrack\alpha\leftarrow\tau\rbrack)\ (e_2\lbrack\alpha\leftarrow\tau\rbrack)
% \\
% (\Lambda\alpha.e)\lbrack\alpha\leftarrow\tau\rbrack &=& \Lambda\alpha.e \\
% (\Lambda\alpha'.e)\lbrack\alpha\leftarrow\tau\rbrack &=&
% \Lambda\alpha'.(e\lbrack\alpha\leftarrow\tau\rbrack)\\
% \textsf{(if } \alpha\not=\alpha'\textsf{)} \\
% (e\ \lbrack\tau'\rbrack)\lbrack\alpha\leftarrow\tau\rbrack &=&
% (e\lbrack\alpha\leftarrow\tau\rbrack)\ \lbrack
% \tau'\lbrack\alpha\leftarrow\tau\rbrack\rbrack \\
% \end{array}
% \]

% \section{Interpreter}

% \begin{verbatim}
% sealed trait Expr
% ...
% case class TFun(a: String, b: Expr) extends Expr
% case class TApp(f: Expr, t: Type) extends Expr
% \end{verbatim}

% Now, consider an interpreter for \lang.

% \begin{verbatim}
% sealed trait Value
% ...
% case class TFunV(a: String, b: Expr, e: Env) extends Value
% \end{verbatim}

% A \code{TFunV} instance represents a type function value.

% \begin{verbatim}
% case TFun(a, b) => TFunV(a, b, env)
% \end{verbatim}

% A type function results in a type function value that captures the current
% environment.

% \begin{verbatim}
% case TApp(f, t) =>
%   val TFunV(a, b, fEnv) = interp(f, env)
%   interp(subst(b, a, t), fEnv)
% \end{verbatim}

% For evaluation of a type application, the type function expression is evaluated
% first. The result must be a type function value. Obtain an expression by
% substituting the type parameter of the type function value with the type
% argument in the body of the type function value. Evaluate the expression under
% the environment captured by the type function value to acquire the final result.

\section{Static Semantics}

Like in \textsf{TVFAE}, the definition of a type environment needs to be
revised. Type environments should be able to store type identifiers in
addition to the types of variables. Unlike type definitions in \textsf{TVFAE},
type identifiers in \lang do not have any further information. However, the type
checking procedure needs to know whether a certain type identifier is free or
not to determine the well-formedness of types. Thus, type environments store
type identifiers to record their existence.

\[
  \embox{TEnv}=(\embox{Id}\cup\embox{TId})\finto(T\cup\{\cdot\})
\]

Now, the codomain of a type environment contains $\cdot$, which is a
meaningless mathematical object. For brevity, let $\Gamma[\alpha]$ denote
$\Gamma[\alpha:\cdot]$.

\subsection{Well-Formed Types}

The well-formedness checking of \lang is similar to \textsf{TVFAE}.
The following three rules are the same as those of \textsf{TVFAE}:

\typerule{Wf-NumT}{
  \wftdn{\tnum}.
}

\vspace{-1em}

\[
  \wftd{\tnum}
  \quad\textsc{[Wf-NumT]}
\]

\typerule{Wf-ArrowT}{
  If
    \wftdn{\tau_1} and
    \wftdn{\tau_2}, \\
  then
    \wftdn{\tarrow{\tau_1}{\tau_2}}.
}

\vspace{-1em}

\[
  \inferrule
  { \wftd{\tau_1} \\
    \wftd{\tau_2} }
  { \wftd{\tarrow{\tau_1}{\tau_2}} }
  \quad\textsc{[Wf-ArrowT]}
\]

\typerule{Wf-IdT}{
  If $\alpha$ in the domain of $\Gamma$,\\
  then \wftdn{\alpha}.
}

\vspace{-1em}

\[
  \inferrule
  { \alpha\in\dom{\Gamma} }
  { \wftd{\alpha} }
  \quad\textsc{[Wf-IdT]}
\]

One remaining sort of a type is a universally quantified type, which is new in
\lang.

\typerule{Wf-ForallT}{
  If
    \wftn{\Gamma[\alpha]}{\tau}, \\
  then
    \wftdn{\tforall{\alpha}{\tau}}.
}

\[
  \inferrule
  { \wft{\Gamma[\alpha]}{\tau} }
  { \wftd{\tforall{\alpha}{\tau}} }
  \quad\textsc{[Wf-ForallT]}
\]

In $\tforall{\alpha}{\tau}$, $\alpha$ is a binding occurrence and, thus, can
appear in $\tau$. Therefore, $\alpha$ must be in the type environment when the
well-formedness of $\tau$ is checked. For example, $\tforall{\alpha}{\alpha}$
is well-formed under the empty type environment, while $\tforall{\alpha}{\beta}$
is ill-formed under the same type environment.

\subsection{Typing Rules}

Now, let us define the typing rules of \lang.

\typerule{Typ-TyAbs}{
  If
    \typeofn{\Gamma[\alpha]}{e}{\tau}, \\
  then
    \typeofdn{\etfun{\alpha}{e}}{\tforall{\alpha}{\tau}}.
}

\vspace{-1em}

\[
  \inferrule
  { \typeof{\Gamma[\alpha]}{e}{\tau} }
  { \typeofd{\etfun{\alpha}{e}}{\tforall{\alpha}{\tau}} }
  \quad\textsc{[Typ-TyAbs]}
\]

The type of ${\etfun{\alpha}{e}}$ is $\tforall{\alpha}{\tau}$ if the type of $e$
is $\tau$. Since $\alpha$ is a binding occurrence whose scope is $e$,
$e$ should be type-checked under the type environment containing $\alpha$.

\typerule{Typ-TyApp}{
  If
    \wftdn{\tau} and
    \typeofdn{e}{\tforall{\alpha}{\tau'}}, \\
  then
    \typeofdn{\etapp{e}{\tau}}{\tau'[\alpha\leftarrow\tau]}.
}

\vspace{-1em}

\[
  \inferrule
  { \wftd{\tau} \\
    \typeofd{e}{\tforall{\alpha}{\tau'}} }
  { \typeofd{\etapp{e}{\tau}}{\tau'[\alpha\leftarrow\tau]} }
  \quad\textsc{[Typ-TyApp]}
\]

If the type of $e$ is ${\tforall{\alpha}{\tau'}}$, the type of $\etapp{e}{\tau}$
is $\tau'[\alpha\leftarrow\tau]$, which is a
type obtained by substituting $\alpha$ with $\tau$ in $\tau'$. Since
$\tau$ is a user-written type annotation, the well-formeness of $\tau$ must be
checked.

In addition, like in \textsf{TVFAE}, Rule~\textsc{Typ-Fun} has to check the well-formedness
of the parameter type annotation.

\typerule{Typ-Fun}{
  If
    \wftdn{\tau_1} and
    \typeofn{\Gamma[x:\tau_1]}{e}{\tau_2}, \\
  then
    \typeofdn{\efunt{x}{\tau_1}{e}}{\tarrow{\tau_1}{\tau_2}}.
}

\vspace{-1em}

\[
  \inferrule
  { \wftd{\tau_1} \\
    \typeof{\Gamma[x:\tau_1]}{e}{\tau_2} }
  { \typeofd{\efunt{x}{\tau_1}{e}}{\tarrow{\tau_1}{\tau_2}} }
  \quad\textsc{[Typ-Fun]}
\]

The following proof trees prove that the type of
$\etapp{(\etfun{\alpha}{\efunt{\cx}{\alpha}{\cx}})}{\tnum}\ 1$
is $\tnum$:

\[
  \small
  \inferrule
  {
    \inferrule
    {
      \wft{\emptyset}{\tnum}
      \\
      \inferrule
      {
        \inferrule
        {
          \inferrule
          { \alpha\in\dom{[\alpha]} }
          { \wft{[\alpha]}{\alpha} }
          \\
          \inferrule
          { \cx\in\dom{[\alpha,\cx:\alpha]} }
          { \typeof{[\alpha,\cx:\alpha]}{\cx}{\alpha} }
        }
        { \typeof{[\alpha]}{\efunt{\cx}{\alpha}{\cx}}{\tarrow{\alpha}{\alpha}} }
      }
      { \typeofe{\etfun{\alpha}{\efunt{\cx}{\alpha}{\cx}}}
        {\tforall{\alpha}{\tarrow{\alpha}{\alpha}}} }
    }
    { \typeofe{\etapp{(\etfun{\alpha}{\efunt{\cx}{\alpha}{\cx}})}{\tnum}}
      {\tarrow{\tnum}{\tnum}} }
    \\
    \typeofe{1}{\tnum}
  }
  {\typeofe{\etapp{(\etfun{\alpha}{\efunt{\cx}{\alpha}{\cx}})}{\tnum}\ 1}{\tnum}}
\]

The current type system of \lang has two problems. First, multiple type
parameters of the same name breaks the type soundness, as multiple type
definitions of the same name does in \textsf{TVFAE}. Second, syntactic
comparison of types makes the type checking too restrictive. For example, if
$\etfun{\beta}{\efunt{\cx}{\beta}{\cx}}$ is given to a function
that expects a value of $\tforall{\alpha}{\tarrow{\alpha}{\alpha}}$, syntactically comparing
$\tforall{\alpha}{\tarrow{\alpha}{\alpha}}$ and $\tforall{\beta}{\tarrow{\beta}{\beta}}$ judges them to be
different and makes the type checking reject the program. However,
$\tforall{\alpha}{\tarrow{\alpha}{\alpha}}$ and $\tforall{\beta}{\tarrow{\beta}{\beta}}$ denote the same type
semantically.

The best solution to both of the problems is de Bruijn indices, introduced in
\refch{nameless-representation-of-expressions}.
\refch{nameless-representation-of-expressions} shows use of de Bruijn indices for
expressions. However, de Bruijn indices are not limited to expressions; they
can be applied to types. For instance, both $\tforall{\alpha}{\tarrow{\alpha}{\alpha}}$
and $\tforall{\beta}{\tarrow{\beta}{\beta}}$ can be represented with
$\Lambda.\tarrow{\underline{0}}{\underline{0}}$, so their semantic equivalence
can be verified with syntactic comparison. In addition, de Bruijn indices
prevent multiple types from being considered as the same type because of their
names.

De Bruijn indices seem to be the best solution, but, still, other solutions can
be used. The three solutions described in \refch{algebraic-data-types} can
applied to \lang in the same manner to resolve the first issue. The second issue
can be fixed by renaming type parameters before the comparison. For example, simple syntactic
transformation can transform $\tforall{\alpha}{\tarrow{\alpha}{\alpha}}$
into $\tforall{\beta}{\tarrow{\beta}{\beta}}$.

% \[
% \tau\equiv\tau
% \]

% \[
% \inferrule
% { \tau\equiv\tau'\lbrack\alpha'\leftarrow\alpha\rbrack }
% { \forall\alpha.\tau\equiv\forall\alpha'.\tau' }
% \]

% \[
% \inferrule
% { \Gamma\vdash e_1:\tau'\rightarrow\tau \\
%   \Gamma\vdash e_2:\tau'' \\
%   \tau'\equiv\tau'' }
% { \Gamma\vdash e_1\ e_2:\tau }
% \]

% \section{Type Checker}

% The following Scala code implements the abstract syntax of \lang:

% \begin{verbatim}
% sealed trait Type
% ...
% case class ForallT(a: String, t: Type) extends Type
% case class IdT(a: String) extends Type
% \end{verbatim}

% An \code{Expr} instance represents a \lang expression; a \code{TFun} instance
% represents a type function; a \code{TApp} instance represents a type application;
% a
% \code{ForallT} instance represents a universal type; a \code{IdT} instance
% represents a
% type identifier as a type. A type identifier is any string.

% \begin{verbatim}
% case class TEnv(
%   vars: Map[String, Type] = Map(),
%   tbinds: Set[String] = Set()
% ) {
%   def +(p: (String, Type)): TEnv =
%     copy(vars = vars + p)
%   def +(x: String): TEnv =
%     copy(tbinds = tbinds + x)
%   def contains(x: String): Boolean = tbinds(x)
% }
% \end{verbatim}

% \code{TEnv} is the type of a type environment. The field \code{vars} stores the
% types of
% variables; the filed \code{tbinds} stores bound type identifiers. The methods
% \code{+} and
% \code{contains} help using type environments. For example, the following shows
% adding
% a variable and a type to a type environment:

% \begin{verbatim}
% env + ("x" -> NumT)
% env + "alpha"
% \end{verbatim}

% Also, the following shows how one can check whether a type identifier is bound
% to a type environment:

% \begin{verbatim}
% env.contains("alpha")
% \end{verbatim}

% The following function \code{subst} defines substitutions on types:

% \begin{verbatim}
% def subst(t1: Type, a: String, t2: Type): Type = t1 match {
%   case NumT => t1
%   case ArrowT(p, r) => ArrowT(subst(p, a, t2), subst(r, a, t2))
%   case IdT(a1) => if (a == a1) t2 else t1
%   case ForallT(a1, t) => if (a == a1) t1 else ForallT(a1, subst(t, a, t2))
% }
% \end{verbatim}

% The function \code{mustSame} now considers equivalence of types:

% \begin{verbatim}
% def mustSame(t1: Type, t2: Type): Type = (t1, t2) match {
%   case (NumT, NumT) => t1
%   case (ArrowT(p1, r1), ArrowT(p2, r2)) =>
%     ArrowT(mustSame(p1, p2), mustSame(r1, r2))
%   case (IdT(a1), IdT(a2)) if a1 == a2 => t1
%   case (ForallT(a1, t1), ForallT(a2, t2)) =>
%     ForallT(a1, mustSame(t1, subst(t2, a2, IdT(a1))))
%   case _ => throw new Exception
% }
% \end{verbatim}

% The function \code{validType} checks whether a given type is well-formed under a
% given type environment:

% \begin{verbatim}
% def validType(t: Type, env: TEnv): Type = t match {
%   case NumT => t
%   case ArrowT(p, r) =>
%     ArrowT(validType(p, env), validType(r, env))
%   case IdT(t) =>
%     if (env.contains(t)) IdT(t)
%     else throw new Exception
%   case ForallT(a, t) => ForallT(a, validType(t, env + a))
% }
% \end{verbatim}

% Now, consider the function \code{typeCheck}.

% \begin{verbatim}
% case TFun(a, b) =>
%   if (env.contains(a)) throw new Exception
%   ForallT(a, typeCheck(b, env + a))
% \end{verbatim}

% The type parameter of a type function must not be in the type environment. The
% type of the body is computed under the type environment with the type parameter.
% The resulting type is a universal type that consists of the type parameter and
% the type of the body.

% \begin{verbatim}
% case TApp(f, t) =>
%   validType(t, env)
%   val ForallT(a, t1) = typeCheck(f, env)
%   subst(t1, a, t)
% \end{verbatim}

% A type argument must be well-formed. The type of an expression at the type
% function position must be a universal type. The resulting type is obtained by
% substituting the type parameter of the universal type with the type argument in
% the body of the universal type.

\section{Exercises}

\begin{enumerate}
\item Draw the type derivation of each the following expressions:
  \begin{enumerate}
    \item
      $(\efunt{\cf}{\tforall{\alpha}{\tarrow{\alpha}{\alpha}}}{
        \etapp{\cf}{\tnum}\ 10})\ (\etfun{\alpha}{\efunt{\cx}{\alpha}{\cx}})$
    \item
      $
        \etapp{\etapp{(\etfun{\alpha}{\etfun{\beta}{
          \efunt{\cf}{\tarrow{\alpha}{\beta}}{
            \efunt{\cx}{\alpha}{\cf\ \cx}
          }
        }})}{\tnum}}{\tnum}\ (\efunt{\cy}{\tnum}{17-\cy})\ 9
      $
  \end{enumerate}

\item Rewrite the following code with type abstractions and type application
to replace all the occurrences of $?$ with types
and to make function calls take explicit type arguments.

$
\ebind{\cf}{\efunt{\code{g}}{?}{\efunt{\cx}{?}{\code{g}\ \cx}}}{\\
\ebind{\code{g}}{\efunt{\cx}{?}{\cx}}{\\
\cf\ \code{g}\ 10}}
$

\item Consider the following language:

\textbf{Syntax}
    \vspace{0.5em}

$
  \small
\begin{array}{@{}r@{\hskip2pt}r@{\hskip2pt}l
  r@{\hskip2pt}l@{\hskip0pt}
  r@{\hskip2pt}r@{\hskip2pt}l@{\hskip10pt}
  r@{\hskip2pt}r@{\hskip2pt}l@{\hskip10pt}
  r@{\hskip2pt}c@{\hskip2pt}l}
  e &::= & n &|& \lambda x.e & \tau &::= & \textsf{num} & \sigma &::=& \tau &
  \sigma&\in&\embox{TScheme}\\[-3pt]
  &|& b &|& e\ e &&|& \tbool &&|& \forall\alpha.\sigma &
  \Gamma&\in&\embox{Id}\finto\embox{TScheme}\\
  &|& x &|& \textsf{val}\ x=e\ \textsf{in}\ e &&|& \tau\rightarrow\tau
  &b&::=&\textsf{true} \\
  &|& e;e &&&&|& \alpha &&|&\textsf{false} \\
\end{array}
$

    \vspace{0.5em}
\textbf{Typing rules}
    \vspace{0.5em}

$
  \small
\begin{array}{@{}c}
  \Gamma\vdash n:\textsf{num}
  \quad
  \Gamma\vdash b:\tbool
  \quad
  \inferrule
  {
    x\in\dom{\Gamma} \\
    \Gamma(x) \succ \tau
  }
  { \Gamma\vdash x:\tau}
  \\[20pt]
  \inferrule
  {
    \Gamma\vdash e_1:\tau_1 \\
    \Gamma\vdash e_2:\tau_2
  }
  { \Gamma\vdash e_1;e_2:\tau_2 }
  \quad
  \inferrule
  { \Gamma[x : \tau]\vdash e : \tau' }
  {\Gamma\vdash \lambda x.e : \tau\rightarrow\tau'}
  \\[20pt]
  \inferrule
  {
    \Gamma\vdash e_1 : \tau\rightarrow\tau' \\
    \Gamma\vdash e_2 : \tau
  }
  { \Gamma\vdash e_1\ e_2 : \tau' }
  \quad
  \inferrule
  {
    \Gamma\vdash e_1 : \tau \\
    \tau\prec_\Gamma\sigma \\
    \Gamma[x:\sigma]\vdash e_2 : \tau'
  }
  {\Gamma\vdash \textsf{val}\ x=e_1\ \textsf{in}\ e_2 : \tau'}
\end{array}
$

    \vspace{0.5em}
\textbf{Miscellaneous definitions}
    \vspace{0.5em}

$
  \small
\begin{array}{@{}r@{\hskip4pt}c@{\hskip4pt}l}
  \fbox{$\sigma\succ\tau$}
  &&
  \forall\alpha_1.\cdots\forall\alpha_n.\tau\succ\tau[\alpha_1\leftarrow\tau_1,\ldots,\alpha_n\leftarrow\tau_n]
  \\[1em]
  \fbox{$\tau\prec_\Gamma\sigma$}
  &&
  \inferrule
  { \textit{ FTV}(\tau)\setminus\textit{ FTV}(\Gamma)=\{\alpha_1,\ldots,\alpha_n\} }
  { \tau\prec_\Gamma\forall\alpha_1.\cdots\forall\alpha_n.\tau }
  \\[2em]
\embox{FTV}(\tnum) &=& \emptyset
\\
\embox{FTV}(\tbool) &=& \emptyset
\\
\embox{FTV}(\tau_1\rightarrow\tau_2) &=& \embox{FTV}(\tau_1) \cup \embox{FTV}(\tau_2)
\\
\embox{FTV}(\alpha) &=& \{\alpha\}
\\
\embox{FTV}(\forall\alpha.\sigma) &=& \embox{FTV}(\sigma) \setminus \{\alpha\}
\\
\embox{FTV}(\Gamma) &=& \bigcup_{x\in\dom{\Gamma}}\embox{FTV}(\Gamma(x))
\\
\end{array}
$
    \vspace{0.5em}

For each of the following expressions, write whether the expression is
well-typed. If so, draw the type derivation. Otherwise, explain why.
\begin{enumerate}
  \item $(\lambda \code{x}.(\code{x}\ 42))\ \lambda \code{z}.\code{z}$
  \item $(\lambda \code{x}.((\code{x}\ 42); (\code{x}\ \textsf{true})))\ \lambda \code{z}.\code{z}$
  \item $\textsf{val}\ \code{x}=\lambda \code{z}.\code{z}\ \textsf{in}\ ((\code{x}\ 42); (\code{x}\ \textsf{true}))$
\end{enumerate}

\item Type annotations do not take any roles during evaluation.
Therefore, this exercise defines type-erasure semantics, which removes
type annotations from an expression.
Consider the following Scala code, which implements type erasure:
\begin{verbatim}
sealed trait Expr
case class Num(n: Int) extends Expr
case class Add(l: Expr, r: Expr) extends Expr
case class Sub(l: Expr, r: Expr) extends Expr
case class Id(x: String) extends Expr
case class Fun(p: String, t: Type, b: Expr) extends Expr
case class App(f: Expr, a: Expr) extends Expr
case class TyFun(a: String, e: Expr) extends Expr
case class TyApp(e: Expr, t: Type) extends Expr

sealed trait Type
case object NumT extends Type
case class ArrowT(p: Type, r: Type) extends Type
case class VarT(a: String) extends Type
case class Forall(a: String, t: Type) extends Type

object Expr {
  sealed trait Expr
  case class Num(n: Int) extends Expr
  case class Add(l: Expr, r: Expr) extends Expr
  case class Sub(l: Expr, r: Expr) extends Expr
  case class Id(x: String) extends Expr
  case class Fun(p: String, b: Expr) extends Expr
  case class App(f: Expr, a: Expr) extends Expr
}

def erase(e: Expr): FAE.Expr = e match {
  case Num(n) => FAE.Num(n)
  case Add(l, r) => FAE.Add(erase(l), erase(r))
  case Sub(l, r) => FAE.Sub(erase(l), erase(r))
  case Id(x) => FAE.Id(x)
  case Fun(p, t, b) => FAE.Fun(p, erase(b))
  case App(f, a) => FAE.App(erase(f), erase(a))
  case TyFun(a, e) => erase(e)
  case TyApp(e, t) => erase(e)
}
\end{verbatim}
Applying the type erasure to a \lang expression results in an \textsf{FAE} expression.
The following is the syntax of \textsf{FAE}:
\[
E\ ::=\ n
\ |\ E+E
\ |\ E-E
\ |\ x
\ |\ \lambda x.E
\ |\ E\ E
\]
\begin{enumerate}
  \item Write the formal definition of the type erasure
    \framebox{$\embox{erase}(e)=E$} according to the Scala code.
\item Write the result of applying the type erasure to each of the following
  expressions:
    \begin{enumerate}
      \item $\etapp{(\etfun{\alpha}{\efunt{\cx}{\alpha}{\cx}})}{\tnum}\ 1$
      \item $\etapp{\etapp{(\etfun{\alpha}{\etfun{\beta}{
          \efunt{\cx}{\alpha}{\efunt{\cy}{\beta}{\cy}}}})}{\tnum}}{\tnum}\ 1\ 2$
    \end{enumerate}
\item Write a well-typed expression that becomes the following
  expression by the type erasure:

    $\efun{\cx}{(\cx\ (\efun{\cx}{\cx})\ (\cx\ 1))}$
\end{enumerate}

\item The following language adds
polymorphic non-recursive type definitions to \texttt{\lang}
(the omitted parts are the same as \texttt{\lang}):

\vspace{1em}
$
  \begin{array}{l}
    e ::=\ \cdots\ \ |\ \ \textsf{type}\ t[\alpha]=x@\tau+x@\tau;\ e\ \ |\ \
           e\ \textsf{match}\ x(x)\rightarrow e,\ x(x)\rightarrow e \\
    v ::=\ \cdots\ \ |\ \ \Lambda\alpha.\langle x\rangle\ \ |\ \ \langle x\rangle\ \ |\ \  x(v) \\
    \tau ::=\ \cdots\ \ |\ \  t[\tau] \\
    \Gamma\ \in\ (\embox{Id}\cup\embox{TId}\cup\embox{TN})\finto
    {(T\cup\{\cdot\}
    \cup(\embox{TId}\times\embox{Id}\times T\times\embox{Id}\times T))}
  \end{array}
$
\vspace{0.5em}

$
  \begin{array}{c}
  \inferrule
  { \eval{\sigma[x_1\mapsto\Lambda\alpha.\langle x_1\rangle,x_2\mapsto\Lambda\alpha.\langle x_2\rangle]}{e}{v} }
  { \evald{\textsf{type}\ t[\alpha]=x_1@\tau_1+x_2@\tau_2;\ e}{v} }
  \quad
  \inferrule
  { \evald{e}{\Lambda \alpha.\langle x\rangle} }
  { \evald{\etapp{e}{\tau}}{\langle x\rangle} }
  \\[2em]
  \inferrule
  { \evald{e_1}{\langle x\rangle} \\ \evald{e_2}{v} }
  { \evald{\eapp{e_1}{e_2}}{x(v)} }
  \\[2em]
  \inferrule
  { \evald{e}{x_1(v')} \\ \eval{\sigma[x_3\mapsto v']}{e_1}{v} }
  { \evald{e\ \textsf{match}\ x_1(x_3)\rightarrow e_1,x_2(x_4)\rightarrow e_2}{v} }
  \\[2em]
  \inferrule
  { \evald{e}{x_2(v')} \\ \eval{\sigma[x_4\mapsto v']}{e_2}{v} }
  { \evald{e\ \textsf{match}\ x_1(x_3)\rightarrow e_1,x_2(x_4)\rightarrow e_2}{v} }
  \end{array}
$
\vspace{0.5em}

For example, programmers can write the following code, which defines a polymorphic
option type, in this language:

\vspace{0.5em}
$
  \begin{array}{l}
    \textsf{type}\ \code{option}[\alpha]=\code{None}@\tnum+\code{Some}@\alpha;\\
    \textsf{val}\ \code{getOrElse}=\etfun{\alpha}{\efun{\code{x}{:}\code{option}[\alpha]}{
      \efun{\code{y}{:}\alpha}{(}} \\
    \ \ \ \ \code{x}\ \textsf{match}} \\
    \ \ \ \ \ \ \ \ \code{None}(\code{z})\rightarrow\code{y}, \\
    \ \ \ \ \ \ \ \ \code{Some}(\code{z})\rightarrow\code{z} \\
    );\\
    \eapp{\eapp{\etapp{\code{getOrElse}}{\tnum}}{(\eapp{\etapp{\code{Some}}{\tnum}}{1})}}{2}
  \end{array}
$
\vspace{0.5em}

On the other hand, the following code is not well-typed
since types are not recursive in this language:

\vspace{0.5em}
$
  \begin{array}{l}
    \textsf{type}\ \code{foo}[\alpha]=\code{bar}@\tnum+\code{baz}@\code{foo}[\alpha];\\
    \ldots
  \end{array}
$
\vspace{0.5em}

Note that \code{foo} appears in the definition of itself, which implies that
\code{foo} is a recursive type.

\begin{enumerate}
\item
  Write the typing rules of the form
    \fbox{$\typeofd{e}{\tau}$} of
    $\textsf{type}\ t[\alpha]=x_1@\tau_1+x_2@\tau_2;\ e$ and
    $e\ \textsf{match}\ x_1(x_3)\rightarrow e_1,\ x_2(x_4)\rightarrow e_2$.
\item
  Write the well-formedness rule of the form
    \fbox{$\Gamma\vdash\tau$} of $t[\tau]$.
\item
  Write the type of \code{getOrElse} of the above example.
\end{enumerate}

\end{enumerate}
