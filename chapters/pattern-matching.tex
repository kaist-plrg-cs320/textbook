
\chapter{Pattern Matching}
\labch{pattern-matching}

This section explains pattern matching of Scala.
Pattern matching is one of the key features of functional programming.
It helps programmers handle complex, but structured data.
We have already used a simple form of pattern matching for lists.
This section discusses the benefits
of pattern matching and various patterns available in Scala.
In addition, it will introduce the option type, which is widely-used in
functional programming.

\section{Algebraic Data Types}

It is common to include values of various shapes in a single type.

A natural number is

\begin{itemize}
\item zero or
\item the successor of a natural number.
\end{itemize}

A list is

\begin{itemize}
\item the empty list or
\item a pair of an element and a list.
\end{itemize}

A binary tree is

\begin{itemize}
\item the empty tree or
\item a tree containing a root element and two child trees.
\end{itemize}

An arithmetic expression is

\begin{itemize}
\item a number,
\item the sum of two arithmetic expressions, or
\item the difference of two arithmetic expressions.
\end{itemize}

An expression of the lambda calculus is

\begin{itemize}
\item a variable,
\item a function, which is a pair of a variable and an expression, or
\item a function application, which is a pair of two expressions.
\end{itemize}

As the examples show, in computer science, a single type often includes values of
various shapes. \textit{Algebraic data types}\index{algebraic data type}
(\acrshort{adtLabel}) express such types. An ADT
is the sum type of product types. That is why it is called ``algebraic.''
A \textit{product type}\index{product type}
is a type whose every element is an enumeration of values of types in the
same specific order. Tuple types are typical product types. A \textit{sum
type}\index{sum type}, whose
another name is a \textit{tagged union type}\index{tagged union type},
has values of multiple types as its values. Unlike a union type,
each component of a sum type has a
tag to be distinguished from other components.
In an ADT, one form of values that can be distinguished from the other forms is
called a \textit{variant}\index{variant}.

For example, an arithmetic expression, which has three variants, is

\begin{itemize}
\item a number,
\item the sum of two arithmetic expressions, or
\item the difference of two arithmetic expressions.
\end{itemize}

Therefore, we can define the \code{AE} type, which is the type of an arithmetic
expression, as the sum type of

\begin{itemize}
\item the \code{Int} type tagged with \code{Num},
\item the \code{AE * AE} type tagged with \code{Add}, and
\item the \code{AE * AE} type tagged with \code{Sub},
\end{itemize}

where \code{AE * AE} denotes the product type of \code{AE} and \code{AE}.

ADTs are common in functional languages. Most functional
languages allow users to define new ADTs. The following OCaml
code defines arithmetic expressions:

\begin{verbatim}
type ae =
| Num of int
| Add of ae * ae
| Sub of ae * ae
\end{verbatim}

Scala does not provide a direct way to define ADTs. Instead, Scala provides
traits and classes, which are more general mechanisms to define new types,
and programmers can express ADTs with traits and classes.

A new type can be defined as a trait.
The syntax of a trait definition is as follows:

\begin{verbatim}
trait [name]
\end{verbatim}

It defines a type whose name is \code{[name]}.
The following code defines the \code{AE} type,
which is the type of arithmetic expressions:

\begin{verbatim}
sealed trait AE
\end{verbatim}

The \code{sealed} modifier prevents \code{AE} being extended outside the file
that defines \code{AE}. We will get back to this point when we discuss the
exhaustivity checking of pattern matching.

Once a type is defined as a trait, the type can be used just like any other
types. For example, we can define an identity function for arithmetic
expressions.

\begin{verbatim}
def identity(ae: AE): AE = ae
\end{verbatim}

However, traits do not have ability to construct new values. It means that there
is no way to create a value of the type \code{AE} yet. We need to define the
variants of \code{AE} as case classes by extending \code{AE}.

\begin{verbatim}
case class Num(value: Int) extends AE
case class Add(left: AE, right: AE) extends AE
case class Sub(left: AE, right: AE) extends AE
\end{verbatim}

As you have seen in \refch{introduction-to-scala}, we can easily create values of case classes.

\begin{verbatim}
val n = Num(10)
val m = Num(5)
val e1 = Add(n, m)
val e2 = Sub(e1, Num(3))
\end{verbatim}

Like traits, case classes also define types. The name of each class is the name
of the defined type. Every instance of a class belongs to the type corresponding
to the class.

\begin{verbatim}
val n: Num = Num(10)
val m: Num = Num(5)
val e1: Add = Add(n, m)
val e2: Sub = Sub(e1, Num(3))
\end{verbatim}

In addition, because of the \code{extends} keyword, \code{Num}, \code{Add}, and
\code{Sub} are subtypes of \code{AE}. It means that any value of the types
\code{Num}, \code{Add}, or \code{Sub} is also a value of the type \code{AE}.

\begin{verbatim}
val n: AE = Num(10)
val m: AE = Num(5)
val e1: AE = Add(n, m)
val e2: AE = Sub(e1, Num(3))
\end{verbatim}

We know that we can access the fields of objects with their names.

\begin{verbatim}
val n: Num = Num(10)
n.value
\end{verbatim}

However, we cannot access the fields of an object when its type becomes \code{AE}.

\begin{verbatim}
val m: AE = Num(10)
m.value
\end{verbatim}
\vspace{-1em}
\begin{mdframed}[hidealllines=true,backgroundcolor=red!10,innerleftmargin=3pt,innerrightmargin=3pt,leftmargin=-3pt,rightmargin=-3pt]
\begin{verbatim}
  ^
error: value value is not a member of AE
\end{verbatim}
\vspace{-2em}
\begin{flushright}
\scriptsize\textsf{Compile-time error}
\end{flushright}
\end{mdframed}

The reason is that \code{m} can be \code{Add} or \code{Sub}, which do not have
the field \code{value}, as \code{AE} consists of not only \code{Num} but also
\code{Add} and \code{Sub}. The compiler thinks that \code{m} may not have the
field \code{value} and considers \code{m.value} as an unsafe expression, which
should be rejected.

The best way to use ADTs is pattern matching. The following function evaluates a
given arithmetic expression and returns the number denoted by the arithmetic
expression.

\begin{verbatim}
def eval(e: AE): Int = e match {
  case Num(n) => n
  case Add(l, r) => eval(l) + eval(r)
  case Sub(l, r) => eval(l) - eval(r)
}

assert(eval(Sub(Add(Num(3), Num(7)), Num(5))) == 5)
\end{verbatim}

When \code{e} is \code{Num(n)}, \code{eval} simply returns \code{n}.
When \code{e} is \code{Add(l, r)}, \code{eval} respectively evaluates \code{l}
and \code{r}, which are arithmetic expressions. \code{eval} returns the sum of
the results of \code{l} and \code{r}.
The \code{Sub(l, r)} case is similar. \code{eval} returns the difference of
the results of \code{l} and \code{r}.

The list type is another good example of an ADT. The Scala standard library defines
lists similar to the following code:

\begin{verbatim}
sealed trait List[+A]
case object Nil extends List[Nothing]
case class ::[A](head: A, tail: List[A]) extends List[A]
\end{verbatim}

This code omits some details but clearly shows the high-level idea to define
lists.\footnote{We will not see what \code{[+A]} and \code{Nothing} are here.
You can understand the overall ADT structure without knowing those concepts.}
A list is either the empty
list or a nonempty list, which is a pair of its head and tail. \code{Nil} is
defined as a case object, not a case class, since there is only one empty list.
Every empty list is identical. We use a case object to express this idea.
\code{Nil} is created only once during entire execution, and every \code{Nil} is
identitcal. The name \code{::} looks a bit weird, but it is for
readability of pattern matching. Scala allows writing class names as infix
operators in patterns. It means that both \code{case ::(h, t) =>} and \code{case
h :: t =>} are allowed. Due to the class name \code{::}, we can write \code{case
h :: t =>} in pattern matching.

\section{Advantages}

\subsection{Conciseness}

Without pattern matching, handling ADTs becomes a complicated job. We need to
use dynamic type tests to distinguish variants and type casting to access the
fields of values.

Below is \code{eval} without pattern matching:

\begin{verbatim}
def eval(e: AE): Int =
  if (e.isInstanceOf[Num])
    e.asInstanceOf[Num].value
  else if (e.isInstanceOf[Add]) {
    val e0 = e.asInstanceOf[Add]
    eval(e0.left) + eval(e0.right)
  } else {
    val e0 = e.asInstanceOf[Sub]
    eval(e0.left) - eval(e0.right)
  }
\end{verbatim}

\code{e.isInstanceOf[Num]} tests whether \code{e} is an instance of class
\code{Num}. If it is true, \code{eval} should return the value of the field
\code{value} of \code{e}. However, \code{value} cannot be directly accessed
since \code{e}'s type is \code{AE}. Because \code{e.isInstanceOf[Num]} is true,
we are sure that \code{e}'s actual type is \code{Num}. In this case, we can
inform this knowledge to the compiler with type casting. \code{e.asInstanceOf[Num]}
does not change the value denoted by \code{e} but lets the compiler know that
the programmer guarantees the type of \code{e} to be \code{Num}. Therefore, the
compiler considers \code{e.asInstanceOf[Num]} to belong \code{Num} and allows
accessing the field \code{value}. These type tests and casting processes should
be done for the other variants, \code{Add} and \code{Sub}, too.

The code is long and complicated despite its simple functionality. Dynamic type
tests and explicit type casting occupy most of the code, while real
computation requires short code. Besides, such code is error-prone.
For example, programmers may write code like below by mistake:

\begin{verbatim}
else if (e.isInstanceOf[Add]) {
  val e0 = e.asInstanceOf[Sub]
  eval(e0.left) + eval(e0.right)
}
\end{verbatim}

While the condition checks whether \code{e} is an instance of \code{Add},
\code{e} becomes casted to \code{Sub}. Such code will trigger an error at run
time and terminate the execution abnormally.
It is easy to check whether \code{eval} is correct because it is short.
However, complex types and computation will increase the possibility of
mistakes.

Pattern matching gives us a much better solution. Pattern matching hides type
tests and casting and makes code concise. At the same time, pattern matching
removes the possibility of mistakes.

\subsection{Exhaustivity Checking}

Pattern matching checks the exhaustivity of patterns. At run time, a match error
occurs when a given value matches none of given patterns.

\begin{verbatim}
def eval(e: AE): Int = e match {
  case Add(l, r) => eval(l) + eval(r)
  case Sub(l, r) => eval(l) - eval(r)
}
\end{verbatim}

The function lacks the \code{Num} pattern.

\begin{verbatim}
eval(Num(3))
\end{verbatim}
\vspace{-1em}
\begin{mdframed}[hidealllines=true,backgroundcolor=red!10,innerleftmargin=3pt,innerrightmargin=3pt,leftmargin=-3pt,rightmargin=-3pt]
\begin{verbatim}
scala.MatchError: Num(3) (of class Num)
\end{verbatim}
\vspace{-2em}
\begin{flushright}
\scriptsize\textsf{Run-time error}
\end{flushright}
\end{mdframed}

An argument of type \code{Num} results in a match error at run time.

Fortunately, we can easily avoid such mistakes.
The Scala compiler checks whether patterns are exhaustive and warns if they are not.

\begin{verbatim}
def eval(e: AE): Int = e match {
  case Add(l, r) => eval(l) + eval(r)
  case Sub(l, r) => eval(l) - eval(r)
}
\end{verbatim}
\vspace{-1em}
\begin{mdframed}[hidealllines=true,backgroundcolor=yellow!10,innerleftmargin=3pt,innerrightmargin=3pt,leftmargin=-3pt,rightmargin=-3pt]
\begin{verbatim}
                       ^
warning: match may not be exhaustive.
It would fail on the following input: Num(_)
\end{verbatim}
\vspace{-2em}
\begin{flushright}
\scriptsize\textsf{Compile-time warning}
\end{flushright}
\end{mdframed}

The compiler warns programmers about that the patterns are not exhaustive. Moreover, it precisely
informs which patterns are missing to help debugging.
Exhaustivity checking is beneficial for complex programs. It helps programmers make
error-free programs and thus is a crucial strength of pattern matching.

For exhaustivity checking, the \code{sealed} modifier of traits is necessary.
Without \code{sealed}, a trait can be extended outside the file that defines it.
The unit of compilation is a single file, so it is impossible to find
all the variants by scanning a single file when a trait is not sealed.
Exhaustivity checking during pattern matching will be impossible.
The \code{sealed} keyword resolves the problem. Since sealed traits cannot be
extended further, it is enough to check only the file that defines a sealed trait to find
every variant of the trait. It is why we use sealed traits to define ADTs.

\subsection{Reachability Checking}

Like \code{switch-case}, pattern
matching compares a value to patterns sequentially from top to bottom and selects
the first matching pattern. If there are duplicated patterns, the latter will
not be reachable.
The compiler warns programmers when they find unreachable patterns to prevent such code.

\begin{verbatim}
def eval(e: AE): Int = e match {
  case Num(n) => 0
  case Add(l, r) => eval(l) + eval(r)
  case Num(n) => n
  case Sub(l, r) => eval(l) - eval(r)
}
\end{verbatim}
\vspace{-1em}
\begin{mdframed}[hidealllines=true,backgroundcolor=yellow!10,innerleftmargin=3pt,innerrightmargin=3pt,leftmargin=-3pt,rightmargin=-3pt]
\begin{verbatim}
  case Num(n) => n
                 ^
warning: unreachable code
\end{verbatim}
\vspace{-2em}
\begin{flushright}
\scriptsize\textsf{Compile-time warning}
\end{flushright}
\end{mdframed}

When code is simple and short, it is easy to check whether there are unreachable
patterns. However, in complex code, programmers often insert unreachable
patterns by mistake and make critical bugs. Reachability checking of the
compiler is an important feature to prevent such bugs.

\section{Patterns in Scala}

\subsection{Constant and Wildcard Patterns}

\code{switch-case} statements divide a given value into multiple cases in
imperative languages. Pattern matching is a general form of \code{switch-case}.
The following code is an example using a \code{switch-case} statement in Java:

\begin{verbatim}
String grade(int score) {
  switch (score / 10) {
    case 10: return "A";
    case 9: return "A";
    case 8: return "B";
    case 7: return "C";
    case 6: return "D";
    default: return "F";
  }
}
\end{verbatim}

Constant and wildcard patterns exist in Scala. Constant patterns are
literals like integers and strings. A constant pattern matches a given
value if a value denoted by the pattern equals the given value. The underscore
denotes the wildcard pattern, which matches every value, and is equivalent to
\code{default} of \code{switch-case}. The following function rewrites the
previous function with pattern matching:

\begin{verbatim}
def grade(score: Int): String =
  (score / 10) match {
    case 10 => "A"
    case 9 => "A"
    case 8 => "B"
    case 7 => "C"
    case 6 => "D"
    case _ => "F"
  }

assert(grade(85) == "B")
\end{verbatim}

\subsection{Or Patterns}

\code{switch-case} statements use the fall-through semantics; if \code{break}
does not exist, after executing code corresponding to a case, the flow of the
execution moves to code corresponding to the next case. Since the results of
cases \code{10} and \code{9} are identical, the function can use fall-through.

\begin{verbatim}
String grade(int score) {
  switch (score / 10) {
    case 10:
    case 9: return "A";
    case 8: return "B";
    case 7: return "C";
    case 6: return "D";
    default: return "F";
  }
}
\end{verbatim}

In contrast, pattern matching disallows fall-through. Instead, or patterns
give a way to write the same expression only once for multiple patterns. The
syntax of an or pattern is \code{[pattern] | [pattern] | …}, which is a sequence
of multiple patterns with vertical bars in between. \code{A | B} matches values
that match \code{A} or \code{B}.

\begin{verbatim}
def grade(score: Int): String =
  (score / 10) match {
    case 10 | 9 => "A"
    case 8 => "B"
    case 7 => "C"
    case 6 => "D"
    case _ => "F"
  }

assert(grade(100) == "A")
\end{verbatim}

\subsection{Nested Patterns}

Nested patterns are patterns containing patterns.
The \code{optimizeAdd} function
optimizes a given arithmetic expression by eliminating additions of zeros.

\begin{verbatim}
def optimizeAdd(e: AE): AE = e match {
  case Num(_) => e
  case Add(Num(0), r) => optimizeAdd(r)
  case Add(l, Num(0)) => optimizeAdd(l)
  case Add(l, r) => Add(optimizeAdd(l), optimizeAdd(r))
  case Sub(l, r) => Sub(optimizeAdd(l), optimizeAdd(r))
}
\end{verbatim}

Nested patterns help programmers treat values with complex structures easily.

\subsection{Patterns with Binders}

Assume that we have one more variant of \code{AE}:

\begin{verbatim}
case class Abs(e: AE) extends AE
\end{verbatim}

It denotes the absolute value of an operand.
Optimizing an arithmetic expression decorated by two
consecutive \code{Abs} operators results in the arithmetic expression with only
one \code{Abs} operator.

\begin{verbatim}
def optimizeAbs(e: AE): AE = e match {
  case Num(_) => e
  case Add(l, r) => Add(optimizeAbs(l), optimizeAbs(r))
  case Sub(l, r) => Sub(optimizeAbs(l), optimizeAbs(r))
  case Abs(Abs(e0)) => optimizeAbs(Abs(e0))
  case Abs(e0) => Abs(optimizeAbs(e0))
}
\end{verbatim}

A flaw of the implementation is that a value matching \code{Abs(e0)}
cannot be an argument of \code{optimizeAbs} directly, and constructing a new
\code{Abs} instance containing a value matching \code{e0} is essential.
The \code{@} symbol makes code efficient by binding a value matching to a pattern to a variable.
Pattern \code{[variable] @ [pattern]} makes the variable refer to a value
matching the pattern.

\begin{verbatim}
def optimizeAbs(e: AE): AE = e match {
  case Num(_) => e
  case Add(l, r) => Add(optimizeAbs(l), optimizeAbs(r))
  case Sub(l, r) => Sub(optimizeAbs(l), optimizeAbs(r))
  case Abs(e0 @ Abs(_)) => optimizeAbs(e0)
  case Abs(e0) => Abs(optimizeAbs(e0))
}
\end{verbatim}

\subsection{Type Patterns}

In \code{optimizeAbs},
the first \verb!Num(_)! pattern does no more than checking whether a value
belongs to type \code{Num}. A type pattern helps to rewrite the function. Type
patterns are in the form of \code{[name]: [type]}. If a value belongs to the
type, it matches the pattern, and the variable refers to the value. The wildcard
pattern can substitute the name if the variable is unnecessary.

\begin{verbatim}
def optimizeAbs(e: AE): AE = e match {
  case _: Num => e
  case Add(l, r) => Add(optimizeAbs(l), optimizeAbs(r))
  case Sub(l, r) => Sub(optimizeAbs(l), optimizeAbs(r))
  case Abs(e0 @ Abs(_)) => optimizeAbs(e0)
  case Abs(e0) => Abs(optimizeAbs(e0))
}
\end{verbatim}

Type patterns are useful for dynamic type checking. The following function takes
any value as an argument and check whether it is a string or
not.\footnote{Every type is a subtype of \code{Any}, i.e. every value belongs to
\code{Any}.}

\begin{verbatim}
def show(x: Any): String = x match {
  case s: String => s + " is a string"
  case _ => "not a string"
}

assert(show("1") == "1 is a string")
assert(show(1) == "not a string")
\end{verbatim}

Note that type patterns cannot check type arguments of polymorphic types. Using
type patterns against polymorphic types is dangerous.

\begin{verbatim}
def show(x: Any): String = x match {
  case _: List[String] => "a list of strings"
  case _ => "not a list of strings"
}
\end{verbatim}
\vspace{-1em}
\begin{mdframed}[hidealllines=true,backgroundcolor=yellow!10,innerleftmargin=3pt,innerrightmargin=3pt,leftmargin=-3pt,rightmargin=-3pt]
\begin{verbatim}
          ^
warning: non-variable type argument String in type pattern
List[String] is unchecked since it is eliminated by erasure
\end{verbatim}
\vspace{-1.5em}
\begin{flushright}
\scriptsize\textsf{Compile-time warning}
\end{flushright}
\end{mdframed}

\begin{verbatim}
val l: List[Int] = List(1, 2, 3)
assert(show(l) == "a list of strings")  // weird result
\end{verbatim}

Although the type of the argument is \code{List[Int]}, it matches the first
pattern. As the warnings imply, the JVM uses type erasure
semantics, and type arguments are unavailable at run time.

\subsection{Tuple Patterns}

The syntax of a tuple pattern is \code{([pattern], …, [pattern])}.
It matches a tuple whose elements respectively match the internal patterns.

The following function uses tuple patterns to check
whether two lists are identical:

\begin{verbatim}
def equal(l0: List[Int], l1: List[Int]): Boolean =
  (l0, l1) match {
    case (h0 :: t0, h1 :: t1) =>
      h0 == h1 && equal(t0, t1)
    case (Nil, Nil) => true
    case _ => false
  }
\end{verbatim}

\subsection{Pattern Guards}

A binary search tree is

\begin{itemize}
\item the empty tree or
\item a tree containing an integral root element and two child trees.
\end{itemize}

\begin{verbatim}
sealed trait Tree
case object Empty extends Tree
case class Node(root: Int, left: Tree, right: Tree) extends Tree
\end{verbatim}

The function \code{add} takes a tree and an integer as arguments and returns a tree
obtained by adding the integer to the tree. If the integer is an element of the
given tree, the tree itself is the return value.

\begin{verbatim}
def add(t: Tree, n: Int): Tree =
  t match {
    case Empty => Node(n, Empty, Empty)
    case Node(m, t0, t1) =>
      if (n < m)
        Node(m, add(t0, n), t1)
      else if (n > m)
        Node(m, t0, add(t1, n))
      else
        t
  }
\end{verbatim}

An expression corresponding to the second pattern uses \code{if-else}. Pattern
guards allow adding constraints to patterns. A pattern in the form of
\code{[pattern] if [expression]} matches a value if the value matches the
pattern, and the expression results in \code{true}. The following version of \code{add}
uses pattern guards:

\begin{verbatim}
def add(t: Tree, n: Int): Tree =
  t match {
    case Empty => Node(n, Empty, Empty)
    case Node(m, t0, t1) if n < m =>
      Node(m, add(t0, n), t1)
    case Node(m, t0, t1) if n > m =>
      Node(m, t0, add(t1, n))
    case _ => t
  }
\end{verbatim}

Guarded patterns may be inexhaustive and need care.

\begin{verbatim}
def add(t: Tree, n: Int): Tree =
  t match {
    case Empty => Node(n, Empty, Empty)
    case Node(m, t0, t1) if n < m =>
      Node(m, add(t0, n), t1)
    case Node(m, t0, t1) if n > m =>
      Node(m, t0, add(t1, n))
  }
\end{verbatim}

The patterns in the above code is not exhaustive, but
the compiler does not warn programmers about the inexhaustivity.

\subsection{Patterns with Backticks}

The function \code{remove} takes a tree and an integer as arguments and returns a
tree obtained by removing the integer from the tree. If the integer is not an
element of the tree, the given tree itself is the return value. \code{removeMin}
is a helper function used by \code{remove}. It returns the pair of the smallest
element of a given tree and a tree obtained by removing the element from the
tree.

\begin{verbatim}
def removeMin(t: Node): (Int, Tree) = {
  t match {
    case Node(n, Empty, t1) =>
      (n, t1)
    case Node(n, t0: Node, t1) =>
      val (min, t2) = removeMin(t0)
      (min, Node(n, t2, t1))
  }
}

def remove(t: Tree, n: Int): Tree = {
  t match {
    case Empty =>
      Empty
    case Node(m, t0, Empty) if n == m =>
      t0
    case Node(m, t0, t1: Node) if n == m =>
      val res = removeMin(t1)
      val min = res._1
      val t2 = res._2
      Node(min, t0, t2)
    case Node(m, t0, t1) if n < m =>
      Node(m, remove(t0, n), t1)
    case Node(m, t0, t1) if n > m =>
      Node(m, t0, remove(t1, n))
  }
}
\end{verbatim}

\verb!Node(`n`, t0, Empty)! can replace
\code{case Node(m, t0, Empty) if n == m}. The pattern \code{Node(n, t0, Empty)} defines
a new variable \code{n} and makes \code{n} refer to the
root element; it does not check whether the root element equals \code{n}.
However, backticks prohibit defining a new variable and allow to compare the root
element to \code{n} in the scope.

\begin{verbatim}
def remove(t: BinTree, n: Int): BinTree = {
  t match {
    case Empty =>
      Empty
    case Node(`n`, t0, Empty) =>
      t0
    case Node(`n`, t0, t1: Node) =>
      val res = removeMin(t1)
      val min = res._1
      val t2 = res._2
      Node(min, t0, t2)
    case Node(m, t0, t1) if n < m =>
      Node(m, remove(t0, n), t1)
    case Node(m, t0, t1) if n > m =>
      Node(m, t0, remove(t1, n))
  }
}
\end{verbatim}

\section{Applications of Pattern Matching}

\subsection{Variable Definitions}

It is possible to define variables with pattern matching.

\begin{verbatim}
val (n, m) = (1, 2)
assert(n == 1 && m == 2)

val (a, b, c) = ("a", "b", "c")
assert(a == "a" && b == "b" && c == "c")

val h :: t = List(1, 2, 3, 4)
assert(h == 1 && t == List(2, 3, 4))

val Add(l, r) = Add(Num(1), Num(2))
assert(l == Num(1) && r == Num(2))
\end{verbatim}

Pattern matching helps programmers declare variables concisely, but a match error occurs
when the pattern does not match the right-hand-side value. It is desirable to use
pattern matching only when there is a guarantee that the match succeeds. Since
a tuple pattern always matches a tuple value of the same length,
tuple patterns are widely used for variable definitions.

\subsection{Anonymous Functions}

The function \code{toSum} takes a list of pairs of two integers as arguments and
returns a list whose elements are the sums of the integers in the pairs.

\begin{verbatim}
def toSum(l: List[(Int, Int)]): List[Int] =
  l.map(p => p match {
    case (n, m) => n + m
  })

val l = List((0, 1), (2, 3), (3, 4))
assert(toSum(l) == List(1, 5, 7))
\end{verbatim}

The anonymous function directly uses parameter \code{p} as the target of the
pattern matching. Scala allows simplification of \verb!x => x match { … }! to
\verb!{ … }!. Therefore, we can use an enumeration of patterns as an anonymous
function.

\begin{verbatim}
def toSum(l: List[(Int, Int)]): List[Int] =
  l.map({ case (n, m) => n + m })
\end{verbatim}

\subsection{For Loops}

\code{toSum} can use a for expression instead of \code{map}.

\begin{verbatim}
def toSum(l: List[(Int, Int)]): List[Int] =
  for (p <- l)
    yield p match { case (n, m) => n + m }
\end{verbatim}

For expressions directly support pattern matching.

\begin{verbatim}
def toSum(l: List[(Int, Int)]): List[Int] =
  for ((n, m) <- l)
    yield n + m
\end{verbatim}

The code is readable and concise.

\section{Options}
\labsec{options}

The option type is a widely-used ADT. It represents a value whose existence is
optional. This section introduces the option type and explains the usage of
options.

Consider the function \code{get}, which takes a list and integer \code{n} as
arguments and returns the \code{n}th element of the list. It is problematic
when \code{n} is negative or exceeds the length of the list. Throwing exceptions
is a widely used solution in imperative languages. In Scala, \code{throw
[expression]} throws an exception. For convenience, we define the function
\code{error}, which throws an exception, like below and use it throughout the
book.

\begin{verbatim}
def error(msg: String) = throw new Exception(msg)

def get(l: List[Int], n: Int): Int =
  if (n < 0)
    error("index out of bounds")
  else l match {
    case Nil =>
      error("index out of bounds")
    case h :: t =>
      if (n == 0)
        h
      else
        get(t, n - 1)
  }
\end{verbatim}

Throwing an exception is a simple and effective solution. However, exceptions
have two problems. First, exceptions should be handled by exception handlers.

\begin{verbatim}
try {
  get(List(1, 2), 2)
} catch {
  case e: Exception =>
    // prints "index out of bounds"
    println(e.getMessage)
}
\end{verbatim}

If an exception is not handled properly, it will eventually cause a run-time
error and terminate the execution.

\begin{verbatim}
get(List(1, 2), 2)
\end{verbatim}
\vspace{-1em}
\begin{mdframed}[hidealllines=true,backgroundcolor=red!10,innerleftmargin=3pt,innerrightmargin=3pt,leftmargin=-3pt,rightmargin=-3pt]
\begin{verbatim}
java.lang.Exception: index out of bounds
\end{verbatim}
\vspace{-2em}
\begin{flushright}
\scriptsize\textsf{Run-time error}
\end{flushright}
\end{mdframed}

The Scala compiler does not check whether exceptions are handled properly.
It means that there will not be any compile-time error even if there is a
possibility of unhandled exceptions.

Another problem of exceptions is that exception handling is not local.
When an exception is thrown, the control flow suddenly jumps to the position of
the nearest exception handler. Non-local transition of the control flow usually
hinders programmers from understanding code.
Therefore, implementing \code{get} without exceptions is desirable.

The first attempt is to use \code{null}. \code{null} is a value that denotes that
it does not refer to any existing object. We can try to make \code{get} return
\code{null} when a given index is invalid.

\begin{verbatim}
def get(l: List[Int], n: Int): Int =
  if (n < 0)
    null
  else l match {
    case Nil => null
    case h :: t =>
      if (n == 0)
        h
      else
        get(t, n - 1)
  }
\end{verbatim}
\vspace{-1em}
\begin{mdframed}[hidealllines=true,backgroundcolor=red!10,innerleftmargin=3pt,innerrightmargin=3pt,leftmargin=-3pt,rightmargin=-3pt]
\begin{verbatim}
    null
    ^
error: an expression of type Null is ineligible
for implicit conversion

    case Nil => null
                ^
error: an expression of type Null is ineligible
for implicit conversion
\end{verbatim}
\vspace{-2em}
\begin{flushright}
\scriptsize\textsf{Run-time error}
\end{flushright}
\end{mdframed}

Unfortunately, \code{null} is not an element of \code{Int} in Scala.
The compiler rejects the code.
Even with the assumption that we can treat \code{null} as \code{Int},
using \code{null} is a bad solution. Dereferencing \code{null} causes a
run-time error, which is the well-known \code{NullPointerException}.
The compiler does not check whether \code{null} is dereferenced.
Therefore, using \code{null} is nothing better than using exceptions.
Use of \code{null} has been criticized enormously because \code{null} is extremely
error-prone. Even Tony Hoare, the inventor of \code{null}, said that inventing
\code{null} was a terrible mistake:

\begin{quote}
I call it my billion-dollar mistake. It was the invention of the null reference
in 1965.\footnote{\url{https://en.wikipedia.org/wiki/Null\_pointer\#History}}
\end{quote}

The second attempt is to use a particular error-indicating value, e.g. \code{-1}.

\begin{verbatim}
def get(l: List[Int], n: Int): Int =
  if (n < 0)
    -1
  else l match {
    case Nil =>
      -1
    case h :: t =>
      if (n == 0)
        h
      else
        get(t, n - 1)
  }
\end{verbatim}

The strategy has an obvious problem. The caller cannot distinguish two
situations:
\begin{itemize}
  \item The list contains \code{-1}.
  \item The index is invalid.
\end{itemize}
Default values can be successful solutions for certain purposes but do not fit \code{get}.

Instead of using a fixed particular value in \code{get}, the caller can specify the default value.

\begin{verbatim}
def getOrElse(l: List[Int], n: Int, default: Int): Int =
  if (n < 0)
    default
  else l match {
    case Nil =>
      default
    case h :: t =>
      if (n == 0)
        h
      else
        getOrElse(t, n - 1, default)
}
\end{verbatim}

It works well when an appropriate default value
exists. However, when checking failures is per se important, the new strategy is
as bad as the previous strategy. There is no way to distinguish an element and
the default value.

Functional languages provide the option type to handle erroneous situations
safely. As the name implies, it represents an optional existence of a value.
The Scala standard library defines the option type like below.\footnote{
We will not see what \code{[+A]} and \code{Nothing} are here.
You can understand the overall ADT structure without knowing those concepts.}

\begin{verbatim}
sealed trait Option[+A]
case object None extends Option[Nothing]
case class Some[A](value: A) extends Option[A]
\end{verbatim}

An option that may have a value of type \code{T} has type \code{Option[T]}.
An option is either \code{None} or \code{Some}.
\code{None} is a value that does not denote any value and similar
to \code{null}. It indicates a problematic situation. Like \code{Nil}, it is
defined as a case object because every \code{None} is identical. \code{Some} constructs a value that
denotes that a value exists. It is similar to a reference to a real object and
indicates that computation has succeeded.

The following code defines \code{getOption}, which returns an option.

\begin{verbatim}
def getOption(l: List[Int], n: Int): Option[Int] =
  if (n < 0)
    None
  else l match {
    case Nil =>
      None
    case h :: t =>
      if (n == 0)
        Some(h)
      else
        getOption(t, n - 1)
  }

assert(getOption(List(1, 2), 0) == Some(1))
assert(getOption(List(1, 2), 2) == None)
\end{verbatim}

For an invalid index, the return value is \code{None}. The caller can notice
that the operation has failed by \code{None}.
Otherwise, the function packs
an element inside \code{Some} to make the return value.

The Scala standard library uses options in many places. Various methods return options.
For example, \code{headOption} of a list returns \code{None} when the list is
empty. Otherwise, \code{Some} containing the head of the list is returned.

\begin{verbatim}
assert(List().headOption == None)
assert(List(1).headOption == Some(1))
\end{verbatim}

Also, \code{get} of a map returns \code{None} when the map does not have a given key.
Otherwise, \code{Some} containing the value corresponding to the key is
returned.

\begin{verbatim}
val m = Map(1 -> "one", 2 -> "two")
assert(m.get(0) == None)
assert(m.get(1) == Some("one"))
\end{verbatim}

Pattern matching allows programmers to deal with options by
distinguishing the \code{None} and \code{Some} cases. In addition, like the
methods of lists, options also provide methods to abstract common patterns.
We are going to see two methods: \code{map} and \code{flatMap}.

\code{map} can be used when we want to apply some computation only when the
previous computation has succeeded. \code{map} takes a single argument, which
must be a function. \code{opt.map(f)} results in \code{None} when \code{opt} is
\code{None}. If \code{opt} is \code{Some(v)}, then \code{opt.map(f)} evaluates
to \code{Some(f(v))}.

As an example, let us consider a map containing students.
Names are the keys, and students are the values. We want to find a student by a name and
get one's height only when the student exists. It can be implemented with
\code{map}.

\begin{verbatim}
def getHeight(
  m: Map[String, Student],
  name: String
): Option[Int] =
  m.get(name).map(_.height)
\end{verbatim}

If \code{m.get(name)} is \code{None}, then \code{m.get(name).map(\_.height)} also
is \code{None}. Otherwise, \code{m.get(name)} should be \code{Some(student)}, and
\code{m.get(name).map(\_.height)} will result in \code{Some(student.height)}.

In summary, \code{map} is useful when the computation consists of two steps, and
the first step can fail.

\code{flatMap} is similar to \code{map} but a bit different. It is useful when
the computation consists of two steps, and both steps can fail.
\code{flatMap} takes a single argument, which must be a function that returns an option.
\code{opt.flatMap(f)} results in \code{None} when \code{opt} is
\code{None}. If \code{opt} is \code{Some(v)}, then \code{opt.flatMap(f)} evaluates
to \code{f(v)}.

Let us consider a list of names and a map like before.
When the list is nonempty, we will find a student with the first name in the
list from the map. It is a typical application of \code{flatMap}.

\begin{verbatim}
def getStudent(
  l: List[String],
  m: Map[String, Student]
): Option[Student] =
  l.headOption.flatMap(m.get)
\end{verbatim}

The standard library provides many other useful methods for
options.\footnote{\url{https://www.scala-lang.org/api/current/scala/Option.html}}
