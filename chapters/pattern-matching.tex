\setchapterpreamble[u]{\margintoc}
\chapter{Pattern Matching}
\labch{pattern-matching}

It is the fourth article about functional programming. The last article dealt
with the \verb!List! and \verb!Option! types of the Scala standard library and
\verb!for! expressions of Scala. This article is about pattern matching. The
previous articles have already used pattern matching for lists and options, but
the form of the pattern matching was simple. The article discusses the benefits
of pattern matching and various patterns available in Scala. The homework of the
course requires pattern matching, and understanding the article thus is crucial
for the homework.

\section{Algebraic Data Types}

It is common to include values of various shapes in a single type.

A natural number is

\begin{itemize}
\item zero or
\item the successor of a natural number.
\end{itemize}

A list is

\begin{itemize}
\item the empty list or
\item the pair of an element and a list.
\end{itemize}

A binary tree is

\begin{itemize}
\item the empty tree or
\item a tree containing a root element and two child trees.
\end{itemize}

An arithmetic expression is

\begin{itemize}
\item a number,
\item the sum of two arithmetic expressions,
\item the difference of two arithmetic expressions,
\item the product of two arithmetic expressions, or
\item the ratio of two arithmetic expressions.
\end{itemize}

In general, an arithmetic expression is

\begin{itemize}
\item a number,
\item an arithmetic expression with a unary operator, or
\item two arithmetic expressions with a binary operator.
\end{itemize}

An expression of the \term{lambda} \term{calculus}, which is one of the topics of
the course, is

\begin{itemize}
\item a variable,
\item a function, which is the pair of a variable and an expression, or
\item a function application, which is the pair of two expressions.
\end{itemize}

As the examples show, in computer science, a type often includes values of
various shapes. Algebraic data types typically express such types. An algebraic
data type is a \term{sum} \term{type} of \term{product} \term{types}. A product
type is a type whose every element is an enumeration of values of types in the
same specific order. Tuple types are typical product types. A sum type, whose
another name is a \term{tagged} \term{union} \term{type}, has values of multiple
types as its values. Unlike a union type, each \term{variant} of a sum type has a
'tag' to be distinguished from other variants.

For example, an arithmetic expression is

\begin{itemize}
\item an integer or
\item the sum of two arithmetic expressions.
\end{itemize}

Therefore, the \verb!AE! type is the sum type of

\begin{itemize}
\item the \verb!Int! type tagged with \verb!Num! and
\item the \verb!AE * AE! type (the product type of \verb!AE! and \verb!AE!) tagged
with \verb!Sum!.
\end{itemize}

\section{Defining Algebraic Data Types in Scala}

Algebraic data types are common in functional languages. Most functional
languages allow users to define new algebraic data types. The following OCaml
code defines arithmetic expressions:

\begin{verbatim}
type ae =
| Num of int
| UnOp of string * ae
| BinOp of string \term{ ae } ae
\end{verbatim}

Scala is functional and object-oriented at the same time. A language defining
types of objects and algebraic data types with two different ways is inconsistent
and complex. Like many other object-oriented languages, Scala lets programmers
define a new type by defining a new class. Types defined with classes can
simulate algebraic data types.

\verb!class [class name]! defines a class of a given name. \verb!class A! defines
a class whose name is \verb!A!. \verb!A! is a name of the class and a type
simultaneously, for defining a class is defining a type. Every instance of class
\verb!A! belongs to type \verb!A!. The following code defines the \verb!AE! type,
which is a type of algebraic expressions:

\begin{verbatim}
class AE
\end{verbatim}

Classes define variants belonging to \verb!AE!. \verb!Num!, \verb!UnOp!, and
\verb!BinOp! are types denoting a number, an arithmetic expression with a unary
operator, and two arithmetic expressions with a binary operator respectively.

\begin{verbatim}
class AE
class Num
class UnOp
class BinOp
\end{verbatim}

However, the code does not inform that \verb!AE! subsumes the variants. Every
value of the \verb!Num!, \verb!UnOp!, and \verb!BinOp! types is not a value of
the \verb!AE! type.

Defining a class by extending a class makes the latter subsume the former. Code
like \verb!class [class name] extends [parent class name]! make the latter class
be a parent of the former class. \verb!class A extends B! implies that class
\verb!A! is a child class, or a \term{subclass}, of class \verb!B!, and class
\verb!B! is a parent class, or a \term{superclass}, of class \verb!A!. The
\verb!A! class \term{inherits} from the \verb!B! class. \verb!A! is a subtype of
\verb!B!; \verb!B! is a \term{supertype} of \verb!A!; every element of type
\verb!A! is an element of type \verb!B!.

\begin{verbatim}
class AE
class Num extends AE
class UnOp extends AE
class BinOp extends AE
\end{verbatim}

\verb!Num!, \verb!UnOp!, and \verb!BinOp! are subtypes of \verb!AE!. Values of
the types belong to \verb!AE!.

Instances of the \verb!Num! class cannot express integers whom they denote. In
the same manner, instances of \verb!UnOp! and \verb!BinOp! cannot express
operators and arithmetic expressions. Class parameters allow objects to store
additional information. Form \verb!class [class name](val [parameter name]: [parameter type], …)!
specifies class parameters. \verb!val! is unnecessary for
some usages but not in the article.

\begin{verbatim}
class AE
class Num(val n: Int) extends AE
class UnOp(val op: String, val e: AE) extends AE
class BinOp(val op: String, val e0: AE, val e1: AE) extends AE
\end{verbatim}

The declarations of \verb!Num!, \verb!UnOp!, and \verb!BinOp! denote that their
instances respectively store an integer, a string and an arithmetic expression,
and a string and two arithmetic expressions.

Calling a \term{constructor} creates an object. Constructor calls are in the form
of \verb!new [class name]([expression], …)!.

\begin{verbatim}
new Num(1): AE
new UnOp("-", new Num(2)): AE
new BinOp("+", new Num(1), new UnOp("-", new Num(2))): AE
\end{verbatim}

Type \verb!AE! includes instances of the three child classes of \verb!AE!.

The parameters of a class are the \term{fields} of the class. The values of
arguments used for a constructor call are the values of fields. As method calls
need periods, field accesses require periods. Expression
\verb![expressions].[field name]! denotes the value of the field of an object
obtained by evaluating the expression.

\begin{verbatim}
new Num(1).n  // 1
new UnOp("-", new Num(2)).op  // "-"
new BinOp("+", new Num(1), new UnOp("-", new Num(2))).op  // "+"
\end{verbatim}

The code is still problematic.

\begin{verbatim}
new AE: AE
\end{verbatim}

An arithmetic expression cannot be anything other than a number, an arithmetic
expression with a unary operator, or two arithmetic expressions with a binary
operator. Writing \verb!new AE! is meaningless. Therefore, it must be impossible
to construct an instance of the \verb!AE! class directly.

\begin{verbatim}
trait AE
class Num(val n: Int) extends AE
class UnOp(val op: String, val e: AE) extends AE
class BinOp(val op: String, val e0: AE, val e1: AE) extends AE
\end{verbatim}

Defining \verb!AE! as a \term{trait} instead of a class prevents creating an
instance of \verb!AE! without the constructors of the child classes. Traits
define new types as classes but differ from classes in several aspects. One
difference is that traits forbid calling their constructors.

\begin{verbatim}
new AE
// trait AE is abstract; cannot be instantiated
\end{verbatim}

Calling the constructor of \verb!AE! results in a compile error because it is a
trait.

\section{Treating Algebraic Data Types in Scala}

The section defines function \verb!interpret!, which calculates an integer
denoted by a given arithmetic expression, with different ways. For simplicity,
assume that \verb!-! is the only possible unary operator, and \verb!+! is the
only possible binary operator. Any other operators result in exceptions.

\subsection{Dynamic Type Checking}

The first approach is checking the type of a given arithmetic expression
dynamically, or at run time.

\begin{verbatim}
def interpret(e: AE): Int = {
  if (...) // e is a number
    e.n
  else if (...) // e has unary operator -
    -interpret(e.e)
  else if (...) // e has binary operator +
    interpret(e.e0) + interpret(e.e1)
  else
    throw new Exception
}
\end{verbatim}

In Scala, \verb!isInstanceOf! checks whether an object is a value of a type.
Expression \verb![expression].isInstanceOf[[type]]! denotes \verb!true! if the
value of the expression belongs to the type and \verb!false! otherwise.

\begin{verbatim}
def interpret(e: AE): Int = {
  if (e.isInstanceOf[Num])
    e.n
  else if (e.isInstanceOf[UnOp] && e.op == "-")
    -interpret(e.e)
  else if (e.isInstanceOf[BinOp] && e.op == "+")
    interpret(e.e0) + interpret(e.e1)
  else
    throw new Exception
}
\end{verbatim}

The code results in a compile error. Compilers know only that a value referred by
\verb!e! is an element of type \verb!AE! about \verb!e!; fields of \verb!e! are
inaccessible. \verb!asInstanceOf!, which inform the compilers that an expression
has a particular type, resolves the problem by \term{casting} \term{types}
explicitly. To type \verb![expression].asInstanceOf[[type]]!, the compilers
ignore a type calculated by themselves and rely on a type given by programmers.

At run time, \verb!asInstanceOf! does not change the value of an expression but
checks the type of the value. Evaluating \verb![expression].asInstanceOf[[type]]!
results in a value denoted by the expression if the value belongs to the type but
throws an exception otherwise.

\begin{verbatim}
def interpret(e: AE): Int = {
  if (e.isInstanceOf[Num])
    e.asInstanceOf[Num].n
  else if (e.isInstanceOf[UnOp] &&
    e.asInstanceOf[UnOp].op == "-")
    -interpret(e.asInstanceOf[UnOp].e)
  else if (e.isInstanceOf[BinOp] &&
    e.asInstanceOf[BinOp].op == "+")
    interpret(e.asInstanceOf[BinOp].e0) +
      interpret(e.asInstanceOf[BinOp].e1)
  else
    throw new Exception
}

// -1 + 2
interpret(new BinOp(
  "+",
  new UnOp("-", new Num(1)),
  new Num(2)
))
// 1
\end{verbatim}

The code is long and complicated despite its simple functionality. Dynamic type
checking and explicit type casting occupy most of the code, while real
computation requires short code. Besides, such code is error-prone. Programmers
give information whom compilers cannot find due to their imprecision---does not
mean that the compilers are wrong but signifies that they fail to find all the
information---with \verb!asInstanceOf!. However, they may be incorrect. Incorrect
information might terminate programs abnormally because of type errors at run
time. It is easy to check whether the code is correct as it is short. In
contrast, complex types or computation increase the possibilities of errors.

\subsection{Method Overloading}

Method \term{overloading} allows defining methods of the same name with different
types of parameters. Many object-oriented languages including Scala feature
method overloading.

\begin{verbatim}
object Show {
  def show(i: Int): String = i + ": Int"
  def show(s: String): String = s + ": String"
}
Show.show(1)  // "1: Int"
Show.show("1")  // "1: String"
\end{verbatim}

Singleton object \verb!Show! has the two \verb!show! methods. Their parameter
types respectively are \verb!Int! and \verb!String!, which differ, and the
compilation thus succeeds. Each invocation of the \verb!show! method chooses a
method proper to the type of the argument.

Is method overloading a correct solution for the \verb!interpret! method?

\begin{verbatim}
object Interpreter {
  def interpret(e: AE): Int =
    throw new Exception("What is it?")
  def interpret(e: Num): Int = e.n
  def interpret(e: UnOp): Int =
    if (e.op == "-") -interpret(e.e)
    else throw new Exception("Only -")
  def interpret(e: BinOp): Int =
    if (e.op == "+")
      interpret(e.e0) + interpret(e.e1)
    else throw new Exception("Only +")
}

Interpreter.interpret(new Num(1))  // 1
\end{verbatim}

It seems to work but, alas, does not.

\begin{verbatim}
// -1 + 2
Interpreter.interpret(new BinOp(
  "+",
  new UnOp("-", new Num(1)),
  new Num(2)
))
// java.lang.Exception: What is it?
\end{verbatim}

Surprisingly, the first method is invoked because the compile-time types rather
than the run-time types of arguments between angle brackets affect dynamic method
dispatch. At compile time, \verb!e.e0! in expression \verb!interpret(e.e0)! has
type \verb!AE!, and the first method is therefore called, while its type is
\verb!UnOp! at run time. As a consequence, method overloading is not a correct
strategy to implement the \verb!interpret! method.

\subsection{The Visitor Pattern}

Object-oriented programmers usually use the \term{visitor} \term{pattern} to
resolve such problems. The pattern is appropriate for the \verb!interpret! method
as well but makes code long and complex and needs some boilerplate code. Despite
the disadvantages, it is the most effective solution for languages without
pattern matching like Java; it is popular. The article does not discuss the
visitor pattern in detail and refers to an article of English
Wikipedia\sidenote{\url{https://en.wikipedia.org/wiki/Visitor\_pattern}} instead.

\subsection{Pattern Matching}

Pattern matching is the best solution for such problems and a typical feature of
functional languages. The following code defines the \verb!interpret! function in
OCaml with pattern matching:

\begin{verbatim}
let rec interpret e =
  match e with
  | Num n -> n
  | UnOp ("-", e0) -> -(interpret e0)
  | BinOp ("+", e0, e1) -> (interpret e0) + (interpret e1)
  | _ -> raise Exception
\end{verbatim}

The code is intuitive and concise even for those who are unknowledgeable about
OCaml.

Scala features pattern matching as well. Pattern matching requires types defined
by \term{case} \term{classes} instead of regular classes. Actually, every type is
available for pattern matching, but the article does not introduce how to do it.
The following code defines arithmetic expressions with case classes:

\begin{verbatim}
trait AE
case class Num(n: Int) extends AE
case class UnOp(op: String, e: AE) extends AE
case class BinOp(op: String, e0: AE, e1: AE) extends AE
\end{verbatim}

Compilers do not treat case classes as special cases but desugar them to simple
classes. Case classes have a few features whom ordinary classes do not have:
pattern matching is possible without additional code; constructing instances does
not require the \verb!new! keyword; class parameters do not need the \verb!val!
keyword. More exist but are out of the scope of the article.

\begin{verbatim}
def interpret(e: AE): Int = e match {
  case Num(n) => n
  case UnOp(op, e0) =>
    if (op == "-") -interpret(e0)
    else throw new Exception
  case BinOp(op, e0, e1) =>
    if (op == "+") interpret(e0) + interpret(e1)
    else throw new Exception
}

// -1 + 2
interpret(BinOp(
  "+",
  UnOp("-", Num(1)),
  Num(2)
))
// 1
\end{verbatim}

The function uses pattern matching. It is more concise than the code using
dynamic type checking and safe due to the lack of explicit type casting.

\section{Advantages of Pattern Matching and Patterns in Scala}

\subsection{Exhaustivity Checking}

Pattern matching checks the exhaustivity of patterns. At run time, a match error
occurs when a given value matches none of patterns.

\begin{verbatim}
def interpret(e: AE): Int = e match {
  case UnOp(op, e0) =>
    if (op == "-") -interpret(e0)
    else throw new Exception
  case BinOp(op, e0, e1) =>
    if (op == "+") interpret(e0) + interpret(e1)
    else throw new Exception
}
\end{verbatim}

The function lacks the \verb!Num! pattern.

\begin{verbatim}
interpret(Num(3))
// scala.MatchError: Num(3) (of class Num)
\end{verbatim}

An argument of type \verb!Num! results in a match error.

Scala compilers check whether patterns are exhaustive and warn if they are not.
However, the compilers do not warn about the function, for they cannot know all
the child classes of the \verb!AE! trait. Every file can define classes extending
\verb!AE!. The unit of compilation is a single file, but it is impossible to find
all the child classes by scanning a single file. The \verb!sealed! keyword
resolves the problem. A \term{sealed} class or trait forbids defining its
children outside a file defining it. As a result, the compilers identify all the
child classes by reading one file and check the exhaustivity of patterns
successfully.

\begin{verbatim}
sealed trait AE
case class Num(n: Int) extends AE
case class UnOp(op: String, e: AE) extends AE
case class BinOp(op: String, e0: AE, e1: AE) extends AE

def interpret(e: AE): Int = e match {
  case UnOp(op, e0) =>
    if (op == "-") -interpret(e0)
    else throw new Exception
  case BinOp(op, e0, e1) =>
    if (op == "+") interpret(e0) + interpret(e1)
    else throw new Exception
}
// warning: match may not be exhaustive.
// It would fail on the following input: Num(_)
//       def interpret(e: AE): Int = e match {
//                                   ^
\end{verbatim}

The compilers warn about inexhaustive patterns as \verb!AE! is a sealed trait.
Exhaustivity checking is beneficial for complex programs. It helps to make safe
programs and thus is a crucial strength of pattern matching.

\subsection{Constant and Wildcard Patterns}

\verb!switch-case! statements divide a given value into multiple cases in
imperative languages. Pattern matching is a general form of \verb!switch-case!.
The following code is an example using a \verb!switch-case! statement in Java:

\begin{verbatim}
String grade(int score) {
  switch (score / 10) {
    case 10: return "A";
    case 9: return "A";
    case 8: return "B";
    case 7: return "C";
    case 6: return "D";
    default: return "F";
  }
}
\end{verbatim}

Constant and wildcard patterns exist in Scala. Constant patterns are
\term{literals} like integers and strings. A constant pattern matches a given
value if a value denoted by the pattern equals the given value. The underscore
denotes the wildcard pattern, which matches every value, and is equivalent to
\verb!default! of \verb!switch-case!. The following function rewrites the
previous function with pattern matching:

\begin{verbatim}
def grade(score: Int): String =
  (score / 10) match {
    case 10 => "A"
    case 9 => "A"
    case 8 => "B"
    case 7 => "C"
    case 6 => "D"
    case _ => "F"
  }

grade(85)  // "B"
\end{verbatim}

\subsection{Or Patterns}

\verb!switch-case! statements use the fall-through semantics; if \verb!break!
does not exist, after executing code corresponding to a case, the flow of the
execution moves to code corresponding to the next case. Since the results of
cases \verb!10! and \verb!9! are identical, the function can use fall-through.

\begin{verbatim}
String grade(int score) {
  switch (score / 10) {
    case 10:
    case 9: return "A";
    case 8: return "B";
    case 7: return "C";
    case 6: return "D";
    default: return "F";
  }
}
\end{verbatim}

In contrast, pattern matching disallows fall-through. Instead, \term{or} patterns
give a way to write the same expression only once for multiple patterns. Or
patterns are in the form of \verb![pattern] | [pattern] …!, which is a sequence
of multiple patterns with vertical bars in between. \verb!A | B! matches values
that match \verb!A! or \verb!B!.

\begin{verbatim}
def grade(score: Int): String =
  (score / 10) match {
    case 10 | 9 => "A"
    case 8 => "B"
    case 7 => "C"
    case 6 => "D"
    case _ => "F"
  }

grade(100)  // "A"
\end{verbatim}

\subsection{Nested Patterns}

Nested patterns are patterns containing patterns. Nested patterns simplify the
\verb!interpret! function.

\begin{verbatim}
def interpret(e: AE): Int = e match {
  case Num(n) => n
  case UnOp("-", e0) => -interpret(e0)
  case BinOp("+", e0, e1) =>
    interpret(e0) + interpret(e1)
  case _ => throw new Exception
}
\end{verbatim}

As the \verb!UnOp! pattern contains constant pattern \verb!"-"!, the second
pattern matches an arithmetic expression only with unary operator \verb!"-"!.
Similarly, the third pattern matches an arithmetic expression only with binary
operator \verb!"+"!.

It is possible to nest multiple patterns. The \verb!optimizeAdd! function
optimizes a given arithmetic expression by eliminating additions of zeros.

\begin{verbatim}
def optimizeAdd(e: AE): AE = e match {
  case Num(_) => e
  case UnOp(op, e0) => UnOp(op, optimizeAdd(e0))
  case BinOp("+", Num(0), e1) => optimizeAdd(e1)
  case BinOp("+", e0, Num(0)) => optimizeAdd(e0)
  case BinOp(op, e0, e1) =>
    BinOp(op, optimizeAdd(e0), optimizeAdd(e1))
}

// 0 + 1 + 2
optimizeAdd(BinOp(
  "+",
  Num(0),
  BinOp("+", Num(1), Num(2))
))
// 1 + 2
\end{verbatim}

Assume that \verb!"abs"!, a unary operator, is available. It denotes the absolute
value of an operand. Optimizing an arithmetic expression decorated by two
consecutive \verb!"abs"! operators results in the arithmetic expression with only
one \verb!"abs"! operator.

\begin{verbatim}
def optimizeAbs(e: AE): AE = e match {
  case Num(_) => e
  case UnOp("abs", UnOp("abs", e0)) =>
    optimizeAbs(UnOp("abs", e0))
  case UnOp(op, e0) => UnOp(op, optimizeAbs(e0))
  case BinOp(op, e0, e1) =>
    BinOp(op, optimizeAbs(e0), optimizeAbs(e1))
}

// | | -1 | |
optimizeAbs(UnOp("abs",
  UnOp("abs",
    UnOp("-", Num(1))
  )
))
// | - 1 |
\end{verbatim}

A flaw of the implementation is that a value matching \verb!UnOp("abs", e0)!
cannot be an argument of \verb!optimizeAbs! directly, and constructing a new
\verb!UnOp! instance containing a value matching \verb!e0! is essential. The at
sign makes code efficient by binding a value matching to a pattern to a variable.
Pattern \verb![variable] @ [pattern]! makes the variable refer to a value
matching the pattern.

\begin{verbatim}
def optimizeAbs(e: AE): AE = e match {
  case Num(_) => e
  case UnOp("abs", e0 @ UnOp("abs", _)) =>
    optimizeAbs(e0)
  case UnOp(op, e0) => UnOp(op, optimizeAbs(e0))
  case BinOp(op, e0, e1) =>
    BinOp(op, optimizeAbs(e0), optimizeAbs(e1))
}
\end{verbatim}

Nested patterns and the at sign help to treat values of complex structures
easily.

\subsection{Unreachable Patterns}

A value is likely to match more than one pattern when a pattern matching
expression contains multiple nested patterns. Like \verb!switch-case!, pattern
matching compares a value to patterns sequentially from top to bottom and selects
the first matching pattern. If both a pattern handling a specific case and a
pattern handling a general case exist, the former must occur earlier than the
latter. However, if the number of patterns is large, the patterns might be in the
wrong order so that a value matches an unintended pattern. Scala compilers warn
when they find \term{unreachable} patterns to prevent such code.

\begin{verbatim}
def optimizeAbs(e: AE): AE = e match {
  case Num(_) => e
  case UnOp(op, e0) => UnOp(op, optimizeAbs(e0))
  case UnOp("abs", e0 @ UnOp("abs", _)) =>
    optimizeAbs(e0)
  case BinOp(op, e0, e1) =>
    BinOp(op, optimizeAbs(e0), optimizeAbs(e1))
}
// warning: unreachable code
//         case UnOp("abs", e0 @ UnOp("abs", _)) => optimizeAbs(e0)
//                                                             ^
\end{verbatim}

Checking the unreachability of patterns also is a significant advantage of
pattern matching. It avoids programs behaving incorrectly because of patterns in
the wrong order. However, not every wrong pattern results in an unreachable
pattern. In some code, values match unintended patterns, but every pattern is
reachable. The compilers do not warn about such code, and programmers thus need
to be careful.

\subsection{Type Patterns}

The \verb!optimizeNeg! function optimizes arithmetic expression by removing two
consecutive \verb!"-"! unary operators.

\begin{verbatim}
def optimizeNeg(e: AE): AE = e match {
  case Num(_) => e
  case UnOp("-", UnOp("-", e0)) => e0
  case UnOp(op, e0) => UnOp(op, optimizeNeg(e0))
  case BinOp(op, e0, e1) =>
    BinOp(op, optimizeNeg(e0), optimizeNeg(e1))
}

// -(-(1 + 1))
optimizeNeg(UnOp("-",
  UnOp("-",
    BinOp("+", Num(1), Num(1))
  )
))
// 1 + 1
\end{verbatim}

The first \verb!Num(_)! pattern does no more than checking whether a value
belongs to type \verb!Num!. A type pattern helps to rewrite the function. Type
patterns are in the form of \verb![variable]: [type]!. If a value belongs to the
type, it matches the pattern, and the variable refers to the value. The wildcard
pattern can substitute the variable if the variable is unnecessary.

\begin{verbatim}
def optimizeNeg(e: AE): AE = e match {
  case _: Num => e
  case UnOp("-", UnOp("-", e0)) => e0
  case UnOp(op, e0) => UnOp(op, optimizeNeg(e0))
  case BinOp(op, e0, e1) =>
    BinOp(op, optimizeNeg(e0), optimizeNeg(e1))
}
\end{verbatim}

Type patterns are useful for dynamic type checking.

\begin{verbatim}
def show(x: Any): String = x match {
  case i: Int => i + ": Int"
  case s: String => s + ": String"
  case _ => x + ": Any"
}

show(1)  // "1: Int"
show("1")  // "1: String"
show(1.0)  // "1.0: Any"
\end{verbatim}

\verb!Any! is the \term{top} type of the Scala type system; every value belongs
to \verb!Any!.

Note that type patterns cannot check type arguments of polymorphic types. Using
pattern matching against polymorphic types is dangerous.

\begin{verbatim}
def show(x: Any): String = x match {
  case l: List[Int] => l + ": List[Int]"
  case l: List[String] => l + ": List[String]"
  case _ => x + ": Any"
}
// warning: non-variable type argument Int
// in type pattern List[Int] is unchecked
// since it is eliminated by erasure
//         case l: List[Int] => l + ": List[Int]"
//                 ^
// warning: non-variable type argument String
// in type pattern List[String] is unchecked
// since it is eliminated by erasure
//         case l: List[String] => l + ": List[String]"
//                 ^
// warning: unreachable code
//         case l: List[String] => l + ": List[String]"
//                                   ^

show("one" :: Nil)  // "List(one): List[Int]"
\end{verbatim}

Although the type of the argument is \verb!List[String]!, it matches the first
pattern. As the warnings imply, the JVM uses the \term{type} \term{erasure}
semantics, and type arguments are therefore not available at run time. A later
article will deal with polymorphic types and type erasures in detail.

\subsection{Tuple Patterns}

The last article has already used tuple patterns. They are in the form of
\verb!([pattern], … )!. The \verb!equals! function uses tuple patterns and checks
whether two lists are identical.

\begin{verbatim}
def equal(l0: List[Int], l1: List[Int]): Boolean =
  (l0, l1) match {
    case (h0 :: t0, h1 :: t1) =>
      h0 == h1 && equal(t0, t1)
    case (Nil, Nil) => true
    case _ => false
  }

equal(List(0, 1), List(0, 1))  // true
equal(List(0, 1), List(0))  // false
\end{verbatim}

\subsection{Pattern Guards}

A binary (search) tree is

* the empty tree or
* a tree containing an integral root element and two child trees.

\begin{verbatim}
sealed trait BST
case object Empty extends BST
case class Node(root: Int, left: BST, right: BST) extends BST
\end{verbatim}

Function \verb!add! takes a tree and an integer as arguments and returns a tree
obtained by adding the integer to the tree. If the integer is an element of the
given tree, the tree itself is the return value.

\begin{verbatim}
def add(t: BST, n: Int): BST =
  t match {
    case Empty => Node(n, Empty, Empty)
    case Node(m, t0, t1) =>
      if (n < m) Node(m, add(t0, n), t1)
      else if (n > m) Node(m, t0, add(t1, n))
      else t
  }
\end{verbatim}

An expression corresponding to the second pattern uses \verb!if-else!. Pattern
guards allow adding constraints to patterns. A pattern in the form of
\verb![pattern] if [expression]! matches a value if the value matches the
pattern, and the expression results in \verb!true!. The following \verb!add!
function uses \term{pattern} \term{guards}:

\begin{verbatim}
def add(t: BST, n: Int): BST =
  t match {
    case Empty => Node(n, Empty, Empty)
    case Node(m, t0, t1) if n < m =>
      Node(m, add(t0, n), t1)
    case Node(m, t0, t1) if n > m =>
      Node(m, t0, add(t1, n))
    case _ => t
  }
\end{verbatim}

Guarded patterns may be inexhaustive.

\begin{verbatim}
def add(t: BST, n: Int): BST =
  t match {
    case Empty => Node(n, Empty, Empty)
    case Node(m, t0, t1) if n < m =>
      Node(m, add(t0, n), t1)
    case Node(m, t0, t1) if n > m =>
      Node(m, t0, add(t1, n))
  }
\end{verbatim}

Compilers do not warn about the inexhaustivity of the code. However, since a
pattern handling cases that a given integer is an element of a given tree is
missing, a match error occurs in the cases.

\subsection{Patterns with Backticks}

Function \verb!remove! takes a tree and an integer as arguments and returns a
tree obtained by removing the integer from the tree. If the integer is not an
element of the tree, the given tree itself is the return value. \verb!removeMin!
is a helper function used by \verb!remove!. It returns the pair of the smallest
element of a given tree and a tree obtained by removing the element from the
tree.

\begin{verbatim}
def removeMin(t: Node): (Int, BST) = {
  t match {
    case Node(n, Empty, t1) =>
      (n, t1)
    case Node(n, t0: Node, t1) =>
      val (min, t2) = removeMin(t0)
      (min, Node(n, t2, t1))
  }
}

def remove(t: BST, n: Int): BST = {
  t match {
    case Empty =>
      Empty
    case Node(m, t0, Empty) if n == m =>
      t0
    case Node(m, t0, t1: Node) if n == m =>
      val (min, t2) = removeMin(t1)
      Node(min, t0, t2)
    case Node(m, t0, t1) if n < m =>
      Node(m, remove(t0, n), t1)
    case Node(m, t0, t1) if n > m =>
      Node(m, t0, remove(t1, n))
  }
}
\end{verbatim}

\begin{verbatim} Node(\verb!n!, t0, Empty)\end{verbatim} can replace
\verb!case None(m, t0, Empty) if n == m!. \verb!case Node(n, t0, Empty)! defines new
\verb!n! other than parameter \verb!n! and makes new \verb!n! to refer to the
root element; it does not check whether the root element equals \verb!n!.
However, backticks prohibit to define new \verb!n! and allow to compare the root
element to \verb!n! in the scope.

\begin{verbatim}
def remove(t: BinTree, n: Int): BinTree = {
  t match {
    case Empty =>
      Empty
    case Node(\verb!n!, t0, Empty) =>
      t0
    case Node(\verb!n!, t0, t1: Node) =>
      val (min, t2) = removeMin(t1)
      Node(min, t0, t2)
    case Node(m, t0, t1) if n < m =>
      Node(m, remove(t0, n), t1)
    case Node(m, t0, t1) if n > m =>
      Node(m, t0, remove(t1, n))
  }
}
\end{verbatim}

\section{Applications of Pattern Matching}

\subsection{Variable Declarations}

It is possible to declare variables with pattern matching.

\begin{verbatim}
val (n, m) = (1, 2)
// n = 1, m = 2
val (a, b, c) = ("a", "b", "c")
// a = "a", b = "b", c = "c"
val h :: t = List(1, 2, 3, 4)
// h = 1, t = List(2, 3, 4)
val BinOp(op, e0, e1) = BinOp("+", Num(1), Num(2))
// op = "+", e0 = Num(1), e1 = Num(2)
\end{verbatim}

Pattern matching helps to declare variables concisely, but a match error occurs
when a pattern does not match a right-hand-side value. It is desirable to use
pattern matching only when a guarantee that a pattern must match exists. Since
tuple patterns always match tuple values, tuple patterns are typical for variable
declarations.

\subsection{Anonymous Functions}

Function \verb!toSum! takes a list of pairs of two integers as arguments and
returns a list whose elements are the sums of the integers in the pairs.

\begin{verbatim}
def toSum(l: List[(Int, Int)]): List[Int] =
  l.map(p => p match {
    case (n, m) => n + m
  })

toSum(List((0, 1), (2, 3), (3, 4)))
// List(1, 5, 7)
\end{verbatim}

An anonymous function using parameter \verb!p! as a value matched is verbose.
Scala compilers consider an expression in the form of \verb!{ case [pattern] => [expression] … }!
as an anonymous function if the expression occurs where they
expect a function. The following function is concise:

\begin{verbatim}
def toSum(l: List[(Int, Int)]): List[Int] =
  l.map({ case (n, m) => n + m })
\end{verbatim}

An argument that is an anonymous function defined by pattern matching does not
require surrounding round brackets. It is a special rule.

\begin{verbatim}
def toSum(l: List[(Int, Int)]): List[Int] =
  l.map { case (n, m) => n + m }
\end{verbatim}

The period also is unnecessary. It is not special; Scala allows using methods as
infix operators; for example, \verb!Nil.::(0)! and \verb!0 :: Nil! are identical.

\begin{verbatim}
def toSum(l: List[(Int, Int)]): List[Int] =
  l map { case (n, m) => n + m }
\end{verbatim}

\subsection{for}

\verb!toSum! can use a \verb!for! expression instead of \verb!map!.

\begin{verbatim}
def toSum(l: List[(Int, Int)]): List[Int] =
  for (p <- l) yield p match {
    case (n, m) => n + m
  }
\end{verbatim}

\verb!for! expressions can use pattern matching.

\begin{verbatim}
def toSum(l: List[(Int, Int)]): List[Int] =
  for ((n, m) <- l) yield n + m
\end{verbatim}

The code is readable and concise.
