\setchapterpreamble[u]{\margintoc}
\chapter{Recursion}
\labch{recursion}

\renewcommand{\plang}{\textsf{FAE}\xspace}
\renewcommand{\lang}{\textsf{RFAE}\xspace}

\newcommand{\sumbody}{\eifz{\cx}{0}{(\eadd{\cx}{\eapp{\code{sum}}{(\esub{\cx}{1})}})}}
\newcommand{\sumbodyf}{\eifz{\cx}{0}{(\eadd{\cx}{\eapp{\code{f}}{(\esub{\cx}{1})}})}}
\newcommand{\sumbodyfv}{\eifz{\cv}{0}{(\eadd{\cv}{\eapp{\code{f}}{(\esub{\cv}{1})}})}}

Recursive functions are widely used in programming. We have discussed importance of
recursion in \refsec{recursion}. This chapter explains the semantics of
recursive functions by defining \lang, which extends \plang with recursive
functions. In addition, we will see that recursive functions also are just syntactic
sugar: we can express recursive functions with first-class functions.

Consider the following Scala program:

\begin{verbatim}
def sum(x: Int): Int =
  if (x == 0)
    0
  else
    x + sum(x – 1)

println(sum(10))
\end{verbatim}

The function \code{sum} takes an integer \code{n} as an argument, and returns
the sum of the integers between \code{0} to \code{n} (including \code{n}).
\sidenote{We ignore the case when the input is negative.}
Therefore, the program prints \code{55}.

How can we implement an equivalent program in \plang? One naïve approach could
be the following expression:\sidenote{\plang does not provide conditional
expressions (\textsf{if0}), which is defined by \lang, but we use it since
we can easily add them to \plang.}

\[\ebind{
  \code{sum}
}{
  \efun{\cx}{\eifz{\cx}{0}{(\eadd{\cx}{\eapp{\code{sum}}{(\esub{\cx}{1})}})}}
}{
  \eapp{\code{sum}}{10}
}\]

However, it is wrong since the scope of \code{sum} includes $\eapp{\code{sum}}{10}$ but excludes
$\efun{\cx}{\eifz{\cx}{0}{(\eadd{\cx}{\eapp{\code{sum}}{(\esub{\cx}{1})}})}}$.
\code{sum} in the body of the function is a free identifier. Evaluation of the
expression termiantes with a run-time error. It is nontrival to define recursive
functions in \plang.

\section{Syntax}

We define \lang by extending \plang with recursive functions. To demonstrate the
usefulness of recursion with examples, we add conditional expressions as well.

The following is the abstract syntax of \lang:
\sidenote{We omit the common part to \plang.}

\[
  e\ ::=\ \cdots\ |\ \eifz{e}{e}{e}\ |\ \erec{x}{x}{e}{e}
\]

\begin{itemize}
  \item $\eifz{e_1}{e_2}{e_3}$

    It is a conditional expression. $e_1$ is the condition; $e_2$ is the true
    branch; $e_3$ is the false branch. We consider the condition to be true when
    it denotes $0$. All the other values, i.e. nonzero integers and closures,
    are considered as false.

  \item $\erec{x_1}{x_2}{e_1}{e_2}$

    It defines a recursive function whose name is $x_1$, parameter is $x_2$, and
    body is $e_1$. Both $x_1$ and $x_2$ are binding occurrences. The scope of
    $x_1$ is $e_1$ and $e_2$; the scope of $x_2$ is $e_1$. If $x_1$ occurs in
    $e_1$, it is a bound occurrence, which implies that the function can be
    recursive.
\end{itemize}

In \lang, we can implement a function computing the sum of consecutive integers
like below.

\[\erec{\code{sum}}{\cx}{\sumbody}{\eapp{\code{sum}}{10}}\]

\section{Semantics}

The semantics of conditional expressions is quite easy. The semantics consists
of two rules: one for when the condition is true and the other for when the
condition is false.

The following rule defines the semantics when the condition is true:

\semanticrule{If0-Z}{
If
  \evaldn{e_1}{0} and
  \evaldn{e_2}{v},\\
then
  \evaldn{\eifz{e_1}{e_2}{e_3}}{v}.
}

\vspace{-1em}

\[
  \inferrule
  {
    \evald{e_1}{0} \\
    \evald{e_2}{v}
  }
  { \evald{\eifz{e_1}{e_2}{e_3}}{v} }
  \quad\textsc{[If0-Z]}
\]

When $e_1$ evaluates to $0$, the condition is considered as true, and the true
branch, $e_2$, is evaluated. The result of $e_2$ is the result of the whole
expression.

The following rule defines the semantics when the condition is false:

\semanticrule{If0-Nz}{
If
  \evaldn{e_1}{v'}, where $v'\not=0$, and
  \evaldn{e_3}{v},\\
then
  \evaldn{\eifz{e_1}{e_2}{e_3}}{v}.
}

\vspace{-1em}

\[
\inferrule
  {
    \evald{e_1}{v'} \\
    v'\not=0 \\
    \evald{e_3}{v}
  }
  { \evald{\eifz{e_1}{e_2}{e_3}}{v} }
  \quad\textsc{[If0-Nz]}
\]

When $e_1$ evaluates to a value other than $0$, the condition is considered as
false, and the false
branch, $e_3$, is evaluated. The result of $e_3$ is the result of the whole
expression.

Now, let us discuss the semantics of recursive functions.
Consider $\erec{x_1}{x_2}{e_1}{e_2}$.
$e_2$ is in the scope of $x_1$, and $x_1$ denotes a function. Therefore, we can
start by defining $\sigma'=[x_1\mapsto\clov{x_2}{e_1}{\sigma}]$ and evaluating
$e_2$ under $\sigma'$. However, this apporach is incorrect. When the body of the
closure, $e_1$, is evaluated, the environment is $\sigma[x_2\mapsto v]$ for some
$v$. The environment does not contain $x_1$, and therefore using $x_1$ in
$e_1$ will incur a free identifier error, which is certainly wrong since $x_1$
is a recursive function. The environment of the closure must include $x_1$ and
its value, which is the closure itself. From this observation, we obtain the
following rule:

\semanticrule{Rec}{
If
  \evaln{\sigma'}{e_2}{v}, where
  $\sigma'=\sigma[x_1\mapsto\clov{x_2}{e_1}{\sigma'}]$, \\
then
  \evaldn{\erec{x_1}{x_2}{e_1}{e_2}}{v}.
}

\vspace{-1em}

\[
  \inferrule
  {
    \sigma'=\sigma[x_1\mapsto\clov{x_2}{e_1}{\sigma'}] \\
    \eval{\sigma'}{e_2}{v}
  }
  { \evald{\erec{x_1}{x_2}{e_1}{e_2}}{v} }
  \quad\textsc{[Rec]}
\]

The environment of the closure is $\sigma'$, which has $x_1$ and the closure.
$\sigma'$ is recursively defined at the meta-level. It is not that surprising.
We are defining a recursive function, so the defined function value itself
should be recursive. When the body of the
closure, $e_1$, is evaluated, the environment is $\sigma'[x_2\mapsto v]$ for some
$v$, which contains $x_1$. The function $x_1$ can be used in its body and thus
recursive.

We can reuse the rules of \plang for the other expressions.

The following proof trees prove that
$\erec{\code{f}}{\cx}{\sumbodyf}{\eapp{\code{f}}{1}}$
evaluates to $1$ under the empty environment.
The proof splits into three trees for readability.
Suppose the following facts:
\[
\begin{array}{rcl}
  e_{\code{f}}  &=& \sumbodyf \\
  v_{\code{f}}  &=& \clov{\cx}{e_{\code{f}}}{\sigma_1} \\
  \sigma_1 &=& [\code{f}\mapsto v_{\code{f}}] \\
  \sigma_2 &=& \sigma_1[\cx\mapsto1] \\
  \sigma_3 &=& \sigma_1[\cx\mapsto0] \\
\end{array}
\]

\[
  \footnotesize
\inferrule
{
  \inferrule
  { \code{f}\in\dom{\sigma_2} }
  { \eval{\sigma_2}{\code{f}}{v_{\code{f}}} }
  \quad
  \inferrule
  {
    \inferrule
    { \cx\in\dom{\sigma_2} }
    { \eval{\sigma_2}{\cx}{1} }
    \quad
    \eval{\sigma_2}{1}{1}
  }
  { \eval{\sigma_2}{\esub{\cx}{1}}{0} }
  \quad
  \inferrule
  {
    \inferrule
    { \cx\in\dom{\sigma_3} }
    { \eval{\sigma_3}{\cx}{0} }
    \quad
    \eval{\sigma_3}{0}{0}
  }
  { \eval{\sigma_3}{e_{\code{f}}}{0} }
}
{ \eval{\sigma_2}{\eapp{\code{f}}{(\esub{\cx}{1})}}{0} }
\]

\[
  \footnotesize
\inferrule
{
  \inferrule
  { \code{f}\in\dom{\sigma_1} }
  { \eval{\sigma_1}{\code{f}}{v_{f}} }
  \quad
  { \eval{\sigma_1}{1}{1} }
  \quad
  \inferrule
  {
    \inferrule
    { \cx\in\dom{\sigma_2} }
    { \eval{\sigma_2}{\cx}{1} }
    \quad
    1\not=0
    \quad
    \inferrule
    {
      \inferrule
      { \cx\in\dom{\sigma_2} }
      { \eval{\sigma_2}{\cx}{1} }
      \quad
      \eval{\sigma_2}{\eapp{\code{f}}{(\esub{\cx}{1})}}{0}
    }
    { \eval{\sigma_2}{\eadd{\cx}{\eapp{\code{f}}{(\esub{\cx}{1})}}}{1} }
  }
  { \eval{\sigma_2}{e_{\code{f}}}{1} }
}
{ \eval{\sigma_1}{\eapp{\code{f}}{1}}{1} }
\]

\[
\inferrule
{
  \sigma_1=[\code{f}\mapsto v_{\code{f}}]
  \\
  \eval{\sigma_1}{\eapp{\code{f}}{1}}{1}
}
{ \evale{\erec{\code{f}}{\cx}{e_{\code{f}}}{\eapp{\code{f}}{1}}}{1} }
\]

\section{Interpreter}

The following Scala code implements the syntax of \lang:
\sidenote{We omit the common part to \plang.}

\begin{verbatim}
sealed trait Expr
...
case class If0(c: Expr, t: Expr, f: Expr) extends Expr
case class Rec(f: String, x: String, b: Expr, e: Expr) extends Expr
\end{verbatim}

\code{If0($e_1$, $e_2$, $e_3$)} represents $\eifz{e_1}{e_2}{e_3}$;
\code{Rec($x_1$, $x_2$, $e_1$, $e_2$)} represents $\erec{x_1}{x_2}{e_1}{e_2}$.

\begin{verbatim}
sealed trait Value
case class NumV(n: Int) extends Value
case class CloV(p: String, b: Expr, var e: Env) extends Value
\end{verbatim}

Values are defined in a similar way to \plang. The only difference is that the
field \code{e}, which denotes the captured environment, of \code{CloV} is now mutable.
Using mutation is the easiest way to make recursive values in Scala, though we
can do it without mutation.

\begin{verbatim}
def interp(e: Expr, env: Env): Value = e match {
  ...
  case If0(c, t, f) =>
    interp(if (interp(c, env) == NumV(0)) t else f, env)
  case Rec(f, x, b, e) =>
    val cloV = CloV(x, b, env)
    val nenv = env + (f -> cloV)
    cloV.e = nenv
    interp(e, nenv)
}
\end{verbatim}

In the \code{If0} case, the condition is evaluated first. According to the
condition, one of the branches is evaluated.

In the \code{Rec} case, we
construct a closure first. However, the closure is not complete at this point.
We next create a new environment: the environment with the closure. The closure
must capture the new environment. To achieve this, we change the environment of
the closure to the new environment. Now, both closure and environment have
recursive structures, and \code{e} can be evaluated under the environment.

\section{Recursion as Syntactic Sugar}

Even if a language does not support recursive functions, we can implement
recursive functions with first-class functions, i.e. we can implement recursive
functions in \plang. The key to desugar recursive functions into \plang is the
following function:

\[
  Z=\efun{\cf}{\eapp
    {(\efun{\cx}{\eapp{\cf}{(\efun{\cv}{\eapp{\eapp{\cx}{\cx}}{\cv}})}})}
    {(\efun{\cx}{\eapp{\cf}{(\efun{\cv}{\eapp{\eapp{\cx}{\cx}}{\cv}})}})}
  }
\]

$Z$ is called a \textit{fixed point combinator}\index{fixed point combinator}.
In mathematics, a \textit{fixed point}\index{fixed point}
of a function $f$ is a solution of the equation $f(x)=x$.
A fixed point combinator is a function that computes a fixed point of a given
function. A recursive function can be considered as a fixed point of a certain
function. For example, consider the following function:

\[ \efun{\cf}{\efun{\cv}{\sumbodyfv}} \]

Let us call the function $f$.
Suppose that $\embox{sum}$ is given to the function as an argument, where
$\embox{sum}$ is a function that takes a natural number $n$ as an argument
and returns $\sum_{i=0}^n i$. Then, the return value of $f$ is

\[
  \efun{\cv}{\eifz{\cv}{0}{(\eadd{\cv}{\eapp{\embox{sum}}{(\esub{\cv}{1})}})}}
\]

It is a function again. When $0$ is given to the function, the result is $0$.
When a positive integer $n$ is given to the function, the result is
$\eadd{n}{\eapp{\embox{sum}}{(n-1)}}$, which equals $\eapp{\embox{sum}}{n}$.
Therefore, $\eapp{f}{\embox{sum}}$ equals $\embox{sum}$, and we can say that
$\embox{sum}$ is a fixed point of $f$.

The fixed point combinator $Z$ takes a function as an argument and returns a
fixed point of the function. Therefore, $\eapp{Z}{f}$ equals $\embox{sum}$. Let
us see the reason. We use $e_f$ as an abbreviation of $\sumbodyfv$.
Then, $\eapp{Z}{f}$ is the same as
\[ \eapp{Z}{(\efun{\code{f}}{\efun{\cv}{e_f}})} \]
It evaluates to $\eapp{F}{F}$ where $F$ denotes
\[
  \efun{\cx}{\eapp{(\efun{\code{f}}{\efun{\cv}{e_f}})}{(\efun{\cv}{\eapp{\eapp{\cx}{\cx}}{\cv}})}}
\]
By expanding only the first $F$ in $\eapp{F}{F}$, we get
\[
  \eapp{(\efun{\cx}{\eapp{(\efun{\code{f}}{\efun{\cv}{e_f}})}{(\efun{\cv}{\eapp{\eapp{\cx}{\cx}}{\cv}})}})}
  {F}
\]
which evaluates to
\[
  \eapp{(\efun{\code{f}}{\efun{\cv}{e_f}})}{(\efun{\cv}{\eapp{\eapp{F}{F}}{\cv}})}
\]
By expanding $e_f$, we get
\[
  \eapp{(\efun{\code{f}}{\efun{\cv}{\sumbodyfv}})}{(\efun{\cv}{\eapp{\eapp{F}{F}}{\cv}})}
\]
which evalutes to
\[
  \efun{\cv}{{\eifz{\cv}{0}{(\eadd{\cv}{\eapp{(\efun{\cv}{\eapp{\eapp{F}{F}}{\cv}})}{(\esub{\cv}{1})}})}}}
\]
Let us call this function $g$. Note that we now know that $\eapp{F}{F}$
evaluates to $g$. If we apply $g$ to
$0$, the result is $0$. If we apply $g$ to a positive integer $n$, the result is
\[
  \eadd{n}{\eapp{(\efun{\cv}{\eapp{\eapp{F}{F}}{\cv}})}{(\esub{n}{1})}}
\]
which evalutes to
\[
  \eadd{n}{\eapp{\eapp{F}{F}}{(\esub{n}{1})}}
\]
Since ${\eapp{F}{F}}$ evaluates to $g$, the above expression evaluates to
\[
  \eadd{n}{\eapp{g}{(\esub{n}{1})}}
\]
Then, the semantics of $g$ is
\[
  \eapp{g}{n}=\begin{cases}
    0 & \text{if } n=0\\
    \eadd{n}{\eapp{g}{(\esub{n}{1})}} & \text{otherwise}
  \end{cases}
\]
It implies that $g$ equals $\embox{sum}$. Actually, this evaluation has started from
$\eapp{Z}{f}$. Therefore, $\eapp{Z}{f}$ equals $\embox{sum}$.

One may ask if we can use the following expression $Z'$ instead of $Z$:
\[
  Z'=\efun{\cf}{\eapp
    {(\efun{\cx}{\eapp{\cf}{(\eapp{\cx}{\cx})}})}
    {(\efun{\cx}{\eapp{\cf}{(\eapp{\cx}{\cx})}})}
  }
\]
It is a natural question because $\eapp{\cx}{\cx}$ does the same thing as
$\efun{\cv}{\eapp{\eapp{\cx}{\cx}}{\cv}}$ when applied to an argument. However,
the answer is no. Eta-expanding $\eapp{\cx}{\cx}$ to $\efun{\cv}{\eapp{\eapp{\cx}{\cx}}{\cv}}$
has the effect of delaying computation on $\eapp{\cx}{\cx}$. While
$\eapp{\cx}{\cx}$ should be immediately evaluated, $\eapp{\cx}{\cx}$ in
$\efun{\cv}{\eapp{\eapp{\cx}{\cx}}{\cv}}$ can be evaluated only when the
function is applied to
an argument. This delaying effect is necessary in $Z$. Suppose that we use $Z'$.
Then, $\eapp{Z'}{f}$ evaluates to $\eapp{F'}{F'}$
where $F'$ denotes
\[
  \efun{\cx}{\eapp{(\efun{\code{f}}{\efun{\cv}{e_f}})}{(\eapp{\cx}{\cx})}}
\]
By expanding only the first $F'$, we get
\[
  \eapp{(\efun{\cx}{\eapp{(\efun{\code{f}}{\efun{\cv}{e_f}})}{(\eapp{\cx}{\cx})}})}
  {F'}
\]
which evaluates to
\[
  \eapp{(\efun{\code{f}}{\efun{\cv}{e_f}})}{(\eapp{F'}{F'})}
\]
To evaluate this expression, we need to evaluate the argument. However, the
argument is $\eapp{F'}{F'}$, which implies that we need the value of
$\eapp{F'}{F'}$ to compute the value of $\eapp{F'}{F'}$. Thus, at this point,
the execution starts to evaluate $\eapp{F'}{F'}$ forever and never terminates.
For this reason, we should use $Z$, not $Z'$, as a fixed point combinator.

Finally, we can define the syntactic transformation rule to desugar recursive
functions: $\erec{x_1}{x_2}{e_1}{e_2}$ is transformed into
$\ebind{x_1}{\eapp{Z}{(\efun{x_1}{\efun{x_2}{e_1}})}}{e_2}$.
If $x_1$ in $\erec{x_1}{x_2}{e_1}{e_2}$ denotes a function $h$, $h$ is a fixed
point of $\efun{x_1}{\efun{x_2}{e_1}}$. Therefore,
$\eapp{Z}{(\efun{x_1}{\efun{x_2}{e_1}})}$ is equal to $h$.
Both $\erec{x_1}{x_2}{e_1}{e_2}$ and $\ebind{x_1}{\eapp{Z}{(\efun{x_1}{\efun{x_2}{e_1}})}}{e_2}$
evaluate $e_2$ under the environment that $x_1$ denotes $h$. Therefore, they
have the same semantics, and the desugaring is correct.

\section{Exercises}

\begin{enumerate}
\item Explain why the following expression does not terminate and describe how
  to fix it.

$\begin{array}{l}
  \ebind{\cz}{(\efun{\cf}{\\
  \ \ \ \ \ebind{\cx}{(\efun{\cy}{\\
  \ \ \ \ \ \ \ \ \ebind{\code{g}}{\eapp{\cy}{\cy}}{\\
  \ \ \ \ \ \ \ \ \eapp{\cf}{\code{g}}}}\\
  \ \ \ \ )}{\\
  \ \ \ \ \eapp{\cx}{\cx}}}\\
  )}{\\
  \ebind{\cf}{\eapp{\cz}{(\efun{\cf}{\efun{\cv}{\sumbodyfv}})}}{\\
  \eapp{\cf}{10}
  }
}
\end{array}$

\item Consider the following definition of $\cz$ and its use to define the recursive
function $\cf$.

$\begin{array}{l}
  \ebind{\cz}{(\efun{\cf}{\\
  \ \ \ \ \ebind{\cx}{(\efun{\cy}{\\
  \ \ \ \ \ \ \ \ \ebind{\code{g}}{\efun{\code{a}}{\eapp{\eapp{\cy}{\cy}}{\code{a}}}}{\\
  \ \ \ \ \ \ \ \ \eapp{\cf}{\code{g}}}}\\
  \ \ \ \ )}{\\
  \ \ \ \ \eapp{\cx}{\cx}}}\\
  )}{\\
  \ebind{\cf}{\eapp{\cz}{(\efun{\cf}{\efun{\cv}{\sumbodyfv}})}}{\\
  \eapp{\cf}{10}
  }
}
\end{array}$

Describe conditions that an argument given to $\cz$ must satisfy so that $\cz$
    can make its recursive version.

\item Consider the following expression:

$\begin{array}{l}
  \ebind{\cf}{(\\
  \ \ \ \ \ebind{\cx}{\efun{\cy}{(\\
  \ \ \ \ \ \ \ \ \ebind{\cf}{\efun{\cv}{\eapp{\eapp{\cy}{\cy}}{\cv}}}{\\
  \ \ \ \ \ \ \ \ \efun{\cv}{\sumbodyfv}}\\
  \ \ \ \ )}}{\\
  \ \ \ \ \eapp{\cx}{\cx}\\
  )}}{\\
  \eapp{\cf}{10}
  }
\end{array}$

\begin{enumerate}
\item Draw arrows on the above expression from each bound variable to its binding occurrence.
\item Draw dotted arrows on the above expression from each shadowing variable to its shadowed variable.
\item Write the value of $\cf$ at the last line by using the following Scala types:
\begin{verbatim}
trait Value
case class NumV(n: Int) extends Value
case class CloV(p: String, b: Expr, e: Env) extends Value
type Env = Map[String, Value]
\end{verbatim}
\end{enumerate}

\item Consider the following language:
\[
\begin{array}{lrrl}
  \text{Expression} & e & ::= & n\ |\ \esub{e}{e}\ |\ b\ |\ e\wedge e\ |\ \neg
  e\ |\ \textsf{if}\ e\ e\ e\ |\ x \\
  &&|& \efun{x\cdots x}{e}\ |\ \eappfo{e}{e,\cdots,e}\ |\
  \erec{x}{x,\cdots,x}{e}{e} \\
  \text{Value} & v & ::= & n\ |\ b\ |\ \clov{x\cdots x}{e}{\sigma} \\
\end{array}
\]
Note that $b$ ranges over boolean values. The language does not support the short-circuiting
semantics, i.e. $e_2$ must be evaluated in expression $e_1\wedge e_2$ whenever $e_1$
evaluates to true or not.
Write the operational semantics of the form \fbox{$\evald{e}{v}$} for the expressions.

\item What are the results of the following expression:
\[
  \begin{array}{l}
  \erec{\cf}{\cx}{\eifz{\cx}{\cx}{(\eadd{1}{(\eapp{\cf}{(\esub{\cx}{1})})})}}{\\
  \ebind{\cf}{\eadd{42}{(\eapp{\cf}{\cx})}}{\\
  \eapp{\cf}{7}
  }
  }
  \end{array}
\]
in different scoping semantics when we evaluate it under the following environment?
\begin{verbatim}
Map("f" -> CloV("x", Add(Num(13), Id("x")), Map()))
\end{verbatim}

\begin{enumerate}
  \item Dynamic scope
  \item Static scope
\end{enumerate}
\end{enumerate}
