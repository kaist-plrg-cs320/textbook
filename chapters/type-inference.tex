\chapter{Type Inference}
\labch{type-inference}

\renewcommand{\plang}{\textsf{TFAE}\xspace}
\renewcommand{\Lang}{\textsf{TIFAE}\xspace}

While type checking helps programmers by preventing run-time errors, it gives
programmers the burden of attaching type annotations to function parameters.
Compare the following code snippets:

\begin{verbatim}
def f(x: Int): Int =
  x + 1
\end{verbatim}

\begin{verbatim}
def f(x):
  return x + 1
\end{verbatim}

They are written in Scala and Python, respectively. While Scala, a statically
typed language, requires type annotations, Python, a dynamically typed language,
does not. Writing correct type annotations requires programmers' time and
effort. Besides, type annotations make code verbose. The necessity of type
annotations is a crucial downside of statically typed languages.

For this reason, many languages provide \textit{type inference}\index{type
inference}, which recovers omitted type annotations. Under the presence of type
inference, programmers can selectively write type annotations. When a particular
type annotation is beneficial for documentation purpose, they can leave it in
the code. Otherwise, they can omit the type annotation and save their time.
Then, the type checker will infer correct types for missing type annotations.

Each real-world language has a different capability of type inference. For
example, Scala's type checker can infer the parameter types of anonymous
functions used as arguments. The following code passes type checking:

\begin{verbatim}
def f(g: Int => Int): Int = g(0)
f(x => x + 1)
\end{verbatim}

The type checker infers the type of $\cx$ to \code{Int}. However, parameter
types cannot be inferred in other cases. Both lines of the following code do
not pass type checking:

\begin{verbatim}
def f(x) = x + 1
(x => x + 1)(0)
\end{verbatim}

As these examples show, Scala's type inference is quite limited. On the other
hand, OCaml and Haskell's type inference is much more powerful. They allow
programmers to omit all the parameter type and return type annotations. For
example, the following code snippets pass type checking:

\begin{itemize}
  \item OCaml:
    \begin{verbatim}
let f x = x + 1
    \end{verbatim}
  \item Haskell:
    \begin{verbatim}
f x = x + 1
    \end{verbatim}
\end{itemize}

This chapter introduces a basic implementation of type inference by implementing a
type checker of \Lang, which extends \plang with type inference.

\section{Syntax}

The syntax of \Lang is as follows:

\[
e\ ::=\ n
\ |\ \eadd{e}{e}
\ |\ x
\ |\ \efun{x}{e}
\ |\ \eapp{e}{e}
\]

The only difference from \plang is the absence of type annotations. \Lang allows
omission of every type annotation, just like OCaml and Haskell, so the syntax
does not require type annotations.

\section{Type Inference as Decision Problem}

The usual type checking problem is formulated as the following decision problem
on an expression $e$:

\begin{quote}
  Is $e$ well-typed?
\end{quote}

A type checking algorithm is an algorithm solving this decision problem. Our
\code{typeCheck} function, which is an implementation of a type checking
algorithm, terminates without an exception if the answer is ``yes'' and throws
an exception if the answer is ``no.''

This section formulates the type inference problem as a decision problem in two
different ways. While they reflect two different perspectives on type
inference, they define the same problem.

\subsection{Type Recovery}

\newcommand{\ersr}[1]{\embox{erase}(#1)}

We first define the \textit{type erasure}\index{type erasure} of \plang. Type
erasure removes type annotations from expressions. The type erasure of \plang is
defined as the following function $\embox{erase}$:

\[
  \begin{array}{r@{~}c@{~}l}
    \ersr{n} &=& n \\
    \ersr{\eadd{e_1}{e_2}} &=& \eadd{\ersr{e_1}}{\ersr{e_2}} \\
    \ersr{x} &=& x \\
    \ersr{\efunt{x}{\tau}{e}} &=& \efun{x}{\ersr{e}} \\
    \ersr{\eapp{e_1}{e_2}} &=& \eapp{\ersr{e_1}}{\ersr{e_2}} \\
  \end{array}
\]

It takes a \plang expression as an input and outputs a \Lang expression.
For example, $\ersr{\efunt{\cx}{\tnum}{\eadd{\cx}{\cx}}}$ is
$\efun{\cx}{\eadd{\cx}{\cx}}$.

The type inference problem of \Lang is the following decision problem on a \Lang
expression $e$:

\begin{quote}
  \hspace{-1em}Is there a well-typed \plang expression $e'$ such that $\ersr{e'}=e$?
\end{quote}

For example, when $\efun{\cx}{\eadd{\cx}{\cx}}$ is given, the answer is ``yes''
because $\efunt{\cx}{\tnum}{\eadd{\cx}{\cx}}$ is such a \plang expression. On
the other hand, when $\efun{\cx}{\eapp{\cx}{\cx}}$ is given, the answer is
``no'' because there is no such expression. An algorithm solving this decision
problem is the type inference algorithm of \Lang.

From this perspective, type inference is to figure out whether there are type
annotations that can make a given expression become a well-typed, fully
type-annotated expression. If such type annotations exist, the expression passes
type checking. Otherwise, it is rejected by the type checker. This formulation
fits our intuitive understanding of type inference.

\subsection{Non-Algorithmic Type System}

Consider the following static semantics of \Lang:

\typerule{Typ-Num}{
  \typeofdnc{n}{\tnum}.
}

\vspace{-1em}

\[
  \typeofd{n}{\tnum}
  \quad\textsc{[Typ-Num]}
\]

\typerule{Typ-Add}{
  If
  \typeofdn{e_1}{\tnum} and \typeofdn{e_2}{\tnum},\\
  then
  \typeofdn{e_1+e_2}{\tnum}.
}

\vspace{-1em}

\[
  \inferrule
  { \typeofd{e_1}{\tnum} \\ \typeofd{e_2}{\tnum} }
  { \typeofd{e_1+e_2}{\tnum} }
  \quad\textsc{[Typ-Add]}
\]

\typerule{Typ-Id}{
  If $x$ is in the domain of $\Gamma$, \\
  then \typeofdn{x}{\Gamma(x)}.
}

\vspace{-1em}

\[
  \inferrule
  { x\in\dom{\Gamma} }
  { \typeofd{x}{\Gamma(x)} }
  \quad\textsc{[Typ-Id]}
\]

\typerule{Typ-Fun}{
  If
  \typeofn{\Gamma[x:\tau_1]}{e}{\tau_2} for some $\tau_1$,\\
  then
  \typeofdn{\efunt{x}{e}}{\tarrow{\tau_1}{\tau_2}}.
}

\vspace{-1em}

\[
  \inferrule
  { \typeof{\Gamma[x:\tau_1]}{e}{\tau_2} }
  { \typeofd{\efun{x}{e}}{\tarrow{\tau_1}{\tau_2}} }
  \quad\textsc{[Typ-Fun]}
\]

\typerule{Typ-App}{
  If
  \typeofdn{e_1}{\tarrow{\tau_1}{\tau_2}} and
  \typeofdn{e_2}{\tau_1}, \\
  then
  \typeofdn{e_1\ e_2}{\tau_2}.
}

\vspace{-1em}

\[
  \inferrule
  { \typeofd{e_1}{\tarrow{\tau_1}{\tau_2}} \\
    \typeofd{e_2}{\tau_1} }
  { \typeofd{e_1\ e_2}{\tau_2} }
  \quad\textsc{[Typ-App]}
\]

It is mostly the same as the type system of \plang. The only difference is in Rule
\textsc{Typ-Fun}. While the expression does not have $\tau_1$, it is used in the
premise. $\tau_1$ just comes out of thin air. How can it be possible?

A type system does not necessarily let us know how to implement a type checker.
Its goal is to declare whether a certain expression is well-typed or not.
Therefore, $\tau_1$ coming out of thin air cannot be any problem. This type
system still tells us the well-typedness of each \Lang expression. For example,
we can prove that $\efun{\cx}{\eadd{\cx}{\cx}}$ is well-typed with the type
system:

\[
  \inferrule
  {
    \inferrule
    {
      \inferrule
      { \cx\in\dom{[\cx:\tnum]} }
      { \typeof{[\cx:\tnum]}{\cx}{\tnum} } \\
      \inferrule
      { \cx\in\dom{[\cx:\tnum]} }
      { \typeof{[\cx:\tnum]}{\cx}{\tnum} }
    }
    { \typeof{[\cx:\tnum]}{\eadd{\cx}{\cx}}{\tnum} }
  }
  { \typeofe{\efun{\cx}{\eadd{\cx}{\cx}}}{\tarrow{\tnum}{\tnum}} }
\]

Why have we chosen $\tnum$ as $\tau_1$? So far, it is just our intuition. The
type system does not let us know how we can choose a type for $\tau_1$, and we
do not have any systematic explanation. However, if we choose $\tnum$ somehow,
we can prove that the type of $\efun{\cx}{\eadd{\cx}{\cx}}$ is
$\tarrow{\tnum}{\tnum}$. In addition, if we try to prove that the type of
$\efun{\cx}{\eapp{\cx}{\cx}}$ is $\tau$ for some $\tau$, we will conclude that
it is impossible after a few attempts. As these examples show, the type system
allows us to determine the well-typedness of each expression, while it does not
tell us how.

We say that this type system is \textit{non-algorithmic}\index{non-algorithmic
type system}, or \textit{declarative}\index{declarative type system}, because it
only declares well-typed expressions but does not directly give us an algorithm
to check the well-typedness of each expression. It is different from most of the
type systems we saw before this chapter. Except the type system of
\textsf{STFAE}, which also is non-algorithmic, all the other type systems are
\textit{algorithmic}\index{algorithmic type system}, or
\textit{syntax-directed}\index{syntax-directed type system}. They let us know
how to implement type checking algorithms in a syntax-directed fashion, i.e., by
giving an exact type-checking procedure for each syntactic form of an
expression.

The type inference problem of \Lang is the following decision problem on a \Lang
expression $e$:

\begin{quote}
  Is $e$ well-typed under this non-algorithmic type system?
\end{quote}

Like before, the answer to $\efun{\cx}{\eadd{\cx}{\cx}}$ is ``yes,'' and the
answer to $\efun{\cx}{\eapp{\cx}{\cx}}$ is ``no.'' The type inference algorithm
of \Lang is the same as the type checking algorithm of this non-algorithmic type
system.

From this perspective, the type inference problem asks us to design a type
checking algorithm of a non-algorithmic type system. Since a non-algorithmic
type system is literally non-algorithmic, finding a type checking algorithm is
a nontrivial problem.

\section{Type Variables and Constraints}

Now, our goal is to design and implement a type inference algorithm of
\Lang. We mostly focus on the first perspective on type inference. Therefore,
the decision problem we want to solve is as follows:

\begin{quote}
  \hspace{-1em}Is there a well-typed \plang expression $e'$ such that $\ersr{e'}=e$?
\end{quote}

Our type checker must terminate without an exception if the answer is ``yes''
and throw an exception if the answer is ``no.'' Before going into the detailed
implementation, this section gives you a high-level idea of type inference.

The goal of type inference is to find proper types for missing type annotations
in a given expression. If such types exist, the type checker accepts the
expression. Otherwise, it rejects the expression.  For this purpose, the type
checker attaches a type variable to each function parameter before starting type
inference. For example, $\efun{\cx}{\eadd{\cx}{\cx}}$ becomes
$\efunt{\cx}{\cX}{\eadd{\cx}{\cx}}$ where $\cX$ is a type variable. Similarly,
$\efun{\cx}{\efun{\cy}{\eadd{\cx}{\cy}}}$ becomes
$\efunt{\cx}{\cX}{\efunt{\cy}{\cY}{\eadd{\cx}{\cy}}}$, and
$\eadd{(\eapp{(\efun{\cx}{\cx})}{1})}{(\eapp{(\efun{\cx}{\cx})}{2})}$ becomes
$\eadd{(\eapp{(\efunt{\cx}{\cX}{\cx})}{1})}{(\eapp{(\efunt{\cx}{\cY}{\cx})}{2})}$.
During type inference, each type variable acts as a type, just like
ordinary types such as $\tnum$ and $\tarrow{\tnum}{\tnum}$. Now, the goal of the
type inference is to find a correct type for each type variable.

Note that type variables are different from program variables and type
parameters. They are similar to variables in mathematical equations. Consider
the following system of equations:

\[
\begin{array}{r@{~}c@{~}l}
  x + y = 3 \\
  2x + 3y = 7
\end{array}
\]

By solving this system, we get $x=2$ and $y=1$ as the solution. Similarly, the
type checker creates a system of equations for type variables. The following
system of equations is an example:

\[
\begin{array}{r@{~}c@{~}l}
  \cX &=& \tnum \\
  \tarrow{\cX}{\tnum} &=& \tarrow{\cY}{\tnum}
\end{array}
\]

By solving the system, the type checker gets $\cX=\tnum$ and $\cY=\tnum$.
In the context of type inference, each equation consisting a system is called
a constraint.

How does the type checker construct such a system of equations? It collects
constraints by checking how each subexpression is used in a given expression.
Let us consider various examples.

\paragraph{Single Solution: Number}

Consider $\efunt{\cx}{\cX}{\eadd{\cx}{\cx}}$. This expression uses $\cx$ for
addition. Addition is possible for only integers. Thus, the type of $\cx$ must
be $\tnum$. Since the type of $\cx$ is $\cX$, the type checker concludes that
$\cX=\tnum$. This is the only constraint that can be found.  Therefore, the type
checker gets the following system:

\[
\begin{array}{r@{~}c@{~}l}
  \cX &=& \tnum \\
\end{array}
\]

The solution to this system is $\cX=\tnum$. By applying this solution to the
original expression, we get $\efunt{\cx}{\tnum}{\eadd{\cx}{\cx}}$, which is a
well-typed \plang expression.

\paragraph{Single Solution: Function}

Consider $\efunt{\cx}{\cX}{\eadd{(\eapp{\cx}{0})}{1}}$. The expression applies
$\cx$ to $0$. The type of $\cx$ must be a function type. In addition, the
parameter type of $\cx$ must be $\tnum$ because the type of its argument, $0$,
is $\tnum$. On the other hand, there is no information to decide the return type
of $\cx$ yet. Thus, the type checker introduces a new type variable $\cY$ and
uses it as the return type. This results in the constraint
$\cX=\tarrow{\tnum}{\cY}$. In addition, the type of $\eapp{\cx}{0}$ is $\cY$.
The return value, whose type is $\cY$, is used for addition. Therefore, The type
checker gets one more constraint, $\cY=\tnum$, and completes the following
system:

\[
\begin{array}{r@{~}c@{~}l}
  \cX &=& \tarrow{\tnum}{\cY} \\
  \cY &=& \tnum \\
\end{array}
\]

The solution to this system is $\cX=\tarrow{\tnum}{\tnum}$ and $\cY=\tnum$. By
applying this solution to the original expression, we get
$\efunt{\cx}{\tarrow{\tnum}{\tnum}}{\eadd{(\eapp{\cx}{0})}{1}}$, which is a
well-typed \plang expression.

\paragraph{Multiple Solutions}

In mathematics, there is a system of equations with multiple solutions. The
following system is such an example:

\[
\begin{array}{r@{~}c@{~}l}
  x - 2y &=& 0 \\
  3x - 6y &=& 0 \\
\end{array}
\]

$x=2$ and $y=1$ are one possible solution. $x=4$ and $y=2$ are also a solution.
Any $x$ and $y$ satisfying $x=2y$ are a solution to this system.

Similarly, there is an expression generating a system with multiple solutions.
Consider $\efunt{\cx}{\cX}{\efunt{\cy}{\cY}{\eapp{\cx}{\cy}}}$. This
expression applies $\cx$ to $\cy$, whose type is $\cY$. Therefore, the type
of $\cx$ is $\tarrow{\cY}{\cZ}$, where $\cZ$ is a new type variable. This gives
$\cX=\tarrow{\cY}{\cZ}$, which is the only constraint. Thus, the type checker
gets the following system:

\[
\begin{array}{r@{~}c@{~}l}
  \cX &=& \tarrow{\cY}{\cZ} \\
\end{array}
\]

This system has multiple solutions. One solution is $\cX=\tnum\rightarrow\tnum$,
$\cY=\tnum$, and $\cZ=\tnum$. It is easy to find more solutions. The existence
of multiple solutions is not a problem at all. If at least one solution exists,
we have type annotations making a given expression well-typed. Therefore,
expressions with multiple solutions also pass type checking.

\paragraph{No Solution: Conflicting Requirements}

In mathematics, there is a system of equations without any solutions. One
example is below.

\[
\begin{array}{r@{~}c@{~}l}
  x - 2y &=& 1 \\
  3x - 6y &=& 0 \\
\end{array}
\]

Similarly, there is an expression generating a system with no solutions.
Consider $\efunt{\cx}{\cX}{\eadd{(\eapp{\cx}{0})}{\cx}}$. Because of
$\eapp{\cx}{0}$, the type checker gets $\cX=\tarrow{\tnum}{\cY}$, where $\cY$
is a new type variable. At the same time, using $\eapp{\cx}{0}$ and $\cx$ for
addition gives $\cY=\tnum$ and $\cX=\tnum$. The resulting system is as follows:

\[
\begin{array}{r@{~}c@{~}l}
  \cX &=& \tarrow{\tnum}{\cY} \\
  \cY &=& \tnum \\
  \cX &=& \tnum \\
\end{array}
\]

By inlining the second constraint in the first constraint, we can simplify the
system as follows:

\[
\begin{array}{r@{~}c@{~}l}
  \cX &=& \tarrow{\tnum}{\tnum} \\
  \cX &=& \tnum \\
\end{array}
\]

To satisfy both constraints, $\cX$ must be $\tarrow{\tnum}{\tnum}$ and $\tnum$
at the same time. Such conflicting requirements cannot be satisfied. It is
impossible to annotate the original expression to make a well-typed \plang
expression. Thus, the type checker rejects the expression.

\paragraph{No Solution: Cyclic Constraint}

Consider $\efunt{\cx}{\cX}{\eapp{\cx}{\cx}}$. The only constraint is
$\cX=\tarrow{\cX}{\cY}$, obtained from \eapp{\cx}{\cx}. Therefore, the type
checker constructs the following system:

\[
\begin{array}{r@{~}c@{~}l}
  \cX &=& \tarrow{\cX}{\cY} \\
\end{array}
\]

This system does not have any solutions either. The reason is that $\cX$ occurs
in $\tarrow{\cX}{\cY}$. If a type occurs in another type, the two types cannot
be the same.

\section{Type Checker}

This section implements a type checker of \Lang.

We define type variables as follows:

\begin{verbatim}
sealed trait Type
...
case class VarT(x: String) extends Type
\end{verbatim}

For instance, \code{VarT("X")} denotes the type variable $\cX$.
We implement the abstract syntax is as usual:

\begin{verbatim}
sealed trait Expr
...
case class Fun(x: String, t: Type, b: Expr) extends Expr
\end{verbatim}

This allows the type checker to attach type variables to parameters.  For
example, \code{Fun("x", VarT("X"), Add(Id("x"), Id("x")))} corresponds to
$\efunt{\cx}{\cX}{\eadd{\cx}{\cx}}$. Note that it is still possible for
programmers to attach type annotations to parameters because the type of
\code{t} is still \code{Type}. However, to focus on type inference, this chapter
does not consider expressions with type annotations.

Following the high-level idea illustrated in the previous section, the type
checker consists of two steps: collecting constraints and solving them.

\subsection{Collecting Constraints}

The constraints-collecting procedure is similar to the usual type checking
procedure. The type of each subexpression in a given expression is computed.
However, while the usual procedure immediately checks whether the type of each
subexpression fits the required type, the new procedure adds constraints to a
global constraint list in order to solve them later.

For this purpose, we define the global constraint list as follows:

\begin{verbatim}
import scala.collection.mutable.ListBuffer
val constraints: ListBuffer[(Type, Type)] = ListBuffer()
\end{verbatim}

A \code{ListBuffer} is a mutable list. While we can avoid mutation by using a
strategy similar to store-passing style, we simply use mutable collections
because the main focus of this chapter is type inference, not mutation. Adding
\code{($\tau_1$, $\tau_2$)} to the list corresponds to adding the constraint
$\tau_1=\tau_2$.

We implement the \code{typeCheck} function, which collects the constraints
from a given expression, as follows:

\begin{verbatim}
def typeCheck(e: Expr, tenv: TEnv): Type = e match {
  case Num(_) => NumT
  case Add(l, r) =>
    val lt = typeCheck(l, tenv)
    val rt = typeCheck(r, tenv)
    constraints.append((lt, NumT))
    constraints.append((rt, NumT))
    NumT
  case Id(x) => tenv(x)
  case Fun(x, t, b) =>
    ArrowT(t, typeCheck(b, tenv + (x -> t)))
  case App(f, a) =>
    val ft = typeCheck(f, tenv)
    val at = typeCheck(a, tenv)
    val rt = VarT(fresh())
    constraints.append((ft, ArrowT(at, rt)))
    rt
}

def fresh(): String = ...
\end{verbatim}

The \code{Num}, \code{Id}, and \code{Fun} cases are the same as before. They do
not add any constraints. On the other hand, the types of both operands for
addition must be $\tnum$, so the \code{Add} case adds two constraints. In a
similar manner, the \code{App} case also adds a constraint. The \code{fresh}
function returns a new type variable.

Let us run \code{typeCheck} with the examples in the previous section.

\begin{itemize}
\item $\efunt{\cx}{\cX}{\eadd{\cx}{\cx}}$

\begin{verbatim}
typeCheck(
  Fun("x", VarT("X"), Add(Id("x"), Id("x"))),
  Map()
)
println(constraints)
\end{verbatim}

It prints

\begin{verbatim}
ListBuffer(
  (VarT(X), NumT),
  (VarT(X), NumT)
)
\end{verbatim}

which is equivalent to the following system:

\[
\begin{array}{r@{~}c@{~}l}
  \cX &=& \tnum \\
  \cX &=& \tnum \\
\end{array}
\]

\item $\efunt{\cx}{\cX}{\eadd{(\eapp{\cx}{0})}{1}}$

\begin{verbatim}
typeCheck(
  Fun("x", VarT("X"), Add(App(Id("x"), Num(0)), Num(1))),
  Map()
)
println(constraints)
\end{verbatim}

It prints

\begin{verbatim}
ListBuffer(
  (VarT(X), ArrowT(NumT, VarT(Y))),
  (VarT(Y), NumT),
  (NumT, NumT)
)
\end{verbatim}

which is equivalent to the following system:

\[
\begin{array}{r@{~}c@{~}l}
  \cX &=& \tarrow{\tnum}{\cY} \\
  \cY &=& \tnum \\
  \tnum &=& \tnum \\
\end{array}
\]

\item $\efunt{\cx}{\cX}{\efunt{\cy}{\cY}{\eapp{\cx}{\cy}}}$

\begin{verbatim}
typeCheck(
  Fun("x", VarT("X"), Fun("y", VarT("Y"), App(Id("x"), Id("y")))),
  Map()
)
println(constraints)
\end{verbatim}

It prints

\begin{verbatim}
ListBuffer((VarT(X), ArrowT(VarT(Y), VarT(Z))))
\end{verbatim}

which is equivalent to the following system:

\[
\begin{array}{r@{~}c@{~}l}
  \cX &=& \tarrow{\cY}{\cZ} \\
\end{array}
\]

\item $\efunt{\cx}{\cX}{\eadd{(\eapp{\cx}{0})}{\cx}}$

\begin{verbatim}
typeCheck(
  Fun("x", VarT("X"), Add(App(Id("x"), Num(0)), Id("x"))),
  Map()
)
println(constraints)
\end{verbatim}

It prints

\begin{verbatim}
ListBuffer(
  (VarT(X), ArrowT(NumT, VarT(Y))),
  (VarT(Y), NumT),
  (VarT(X), NumT)
)
\end{verbatim}

which is equivalent to the following system:

\[
\begin{array}{r@{~}c@{~}l}
  \cX &=& \tarrow{\tnum}{\cY} \\
  \cY &=& \tnum \\
  \cX &=& \tnum \\
\end{array}
\]

\item $\efunt{\cx}{\cX}{\eapp{\cx}{\cx}}$

\begin{verbatim}
typeCheck(
  Fun("x", VarT("X"), App(Id("x"), Id("x"))),
  Map()
)
println(constraints)
\end{verbatim}

It prints

\begin{verbatim}
ListBuffer((VarT(X), ArrowT(VarT(X), VarT(Y))))
\end{verbatim}

which is equivalent to the following system:

\[
\begin{array}{r@{~}c@{~}l}
  \cX &=& \tarrow{\cX}{\cY} \\
\end{array}
\]
\end{itemize}

As these examples show, the constraints collected by the procedure are the same as
our expectation, only except that there additionally are redundant or
trivially satisfied constraints.

\subsection{Solving Constraints}

Let us see how to solve a system of equations for type variables. We use a
mutable map to denote a solution. Like before, we use mutation to focus only on
type inference, although we can avoid mutation.

\begin{verbatim}
import scala.collection.mutable.Map
val solution: Map[String, Type] = Map()
\end{verbatim}

By importing \code{scala.collection.mutable.Map}, we can use mutable maps.  When
\code{solution} contains \code{$X$ -> $\tau$}, the solution to a type variable
$X$ is a type $\tau$. For example, \code{"X" -> NumT} means that the solution to
$\cX$ is $\tnum$. On the other hand, if map does not contain $X$ as a key, then
$X$ can be any type. Initially, the map is empty since we have not solved the
system and each type variable can be any type. During the solving procedure,
the type checker adds entries to the map by considering each constraint.

The main function of the solving procedure is the following \code{solve}
function:

\begin{verbatim}
def solve() =
  for ((t1, t2) <- constraints)
    unify(t1, t2)
\end{verbatim}

It iterates over the \code{constraints} list and possibly updates the
\code{solution} map at each iteration. The \code{unify} function updates the
map. It takes two types as arguments and updates the map by using the fact that
the two types are equal.

\begin{verbatim}
def unify(t1: Type, t2: Type): Unit =
  (t1, t2) match {
    ...
  }
\end{verbatim}

Its behavior depends on what given types are. The simplest case is
when both are $\tnum$.

\begin{verbatim}
case (NumT, NumT) =>
\end{verbatim}

Since they are already equal, there is nothing to do. The function immediately
returns.

On the other hand, if they are different, it throws an exception.

\begin{verbatim}
case (NumT, ArrowT(_, _)) | (ArrowT(_, _), NumT) => error()
\end{verbatim}

When both types are function types, their parameter types and return types are
unifed, respectively.

\begin{verbatim}
case (ArrowT(pt1, rt1), ArrowT(pt2, rt2)) =>
  unify(pt1, pt2)
  unify(rt1, rt2)
\end{verbatim}

It follows the fact that if $\tarrow{\tau_1}{\tau_1'}=\tarrow{\tau_2}{\tau_2'}$,
then $\tau_1=\tau_2$ and $\tau_1'=\tau_2'$.

When at least one of given types is a type variable, it becomes a little more
complicated. Suppose that \code{t1} is a type variable.

\begin{verbatim}
case (t1 @ VarT(x), t2) =>
  if (t1 == t2)
    ()
  else
    solution(x) = t2
\end{verbatim}

If \code{t2} is the same type variable, they are already equal, so there is
nothing to do, just like the \code{(NumT, NumT)} case. Otherwise, to make the
two types equal, the type checker sets \code{t2} as the solution to \code{t1} by
updating the \code{solution} map.

However, the above code is not quite right for two reasons. First, there can be
a solution to \code{t1} already. Suppose that \code{t1} is \code{VarT("X")} and
\code{solution} has \code{"X" -> NumT}. If \code{t2} is \code{ArrowT(NumT,
NumT)}, the type variable $\cX$ should be $\tnum$ and $\tarrow{\tnum}{\tnum}$ at
the same time, which is impossible. In this case, the type checker must throw
an exception.

One may think considering the value of \code{solution.get(t1)} is enough to
solve this problem. Unfortunately, it is wrong. The reason is that a solution to
a type variable can be another type variable. For example, the map can contain
\code{"X" -> VarT("Y")} and \code{"Y" -> NumT}. In this case, the type denoted
by $\cX$ is $\tnum$, not $\cY$. Therefore, we define a recursive function to
find the type denoted by each type variable.

\begin{verbatim}
def resolve(ty: Type): Type = ty match {
  case VarT(x) => solution.get(x) match {
    case None => ty
    case Some(t) => resolve(t)
  }
  case _ => ty
}
\end{verbatim}

The \code{resolve} function computes the type denoted by a given type, which can
be a type variable. If the given type is not a type variable, it returns the
type per se. If the type is a type variable, it checks the \code{solution} map.
If the map does not contain the type variable, the type variable can be any
type, so it just returns the type variable as it is. If the map has a solution
type to the type variable, it makes a recursive call on the type to handle cases
that the type is another type variable.

The results of \code{resolve(VarT("X"))} under various conditions are as follows:

\begin{itemize}
  \item When \code{solution} is \code{Map()}: \code{VarT("X")}
  \item When \code{solution} is \code{Map("X" -> NumT)}: \code{NumT}
  \item When \code{solution} is \code{Map("X" -> VarT("Y"))}: \code{VarT("Y")}
  \item When \code{solution} is \code{Map("X" -> VarT("Y"), "Y" -> NumT)}: \code{NumT}
\end{itemize}

Now, we resolve types before unifying them.

\begin{verbatim}
def unify(t1: Type, t2: Type): Unit =
  (resolve(t1), resolve(t2)) match {
    ...
  }
\end{verbatim}

It prevents a type variable from having a solution in the map already. If the
resolved type is a type variable, the type variable is guaranteed not to have a
solution. It allows the type checker to set \code{t2} as the solution to
\code{t1} without worrying any conflicts.

Another problem is that \code{t1} can occur in \code{t2}. For example, \code{t2}
can be \code{ArrowT(VarT("X"), NumT)}, where \code{t1} is \code{VarT("X")}. As
mentioned already, there are no solutions in such cases. Thus, the type checker
must throw an exception. To detect such cases, it must check whether \code{t1}
occurs in \code{t2}. For this purpose, we define the following \code{occurs}
function:

\begin{verbatim}
def occurs(t1: VarT, t2: Type): Boolean =
  resolve(t2) match {
    case NumT => false
    case ArrowT(l, r) => occurs(t1, l) || occurs(t1, r)
    case t2 @ VarT(_) => t1 == t2
  }
\end{verbatim}

The function checks whether a type variable \code{t1} occurs in a type
\code{t2}.  When \code{t2} is \code{NumT}, \code{t1} does not occur in
\code{t2}. On the other hand, when \code{t2} is a function type, we can say that
\code{t1} occurs in \code{t2} if \code{t1} occurs in either parameter type or
return type. When \code{t2} is a type variable, \code{t1} occurs in \code{t2} if
they are the same.

Now, we fix \code{unify} to call \code{occurs}.

\begin{verbatim}
case (t1 @ VarT(x), t2) =>
  if (t1 == t2)
    ()
  else if (occurs(t1, t2))
    error()
  else
    solution(x) = t2
\end{verbatim}

Like before, if \code{t1} and \code{t2} are the same type variable, there is
nothing to do. If \code{t1} occurs in \code{t2}, it throws an exception.
Otherwise, \code{t2} becomes the solution to \code{t1}.

The only remaining case is when \code{t2} is a type variable. As we already have
implemented the case when \code{t1} is a type variable, it is enough to call
\code{unify} again with \code{t1} and \code{t2} swapped.

\begin{verbatim}
case (t1, t2 @ VarT(_)) => unify(t2, t1)
\end{verbatim}

By combining all the cases, we complete \code{unify} as below.

\begin{verbatim}
def unify(t1: Type, t2: Type): Unit =
  (resolve(t1), resolve(t2)) match {
    case (t1 @ VarT(x), t2) =>
      if (t1 == t2)
        ()
      else if (occurs(t1, t2))
        error()
      else
        solution(x) = t2
    case (t1, t2 @ VarT(_)) => unify(t2, t1)
    case (NumT, NumT) =>
    case (ArrowT(t3, t4), ArrowT(t5, t6)) =>
      unify(t3, t5)
      unify(t4, t6)
    case (NumT, ArrowT(_, _)) |
         (ArrowT(_, _), NumT) => error()
  }
\end{verbatim}

Let us run \code{solve} with some of the examples in the previous section.

\begin{itemize}
\item $\efunt{\cx}{\cX}{\eadd{\cx}{\cx}}$

The \code{constraints} list is as follows:

\begin{verbatim}
ListBuffer(
  (VarT(X), NumT),
  (VarT(X), NumT)
)
\end{verbatim}

    In the first iteration, \code{VarT("X")} and \code{NumT} are unified.
    \code{unify} adds \code{"X" -> NumT} to \code{solution}. In the second
    iteration, \code{VarT("X")} and \code{NumT} are unified again. Since
    resolving \code{VarT("X")} gives \code{NumT}, it goes to the \code{(NumT,
    NumT)} case, which does nothing. Therefore, \code{solution} becomes
    \code{Map("X" -> NumT)} after running \code{solve}.

\item $\efunt{\cx}{\cX}{\eadd{(\eapp{\cx}{0})}{\cx}}$

The \code{constraints} list is as follows:

\begin{verbatim}
ListBuffer(
  (VarT(X), ArrowT(NumT, VarT(Y))),
  (VarT(Y), NumT),
  (VarT(X), NumT)
)
\end{verbatim}

    After the first two iterations, \code{solution} has \code{"X" ->
    ArrowT(NumT, VarT(Y))} and \code{"Y" -> NumT}. In the third iteration,
    \code{VarT("X")} and \code{NumT} are unified. Since resolving
    \code{VarT("X")} gives \code{ArrowT(NumT, VarT(Y))}, it goes to the
    \verb!(ArrowT(_, _), NumT)! case, which throws an exception.

\item $\efunt{\cx}{\cX}{\eapp{\cx}{\cx}}$

The \code{constraints} list is as follows:

\begin{verbatim}
ListBuffer((VarT(X), ArrowT(VarT(X), VarT(Y))))
\end{verbatim}

    Since \code{occurs(VarT("X"), ArrowT(VarT("X"), VarT("Y")))} is
    \code{true}, \code{unify} throws an exception.

\end{itemize}

\section{Improving Type Checker}

\subsection{Solving while Collecting}

While the current type checker starts solving after collecting all the
constraints, we can implement a type checker that gradually improves the
solution each time it collects a new constraint. Since the type checker
immediately reflects each constraint in the solution, there is no need to store
constraints in the global list. Therefore, we do not use the \code{collections}
list anymore. In addition, \code{solve} is not required either because there is
no separate solving step. After removing \code{collections} and \code{solve},
the only remaining change is to fix \code{typeCheck} to immediately call
\code{unify} for each constraint instead of storing the constraint in
\code{collections}.

\begin{verbatim}
def typeCheck(e: Expr, tenv: TEnv): Type = e match {
  ...
  case Add(l, r) =>
    val lt = typeCheck(l, tenv)
    val rt = typeCheck(r, tenv)
    unify(lt, NumT)
    unify(rt, NumT)
    NumT
  case App(f, a) =>
    val ft = typeCheck(f, tenv)
    val at = typeCheck(a, tenv)
    val rt = VarT(fresh())
    unify(ft, ArrowT(at, rt))
    rt
}
\end{verbatim}

All the other functions remain the same.

\subsection{Removing Names of Type Variables}

Instead of giving a name to each type variable and storing their types in the
\code{solution} map, we can store a type in each nameless type variable and
avoid using the map. As the first step, we change the definition of \code{VarT}
as follows:

\begin{verbatim}
case class VarT(var ty: Option[Type]) extends Type
\end{verbatim}

When a \code{VarT} has \code{None}, the type variable can be any type. It
corresponds to \code{solution} not having a certain type variable as a key.
When a \code{VarT} has \code{Some($\tau$)}, the solution to the type variable is
$\tau$. To make such updates possible, \code{ty} is defined as a mutable field.

Note that updating one type variable does not affect other type variables.
Consider the following code:

\begin{verbatim}
val tvx = VarT(None)
val tvy = VarT(None)

tvx.ty = Some(NumT)
\end{verbatim}

After running this code, \code{tvx.ty} becomes \code{Some(NumT)}, but
\code{tvy.ty} remains the same as \code{None}.

Due to this change, each expression initially has type variables containing
\code{None}, instead of their names. Containing only \code{None}, not
\code{Some}, corresponds to \code{solution} being empty in the beginning. Some
examples of expressions are below.

\begin{itemize}
\item $\efunt{\cx}{\cX}{\eadd{\cx}{\cx}}$
\begin{verbatim}
Fun("x", VarT(None), Add(Id("x"), Id("x")))
\end{verbatim}
\item $\efunt{\cx}{\cX}{\efunt{\cy}{\cY}{\eapp{\cx}{\cy}}}$
\begin{verbatim}
Fun("x", VarT(None),
  Fun("y", VarT(None), App(Id("x"), Id("y"))))
\end{verbatim}
\end{itemize}

In addition, we need to modify the \code{App} case of \code{typeCheck} as
follows:

\begin{verbatim}
case App(f, a) =>
  val ft = typeCheck(f, tenv)
  val at = typeCheck(a, tenv)
  val rt = VarT(None)
  unify(ft, ArrowT(at, rt))
  rt
\end{verbatim}

Thus, \code{fresh} is no longer required.

Now, we need to fix \code{unify}, \code{occurs}, and \code{resolve}. Let us
start with \code{unify}. The first case of \code{unify} requires changes.

\begin{verbatim}
case (t1 @ VarT(_), t2) =>
  if (t1 eq t2)
    ()
  else if (occurs(t1, t2))
    error()
  else
    t1.ty = Some(t2)
\end{verbatim}

There are two changes. One is \code{t1 == t2} being \code{t1 eq t2}. In Scala,
\code{==} compares values structurally, but \code{eq} compares values
physically. Consider the following code:

\begin{verbatim}
val tvx = VarT(None)
val tvy = VarT(None)

println(tvx == tvy)
println(tvx eq tvy)
\end{verbatim}

It prints \code{true} and \code{false}. Since both have \code{None}, they are
structurally equal. However, they are two distinct \code{VarT} objects stored in
different memory locations, so they are physically different. We want to check
whether two type variables are the same type variable, not whether they have the
same solution. Thus, we should use \code{eq} instead of \code{==}.

Another change is \code{solution(x) = t2} being \code{t1.ty = Some(t2)}. This is
because that we now store the solution to each type variable in the type
variable itself, not the \code{solution} map.

We need a similar change in \code{occurs} as well.

\begin{verbatim}
def occurs(t1: VarT, t2: Type): Boolean =
  resolve(t2) match {
    ...
    case t2 @ VarT(_) => t1 eq t2
  }
\end{verbatim}

For the same reason, \code{==} becomes \code{eq}.

Lastly, we fix \code{resolve}.

\begin{verbatim}
def resolve(ty: Type): Type = ty match {
  case VarT(Some(t)) => resolve(t)
  case _ => ty
}
\end{verbatim}

Instead of looking up \code{solution}, it simply checks the optional type in a
\code{VarT}.

Combining all, our type checker implementation is as follows:

\begin{verbatim}
def typeCheck(e: Expr, tenv: TEnv): Type = e match {
  case Num(_) => NumT
  case Add(l, r) =>
    val lt = typeCheck(l, tenv)
    val rt = typeCheck(r, tenv)
    unify(lt, NumT)
    unify(rt, NumT)
    NumT
  case Id(x) => tenv(x)
  case Fun(x, t, b) =>
    ArrowT(t, typeCheck(b, tenv + (x -> t)))
  case App(f, a) =>
    val ft = typeCheck(f, tenv)
    val at = typeCheck(a, tenv)
    val rt = VarT(None)
    unify(ft, ArrowT(at, rt))
    rt
}

def unify(t1: Type, t2: Type): Unit =
  (resolve(t1), resolve(t2)) match {
    case (t1 @ VarT(x), t2) =>
      if (t1 eq t2)
        ()
      else if (occurs(t1, t2))
        error()
      else
        t1.ty = Some(t2)
    case (t1, t2 @ VarT(_)) => unify(t2, t1)
    case (NumT, NumT) =>
    case (ArrowT(t3, t4), ArrowT(t5, t6)) =>
      unify(t3, t5)
      unify(t4, t6)
    case (NumT, ArrowT(_, _)) |
         (ArrowT(_, _), NumT) => error()
  }

def occurs(t1: VarT, t2: Type): Boolean =
  resolve(t2) match {
    case NumT => false
    case ArrowT(l, r) => occurs(t1, l) || occurs(t1, r)
    case t2 @ VarT(_) => t1 eq t2
  }

def resolve(ty: Type): Type = ty match {
  case VarT(Some(t)) => resolve(t)
  case _ => ty
}
\end{verbatim}

Note that we do not need any global variables now.

\section{Exercises}

\begin{exercise}
\labex{type-inference-pair}

This exercise extends \Lang with pairs.

\begin{verbatim}
sealed trait Expr
...
case class Pair(l: Expr, r: Expr) extends Expr
case class Fst(e: Expr) extends Expr
case class Snd(e: Expr) extends Expr

sealed trait Type
...
case class PairT(l: Type, r: Type) extends Type

def resolve(ty: Type): Type = ...

def unify(t1: Type, t2: Type): Unit =
  (resolve(t1), resolve(t2)) match {
    ...
    case (PairT(l1, r1), PairT(l2, r2)) => ???
  }

def occurs(t1: VarT, t2: Type): Boolean =
  resolve(t2) match {
    ...
    case PairT(l, r) => ???
  }

def typeCheck(e: Expr, tenv: TEnv): Type = e match {
  ...
  case Pair(l, r) => ???
  case Fst(e) => ???
  case Snd(e) => ???
}
\end{verbatim}

    \begin{enumerate}
        \item
Complete the implementation.
You may refer to the following typing rules:

\[
\inferrule
{ \typeofd{e_1}{\tau_1} \\ \typeofd{e_2}{\tau_2} }
{ \typeofd{(e_1,e_2)}{\tau_1\times\tau_2} }
\quad
\inferrule
{ \typeofd{e}{\tau_1\times\tau_2} }
{ \typeofd{e\textsf{.1}}{\tau_1} }
\quad
\inferrule
{ \typeofd{e}{\tau_1\times\tau_2} }
{ \typeofd{e\textsf{.2}}{\tau_2} }
\]

where

        \begin{itemize}
        \item $(e_1,e_2)$ denotes \code{Pair($e_1$,$e_2$)}.
        \item $e\textsf{.1}$ denotes \code{Fst($e$)}.
        \item $e\textsf{.2}$ denotes \code{Snd($e$)}.
        \item $\tau_1\times\tau_2$ denotes \code{PairT($\tau_1$,$\tau_2$)}.
        \end{itemize}
  \item
    Write the result of \code{typeCheck($e$, Map())}, where $e$ is
    \code{Fun("x", VarT(None), Add(Fst(Id("x")), Num(42)))},
    which represents
    $\efunt{\cx}{\cX}{(\eadd{(\cx\textsf{.1})}{42})}$.
\end{enumerate}

\end{exercise}

\begin{exercise}
\labex{type-inference-box}

This exercise extends \Lang with boxes. Complete the following implementation:
\begin{verbatim}
sealed trait Expr
...
case class NewBox(e: Expr) extends Expr
case class OpenBox(b: Expr) extends Expr
case class SetBox(b: Expr, e: Expr) extends Expr

sealed trait Type
...
case class BoxT(t: Type) extends Type

def resolve(ty: Type): Type = ...

def unify(t1: Type, t2: Type): Unit =
  (resolve(t1), resolve(t2)) match {
    ...
    case (BoxT(t3), BoxT(t4)) => ???
  }

def occurs(t1: VarT, t2: Type): Boolean =
  resolve(t2) match {
    ...
    case BoxT(t) => ???
  }

def typeCheck(e: Expr, tenv: TEnv): Type = e match {
  ...
  case NewBox(e) => ???
  case OpenBox(b) => ???
  case SetBox(b, e) => ???
}
\end{verbatim}

You may refer to the following typing rules:

\[
    \inferrule
    { \typeofd{e}{\tau} }
    { \typeofd{\eref{e}}{\eref{\tau}} }
    \quad
    \inferrule
    { \typeofd{e}{\eref{\tau}} }
    { \typeofd{\ederef{e}}{\tau} }
    \quad
    \inferrule
    { \typeofd{e_1}{\eref{\tau}} \\
      \typeofd{e_2}{\tau} }
    { \typeofd{\eset{e_1}{e_2}}{\tau} }
\]

where

\begin{itemize}
  \item $\eref{e}$ denotes \code{NewBox($e$)}.
  \item $\ederef{e}$ denotes \code{OpenBox($e$)}.
  \item $\eset{e_1}{e_2}$ denotes \code{SetBox($e_1$, $e_2$)}.
  \item $\eref{\tau}$ denotes \code{BoxT($\tau$)}.
\end{itemize}

\end{exercise}

\begin{exercise}
\labex{type-inference-list}

This exercise extends \Lang with lists. Complete the following implementation:
\begin{verbatim}
sealed trait Expr
...
case object Nil extends Expr
case class Cons(h: Expr, t: Expr) extends Expr
case class Head(e: Expr) extends Expr
case class Tail(e: Expr) extends Expr

sealed trait Type
...
case class ListT(t: Type) extends Type

def resolve(ty: Type): Type = ...

def unify(t1: Type, t2: Type): Unit =
  (resolve(t1), resolve(t2)) match {
    ...
    case (ListT(t3), ListT(t4)) => ???
  }

def occurs(t1: VarT, t2: Type): Boolean =
  resolve(t2) match {
    ...
    case ListT(t) => ???
  }

def typeCheck(e: Expr, tenv: TEnv): Type = e match {
  ...
  case Nil => ???
  case Cons(h, t) => ???
  case Head(e) => ???
  case Tail(e) => ???
}
\end{verbatim}

You may refer to the following typing rules:

\[
\begin{array}{c}
  \inferrule
  {}
  { \typeofd{\textsf{nil}}{\textsf{list}\ \tau} }
  \qquad\qquad
  \inferrule
  { \typeofd{e_1}{\tau} \\ \typeofd{e_2}{\textsf{list}\ \tau} }
  { \typeofd{\textsf{cons}\ e_1\ e_2}{\textsf{list}\ \tau} }
  \\[2em]
  \inferrule
  { \typeofd{e}{\textsf{list}\ \tau} }
  { \typeofd{\textsf{head}\ e}{\tau} }
  \qquad\qquad
  \inferrule
  { \typeofd{e}{\textsf{list}\ \tau} }
  { \typeofd{\textsf{tail}\ e}{\textsf{list}\ \tau} }
\end{array}
\]

where

\begin{itemize}
  \item $\textsf{nil}$ denotes \code{Nil}.
  \item $\textsf{cons}\ e_1\ e_2$ denotes \code{Cons($e_1$, $e_2$)}.
  \item $\textsf{head}\ e$ denotes \code{Head($e$)}.
  \item $\textsf{tail}\ e$ denotes \code{Tail($e$)}.
  \item $\textsf{list}\ \tau$ denotes \code{ListT($\tau$)}.
\end{itemize}

\end{exercise}

\begin{exercise}
\labex{type-inference-option}

This exercise extends \Lang with options. Complete the following implementation:
\begin{verbatim}
sealed trait Expr
...
case object NoneE extends Expr
case class SomeE(e: Expr) extends Expr
case class Match(e: Expr, e1: Expr, x: String, e2: Expr) extends Expr

sealed trait Type
...
case class OptionT(t: Type) extends Type

def resolve(ty: Type): Type = ...

def unify(t1: Type, t2: Type): Unit =
  (resolve(t1), resolve(t2)) match {
    ...
    case (OptionT(t3), OptionT(t4)) => ???
  }

def occurs(t1: VarT, t2: Type): Boolean =
  resolve(t2) match {
    ...
    case OptionT(t) => ???
  }

def typeCheck(e: Expr, tenv: TEnv): Type = e match {
  ...
  case NoneE => ???
  case SomeE(e) => ???
  case Match(e, e1, x, e2) => ???
}
\end{verbatim}

You may refer to the following typing rules:

\[
\inferrule
{}
{ \typeofd{\textsf{none}}{\textsf{option}\ \tau} }
\qquad
\inferrule
{ \typeofd{e}{\tau} }
{ \typeofd{\textsf{some}\ e}{\textsf{option}\ \tau} }
\qquad
\inferrule
{
  \typeofd{e}{\textsf{option}\ \tau'} \\
  \typeofd{e_1}{\tau} \\
  \typeof{\Gamma[x:\tau']}{e_2}{\tau} \\
}
{ \typeofd{e\ \textsf{match}\ \textsf{none}\rightarrow e_1,\textsf{some}\ x\rightarrow e_2}{\tau} }
\]

where

\begin{itemize}
  \item $\textsf{none}$ denotes \code{NoneE}.
  \item $\textsf{some}\ e$ denotes \code{SomeE($e$)}.
  \item ${e\ \textsf{match}\ \textsf{none}\rightarrow e_1,
    \textsf{some}\ x\rightarrow e_2}$ denotes \code{Match($e$, $e_1$, $x$, $e_2$)}.
  \item $\textsf{option}\ \tau$ denotes \code{OptionT($\tau$)}, which is the type
    of an option possibly containing a value of $\tau$.
\end{itemize}

\end{exercise}

\begin{exercise}
\labex{type-inference-sysf}

The following code implements type erasure of \textsf{PTFAE}:

\begin{verbatim}
object Erased {
  sealed trait Expr
  case class Num(n: Int) extends Expr
  case class Add(l: Expr, r: Expr) extends Expr
  case class Sub(l: Expr, r: Expr) extends Expr
  case class Id(x: String) extends Expr
  case class Fun(p: String, b: Expr) extends Expr
  case class App(f: Expr, a: Expr) extends Expr
}

def erase(e: Expr): Erased.Expr = e match {
  case Num(n) => Erased.Num(n)
  case Add(l, r) => Erased.Add(erase(l), erase(r))
  case Sub(l, r) => Erased.Sub(erase(l), erase(r))
  case Id(x) => Erased.Id(x)
  case Fun(p, t, b) => Erased.Fun(p, erase(b))
  case App(f, a) => Erased.App(erase(f), erase(a))
  case TyFun(a, e) => erase(e)
  case TyApp(e, t) => erase(e)
}
\end{verbatim}

\begin{enumerate}
  \item Formally define the type erasure function
    \framebox{$\ersr{e}=e$} according to the code.
  \item Write the result of applying $\embox{erase}$ to each of the following
    expressions:
    \begin{enumerate}
      \item
        $\eapp{\etapp{(\etfun{\alpha}{\efunt{\cx}{\alpha}{\cx}})}{\tnum}}{1}$
      \item
        $\eapp{\eapp{\etapp{\etapp{(\etfun{\alpha}{\etfun{\beta}{\efunt{\cx}{\alpha}{\efunt{\cy}{\beta}{\cy}}}})}{\tnum}}{\tnum}}{1}}{2}$
    \end{enumerate}
\item Write a well-typed expression that becomes the each of the following
  expressions by $\embox{erase}$.

    \begin{enumerate}
      \item
        $\efun{\cx}{\eapp{(\eapp{\cx}{(\efun{\cx}{\cx})}}{(\cx\ 1))}}$
      \item
        $\eapp{\eapp{(\efun{\cx}{\efun{\cy}{\eapp{\eapp{\cx}{\cx}}{\cy}}})}{(\efun{\cx}{\cx})}}{1}$
      \item
        $\eapp{(\efun{\cx}{\eapp{\eapp{\cx}{\cx}}{1}})}{\efun{\cy}{\cy}}$
    \end{enumerate}

    Although Wells proved that the type inference problem of \textsf{PTFAE} is
    undecidable in 1994~\cite{wells1994typability}, you can solve this problem
    with your intution.

\end{enumerate}

\end{exercise}

\begin{exercise}
\labex{type-inference-hm}

Consider the following language:

\vspace{-1em}

\[
\begin{array}{rrlrlrrlrrl}
  e &::=& n &|& \efun{x}{e} & b &::=& \true & \tau &::=& \tnum \\
  &|& b &|& \eapp{e}{e} & &|& \false &&|& \tbool \\
  &|& x &|& \ebind{x}{e}{e} &&&&&|& \tarrow{\tau}{\tau} \\
  &|& \eseq{e}{e} &&&&&&&|& \alpha \\
\end{array}
\]

A type scheme $\sigma$ is a possibly polymorphic type,
and a type environment contains type schemes instead of types:

\vspace{-1em}

\[ \sigma\ ::=\ \tau\ |\ \tforall{\alpha}{\sigma} \]
\[ \Gamma \in \embox{Id}\finto\text{TypeScheme} \]

Consider the following non-algorithmic type system:

\vspace{-1em}

\[
\begin{array}{@{}c@{}}
  \inferrule
  {}
  { \typeofd{n}{\tnum} }
  \qquad
  \qquad
  \qquad
  \inferrule
  {}
  { \typeofd{b}{\tbool} }
  \\[2em]
  \inferrule
  {
    x\in\dom{\Gamma} \\
    \Gamma(x) \succ \tau
  }
  { \typeofd{x}{\tau} }
  \qquad
  \inferrule
  {
    \typeofd{e_1}{\tau_1} \\
    \typeofd{e_2}{\tau_2}
  }
  { \typeofd{\eseq{e_1}{e_2}}{\tau_2} }
  \\[2em]
  \inferrule
  { \typeof{\Gamma[x:\tau]}{e}{\tau'} }
  { \typeofd{\efun{x}{e}}{\tarrow{\tau}{\tau'}} }
  \qquad
  \inferrule
  {
    \typeofd{e_1}{\tarrow{\tau}{\tau'}} \\
    \typeofd{e_2}{\tau}
  }
  { \typeofd{\eapp{e_1}{e_2}}{\tau'} }
  \\[2em]
  \inferrule
  {
    \typeofd{e_1}{\tau} \\
    \tau\prec_\Gamma\sigma \\
    \typeof{\Gamma[x:\sigma]}{e_2}{\tau'}
  }
  { \typeofd{\ebind{x}{e_1}{e_2}}{\tau'} }
\end{array}
\]

where

\vspace{-1em}

\[
\begin{array}{@{}r@{\hskip4pt}c@{\hskip4pt}l@{}}
  \begin{array}{c}
    \fbox{$\sigma\succ\tau$} \\[0.5em]
    \ \\[1em]
  \end{array}
  &&
  \begin{array}{r@{\hskip4pt}c@{\hskip4pt}l}
    \tau&\succ&\tau \\[0.5em]
    \tforall{\alpha_1}{\cdots\tforall{\alpha_n}{\tau}}&\succ&
    \tau[\alpha_1\leftarrow\tau_1,\cdots,\alpha_n\leftarrow\tau_n]\\[1em]
  \end{array}
  \\[2em]
  \fbox{$\tau\prec_\Gamma\sigma$}
  &&
  \inferrule
  { \embox{FTV}(\tau)\setminus\embox{FTV}(\Gamma)=\{\alpha_1,\cdots,\alpha_n\} }
  { \tau\prec_\Gamma\tforall{\alpha_1}{\cdots\tforall{\alpha_n}{\tau}} }
  \\[2em]
\embox{FTV}(\tnum) &=& \emptyset
\\
\embox{FTV}(\tbool) &=& \emptyset
\\
\embox{FTV}(\tarrow{\tau_1}{\tau_2}) &=& \embox{FTV}(\tau_1) \cup \embox{FTV}(\tau_2)
\\
\embox{FTV}(\alpha) &=& \{\alpha\}
\\
  \embox{FTV}(\tforall{\alpha}{\sigma}) &=& \embox{FTV}(\sigma) \setminus \{\alpha\}
\\
\embox{FTV}(\Gamma) &=& \bigcup_{x\in\dom{\Gamma}}\embox{FTV}(\Gamma(x))
\\
\end{array}
\]

($\succ$ denotes specialization of a type scheme to a type. $\prec$ denotes
generalization of a type to a type scheme. $\embox{FTV}$ denotes free type
variables in a type or a type environment.)

Is each of the following expressions well-typed? If so, draw a type derivation
tree. Otherwise, explain why.

\begin{enumerate}
  \item $\eapp{(\efun{\cx}{\eapp{\cx}{42}})}{\efun{\cy}{\cy}}$
  \item $\eapp{(\efun{\cx}{\eseq{\eapp{\cx}{42}}{\eapp{\cx}{\true}}})}{\efun{\cy}{\cy}}$
  \item $\ebind{\cx}{\efun{\cy}{\cy}}{\eseq{\eapp{\cx}{42}}{\eapp{\cx}{\true}}}$
\end{enumerate}

Note that this type system is called a \textit{Hindley-Milner type
system}\index{Hindley-Milner type system}, named after two independent inventors
of the type system, Hindley~\cite{hindley1969principal} and
Milner~\cite{milner1978theory}. OCaml and Haskell are well-known for the use of
this type system.

\end{exercise}
