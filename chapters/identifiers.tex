\setchapterpreamble[u]{\margintoc}
\chapter{Identifiers}
\labch{identifiers}

AE defined by the last article features only integers, sums, and differences.
This article defines VAE by adding local variables to AE.

\section{Identifiers}

\term{Identifiers} name entities such as variables, functions, types, and
packages. Most languages use strings satisfying specific conditions as
identifiers.

\begin{verbatim}
f(0)
def f(x: Int): Int = {
  val y = 2
  x + y
}
f(1)
x - y
\end{verbatim}

\verb!f!, \verb!x!, and \verb!y! are identifiers. Strictly speaking, \verb!Int!
also is an identifier, but the article ignores it. \verb!f! is the name of a
function; \verb!x! is the name of the parameter of \verb!f!; \verb!y! is the name
of the local variable of \verb!f!.

Three kinds of identifiers exist.

The first kind is a \term{binding occurrence}. If the occurrence associates an
identifier with an entity, it is a binding occurrence. Both \verb!f! and \verb!x!
in \verb!def f(x: Int): Int! are binding occurrences; \verb!y! in \verb!val y = 2!
is a binding occurrence. \verb!f! refers to a function, \verb!(x: Int) => { val y = 2; x + y }!;
\verb!x! refers to an argument passed to \verb!f!; \verb!y!
refers to \verb!2!.

The second kind is a \term{bound occurrence}. The \term{scope} of a binding
occurrence means a code region where an identifier introduced by the binding
occurrence can appear. An occurrence of an identifier in a scope is a bound
occurrence. The identifier denotes an entity referred by the identifier. Both
\verb!x! and \verb!y! in \verb!x + y! are bound occurrences; the binding
occurrences of \verb!x! and \verb!y! respectively bind the bound occurrences.
\verb!x! denotes an argument used at a function call-site; \verb!y! denotes
\verb!2!. \verb!f! in \verb!f(1)! is a bound occurrence; the binding occurrence
of \verb!f! binds the bound occurrence. \verb!f! denotes a function referred by
\verb!f!, and \verb!f(1)! thus is a correct function call.

The last kind is a \term{free identifier}. An identifier not belonging to any
scopes of the binding occurrences of the identifier is a free identifier. The
scope of the binding occurrence of \verb!f! includes the body of \verb!f! and
every line below the definition of \verb!f!. Therefore, \verb!f! in \verb!f(0)!
is a free identifier. The scope of \verb!x! equals the body of \verb!f!, and the
scope of \verb!y! starts at a line defining \verb!y! and ends at the end of the
definition of \verb!f!. Both \verb!x! and \verb!y! in \verb!x – y! are free
identifiers. Unlike bound occurrences, free identifiers do not denote any
entities. Evaluating an expression containing free identifiers causes an error in
most languages; compiling code containing free identifiers results in a compile
error.

Consider a bound occurrence that resides in the scopes of more than one binding
occurrence simultaneously. Most languages regard an entity referred by a binding
occurrence owning the smallest scope as whom the bound occurrence denotes.
\term{Shadowing} means that a binding occurrence with a smaller scope takes
priority over a binding occurrence with a larger scope.

\begin{verbatim}
def f(x: Int): Int = {
  def g(x: Int): Int = x
  g(x)
}
\end{verbatim}

The code is an example of shadowing. Both \verb!x!'s in \verb!def f(x: Int): Int!
and \verb!def g(x: Int): Int! are binding occurrences. The occurrence of \verb!x!
at the end of the second line is a bound occurrence. It belongs to the scopes of
the two binding occurrences at the same time. As having a smaller scope than the
first binding occurrence, the second binding occurrence binds the bound
occurrence. The bound occurrence denotes an argument passed to \verb!g!. On the
other hand, \verb!x! in \verb!g(x)! belongs to only the scope of the first
binding occurrence of \verb!x!. The first binding occurrence binds it, and it
denotes an argument passed to \verb!f!.

\section{VAE}

Expressions of VAE are arithmetic expressions that can define immutable local
variables and refer to variables.

\subsection{Syntax}

The following is the abstract syntax of VAE:

\[
\begin{array}{lrcl}
\text{Integer} & n & \in & \mathbb{Z} \\
\text{Variable} & x & \in & \textit{Id} \\
\text{Expression} & e & ::= & n \\
&& | & e + e \\
&& | & e - e \\
&& | & \textsf{val}\ x = e\ \textsf{in}\ e \\
&& | & x \\
\text{Value} & v & ::= & n
\end{array}
\]

Metavariable \(n\) ranges over integers.

Metavariable \(x\) ranges over variables, which belong to set \(\mathit{Id}\).
\(\mathit{Id}\) is a set of objects differing from elements of other sets; it is
the set of every possible identifier.

Metavariable \(e\) ranges over expressions. Expression \(\textsf{val}\ x=e_1\
\textsf{in}\ e_2\) declares a local variable. The occurrence of \(x\) is a
binding occurrence. \(x\) refers to a value denoted by \(e_1\). The scope of the
occurrence covers entire \(e_2\) but not \(e_1\). Expression \(x\) is either a
bound occurrence of \(x\) or a free identifier. If it is a bound occurrence, it
denotes a value associated with the identifier. For example, the first occurrence
of \(x\) in \(\textsf{val}\ x=1\ \textsf{in}\ x\) binds the second occurrence of
\(x\). On the other hand, the second \(x\) in \(\textsf{val}\ x=x\ \textsf{in}\
1\) is a free identifier.

Metavariable \(v\) ranges over values. Values of VAE are integers as those of AE
are.

\subsection{Semantics}

The natural semantics of AE defines the semantics of integers, sums, and
differences. The natural semantics of VAE additionally requires the semantics of
binding and bound occurrences. Consider \(\textsf{val}\ x=1\ \textsf{in}\ x\).
The following Scala expression expresses the same thing:

\begin{verbatim}
{
  val x = 1
  x
}
\end{verbatim}

It denotes \verb!1!. Since the binding occurrence associates \verb!x! with
\verb!1!, the bound occurrence denotes \verb!1!. The value denoted by the entire
expression equals \verb!1!, the value of the bound occurrence.

\(\textsf{val}\ x=1\ \textsf{in}\ x\) is evaluated in the same manner. The value
denoted by the whole expression equals the value denoted by \(x\), the body of
the expression. However, it is not possible to evaluate \(x\) without any
information. A value associated with \(x\) by the binding occurrence is
necessary. VAE newly defines \term{environments} to store such information. An
environment is a dictionary of finite size; its keys are identifiers, and its
values are values of VAE. Mathematically, it is a \term{partial function} from
\(\mathit{Id}\) to \(\text{Value}\).

\[
\begin{array}{lrcl}
\text{Environment} & \sigma & \in & \mathit{Id}\hookrightarrow \text{Value}
\end{array}
\]

Metavariable \(\sigma\) ranges over environments. Binding occurrences add
information to environments, and bound occurrences use information in
environments.

The natural semantics of AE is a binary relation over \(\text{Expression}\) and
\(\text{Value}\). The natural semantics of VAE is not a binary relation of the
two sets since evaluating an expression requires an environment providing the
values of identifiers in the expression. The natural semantics is a ternary
relation over \(\text{Environment}\), \(\text{Expression}\), and
\(\text{Value}\).


\[\Rightarrow\subseteq\text{Environment}\times\text{Expression}\times\text{Value}\]

\((\sigma,e,v)\in\Rightarrow\) implies that evaluating expression \(e\) under
environment \(\sigma\) results in value \(v\). \(\sigma\vdash e\Rightarrow v\)
replaces the notation. Intuitively, an environment and an expression are given
input, and a value is output obtained by computation.

Inference rules define the natural semantics of VAE.

\[
\sigma\vdash n\Rightarrow n
\]

The rule equals that of AE.

\[
\inferrule
{ \sigma\vdash e_1\Rightarrow n_1 \\ \sigma\vdash e_2\Rightarrow n_2 }
{ \sigma\vdash e_1+e_2\Rightarrow n_1+n_2 }
\]

\[
\inferrule
{ \sigma\vdash e_1\Rightarrow n_1 \\ \sigma\vdash e_2\Rightarrow n_2 }
{ \sigma\vdash e_1-e_2\Rightarrow n_1-n_2 }
\]

The rules equal those of AE except evaluating \term{subexpressions} \(e_1\) and
\(e_2\) also requires environments. The premises and the conclusion use the same
environment.

The rule for a binding occurrence needs a way to add information to an
environment. The following defines an extension of an environment.

\[
\sigma \lbrack x\mapsto
v\rbrack=\sigma\setminus\{(x',v'):x'=x\land\sigma(x')=v'\}\cup\{(x,v)\}
\]

An environment obtained by adding that identifier \(x\) refers to value \(v\) to
environment \(\sigma\) is \(\sigma\) with \((x,v)\) but without any other pairs
about \(x\). It has the following property:

\[
\sigma \lbrack x\mapsto v\rbrack(x') =
\begin{cases}
v & \text{if}\ x=x' \\
v' & \text{if}\ x\neq x'\land\sigma(x')=v'
\end{cases}
\]

The \(x=x'\) case shows shadowing.

\[
\inferrule
{
  \sigma\vdash e_1\Rightarrow v_1 \\
  \sigma\lbrack x\mapsto v_1\rbrack\vdash e_2\Rightarrow v_2
}
{ \sigma\vdash \textsf{val}\ x=e_1\ \textsf{in}\ e_2\Rightarrow v_2 }
\]

Evaluating \(\textsf{val}\ x=e_1\ \textsf{in}\ e_2\) requires a value denoted by
identifier \(x\). Firstly, evaluating \(e_1\) attains \(v_1\). The scope of the
binding occurrence excludes \(e_1\) so that the evaluation uses \(\sigma\). A
value denoted by the whole expression equals \(v_2\), denoted by \(e_2\). The
evaluation of \(e_2\) uses \(\sigma\lbrack x\mapsto v_1\rbrack\), which knows
that \(x\) refers to \(v_1\) in addition to information from \(\sigma\).

If an expression is an identifier, it is essential to check whether the
identifier is a bound occurrence or a free identifier. The following function
calculates the domain of an environment:

\[
\mathit{Domain}(\sigma)=\{x:\exists v.(x,v)\in\sigma\}
\]

\(\mathit{Domain}(\sigma)\) is the domain of \(\sigma\). An identifier that
belongs to the domain is a bound occurrence. Otherwise, it is a free identifier.

\[
\inferrule
{ x\in\mathit{Domain}(\sigma) }
{ \sigma\vdash x\Rightarrow \sigma(x)}
\]

The bound occurrence of \(x\) denotes a value whom \(x\) refers to in the
environment. If \(x\) is a free identifier, the premise is false, and, therefore,
an expression containing a free identifier does not denote any values. It
explains that executing a program with free identifiers results in an error.

The following inference rules are all of the natural semantics of VAE:

\[
\sigma\vdash n\Rightarrow n
\]

\[
\inferrule
{ \sigma\vdash e_1\Rightarrow n_1 \\ \sigma\vdash e_2\Rightarrow n_2 }
{ \sigma\vdash e_1+e_2\Rightarrow n_1+n_2 }
\]

\[
\inferrule
{ \sigma\vdash e_1\Rightarrow n_1 \\ \sigma\vdash e_2\Rightarrow n_2 }
{ \sigma\vdash e_1-e_2\Rightarrow n_1-n_2 }
\]

\[
\inferrule
{
  \sigma\vdash e_1\Rightarrow v_1 \\
  \sigma\lbrack x\mapsto v_1\rbrack\vdash e_2\Rightarrow v_2
}
{ \sigma\vdash \textsf{val}\ x=e_1\ \textsf{in}\ e_2\Rightarrow v_2 }
\]

\[
\inferrule
{ x\in\mathit{Domain}(\sigma) }
{ \sigma\vdash x\Rightarrow \sigma(x)}
\]

\subsection{Implementing an Interpreter}

The following Scala code expresses the abstract syntax of VAE:

\begin{verbatim}
sealed trait Expr
case class Num(n: Int) extends Expr
case class Add(l: Expr, r: Expr) extends Expr
case class Sub(l: Expr, r: Expr) extends Expr
case class Val(x: String, i: Expr, b: Expr) extends Expr
case class Id(x: String) extends Expr
\end{verbatim}

An identifier is an arbitrary string. \verb!Val! constructs an expression
declaring a local variable; \verb!Id! constructs an expression using a local
variable.

\verb!Map! in the Scala standard library can represent environments. \verb!Map!
creates dictionaries.

\begin{verbatim}
val m0: Map[String, Int] = Map.empty
val m1: Map[String, Int] = m0 + ("one" -> 1)
m1.getOrElse("one", -1)  // 1
m1.getOrElse("two", -1)  // -1
\end{verbatim}

The \verb!Map! type has two type parameters: the first one is the type of keys,
and the second one is the type of values. \verb!Map[String, Int]! is the type of
a dictionary whose keys and values are strings and integers respectively.
\verb!Map.empty! is a dictionary without any keys and values. The library defines
various methods: the \verb!+! method adds the pair of a key and a value to a
dictionary; the \verb!getOrElse! method takes two arguments and returns a value
corresponding to a key, the first argument, in a dictionary. \verb!getOrElse!
evaluates the second argument and returns the result only if the first argument
is not a key of the dictionary.

\begin{verbatim}
type Env = Map[String, Int]
\end{verbatim}

An environment is a partial function from identifiers to values. Identifiers are
strings; values are integers. Therefore, the type of an environment is
\verb!Map[String, Int]!. The above expression defines \verb!Env!, a type alias of
\verb!Map[String, Int]!. The two types are identical, and an environment thus has
type \verb!Env!.

\begin{verbatim}
def lookup(x: String, env: Env): Int =
  env.getOrElse(x, throw new Exception)
\end{verbatim}

The \verb!lookup! function returns a value referred by an identifier, the first
argument, in an environment, the second argument. It throws an exception if the
environment lacks information about the identifier.

\begin{verbatim}
def interp(e: Expr, env: Env): Int = e match {
  case Num(n) => n
  case Add(l, r) => interp(l, env) + interp(r, env)
  case Sub(l, r) => interp(l, env) - interp(r, env)
  case Val(x, i, b) => interp(b, env + (x -> interp(i, env)))
  case Id(x) => lookup(x, env)
}
\end{verbatim}

The \verb!Num! case equals that of the interpreter, implemented by the last
article, of AE. The \verb!Add! and \verb!Sub! cases pass \verb!env! as arguments.
The \verb!Val! case calculates the value of \verb!b! under an environment
extended with the result of evaluating \verb!i! under \verb!env!. The \verb!Id!
case calls the \verb!lookup! function to find a value denoted by an identifier.

Calling the \verb!interp! function with arguments \(\textsf{val}\ x=1\
\textsf{in}\ x+x\) and the empty environment yields \verb!2!.

\begin{verbatim}
// val x = 1 in x + x
interp(
  Val("x", Num(1),
    Add(Id("x"), Id("x"))
  ),
  Map.empty
)
// 2
\end{verbatim}

The following proof tree proves that \(\textsf{val}\ x=1\ \textsf{in}\ x+x\)
denotes \(2\) under \(\emptyset\), the empty environment:

\[
\inferrule
{
  \emptyset\vdash 1\Rightarrow 1 \\
  \inferrule
  {
    \inferrule
    { x\in\mathit{Domain}(\lbrack x\mapsto 1\rbrack) }
    { \lbrack x\mapsto 1\rbrack\vdash x\Rightarrow 1 } \\
    \inferrule
    { x\in\mathit{Domain}(\lbrack x\mapsto 1\rbrack) }
    { \lbrack x\mapsto 1\rbrack\vdash x\Rightarrow 1 }
  }
  {\lbrack x\mapsto 1\rbrack\vdash x+x\Rightarrow 2 }
}
{ \emptyset\vdash \textsf{val}\ x=1\ \textsf{in}\ x+x\Rightarrow 2 }
\]

\section{Exercises}
