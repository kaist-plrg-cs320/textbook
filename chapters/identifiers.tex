\setchapterpreamble[u]{\margintoc}
\chapter{Identifiers}
\labch{identifiers}

\renewcommand{\plang}{\textsf{AE}\xspace}
\renewcommand{\lang}{\textsf{VAE}\xspace}

Variables are one of the basic concepts of programming languages. A
\textit{variable}\index{variable} relates a name to a value. We use the
value of a variable by writing the name of the variable. For example, the following Scala program
prints \code{3}.

\begin{verbatim}
val x = 3
println(x)
\end{verbatim}

The program defines a variable whose name is \code{x} and value is \code{3}. At
the second line, the name \code{x} denotes the value \code{3}.

We call the names of variables identifiers. An
\textit{identifier}\index{identifier} is a name related to a
certain entity in a program. Not only the names of variables are identifiers;
there are various kinds of identifiers:

\begin{itemize}
\item Function names, which are related to functions
\item Parameter names, which are related to the values of arguments
\item Field names, which are related to values of fields
\item Method names, which are related to methods
\item Class names, which are related to classes
\end{itemize}

This chapter introduces identifiers. Identifiers in programs can split into
three groups: binding occurrences, bound occurrences, and free identifiers.
We will see what they are. This chapter discusses identifiers based on the use
of variables in programs. We will define \lang by extending \plang of
\refch{syntax-and-semantics} with variables. Variables of \lang are immutable.
We will deal with mutable variables in \refch{mutable-variables}.
In \lang, the names of variables are
the only identifiers. However, as you have seen already, real-world programming
languages have many kinds of identifiers.

\section{Identifiers}

Identifiers name entities like variables and functions.
Let us discuss notions related to identifiers with the following Scala program:

\begin{verbatim}
f(0)
def f(x: Int): Int = {
  val y = 2
  x + y
}
f(1)
x - z
\end{verbatim}

In this program, \code{f}, \code{x}, \code{y}, and \code{z} are identifiers. Strictly speaking,
\code{Int} also is an identifier, but we ignore it because we do not want to take
types into account here.

A single identifier can occur multiple times in a program. For instance,
\code{f} occurs three times in the program: line 1, line 2, and line 6.
We can classify occurrences of identifiers into three categories:
binding occurrences, bound occurrences, and free identifiers.

An occurrence of an identifier is called a \textit{binding occurrence}\index{binding occurrence}
if the identifier occurs to be defined. A binding occurrence relates the
identifier to a particular entity. The program has three binding occurrences:

\begin{itemize}
  \item \code{f} at line 2

    It relates \code{f} to a function.

  \item \code{x} at line 2

    It relates \code{x} to the value of an argument given to \code{f}.

  \item \code{y} at line 3

    It relates \code{y} to the value \code{2}.
\end{itemize}

Every binding occurrence has its own scope. The \textit{scope}\index{scope} of a binding
occurrence means a code region where the indentifier defined by the binding
occurrence is alive, i.e. usable. The scope of each identifier in the program is as follows:

\begin{itemize}
  \item \code{f}

    A function can be used in its body (as Scala allows recursive function
    definitions) and at the lines below its definition. The scope of
    \code{f} is from line 3 to line 7.

  \item \code{x}

    A parameter of a function can be used only in the function body. The scope of
    \code{x} is line 3 and line 4.

  \item \code{y}

    A variable can be used at the lines below its definition. The scope of
    \code{y} is line 4.
\end{itemize}

An occurrence of an identifier is called a \textit{bound occurrence}\index{bound occurrence}
if the identifier occurs to use the entity related to itself. Since an
identifier becomes related to an entity by its binding occurrence, any bound
occurrences must reside in the scope of the binding occurrence.
The program has three bound occurrences:

\begin{itemize}
  \item \code{f} at line 6

    It denotes the function defined at line 2.

  \item \code{x} at line 4

    It denotes the value of an argument passed to \code{f}.

  \item \code{y} at line 4

    It denotes the value \code{2}.
\end{itemize}

An occurrence of an identifier is called a \textit{free identifier}\index{free
identifier} if it is
neither binding nor bound. A free identifier neither introduces a new name nor
uses a name defined already. It is not in the scope of any binding occurrence of
the same identifier. The program has three free identifiers:

\begin{itemize}
  \item \code{f} at line 1

    It is outside the scope of \code{f}.

  \item \code{x} at line 7

    It is outside the scope of \code{x}.

  \item \code{z} at line 7

    The program never defines \code{z}.
\end{itemize}

We call a free identifier a \textit{free variable}\index{free variable} when it
is the name of a variable. Therefore, both \code{x} and \code{z} at line 7 are
free variables.

Now, consider a binding occurrence that resides in the scope of a binding
occurrence of the same identifier. For example, the following program has two
binding occurrences of \code{x}, and the second binding occurrence is in the
scope of the first binding occurrence.

\begin{verbatim}
def f(x: Int): Int = {
  def g(x: Int): Int =
    x
  g(x)
}
\end{verbatim}

In this case, shadowing happens. \textit{Shadowing}\index{shadowing} means that
the innermost binding occurrence \textbf{shadow}s, i.e. temporarily invalidates,
the outer binding occurrences of the same name. Therefore, \code{x} at line 2
shadows \code{x} at line 1.
\code{x} at line 3 belongs to the scope of both binding occurrences simultaenously.
It denotes the value of an argument given to \code{g}, not \code{f}, because of
shadowing. On the other hand, \code{x} at line 4 denotes the value of an
argument given to \code{f} since it belongs to the scope of only \code{x} at
line 1.

\section{Syntax}

Let us define the abstract syntax of \lang. We do not consider concrete
syntax anymore. Therefore, the term syntax will be used to mean abstract
syntax. Also, from now on, we use the term
\textit{expressions}\index{expression} rather than
programs when we discuss languages like \lang. For example, we say that
$1+2$ is an expression of \plang, and $1$ and $2$ are the subexpressions of
$1+2$.

Recall the example at the beginning of the chapter:

\begin{verbatim}
val x = 3
println(x)
\end{verbatim}

To add variables to \plang, we need two kinds of expressions. The first kind is
expressions defining a variable, i.e. binding an identifier. In the example,
\code{val x = 3; println(x)} is such an expression. It defines the
variable \code{x} and starts the scope of \code{x} so that \code{x} can be used
in \code{println(x)}. We can conclude that an expression defining a variable
consists of three parts: the name of the variable, an expression determining the
value of the variable, and an expression that can use the variable. These parts
are \code{x}, \code{3}, and \code{println(x)}, respectively, in the example. The
second kind is expressions using a variable, i.e. a bound occurrence. In the
example, \code{x} at the second line is such an expression. It uses the variable \code{x} to denote
the value \code{3}. Based on this observation, we can define the syntax of
\lang.

First, we need to add a new syntactic element: identifiers. The metavariable
$x$ ranges over identifiers. Let $\embox{Id}$ be the set of every
identifier.

\[x\in\embox{Id}\]

We do not care what $\embox{Id}$ really is.

The syntax of \lang is as follows:\sidenote{We omit the common part
to \plang.}

\[e\ ::=\ \cdots\ |\ \ebind{x}{e}{e}\ |\ x\]

\begin{itemize}
  \item $\ebind{x}{e_1}{e_2}$

    It defines a new variable whose name is $x$. Therefore, the occurrence of $x$ is a
    binding occurrence. $e_1$ decides the value denoted by the variable. The
    scope of the variable includes $e_2$ but excludes $e_1$.

  \item $x$

    It uses a variable; it is either a bound occurrence of $x$ or a free identifier.
    If it belongs to the scope of a binding occurrence of the same name, then it is a
    bound occurrence and denotes the value associated with the identifier.
    Otherwise, it is a free identifier, which denotes nothing.
\end{itemize}

\section{Semantics}

To define the semantics of \lang, we need an additional semantic element that
stores the values denoted by variables. Without such an element, we cannot know the
value of each variable. We call the element an
\textit{environment}\index{environment}. An environment is a finite partial
function.\sidenote{A finite partial function is a partial function whose domain
is a finite set.} The metavariable $\sigma$ ranges over environments.

\[\embox{Env}=\embox{Id}\finto\mathbb{Z}\]
\[\sigma\in\embox{Env}\]

For example, consider an environment $\sigma$.
If $\sigma(\cx)=1$, the value of a variable named \code{x} is $1$.
An environment is a partial function because it does not have the values
related to free identifiers. If a variable named \code{y} is free in
$\sigma$, then $\sigma(\cy)$ is undefined.
In addition, it is finite since every program
defines only finitely many identifiers.

Every expression in \lang can evaluate to an integer only under some
environment. The reason is obvious: without environments, there is no way to
find the values of variables, and thus environments are essential to evaluation.

The following rule defines the semantics of $x$:

\semanticrule{Id}{
If
  $x$ is in the domain of $\sigma$,\\
then
  \evaldn{x}{\sigma(x)}.
}

If $x$ is an element of the domain of $\sigma$, $x$ is a bound occurrence. The
environment gives us the value denoted by $x$, which is $\sigma(x)$. Then, the
result is $\sigma(x)$. Otherwise, $x$ is not in the domain and is a free
identifier. In that case, we cannot evaluate $x$. The evaluation terminates
immediately. It can be interpreted as a run-time error.

Formally, the semantics of \lang is a
ternary relation over $\embox{Env}$, $E$, and $\mathbb{Z}$ since it must take
environments into account.

\[\Rightarrow\subseteq\embox{Env}\times E\times\mathbb{Z}\]

$(\sigma,e,n)\in\Rightarrow$ is true if and only if
$e$ evaluates to $n$ under $\sigma$.
We write $\sigma\vdash e\Rightarrow n$ instead of $(\sigma,e,n)\in\Rightarrow$.
Intuitively, $\sigma$ and $e$ are inputs, and $n$ is the corresponding output.

Rule \textsc{Id} can be formulated as the following inference rule:
\sidenote{$\dom{\sigma}$ denotes the domain of $\sigma$.}

\[
  \inferrule
  { x\in\dom{\sigma} }
  { \evald{x}{\sigma(x)} }
  \quad\textsc{[Id]}
\]

When a variable is defined, the value of the variable is added to the environment.
We write $\sigma[x\mapsto n]$ to denote an environment obtained by adding the
fact that $x$ denotes $n$ to $\sigma$. Then, the following property holds:

\[
\sigma [x\mapsto n](x') =
\begin{cases}
  n & \text{if}\ x=x' \\
  \sigma(x') & \text{if}\ x\neq x'
\end{cases}
\]

The following rule defines the semantics of $\ebind{x}{e_1}{e_2}$:

\semanticrule{Val}{
If
  \evaldn{e_1}{n_1}, and
  \evaln{\sigma[x\mapsto n_1]}{e_2}{n_2},\\
then
  \evaldn{\ebind{x}{e_1}{e_2}}{n_2}.
}

To evaluate $\ebind{x}{e_1}{e_2}$, we need to determine the value of $x$ first.
It can be done by evaluating $e_1$. Since the scope of $x$ excludes $e_1$,
the evaluation proceeds under $\sigma$, which is a given environment.
The result of $e_1$ is the value of $x$, and this information must be added to
the environment. By adding the fact to $\sigma$, we get $\sigma[x\mapsto n_1]$.
As $e_2$ is the scope of $x$, $e_2$ is evaluated under $\sigma[x\mapsto n_1]$.
The result of $e_2$ is the final result.

This semantics naturally explains shadowing. Let $x$ already be in the domain of
$\sigma$. Suppose that $\sigma(x)=n$. However, $e_2$ is evaluated under
$\sigma[x\mapsto n_1]$, and $\sigma[x\mapsto n_1](x)=n_1$. When $x$ is used in
$e_2$, its value is $n_1$, not $n$. Therefore, we can say that the
innermost definition of $x$ is used for the evaluation of $e_2$. This behavior
exactly matches the concept of shadowing explained before.

Rule \textsc{Val} can be expressed as the following inference rule:

\[
\inferrule
{
  \evald{e_1}{n_1} \\
  \eval{\sigma[x\mapsto n_1]}{e_2}{n_2}
}
{ \evald{\ebind{x}{e_1}{e_2}}{n_2} }
\quad\textsc{[Val]}
\]

The remaining cases are $n$, $\eadd{e_1}{e_2}$, and $\esub{e_1}{e_2}$.
Rules for those cases are basically identical to the rules of \plang.
However, we need to additionally take environments into account.

\semanticrule{Num}{
  \evaldn{n}{n}.
}

\vspace{-1em}

\semanticrule{Add}{
If
  \evaldn{e_1}{n_1}, and \evaldn{e_2}{n_2},\\
then
  \evaldn{\eadd{e_1}{e_2}}{n_1+n_2}.
}

\vspace{-1em}

\semanticrule{Sub}{
If
  \evaldn{e_1}{n_1}, and \evaldn{e_2}{n_2},\\
then
  \evaldn{\esub{e_1}{e_2}}{n_1-n_2}.
}

Integers, addition, and subtraction never update environments.
An integer evaluates to itself. Addition and subtraction evaluates their
subexpressions under the same environment.

We can express the rules in a natural language as the following inference rules:

\[
  \evald{n}{n}\quad\textsc{[Num]}
\]

\[
  \inferrule
  { \evald{e_1}{n_1} \\ \evald{e_2}{n_2} }
  { \evald{\eadd{e_1}{e_2}}{n_1+n_2} }
  \quad\textsc{[Add]}
\]

\[
  \inferrule
  { \evald{e_1}{n_1} \\ \evald{e_2}{n_2} }
  { \evald{\esub{e_1}{e_2}}{n_1-n_2} }
  \quad\textsc{[Sub]}
\]

The following proof tree proves that $\ebind{\cx}{1}{\eadd{\cx}{\cx}}$ evaluates
to $2$ under the empty environment. Note that $[x_1\mapsto n_1,\ldots,x_m\mapsto
n_m]$ denotes an environment whose domain includes from $x_1$ to $x_m$ and each
$x_i$ is mapped to $n_i$.

\[
\inferrule
{
  \emptyset\vdash 1\Rightarrow 1 \\
  \inferrule
  {
    \inferrule
    { \cx\in\dom{[ \cx\mapsto 1]} }
    { [ \cx\mapsto 1]\vdash \cx\Rightarrow 1 } \\
    \inferrule
    { \cx\in\dom{[ \cx\mapsto 1]} }
    { [ \cx\mapsto 1]\vdash \cx\Rightarrow 1 }
  }
  {[ \cx\mapsto 1]\vdash \eadd{\cx}{\cx}\Rightarrow 2 }
}
{ \emptyset\vdash \ebind{\cx}{1}{\eadd{\cx}{\cx}}\Rightarrow 2 }
\]

\section{Interpreter}

The following Scala code implements the abstract syntax of \lang:

\begin{verbatim}
sealed trait Expr
case class Num(n: Int) extends Expr
case class Add(l: Expr, r: Expr) extends Expr
case class Sub(l: Expr, r: Expr) extends Expr
case class Val(x: String, i: Expr, b: Expr) extends Expr
case class Id(x: String) extends Expr
\end{verbatim}

An identifier is an arbitrary string.
\code{Val($x$, $e_1$, $e_2$)} corresponds to $\ebind{x}{e_1}{e_2}$;
\code{Id($x$)} corresponds to $x$.

We use a map to represent an environment. The type of an environment is
\code{Map[String, Int]}.

\begin{verbatim}
type Env = Map[String, Int]
\end{verbatim}

We can add a pair of a key and a value to a map with the \code{+} operator.
For example,
where \code{m} is \code{Map(1 -> "one")}, \code{m + (2 -> "two")} is the same as
\code{Map(1 -> "one", 2 -> "two")}.

\begin{verbatim}
def interp(e: Expr, env: Env): Int = e match {
  case Num(n) => n
  case Add(l, r) => interp(l, env) + interp(r, env)
  case Sub(l, r) => interp(l, env) - interp(r, env)
  case Val(x, i, b) => interp(b, env + (x -> interp(i, env)))
  case Id(x) => env(x)
}
\end{verbatim}

Since the structure of the code is almost identical to the semantics rules, there
is nothing much to explain. In the \code{Id} case, when \code{x} is a key in
\code{env}, the corresponding value becomes the result of \code{interp}.
Otherwise, an exception is thrown, and the execution
terminates without producing any results.

\section{Exercises}

\begin{exercise}

For each of the following expression:
\begin{itemize}
  \item $\ebind{\cx}{(\ebind{\cx}{3}{\esub{5}{\cx}})}{\eadd{1}{\cx}}$
  \item $\ebind{\cx}{3}{\ebind{\cy}{5}{\eadd{1}{\cx}}}$
  \item $\ebind{\cx}{3}{\ebind{\cx}{5}{\eadd{1}{\cx}}}$
\end{itemize}
\begin{enumerate}
  \item Draw arrows from each bound occurrence to its binding occurrence.
  \item Draw dotted arrows from each shadowing variable to its shadowed variable.
\end{enumerate}

\end{exercise}

\begin{exercise}

Explain whether each of the following expressions is an example of shadowing:
\begin{enumerate}
  \item $\ebind{\cx}{(\ebind{\cx}{3}{\esub{5}{\cx}})}{\eadd{1}{\cx}}$
  \item $\ebind{\cx}{3}{\ebind{\cy}{5}{\eadd{1}{\cx}}}$
  \item $\ebind{\cx}{3}{\ebind{\cx}{5}{\eadd{1}{\cx}}}$
\end{enumerate}

\end{exercise}

\begin{exercise}
\labex{identifiers-shadowing-impl}

This exercise asks you to implement the \code{shadowing} function, which
takes a \lang expression as an argument and returns the set of every
identifier that becomes shadowed at least once in the expression. For
example, \code{shadowing($e$)} equals \code{Set("x")} where $e$ denotes
$\ebind{\cx}{\cy}{\ebind{\cx}{1}{\cz}}$.  The \code{shadowing} function
calls the \code{helper} function, which tracks the set of every identifier
defined already to detect shadowing.

\begin{verbatim}
def shadowing(e: Expr): Set[String] = helper(e, Set())

def helper(e: Expr, env: Set[String]): Set[String] =
  e match {
    case Num(n) => ???
    case Add(l, r) => ???
    case Id(x) => ???
    case Val(x, e, b) => ???
  }
\end{verbatim}

Fill every \code{???} to complete the implementation. You may refer to the following
to deal with sets:
\begin{itemize}
  \item \code{Set()} is the empty set.
  \item \code{Set(v)} is the set consisting of a single element \code{v}.
  \item Where \code{s} is a set, \code{s(v)} is \code{true} if \code{v} is an
    element of \code{s} and \code{false} otherwise.
  \item Where \code{s} is a set, \code{s + v} is the union of \code{s} and $\{\code{v}\}$.
  \item Where \code{s1} and \code{s2} are sets, \code{s1 ++ s2} is the union of \code{s1}
    and \code{s2}.
\end{itemize}

\end{exercise}
