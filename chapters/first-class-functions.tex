\setchapterpreamble[u]{\margintoc}
\chapter{First-Class Functions}
\labch{first-class-ftns}

It is the second article about functional programming. The last article discussed
immutability, recursion, functional lists, and tail call optimization. This
article focuses on functions. It defines a \term{first-class function} and a way
to use \term{anonymous functions} in Scala. Besides, it generalizes the functions
defined in the last article by using first-class functions. It also introduces
\term{option types}, which handle erroneous situations in a functional way.

\section{First-Class Functions}

An entity in a programming language is a first-class citizen if it satisfies the
following three conditions:

\begin{enumerate}
\item It can be an argument of a function call.
\item It can be a return value of a function.
\item A variable can refer to it.
\end{enumerate}

A first-class citizen can be used as a value. Functions are highly important and
treated as values in functional languages. Functions in Scala also are
first-class.

\begin{verbatim}
def f(x: Int): Int = x
def g(h: Int => Int): Int = h(0)
g(f)
\end{verbatim}

Function \verb!g! has one parameter \verb!h!. The type of \verb!h! is \verb!Int => Int!.
An argument passed to \verb!g! is a function that takes one integral
parameter and returns an integral value. In Scala, \verb!=>! expresses the types
of functions. Functions without parameters have types of form \verb!() => [return type]!.
\verb![parameter type] => [return type]! is the type of a function with a
single parameter. Round brackets express the types of functions with more than
one parameter: \verb!([parameter type], … ) => [return type]!. Function \verb!f!
has one integral parameter and returns an integer so can be an argument for
\verb!g!. Evaluating \verb!g(f)! equals to evaluating \verb!f(0)!, which results
in \verb!0!.

\begin{verbatim}
def f(y: Int): Int => Int = {
  def g(x: Int): Int = x
  g
}
f(0)(0)
\end{verbatim}

Function \verb!f! returns function \verb!g!. Since \verb!f! has return type
\verb!Int => Int!, a return value must be a function that takes an integer as a
parameter and returns an integer. \verb!g! satisfies the condition. \verb!f(0)!
is same as \verb!g! and therefore is a function. \verb!f(0)(0)! is \verb!g(0)!,
which returns \verb!0!.


\begin{verbatim}
val h0 = f(0)
h0(0)
\end{verbatim}

A variable can refer to \verb!f(0)!. \verb!h0! refers to the return value of
\verb!f(0)! and has type \verb!Int => Int!. Calling variables referring to
function values is possible. \verb!h0(0)! is a valid expression and results in
\verb!0!.

\begin{verbatim}
val h1 = f
         ^
error: missing argument list for method f

Unapplied methods are only converted to functions
when a function type is expected.

You can make this conversion explicit
by writing \verb!f _! or \verb!f(_)! instead of \verb!f!.
\end{verbatim}

On the other hand, defining a variable referring to \verb!f! results in a compile
error. In Scala, a function defined by \verb!def! is not a value. Since \verb!f!
is the name of a function but not a variable referring to a value, \verb!h1!
cannot refer to the value of \verb!f!. As the above error message implies,
underscores convert function names into function values.

\begin{verbatim}
val h1 = f _
h1(0)(0)
\end{verbatim}

Compiling the above code succeeds. The type of \verb!h1! is \verb!Int => (Int => Int)!.
\verb!Int => Int => Int! denotes the same type because function types are
right-associative. \verb!h1(0)(0)! is valid and yields \verb!0!.

Expressions preceding \verb!val h1 = f! use function names as values
successfully. Scala compilers transform function names into function values if
they occur where function types are expected. Therefore, enforcing the type of
\verb!h1! to be a function type corrects the code without underscores. The
following code works well:

\begin{verbatim}
val h1: Int => Int => Int = f
h1(0)(0)
\end{verbatim}

When programmers use function names as values, they usually place the names where
function types are expected. In these cases, underscores and explicit type
annotations are unnecessary. Rarely code becomes problematic and needs
modifications like the above to enforce the transformations.

How do the compilers create function values from function names? If the parameter
type of function \verb!f! is \verb!Int!, the corresponding function value is
\verb!(x: Int) => f(x)!. The transformation is called eta expansion. \verb!(x: Int) => f(x)!
is a function value without a name and does the same thing as
\verb!f!. The following section covers functions without names.

\subsection{Anonymous Functions}

In functional programming, functions often appear only once as an argument or a
return value. Naming the functions used only once is unnecessary. The meanings of
function values are how they act. The parameters and bodies of functions decide
the meanings, but the names do not have a role. Naturally, most functional
languages provide syntax to define functions without giving them names. Such
functions are anonymous functions.

Anonymous functions in Scala have form \verb!([parameter name]: [parameter type], …) => [expression]!.
Since curly brackets bundle multiple expressions and create
a single expression, the bodies of anonymous functions can have multiple
expressions: \verb!(…) => { … }!. Like functions declared by \verb!def!,
anonymous functions can be arguments, return values, or values referred by
variables. Directly calling them is possible as well.

\begin{verbatim}
((x: Int) => x)(0)
def g(h: Int => Int): Int = h(0)
g((x: Int) => x)
def f(): Int => Int = (x: Int) => x
val h = (x: Int) => x
\end{verbatim}

The code does similar things to the previous code but uses anonymous functions.

Anonymous functions need explicit parameter types as named functions do. However,
annotating every parameter type is verbose and inconvenient. Scala compilers
infer the types of parameters when anonymous functions occur where the compilers
expect function types.

\begin{verbatim}
def g(h: Int => Int): Int = h(0)
g(x => x)
\end{verbatim}

Since \verb!g! has a parameter of type \verb!Int => Int!, the compilers expect
\verb!x => x! to have type \verb!Int => Int!. They infer the type of \verb!x! as
\verb!Int!.

\begin{verbatim}
def f0(): Int => Int = x => x
def f1() = (x: Int) => x
def f2() = x => x  // a compile error
\end{verbatim}

\verb!f0! has the explicit return type. \verb!Int => Int! is the expected type of
\verb!x => x!. The compilers infer the type of \verb!x! as \verb!Int!. While
\verb!f1! does not result in a compile error, \verb!f2! is problematic. Since
\verb!(x: Int) => Int! specifies the parameter type, the compilers know that its
type is \verb!Int => Int!. In contrast, there is no information to infer the type
of \verb!x => x!, the body of \verb!f2!.

\begin{verbatim}
val h0: Int => Int = x => x
val h1 = (x: Int) => x
val h2 = x => x  // a compile error
\end{verbatim}

As shown in the previous example, \verb!h0! and \verb!h1! are valid, but
\verb!h2! is not.

Most cases using anonymous functions are arguments for function calls so that the
functions do not require explicit parameter types. However, beginners might not
be sure about whether omitting parameter types is allowed or not. Specifying
parameter types is safe when it is not assured.

Scala provides one more syntax for anonymous functions. The syntax uses
underscores. Underscores help programmers to define concise and intuitive
anonymous functions. They make code readable but can appear only in particular
situations. Every parameter occurs exactly once in the body of a function in the
order. Moreover, the function is not an identity function like \verb!(x: Int) => x!.
In such a function, underscores can replace parameters in the body.
Otherwise, it is impossible to use underscores to define anonymous functions.

\begin{verbatim}
def g0(h: Int => Int): Int = h(0)
g0(_ + 1)

def g1(h: (Int, Int) => Int): Int = h(0, 0)
g1(_ + _)
\end{verbatim}

The compilers \term{desugar} \verb!_ + 1! and obtain \verb!x => x + 1!. \verb!_ + _!
becomes \verb!(x, y) => x + y!. The compilers automatically create parameters
as many as underscores and substitute the underscores with the parameters. The
mechanism clearly shows why the restriction exists.

\begin{verbatim}
val h0 = (_: Int) + 1
val h1 = (_: Int) + (_: Int)
\end{verbatim}

Underscores can have explicit types.

In the code, the compilers cannot infer parameter types and therefore require
explicit parameter types to succeed compiling.

The transformation happens for the shortest expression containing underscores.
Expressing anonymous functions with complex bodies using underscores is tricky.

\begin{verbatim}
def f(x: Int): Int = x
def g1(h: Int => Int): Int = h(0)
def g2(h: (Int, Int) => Int): Int = h(0, 0)
g(f(_))
g1(f(_ + 1))  // a compile error
g2(f(_) + _)
g2(f(_ + 1) + _)  // a compile error
\end{verbatim}

As intended, \verb!f(_)! becomes \verb!x => f(x)!, and \verb!f(_) + _! becomes
\verb!(x, y) => f(x) + y!. (Actually, there is no need to write \verb!g(f(_))!
because it is equal to \verb!g(f)!.) On the other hand, \verb!f(_ + 1)! becomes
\verb!f(x => x + 1)! but not \verb!x => f(x + 1)!. \verb!f(_ + 1) + _! becomes
\verb!y => f(x => x + 1) + y! but not \verb!(x, y) => f(x + 1) +y!.

Like type inference of parameter types, novices may not be sure about how
anonymous functions with underscores change. I recommend using normal anonymous
functions without underscores for those who are not confident about the mechanism
of underscores.

\subsection{Closures}

Functions are values in some non-functional languages also. For example, C
provides \term{function pointers}. Like other pointers, function pointers can be
arguments, return values, and referred by variables.

\begin{verbatim}
int f(int x) {
    return x;
}

int g(int (*h)(int)) {
    return h(0);
}

int (*i())(int) {
    return f;
}

int main() {
    g(f);
    int (*h)(int) = f;
    i()(0);
}
\end{verbatim}

Functions in functional languages are much more expressive than function pointers
in C. Functions are \term{closures} in functional languages, but function
pointers are not closure. Closures are function values that capture
\term{environments}---states---when they are defined. The bodies of closures may
have \term{free variables}, and their environments stores values referred by the
free variables.

\begin{verbatim}
def makeAdder(x: Int): Int => Int = {
  def adder(y: Int): Int = x + y
  adder
}
\end{verbatim}

The definition of \verb!adder!, \verb!def adder(y: Int): Int = x + y!, does not
bind \verb!x!. \verb!x! is a free variable. However, the code is correct.

\begin{verbatim}
val add1 = makeAdder(1)
add1(2)
val add2 = makeAdder(2)
add2(2)
\end{verbatim}

\verb!add1! and \verb!add2! refer to the same \verb!adder! function, but the
former returns an integer one larger than an argument, and the latter returns an
integer two larger than an argument. The results of \verb!add(1)! and
\verb!add(2)! are \verb!3! and \verb!4!, respectively. It is possible because the
closures capture the environments when they are created. \verb!add1! refers to a
thing like \verb!(adder, x = 1)! instead of simple \verb!adder!. Similarly,
\verb!add2! is actually \verb!(adder, x = 2)!. Since the environment of
\verb!add1! stores the fact that \verb!x! is \verb!1!, \verb!add1(2)! results in
\verb!3!. Under the environment of \verb!add2!, \verb!x! denotes \verb!2!, and
thus \verb!x + y!, or add2(2)\verb!, is !4`.

Function pointers in C do not allow such application. However, one can simulate
closures using function pointers. It requires some efforts.

\begin{verbatim}
#include <stdio.h>
#include <stdlib.h>

struct closure {
    int (\term{f)(struct closure }, int);
    void *env;
};

int call_closure(struct closure *c, int arg0) {
    return c->f(c, arg0);
}

int adder(struct closure *c, int y) {
    return ((int *) (c->env))[0] + y;
}

struct closure *makeAdder(int x) {
    struct closure *c = malloc(sizeof(struct closure));
    c->f = adder;
    c->env = malloc(sizeof(int));
    ((int *) (c->env))[0] = x;
    return c;
}

int main() {
    struct closure *add1 = makeAdder(1);
    struct closure *add2 = makeAdder(2);

    int n1 = call_closure(add1, 2);
    int n2 = call_closure(add2, 2);

    printf("%d %d\n", n1, n2);  // 3 4
}
\end{verbatim}

In the course, students learn how to implement interpreters for languages with
closures. Implementing interpreter helps to understand the concept of a closure,
which captures an environment.

\section{First-Class Functions and Lists}

The section shows how first-class functions allow generalization of the functions
defined in the last article.

\subsection{map}

\begin{verbatim}
def inc1(l: List): List = l match {
  case Nil => Nil
  case Cons(h, t) => Cons(h + 1, inc1(t))
}

def square(l: List): List = l match {
  case Nil => Nil
  case Cons(h, t) => Cons(h * h, square(t))
}
\end{verbatim}

\verb!inc1! increases every element of a given list by one, and \verb!square!
squares every element. The two functions are remarkably similar. To make the
similarity clearer, let us rename the functions to \verb!g!.

\begin{verbatim}
def g(l: List): List = l match {
  case Nil => Nil
  case Cons(h, t) => Cons(h + 1, g(t))
}

def g(l: List): List = l match {
  case Nil => Nil
  case Cons(h, t) => Cons(h * h, g(t))
}
\end{verbatim}

The only difference is the first argument of \verb!Cons! in the third line:
\verb!h + 1! versus \verb!h * h!. By adding one parameter, the functions become
entirely identical.

\begin{verbatim}
def g(l: List, f: Int => Int): List = l match {
  case Nil => Nil
  case Cons(h, t) => Cons(f(h), g(t, f))
}
g(l, h => h + 1)

def g(l: List, f: Int => Int): List = l match {
  case Nil => Nil
  case Cons(h, t) => Cons(f(h), g(t, f))
}
g(l, h => h * h)
\end{verbatim}

In the article, I call the function \verb!list_map!. An argument and the return
value have elements *\term{map}*ped by a given function.

\begin{verbatim}
def list_map(l: List, f: Int => Int): List = l match {
  case Nil => Nil
  case Cons(h, t) => Cons(f(h), list_map(t, f))
}
\end{verbatim}

\verb!inc1! and \verb!square! can be redefined using \verb!list_map!.

\begin{verbatim}
def inc1(l: List): List = list_map(l, h => h + 1)
def square(l: List): List = list_map(l, h => h * h)
\end{verbatim}

An underscore makes \verb!inc1! concise.

\begin{verbatim}
def inc1(l: List): List = list_map(l, _ + 1)
\end{verbatim}

Implement the \verb!incBy! function, which takes a list and an integer as
arguments and increases every element of the list by the given integer. Use
\verb!list_map!.

\begin{verbatim}
def incBy(l: List, n: Int): List = list_map(l, h => h + n)
def incBy(l: List, n: Int): List = list_map(l, _ + n)
\end{verbatim}

\subsection{filter}

Let us compare \verb!odd! and \verb!positive!.

\begin{verbatim}
def odd(l: List): List = l match {
  case Nil => Nil
  case Cons(h, t) =>
    if (h % 2 != 0) Cons(h, odd(t))
    else odd(t)
}

def positive(l: List): List = l match {
  case Nil => Nil
  case Cons(h, t) =>
    if (h > 0) Cons(h, positive(t))
    else positive(t)
}
\end{verbatim}

They look similar. Rename the functions and add a parameter. The functions become
identical. I call the function \verb!list_filter!. The function *\term{filter}*s
unwanted elements in an argument.

\begin{verbatim}
def list_filter(l: List, f: Int => Boolean): List = l match {
  case Nil => Nil
  case Cons(h, t) =>
    if (f(h)) Cons(h, list_filter(t, f))
    else list_filter(t, f)
}
\end{verbatim}

\verb!odd! and \verb!positive! can be redefined using \verb!list_filter!.

\begin{verbatim}
def odd(l: List): List = list_filter(l, h => h % 2 != 0)
def positive(l: List): List = list_filter(l, h => h > 0)
\end{verbatim}

Underscores make the functions concise.

\begin{verbatim}
def odd(l: List): List = list_filter(l, _ % 2 != 0)
def positive(l: List): List = list_filter(l, _ > 0)
\end{verbatim}

Implement the \verb!gt! function, which takes a list and an integer as arguments
and filters elements less than or equal to the given integer out from the list.
Use \verb!list_filter!.

\begin{verbatim}
def gt(l: List, n: Int): List = list_filter(l, h => h > n)
def gt(l: List, n: Int): List = list_filter(l, _ > n)
\end{verbatim}

\subsection{foldRight}

Let us compare \verb!sum! and \verb!product! without tail recursion.

\begin{verbatim}
def sum(l: List): Int = l match {
  case Nil => 0
  case Cons(h, t) => h + sum(t)
}

def product(l: List): Int = l match {
  case Nil => 1
  case Cons(h, t) => h * product(t)
}
\end{verbatim}

After renaming the names to \verb!g!, two differences exist: \verb!0! versus
\verb!1! and \verb!h + g(t)! versus \verb!h * g(t)!. By adding two parameters, an
initial value and a function taking \verb!h! and \verb!g(t)! as arguments, the
functions become identical. I call the function \verb!list_foldRight!. The
function appends an initial value at the *\term{right}\term{ side of a list and
}\term{fold}\term{s the list from the }\term{right}* side using a given function.

\begin{verbatim}
def list_foldRight(l: List, n: Int, f: (Int, Int) => Int): Int = l match {
  case Nil => n
  case Cons(h, t) => f(h, list_foldRight(t, n, f))
}
\end{verbatim}

\verb!sum! and \verb!product! can be redefined using \verb!list_foldRight!.

\begin{verbatim}
def sum(l: List): Int = list_foldRight(l, 0, (h, gt) => h + gt)
def product(l: List): Int = list_foldRight(l, 1, (h, gt) => h * gt)
\end{verbatim}

They may use underscores for conciseness.

\begin{verbatim}
def sum(l: List): Int = list_foldRight(l, 0, _ + _)
def product(l: List): Int = list_foldRight(l, 1, _ * _)
\end{verbatim}

The following gives an intuitive interpretation of the function:

\begin{verbatim}
  list_foldRight(List(a, b, .., y, z), n, f)
= f(a, f(b, .. f(y, f(z, n)) .. ))

  list_foldRight(List(1, 2, 3), 0, +)
= +(1, +(2, +(3, 0)))

  list_foldRight(List(1, 2, 3), 1, *)
= \term{(1, }(2, *(3, 1)))
\end{verbatim}

Implement \verb!length! with \verb!list_foldRight!.

\begin{verbatim}
def length(l: List): List = list_foldRight(l, 0, (h, gt) => 1 + gt)
\end{verbatim}

\subsection{foldLeft}

Let us compare tail-recursive \verb!sum! and \verb!product!.

\begin{verbatim}
def sum(l: List): Int = {
  @tailrec def aux(l: List, inter: Int): Int = l match {
    case Nil => inter
    case Cons(h, t) => aux(t, inter + h)
  }
  aux(l, 0)
}

def product(l: List): Int = l match {
  @tailrec def aux(l: List, inter: Int): Int = l match {
    case Nil => inter
    case Cons(h, t) => aux(t, inter * h)
  }
  aux(l, 1)
}
\end{verbatim}

After renaming, there are two differences: \verb!inter + h! versus \verb!inter * h!
and \verb!0! versus \verb!1!. Similarly, adding two parameters makes the
functions identical.

\begin{verbatim}
def list_foldLeft(l: List, n: Int, f: (Int, Int) => Int): Int = {
  @tailrec def aux(l: List, inter: Int): Int = l match {
    case Nil => inter
    case Cons(h, t) => aux(t, f(inter, h))
  }
  aux(l, n)
}
\end{verbatim}

I call the function \verb!list_foldLeft!. Its semantics is different from
\verb!list_foldRight!. While \verb!list_foldRight! appends an initial value at
the right side and folds a list from the right side, \verb!list_foldLeft!
prepends an initial value at the *\term{left}\term{ side and }\term{fold}\term{s
a list from the }\term{left}* side. The following gives an intuitive
interpretation:

\begin{verbatim}
  list_foldLeft(List(a, b, .., y, z), n, f)
= f(f( .. f(f(n, a), b), .. , y), z)

  list_foldLeft(List(1, 2, 3), 0, +)
= +(+(+(0, 1), 2), 3)

  list_foldLeft(List(1, 2, 3), 1, *)
= \term{(}(*(1, 1), 2), 3)
\end{verbatim}

The order traversing a list does not affect the results of \verb!sum!,
\verb!product!, and \verb!length!. Both \verb!list_foldRight! and
\verb!list_foldLeft! can express the functions.

\begin{verbatim}
def sum(l: List): Int = list_foldLeft(l, 0, _ + _)
def product(l: List): Int = list_foldLeft(l, 1, _ * _)
def length(l: List): Int = list_foldLeft(l, 0, (inter, h) => inter + 1)
\end{verbatim}

On the other hand, the order is important for function such as \verb!addBack! and
\verb!reverse!. Using one of \verb!list_foldRight! and \verb!list_foldLeft! is
more efficient than using the other. \verb!list_foldRight! fits \verb!addBack!
and \verb!list_foldLeft! fits \verb!reverse!. (The following code is incorrect
because of types. Consider it as a conceptual example.)

\begin{verbatim}
def addBack(l: List, n: Int): List =
  list_foldRight(l, Cons(n, Nil), (h, gt) => Cons(h, gt))
def addBack(l: List, n: Int): List =
  list_foldRight(l, Cons(n, Nil), Cons)
def reverse(l: List): List =
  list_foldLeft(l, Nil, (inter, h) => Cons(h, inter))
\end{verbatim}

\verb!list_map!, \verb!list_filter!, \verb!list_foldRight!, and
\verb!list_foldLeft! are powerful functions. The four functions offer concise
implementation for most procedures dealing with lists. In most functional
languages, libraries provide functions similar to the four functions. The
\verb!List! class of the Scala standard library defines \verb!map!,
\verb!filter!, \verb!foldRight!, and \verb!foldLeft! methods. The next article
introduces the \verb!List! class and its methods.

\section{Option Types}

Consider the \verb!list_get! function, which takes a list and integer \verb!n! as
arguments and returns the \verb!n!th element of the list. The case when \verb!n!
is negative or exceeds the length of a list is troublesome. Throwing exceptions
is a widely used solution in imperative languages.

\begin{verbatim}
def list_get(l: List, n: Int): Int =
  if (n < 0)
    throw new Exception("Negative index.")
  else l match {
    case Nil =>
      throw new Exception("Index out of bound.")
    case Cons(h, t) =>
      if (n == 0) h else list_get(t, n - 1)
  }
\end{verbatim}

It is simple and effective, but the program terminates abnormally if
\term{exception handlers} do not exist at call sites of the function. Most type
systems do not check exceptions. They do not enforce programmers to handle
exceptions. For this reason, Java has introduced \term{checked exceptions}, whom
compilers check. However, programmers usually do not adequately handle exceptions
but use only conventional \verb!try-catch! statements to make programs pass type
checking. Many people have criticized the concept of a checked exception. Another
problem of throwing exceptions is that exception handling is not local.
Exceptions spread through function call stacks so that the \term{control flow} of
programs suddenly jumps to the positions of exception handlers. It disturbs
programmers to understand code. Implementing \verb!list_get! without exceptions
is desirable.

\begin{verbatim}
def list_get(l: List, n: Int): Int =
  if (n < 0) null
  else l match {
    case Nil => null
    case Cons(h, t) =>
      if (n == 0) h else list_get(t, n - 1)
  }
\end{verbatim}

The first attempt is using \verb!null!. \verb!null! is a value that denotes that
it does not refer to any existing object. In Java and therefore Scala, \verb!Int!
is a primitive type, and \verb!null! is not an element of \verb!Int!. The code is
invalid. Even though we assume that \verb!null! belongs to \verb!Int!, without
checking whether a return value is \verb!null!, using the return value may lead
to \verb!NullPointerException!. Like exceptions, \verb!null! is beyond the scopes
of type systems of Java and Scala. Some modern languages including Kotlin have
introduced \term{nullable types} and \term{non-null types} to make programs safe
from \verb!NullPointerException!. The concept of a nullable type is similar to an
option type, the subject of the section.

\begin{verbatim}
def list_get(l: List, n: Int): Int =
  if (n < 0) -1
  else l match {
    case Nil => -1
    case Cons(h, t) =>
      if (n == 0) h else list_get(t, n - 1)
  }
\end{verbatim}

The second attempt is using a particular error-indicating value, \verb!-1! for
example. The strategy has an obvious problem: at call-sites, programs cannot
judge whether lists contain \verb!-1! as elements or indices are wrong. It can be
successful for certain purposes but does not fit the \verb!list_get! function in
general.

\begin{verbatim}
def list_getOrElse(l: List, n: Int, default: Int): Int =
  if (n < 0) default
  else l match {
    case Nil => default
    case Cons(h, t) =>
      if (n == 0) h else list_getOrElse(t, n - 1, default)
}
\end{verbatim}

Instead of using a fixed particular value, specifying default values for failures
at call sites is possible. It works well when an appropriate default value
exists. However, when checking failures is per se important, the new strategy is
as bad as the previous strategy. There is no way to distinguish an element and a
default value.

Functional languages provide option types to handle erroneous situations safely
without any mutation. Many languages including Scala use \verb!Option! as the
name, but some call them \verb!Maybe!. As the name implies, an option type
represents an optional existence of a value. The article defines an option type
for \verb!Int! and functions treating values of the type.

\begin{verbatim}
trait Option
case object None extends Option
case class Some(n: Int) extends Option
\end{verbatim}

The code is similar to the code defining \verb!List!, \verb!Nil!, and
\verb!Cons!. A value of the \verb!Option! type either is \verb!None! or belongs
to \verb!Some!. \verb!None! is a value that does not denote any value and similar
to \verb!null!. It indicates a problematic situation. Like \verb!Nil!, it is a
singleton object defined by \verb!object!. \verb!Some! constructs a value that
denotes that a value exists. It is similar to a reference to a real object and
indicates that computation succeeds without exceptions.

The following defines \verb!list_getOption! using the \verb!Option! type:

\begin{verbatim}
def list_getOption(l: List, n: Int): Option =
  if (n < 0) None
  else l match {
    case Nil => None
    case Cons(h, t) =>
      if (n == 0) Some(h) else list_getOption(t, n - 1)
  }
\end{verbatim}

For wrong indices, the return value is \verb!None!. Otherwise, the function packs
an element inside \verb!Some! to make the return value.

Define the \verb!div100! function, which takes an integer as an argument and
returns an optional value to handle a division by zero safely.

\begin{verbatim}
def div100(n: Int): Option =
  if (n == 0) None else Some(100 / n)
\end{verbatim}

Pattern matching allows programmers to deal with optional values by
distinguishing the \verb!None! and the \verb!Some! cases. Like the functions for
lists, common patterns handling optional values exist. It is desirable to define
functions generalizing the patterns.

\begin{verbatim}
def option_map(opt: Option, f: Int => Int): Option = opt match {
  case None => None
  case Some(n) => Some(f(n))
}
\end{verbatim}

Consider computation that might fail. After the computation, one wants to do
additional computation only if the computation has succeeded and to do nothing
otherwise. The situation is the typical usage of the \verb!option_map! function.
It takes an optional value and a function as arguments and applies the function
to a value wrapped in the optional value only if the value belongs to
\verb!Some!.

Define the \verb!getSquare! function, which takes a list and integer \verb!n! as
arguments and returns the square of the \verb!n!th element of the list. The
return type of the function must be \verb!Option!. Use \verb!list_getOption! and
\verb!option_map!.

\begin{verbatim}
def getSquare(l: List, n: Int): Option =
  option_map(list_getOption(l, n), n => n * n)
\end{verbatim}

In this time, consider a situation that additional computation also can fail. It
is the place for the \verb!option_flatMap! function. It takes a function whose
return type is \verb!Option! as the second argument. If every computation
succeeds, the result belongs to \verb!Some!. Otherwise, it is \verb!None!.

Define the \verb!getAndDiv100! function, which takes a list and integer \verb!n!
as arguments and returns an integer obtained by dividing \verb!100! by the
\verb!n!th element of the list. The function must return an optional value. Use
\verb!list_getOption!, \verb!div100!, and \verb!option_flatMap!.

\begin{verbatim}
def getAndDiv100(l: List, n: Int): Option =
  option_flatMap(list_getOption(l, n), div100)
\end{verbatim}

Option types are powerful tools to handle erroneous cases in a functional way.
Functional programming uses lists, first-class functions, anonymous functions,
and optional values a lot.
