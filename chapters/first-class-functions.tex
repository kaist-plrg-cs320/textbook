\setchapterpreamble[u]{\margintoc}
\chapter{First-Class Functions}
\labch{first-class-functions}

\renewcommand{\plang}{\textsf{VAE}\xspace}
\renewcommand{\lang}{\textsf{FVAE}\xspace}

\textit{First-class functions}\index{first-class function} are functions that
can be used as values. They are much more expressive than first-order functions,
which are the topic of the previous chapter. This chapter explains the semantics
of first-class functions. We need to introduce the notion of a closure to make
first-class functions work properly. We will see what closures are and why they
are necessary.

This chapter defines \lang by extending \plang with first-class functions.
The only way to create a function in \lang is to make an \textit{anonymous
function}\index{anonymous function}, which is a function without a name.
However, we can add named functions as syntactic sugar. In addition,
we will see that even variable definitions can be considered as syntactic sugar.

\section{Syntax}

Consider an anonymous function in Scala:

\begin{verbatim}
(x: Int) => x + x
\end{verbatim}

If we ignore its type annotation, it consists of two parts: \code{x} and \code{x
+ x}. \code{x} is the parameter of the function; \code{x + x} is the body of the
function. This observation lets us know that an anonymous function consists of
its name and parameter.

In \textsf{F1VAE}, the sytnax of a function call is $\eappfo{x}{e}$. To call a
function, the name of the function should be given. However, it is not true in
languages with first-class functions. Let us see some function calls in Scala.

\begin{verbatim}
def twice(x: Int): Int = x + x
twice(1)
\end{verbatim}

\code{twice(1)} is a function call, and it designates a function by its name.

\begin{verbatim}
def makeAdder(x: Int): Int => Int =
  (y: Int) => x + y
makeAdder(3)(5)
\end{verbatim}

\code{makeAdder} is a function that returns a function. \code{makeAdder(3)} is a
function call, and its result is a function. Therefore, we can call the
resulting function again. \code{makeAdder(3)(5)} is an expression that calls
\code{makeAdder(3)}. It designates a function by an expression, rather than just
a name. We can conclude that the syntax of a function call in \lang should be
more general than \textsf{F1VAE} because of the presence of first-class
functions. In \lang, a function call consists of two expressions: one determines
the function to be called and the other determines the value of the argument.

We have used the term function call so far. In the context of functional
programming, we use the term \textit{function application}\index{function
application} more frequently. When we see \code{f(1)}, we say ``\code{f} is
applied to \code{1}'' instead of ``\code{f} is called with the argument
\code{1}.'' Applications sound more natural than calls especially when we are
talking about first-class functions. For example, we usually say
``\code{makeAdder(3)} is applied to \code{5}'' rather than ``\code{makeAdder(3)}
is called with the argument \code{5}.''

From the above observation on anonymous functions and function applications,
we can define the syntax of \lang. The following is the syntax of \lang:
\sidenote{We omit the common part to \plang.}

\[ e\ ::=\ \cdots\ |\ \efun{x}{e}\ |\ \eapp{e}{e} \]

\begin{itemize}
  \item $\efun{x}{e}$

    It is called an anonymous function or a \textit{lambda abstraction}.
    \index{lambda abstraction}
    It denotes a function whose parameter is $x$ and body is $e$. $x$ is a
    binding occurrence, and its scope is $e$.
    A function has zero or more parameters in many real-world languages,
    but we restrict a function in \lang to have one and only one parameter for
    simplicity as before.

  \item $\eapp{e_1}{e_2}$

    It is a function application, or just an application in short.
    $e_1$ denotes the function; $e_2$ denotes the argument.
\end{itemize}

\section{Semantics}

Integers are the only values in \plang. It is not true in \lang. Since
first-class functions are values, a value of \lang is either an integer or a
function. Thus, we define a new kind of semantic element,
\textit{value}\index{value}. The
metavariable $v$ ranges over values. Also, let $V$ be the set of every value.

\[ v\ ::=\ n\ |\ \clov{x}{e}{\sigma} \]

A value is either an integer or a closure. A \textit{closure}\index{closure}, which is a
function as a value, has the form $\clov{x}{e}{\sigma}$.
It is a pair of a lambda abstraction and an environment.
A lambda abstraction in a closure may have free identifiers,
but the environment of the closure can store the values denoted by the
free identifiers.

To discuss the necessity of closures, consider the following expression:

\[\eapp{\eapp{(\efun{\cx}{\efun{\cy}{\eadd{\cx}{\cy}}})}{1}}{2}\]

It is equivalent to \code{((x: Int) => (y: Int) => x + y)(1)(2)} in Scala.
The function $\efun{\cx}{\efun{\cy}{\eadd{\cx}{\cy}}}$ does not have any free
identifiers. The scope of \code{x} is ${\efun{\cy}{\eadd{\cx}{\cy}}}$; the scope
of \code{y} is ${\eadd{\cx}{\cy}}$. Therefore, both \code{x} and \code{y} in
$\eadd{\cx}{\cy}$ are bound occurrences. The whole expression must result in $3$,
which equals $1+2$, without a run-time error. You can check it by running the
equivalent Scala program.

If we consider a function value as just a lambda abstraction, not a closure,
evaluation of the above expression becomes problematic.
When the expression is evaluated, $\efun{\cx}{\efun{\cy}{\eadd{\cx}{\cy}}}$ is
applied to $1$ first. The result is a function value, which is a lambda
abstraction: $\efun{\cy}{\eadd{\cx}{\cy}}$. Next, $\efun{\cy}{\eadd{\cx}{\cy}}$
is applied to $2$. The result of the application can be computed by evaluating
$\eadd{\cx}{\cy}$ under the environment containing that $\cy$ denotes $2$.
However, there is no way to find the value of $\cx$. $\cx$ has become free
identifier although it was not in the beginning.

We adopt the notion of a closure to resolve the problem. When a lambda
expression evaluates to a function value, which is a closure, it captures the
environment. Since $\efun{\cy}{\eadd{\cx}{\cy}}$ is evaluated under the
environment containing that $\cx$ denotes $1$, its result is
$\clov{\cy}{\eadd{\cx}{\cy}}{[\cx\mapsto1]}$. The captured environment of the
closure records that $\cx$ is not a free identifier and denotes $1$.
When the closure is applied to $2$, its body ${\eadd{\cx}{\cy}}$ is evaluated
under $[\cx\mapsto1,\cy\mapsto2]$, not $[\cy\mapsto2]$. The addition
successfully results in $3$.

In summary, we need closures to retain the static scope semantics. A first-class
function can be passed as a value and thus applied to an argument at a different
place from where it has been defined. However, the environments used for the
evaluation of their bodies must be determined statically. In other words,
the denotation of identifiers in the bodies of functions must be determined when
the functions are defined, not used. Therefore, each closure captures the surrounding
environment when it is created.

Now, let us define the semantics of \lang. Most things are the same as the
semantics of \plang, but we should be aware of that values now include not only
integers but also closures.

An environment is a finite partial function from identifiers to values.

\[ \embox{Env}=\embox{Id}\finto V \]

The semantics of \lang is a ternay relation over $\embox{Env}$, $E$, and $V$.

\[\Rightarrow\subseteq\embox{Env}\times E\times V\]

$\evald{e}{v}$ is true if and only if $e$ evaluates to $v$ under $\sigma$.

A lambda abstraction creates a closure containing the current environment.

\semanticrule{Fun}{
  \evaldn{\efun{x}{e}}{\clov{x}{e}{\sigma}}.
}

\vspace{-1em}

\[
  \evald{\efun{x}{e}}{\clov{x}{e}{\sigma}}
  \quad\textsc{[Fun]}
\]

A function application evaluates its both subexpressions. Then, it evaluates the
body of the closure under the environment obtained by adding the value of the
argument to the environment of the closure.

\semanticrule{App}{
  \begin{tabular}{@{\hskip0pt}l@{\hskip0pt}l}
    If \\
    & \evaldn{e_1}{\clov{x}{e}{\sigma'}}, \\
    & \evaldn{e_2}{v'}, and \\
    & \evaln{\sigma'[x\mapsto v']}{e}{v}, \\
    then \\
    & \evaldn{\eapp{e_1}{e_2}}{v}.
  \end{tabular}
}

\vspace{-1em}

\[
  \inferrule
  {
    \evald{e_1}{\clov{x}{e}{\sigma'}} \\
    \evald{e_2}{v'} \\
    \eval{\sigma'[x\mapsto v']}{e}{v}
  }
  { \evald{\eapp{e_1}{e_2}}{v} }
  \quad\textsc{[App]}
\]

We can reuse Rule \textsc{Num}, Rule \textsc{Add}, Rule \textsc{Sub}, and Rule \textsc{Id} of
\plang. However, it is important to note that \lang has more
cases that evaluation can fail than \plang. For example, consider Rule \textsc{Add}.

\semanticrule{Add}{
If
  \evaldn{e_1}{n_1}, and
  \evaldn{e_2}{n_2},\\
then
  \evaldn{\eadd{e_1}{e_2}}{n_1+n_2}.
}

\vspace{-1em}

\[
  \inferrule
  {
    \evald{e_1}{n_1} \\
    \evald{e_2}{n_2}
  }
  { \evald{\eadd{e_1}{e_2}}{n_1+n_2} }
  \quad\textsc{[Add]}
\]

The rule assumes the results of $e_1$ and $e_2$ to be integers. If the
assumption is violated, a run-time error happens. For example,
$\eadd{(\efun{\cx}{\cx})}{1}$ incurs a run-time error becuase the left operand
is a closure, not an integer.

We need to revise Rule \textsc{Val} of \plang a bit.
Since every value is an integer in \plang, a variable of \plang can denote only
an integer. In \lang, a variable should be able to denote a general value, not
only an integer.

\semanticrule{Val}{
If
  \evaldn{e_1}{v_1}, and
  \evaln{\sigma[x\mapsto v_1]}{e_2}{v_2},\\
then
  \evaldn{\ebind{x}{e_1}{e_2}}{v_2}.
}

\vspace{-1em}

\[
  \inferrule
  {
    \evald{e_1}{v_1} \\
    \eval{\sigma[x\mapsto v_1]}{e_2}{v_2}
  }
  { \evald{\ebind{x}{e_1}{e_2}}{v_2} }
  \quad\textsc{[Val]}
\]

Now, a variable can denote a value, not only an integer.

The following proof trees prove that
$\eapp{\eapp{(\efun{\cx}{\efun{\cy}{\eadd{\cx}{\cy}}})}{1}}{2}$
evaluates to $3$ under the empty environment.
The proof splits into three trees for readability.
Suppose that $\sigma_1=[\cx\mapsto1]$ and $\sigma_2=[\cx\mapsto1,\cy\mapsto2]$.

\[
  \inferrule
  {
    \eval{\emptyset}{
      \efun{\cx}{\efun{\cy}{\eadd{\cx}{\cy}}}
    }{\clov{\cx}{\efun{\cy}{\eadd{\cx}{\cy}}}{\emptyset}}
    \\
    \eval{\emptyset}{1}{1}
    \\
    \eval{\sigma_1}{
      \efun{\cy}{\eadd{\cx}{\cy}}
    }{\clov{\cy}{\eadd{\cx}{\cy}}{\sigma_1}}
  }
  { \eval{\emptyset}{
      \eapp{(\efun{\cx}{\efun{\cy}{\eadd{\cx}{\cy}}})}{1}
    }{\clov{\cy}{\eadd{\cx}{\cy}}{\sigma_1}}
  }
\]

\[
  \inferrule
  {
    \inferrule
    { \cx\in\dom{\sigma_2} }
    { \eval{\sigma_2}{\cx}{1} }
    \\
    \inferrule
    { \cy\in\dom{\sigma_2} }
    { \eval{\sigma_2}{\cy}{2} }
  }
  { \eval{\sigma_2}{\eadd{\cx}{\cy}}{3} }
\]

\[
\inferrule
{
  \eval{\emptyset}{
      \eapp{(\efun{\cx}{\efun{\cy}{\eadd{\cx}{\cy}}})}{1}
    }{\clov{\cy}{\eadd{\cx}{\cy}}{\sigma_1}}
  \\
  \emptyset\vdash2\Rightarrow 2
  \\
  \eval{\sigma_2}{\eadd{\cx}{\cy}}{3}
}
{ \eval{\emptyset}{
    \eapp{\eapp{(\efun{\cx}{\efun{\cy}{\eadd{\cx}{\cy}}})}{1}}{2}
  }{3}
}
\]

\section{Interpreter}

The following Scala code implements the syntax of \lang:
\sidenote{We omit the common part to \plang.}

\begin{verbatim}
sealed trait Expr
...
case class Fun(x: String, b: Expr) extends Expr
case class App(f: Expr, a: Expr) extends Expr
\end{verbatim}

\code{Fun($x$, $e$)} represents $\efun{x}{e}$; \code{App($e_1$, $e_2$)}
represents $\eapp{e_1}{e_2}$.

A value of \lang is either an integer or a closure. Thus, we represent a value
as an ADT.

\begin{verbatim}
sealed trait Value
case class NumV(n: Int) extends Value
case class CloV(p: String, b: Expr, e: Env) extends Value
\end{verbatim}

\code{NumV($n$)} represents $n$; \code{CloV($x$, $e$, $\sigma$)} represents
$\clov{x}{e}{\sigma}$.

An environment is a finite partial function from identifiers to values.
Therefore, the type of an environment is \code{Map[String, Value]}.

\begin{verbatim}
type Env = Map[String, Value]
\end{verbatim}

The following function evaluates a given expression under a given environment:

\begin{verbatim}
def interp(e: Expr, env: Env): Value = e match {
  case Num(n) => NumV(n)
  case Add(l, r) =>
    val NumV(n) = interp(l, env)
    val NumV(m) = interp(r, env)
    NumV(n + m)
  case Sub(l, r) =>
    val NumV(n) = interp(l, env)
    val NumV(m) = interp(r, env)
    NumV(n - m)
  case Id(x) => env(x)
  case Fun(x, b) => CloV(x, b, env)
  case App(f, a) =>
    val CloV(x, b, fEnv) = interp(f, env)
    interp(b, fEnv + (x -> interp(a, env)))
}
\end{verbatim}

In the \code{Num} case, the return value is \code{NumV(n)}, not \code{n},
since the function must return a value of the type \code{Value}.

In the
\code{Add} and \code{Sub} cases, we cannot assume that the operands are integers
any longer. We use pattern matching to discriminate integers from closures. If
both operands are integers, addition or subtraction succeeds. Otherwise, at
least one of them is a closure, and the interpreter crashes due to a pattern
matching failure. Note that this code is equivalent to the following code:

\begin{verbatim}
case Add(l, r) =>
  interp(l, env) match {
    case NumV(n) => interp(r, env) match {
      case NumV(m) => NumV(n + m)
      case _ => error("not an integer")
    }
    case _ => error("not an integer")
  }
\end{verbatim}

Similarly, in the \code{App} case, we use pattern matching to discriminate
closures from integers. The first expression of \code{App} must yield a clsoure,
not an integer, to make the execution succeed.

\section{Syntactic Sugar}
\labsec{fae}

We can add named local functions to \lang with the following change in the
syntax:

\[
  e\ ::=\ \cdots\ |\ \erec{x}{x}{e}{e}
\]

$\erec{x_1}{x_2}{e_1}{e_2}$ defines a function whose name is $x_1$, parameter is
$x_2$, and body is $e_1$. The scope of $x_1$ is $e_2$, and thus the function
does not allow recursion.

Instead of changing the semantics, \lang can provide named local functions as
syntactic sugar. Let $s$ be a string transformed into $\erec{x_1}{x_2}{e_1}{e_2}$
by the parser of \lang with named local functions embeded in the semantics.
To treat named local functions as syntactic sugar, the parser should transform
$s$ into $\ebind{x_1}{\efun{x_2}{e_1}}{e_2}$.

Variable definitions can be considered as syntactic sugar as well.
Let $s$ be a string transformed into $\ebind{x}{e_1}{e_2}$.
To make variable definitions syntactic sugar, the parser can transform $s$ into
$\eapp{(\efun{x}{e_2})}{e_1}$. The evaluation of $\eapp{(\efun{x}{e_2})}{e_1}$
evaluates $e_1$ first. Then, $e_2$ is evaluated under the environment that $x$ denotes
the result of $e_1$. This semantics is exactly the same as that of
$\ebind{x}{e_1}{e_2}$. Therefore, we can say that variable definitions are just
syntactic sugar in \lang.

Hereafter, we remove variable definitions from \lang and call the language
\textsf{FAE}. However, we may still use variable definitions in examples. It is
completely fine because they are considered as syntactic sugar.

Furthermore, we can treat even integers, addition, and subtraction as syntactic
sugar. The only things we need are variables, lambda abstractions, and function
applications. We can write any programs with these three kinds of expressions.
The \textit{lambda calculus}\index{lambda calculus} is a language that provides
only the three features. This book does not discuss how integers, addition, and
subtraction can be desugared into the lambda calculus.

\section{Exercises}

\begin{enumerate}
\item Consider the following expression:

\[
\ebind{\cx}{5}{
    \ebind{\code{f}}{\efun{\cy}{\eadd{\cy}{\cx}}}{
        \eapp{(\efun{\code{g}}{\eapp{\code{f}}{(\eapp{\code{g}}{1})}})}
        {(\efun{\cx}{\cx})}
    }
}
\]
Describe a trace of the evalaution in terms of arguments to the \code{interp}
function for every call. (There will be 16 calls.) The \code{interp} function
takes two arguments---an expression and an environment---so show both for each call.
For \code{Num}, \code{Id}, and \code{Fun} expressions, show their result values, which
are immediate. You can use the following abbreviations and possibly more abbreviations:

\begin{center}
\begin{tabular}{lcl}
$E_0$ & = & the whole expression \\
$E_1$ & = & $\efun{\cy}{\eadd{\cy}{\cx}}$ \\
$E_2$ & = & $\efun{\code{g}}{\eapp{\code{f}}{(\eapp{\code{g}}{1})}}$ \\
$E_3$ & = & $\efun{\cx}{\cx}$ \\
$E_4$ & = & $\ebind{\code{f}}{E_1}{\eapp{E_2}{E_3}}$
\end{tabular}
\end{center}

\item This exercise examines differences between semantics by changing scope.
The following code is an implementation of an interpreter:

\begin{verbatim}
def interp(e: Expr, env: Env): Value = e match {
  case Num(n) => NumV(n)
  case Add(l, r) =>
    val NumV(n) = interp(l, env)
    val NumV(m) = interp(r, env)
    NumV(n + m)
  case Sub(l, r) =>
    val NumV(n) = interp(l, env)
    val NumV(m) = interp(r, env)
    NumV(n - m)
  case Id(x) => lookup(x, env)
  case Fun(x, b) => CloV(x, b, env)
  case App(f, a) =>
    val CloV(x, b, fEnv) = interp(f, env)
    interp(b, __________ + (x -> interp(a, env)))
}
\end{verbatim}

Describe the semantics of the \code{App} case in prose
when we use each of the following for the blank above:
\begin{itemize}
  \item \code{env}
  \item \code{Map()}
  \item \code{fEnv}
\end{itemize}

\item This exercise extends \lang to support multiple parameters.
Consider the following language:
\[
\begin{array}{lrrl}
  \text{Expression}& e & ::= & \cdots\ |\ \efun{x\cdots x}{e}\ |\
  \eappfo{e}{e,\cdots,e}\\
  \text{Value}& v & ::= & \cdots\ |\ \clov{x\cdots x}{e}{\sigma} \\
\end{array}
\]
The semantics of some constructs are as follows:
\begin{itemize}
  \item Evaluating $\lambda x_1\ \cdots\ x_n. e$ under $\sigma$
      yields a closure $\langle \lambda x_1 \cdots\ x_n.e,\sigma \rangle$.
  \item If
    \begin{itemize}
    \item evaluating $e_0$ under $\sigma$ yields a closure $\langle \lambda x_1
      \cdots\ x_n.e,\sigma' \rangle$,
    \item evaluating $e_i$ under $\sigma$ yields $v_i$ for each $1 \leq i \leq
      n$, and
    \item evaluating $e$ under $\sigma'[x_1 \mapsto v_1, \cdots, x_n \mapsto
      v_n]$ yields $v$,
    \end{itemize}
\item[] then evaluating $\eappfo{e}{e,\cdots,e}$ under $\sigma$ yields $v$.
\end{itemize}

\begin{enumerate}
  \item Write the operational semantics of the form \fbox{$\sigma\vdash e \Rightarrow v$} for the expressions.
  \item Write the evaluation derivation of the following expression:
\[
\inferrule
{ \hspace*{0.8\textwidth} }
{
  \eval{\emptyset}{
    \eappfo{(\efun{\code{f}\
    \code{m}}{\eappfo{\code{f}}{\code{m}}})}{\efun{\cx}{\cx},8}
  }{~~~~~~~~~~}
}
\]
\end{enumerate}

\item Rewrite the following \lang expression to an \textsf{FAE} expression.
  You should not ``evaluate'' it. Consider the approach of \refsec{fae}.

\[
  \ebind{\cx}{\efun{\cy}{\eadd{8}{\cy}}}{\efun{\cy}\eapp{\eapp{\cx}{(\esub{10}{\cy})}}}
\]

\item This exercise modifies \lang to check body expressions when evaluating
  function expressions.
Consider we extend the value of \lang to include a special value $\uparrow$
to represent an error during function body checking.
Write the operational semantics of the form
\fbox{$\evald{e}{v}$} for a function expression
$\efun{x}{e}$, when its semantics  changes as follows:
\begin{itemize}
\item If every free identifier of $e$ is in the domain of $\sigma$ or is $x$,
  then evaluation of $\efun{x}{e}$ under $\sigma$ yields a closure
  containing the function expression and the environment.
\item Otherwise, evaluation of $\efun{x}{e}$ under $\sigma$ yields $\uparrow$.
\end{itemize}
You may use the semantic function \embox{fv}, which takes an
expression and returns the set of every free identifier in the expression.
For example, $\embox{fv}(\efun{\cx}{\eapp{\cy}{\cx}}) = \{ \cy \}$.

\item This exercise extends \lang with records.
  Consider the following language:
\[
\begin{array}{lrrl}
  \text{Field} & f & \in & \textit{Field} \\
  \text{Record} & \rho & \in & \embox{Record} = \embox{Field} \finto \embox{Value} \\
  \text{Expression}& e & ::= & \cdots\ |\ \{f\ e,\ \ldots,\ f\ e\}\ |\ e.f\ |\
  e;e \\
  \text{Value} & v & ::= & \cdots\ |\ \rho \\
\end{array}
\]

The semantics of some constructs are as follows:
\begin{itemize}
  \item The evaluation of $\{f_1\ e_1,\ \cdots,\ f_k\ e_k\}$
    under $\sigma$ yields a finite map $\rho$,
which maps $f_i \in \{f_1\ \cdots,\ f_k\}$
to the value $v_i$ which is evaluated from the expression $e_i$ under $\sigma$.
  \item The evaluation of $e.f$ under $\sigma$ yields the value of the field $f$ in the record $\rho$,
      where evaluation $e$ under $\sigma$ yields $\rho$.
  \item If evaluation of $e_1$ yields some value under $\sigma$, and evaluation
    of $e_2$ yields $v$ under $\sigma$,
      then evaluation of $e_1; e_2$ yields $v$ under $\sigma$.
\end{itemize}

Write the operational semantics of the form
$\boxed{\sigma \vdash e \Rightarrow v}$

\item This exercise extends \lang with pairs. Consider the following language:
\[
\begin{array}{lrrl}
  \text{Expression}& e & ::= & \cdots\ |\ (e,e)\ |\ e\textsf{.1}\ |\ e\textsf{.2} \\
  \text{Value} & v & ::= & \cdots\ |\ \fbox{ (a) }
\end{array}
\]

\begin{enumerate}
  \item Write the syntax of a pair value in \fbox{ (a) } and
    the operational semantics of the form \fbox{$\evald{e}{v}$} for the expressions.
  \item Write the evaluation derivation of the following expression:
    \[
      \inferrule
      {\hspace*{0.8\textwidth}}
      { \eval{\emptyset}{(8,(320,42)\textsf{.1})\textsf{.2}}{~~~~~} }
    \]
\end{enumerate}

\item This exercise replaces the environment-based semantics of \lang with
substitution-based semantics.
Consider the following implementation:
\begin{verbatim}
trait Expr
trait Value extends Expr
case class Num(n: Int) extends Expr with Value
case class Add(l: Expr, r: Expr) extends Expr
case class Sub(l: Expr, r: Expr) extends Expr
case class Val(x: String, e: Expr, b: Expr) extends Expr
case class Id(x: String) extends Expr
case class Fun(x: String, b: Expr) extends Expr with Value
case class App(f: Expr, a: Expr) extends Expr

def subst(e: Expr, x: String, v: Value): Expr = e match {
  case Num(n) => e
  case Add(l, r) =>
    Add(subst(l, x, v), subst(r, x, v))
  case Sub(l, r) =>
    Sub(subst(l, x, v), subst(r, x, v))
  case Val(y, i, b) =>
    val nb = if (y == x) b else subst(b, x, v)
    Val(y, subst(i, x, v), nb)
  case Id(name) =>
    if (name == x) v else e
  case Fun(y, b) =>
    Fun(y, if (y == x) b else subst(b, x, v))
  case App(f, a) =>
    App(subst(f, x, v), subst(a, x, v))
}

def interp(e: Expr): Value = e match {
  case Num(n) => Num(n)
  case Add(l, r) =>
    val Num(n) = interp(l)
    val Num(m) = interp(r)
    Num(n + m)
  case Sub(l, r) =>
    val Num(n) = interp(l)
    val Num(m) = interp(r)
    Num(n + m)
  case Val(x, i, b) =>
    interp(subst(b, x, interp(i)))
  case Id(x) => error("free identifier")
  case Fun(x, b) => Fun(x, b)
  case App(f, a) =>
    val Fun(x, b) = interp(f)
    interp(subst(b, x, interp(a)))
}
\end{verbatim}

In this implementation, a value is either an integer or a lambda abstraction:

\[ v\ ::=\ n\ |\ \efun{x}{e} \]

\begin{enumerate}
  \item
    Write the operational semantics of the above implementation
    of the form \fbox{$e \Rightarrow v$}
    where $e[x/v]$ denotes \code{subst($e$, $x$, $v$)}.

  \item Write the definition of the substitution $e[x/v]$
    of the form \fbox{$e[x/v]=e$}:

  \item Consider the following expression:
\[
\ebind{\code{z}}{\efun{\cx}{\esub{\cx}{\cy}}}{
    \ebind{\cy}{10}{
        (\eapp{\code{z}}{32})
    }
}
\]

\begin{enumerate}
\item What is the result of evaluating the expression under the empty
  environment in substitution-based \lang?
\item What is the result of evaluating the expression under the empty
  environment in environment-based \lang?
\item Why are the results different?
  \sidenote{We can make the semantics of substitution-based \lang equivalent to
    environment-based \lang by modifying \code{subst} function.}
\end{enumerate}
\end{enumerate}

\item Consider the following language:
\[
  \begin{array}{lrrl}
  \text{Expression}& e & ::= & n\ |\ x\ |\ \efun{x\cdots x}{e}\ |\
  \eappfo{e}{e,\cdots,e}\ |\ \textsf{get}\ e \\
  \text{Value}& v & ::= & n\ |\ \textsf{undefined}\ |\ \clov{x\cdots x}{e}{\sigma} \\
  \end{array}
\]
The semantics of some constructs are as follows:
\begin{itemize}
  \item The value of a function expression $\efun{x_1\cdots x_n}{e}$
    at an environment $\sigma$ is a closure $\clov{x_1\cdots x_n}{e}{\sigma}$.
  \item A function application $\eappfo{e_0}{e_1,\cdots,e_n}$ is evaluated as follows:
    \begin{itemize}
      \item Evaluate the subexpressions in order.
        The value of $e_0$ should be a closure
        $\clov{x_1\cdots x_m}{e}{\sigma}$
        that has $m$ parameters.
      \item Create an array $\alpha$ of size $n$ and
        initialize the $i$-th value of the array with the value of $e_{i+1}$
        where $0 \le i \le n-1$.
      \item Evaluate the closure body $e$ under the environment $\sigma$
        extended as follows:
        \begin{itemize}
          \item The value of the $i$-th parameter is the value of $e_i$
            where $1 \le i \le m \le n$.
          \item The value of the $j$-th parameter is the \textsf{undefined}
            value where $n < j \le m$.
        \end{itemize}
        and the array $\alpha$.
    \end{itemize}
  \item The value of $\textsf{get}\ e$ is the $n$-th value of the array $\alpha$
    where $n$ is the value of $e$ and the array indices start from $0$.
\end{itemize}

For example,
$\eappfo{(\efun{\cx\cy}{\cy})}{4}$
evaluates to \textsf{undefined}, and
$\eappfo{(\efun{\cx}{\textsf{get}\ 0})}{5}$
evaluate to $5$.

\begin{enumerate}
  \item Write the operational semantics of the form
    \fbox{$\eval{\sigma,\alpha}{e}{v}$}.
  \item Write the evaluation derivation of the following expression:
  \[
    \inferrule
    {\hspace*{0.8\textwidth}}
    {\eval{\emptyset,\emptyset}{\eappfo{(\efun{\cx\ \cy}{\textsf{get}\ x})}{2,19,141}}{~~~~~}}
  \]
\end{enumerate}

\item The following quote describes the JavaScript sequencing semantics:
\sidenote{\url{https://tc39.es/ecma262/\#sec-block-runtime-semantics-evaluation}}

\begin{quote}
The value of a \embox{StatementList} is the value of the last
value-producing item in the \embox{StatementList}.  For example, the
following calls to the \verb!eval! function all return the value 1:
\begin{verbatim}
eval("1;;;;;")
eval("1;()")
eval("1;var a;")
\end{verbatim}
\end{quote}
Consider the following language:
\[
  \begin{array}{lrrl}
    \text{Expression} & e & ::= & \textsf{()}\ |\ x\ |\ \efun{x}{e}\ |\ \eapp{e}{e}\ |\
    e;\cdots;e \\
    \text{Value} & v & ::= & \textsf{()}\ |\ \clov{x}{e}{\sigma} \\
  \end{array}
\]
The value of the sequence expression $e_1;\cdots;e_n$
is the value of the last expression whose value is not $\textsf{()}$.
If the values of all the expressions $e_1,\cdots,e_n$ are $\textsf{()}$,
the value of the sequence expression is $\textsf{()}$.
Write the operational semantics of each expression of the form
\fbox{$\evald{e}{v}$}.

\item Consider the following language:
\[
\begin{array}{rll}
e ::= & a& \text{atomic expression}\\
\mid& e\ a& \text{function application}\\
\mid& \textsf{fn}\ m& \text{function expression}\\
a ::= & n & \text{number}\\
\mid&x & \text{identifier}\\
m ::= & p\ \leadsto\ e & \text{pattern matching}\\
\mid& p\ \leadsto\ e\ \verb!|!\ m & \text{pattern matching sequence}\\
p ::= & \verb!_! & \text{wildcard pattern}\\
\mid&n& \text{number pattern}\\
\mid&x& \text{identifier pattern}
\end{array}
\]
where a value of the language $v$ is either a number $n$ or a closure $\langle m, \sigma\rangle$,
a result of evaluation $r$ is either a value $v$ or a failure in pattern matching $\uparrow$,
which is different from run-time errors,
and an environment $\sigma$ maps identifiers to their values.

The operational semantics rules for expressions and atomic expressions are as follows:

\fbox{$\sigma\vdash e \Rightarrow r$}
\[
\begin{array}{c}
  \inferrule
  {\sigma \vdash a \hookrightarrow v}
  {\sigma \vdash a \Rightarrow v}
  \qquad
  \inferrule
  {\sigma \vdash e \Rightarrow \uparrow}
  {\sigma \vdash e\ a \Rightarrow \uparrow}
  \qquad
  \sigma \vdash \textsf{fn}\ m \Rightarrow \langle m, \sigma \rangle
  \\[1.5em]
  \inferrule
  {
    \sigma \vdash e \Rightarrow \langle m, \sigma' \rangle \\
    \sigma \vdash a \Rightarrow v \\
    (\sigma', v) \vdash m \Rightarrow v'
  }
  {\sigma \vdash e\ a \Rightarrow v'}
  \\[1.5em]
  \inferrule
  {
    \sigma \vdash e \Rightarrow \langle m, \sigma' \rangle \\
    \sigma \vdash a \Rightarrow v \\
    (\sigma', v) \vdash m \Rightarrow \uparrow
  }
  {\sigma \vdash e\ a \Rightarrow \uparrow}
\end{array}
\]

\fbox{$\sigma\vdash a \hookrightarrow v$}
\[
\begin{array}{c}
  \sigma \vdash n \hookrightarrow n
  \qquad\qquad
  \inferrule
  {x \in \dom{\sigma}}
  {\sigma \vdash x \hookrightarrow \sigma(x)}
\end{array}
\]

The semantics of pattern matching $m$ and pattern $p$ are as follows:
\begin{itemize}
\item Evaluation of $p\ \leadsto\ e$ under $(\sigma, v)$ has two possibilities.
First, when evaluation of $p$ results in a new environment $\sigma'$,
the result of this pattern matching is the result of evaluation of $e$ under $\sigma+\sigma'$,
where $\sigma+\sigma'$ is a disjoint union of $\sigma$ and $\sigma'$.
Second, when evaluation of $p$ produces $\uparrow$,
the evaluation of this pattern matching produces $\uparrow$ as well.
\item Evaluation of ``$p\ \leadsto\ e\ \verb!|!\ m$'' under $(\sigma, v)$
also has two possibilities.
First, when evaluation of $p\ \leadsto\ e$ succeeds with a value $v'$,
the value of this pattern matching sequence is $v'$.
Second, when evaluation of $p\ \leadsto\ e$ fails,
the result of evaluation of this pattern matching sequence is
the result of evaluation of $m$.
\item Evaluation of the wildcard pattern \verb!_! under $(\sigma, v)$
produces the empty environment.
\item Evaluation of the number pattern $n$ under $(\sigma, v)$ has two possibilities.
If $v=n$,
it produces the empty environment.  Otherwise, it produces $\uparrow$.
\item Evaluation of the identifier pattern $x$ under $(\sigma, v)$
produces a singleton environment $\{x \mapsto v\}$
if $x$ is not in the domain of $\sigma$.
\end{itemize}
Write the operational semantics for $m$ and $p$
of the forms \fbox{$(\sigma, v)\vdash m \Rightarrow r$} and
\fbox{$(\sigma, v)\vdash p \Rightarrow \sigma/\uparrow$}, respectively,
where \fbox{$(\sigma, v)\vdash p \Rightarrow \sigma/\uparrow$} denotes
\fbox{$(\sigma, v)\vdash p \Rightarrow \sigma$} or
\fbox{$(\sigma, v)\vdash p \Rightarrow \uparrow$}.
Remember that the operational semantics do not specify run-time errors.
\end{enumerate}
