\setchapterpreamble[u]{\margintoc}
\chapter{Syntax}
\labch{syntax}

The course defines programming languages. Defining a language is defining the
\term{syntax} and the \term{semantics} of the language. The article is about
syntax. Before going into detail about syntax, it firstly explains why defining a
language is essential.

Languages defined by the course are tiny and whom people do not use in practice.
For example, they cannot get input from users or print results; they do not have
typical types, including a string type and a floating-point number type. It seems
meaningless to define languages that do not have any usages. Defining such tiny
languages does not aim to make the course easy for undergraduate students.
Surprisingly, many kinds of PL research deal with small unused languages.

PL research often aims to prove that a language satisfies a specific property.
The language might be a language used by plenty of people at the moment or an
improved, unimplemented version of an existing language with new features. The
sentence uses the term 'property' in a broad sense: it refers to a property
derived from the definition of the language; it refers to the characteristics of
results obtained by applying a specific algorithm to code written in the
language.

Researchers define small languages because real-world languages are complicated
to be the subjects of research. The real-world languages have many features
helping programmers, such as syntactic sugar. Verifying a property of a language
containing all such features takes a long time. If all the features affected the
property, they sadly would have to deal with a language with all the features.
However, most features are not related to the property, whom the researchers want
to show. It is efficient to work on a small language containing only important
characteristics by identifying features influencing the property.

Besides, it is hard to apply research on a specific existing language to other
languages. If researchers proved a property while reflecting all the features of
the language, they would not be able to conclude that other languages with a
portion of the features satisfy the property. In contrast, if they research a
small language containing features affecting the property, they will be able to
apply the result to such other languages without considerable cost.

Research on Scala is a concrete example. Scala features objects with \term{type
members}---ignore what it is. Popular languages preceding Scala had not featured
them. It had not been sure whether type systems with objects with type members
are \term{type-sound}---ignore what type soundness is. Researchers had defined
DOT (dependent object type), which is a small language with objects with type
members, and proved the type-soundness of DOT. If they had tried to prove the
type-soundness of Scala, they would have spent decades and exerted themselves for
features orthogonal to objects with type members. However, not spending much
time, they had proved the type-soundness of DOT and could apply the result to
languages sharing the feature, such as Wyvern. Alas, even though DOT models the
feature precisely, the type-soundness of DOT does not imply the type-soundness of
Scala. Nonetheless, proving the safety of the core of Scala is crucial for those
who want to trust Scala. As features other than objects with type members of
Scala have been already verified with other researches, verifying only DOT is
quite enough.

In summary, the following is a typical flow of PL research.

1. Want to prove that feature \verb!B! of language \verb!A! satisfies property
\verb!C!.
2. Define small language \verb!a! representing \verb!B!.
3. Define property \verb!c! for \verb!a! as \verb!C! for \verb!A!.
4. Prove that \verb!a! satisfies \verb!c!.
5. \verb!A! probably satisfies \verb!C!, and other languages which feature
\verb!B! may satisfy \verb!C!.

Mind that numerous sorts of PL research do not follow the flow. PL researchers
make, prove, and verify real-world languages, programs, and systems. They invent
tools for practical usages. For instance, Infer of Facebook is a static analyzer
developed by PL researchers. Companies including Facebook and Amazon have been
using Infer.

Such practical research cannot exist without a theoretical background for core
properties and algorithms produced by research dealing with small languages. [The
foundations of
Infer](https://fbinfer.com/docs/separation-logic-and-bi-abduction.html) are
theories suggested by a few papers that are not on real-world languages. Since
the objects of the papers are small but general, Infer can analyze Java, C, C++,
and Objective-C rather than a single language.

The most crucial thing of PL research is to define and solve a small precise
problem expressing a problem of interest. So does the course. The course focuses
on essential features provided by most languages and defines tiny languages
representing the features. The course is a starting point of PL research. At the
same time, the course gives students who are not interested in PL basic knowledge
to understand and to use new languages.

\section{Syntax}

The syntax of a language determines whether code is correct code written in the
language.

\begin{verbatim}
class A
\end{verbatim}

\begin{verbatim}
class A {
\end{verbatim}

The former is code written in Scala, but the latter is not. The syntax of Scala
determines it.

In a mathematical sense, assume that the set of all possible code exists; the set
of all correct code written in language A is a subset of the former set. The
syntax of A defines the subset.

Syntax is either concrete or abstract. Despite the lack of strict definitions of
concrete and abstract syntax, they have distinct properties and are thus easily
distinguished. The course explains them briefly: concrete syntax is for people;
abstract syntax is for computers. The explanation intuitively shows what they
are.

\subsection{Concrete Syntax}

Existing for humans, \term{concrete syntax} deals with code written by people. It
defines a rule for strings and cares about all the characters including
whitespaces and newlines; it specifies rules like "two quotation marks are at the
start and the end of a string," "two consecutive backslashes indicate the start
of a comment," and "every operator is an infix operator." The specifications of
most languages describe the concrete syntax of the languages since programmers
write code according to the specifications.

\term{Backus-Naur form} (BNF) is the most popular way to describe syntax. A form
includes one or more rules. Each rule is in the form of
\verb!<symbol> ::= expression | expression …!. A symbol between angle brackets is a
\term{metavariable}, which denotes a set of strings. An expression is an
enumeration of metavariables and strings. A set denoted by the metavariable
includes strings obtained by substituting metavariables with elements of the
metavariables in one of the expressions. Every string starts and ends with
quotation marks.

The article defines the syntax of AE, a language for arithmetic expressions.

An expression of AE is

\begin{itemize}
\item an integer,
\item the sum of two expressions, or
\item the difference of two expressions.
\end{itemize}

The following is the concrete syntax of AE in the BNF:

\[
\begin{array}{l}
\texttt{digit ::= "0" | "1" | "2" | "3" | "4"} \\
\texttt{\ \ \ \ \ \ \ \ }\texttt{| "5" | "6" | "7" | "8" | "9"} \\
\texttt{nat}\texttt{\ \ \ }\texttt{::= digit | digit nat} \\
\texttt{num}\texttt{\ \ \ }\texttt{::= nat | "-" nat} \\
\texttt{expr}\texttt{\ \ }\texttt{::= num} \\
\texttt{\ \ \ \ \ \ \ \ }\texttt{| "(" expr "+" expr ")"} \\
\texttt{\ \ \ \ \ \ \ \ }\texttt{| "(" expr "-" expr ")"} \\
\end{array}
\]

The remaining part of the section shows how to interpret syntax in the BNF.
\(Digit\) is a set denoted by \(\texttt{digit}\); \(Nat\) is a set denoted by \(\texttt{
nat}\); \(Num\) is a set denoted by \(\texttt{num}\); \(Expr\) is a set denoted by
\(\texttt{expr}\).

\(Digit\) equals , a set of the digits of decimals.

\(Nat\) is the smallest set satisfying the following two conditions; it denotes
the set of every natural number. The \(\cdot\) operator denotes string
concatenation.

\begin{enumerate}
\item \(\forall d\in Digit.d\in Nat\)
\item \(\forall d\in Digit.\forall n\in Nat.d \cdot n\in Nat\)
\end{enumerate}

\(Num\) is the smallest set satisfying the following two conditions; it denotes
the set of every integer.

\begin{enumerate}
\item \(\forall n\in Nat.n\in Num\)
\item \(\forall n\in Nat.\texttt{"-"}\cdot n\in Num\)
\end{enumerate}

\(Expr\) is the smallest set satisfying the following three conditions; it
denotes the set of every arithmetic expression.

\begin{enumerate}
\item \(\forall n\in Num.n\in Expr\)
\item \(\forall e_1\in Expr.\forall e_2\in Expr.{\texttt{"("}}\cdot e_1\cdot{\texttt{"+"}}\cdot
e_2\cdot{\texttt{")"}}\in Expr\)
\item \(\forall e_1\in Expr.\forall e_2\in Expr.{\texttt{"("}}\cdot
e_1\cdot\texttt{"-"}\cdot e_2\cdot{\texttt{")"}}\in Expr\)
\end{enumerate}

\(\texttt{"(1+2)"}\) is an element of \(Expr\), but \(\texttt{"1+2"}\) is not an element
of \(Expr\) due to the lack of parentheses.

\subsection{Abstract Syntax}

Most kinds of PL research define languages with \term{abstract syntax} instead of
concrete syntax, which is unnecessarily precise. As the previous section shows,
concrete syntax cares unimportant details.

Abstract syntax is an abstract data structure describing code. Unlike concrete
syntax, which deals with strings, it deals with abstract objects. Since people
mostly use strings to represent information, strings often describe abstract
syntax. However, the essence of abstract syntax is not about strings. For
example, strings "1" and "one" represent the number one, but the essence of the
number one is that it is the successor of zero but not how people write it on
papers. In the same manner, regardless of a way of describing abstract syntax,
abstract deals with abstract objects but not strings.

The following is the abstract syntax of AE in the BNF:

\[
\begin{array}{rcl}
n & \in & \mathbb{Z} \\
e & ::= & n \\
& | & e+e \\
& | & e-e \\
\end{array}
\]

Metavariable \(n\) ranges over integers; metavariable \(e\) ranges over
expressions.

Like concrete syntax, the abstract syntax in the BNF defines a set. Let
\(\mathcal{A}\) is a set denoted by \(e\). \(\mathcal{A}\) is the smallest set
satisfying the following three conditions.

\begin{enumerate}
\item \(\forall n\in\mathbb{Z}.n\in \mathcal{A}\)
\item \(\forall e_1\in\mathcal{A}.\forall e_2\in\mathcal{A}.e_1+e_2\in\mathcal{A}\)
\item \(\forall e_1\in\mathcal{A}.\forall e_2\in\mathcal{A}.e_1-e_2\in\mathcal{A}\)
\end{enumerate}

\term{Inference rules} can define abstract syntax as well. Inference rules
typically define the semantics of languages, but the article defines abstract
syntax with inference rules to make readers familiar with inference rules. It is
possible to define concrete syntax with inference rules, but I think that it is
redundant and unnecessary.

Inference rules derive a \term{proposition} from propositions. An inference rule
is composed of a horizontal line, zero or more propositions above the line, and a
proposition below the line. If no proposition exists above the line, then the
line can be omitted. The propositions above the line are premises; the
proposition below the line is a conclusion. Every proposition in the rule may
have metavariables.

For instance, an inference rule can encode \term{modus ponens}, which implies
that for any propositions \(P\) and \(Q\), if \(P\rightarrow Q\) and \(P\), then
\(Q\). Let metavariables \(p\) and \(q\) range over propositions.

\[
\inferrule
{ p\rightarrow q \\ p }
{ q }
\]

If substituting every metavariable with an element of the metavariable in a rule
makes every premise of the rule true, then the conclusion of the rule also is
true. Assume that \(P\) and \(Q\) are propositions, and both \(Q\rightarrow P\)
and \(Q\) are true. Substituting \(p\) and \(q\) with \(Q\) and \(P\) results in
two true premises and conclusion \(P\). The following \term{proof tree} is a
proof of \(P\):

\[
\inferrule
{ Q\rightarrow P \\ Q }
{ P }
\]

One can use inference rules multiple times to prove a proposition. Assume that
\(P\), \(Q\), and \(R\) are propositions, and \(P\rightarrow(Q\rightarrow R)\),
\(P\), and \(Q\) are true. Substituting \(p\) and \(q\) with \(P\) and
\(Q\rightarrow R\) yields that \(Q\rightarrow R\) is true. Substituting \(p\) and
\(q\) with \(Q\) and \(R\) finally proves \(R\). The following proof tree gives a
proof:

\[
\inferrule
{
{\inferrule
  { P\rightarrow(Q\rightarrow R) \\ P }
  { Q\rightarrow R } }\\
  Q }
{ R }
\]

The following inference rules define the abstract syntax of AE:

\[
\inferrule
{ n\in\mathbb{Z} }
{ n\in\mathcal{A} }
\\
\inferrule
{ e_1\in\mathcal{A} \\ e_2\in\mathcal{A} }
{ e_1+e_2\in\mathcal{A} }
\\
\inferrule
{ e_1\in\mathcal{A} \\ e_2\in\mathcal{A} }
{ e_1-e_2\in\mathcal{A} }
\]

The following proof tree proves that \(4+(2-1)\) is an element of
\(\mathcal{A}\).
Note that we can use parentheses to resolve ambiguity in abstract syntax since
it defines mathematical notation.

\[
\inferrule
{
\inferrule
  { 4\in\mathbb{Z} }
  { 4\in\mathcal{A} } \\
  \inferrule
  { \inferrule
    { 2\in\mathbb{Z} }
    { 2\in\mathcal{A} } \\
    \inferrule
    { 1\in\mathbb{Z} }
    { 1\in\mathcal{A} }
  }
  { (2-1)\in\mathcal{A} }
}
{ 4+(2-1)\in\mathcal{A} }
\]

Scala code also can represent the abstract syntax of AE. It is a typical ADT; a
sealed trait and case classes define it:

\begin{verbatim}
sealed trait Expr
case class Num(n: Int) extends Expr
case class Add(l: Expr, r: Expr) extends Expr
case class Sub(l: Expr, r: Expr) extends Expr
\end{verbatim}

The following Scala code represents \(4+(2-1)\):

\begin{verbatim}
Add(Num(4), Sub(Num(2), Num(1)))
\end{verbatim}

Most sorts of abstract syntax define tree shapes. Trees following abstract syntax
are \term{abstract syntax trees} (ASTs). The below tree visualizes \(4+(2-1)\).
The structure of an object defined by the above Scala code equals the tree.

\subsection{Parsing}

\term{Parsing} is a process that transforms strings following concrete syntax
into ASTs and rejects strings not following the concrete syntax. A \term{parser}
is a parsing program. Parsing is out of the scope of the course and thus is out
of the scope of the article.

The Scala standard library provides \term{parser combinators}. Programmers can
implement parsers without detailed knowledge about parsing. The below code
implements a parser of AE. The parser takes a string as input and produces an AST
of AE; it throws an exception if the string does not follow the concrete syntax
of AE. Note that strings may contain whitespaces freely, while concrete syntax
defined by the article is tight with whitespaces.

\begin{verbatim}
import scala.util.parsing.combinator._

object Expr extends RegexParsers {
  def wrap[T](e: Parser[T]): Parser[T] = "(" ~> e <~ ")"
  lazy val n: Parser[Int] = "-?\\d+".r ^^ (_.toInt)
  lazy val e: Parser[Expr] =
    n                    ^^ Num                         |
    wrap((e <~ "+") ~ e) ^^ { case l ~ r => Add(l, r) } |
    wrap((e <~ "-") ~ e) ^^ { case l ~ r => Sub(l, r) }

  def parse(s: String): Expr =
    parseAll(e, s).getOrElse(throw new Exception)
}

Expr.parse("1")
// Num(1)

Expr.parse("(4 + (2 - 1))")
// Add(Num(4),Sub(Num(2),Num(1)))

Expr.parse("1 + 2")
// java.lang.Exception
\end{verbatim}

\newpage
\section{Exercises}

\begin{enumerate}
\item Given the following grammar:
%
\begin{verbatim}
    <WAE> ::= <num>
            | {+ <WAE> <WAE>}
            | {* <WAE> <WAE>}
            | {let {<id> <WAE>} <WAE>}
            | <id>
\end{verbatim}
%
Describe whether each of the following is \verb+<WAE>+ and why:

\begin{itemize}
\item[a)]
\begin{verbatim}
{let {x 5} {+ 8 {* x 2 3}}}
\end{verbatim}

\item[b)]
\begin{verbatim}
{with {x 0} {with {x 7}}}
\end{verbatim}

\item[c)]
\begin{verbatim}
{let {3 5} {+ 8 {- x 2}}}
\end{verbatim}

\item[d)]
\begin{verbatim}
{let {3 y} {+ 8 {* x 2}}}
\end{verbatim}

\item[e)]
\begin{verbatim}
{let {x y} {+ 8 {* x 2}}}
\end{verbatim}
\end{itemize}

\item Given the following grammar:
%
\newcommand{\BNF}[1]{$\langle$#1$\rangle$}
\newcommand{\coffee}{\mbox{\BNF{coffee}}}
\newcommand{\milk}{\mbox{\BNF{milk}}}
\newcommand{\flavor}{\mbox{\BNF{flavor}}}

\[
\begin{array}{ccc}
{\texttt{espresso} \in \coffee}
&&
\newinfrule
{e_1 \in \milk\qquad
e_2 \in \coffee
}
{\ e_1\ \texttt{on}\ e_2 \in \coffee}
\\[2em]
\newinfrule
{e_1 \in \coffee\qquad
e_2 \in \milk
}
{\ e_1\ \texttt{on}\ e_2 \in \coffee}
&&
\newinfrule
{e_1 \in \flavor \qquad
e_2 \in \coffee
}
{\ e_1\ \texttt{on}\ e_2 \in \coffee}
\\[2em]
{\texttt{milk-foam} \in \milk}
&&
{\texttt{steamed-milk} \in \milk}
\\[2em]
{\texttt{caramel} \in \flavor}
&&
{\texttt{cinnamon} \in \flavor}
\\[2em]
{\texttt{cocoa-powder} \in \flavor}
&&
{\texttt{chocolate-syrup} \in \flavor}
\end{array}
\]
where \texttt{on} is right-associative.

\begin{itemize}
\item[a)] Which of the following are examples of \BNF{coffee}?
%
\begin{enumerate}

  \item[1)] \texttt{caramel latte macchiato} % No

  \item[2)] \texttt{espresso} % Yes

  \item[3)] \texttt{steamed-milk on caramel on milk-foam on espresso} % Yes

  \item[4)] \texttt{chocolate-syrup on cocoa-powder on cinnamon on milk-foam on steamed-milk on espresso} % Yes

  \item[5)] \texttt{steamed-milk on espresso on chocolate-syrup} % No

\end{enumerate}

\item[b)] Draw a proof of why the following is or is not \BNF{coffee}:

\begin{center}
\texttt{cocoa-powder on milk-foam on steamed-milk on espresso}
\end{center}
\end{itemize}

\item Given the following grammar:
%
\begin{center}
\begin{tabular}{lll}
 \BNF{ice-cream} & $::=$ & \texttt{sprinkles on \BNF{ice-cream}} \\
  & $|$ & \texttt{cherry on \BNF{ice-cream}} \\
  & $|$ & \texttt{scoop of \BNF{flavor} on \BNF{ice-cream}} \\
  & $|$ & \texttt{sugar-cone} \\
  & $|$ & \texttt{waffle-cone} \\
 \BNF{flavor} & $::=$ & \texttt{vanilla} \\
  & $|$ & \texttt{lettuce}
\end{tabular}
\end{center}
%
\begin{itemize}
  \item[a)] Which of the following are examples of \BNF{ice-cream}?
%
\begin{itemize}
  \item[1)] \texttt{sprinkles}
  \item[2)] \texttt{sugar-cone}
  \item[3)] \texttt{vanilla}
  \item[4)] \texttt{scoop of vanilla on waffle-cone}
  \item[5)] \texttt{sprinkles on lettuce on waffle-cone}
  \item[6)] \texttt{scoop of vanilla on sprinkles on waffle-cone}
\end{itemize}

  \item[b)] Explain why the following is an \BNF{ice-cream}:

\begin{center}
\texttt{cherry on scoop of lettuce on scoop of vanilla on sugar-cone}
\end{center}
\end{itemize}

\end{enumerate}
