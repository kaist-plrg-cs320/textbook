\setchapterpreamble[u]{\margintoc}
\chapter{First-Order Representation of Continuations}
\labch{first-order-representation-of-continuations}

\renewcommand{\lang}{\textsf{KFAE}\xspace}

The previous chapter implements an interpreter of \lang with first-class
functions in Scala. The interpreter treats
continuations as Scala functions. Since the interpreter uses CPS, continuations
are passed from functions to functions. In addition, continuations sometimes need to be
stored in \code{ContV} because \lang supports first-class continuations.
\code{ContV} instances are stored inside environments or returned from
\code{interp}. Therefore, the interpreter relies on the fact that Scala provides
first-class functions. Since functions are values in Scala, the interpreter can
represent continuations with functions and uses them as values.

The use of first-class functions is problematic for some cases. First, low-level languages, such
as C, lack first-class functions.\sidenote{C provides function pointers but not
closures. Closures are necessary to represent continuations.} There must be
another way to implement an interpreter of \lang for those who use low-level languages.
Second, functions do not give useful information. The only ability of functions
is being applied to arguments. However, in particular programs like debuggers, it
is necessary to figure out what a given first-class continuation does. The
current implementation disallows such analysis on continuations. On the other
hand, a \code{CloV} instance, which represents a closure, can give the exact
information about the parameter, body, and environment of the closure. Alas,
\code{ContV} instances do not have such capabilities.

This chapter shows how we can represent continuations without
first-class functions. By avoiding first-class functions, an interpreter of a
language with first-class continuations can be written in low-level languages. In
addition, if continuations are not functions and have specific structures
instead, debuggers can analyze what a given continuation denotes.

\section{First-Order Representation of Continuations}

The following code shows the \lang interpreter implemented in the previous
chapter:\sidenote{Since the \code{Sub} case is similar to the \code{Add} case,
this chapter omits the \code{Sub} case.}

\begin{verbatim}
def interp(e: Expr, env: Env, k: Cont): Value = e match {
  case Num(n) => k(NumV(n))
  case Id(x) => k(lookup(x, env))
  case Fun(x, b) => k(CloV(x, b, env))
  case Add(e1, e2) =>
    interp(e1, env, v1 =>
      interp(e2, env, v2 =>
        k(add(v1, v2))
      )
    )
  case App(e1, e2) =>
    interp(e1, env, v1 =>
      interp(e2, env, v2 => v1 match {
        case CloV(x, e, fenv) =>
          interp(e, fenv + (x -> v2), k)
        case ContV(nk) => nk(v2)
      })
    )
  case Vcc(x, b) =>
    interp(b, env + (x -> ContV(k)), k)
}
\end{verbatim}

We can change the implementation a bit by replacing
\code{k(v)} with \code{continue(k, v)}.

\begin{verbatim}
def continue(k: Cont, v: Value): Value = k(v)

def interp(e: Expr, env: Env, k: Cont): Value = e match {
  case Num(n) => continue(k, NumV(n))
  case Id(x) => continue(k, lookup(x, env))
  case Fun(x, b) => continue(k, CloV(x, b, env))
  case Add(e1, e2) =>
    interp(e1, env, v1 =>
      interp(e2, env, v2 =>
        continue(k, add(v1, v2))
      )
    )
  case App(e1, e2) =>
    interp(e1, env, v1 =>
      interp(e2, env, v2 => v1 match {
        case CloV(x, e, fenv) =>
          interp(e, fenv + (x -> v2), k)
        case ContV(nk) => continue(nk, v2)
      })
    )
  case Vcc(x, b) =>
    interp(b, env + (x -> ContV(k)), k)
}
\end{verbatim}

Since evaluating \code{k(v)} is everything \code{continue(k, v)} does, the new
implementation behaves the same as the previous implementation.
For now, the change seems needless, but it will become useful soon.

We need to know which continuations are used in the original interpreter to
define values representing continuations. There are four sorts of continuations
in the interpreter:

\begin{itemize}
  \item \code{v1 => interp(e2, env, v2 => continue(k, add(v1, v2)))}
  \item \code{v2 => continue(k, add(v1, v2))}
  \item \code{v1 => interp(e2, env, v2 => v1 match ...)}
  \item \code{v2 => v1 match ...}
\end{itemize}

The first continuation, \code{v1 => interp(e2, env, v2 => continue(k, add(v1,
v2)))}, is used after the evaluation of the left operand of addition. It
evaluates the right operand, calculates the sum, and passes the sum to the
continuation of the entire addition. The parameter \code{v1} denotes the value of
the left operand. The function body contains three free variables: \code{e2},
\code{env}, and \code{k}. \code{e2} is the right operand; \code{env} is the
current environment; \code{k} is the continuation of the addition.
If the values of the free variables are determined, the
behavior of the continuation is also determined. Therefore, \code{(e2, env, k)},
which is a triple of an expression, an environment, and a continuation,
can represent the continuation.

Currently, \code{continue} continues the evaluation with a given function, which
represents the continuation. Since the continuation is a Scala function, it can be
directly applied to a given value. However, if we use a triple to represent the
continuation instead of a function, it cannot be applied to a value. We need a new
way to continue evaluation when a continuation and a value are given. The clue
already exists---look at the body of the function representing a continuation.
When the function is applied to \code{v1}, the result is \code{interp(e2, env,
v2 => continue(k, add(v1, v2)))}. Now, \code{(e2, env, k)} and \code{v1} are
provided instead of the function and \code{v1}. It is enough to evaluate
\code{interp(e2, env, v2 => continue(k, add(v1, v2)))} with \code{v1},
\code{e2}, \code{env}, and \code{k}. It evaluates the same thing as the
original function application.

Below compare the previous and current strategies:

\begin{itemize}
  \item Previous: \code{v1 => interp(e2, env, v2 => continue(k ,add(v1, v2)))} and
  \code{v1} are given. Then, evaluate \code{interp(e2, env, v2 => continue(k,
  add(v1, v2)))} by applying \code{v1 => interp(e2, env, v2 => continue(k
  ,add(v1, v2)))} to \code{v1}.
  \item Current: \code{(e2, env, k)} and \code{v1} are given. Then, evaluate
  \code{interp(e2, env, v2 => continue(k, add(v1, v2)))} with \code{e2},
  \code{env}, \code{k}, and \code{v1}.
\end{itemize}

Both strategies evaluate \code{interp(e2, env, v2 => continue(k, add(v1,
v2)))} in the end. While the previous strategy represents continuations with
functions, the current strategy represents continuations with triples.

Once you understand the first sort of continuations, the remaining ones are
straightforward. The second continuation, \code{v2 => continue(k, add(v1,
v2))}, is used after the evaluation of the right operand of addition. It
calculates the sum of the operands and passes the sum to the continuation of the
addition. The parameter \code{v2} denotes the value of the right operand. The
function body contains two free variables: \code{k} and \code{v1}. \code{k} is
the continuation of the addition; \code{v1} is the value of the left operand. In
a similar fashion, \code{v1} and \code{k} are enough to determine what the
continuation does. Therefore, \code{(v1, k)}, a pair of a value and a
continuation, can represent the continuation. To continue the evaluation when
\code{(v1, k)} and \code{v2} are given, \code{continue(k, add(v1, v2))} needs
to be evaluated.

\begin{itemize}
  \item Previous: \code{v2 => continue(k, add(v1, v2))} and \code{v2} are given.
  Then, evaluate \code{continue(k, add(v1, v2))} by applying \code{v2 =>
  continue(k, add(v1, v2))} to \code{v2}.
  \item Current: \code{(v1, k)} and \code{v2} are given. Then, evaluate
  \code{continue(k, add(v1, v2))} with \code{v1}, \code{k}, and \code{v2}.
\end{itemize}

The third continuation, \code{v1 => interp(e2, env, v2 => v1 match ...)}, is
used after the evaluation of the expression at the function position of
a function application. It evaluates an argument and applies a function (or
a continuation) to the argument.
\code{v1} denotes the value of the expression at the function position. The body
of the function representing the continuation contains three free variables:
\code{e2}, \code{env}, and \code{k}.\sidenote{\code{k} is in \code{...}.}
\code{e2} is the expression at the argument
position; \code{env} is the current environment; \code{k} is the continuation of
the application. Therefore, \code{e2}, \code{env},
and \code{k} determine what the continuation does, and \code{(e2, env, k)}, a
triple of an expression, an environment, and a continuation, can represent the
continuation. Continuing the evaluation is evaluating \code{interp(e2, env, v2 =>
v1 match ...)}, which can be done with \code{(e2, env, k)} and \code{v1}.

\begin{itemize}
  \item Previous: \code{v1 => interp(e2, env, v2 => v1 match ...)} and \code{v1}
  are given. Then, evaluate \code{v1 => interp(e2, env, v2 => v1 match ...)} by
  applying \code{v1 => interp(e2, env, v2 => v1 match ...)} to \code{v1}.
  \item Current: \code{(e2, env, k)} and \code{v1} are given. Then, evaluate
  \code{interp(e2, env, v2 => v1 match ...)} with \code{e2}, \code{env},
  \code{k}, and \code{v1}.
\end{itemize}

The fourth continuation, \code{v2 => v1 match ...}, is used after the
evaluation of the argument of a function application. It applies a function (or
a continuation) to the argument.
\code{v2} denotes the value of the argument. The body of the function
representing the continuation contains two free variables: \code{v1} and
\code{k}.\sidenote{\code{k} is in \code{...}.}
\code{v1} is the value of the expression at the function position;
\code{k} is the continuation of the application. Therefore, \code{(v1, k)},
a pair of a value and a continuation, can represent
the continuation. Continuing the evaluation is evaluating \code{v1 match ...}
with \code{(v1, k)} and \code{v2}.

\begin{itemize}
  \item Previous: \code{v2 => v1 match ...} and \code{v2} are given. Then, evaluate
  \code{v1 match ...} by applying \code{v2 => v1 match ...} to \code{v2}.
  \item Current: \code{(v1, k)} and \code{v2} are given. Then, evaluate \code{v1 match
  ...} with \code{v1}, \code{k}, and \code{v2}.
\end{itemize}

In fact, there is one more continuation, which does not appear in the implementation of
\code{interp}. It is the one that is represented as the identity function and is
passed to \code{interp} in the beginning. The identity function returns a given
argument without any changes. No additional information is necessary to determine the
behavior of the continuation. Therefore, \code{()}, the zero-length tuple (the
\code{Unit} value in Scala) can represent the continuation. To continue the
evaluation with the continuation and a value \code{v}, it is enough to give
\code{v} as the result.

\begin{itemize}
  \item Previous: An identity function and \code{v} are given. Then, evaluate \code{v}
  by applying the identity function to \code{v}.
  \item Current: \code{()} and \code{v} are given. Then, evaluate \code{v} with
  \code{v}.
\end{itemize}

In summary, the \lang interpreter uses continuations of the following five
sorts:

\begin{itemize}
  \item \code{(e2: Expr, env: Env, k: Cont)}
  \item \code{(v1: Value, k: Cont)}
  \item \code{(e2: Expr, env: Env, k: Cont)}
  \item \code{(v1: Value, k: Cont)}
  \item \code{()}
\end{itemize}

Note that the first and the third are different even though they look the same.
The first continuation computes \code{interp(e2, env, v2 => continue(k,
add(v1, v2)))} with its data, while the third continuation computes
\code{interp(e2, env, v2 => v1 match ...)} with its data. Similarly, the
second and the fourth are diffent as well. The second computes \code{continue(k,
add(v1, v2))}, while the fourth computes \code{v1 match ...}.

As explained in \refch{pattern-matching}, an ADT is the best way to implement a
type that consists of values of various shapes.
Thus, \code{Cont} can be newly defined with a \code{sealed trait} and \code{case
class}es as follows:

\begin{verbatim}
sealed trait Cont
case class AddSecondK(e2: Expr, env: Env, k: Cont) extends Cont
case class DoAddK(v1: Value, k: Cont) extends Cont
case class AppArgK(e2: Expr, env: Env, k: Cont) extends Cont
case class DoAppK(v1: Value, k: Cont) extends Cont
case object MtK extends Cont
\end{verbatim}

The names of the classes do not matter, though they are named carefully so that
the names can reflect what they are for. The important things are data carried by
each continuation. Following the Scala convention, the last sort, which can be
represented by the empty tuple, is now represented by a singleton object. One
may use \code{case class MtK() extends Cont} instead without changing the semantics,
but the singleton object is more efficient than the case class from the
implementation perspective. Now, the implementation of continuations does not
require first-class functions.

Now, we need to revise the \code{continue} function. The previous implementation
is \code{def continue(k: Cont, v: Value): Value = k(v)}. It works because
\code{Cont} is a function type before our change. However, \code{Cont} is not a
function now, and
\code{continue} needs a fix. In fact, we already know everything to make a
correct fix. Previously, \code{continue} applies \code{k} to \code{v} when \code{k}
and \code{v} are given. Now, it should check \code{k} and do the correct
computation according to the data in \code{k}. Below is the repetition of the
previous explanations, but with the names of the case classes and object.

\begin{itemize}
  \item \code{AddSecondK(e2, env, k)} and \code{v1} are given. Then, evaluate \code{interp(e2,
  env, v2 => continue(k, add(v1, v2)))} with \code{e2}, \code{env}, \code{k},
  and \code{v1}.
  \item \code{DoAddK(v1, k)} and \code{v2} are given. Then, evaluate \code{continue(k,
  add(v1, v2))} with \code{v1}, \code{k}, and \code{v2}.
  \item \code{AppArgK(e2, env, k)} and \code{v1} are given. Then, evaluate \code{interp(e2,
  env, v2 => v1 match ...)} with \code{e2}, \code{env}, \code{k}, and
  \code{v1}.
  \item \code{DoAppK(v1, k)} and \code{v2} are given. Then, evaluate \code{v1 match ...}
  with \code{v1}, \code{k}, and \code{v2}.
  \item \code{MtK} and \code{v} are given. Then, evaluate \code{v} with \code{v}.
\end{itemize}

The first and third explanations still pass functions to \code{interp} even
though continuations are not functions anymore. They need small changes. Now,
\code{DoAddK(v1, k)} represents \code{v2 => continue(k, add(v1, v2))}, and
\code{DoAppK(v1, k)} represents \code{v2 => v1 match ...}.

\begin{itemize}
  \item \code{AddSecondK(e2, env, k)} and \code{v1} are given. Then, evaluate \code{interp(e2,
  env, DoAddK(v1, k))} with \code{e2}, \code{env}, \code{k}, and \code{v1}.
  \item \code{AppArgK(e2, env, k)} and \code{v1} are given. Then, evaluate \code{interp(e2,
  env, DoAppK(v1, k))} with \code{e2}, \code{env}, \code{k}, and \code{v1}.
\end{itemize}

The new implementation of \code{continue} is as follows:

\begin{verbatim}
def continue(k: Cont, v: Value): Value = k match {
  case AddSecondK(e2, env, k) => interp(e2, env, DoAddK(v, k))
  case DoAddK(v1, k) => continue(k, add(v1, v))
  case AppArgK(e2, env, k) => interp(e2, env, DoAppK(v, k))
  case DoAppK(v1, k) => v1 match {
    case CloV(x, e, fenv) =>
      interp(e, fenv + (x -> v), k)
    case ContV(nk) => continue(nk, v)
  }
  case MtK => v
}
\end{verbatim}

The code is straightforward since it is exactly the same as the explanation.

The \code{interp} function also needs a fix to follow the new definition of
\code{Cont}:

\begin{verbatim}
def interp(e: Expr, env: Env, k: Cont): Value = e match {
  case Num(n) => continue(k, NumV(n))
  case Id(x) => continue(k, lookup(x, env))
  case Fun(x, b) => continue(k, CloV(x, b, env))
  case Add(e1, e2) => interp(e1, env, AddSecondK(e2, env, k))
  case App(e1, e2) => interp(e1, env, AppArgK(e2, env, k))
  case Vcc(x, b) => interp(b, env + (x -> ContV(k)), k)
}
\end{verbatim}

Only the \code{Add} and \code{App} cases are different from before. The \code{Add} case
uses \code{AddSecondK(e2, env, k)} to represent \code{v1 => interp(e2, env, v2 =>
continue(k, add(v1, v2)))}; the \code{App} case uses \code{AppArgK(e2, env,
k)} to represent \code{v1 => interp(e2, env, v2 => v1 match ...)}.

Note that \code{MtK} does not appear in \code{interp}. \code{MtK} is used to
call \code{interp} in the beginning. One should write \code{interp($e$, Map(),
MtK)} to evaluate $e$.

\section{Big-Step Semantics of \lang}

Representing continuations without first-class functions additionally gives us a clue to
define the big-step semantics of \lang. This section defines
the big-step semantics of \lang from the new interpreter implementation.

First, we can define continuations inductively.
It is straightforward from the implementation.

\begin{verbatim}
sealed trait Cont
case class AddSecondK(e2: Expr, env: Env, k: Cont) extends Cont
case class DoAddK(v1: Value, k: Cont) extends Cont
case class AppArgK(e2: Expr, env: Env, k: Cont) extends Cont
case class DoAppK(v1: Value, k: Cont) extends Cont
case object MtK extends Cont
\end{verbatim}

The above ADT can be formalized as below:

\[
\kappa\ ::=\ [\square+(e,\sigma)]::\kappa
\ |\ [v+\square]::\kappa
\ |\ [\square\ (e,\sigma)]::\kappa
\ |\ [v\ \square]::\kappa
\ |\ [\square]
\]

where the metavariable $\kappa$ ranges over continuations.
Note that $\square$ in the definition is different from $\square$ used by the
small-step semantics. While $\square$ denotes the empty computation stack in the
small-step semantics, $\square$ in the above definition is just a part of the
notation. It does not have any formal meaning but conceptually represents a hole
in an intuitive way.

Previous chapters define only one sort of a proposition for the big-step
semantics because their interpreters have only \code{interp}. On the other hand,
the interpreter of this chapter has both \code{interp} and \code{continue}, and
each of them has a different role from the other. Therefore, we formalize each
function with a different sort of a proposition. The first sort is
$\sigma,\kappa\vdash e\Rightarrow v$, which denotes that the result of
\code{interp($e$, $\sigma$, $\kappa$)} is $v$. The other is
$v_1\mapsto\kappa\Downarrow v_2$, which denotes that the result of
\code{continue($\kappa$, $v_1$)} is $v_2$.

Let us define inference rules related to \code{interp} first.

\begin{verbatim}
def interp(e: Expr, env: Env, k: Cont): Value = e match {
  case Num(n) => continue(k, NumV(n))
  case Id(x) => continue(k, lookup(x, env))
  case Fun(x, b) => continue(k, CloV(x, b, env))
  case Add(e1, e2) => interp(e1, env, AddSecondK(e2, env, k))
  case App(e1, e2) => interp(e1, env, AppArgK(e2, env, k))
  case Vcc(x, b) => interp(b, env + (x -> ContV(k)), k)
}
\end{verbatim}

Each case of the \code{interp} function produces a single inference rule.

\[
  \inferrule
  { n\mapsto\kappa\Downarrow v }
  { \sigma,\kappa\vdash n\Rightarrow v }
  \quad\textsc{[Interp-Num]}
\]

The conclusion, $\sigma,\kappa\vdash n\Rightarrow v$, denotes that the result of
\code{interp($n$, $\sigma$, $\kappa$)} is $v$. The result of \code{interp($n$,
$\sigma$, $\kappa$)} is the same as that of \code{continue($\kappa$, NumV($n$))}.
$n\mapsto\kappa\Downarrow v$ denotes that the result of \code{continue($\kappa$,
NumV($n$))} is $v$.

\[
  \inferrule
  { x\in\dom{\sigma} \\ \sigma(x)\mapsto\kappa\Downarrow v }
  { \sigma,\kappa\vdash x\Rightarrow v }
  \quad\textsc{[Interp-Id]}
\]

\[
  \inferrule
  { \langle\lambda x.e,\sigma\rangle\mapsto\kappa\Downarrow v }
  { \sigma,\kappa\vdash \lambda x.e\Rightarrow v }
  \quad\textsc{[Interp-Fun]}
\]

The rules for variables and functions are similar to the rule for integers.

\[
  \inferrule
  { \sigma,[\square+(e_2,\sigma)]::\kappa\vdash e_1\Rightarrow v }
  { \sigma,\kappa\vdash e_1+e_2\Rightarrow v }
  \quad\textsc{[Interp-Add]}
\]

The conclusion, $\sigma,\kappa\vdash e_1+e_2\Rightarrow v$, denotes that the
result of \code{interp(Add($e_1$,$e_2$), $\sigma$, $\kappa$)} is $v$. The result
of \code{interp(Add($e_1$,$e_2$), $\sigma$, $\kappa$)} is the same as that of
\code{interp($e_1$, $\sigma$, AddSecondK($e_2$, $\sigma$, $\kappa$))}. Note that
$[\square+(e_2,\sigma)]::\kappa$ denotes \code{AddSecondK($e_2$, $\sigma$,
$\kappa$)}.
$\sigma,\kappa'\vdash e_1\Rightarrow v$ denotes that the result of
\code{interp($e_1$, $\sigma$, $\kappa'$)} is $v$, where $\kappa'$ is
$[\square+(e_2,\sigma)]::\kappa$.

\[
  \inferrule
  { \sigma,[\square\ (e_2,\sigma)]::\kappa\vdash e_1\Rightarrow v }
  { \sigma,\kappa\vdash e_1\ e_2\Rightarrow v }
  \quad\textsc{[Interp-App]}
\]

The rule for function application is similar to the rule for addition.

\[
  \inferrule
  { \sigma[x\mapsto\kappa],\kappa\vdash e\Rightarrow v }
  { \sigma,\kappa\vdash \textsf{vcc}\ x;\ e\Rightarrow v }
  \quad\textsc{[Interp-Vcc]}
\]

The conclusion, $\sigma,\kappa\vdash \textsf{vcc}\ x;\ e\Rightarrow v$, denotes that
the result of \code{interp(Vcc($x$, $e$), $\sigma$, $\kappa$)} is $v$. The result
of \code{interp(Vcc($x$, $e$), $\sigma$, $\kappa$)} is the same as that of
\code{interp($e$, $\sigma[x\mapsto\kappa]$, $\kappa$)}. $\sigma[x\mapsto\kappa],\kappa\vdash
e\Rightarrow v$ denotes that the result of \code{interp($e$,
$\sigma[x\mapsto\kappa]$, $\kappa$)} is $v$.

Now, we define inference rules related to \code{continue}.

\begin{verbatim}
def continue(k: Cont, v: Value): Value = k match {
  case AddSecondK(e2, env, k) => interp(e2, env, DoAddK(v, k))
  case DoAddK(v1, k) => continue(k, add(v1, v))
  case AppArgK(e2, env, k) => interp(e2, env, DoAppK(v, k))
  case DoAppK(v1, k) => v1 match {
    case CloV(xv1, ev1, sigmav1) =>
      interp(ev1, sigmav1 + (xv1 -> v), k)
    case ContV(k) => continue(k, v)
  }
  case MtK => v
}
\end{verbatim}

Each case of the \code{continue} function also produces a single inference rule.
The only exception is the \code{DoAppK} case because it requires two rules: one
for the \code{CloV} case and the other for the \code{ContV} case.

\[
  \inferrule
  { \sigma,[v_1+\square]::\kappa\vdash e_2\Rightarrow v_2 }
  { v_1\mapsto[\square+(e_2,\sigma)]::\kappa\Downarrow v_2 }
  \quad\textsc{[Continue-AddSecondK]}
\]

The conclusion, $v_1\mapsto[\square+(e_2,\sigma)]::\kappa\Downarrow v_2$, denotes
that the result of \code{continue(AddSecondK($e_2$, $\sigma$, $\kappa$), $v_1$)}
is $v_2$. The result of \code{continue(AddSecondK($e_2$, $\sigma$, $\kappa$),
$v_1$)} is the same as that of \code{interp($e_2$, $\sigma$, DoAddK($v_1$,
$\kappa$))}. Note that $[v_1+\square]$ denotes \code{DoAddK($v_1$, $\kappa$)}.
$\sigma,\kappa'\vdash e_2\Rightarrow v_2$ denotes that the result of
\code{interp($e_2$, $\sigma$, $\kappa'$)} is $v_2$ where $\kappa'$ is $[v_1+\square]$.

\[
  \inferrule
  { n_1+n_2\mapsto\kappa\Downarrow v }
  { n_2\mapsto[n_1+\square]::\kappa\Downarrow v }
  \quad\textsc{[Continue-DoAddK]}
\]

The conclusion, $n_2\mapsto[n_1+\square]::\kappa\Downarrow v$, denotes that the
result of \code{continue(DoAddK(NumV($n_1$), $\kappa$),
NumV($n_2$))} is $v$. The result of \code{continue(DoAddK(NumV($n_1$),
$\kappa$), NumV($n_2$))} is the same as that of \code{continue($\kappa$,
add(NumV($n_1$), NumV($n_2$)))}. Note that \code{add(NumV($n_1$),
NumV($n_2$))} equals \code{NumV($n_1+n_2$)}.

$n_1+n_2\mapsto\kappa\Downarrow v$ denotes that the result of
\code{continue($\kappa$, NumV($n_1+n_2$))} is $v$.

\[
  \inferrule
  { \sigma,[v_1\ \square]::\kappa\vdash e\Rightarrow v_2 }
  { v_1\mapsto[\square\ (e,\sigma)]::\kappa\Downarrow v_2 }
  \quad\textsc{[Continue-AppArgK]}
\]

This rule is similar to the rule when the continuation is
$[\square+(e_2,\sigma)]$.

\[
  \inferrule
  { \sigma[x\mapsto v_2],\kappa\vdash e\Rightarrow v }
  { v_2\mapsto[\langle\lambda x.e,\sigma\rangle\ \square]::\kappa\Downarrow v }
  \quad\textsc{[Continue-DoAppK-CloV]}
\]

The conclusion, $v_2\mapsto[\langle\lambda x.e,\sigma\rangle\
\square]::\kappa\Downarrow v$, denotes that the result of
\code{continue(DoAppK(CloV($x$, $e$, $\sigma$), $\kappa$), $v_2$)} is $v$. The result of
\code{continue(DoAppK(CloV($x$, $e$, $\sigma$), $\kappa$), $v_2$)} is the same as that
of \code{interp($e$, $\sigma[x\mapsto v_2]$, $\kappa$)}. $\sigma[x\mapsto
v_2],\kappa\vdash e\Rightarrow v$ denotes that the result of \code{interp($e$,
$\sigma[x\mapsto v_2]$, $\kappa$)} is $v$.

\[
  \inferrule
  { v_2\mapsto\kappa_1\Downarrow v }
  { v_2\mapsto[\kappa_1\ \square]::\kappa\Downarrow v }
  \quad\textsc{[Continue-DoAppK-ContV]}
\]

The conclusion, $v_2\mapsto[\kappa_1\ \square]::\kappa\Downarrow v$, denotes that
the result of \code{continue(DoAppK(ContV($\kappa_1$), $\kappa$), $v_2$)}
is $v$. The result of \code{continue(DoAppK(ContV($\kappa_1$),
$\kappa$), $v_2$)} is the same as that of \code{continue($\kappa_1$, $v_2$)}.
$v_2\mapsto\kappa_1\Downarrow v$ denotes that the result of
\code{continue($\kappa_1$, $v_2$)} is $v$.

\[
  v\mapsto[\square]\Downarrow v
  \quad\textsc{[Continue-MtK]}
\]

The conclusion, $v\mapsto[\square]\Downarrow v$, denotes that the result of
\code{continue(MtK, $v$)} is $v$. The result of \code{continue(MtK, $v$)} is
actually $v$.

\section{Exercises}

\begin{enumerate}

\item This exercise asks you to implement an interpreter of \textsf{FAE}
    with pairs in CPS without first-class functions.

\begin{verbatim}
sealed trait Expr
...
case class Pair(f: Expr, s: Expr) extends Expr
case class Fst(p: Expr) extends Expr
case class Snd(p: Expr) extends Expr

sealed trait Value
...
case class PairV(f: Value, s: Value) extends Value

sealed trait Cont
...
case class PairSecondK(???) extends Cont
case class DoPairK(???) extends Cont
case class DoFstK(???) extends Cont
case class DoSndK(???) extends Cont

def continue(k: Cont, v: Value): Value = k match {
  ...
  case PairSecondK(???) => ???
  case DoPairK(???) => ???
  case DoFstK(???) => ???
  case DoSndK(???) => ???
}

def interp(e: Expr, env: Env, k: Cont): Value = e match {
  ...
  case Pair(f, s) => ???
  case Fst(p) => ???
  case Snd(p) => ???
}
\end{verbatim}

\code{Pair(e1, e2)} creates a new pair; \code{Fst(e)} gives the first value of a
pair; \code{Snd(e)} gives the second value of a pair.  Fill every \code{???}
to complete the implementation.

\end{enumerate}
