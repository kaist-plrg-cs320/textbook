\setchapterpreamble[u]{\margintoc}
\chapter{Functions}
\labch{functions}

This section focuses on use of functions in functional programming.
In functional programming, functions are first-class. First-class functions
allow programmers to abstract complex computation easily.
This section explains what first-class functions are.
In addition, anonymous functions and closures, which are related to first-class
functions, will be introduced. To show the power of first-class functions, we
will re-implement the functions in \refch{immutability} (\code{inc1},
\code{square}, etc.) with first-class functions.

\section{First-Class Functions}

An entity in a programming language is \textit{first-class}\index{first-class} if it satisfies the
following conditions:

\begin{itemize}
\item It can be an argument of a function call.
\item It can be a return value of a function.
\item A variable can refer to it.
\end{itemize}

Anything that is first-class can be used as a value. Functions are highly important and
treated as values in functional languages.
Functions that are first-class are called \textit{first-class functions}.\index{first-class function}

Some people use the term higher-order functions. \textit{Higher-order
functions}\index{higher-order function} are
functions that are not first-order, where first-order functions neither take
functions as arguments nor return functions. Therefore, higher-order functions
can take functions as arguments and return functions. Strictly speaking, they
are different from first-class functions because first-class functions are
functions that can be passed as arguments or returned from functions.
However, any languages that support first-class functions support higher-order
functions and vice versa.
The reason is obvious: to pass first-class functions as arguments, there should
be higher-order functions, and to pass functions to higher-order functions,
there should be first-class functions.
Consequently, in most contexts, people do not distinguish
first-class functions and higher-order functions, and you can consider
first-class functions and higher-order functions as exchangeable terms.

Now, let us see how we can use first-class functions in Scala with some code
examples.

\begin{verbatim}
def f(x: Int): Int = x
def g(h: Int => Int): Int = h(0)

assert(g(f) == 0)
\end{verbatim}

The function \code{g} has one parameter \code{h}. The type of \code{h} is \code{Int => Int}.
An argument passed to \code{g} is a function that receives one integer
and returns an integer. In Scala, \code{=>} expresses the types
of functions. Functions without parameters have types of the form \code{() => [return type]}.
\code{[parameter type] => [return type]} is the type of a function with a
single parameter. Parentheses are required to express the types of functions
with two or more parameters:
\code{([parameter type], … ) => [return type]}. The function \code{f}
has one integer parameter and returns an integer, i.e. its type is \code{Int =>
Int}. Thus, it can be an argument for
\code{g}. Evaluating \code{g(f)} equals evaluating \code{f(0)}, which results
in \code{0}.

\begin{verbatim}
def f(y: Int): Int => Int = {
  def g(x: Int): Int = x
  g
}

assert(f(0)(0) == 0)
\end{verbatim}

The function \code{f} returns the function \code{g}. Since the return type of \code{f}
is \code{Int => Int}, its return value must be a function that takes an integer
as an argument and returns an integer. \code{g} satisfies the condition. \code{f(0)}
is the same as \code{g} and therefore is a function. \code{f(0)(0)} equals \code{g(0)},
which returns \code{0}.

\begin{verbatim}
val h0 = f(0)

assert(h0(0) == 0)
\end{verbatim}

A variable can refer to \code{f(0)}. \code{h0} refers to the return value of
\code{f(0)} and has type \code{Int => Int}. Calling variables referring to
function values is possible. \code{h0(0)} is a valid expression and results in
\code{0}.

\begin{verbatim}
val h1 = f
\end{verbatim}
\vspace{-1em}
\begin{mdframed}[hidealllines=true,backgroundcolor=red!10,innerleftmargin=3pt,innerrightmargin=3pt,leftmargin=-3pt,rightmargin=-3pt]
\begin{verbatim}
         ^
error: missing argument list for method f

Unapplied methods are only converted to functions
when a function type is expected.

You can make this conversion explicit
by writing `f _` or `f(_)` instead of `f`.
\end{verbatim}
\vspace{-2em}
\begin{flushright}
\scriptsize\textsf{Compile-time error}
\end{flushright}
\end{mdframed}

On the other hand, defining a variable referring to \code{f} results in a compile
error. In Scala, a function defined by \code{def} is not a value per se. Since \code{f}
is the name of a function but not a variable referring to a value, \code{h1}
cannot refer to the value of \code{f}. As the above error message implies,
underscores convert function names into function values.

\begin{verbatim}
val h1 = f _

assert(h1(0)(0) == 0)
\end{verbatim}

Compiling the above code succeeds. The type of \code{h1} is \code{Int => (Int => Int)}.
\code{Int => Int => Int} denotes the same type because \code{=>} is
a right-associative type operator. \code{h1(0)(0)} is valid and yields \code{0}.

Actually, above expressions except \code{val h1 = f} use function names as values
successfully. The Scala compiler transforms function names into function values
when they occur where function types are expected. Therefore, enforcing the type of
\code{h1} to be a function type corrects the code without the underscore. The
following code works well:

\begin{verbatim}
val h1: Int => Int => Int = f
\end{verbatim}

When programmers use function names as values, they usually place the names where
function types are expected. In these cases, underscores and explicit type
annotations are unnecessary. Code rarely becomes problematic and needs
underscores or type annotations like the above to enforce the transformations.

How does the compiler create function values from function names? If the parameter
type of function \code{f} is \code{Int}, the corresponding function value is
\code{(x: Int) => f(x)}. The transformation is called \textit{eta expansion}.
\index{eta expansion}
\code{(x: Int) => f(x)} is a function value without a name and does the same thing as
\code{f}. The following section covers functions without names.

\section{Anonymous Functions}

In functional programming, functions often appear only once as an argument or a
return value. Naming functions used only once is unnecessary. The meaning of
a function value is how it behave. While the parameters and body of a function
decide its behavior, its name does not affect the behavior. Naturally, functional
languages provide syntax to define functions without giving them names. Such
functions are \textit{anonymous functions}.\index{anonymous function}

The syntax of an anonymous function in Scala is as follows:

\begin{verbatim}
([parameter name]: [parameter type], …) => [expression]
\end{verbatim}

Like functions declared by \code{def},
anonymous functions can be arguments, return values, or values referred by
variables. Directly calling them is possible as well.

\begin{verbatim}
def g(h: Int => Int): Int = h(0)
g((x: Int) => x)

def f(): Int => Int = (x: Int) => x
f()(0)

val h = (x: Int) => x
h(0)

((x: Int) => x)(0)
\end{verbatim}

The code does similar things to the previous code but uses anonymous functions.

Anonymous functions need explicit parameter types as named functions do. However,
annotating every parameter type is verbose and inconvenient. The Scala compiler
infers the types of parameters when anonymous functions occur where the compiler
expects function types.

\begin{verbatim}
def g(h: Int => Int): Int = h(0)
g(x => x)
\end{verbatim}

Since \code{g} has a parameter of type \code{Int => Int}, the compiler expects
\code{x => x} to have the type \code{Int => Int}. It infers the type of \code{x} as
\code{Int}.

\begin{verbatim}
val h: Int => Int = x => x
\end{verbatim}

\code{h} has an explicit type annotation. \code{Int => Int} is the expected type of
\code{x => x}. The compiler infers the type of \code{x} as \code{Int}.

\begin{verbatim}
val h = x => x
\end{verbatim}
\vspace{-1em}
\begin{mdframed}[hidealllines=true,backgroundcolor=red!10,innerleftmargin=3pt,innerrightmargin=3pt,leftmargin=-3pt,rightmargin=-3pt]
\begin{verbatim}
         ^
error: missing parameter type
\end{verbatim}
\vspace{-2em}
\begin{flushright}
\scriptsize\textsf{Compile-time error}
\end{flushright}
\end{mdframed}

Unlike previous one, this code is problematic. Since
there is no information to infer the type of \code{x} in \code{x => x},
the compiler generates an error.

Most cases using anonymous functions are arguments for function calls.
Those functions do not require explicit parameter types. However, beginners might not
be sure about whether parameter types can be omitted or not. Specifying
parameter types is safe when you are not sure.

Scala provides one more syntax for anonymous functions: syntax using
underscores. Underscores help programmers to create anonymous functions in a
concise and intuitive way.
Underscores can be used only when certain conditions are satisfied.
Every parameter must occur exactly once in the body of a function in the
order. Moreover, the function must not be an identity function like \code{(x: Int) => x}.
In functions satisfying the conditions, underscores can replace parameters in the body.
Otherwise, it is impossible to use underscores to create anonymous functions.

\begin{verbatim}
def g0(h: Int => Int): Int = h(0)
g0(_ + 1)

def g1(h: (Int, Int) => Int): Int = h(0, 0)
g1(_ + _)
\end{verbatim}

The compiler transforms \verb!_ + 1! into \code{x => x + 1}.
Similarly, \verb!_ + _!
becomes \code{(x, y) => x + y}. The compiler automatically creates parameters
as many as underscores and substitutes the underscores with the parameters. The
mechanism clearly shows why the aforementioned conditions exist.

\begin{verbatim}
val h0 = (_: Int) + 1
val h1 = (_: Int) + (_: Int)
\end{verbatim}

Underscores can have explicit types.
Programmers should supply parameter types to succeed compiling when
the compiler cannot infer them.

The transformation happens for the shortest expression containing underscores.
Expressing anonymous functions with underscores is sometimes tricky.

\begin{verbatim}
def f(x: Int): Int = x
def g1(h: Int => Int): Int = h(0)
g1(f(_))
\end{verbatim}

As intended, \verb!f(_)! becomes \code{x => f(x)}, whose type is \code{Int =>
Int}.\sidenote{Actually, there is no need to write
\code{g(f(\_))} because it is equal to \code{g(f)}.}

\begin{verbatim}
g1(f(_ + 1))
\end{verbatim}
\vspace{-1em}
\begin{mdframed}[hidealllines=true,backgroundcolor=red!10,innerleftmargin=3pt,innerrightmargin=3pt,leftmargin=-3pt,rightmargin=-3pt]
\begin{verbatim}
     ^
error: missing parameter type for expanded function
((<x$1: error>) => x$1.$plus(1))
\end{verbatim}
\vspace{-2em}
\begin{flushright}
\scriptsize\textsf{Compile-time error}
\end{flushright}
\end{mdframed}

On the other hand, \verb!f(_ + 1)! becomes
\code{f(x => x + 1)} but not \code{x => f(x + 1)}.
As \code{f} takes an integer, not a function, it results in a compile-time error.

\begin{verbatim}
def g2(h: (Int, Int) => Int): Int = h(0, 0)
g2(f(_) + _)
\end{verbatim}

\verb!f(_) + _! becomes \code{(x, y) => f(x) + y}, whose type is \code{(Int,
Int) => Int}, and the compilation succeeds.

\begin{verbatim}
g2(f(_ + 1) + _)
\end{verbatim}
\vspace{-1em}
\begin{mdframed}[hidealllines=true,backgroundcolor=red!10,innerleftmargin=3pt,innerrightmargin=3pt,leftmargin=-3pt,rightmargin=-3pt]
\begin{verbatim}
              ^
error: missing parameter type for expanded function
((<x$2: error>) => f(((<x$1: error>) => x$1.$plus(1)))
  .<$plus: error>(x$2))
\end{verbatim}
\vspace{-2em}
\begin{flushright}
\scriptsize\textsf{Compile-time error}
\end{flushright}
\end{mdframed}

\verb!f(_ + 1) + _! becomes
\code{y => f(x => x + 1) + y} but not \code{(x, y) => f(x + 1) + y}.

Like type inference of parameter types, novices may not be sure about how
anonymous functions with underscores are transformed. It is recommended to use normal anonymous
functions without underscores for those who are not confident about the mechanism
of underscores.

\section{Closures}

\textit{Closures}\index{closure} are function values that capture
environments, which store the values of existing variables, when they are defined.
The bodies of closures may
have variables not defined in themselves, and the environments store the values
of those variables.

\begin{verbatim}
def makeAdder(x: Int): Int => Int = {
  def adder(y: Int): Int = x + y
  adder
}
\end{verbatim}

The definition of \code{adder}, \code{def adder(y: Int): Int = x + y}, does not
define but uses \code{x}. However, the code is correct.

\begin{verbatim}
val add1 = makeAdder(1)
assert(add1(2) == 3)

val add2 = makeAdder(2)
assert(add2(2) == 4)
\end{verbatim}

\code{add1} and \code{add2} refer to the same \code{adder} function, but the
former returns an integer one larger than an argument, and the latter returns an
integer two larger than an argument. The results of \code{add1(2)} and
\code{add2(2)} are \code{3} and \code{4}, respectively. It is possible because the
closures capture the environments when they are created. \code{add1} refers to a
thing like \code{(adder, x = 1)} instead of just \code{adder}. Similarly,
\code{add2} is actually \code{(adder, x = 2)}. Since the environment of
\code{add1} stores the fact that \code{x} is \code{1}, \code{add1(2)} results in
\code{3}. Under the environment of \code{add2}, \code{x} denotes \code{2}, and
thus \code{x + y} is \code{4} when \code{y} is \code{2}.

\section{First-Class Functions and Lists}

This section shows how first-class functions allow generalization of the functions
defined in \refch{immutability}.

\begin{verbatim}
def inc1(l: List[Int]): List[Int] = l match {
  case Nil => Nil
  case h :: t => h + 1 :: inc1(t)
}

def square(l: List[Int]): List[Int] = l match {
  case Nil => Nil
  case h :: t => h * h :: square(t)
}
\end{verbatim}

\code{inc1} increases every element of a given list by one, and \code{square}
squares every element. The two functions are remarkably similar. To make the
similarity clearer, let us rename the functions to \code{g}.

\begin{verbatim}
def g(l: List[Int]): List[Int] = l match {
  case Nil => Nil
  case h :: t => h + 1 :: g(t)
}

def g(l: List[Int]): List[Int] = l match {
  case Nil => Nil
  case h :: t => h * h :: g(t)
}
\end{verbatim}

The only difference is the left operand of \code{::} in the third line:
\code{h + 1} versus \code{h * h}. By adding one parameter, the functions become
entirely identical.

\begin{verbatim}
def g(l: List[Int], f: Int => Int): List[Int] = l match {
  case Nil => Nil
  case h :: t => f(h) :: g(t, f)
}
g(l, h => h + 1)

def g(l: List[Int], f: Int => Int): List[Int] = l match {
  case Nil => Nil
  case h :: t => f(h) :: g(t, f)
}
g(l, h => h * h)
\end{verbatim}

This function is called \code{map}. The returned list
has elements obtained by \textbf{map}ping a given function to the elements of a
given list.

\begin{verbatim}
def map(l: List[Int], f: Int => Int): List[Int] = l match {
  case Nil => Nil
  case h :: t => f(h) :: map(t, f)
}
\end{verbatim}

\code{inc1} and \code{square} can be redefined with \code{map}.

\begin{verbatim}
def inc1(l: List[Int]): List[Int] = map(l, h => h + 1)
def square(l: List[Int]): List[Int] = map(l, h => h * h)
\end{verbatim}

An underscore makes \code{inc1} conciser.

\begin{verbatim}
def inc1(l: List[Int]): List[Int] = map(l, _ + 1)
\end{verbatim}

Let us compare \code{odd} and \code{positive}.

\begin{verbatim}
def odd(l: List[Int]): List[Int] = l match {
  case Nil => Nil
  case h :: t =>
    if (h % 2 != 0)
      h :: odd(t)
    else
      odd(t)
}

def positive(l: List[Int]): List[Int] = l match {
  case Nil => Nil
  case h :: t =>
    if (h > 0)
      h :: positive(t)
    else
      positive(t)
}
\end{verbatim}

They look similar. They can become identical by renaming and adding parameters.

\begin{verbatim}
def filter(l: List[Int], f: Int => Boolean): List[Int] = l match {
  case Nil => Nil
  case h :: t =>
    if (f(h))
      h :: filter(t, f)
    else
      filter(t, f)
}
\end{verbatim}


The function is called \code{filter} because it \textbf{filter}s
unwanted elements out from a given list.

\code{odd} and \code{positive} can be redefined with \code{filter}.

\begin{verbatim}
def odd(l: List[Int]): List[Int] =
  filter(l, h => h % 2 != 0)
def positive(l: List[Int]): List[Int] =
  filter(l, h => h > 0)
\end{verbatim}

Underscores make the functions conciser.

\begin{verbatim}
def odd(l: List[Int]): List[Int] = filter(l, _ % 2 != 0)
def positive(l: List[Int]): List[Int] = filter(l, _ > 0)
\end{verbatim}

Let us compare \code{sum} and \code{product} without tail recursion.

\begin{verbatim}
def sum(l: List[Int]): Int = l match {
  case Nil => 0
  case h :: t => h + sum(t)
}

def product(l: List[Int]): Int = l match {
  case Nil => 1
  case h :: t => h * product(t)
}
\end{verbatim}

After renaming the names to \code{g}, two differences exist: \code{0} versus
\code{1} and \code{h + g(t)} versus \code{h * g(t)}. By adding two parameters, an
initial value and a function taking \code{h} and \code{g(t)} as arguments, the
functions become identical.

\begin{verbatim}
def foldRight(
  l: List[Int],
  n: Int,
  f: (Int, Int) => Int
): Int = l match {
  case Nil => n
  case h :: t => f(h, foldRight(t, n, f))
}
\end{verbatim}

This function is called \code{foldRight} since it
appends an initial value at the right side of a list and
\textbf{fold}s the list from the \textbf{right} side with a given function.

\code{sum} and \code{product} can be redefined with \code{foldRight}.

\begin{verbatim}
def sum(l: List[Int]): Int =
  foldRight(l, 0, (h, gt) => h + gt)
def product(l: List[Int]): Int =
  foldRight(l, 1, (h, gt) => h * gt)
\end{verbatim}

They may use underscores for conciseness.

\begin{verbatim}
def sum(l: List[Int]): Int = foldRight(l, 0, _ + _)
def product(l: List[Int]): Int = foldRight(l, 1, _ * _)
\end{verbatim}

The following equations give an intuitive interpretation of \code{foldRight}:

\begin{verbatim}
  foldRight(List(a, b, .., y, z), n, f)
= f(a, f(b, .. f(y, f(z, n)) .. ))

  foldRight(List(1, 2, 3), 0, add)
= add(1, add(2, add(3, 0)))

  foldRight(List(1, 2, 3), 1, mul)
= mul(1, mul(2, mul(3, 1)))
\end{verbatim}

Let us compare tail-recursive \code{sum} and \code{product}.

\begin{verbatim}
def sum(l: List[Int]): Int = {
  def aux(l: List[Int], inter: Int): Int = l match {
    case Nil => inter
    case h :: t => aux(t, inter + h)
  }
  aux(l, 0)
}

def product(l: List[Int]): Int = {
  def aux(l: List[Int], inter: Int): Int = l match {
    case Nil => inter
    case h :: t => aux(t, inter * h)
  }
  aux(l, 1)
}
\end{verbatim}

After renaming, there are two differences: \code{inter + h} versus \code{inter * h}
and \code{0} versus \code{1}. Similarly, adding two parameters makes the
functions identical.

\begin{verbatim}
def foldLeft(
  l: List[Int],
  n: Int,
  f: (Int, Int) => Int
): Int = {
  def aux(l: List[Int], inter: Int): Int = l match {
    case Nil => inter
    case h :: t => aux(t, f(inter, h))
  }
  aux(l, n)
}
\end{verbatim}

This function is called \code{foldLeft}. Its semantics is different from
\code{foldRight}. While \code{foldRight} appends an initial value at
the right side and folds a list from the right side, \code{foldLeft}
prepends an initial value at the left side and \textbf{fold}s
a list from the \textbf{left} side. The following equations give an intuitive
interpretation:

\begin{verbatim}
  foldLeft(List(a, b, .., y, z), n, f)
= f(f( .. f(f(n, a), b), .. , y), z)

  foldLeft(List(1, 2, 3), 0, add)
= add(add(add(0, 1), 2), 3)

  foldLeft(List(1, 2, 3), 1, mul)
= mul(mul(mul(1, 1), 2), 3)
\end{verbatim}

The order traversing a list does not affect the results of \code{sum} and
\code{product}. Both \code{foldRight} and \code{foldLeft} can express the functions.

\begin{verbatim}
def sum(l: List[Int]): Int = foldLeft(l, 0, _ + _)
def product(l: List[Int]): Int = foldLeft(l, 1, _ * _)
\end{verbatim}

On the other hand, the order is important for some functions.
Consider a function that takes a list of digits as arguments and returns the
decimal number obtained by concatenating the digits.
\code{foldLeft} is the easiest way to implement this function.

\begin{verbatim}
def digitToDecimal(l: List[Int]) =
  foldLeft(l, 0, _ * 10 + _)

  foldLeft(List(1, 2, 3), 0, f)
= f(f(f(0, 1), 2), 3)
= ((0 * 10 + 1) * 10 + 2) * 10 + 3
= (1 * 10 + 2) * 10 + 3
= 12 * 10 + 3
= 123
\end{verbatim}

Using \code{foldRight} with the same arguments will yield completely different
result.

\begin{verbatim}
def digitToDecimal(l: List[Int]) =
  foldRight(l, 0, _ * 10 + _)

  foldRight(List(1, 2, 3), 0, f)
= f(1, f(2, f(3, 0)))
= 1 * 10 + (2 * 10 + (3 * 10 + 0))
= 1 * 10 + (2 * 10 + 30)
= 1 * 10 + 50
= 60
\end{verbatim}

\code{map}, \code{filter}, \code{foldRight}, and
\code{foldLeft} are powerful functions. The four functions offer concise
implementation for many procedures dealing with lists.
Since they are so useful, the Scala standard library provides \code{map},
\code{filter}, \code{foldRight}, and \code{foldLeft} as the methods of the
\code{List} class. You do not need to implement \code{map},
\code{filter}, \code{foldRight}, and \code{foldLeft} by yourself.

\code{map(l, f)} can be rewritten to \code{l.map(f)} by using the \code{map}
method instead.

\begin{verbatim}
def inc1(l: List[Int]): List[Int] = l.map(_ + 1)
def square(l: List[Int]): List[Int] = l.map(h => h * h)
\end{verbatim}

\code{filter(l, f)} can be rewritten to \code{l.filter(f)} by using the
\code{filter} method instead.

\begin{verbatim}
def odd(l: List[Int]): List[Int] = l.filter(_ % 2 != 0)
def positive(l: List[Int]): List[Int] = l.filter(_ > 0)
\end{verbatim}

\code{foldRight(l, n, f)} can be rewritten to \code{l.foldRight(n)(f)} by using the
\code{foldRight} method instead.

\begin{verbatim}
def sum(l: List[Int]): Int = l.foldRight(0)(_ + _)
def product(l: List[Int]): Int = l.foldRight(1)(_ * _)
\end{verbatim}

\code{foldLeft(l, n, f)} can be rewritten to \code{l.foldLeft(n)(f)} by using the
\code{foldLeft} method instead.

\begin{verbatim}
def sum(l: List[Int]): Int = l.foldLeft(0)(_ + _)
def product(l: List[Int]): Int = l.foldLeft(1)(_ * _)
def digitToDecimal(l: List[Int]) = l.foldLeft(0)(_ * 10 + _)
\end{verbatim}

The methods in the standard library are polymorphic, i.e. they can take
arguments of various types. For example, our \code{map} function takes only a
list of integers. To use \code{map} for a list of students, we need to define a
new version of \code{map}. However, the \code{map} method in the standard
library can take lists of any types as arguments.

\begin{verbatim}
case class Student(name: String, height: Int)

def heights(l: List[Student]): List[Int] = l.map(_.height)
\end{verbatim}

The standard library provides many other useful methods for
lists.\sidenote{\url{https://www.scala-lang.org/api/current/scala/collection/immutable/List.html}}

\section{For Loops}

Scala has for loops.
In Scala, a for loop is an expression, which evaluates to a value.
For expressions are highly expressive.
Unlike \code{while}, which work with mutable variables or objects,
\code{for} of Scala helps programmers to write code in a functional and readable way.

The syntax of a for expression is as follows:

\begin{verbatim}
for ([name] <- [expression])
  yield [expression]
\end{verbatim}

The first expression should result in a collection.
Its result is a collection containing the result of evaluating the second expression
at each iteration.
Therefore, for expressions can replace use of the \code{map} method.

\begin{verbatim}
val l = for (n <- List(0, 1, 2)) yield n * n
assert(l == List(0, 1, 4))
\end{verbatim}

For expressions can appear at any places expecting expressions.

\begin{verbatim}
def square(l: List[Int]): List[Int] =
  for (n <- l)
    yield n * n
\end{verbatim}

In Scala, \code{for} is just syntactic sugar. Instead of giving specific
semantics to \code{for}, syntactic rules transform code using \code{for} into the
code using methods of collections and anonymous functions. The above function becomes
the following function, which uses \code{map}, by the transformation:

\begin{verbatim}
def square(l: List[Int]): List[Int] =
  l.map(n => n * n)
\end{verbatim}

For this reason, for expressions are powerful. Any
user-defined types can appear in for expressions if the
types define \code{map}.

For expressions can replace use of the \code{filter} method as well.

\begin{verbatim}
def positive(l: List[Int]): List[Int] =
  for (n <- l if n > 0)
    yield n
\end{verbatim}

Elements not satisfying a given condition will be omitted during iteration.

In addition, combination of \code{map} and \code{filter} can be expressed with a
for loop concisely. Consider a function that takes a list of students and
returns a list of the names of students whose height is greater than 170.
The function can be implemented with \code{map} and \code{filter} like below.

\begin{verbatim}
def tall(l: List[Student]): List[String] =
  l.filter(_.height > 170).map(_.name)
\end{verbatim}

We can use a for expression instead.

\begin{verbatim}
def tall(l: List[Student]): List[String] =
  for (s <- l if s.height > 170)
    yield s.name
\end{verbatim}

\section{Exercises}

\begin{exercise}
\labex{functions-incby}

Implement a function \code{incBy}:
\begin{verbatim}
def incBy(l: List[Int], n: Int): List[Int] = ???
\end{verbatim}
that takes a list of integers and an integer as
arguments and increases every element of the list by the given integer. Use
the \code{map} method.

\end{exercise}

\begin{exercise}
\labex{functions-gt}

Implement a function \code{gt}:
\begin{verbatim}
def gt(l: List[Int], n: Int): List[Int] = ???
\end{verbatim}
that takes a list of integers and an integer as arguments
and filters elements less than or equal to the given integer out from the list.
Use the \code{filter} method.

\end{exercise}

\begin{exercise}
\labex{functions-append}

Implement a function \code{append}:
\begin{verbatim}
def append(l: List[Int], n: Int): List[Int] = ???
\end{verbatim}
that takes a list of integers and an integer as arguments
and returns a list obtained by appending the integer at the end of the list.
Use the \code{foldRight} method.

\end{exercise}

\begin{exercise}
\labex{functions-reverse}

Implement a function \code{reverse}:
\begin{verbatim}
def reverse(l: List[Int]): List[Int] = ???
\end{verbatim}
that takes a list of integers
and returns a list obtained by reversing the order between the elements.
Use the \code{foldLeft} method.

\end{exercise}
