\setchapterpreamble[u]{\margintoc}
\chapter{Mutable Variables}
\labch{mutable-variables}

\renewcommand{\plang}{\textsf{FAE}\xspace}
\newcommand{\bfae}{\textsf{BFAE}\xspace}
\renewcommand{\lang}{\textsf{MFAE}\xspace}

\bfae of the previous chapter provides boxes. Boxes are good abstraction
of mutable objects and data structures but do not explain mutable variables
well. Boxes, mutable objects, mutable data structures are values, while mutable
variables are names. Mutable variables allow the values associated with
names to change. We can find the notion of a mutable variable in many
real-world languages except a few functional languages including OCaml and
Haskell.

The semantics of mutable variables seem trivial. We can change the values of
mutable variables. However, if we use mutable variables with closures, we can do
many interesting things. Consider the following Scala program:

\begin{verbatim}
def makeCounter(): () => Int = {
  var x = 0
  def counter(): Int = {
    x += 1
    x
  }
  counter
}

val counter1 = makeCounter()
val counter2 = makeCounter()

println(counter1())
println(counter2())
println(counter1())
println(counter2())
\end{verbatim}

The program defines the function \code{makeCounter}. The function has a mutable
variable \code{x} whose initial value is \code{0}. Also, it defines and returns the function
\code{counter}. \code{counter} increases the value of \code{x} by
one every time it is called. We make two counters by calling \code{makeCounter}
twice. Then, we call each counter in turn and print the return value.
What does the program print? The first value will be \code{1} since
\code{counter1} will increase \code{x} by one from zero and return \code{x}.
However, predicting the other ones is difficult. We need the exact semantics of
mutable variables to answer the question.

This chapter defines \lang by extending \plang with mutable variables.
We will see the semantics of mutable variables. Addition of mutable
variables gives us a chance to explore a different design of the function
application semantics. We will see what is the call-by-reference semantics and how
it differs from the call-by-value semantics.

\section{Syntax}

As variables are mutable in \lang, we need to add expressions that change the
values of variables. The following is the abstract syntax of \lang:
\sidenote{We omit the common part to \plang.}

\[ e\ ::=\ \cdots\ |\ \eset{x}{e} \]

$\eset{x}{e}$ is an expression changing the value of a variable. $x$ is the
variable to be updated; $e$ determines the new value of the variable. Unlike
$\eset{e_1}{e_2}$ in \bfae, the left-hand-side of an assignment is restricted to
a variable. The reason is that variables are not values. We cannot get a
variable by evaluating an expression. The only way to designate a variable is
to write the name of the variable, and the syntax reflects this point.

Note that \lang lacks sequencing expressions, which exist in \bfae. Actually, it
is not problematic at all. We can desugar sequencing expressions into
lambda abstractions and function applications: transform $\eseq{e_1}{e_2}$ into
$\eapp{(\efun{x}{e_2})}{e_1}$, where $x$ is not free in $e_2$. The semantics of
$\eseq{e_1}{e_2}$ is that evaluating $e_1$ first and then $e_2$. The evaluation
of $\eapp{(\efun{x}{e_2})}{e_1}$ is the same. First, $\efun{x}{e_2}$ evaluates
to a closure, which means that $e_2$ is not evaluated. Then, the argument,
$e_1$, is evaluated. Finally, the body of the closure, $e_2$, is evaluated. Therefore,
$e_1$ is evaluated before $e_2$. Also, since $x$ is not a free identifier in
$e_2$, the result of $e_1$ is never used even though it is passed to the
function as an argument. Thus, we can conclude that the desugaring is correct.
We may use sequencing expressions in examples as they can be easily desugared.

\section{Semantics}

Since \lang provides mutation, its semantics uses store-passing just like \bfae.
Therefore, a store is a finite partial function from addresses to values.

\[ \embox{Sto} = \embox{Addr}\finto V \]
\[ M \in \embox{Sto} \]

The semantics is a
relation over $\embox{Env}$, $\embox{Sto}$, $E$, $V$, and $\embox{Sto}$.

\[\Rightarrow\subseteq\embox{Env}\times\embox{Sto}\times E\times V\times\embox{Sto}\]

Like \plang, a value of \lang is either an integer or a closure. It is different
from \bfae, which allows addresses to be values. \bfae treats addresses as
values because expressions creating boxes evaluate to addresses. However, there
are mutable variables instead of boxes in \lang. \lang has addresses to support
mutation, but they are used only for tracking the value of each variable. Addresses
are not exposed to programmers as values.

An environment of \lang is a finite partial function from identifiers to addresses,
but not values.

\[ \embox{Env}=\embox{Id}\finto\embox{Addr} \]
\[ \sigma \in \embox{Env} \]

The semantics needs environments to find the value denoted by
a variable. \plang, whose variables are immutable, is
satisfied with environments that take identifiers as input and return values.
However, variables of \lang are mutable. Evaluation outputs a value and a store.
Environments are not the output of evaluation. Therefore, we cannot use
environments to record changes in the values of variables. On the other hand, we
can use stores to record the changes as stores are output of evaluation.
It implies that stores must contain the values of variables to make variables
mutable. Since the value of a certain variable is stored at a particular address
of a store, an environment must know the address of each variable.

One may ask if we can remove environments from the semantics and consider a
store as a partial function from identifiers to values. However, in fact,
removing environments from the semantics prevents use of static scope.
Assume that the semantics lacks environments, and a store is a partial
function from an identifier to a value.
Consider $\eseq{(\eapp{(\efun{\cx}{\eset{\cx}{1}})}{0})}{\cx}$.
To evaluate the function application, the value of the argument should be
recorded in the store. After evaluating the function body, the store will
be passed to the evaluation of $\cx$. Then, $\cx$ evaluates to $1$
without a run-time error since $\cx$ is in the store.
On the contrary, under static scope, the scope of $\cx$ includes only $x:=1$.
The expression should result in a run-time error because $\cx$ outside the
function is a free identifier. Environments are essential for resolving this
problem. Environments enable static scope, and stores make variables mutable.
The semantics must have both.

Because of the change in the definition of an environment, the semantics of
identifiers need to be revised. An environment has the address of a given
identifier, and a store has the value at a given address. Therefore,
we need two steps to find the value of a variable: find the address of a variable
from the environment and find the value at the address from the store.

\semanticrule{Id}{
  \begin{tabular}{@{\hskip0pt}l@{\hskip-10pt}l}
    If \\
    & x is in the domain of $\sigma$, and \\
    & $\sigma(x)$ is in the domain of $M$,\\
    then \\
    & \sevaldn{}{x}{M(\sigma(x))}{}.
  \end{tabular}
}

\vspace{-1em}

\[
  \inferrule
  {
    x\in\dom{\sigma} \\
    \sigma(x)\in\dom{M}
  }
  { \sevald{}{x}{M(\sigma(x))}{} }
  \quad\textsc{[Id]}
\]

Like boxes in \bfae, each variable of \lang has its own address. New variables
can be defined only by function applications. Hence,
function applications are the only expressions that create new addresses.
Let us see the semantics of function applications.

\semanticrule{App}{
  \begin{tabular}{@{\hskip0pt}l@{\hskip-10pt}l}
    If \\
    & \sevaldn{}{e_1}{\clov{x}{e}{\sigma'}}{1}, \\
    & \sevaldn{1}{e_2}{v'}{2}, \\
    & $a$ is not in the domain of $M_2$, and \\
    & \sevaln{\sigma'[x\mapsto a]}{M_2[a\mapsto v']}{e}{v}{M_3}, \\
    then \\
    & \sevaldn{}{\eapp{e_1}{e_2}}{v}{3}.
  \end{tabular}
}

\vspace{-1em}

\[
  \inferrule
  {
    \sevald{}{e_1}{\clov{x}{e}{\sigma'}}{1} \\
    \sevald{1}{e_2}{v'}{2} \\
    a\not\in \dom{M_2} \\
    \seval{\sigma'[x\mapsto a]}{M_2[a\mapsto v']}{e}{v}{M_3}
  }
  { \sevald{}{\eapp{e_1}{e_2}}{v}{3} }
  \quad\textsc{[App]}
\]

The evaluation of the subexpressions is the same as \bfae. However, the remaining
procedure is different. We cannot store the value of the argument in the
environment. It should go into the store. To put the value into the store, we
need a fresh address. The name of the parameter becomes associated with the
address in the environment; the address becomes associated with the value of the
argument in the store. Finally, the function body is evaluated.

Changing the value of a variable is similar to changing the value of a box of
\bfae. However, we have to evaluate an expression to find the address of the box
to be updated in \bfae. On the other hand, we can find the address of the
variable to be updated from the environment by using its name in \lang.

\semanticrule{Set}{
  \begin{tabular}{@{\hskip0pt}l@{\hskip-10pt}l}
    If \\
    & $x$ is in the domain of $\sigma$, and \\
    & \sevaldn{}{e}{v}{1}, \\
    then \\
    & \sevaln{\sigma}{M}{\eset{x}{e}}{v}{M_1[\sigma(x)\mapsto v]}.
  \end{tabular}
}

\vspace{-1em}

\[
  \inferrule
  {
    x\in\dom{\sigma} \\
    \sevald{}{e}{v}{1}
  }
  { \seval{\sigma}{M}{\eset{x}{e}}{v}{M_1[\sigma(x)\mapsto v]} }
  \quad\textsc{[Set]}
\]

We can reuse the rules of \bfae for the other expressions.

Now, we can answer the question in the beginning of the chapter. At each call to
\code{makeCounter}, a new address is allocated to store the value of \code{x}.
Therefore, \code{x} of \code{counter1} uses a different address from \code{x} of
\code{counter2}. Both of the first two lines of \code{println} print \code{1}.
Also, each address is permanent throughout the execution. When
a call to \code{counter1} updates the value of \code{x}, the change remains
until the next call to \code{counter1}. Thus, both of the last two lines of
\code{println} print \code{2}.

\section{Interpreter}

The following Scala code implements the syntax of \lang:
\sidenote{We omit the common part to \plang.}

\begin{verbatim}
sealed trait Expr
...
case class Set(x: String, e: Expr) extends Expr
\end{verbatim}

\code{Set($x$, $e$)} represents $\eset{x}{e}$.

The types of an address and a store can be defined as in \bfae.

\begin{verbatim}
type Addr = Int
type Sto = Map[Addr, Value]
\end{verbatim}

We need to change the type of an environment.

\begin{verbatim}
type Env = Map[String, Addr]
\end{verbatim}

As in \bfae, \code{interp} takes an expression, an environment, and a store as
arguments and returns a pair of a value and a store.
\sidenote{We omit the common part to \bfae.}

\begin{verbatim}
def interp(e: Expr, env: Env, sto: Sto): (Value, Sto) =
  e match {
    ...
    case Id(x) => (sto(env(x)), sto)
    case App(f, a) =>
      val (CloV(x, b, fEnv), ls) = interp(f, env, sto)
      val (v, rs) = interp(a, env, ls)
      val addr = rs.keys.maxOption.getOrElse(0) + 1
      interp(b, fEnv + (x -> addr), rs + (addr -> v))
    case Set(x, e) =>
      val (v, s) = interp(e, env, sto)
      (v, s + (env(x) -> v))
  }
\end{verbatim}

In the \code{Id} case, the function finds the address of the variable first and
then the value at the address.

In the \code{App} case, we use the same strategy to
the interpreter of \bfae to compute a new address. The body of the function is
evaluated under the extended environment and the extended store.

The \code{Set} case uses the environment to find the address of the variable.
Then, it updates the store to change the value of the variable.

\section{Call-by-Reference}

Novices in programming often implement a swap function incorrectly. For example,
consider the following C++ program:

\begin{verbatim}
void swap(int x, int y) {
    int tmp = x;
    x = y;
    y = tmp;
}

int a = 1, b = 2;
swap(a, b);
std::cout << a << " " << b << std::endl;
\end{verbatim}

They expect the program to print \code{2 1} as \code{swap} has been called.
On the contrary, their expectation is wrong. The result is \code{1 2}.
We can explain the reason based on
the content of this chapter. When \code{swap} is called, two new fresh addresses
are allocated for \code{x} and \code{y}. The values of \code{a} and \code{b} are
copied and stored in the addresses, respectively. The function affects only the values in
the addresses of \code{x} and \code{y}. It never touches the addresses of
\code{a} and \code{b}. As a consequence, while the values of \code{x} and
\code{y} are swapped, the values of \code{a} and \code{b} are not.

This is the usual semantics of function applications. The values of arguments are
copied and saved at fresh addresses. This semantics is called
\textit{call-by-value}\index{call-by-value} (\acrshort{cbvLabel})
as function calls pass the values of arguments.

People have explored another semantics for function applications to implement
functions like \code{swap} easily. The semantics is called
\textit{call-by-reference}\index{call-by-reference} (\acrshort{cbrLabel}).
In this semantics, function calls pass the references, i.e. addresses,
when variables are used as arguments.

The following rule defines the semantics of a function application using CBR when its
argument is a variable:

\semanticrule{App-Cbr}{
  \begin{tabular}{@{\hskip0pt}l@{\hskip-10pt}l}
    If \\
    & \sevaldn{}{e}{\clov{x'}{e'}{\sigma'}}{1}, \\
    & $x$ is in the domain of $\sigma$, and \\
    & \sevaln{\sigma'[x'\mapsto\sigma(x)]}{M_1}{e'}{v}{M_2}, \\
    then \\
    & \sevaldn{}{\eapp{e}{x}}{v}{2}.
  \end{tabular}
}

\vspace{-1em}

\[
  \inferrule
  {
    \sevald{}{e}{\clov{x'}{e'}{\sigma'}}{1} \\
    x\in \dom{\sigma} \\
    \seval{\sigma'[x'\mapsto\sigma(x)]}{M_1}{e'}{v}{M_2}
  }
  { \sevald{}{\eapp{e}{x}}{v}{2} }
  \quad\textsc{[App-Cbr]}
\]

The rule does not evaluate the argument to get a value. It simply uses the
address of the variable. Then, the parameter of the function has the exactly
same address to the argument. Any change in the parameter that happens in the
function body affects the variable outside the function. We say that the
parameter is an \textit{alias}\index{alias} of the argument as they share the
same address.

Even if we want to adopt the CBR semantics in \lang, we cannot use it when the argument is not
a variable. We cannot get an address from an expression that is not a variable.
In such cases, we fall back to the CBV semantics. The following rule specifies
such cases:

\semanticrule{App-Cbv}{
  \begin{tabular}{@{\hskip0pt}l@{\hskip-10pt}l}
    If \\
    & \sevaldn{}{e_1}{\clov{x}{e}{\sigma'}}{1}, \\
    & $e_2$ is not an identifier, \\
    & \sevaldn{1}{e_2}{v'}{2}, \\
    & $a$ is not in the domain of $M_2$, and \\
    & \sevaln{\sigma'[x\mapsto a]}{M_2[a\mapsto v']}{e}{v}{M_3}, \\
    then \\
    & \sevaldn{}{\eapp{e_1}{e_2}}{v}{3}.
  \end{tabular}
}

\vspace{-1em}

\[
  \inferrule
  {
    \sevald{}{e_1}{\clov{x}{e}{\sigma'}}{1} \\
    e_2\not\in\embox{Id} \\
    \sevald{1}{e_2}{v'}{2} \\
    a\not\in \dom{M_2} \\
    \seval{\sigma'[x\mapsto a]}{M_2[a\mapsto v']}{e}{v}{M_3}
  }
  { \sevald{}{\eapp{e_1}{e_2}}{v}{3} }
  \quad\textsc{[App-Cbv]}
\]

It is the same as Rule \textsc{App} except that it has one more premise to
ensure that the argument is not a variable.

The interpreter needs the following change:

\begin{verbatim}
case App(f, a) =>
  val (CloV(x, b, fEnv), ls) = interp(f, env, sto)
  a match {
    case Id(y) =>
      interp(b, fEnv + (x -> env(y)), ls)
    case _ =>
      val (v, rs) = interp(a, env, ls)
      val addr = rs.keys.maxOption.getOrElse(0) + 1
      interp(b, fEnv + (x -> addr), rs + (addr -> v))
  }
\end{verbatim}

It uses pattern matching on the argument expression of a function application.
When it is an identifier, the CBR semantics can be used. Otherwise, it falls
back to the CBV semantics.

We can find a few languages that support CBR in real-world. One example is C++.
In C++, if there is an ampersand in front of the name of a parameter,
the parameter uses the CBR semantics. We can fix the function \code{swap} with
this feature.

\begin{verbatim}
void swap(int &x, int &y) {
    int tmp = x;
    x = y;
    y = tmp;
}

int a = 1, b = 2;
swap(a, b);
std::cout << a << " " << b << std::endl;
\end{verbatim}

It is enough to fix only the first line to make the parameters use CBR. When
\code{swap} is applied to \code{a} and \code{b}, the addresses of \code{a} and
\code{b} are passed. The address of \code{x} is the same as that of
\code{a}, and the address of \code{y} is the same as that of \code{b}.
Therefore, the function swaps not only the values of \code{x} and \code{y} but
also the values of \code{a} and \code{b}. The program prints \code{2 1} as
intended.

\section{Exercises}

\begin{exercise}
\labex{mutable-variables-cbr}

The following code is an excerpt from the implementation of the interpreter for
\lang:
\begin{verbatim}
def interp(e:Expr, env:Env, sto:Sto): (Value, Sto) =
  e match {
    ...
    case App(f, a) =>
      val (CloV(x, b, fEnv), ls) = interp(f, env, sto)
      a match {
        case Id(y) =>
          interp(b, fEnv + (x -> env(y)), ls)
        case _ =>
          val (v, rs) = interp(a, env, ls)
          val addr = rs.keys.maxOption.getOrElse(0) + 1
          interp(b, fEnv + (x -> addr), rs + (addr -> v))
      }
  }
\end{verbatim}

\begin{enumerate}
  \item What is this semantics? CBV or CBR?
\end{enumerate}

Consider the following expression:

$\begin{array}{@{}l}
  \ebind{\code{n}}{42}{ \\
  \ebind{\code{f}}{\efun{\code{g}}{\eapp{\code{g}}{\code{n}}}}{ \\
  \eapp{\code{f}}{(\efun{\cx}{\eadd{\cx}{8}})}}}
\end{array}$

\begin{enumerate}
\setcounter{enumi}{1}
\item Show the environment and store just before evaluating addition in the CBV semantics.
\item Show the environment and store just before evaluating addition in the CBR semantics.
\end{enumerate}

\end{exercise}

\begin{exercise}
\labex{mutable-variables-record}

This question extends \lang with mutable records as follows:

\[
\begin{array}{@{}rrl@{}}
  e & ::= & \cdots\ |\ \{f{:}e,\cdots,f{:}e\}\ |\ e.f \\
  v & ::= & \cdots\ |\ \{f{:}a,\cdots,f{:}a\}
\end{array}
\]

\[
\begin{array}{@{}c@{}}
\inferrule
{
  \seval{\sigma}{M_0}{e_1}{v_1}{M_1'} \quad\cdots\quad
  \seval{\sigma}{M_{n-1}}{e_n}{v_n}{M_n'} \\
  a_1\not\in\dom{M_1'} \quad\cdots\quad
  a_n\not\in\dom{M_n'} \\
  M_1=M_1'[a_1\mapsto v_1] \quad\cdots\quad
  M_n=M_n'[a_n\mapsto v_n]
}
{ \sevald{0}{\{f_1{:}e_1,\cdots,f_n{:}e_n\}}{\{f_1{:}a_1,\cdots,f_n{:}a_n\}}{n} }
  \\[2em]
\inferrule
{ \sevald{}{e}{\{\cdots,f{:}a,\cdots\}}{1} \\ a\in\dom{M_1} }
{ \sevald{}{e.f}{M_1(a)}{1} }
\end{array}
\]

Consider the following interpreter implementation of the language:
\begin{verbatim}
case class Get(r: Expr, f: String) extends Expr
case class RecV(fs: Map[String, Addr]) extends Value

def interp(e: Expr, env: Env, sto: Sto): (Value, Sto) = e match {
  ...
  case App(f, a) =>
    val (fv, fs) = interp(f, env, sto)
    fv match {
      case CloV(x, b, fenv) =>
        a match {
          case Get(r, f) => ???
          case _ =>
            val (av, as) = interp(a, env, fs)
            val addr = malloc(as)
            interp(b, fenv + (x -> addr), as + (addr -> av))
        }
      case _ => error()
    }
}
\end{verbatim}
Note that
\code{Get($e$, $f$)} is $e.f$ and \code{RecV(Map($f_1$ -> $a_1$, $\cdots$, $f_n$
-> $a_n$))} is $\{f_1{:}a_1,\cdots,f_n{:}a_n\}$.

\begin{enumerate}
  \item Fill \code{???} to complete the implementation. The
  language uses CBR when a record field is an argument of a
  function call.
\end{enumerate}

Consider the following expression:

$
\begin{array}{@{}l@{}}
  \ebind{\cx}{\{\cz{:}1\}}{ \\
  \ebind{\cf}{\efun{\cy}{\eset{\cy}{2}}}{ \\
  \eseq{\eapp{\cf}{\cx.\cz}}{ \\
  \cx.\cz}}}
\end{array}
$

\begin{enumerate}
  \setcounter{enumi}{1}
  \item Write the result and the store at the end of the evaluation of
  the above expression.
  \item Write the result and the store at the end of the evaluation of
  the above expression
  if the language uses CBV when a record field is an argument
  of a function call.
\end{enumerate}

\end{exercise}

\begin{exercise}
\labex{mutable-variables-ptr}

This exercise extends \lang with pointers.
Consider the following language:
\[
  \begin{array}{@{}r@{~}r@{~}l@{}}
    e & ::= & \cdots\ |\ \ast e \ |\ \& x\ |\ \ast\eset{e}{e} \\
    v & ::= & \cdots\ |\ a \\
  \end{array}
\]
The semantics of some constructs are as follows:
\begin{itemize}
  \item The value of $\ast e$ is the value in the store at the address denoted by
    the expression.
  \item The value of $\& x$ is the address denoted by the identifier in the environment.
  \item The evaluation of $\ast\eset{e_1}{e_2}$ evaluates $e_2$ first, which is
    the value of the whole expression. Then, it evaluates $e_1$, and it maps the
    address denoted $e_1$ to the value of $e_2$.
\end{itemize}
Write the operational semantics of the form \fbox{$\sevald{}{e}{v}{}$} for the expressions.

\end{exercise}

\begin{exercise}
\labex{mutable-variables-imp}

Consider the following language:

\vspace{-1em}

\[
\begin{array}{rcl}
  e & ::= & n\ |\ x\ |\ \eadd{e}{e} \\
  c & ::= & \eskip\ |\ \eset{x}{e}\ |\ \eifz{e}{c}{c}
  \ |\ \ewhilez{e}{c}\ |\ \eseq{c}{c} \\
  v & ::= & n \\
\end{array}
\]

where $c$ ranges over commands.

Under a given environment, $e$ evaluates to a value without modifying the
environment:

\[
  \inferrule
  {}
  { \evald{n}{n} }
  \quad
  \inferrule
  { x\in\dom{\sigma} }
  { \evald{x}{\sigma(x)} }
  \quad
  \inferrule
  { \evald{e_1}{n_1} \\ \evald{e_2}{n_2} }
  { \evald{\eadd{e_1}{e_2}}{n_1+n_2} }
\]

\begin{enumerate}
  \item
    A command takes an environment as input and produces a new environment as
    output. We write $\evald{c}{\sigma'}$ if $c$ produces $\sigma'$ when
    $\sigma$ is given.
    The following sentences describe the semantics of commands:
    \begin{itemize}
      \item $\eskip$ does not change a given environment.
      \item $\eset{x}{e}$ evaluates $e$ to get a value $v$ and
        updates the value of $x$ to $v$. $x$ does not need to be in a given
        environment.
      \item $\eifz{e}{c_1}{c_2}$ evaluates $e$ to get a value $v$.
        If $v$ is $0$, $c_1$ is evaluated. Otherwise, $c_2$ is evaluated.
      \item $\ewhilez{e}{c}$ evaluates $e$ to get a value $v$.
        If $v$ is not $0$, the command terminates without changing a given
        environment. Otherwise, it evaluates $c$ and then checks the result of $e$ again
        under the new environment.
        This process repeats until the result of $e$ becomes nonzero.
      \item $\eseq{c_1}{c_2}$ evaluates $c_1$ first to get a new environment.
        Then, it evaluates $c_2$ under the new environment.
    \end{itemize}
    Write the operational semantics of commands of the form
    \fbox{$\evald{c}{\sigma}$}.

  \item
    Draw the evaluation derivation of
    $\eseq{\eset{\cx}{0}}{\eifz{\cx}{\eskip}{\eset{\cx}{1}}}$
    under the empty environment.
\end{enumerate}

\end{exercise}
