\setchapterpreamble[u]{\margintoc}
\chapter{Mutable Boxes}
\labch{mutable-boxes}

The article defines BFAE, a language with \term{boxes}, which are mutable spaces
storing data.

\section{Syntax
}

The below is the abstract syntax of BFAE. It shows only parts not in FAE.

\[
\begin{array}{lrcl}
\text{Expression} & e & ::= & \cdots \\
&& | & \textsf{ref}\ e \\
&& | & e:=e \\
&& | & !e \\
&& | & e;e
\end{array}
\]

\(\textsf{ref}\) creates a new box. \(\textsf{ref}\ e\) evaluates \(e\), creates
a new box, and stores the result of \(e\) in the box. The whole expression
denotes the box. A \term{low-level} interpretation of a box is a memory address
where its value exists. \(\textsf{ref}\) is similar to \verb!malloc! of C or
\verb!new! of object-oriented languages because it allocates spaces whom
programmers can use in a memory. For example, rewriting \(\textsf{ref}\ 1\) in C
yields \verb!int \term{p = (int }) malloc(sizeof(int)); *p = 1; return p;!.

\(:=\) modifies the content of a box. \(e_1:=e_2\) evaluates \(e_1\) and then
\(e_2\) and puts a value denoted by \(e_2\) in a box denoted by \(e_1\). The
result of the whole expression equals the value. If \(e_1\) does not denote a
box, a run-time error occurs. The expression directly modifies the memory. It is
similar to an assignment statement whose left-hand side is a pointer
\term{dereference} in C. For example, if \(x\) denotes a box or a pointer,
\(x:=2\) equals \(*x=2\).

\(!\) opens a box. If \(e\) results in a box, \(!e\) results in the content of
the box. It is similar to a pointer dereference not at the left-hand side of an
assignment statement in C. \(!x\) equals \(*x\).

\(e_1;e_2\) is an expression obtained by \term{sequencing} \(e_1\) and \(e_2\).
It evaluates \(e_1\) and then \(e_2\). The whole expression denotes a value
denoted by \(e_2\). Languages covered by the previous articles lack expression
sequencing since it is meaningless. The result of the former expression is always
discarded without being used by the latter expression. On the other hand, in the
case of BFAE, the former expression can create a box or modify the content of a
box and is therefore meaningful although its result is ignored. Most languages
allow programmers to write sequences of multiple statements or expressions
separated by commas or line breaks. Expression sequencing of BFAE equals them.

\section{Semantics
}

Unlike the previous languages, BFAE is not a purely functional language but
provides mutable boxes. Like imperative languages, execution of a BFAE program
mutates a state. Despite mutability, the semantics of BFAE retains the functional
viewpoint, which interprets an expression as a program and executes the program
by finding a value denoted by the expression.

Defining a mutable memory is crucial to define the semantics of BFAE. The article
calls such memories \term{stores}. A store saves values in boxes existing during
execution of a program. Boxes are distinguishable as they have distinct names.
The article calls the names \term{addresses}. Let \(Addr\) be the set of every
possible address. A store is a partial function from an address to a value. A
value mapped from the address of a box by a store is the content of the box.

\[
\begin{array}{lrcl}
\text{Address} & a & \in & \mathit{Addr} \\
\text{Store} & M & \in & \text{Address}\hookrightarrow\text{Value}
\end{array}
\]

Metavariable \(a\) ranges over addresses; \(M\) ranges over stores.

The semantics does not require a concrete definition of a box. Since the address
of a box determines the meaning of the box, addresses are enough for the
semantics. The result of evaluating an expression denoting a box is an address.
For example, \(\textsf{ref}\ e\) yields an address. The previous languages allow
only integers and closures to be values, but BFAE additionally allows addresses
to be values.

\[
\begin{array}{lrcl}
\text{Value} & e & ::= & \cdots \\
&& | & a
\end{array}
\]

Note that the remaining part of the article keeps using the concept of a box.
Even though the semantics abstracts boxes with addresses, in the programmers'
perspective, boxes do exist. Besides, boxes are more intuitive than addresses for
explanations.

Evaluating \(!e\) needs not only an environment but also a store. If \(e\)
denotes a box, the store has a value stored in the box. Hence, evaluating an
expression requires a store, and \(\Rightarrow\) must be a relation over
environments, stores, expressions, and values.

Evaluating \(\textsf{ref}\ e\) creates a new box; evaluating \(e_1:=e_2\) changes
the content of a box. Both modify stores. Modifying a store differs from
extending an environment with a new variable. Evaluating \(\textsf{val}\ x=e_1\
\textsf{in}\ e_2\) adds \(x\) to the environment, but only \(e_2\) uses the
extended environment because the scope of the binding occurrence of \(x\) is
\(e_2\) but nowhere else. For instance, let \(e\) denote the expression, then
both \(e'\) and addition of \(e+e'\) do not require the extended one. On the
other hand, the modified store is unnecessary for the subexpressions of an
expression that creates or modifies a box while other parts of the program need
the modified one. Consider \(x:=2;!x\) as an example. \(!x\) must know that
\(x:=2\) has changed the content of a box denoted by \(x\) into \(2\). Therefore,
how stores change due to expressions is important. If an expression contains two
subexpressions, a store obtained by evaluating the first subexpression has to be
passed to the evaluation of the second expressions. Since the semantics needs to
yield both of the resulting value and a new store, the final correct definition
of \(\Rightarrow\) is a relation over environments, stores, expressions, values,
and stores. Evaluation reads the former store and creates the second store.


\[\Rightarrow\subseteq\text{Environment}\times\text{Store}\times\text{Expression}\times\text{Value}\times\text{Store}\]

\(\sigma,M\vdash e\Rightarrow v,M'\) implies that evaluating \(e\) under
\(\sigma\) and \(M\) results in \(v\) and creates \(M'\). The semantics uses the
\term{store passing style}. The style allows defining BFAE, featuring mutable
boxes, without any mutable concepts.

The order among subexpressions matters as the subexpressions can modify stores.
Suppose that \(x\) denotes a box, and the box contains \(1\). \( (x:=2)+(!x) \)
yields \(4\) if \(x:=2\) comes first so that \(!x\) equals \(2\). In contrast, \(
(x:=2)+(!x) \) yields \(3\) if \(!x\) comes first so that \(!x\) equals \(1\).
Inference rules naturally set orders by passing stores.

The inference rules for integers, variables, lambda abstractions equal those of
FAE but additionally require stores. They maintain the contents of given stores.

\[
\sigma,M\vdash n\Rightarrow n,M
\]

\[
\inferrule
{ x\in\mathit{Domain}(\sigma) }
{ \sigma,M\vdash x\Rightarrow \sigma(x),M }
\]

\[
\sigma,M\vdash \lambda x.e\Rightarrow \langle\lambda x.e,\sigma\rangle,M
\]

A sequenced expression per se cannot modify a given store, but its subexpressions
can. The left subexpression comes before the right one. The evaluation of the
right considers any modifications of the store made by the left.

\[
\inferrule
{ \sigma,M\vdash e_1\Rightarrow v_1,M_1 \\
  \sigma,M_1\vdash e_2\Rightarrow v_2,M_2 }
{ \sigma,M\vdash e_1;e_2\Rightarrow v_2,M_2 }
\]

The rule passes \(M_1\), obtained by evaluating the left, to the evaluation of
the right and discards the result of the left. The final result equals the result
of the right.

Sums, differences, and function applications are similar to sequenced
expressions. They cannot modify stores, but their subexpressions can. The
evaluation orders are the same as those of sequenced expressions. BFAE chooses
the left-to-right order for every expression, but orders vary with languages.

\[
\inferrule
{ \sigma,M\vdash e_1\Rightarrow n_1,M_1 \\
  \sigma,M_1\vdash e_2\Rightarrow n_2,M_2 }
{ \sigma,M\vdash e_1+e_2\Rightarrow n_1+n_2,M_2 }
\]

\[
\inferrule
{ \sigma,M\vdash e_1\Rightarrow n_1,M_1 \\
  \sigma,M_1\vdash e_2\Rightarrow n_2,M_2 }
{ \sigma,M\vdash e_1-e_2\Rightarrow n_1-n_2,M_2 }
\]

\[
\inferrule
{ \sigma,M\vdash e_1\Rightarrow \langle\lambda x.e,\sigma'\rangle,M_1 \\
  \sigma,M_1\vdash e_2\Rightarrow v_1,M_2 \\
  \sigma'\lbrack x\mapsto v_1\rbrack,M_2\vdash e\Rightarrow v_2,M_3 }
{ \sigma,M\vdash e_1\ e_2\Rightarrow v_2,M_3 }
\]

Note that the evaluation of the body of a closure can modify the store as well.

The remaining, creating a box, modifying a box, and opening a box, change or read
stores directly.

\[
\inferrule
{ \sigma,M\vdash e\Rightarrow v,M' \\
  a\not\in \mathit{Domain}(M') }
{ \sigma,M\vdash \textsf{ref}\ e\Rightarrow a,M'\lbrack a\mapsto v\rbrack }
\]

When a box is created, the address of the box must not belong to the domain of a
store attained by evaluating the subexpression. The result contains the address
and the store plus a mapping from the address to the value of the subexpression.

\[
\inferrule
{ \sigma,M\vdash e_1\Rightarrow a,M_1 \\
  \sigma,M_1\vdash e_2\Rightarrow v,M_2 }
{ \sigma,M\vdash e_1:=e_2\Rightarrow v,M_2\lbrack a\mapsto v\rbrack }
\]

Like other expressions, an expression modifying a box uses the left-to-right
order. If the left subexpression results in an address, a value associated with
the address in the store changes into the value of the right subexpression. The
whole expression denotes the value.

\[
\inferrule
{ \sigma,M\vdash e\Rightarrow a,M' \\
  a\in \mathit{Domain}(M') }
{ \sigma,M\vdash !e\Rightarrow M'(a),M' }
\]

The subexpression of an expression opening a box has to yield an address. A value
denoted by the expression is a value at the address in the store. The store
differs from a given store but is the outcome of evaluating the subexpression.
For example, \(!(\textsf{ref}\ 1)\) correctly results in \(1\) only if the
opening uses the store given by \(\textsf{ref}\ 1\).

\section{Implementing an Interpreter
}

The following Scala code implements the abstract syntax, environments, and stores
of BFAE:

\begin{verbatim}
sealed trait Expr
case class Num(n: Int) extends Expr
case class Add(l: Expr, r: Expr) extends Expr
case class Sub(l: Expr, r: Expr) extends Expr
case class Id(x: String) extends Expr
case class Fun(x: String, b: Expr) extends Expr
case class App(f: Expr, a: Expr) extends Expr
case class NewBox(e: Expr) extends Expr
case class SetBox(b: Expr, e: Expr) extends Expr
case class OpenBox(b: Expr) extends Expr
case class Seqn(l: Expr, r: Expr) extends Expr

sealed trait Value
case class NumV(n: Int) extends Value
case class CloV(p: String, b: Expr, e: Env) extends Value
case class BoxV(a: Addr) extends Value

type Env = Map[String, Value]
def lookup(x: String, env: Env): Value =
  env.getOrElse(x, throw new Exception)

type Addr = Int
type Sto = Map[Addr, Value]
def storeLookup(a: Addr, sto: Sto): Value =
  sto.getOrElse(a, throw new Exception)
def malloc(sto: Sto): Addr =
  sto.keys.maxOption.getOrElse(0) + 1
\end{verbatim}

The \verb!NewBox! class corresponds to creating a box; the \verb!SetBox! class
corresponds to modifying a box; the \verb!OpenBox! class corresponds to opening a
box; the \verb!Seqn! class corresponds to sequencing expressions. \verb!BoxV!
instances model values that are addresses. \verb!Addr! denotes the type of an
address and is \verb!Int!. \verb!Sto! is the type of a store and is a map from
\verb!Addr! to \verb!Value!. The \verb!lookup! function finds a value from a
given environment; the \verb!storeLookup! function finds from a given store. The
\verb!malloc! function computes an address unused by a given store.

\verb!interp! takes an expression, an environment, and a store as arguments and
returns the pair of a value and a store.

\begin{verbatim}
def interp(e: Expr, env: Env, sto: Sto): (Value, Sto) = e match { ... }
\end{verbatim}

I discuss the cases of the pattern matching in the same order as the inference
rules.

\begin{verbatim}
case Num(n) => (NumV(n), sto)
case Id(x) => (lookup(x, env), sto)
case Fun(x, b) => (CloV(x, b, env), sto)
\end{verbatim}

The \verb!Num!, \verb!Id!, and \verb!Fun! cases use given stores as the results.

\begin{verbatim}
case Seqn(l, r) =>
  val (_, ls) = interp(l, env, sto)
  interp(r, env, ls)
case Add(l, r) =>
  val (NumV(n), ls) = interp(l, env, sto)
  val (NumV(m), rs) = interp(r, env, ls)
  (NumV(n + m), rs)
case Sub(l, r) =>
  val (NumV(n), ls) = interp(l, env, sto)
  val (NumV(m), rs) = interp(r, env, ls)
  (NumV(n - m), rs)
case App(f, a) =>
  val (CloV(x, b, fEnv), ls) = interp(f, env, sto)
  val (v, rs) = interp(a, env, ls)
  interp(b, fEnv + (x -> v), rs)
\end{verbatim}

The \verb!Seqn!, \verb!Add!, \verb!Sub!, and \verb!App! cases do not directly
modify or read stores, but pass stores returned from the recursive calls to the
other recursive calls or use them as the results.

\begin{verbatim}
case NewBox(e) =>
  val (v, s) = interp(e, env, sto)
  val a = malloc(s)
  (BoxV(a), s + (a -> v))
\end{verbatim}

The \verb!NewBox! case calls \verb!malloc! to compute an unused address after
evaluating the subexpression. The result contains the address and the extended
store.

\begin{verbatim}
case SetBox(b, e) =>
  val (BoxV(a), bs) = interp(b, env, sto)
  val (v, es) = interp(e, env, bs)
  (v, es + (a -> v))
\end{verbatim}

The \verb!SetBox! case evaluates the two subexpressions and modifies a given
store. The result contains the result of the second subexpression and the
modified store.

\begin{verbatim}
case OpenBox(e) =>
  val (BoxV(a), s) = interp(e, env, sto)
  (storeLookup(a, s), s)
\end{verbatim}

The \verb!OpenBox! case finds a value corresponding to an address denoted by the
subexpression in a given store. The store remains unchanged.

I have covered all the cases. The following shows the whole code at once:

\begin{verbatim}
sealed trait Expr
case class Num(n: Int) extends Expr
case class Add(l: Expr, r: Expr) extends Expr
case class Sub(l: Expr, r: Expr) extends Expr
case class Id(x: String) extends Expr
case class Fun(x: String, b: Expr) extends Expr
case class App(f: Expr, a: Expr) extends Expr
case class NewBox(e: Expr) extends Expr
case class SetBox(b: Expr, e: Expr) extends Expr
case class OpenBox(b: Expr) extends Expr
case class Seqn(l: Expr, r: Expr) extends Expr

sealed trait Value
case class NumV(n: Int) extends Value
case class CloV(p: String, b: Expr, e: Env) extends Value
case class BoxV(a: Addr) extends Value

type Env = Map[String, Value]
def lookup(x: String, env: Env): Value =
  env.getOrElse(x, throw new Exception)

type Addr = Int
type Sto = Map[Addr, Value]
def storeLookup(a: Addr, sto: Sto): Value =
  sto.getOrElse(a, throw new Exception)
def malloc(sto: Sto): Addr =
  sto.keys.maxOption.getOrElse(0) + 1

def interp(e: Expr, env: Env, sto: Sto): (Value, Sto) = e match {
  case Num(n) => (NumV(n), sto)
  case Id(x) => (lookup(x, env), sto)
  case Fun(x, b) => (CloV(x, b, env), sto)
  case Seqn(l, r) =>
    val (_, ls) = interp(l, env, sto)
    interp(r, env, ls)
  case Add(l, r) =>
    val (NumV(n), ls) = interp(l, env, sto)
    val (NumV(m), rs) = interp(r, env, ls)
    (NumV(n + m), rs)
  case Sub(l, r) =>
    val (NumV(n), ls) = interp(l, env, sto)
    val (NumV(m), rs) = interp(r, env, ls)
    (NumV(n - m), rs)
  case App(f, a) =>
    val (CloV(x, b, fEnv), ls) = interp(f, env, sto)
    val (v, rs) = interp(a, env, ls)
    interp(b, fEnv + (x -> v), rs)
  case NewBox(e) =>
    val (v, s) = interp(e, env, sto)
    val a = malloc(s)
    (BoxV(a), s + (a -> v))
  case SetBox(b, e) =>
    val (BoxV(a), bs) = interp(b, env, sto)
    val (v, es) = interp(e, env, bs)
    (v, es + (a -> v))
  case OpenBox(e) =>
    val (BoxV(a), s) = interp(e, env, sto)
    (storeLookup(a, s), s)
}
\end{verbatim}

The below code evaluates \( (\lambda x.(x:=1);!x)\ (\textsf{ref}\ 2) \). The
result is \(1\), and the final store has a single box whose content is \(1\).

\begin{verbatim}
// (lambda x.(x:=1); !x) (ref 2)
interp(
  App(
    Fun("x",
      Seqn(
        SetBox(Id("x"), Num(1)),
        OpenBox(Id("x"))
      )
    ),
    NewBox(Num(2))
  ),
  Map.empty,
  Map.empty
)
// (NumV(1), Map(1 -> NumV(1)))
\end{verbatim}

\section{Exercises}

\begin{enumerate}
\item Given the following expression:

\begin{verbatim}
(lambda b1. ((lambda b2. b1 := 8; !b2) b1)) (ref 7)
\end{verbatim}

write out the arguments to and results of \texttt{interp} each time it is called during
the evaluation of a program.  Write them out in the order in which the calls to \texttt{interp}
occur during evaluation.  Use the \texttt{interp} code at the end.

Write down the environments and stores using the \verb+{}+ notation,
not using the more verbose data constructors.
Also, use an arrow notation for both the store and the environment.

An example, if the first call to \texttt{interp} were:

\begin{verbatim}
interp(Add(Id("x"), Id("y")),
       Map("x" -> NumV(3), "y" -> NumV(4)),
       Map.empty)
\end{verbatim}

then a model solution would be:

\begin{verbatim}
exp: x + y
env: {x -> NumV(3), y -> NumV(4)}
sto: {}
ans: NumV(7) {}

exp: x
env: {x -> NumV(3), y -> NumV(4)}
sto: {}
ans: NumV(3) {}

exp: y
env: {x -> NumV(3), y -> NumV(4)}
sto: {}
ans: NumV(4) {}
\end{verbatim}

Note that the environment and store read in order from left to right.

\end{enumerate}
