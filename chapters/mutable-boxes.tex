\setchapterpreamble[u]{\margintoc}
\chapter{Boxes}
\labch{boxes}

\renewcommand{\plang}{\textsf{FAE}\xspace}
\renewcommand{\lang}{\textsf{BFAE}\xspace}

Mutation is a widely-used feature. It is an important concept in imperative
languages. Even functional languages support mutation. Few languages are purely
functional, i.e. do not allow any mutation: e.g. Haskell and Coq. Muation is
important since many programs can be implemented concisely and efficiently with
mutation. At the same time, mutation often makes programs difficult to be
reasoned about and error-prone. While binding of identifiers works modularly and
allows local reasoning, mutation has a global effect on execution and enables
uncontrolled interference between distinct parts of a program. Mutation should
be used with extreme care of programmers.

This chapter introduces mutation by defining \lang, which extends \plang with
boxes. A \textit{box}\index{box} is a cell in memory that contains a single value. The value
contained in a box can be modified anytime after the creation of the box.
Boxes in \lang are higher-order. Each box can contain any value, which can be
a box or a closure, rather than only an integer.
A box itself is rarely found in real-world languages: it is almost the same as
a reference in OCaml (\code{ref}). However, it is a good abstraction of more
general mutation mechanisms including mutable objects and data
structures. We can find such concepts in most languages, and boxes are useful to
understand those concepts.

We can consider mutable objects in Scala as generalization of boxes. By
going the opposite direction, we can say that boxes can be represented as
objects. Consider the following class definition in Scala:

\begin{verbatim}
case class Box(var value: Any)
\end{verbatim}

The class \code{Box} has one field: \code{value}. Like any other classes, we can
construct instances of \code{Box} and read the fields of the instances.

\begin{verbatim}
val b = Box(3)
println(b.value)
\end{verbatim}

It prints \code{3}.
In addition, since the field \code{value} is declared as mutable, we can mutate
its value.

\begin{verbatim}
b.value = 10
println(b.value)
\end{verbatim}

It changes the value of the field to \code{10} and prints \code{10}.

\section{Syntax}

The above Scala example shows three kinds of expressions regarding boxes:
creating a new box, reading the value of a box, and changing the value of a box.
To create a box, we need an expression that determines the initial value of the
box. To read the value of a box, we need an expression that determines the box.
To change the value of a box, we need an expression that determines the box and
an expression that determines the new value of the box.

In addition, there is another kind of expression that has been implicitly
used: the sequencing expression. Usually, an expression mutating a box is useless
per se. There should be other expressions that observe the change and do
other interesting things based on the change. To do so, we need to combine
multiple expressions to form a single expression. Such an expression
is the sequencing expression.

We can define the syntax of \lang based on the observations.
The following is the syntax of \lang:
\sidenote{We omit the common part to \plang.}

\[
  e\ ::=\ \cdots\ |\ \eref{e}\ |\ \ederef{e}\ |\ \eset{e}{e}\ |\ \eseq{e}{e}
\]

\begin{itemize}
  \item $\eref{e}$

    It creates a new box, cf. \code{Box(3)} in the example.
    $e$ determines the initial value of the box.

  \item $\ederef{e}$

    It reads the value of a box, cf. \code{b.value} in the example.
    $e$ determines the box to be read.

  \item $\eset{e_1}{e_2}$

    It changes the value of a box, cf. \code{b.value = 10} in the example.
    $e_1$ determines the box to be updated; $e_2$ determines the new value.

  \item $\eseq{e_1}{e_2}$

    It is a sequencing expression. $e_1$ is the first expression to be
    evaluated; $e_2$ is the second. Many real-world languages allow sequencing
    of an arbitrary number of expressions. For brevity, \lang allows only
    sequencing of two expressions. Sequencing of multiple expressions can be
    easily expresssed by nested sequencing. For example, $e_1;e_2;e_3$ can be
    expresssed as $(e_1;e_2);e_3$.
\end{itemize}

\section{Semantics}

Defining mutable memory is crucial to define the semantics of \lang.
We call the memory of a program a \textit{store}\index{store}. A store records the values of boxes.
Each box is distinguished from another box by its address, i.e. every box has
its own address, which differs from the addresses of any other boxes.
The metavarible $a$ ranges over addresses. Let $\embox{Addr}$ be the set of every
possible address. We do not care about what $\embox{Addr}$ really is.

\[ a \in \embox{Addr} \]

A store is a finite partial function from addresses to values.
If the store maps an address $a$ to a value $v$, the value of the box whose
address is $a$ is $v$. The metavarible $M$ ranges over stores. Let $\embox{Sto}$
be the set of every store.

\[ \embox{Sto} = \embox{Addr}\finto V \]
\[ M \in \embox{Sto} \]

The semantics does not require a concrete notion of a box. Since every
box is uniquely identified by an address, the semantics can consider each
address as a box. Thus, we treat an address as a value of \lang, instead of
introducing a new semantic element denoting boxes. For example, an expression
creating a box evaluates to an address. We need to revise the definition of a
value to include addresses.
\sidenote{We omit the common part to \plang.}

\[ v\ ::=\ \cdots\ |\ a \]

Note that we keep using the concept of a box for explanation.
Even though the semantics abstracts boxes with addresses,
boxes do exist from the programmers' perspective.
The term box and the term address will be interchangeably used.

How are stores used in the semantics? First, consider an expression reading a
box. Evaluating \(\ederef{e}\) needs not only an environment but also a store.
If $e$ denotes a box, the store has the value of the box. The value becomes
the result of \(\ederef{e}\). Without a store, there is no way to find the value
of a box and yield a result. It implies that evaluation requires a store to be
given.

Now, let us consider the other kinds of expressions related to boxes.
$\eref{e}$ creates a new box; $\eset{e_1}{e_2}$ changes
the content of a box. Both modify stores. Modifying a store differs from
extending an environment with a new identifier.

A change in an environment is propagated to the subexpressions of an expression
that has caused the change. Consider $\ebind{x}{e_1}{e_2}$.
It extends the environment with $x$, but only $e_2$ uses the
extended environment because the scope of $x$ is $e_2$ but nowhere else.
A variable definition can affect only its subexpressions.
For instance, in $\eadd{(\ebind{x}{e_1}{e_2})}{e_3}$, $e_3$ does not belong to
the scope of $x$. The extended environment must be used for only $e_2$, but not
$e_3$. Therefore, we say that binding and environments are local and modular.

On the other hand, the modified store is unnecessary for the subexpressions of an
expression that modifies the store, while other parts of the program need
the modified one. Consider $\eseq{(\eset{\cx}{2})}{\ederef{\cx}}$ as an example.
Assume that $\cx$ denotes a box.
$\ederef{\cx}$ must be aware of that $\eset{\cx}{2}$ has changed the value of the box
to $2$. Otherwise, $\ederef{\cx}$ will get the previous value of the box and
produce a wrong result. Note that $\ederef{\cx}$ is not a subexpression of
$\eset{\cx}{2}$. However, in $\eset{\cx}{2}$, the evaluation of $2$, which is a
subexpression of $\eset{\cx}{2}$, must not be affected by the change in the value
of the box since the change happens after the evaluation of $2$. Therefore,
how stores change due to expressions is important. If an expression contains two
subexpressions, the store obtained by evaluating the first subexpression has to be
passed to the evaluation of the second subexpression. Stores are completely
different from environments. Any change in a store affects the entire remaining
computation. Stores are global and not modular.

From the observation, we can conclude that evaluation of an expression needs to
take a store in addition to an environment as input and output a new store along
with a result value. We can define the semantics as a
relation over $\embox{Env}$, $\embox{Sto}$, $E$, $V$, and $\embox{Sto}$.
The former store is input, and the latter store is output.

\[\Rightarrow\subseteq\embox{Env}\times\embox{Sto}\times E\times V\times\embox{Sto}\]

$\sevald{1}{e}{v}{2}$ is true if and only if \sevaldn{1}{e}{v}{2}.
We call this way of defining semantics a \textit{store-passing
style}\index{store-passing style}. The style allows defining \lang, featuring
mutability, without any mutable concepts at the meta-level.

Now, let us define the semantics of each expression. We can easily reuse Rule
\textsc{Num}, Rule \textsc{Id}, and Rule \textsc{Fun} of \lang by adding stores.
They maintain the contents of given stores.

\semanticrule{Num}{
  \sevaldn{}{n}{n}{}.
}

\vspace{-1em}

\[
  \sevald{}{n}{n}{}
  \quad\textsc{[Num]}
\]

\vspace{-1em}

\semanticrule{Id}{
  If $x$ is in the domain of $\sigma$,\\
  then \sevaldn{}{x}{\sigma(x)}{}.
}

\vspace{-1em}

\[
  \inferrule
  { x\in\dom{\sigma} }
  { \sevald{}{x}{\sigma(x)}{} }
  \quad\textsc{[Id]}
\]

\vspace{-1em}

\semanticrule{Fun}{
  \sevaldn{}{\efun{x}{e}}{\clov{x}{e}{\sigma}}{}.
}

\vspace{-1em}

\[
  \sevald{}{\efun{x}{e}}{\clov{x}{e}{\sigma}}{}
  \quad\textsc{[Fun]}
\]

During the evaluation of a certain expression,
the order of the evaluation among the subexpressions matters as they can modify the
store. Suppose that $x$ denotes a box, and the box contains $1$.
If the left operand of addition is evaluated before the right operand, in
$\eadd{(\eset{\cx}{2})}{\ederef{\cx}}$, $\ederef{\cx}$ evaluates to $2$ since it is
affected by the previous change. On the other hand, if the
right operand is evaluated first, $\ederef{\cx}$ evaluates to $1$ because
its evaluation precedes the modification and can observe only the original
value.

The order among the premises in a semantics rule does not specify the order of
evaluation. So far, we have not specified the order of evaluation in the
semantics as the order does not matter if there are no side
effects.\sidenote{Side effects are any observable behaviors of expressions except the
results. Mutation and exceptions are side effects.}
 However, \lang supports mutation, and we should specify
the order in the semantics. This goal can be naturally achieved by passing
stores. If we define the semantic to use the store that comes out from the
evaluation of the left operand as input of the evaluation of the right operand,
the order is determined to evaluate the left first.

A sequencing expression per se cannot modify a given store, but its subexpressions
can.

\semanticrule{Seq}{
  \begin{tabular}{@{\hskip0pt}l@{\hskip-10pt}l}
    If \\
    & \sevaldn{}{e_1}{v_1}{1}, and \\
    & \sevaldn{1}{e_2}{v_2}{2}, \\
    then \\
    & \sevaldn{}{\eseq{e_1}{e_2}}{v_2}{2}. \\
  \end{tabular}
}

\vspace{-1em}

\[
  \inferrule
  {
    \sevald{}{e_1}{v_1}{1} \\
    \sevald{1}{e_2}{v_2}{2} \\
  }
  { \sevald{}{\eseq{e_1}{e_2}}{v_2}{2} }
  \quad\textsc{[Seq]}
\]

$e_1$ is evaluated before $e_2$. The evaluation of $e_2$ must be aware of
any modifications of the store made by $e_1$.
For this purpose, the rule passes $M_1$, obtained by evaluating
$e_1$, to the evaluation of $e_2$. The result of $e_1$ is just discarded.
The final result is the same as the result of $e_2$.

Rule \textsc{Add}, Rule \textsc{Sub}, and Rule \textsc{App} are similar to
Rule \textsc{Seq}. They cannot modify stores, but their subexpressions can. The
evaluation order is the same as the sequencing expression. \lang chooses
the left-to-right order for every expression, but other languages
may use a different order.

\semanticrule{Add}{
  \begin{tabular}{@{\hskip0pt}l@{\hskip-10pt}l}
    If \\
    & \sevaldn{}{e_1}{n_1}{1}, and \\
    & \sevaldn{1}{e_2}{n_2}{2}, \\
    then \\
    & \sevaldn{}{\eadd{e_1}{e_2}}{n_1+n_2}{2}.
  \end{tabular}
}

\vspace{-1em}

\[
  \inferrule
  {
    \sevald{}{e_1}{n_1}{1} \\
    \sevald{1}{e_2}{n_2}{2}
  }
  { \sevald{}{\eadd{e_1}{e_2}}{n_1+n_2}{2} }
  \quad\textsc{[Add]}
\]

\vspace{-1em}

\semanticrule{Sub}{
  \begin{tabular}{@{\hskip0pt}l@{\hskip-10pt}l}
    If \\
    & \sevaldn{}{e_1}{n_1}{1}, and \\
    & \sevaldn{1}{e_2}{n_2}{2}, \\
    then \\
    & \sevaldn{}{\esub{e_1}{e_2}}{n_1-n_2}{2}.
  \end{tabular}
}

\vspace{-1em}

\[
  \inferrule
  {
    \sevald{}{e_1}{n_1}{1} \\
    \sevald{1}{e_2}{n_2}{2}
  }
  { \sevald{}{\esub{e_1}{e_2}}{n_1-n_2}{2} }
  \quad\textsc{[Sub]}
\]

\vspace{-1em}

\semanticrule{App}{
  \begin{tabular}{@{\hskip0pt}l@{\hskip-10pt}l}
    If \\
    & \sevaldn{}{e_1}{\clov{x}{e}{\sigma'}}{1}, \\
    & \sevaldn{1}{e_2}{v'}{2}, and\\
    & \sevaln{\sigma'[x\mapsto v']}{M_2}{e}{v}{M_3}, \\
    then \\
    & \sevaldn{}{\eapp{e_1}{e_2}}{v}{3}.
  \end{tabular}
}

\vspace{-1em}

\[
  \inferrule
  {
    \sevald{}{e_1}{\clov{x}{e}{\sigma'}}{1} \\
    \sevald{1}{e_2}{v'}{2} \\
    \seval{\sigma'[x\mapsto v']}{M_2}{e}{v}{M_3}
  }
  { \sevald{}{\eapp{e_1}{e_2}}{v}{3} }
  \quad\textsc{[App]}
\]

Note that the evaluation of the body of a closure can modify the store as well.

Now, let us define the semantics of expressions treating boxes.
$\eref{e}$ is an expression creating a new box. The result of $e$ becomes the
initial value of the box. The result of $\eref{e}$ is the new box.

\semanticrule{NewBox}{
  \begin{tabular}{@{\hskip0pt}l@{\hskip-10pt}l}
    If \\
    & \sevaldn{}{e}{v}{1}, and \\
    & $a$ is not in the domain of $M_1$, \\
    then \\
    & \sevaln{\sigma}{M}{\eref{e}}{a}{M_1[a\mapsto v]}.
  \end{tabular}
}

\vspace{-1em}

\[
  \inferrule
  {
    \sevald{}{e}{v}{1} \\
    a\not\in \dom{M_1}
  }
  { \seval{\sigma}{M}{\eref{e}}{a}{M_1[a\mapsto v]} }
  \quad\textsc{[NewBox]}
\]

To get the initial value, $e$ is evaluated first. The address of the new
box must not belong to $M_1$, the store attained by evaluating $e$.
There is no additional condition the address must satisfy, so we can freely
choose any address that is not in $M_1$.
Note that if we check the domain of $M$, not $M_1$, we result in multiple
boxes sharing the same address, which is certainly wrong, when $e$ also creates
boxes.
The result is the address of the box. Also, we add a mapping from the address of
the box to the value of the box to the final store.

$\ederef{e}$ is an expression reading the value of a box. $e$ determines the box
to be read. If $e$ does not evaluate to a box, a run-time error occurs.
Otherwise, $e$ is some box, and the final result is the value of the box.

\semanticrule{OpenBox}{
  \begin{tabular}{@{\hskip0pt}l@{\hskip-10pt}l}
    If \\
    & \sevaldn{}{e}{a}{1}, and \\
    & $a$ is in the domain of $M_1$, \\
    then \\
    & \sevaln{\sigma}{M}{\ederef{e}}{M_1(a)}{M_1}.
  \end{tabular}
}

\vspace{-1em}

\[
  \inferrule
  {
    \sevald{}{e}{a}{1} \\
    a\in \dom{M_1}
  }
  { \seval{\sigma}{M}{\ederef{e}}{M_1(a)}{M_1} }
  \quad\textsc{[OpenBox]}
\]

To get a box, $e$ is evaluated. The result of $e$ must be an address that
belongs to $M_1$. The rule uses $M_1$ instead
of $M$ to find the value of the box. The reason is that $e$ can create a new
box and give the box as a result. Consider $\ederef{(\eref{1})}$. If the semantics
is correct, this expression must evaluate to $1$. The initial
store is empty, but evaluating $\eref{1}$ makes the store contain the address of
the box. It means that the address can be obtained only by looking into $M_1$,
not $M$. Therefore, the correct semantics uses $M_1$ to find the value of the
box.

$\eset{e_1}{e_2}$ is an expression changing the value of a box. $e_1$ determines the box
to be updated, and $e_2$ determines the new value of the box. Just like when we
open a box, $e_1$ must evaluate to a box. Otherwise, a run-time error will happen.

\semanticrule{SetBox}{
  \begin{tabular}{@{\hskip0pt}l@{\hskip-10pt}l}
    If \\
    & \sevaldn{}{e_1}{a}{1}, and \\
    & \sevaldn{1}{e_2}{v}{2}, \\
    then \\
    & \sevaln{\sigma}{M}{\eset{e_1}{e_2}}{v}{M_2[a\mapsto v]}.
  \end{tabular}
}

\vspace{-1em}

\[
  \inferrule
  {
    \sevald{}{e_1}{a}{1} \\
    \sevald{1}{e_2}{v}{2}
  }
  { \seval{\sigma}{M}{\eset{e_1}{e_2}}{v}{M_2[a\mapsto v]} }
  \quad\textsc{[SetBox]}
\]

Like all the other expressions, an expression modifying a box uses the left-to-right
order. If $e_1$ evaluates to an address $a$, the value associated with $a$ in
the store changes into the value denoted by $e_2$. Also, the value is the result
of the whole expression. This semantics follows the semantics of many real-world
imperative languages. For example, \code{x = 1} in C changes the value of
\code{x} to \code{1} and results in \code{1}. On the other hand, functional
languages usually use unit as the results of expressions for mutation. We can
easily adopt the semantics in \lang by adding unit to the language.

\section{Interpreter}

The following Scala code implements the syntax of \lang:
\sidenote{We omit the common part to \plang.}

\begin{verbatim}
sealed trait Expr
...
case class NewBox(e: Expr) extends Expr
case class OpenBox(b: Expr) extends Expr
case class SetBox(b: Expr, e: Expr) extends Expr
case class Seqn(l: Expr, r: Expr) extends Expr
\end{verbatim}

\code{NewBox($e$)} represents $\eref{e}$;
\code{OpenBox($e$)} represents $\ederef{e}$;
\code{SetBox($e_1$, $e_2$)} represents $\eset{e_1}{e_2}$;
\code{Seqn($e_1$, $e_2$)} represents $\eseq{e_1}{e_2}$.

Addresses should be defined. We treat addresses as integers in Scala.

\begin{verbatim}
type Addr = Int
\end{verbatim}

In addition, we add a new variant of \code{Value} to represent boxes.
\sidenote{We omit the common part to \plang.}

\begin{verbatim}
sealed trait Value
...
case class BoxV(a: Addr) extends Value
\end{verbatim}

\code{Box($a$)} represents $a$.

We use a map to represent a store. The type of a store is \code{Map[Addr, Value]}.

\begin{verbatim}
type Sto = Map[Addr, Value]
\end{verbatim}

\code{interp} takes an expression, an environment, and a store as arguments and
returns a pair of a value and a store.

\begin{verbatim}
def interp(e: Expr, env: Env, sto: Sto): (Value, Sto) =
  e match {
    ...
  }
\end{verbatim}

Let us see each case of the pattern matching.

\begin{verbatim}
case Num(n) => (NumV(n), sto)
case Id(x) => (env(x), sto)
case Fun(x, b) => (CloV(x, b, env), sto)
\end{verbatim}

The \code{Num}, \code{Id}, and \code{Fun} cases use given stores as the results.

\begin{verbatim}
case Seqn(l, r) =>
  val (_, ls) = interp(l, env, sto)
  interp(r, env, ls)
case Add(l, r) =>
  val (NumV(n), ls) = interp(l, env, sto)
  val (NumV(m), rs) = interp(r, env, ls)
  (NumV(n + m), rs)
case Sub(l, r) =>
  val (NumV(n), ls) = interp(l, env, sto)
  val (NumV(m), rs) = interp(r, env, ls)
  (NumV(n - m), rs)
case App(f, a) =>
  val (CloV(x, b, fEnv), ls) = interp(f, env, sto)
  val (v, rs) = interp(a, env, ls)
  interp(b, fEnv + (x -> v), rs)
\end{verbatim}

The \code{Seqn}, \code{Add}, \code{Sub}, and \code{App} cases do not directly
modify or read stores, but pass the stores returned from the recursive calls to the
next recursive calls or use them as results.

\begin{verbatim}
case NewBox(e) =>
  val (v, s) = interp(e, env, sto)
  val a = s.keys.maxOption.getOrElse(0) + 1
  (BoxV(a), s + (a -> v))
\end{verbatim}

The \code{NewBox} case computes the initial value of the box first. Then, it
computes an address not used in the store. We use the method \code{maxOption}.
If a collection is empty, the method returns \code{None}. Otherwise, the
result is \code{Some(n)}, where \code{n} is the greatest value in the collection.
By \code{.getOrElse(0)}, we can get \code{n} from \code{Some(n)} and \code{0}
from \code{None}. Consequently, \code{sto.keys.maxOption.getOrElse(0)} results
in the maximum key in the store when the store is nonempty and \code{0}
otherwise. \code{a} is one greater than that value and thus does not belong to
the store. Therefore, we can use \code{a} as the address of the box.
The result of the function consists of the address and the extended store.

\begin{verbatim}
case OpenBox(e) =>
  val (BoxV(a), s) = interp(e, env, sto)
  (s(a), s)
\end{verbatim}

The \code{OpenBox} case evaluates the subexpression to get an address.
If the result is not a box, an exception is thrown due to a pattern matching
failure. The address is used to find the value of the box from the store.
The result of the function consists of the value of the box and the store from
the evaluation of the subexpression.

\begin{verbatim}
case SetBox(b, e) =>
  val (BoxV(a), bs) = interp(b, env, sto)
  val (v, es) = interp(e, env, bs)
  (v, es + (a -> v))
\end{verbatim}

The \code{SetBox} case evaluates both subexpressions and modifies the store.

\section{Exercises}

\begin{enumerate}
\item Consider the following expression:

\[
  \eapp{(\efun{\cx}{\eapp{(\efun{\cy}{\eseq{\eset{\cx}{8}}{\ederef{\cy}}})}{\cx}})}{\eref{7}}
\]

Write out the arguments to and results of \code{interp} each time it is called during
the evaluation of the expression. Write them out in the order in which the calls to \code{interp}
occur during evaluation.

Write down the environments and stores using the \verb+{}+ notation.
Also, use the arrow notation for both stores and environments.

For example, the solution to the following \code{interp} call is like below.

\begin{verbatim}
interp(Add(Id("x"), Id("y")),
       Map("x" -> NumV(3), "y" -> NumV(4)),
       Map.empty)
\end{verbatim}

\begin{verbatim}
exp: x + y
env: {x -> NumV(3), y -> NumV(4)}
sto: {}
res: NumV(7) {}

exp: x
env: {x -> NumV(3), y -> NumV(4)}
sto: {}
res: NumV(3) {}

exp: y
env: {x -> NumV(3), y -> NumV(4)}
sto: {}
res: NumV(4) {}
\end{verbatim}

\end{enumerate}
