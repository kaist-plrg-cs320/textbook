\setchapterpreamble[u]{\margintoc}
\chapter{Lazy Evaluation}
\labch{lazy-evaluation}

\renewcommand{\plang}{\textsf{FAE}\xspace}
\renewcommand{\lang}{\textsf{LFAE}\xspace}

This chapter is about lazy evaluation. \textit{Lazy evaluation}\index{lazy
evaluation} means delaying the evaluation of an expression until the result
is required. The opposite of lazy evaluation is \textit{eager
evaluation}\index{eager evaluation}, which evaluates an expression even if in
the case that it is unknown whether the result of the expression is necessary
for future computation. There are multiple features that lazy evaluation can be
applied to in programming languages. For example, arguments for function
applications can be evaluated lazily; each argument of a function application is
evaluated when the argument is used in the function body, not before the
evaluation of the function body starts. Another example is a variable
definition. The initialization of variable happens when the variable is
used for the first time, not when it is defined. Actually, as you have seen
already, local variables can be consdiered as syntactic sugar of function
parameters, it is enough to focus on laziness in function applications.
Therefore, this book considers lazy evaluation only on arguments of function
applications.

All the previously defined languages in the book are eager languages. They use
the CBV semantics for function applications. CBV can be consdiered equivalent to
eager evaluation. CBV means that every argument is passed as a value. The value
of an argument can be acquired only by evaluating the argument expression. It
implies that every argument is evaluated before the evaluation of the function
body regardless of whether the argument is used during the body evaluation.
Thus, the CBV semantics is equal to the eager evaluation semantics.\sidenote{One
may wonder whether CBR is eager or not. The best answer is that CBR is
irrelevant to distinction between eagerness and laziness. As shown in the previous chapter, CBR is
possible only when an argument is a variable, whose address is known. In that
case, there is nothing to evaluate. We have to make a choice between passing the
address (CBR) and passing the value at the address (CBV). On the other hand,
choosing one of eagerness and laziness is about a choice between evaluating the
argument and not evaluating the argument. CBR can be used in both eager and lazy
languages.}

This chapter mainly discusses \textit{call-by-name}\index{call-by-name}
(\acrshort{cbnLabel}) semantics, which is one form of lazy evaluation.
In the CBN semantics, each argument is passed as its name, i.e. the expression
itself, rather than the value denoted by the expression. Since it is passed as an
expression, there is no need to evaluate the expression to obatin its value. The
expression will be evalauted when the value is required by the function body.
As you can see, CBN delays the computation of arguments by passing them as
expressions and can be considered as lazy evaluation. We will define \lang, which
is a lazy version of \plang, in this chapter. \lang adopts the CBN semantics. In
addition, we will introduce call-by-need,\sidenote{Some people use the term lazy
evaluation to denote call-by-need only. In that sense, CBN is not considered as
lazy evaluation. However, this book views lazy evaluation as a term that can be
used broadly to mean any form of delayed computation. In this sense, both CBN
and call-by-need belong to lazy evaluation.} which is another form of lazy
evaluation, as an optimized version of CBN.

Before we discuss the CBN semantics and \lang, let us see why lazy evaluation is
valuable in practice. We can find lazy evaluation in a few real-world languages.
Haskell is well-known as treating every computation lazily by
default.\sidenote{Programmers can force evaluation if they really want.} On the
other hand, Scala is an eager language but allows programmers to selectively
apply lazy evaluation to their programs. We will see code examples in Scala as
it is the language used in this book, though Haskell is the most famous lazy
language. Consider the following Scala code:

\begin{verbatim}
def square(x: Int): Int = {
  Thread.sleep(5000)  // models some expensive computation
  x * x
}

def foo(x: Int, b: Boolean): Int =
  if (b) x else 0

val file = new java.io.File("a.txt")
foo(square(10), file.exists)
\end{verbatim}

The function \code{square} takes an integer as an argument and returns its
square. It always takes five seconds to return due to \code{Thread.sleep(5000)},
which makes the thread sleep for five seconds. Of course, no one will write
such code in practice, but it is an analogy of highly expensive computation that
takes a long time.

The function \code{foo} takes one integer and one boolean. It
returns the integer when the boolean is true and zero otherwise. Therefore, the
integer value is required only when the boolean is true.

The last line of the program applies \code{foo} to \code{square(10)} and
\code{file.exists}. The second argument is true if and only
if there exists a file named \code{a.txt}. If the file does not exist, \code{foo}
returns zero, and thus the value of \code{square(10)} is unnecessary.
However, as Scala uses eager evaluation by default, \code{square(10)} is
evaluated and spends five seconds regardless of the existence of the file.
If we modify the program not to evaluate \code{square(10)} when the file is
absent, we can save time in many cases without changing the behavior of the
program.

Lazy evaluation gives us an easy solution to this issue. If the first argument
for \code{foo} is evaluated lazily, the program will evaluate
\code{square(10)} only when the file exists. In Scala, we can make a certain
parameter use the CBN semantics by adding \code{=>} in front of the type of the
parameter. Thus, the following fix completely resolves the problem:

\begin{verbatim}
def foo(x: => Int, b: Boolean): Int =
  if (b) x else 0
\end{verbatim}

We call \code{x} a by-name parameter in Scala. Since \code{x} is a by-name
parameter, the first argument for \code{foo} is evaluated only when the value of
\code{x} is needed during the evaluation of the body.

\section{Semantics}

We do not explain the syntax of \lang as it is the same as \plang.
We can move on to the semantics immediately. The definition of an environment is
the same as \plang. Also, as in \plang, $\evald{e}{v}$ is true if and only if
\evaldn{e}{v}. We can use Rule \textsc{Num} of \lang since evaluation of
integers are not affected by lazy evaluation. Similarly, Rule \textsc{Fun} can
be reused as well. Rule \textsc{Id} also remains the same.
The value of an identifier can be found in an environment.

One that certainly requires a change is the semantics of function applications.
It is the most distinctive feature of lazy languages compared to eager
languages. In the CBV semantics, we store the values of arguments in
environments and use the environments to evaluate function bodies. We still need
to store arguments in environments in the CBN semantics. However, arguments are
passed as expressions, and expressions are not values. We need a way to put
expressions in environments. The simplest solution is to define a new kind of
values as follows:
\sidenote{We omit the common part to \plang.}

\[ v\ ::=\ \cdots\ |\ \exprv{e}{\sigma} \]

$\exprv{e}{\sigma}$ is an expression as a value; we call it an expression-value.
It denotes that the computation
of $e$ under $\sigma$ has been delayed. We must keep the environment together
with the expression since the expression can have free identifiers whose values
are available in the environment. The reason is the same as why we need the
notion of a closure, which is a function with an environment. The structure of
$\exprv{e}{\sigma}$ is quite similar to the structure of a closure,
$\clov{x}{e}{\sigma}$, except that an expression-value lacks the name of a
parameter. The similarity is not just a coincidence. Both kinds of values denote
delay of computation. The evaluation of $e$ in $\exprv{e}{\sigma}$ is postponed
until the value of the argument becomes required, and the evaluation of $e$ in
$\clov{x}{e}{\sigma}$ is postponed until the closure is applied to a value.

Because of the addition of expression-values, we need to define another form of
evaluation. We call it strict evaluation. The purpose of strict evaluation is to
force an expression-value to be a ``normal'' value, which is an integer or a
closure. Strict evaluation is required because the ability of an
expression-value is limited. It can be passed as an argument or stored in an
environment like normal values but cannot be used as an operand of an arithmetic
expression or applied to a value as a function. There must be a way to convert
an expression-value to a normal value, and strict evaluation takes this role.

Strict evaluation is defined as a binary relation over $V$ and $V$. We use
$\Downarrow$ to denote strict evaluation.

\[ \Downarrow\subseteq V\times V \]

$\stricte{v_1}{v_2}$ is true if and only if $v_1$ strictly evaluates to $v_2$.
Here, $v_1$ can be any value. However, $v_2$ cannot be an expression-value; it
must be a normal value. The reason obviously comes from the purpose of strict
evaluation: converting an expression-value to a normal value.

The following rules define strict evaluation of normal values:

\semanticrule{Strict-Num}{
  \stricten{n}{n}
}

\vspace{-1em}

\[
  \stricte{n}{n}
  \quad\textsc{[Strict-Num]}
\]

\vspace{-1em}

\semanticrule{Strict-Clo}{
  \stricten{\clov{x}{e}{\sigma}}{\clov{x}{e}{\sigma}}
}

\vspace{-1em}

\[
  \stricte{\clov{x}{e}{\sigma}}{\clov{x}{e}{\sigma}}
  \quad\textsc{[Strict-Clo]}
\]

A normal value strictly evaluates to itself since it is already a normal value.

The following rule defines strict evaluation of expression-values:

\semanticrule{Strict-Expr}{
  If \evaldn{e}{v_1}, and \stricten{v_1}{v_2},
  then \stricten{\exprv{e}{\sigma}}{v_2}.
}

\vspace{-1em}

\[
  \inferrule
  { \evald{e}{v_1} \\ \stricte{v_1}{v_2} }
  { \stricte{\exprv{e}{\sigma}}{v_2} }
  \quad\textsc{[Strict-Expr]}
\]

An expression-value $\exprv{e}{\sigma}$ is strictly evaluated by evaluating $e$
under $\sigma$. The result of $e$ can be an expression-value again, and thus we
need repeated strict evaluation until reaching a normal value.

Now, we can define the semantics of function applications by using the notions of
expression-values and strict evaluation.

\semanticrule{App}{
  \begin{tabular}{@{\hskip0pt}l@{\hskip-10pt}l}
    If \\
    & \evaldn{e_1}{v_1}, \\
    & \stricten{v_1}{\clov{x}{e}{\sigma'}}, and \\
    & \evaln{\sigma'[x\mapsto\exprv{e_2}{\sigma}]}{e}{v}, \\
    then \\
    & \evaldn{\eapp{e_1}{e_2}}{v}.
  \end{tabular}
}

\vspace{-1em}

\[
  \inferrule
  {
    \evald{e_1}{v_1} \\
    \stricte{v_1}{\clov{x}{e}{\sigma'}} \\
    \eval{\sigma'[x\mapsto\exprv{e_2}{\sigma}]}{e}{v}
  }
  { \evald{\eapp{e_1}{e_2}}{v} }
  \quad\textsc{[App]}
\]

$e_1$ evaluates to $v_1$ first. $v_1$ may be an expression-value, while we need a
closure. Therefore, we strictly evaluate $v_1$ to get a closure. On the other
hand, in the CBN semantics, the argument must not be evaluated before the
evaluation of the function body. Instead of
evaluating $e_2$, we make an expression-value with $e_2$ and $\sigma$ and then
put the value into the environment.

Let us see the semantics of addition and subtraction.

\semanticrule{Add}{
  \begin{tabular}{@{\hskip0pt}l@{\hskip-10pt}l}
    If \\
    & \evaldn{e_1}{v_1}, \\
    & \stricten{v_1}{n_1}, \\
    & \evaldn{e_2}{v_2}, and\\
    & \stricten{v_2}{n_2}, \\
    then \\
    & \evaldn{\eadd{e_1}{e_2}}{n_1+n_2}.
  \end{tabular}
}

\vspace{-1em}

\[
  \inferrule
  {
    \evald{e_1}{v_1} \\
    \stricte{v_1}{n_1} \\
    \evald{e_2}{v_2} \\
    \stricte{v_2}{n_2} \\
  }
  { \evald{\eadd{e_1}{e_2}}{n_1+n_2} }
  \quad\textsc{[Add]}
\]

\vspace{-1em}

\semanticrule{Sub}{
  \begin{tabular}{@{\hskip0pt}l@{\hskip-10pt}l}
    If \\
    & \evaldn{e_1}{v_1}, \\
    & \stricten{v_1}{n_1}, \\
    & \evaldn{e_2}{v_2}, and\\
    & \stricten{v_2}{n_2}, \\
    then \\
    & \evaldn{\esub{e_1}{e_2}}{n_1-n_2}.
  \end{tabular}
}

\vspace{-1em}

\[
  \inferrule
  {
    \evald{e_1}{v_1} \\
    \stricte{v_1}{n_1} \\
    \evald{e_2}{v_2} \\
    \stricte{v_2}{n_2} \\
  }
  { \evald{\esub{e_1}{e_2}}{n_1-n_2} }
  \quad\textsc{[Sub]}
\]

There is nothing difficult. They are similar to the rules of \plang but
additionally require strict evaluation since addition and subtraction are
possible only by using integers, not expression-values.

The semantics is a correct instance of CBN but has a flaw from a practical
perspective. Consider $\eapp{(\efun{\cx}{\cx})}{(\eadd{1}{1})}$. It results in
$\exprv{\eadd{1}{1}}{\emptyset}$, not $2$. Most programmers are likely to perfer
$2$ as a result. We need to apply one last strict evaluation at the end of the
evaluation to resolve the problem. It is to say that ``the result of a program
$e$ is $v$ when $\evale{e}{v'}$ and $\stricte{v'}{v}$.'' Note that it is
different from applying strict evaluation to the evaluation of every expression
in the program. Strict evaluation is applied to only the result of the whole
expression, which is the program. In this way, we can make the result of the
above expression $2$ and eliminate the flaw.\sidenote{It is not a flaw in real-world
programming languages like Haskell. A program shows its result by output operations
(e.g. to files) rather than the value of a single expression. Each output
operation applies strict evaluation to its argument (like Rule \textsc{Add},
Rule \textsc{Sub}, and Rule \textsc{App} in \lang), and the value of
each expression does not need to be a normal value.}

If evaluating an expression in the CBV semantics results in a value,
then the CBN semantics yields the same value. It is known as a corollary of the
standardization theorem of lambda calculus~\cite{theories-of-pl}.
Note that it is true only in languages without side effects.
The result of an expression with side effects varies in the order of
the evaluation. For example, if an argument is an expression changing the value
of a box, and the body of the function reads the value of the box without using
the argument, the program can behave differently in CBV and CBN. In CBV, the
read value will be the value after the update. On the other hand, in CBN, the
update never happens, and the read value will be the original value of the box.

While the CBN semantics preserves the results of the CBV semantics,
the converse is false even without mutation, i.e. there are expressions that
yield results only in CBN.
For instance, consider a function application whose argument is a nonterminating
expression. If the function returns zero without using the argument, evaluation
with CBN results in zero, while evaluation with CBV does not terminate.

\section{Interpreter}

We need to add a new variant to \code{Value} to represent expression-values.
\sidenote{We omit the common part to \plang.}

\begin{verbatim}
sealed trait Value
...
case class ExprV(e: Expr, env: Env) extends Value
\end{verbatim}

\code{ExprV($e$, $\sigma$)} represents $\exprv{e}{\sigma}$.

The following function implements strict evlauation:

\begin{verbatim}
def strict(v: Value): Value = v match {
  case ExprV(e, env) => strict(interp(e, env))
  case _ => v
}
\end{verbatim}

We can implement \code{interp} as follows:
\sidenote{We omit the common part to \plang.}

\begin{verbatim}
def interp(e: Expr, env: Env): Value = e match {
  ...
  case Add(l, r) =>
    val NumV(n) = strict(interp(l, env))
    val NumV(m) = strict(interp(r, env))
    NumV(n + m)
  case Sub(l, r) =>
    val NumV(n) = strict(interp(l, env))
    val NumV(m) = strict(interp(r, env))
    NumV(n - m)
  case App(f, a) =>
    val CloV(x, b, fEnv) = strict(interp(f, env))
    interp(b, fEnv + (x -> ExprV(a, env)))
}
\end{verbatim}

Each case matches the corresponding rule, so there is nothing difficult.

\section{Call-by-Need}

The current implementation is efficient when a parameter appears once or less in
the function body. However, using a parameter twice or more leads to redundant
calculation. Consider the following Scala program:

\begin{verbatim}
def square(x: Int): Int = {
  Thread.sleep(5000)  // models some expensive computation
  x * x
}

def bar(x: => Int, b: Boolean): Int =
  if (b) x + x else 0

val file = new java.io.File("a.txt")
bar(square(10), file.exists)
\end{verbatim}

\code{x} appears twice in the body of \code{bar}. If \code{a.txt} exists, \code{square(10)} is
evaluated twice. Actually, we do not need to evaluate \code{square(10)} twice
since its result is always the same. Since \code{square} is an expensive
function, it is desirable to reduce the number of function calls as much as possible.
If we use CBV instead of CBN, it is possible to evaluate \code{square(10)} only
once when the file exists.
However, going back to CBV is not a good choice. It will make the program
evaluate \code{square(10)} even when the file does not exist.

The way to solve this problem is to store the value of an argument and use the
value again. This strategy is as optimal as CBV when a parameter appears
multiple times; it is as optimal as CBN when a parameter is not used at all.
For programmers, it is tedious to implement such logic in their programs by
themselves. Instead, programming languages can provide the optimization. This
optimization is called \textit{call-by-need}\index{call-by-need} as each
argument is evaluated based on need for its value. It is evaluated once if
needed and is not otherwise.

Call-by-need is not different semantics from CBN in purely functional languages.
The behaviors of a program in call-by-need and CBN are completely equal.
Call-by-need is just an optimization strategy of interpreters and compilers. On
the other hand, call-by-need is different semantics from CBN in languages with
side effects. In such languages, the number of computation of a certain
expression can affect the result. For example, consider an argument that is an expression
that increases the value of the box by one. Suppose that its value is used twice
in the function body. Then, the value of the box increases by two in CBN, while
it increases by one in call-by-need.

Since \lang lacks side effects, we can adopt call-by-need to the language as
optimization of the interpreter. There is no need to newly define the call-by-need
version of the semantics.

To store the strict value of an expression-value, we add a new field to
the class \code{ExprV}.

\begin{verbatim}
case class ExprV(
  e: Expr, env: Env, var v: Option[Value]
) extends Value
\end{verbatim}

The field is declared as mutable. Initially,
the value of the expression is unknown, and \code{v} equals \code{None}. When
the value is calculated for the first time, the value is stored in \code{v}.
The fact that \code{v} equals \code{Some(a)} for some \code{a} implies that the
value of the expression is \code{a}. In the next time we need the value again,
\code{a} can be used without any redundant computation.

The function \code{strict} requires the following change:

\begin{verbatim}
def strict(v: Value): Value = v match {
  case ExprV(_, _, Some(cache)) => cache
  case ev @ ExprV(e, env, None) =>
    val cache = strict(interp(e, env))
    ev.v = Some(cache)
    cache
  case _ => v
}
\end{verbatim}

It checks whether there exists a cached value. If it is the case, the function
simply returns the cached value. Otherwise, \code{e} is evaluated under
\code{env} like before. In addition, the function stores the value in \code{v}.

The function \code{interp} needs only one fix. When a new \code{ExprV} instance
is created in the \code{App} case, one additional argument is required to
initialize the field \code{v}.

\begin{verbatim}
case App(f, a) =>
  val CloV(x, b, fEnv) = strict(interp(f, env))
  interp(b, fEnv + (x -> ExprV(a, env, None)))
\end{verbatim}

Since we do not know the value of \code{a}, the initial value of \code{v} is \code{None}.

Purely functional languages with lazy evaluation usually adopt call-by-need
because it is just optimization but not a change in their semantics. On the
other hand, impure languages cannot consider call-by-need as optimization and
often allow programmers to choose one of them at each place. For example, Scala
uses CBN for by-name parameters and call-by-need for lazy variables. We can
define lazy variables with the \code{lazy} modifier.

\begin{verbatim}
lazy val x = {
  println(1)
  1
}
val y = x + x
\end{verbatim}

The program prints \code{1} only once. By using both by-name parameter and lazy
variable, we can simulate the call-by-need semantics in Scala.

\begin{verbatim}
def bar(_x: => Int, b: Boolean): Int = {
  lazy val x = _x
  if (b) x + x else 0
}
\end{verbatim}

If \code{b} is true, the first argument is evaluated only once. Otherwise, it is
not evaluated at all.

\section{Exercises}

\begin{enumerate}
\item Which of the following produce different results in a
 CBV language and a CBN language? Both produce the
 same result if they both produce the same number or they both produce
 closures (even if they do not behave exactly the same
 when applied).

\begin{enumerate}
  \item $\eapp{(\efun{\cy}{\eapp{\cy}{3}})}{(\efun{\cx}{\eapp{1}{2}})}$
  \item
    $\eapp{(\efun{\cy}{\eapp{\cy}{\efun{\cx}{10}}})}{(\efun{\cx}{\eapp{\cx}{(\eapp{1}{2})}})}$
  \item
    $\eapp{(\efun{\cy}{\eapp{\cy}{\efun{\cx}{10}}})}{(\efun{\cx}{\eapp{1}{2}})}$
  \item
    $\eapp{(\efun{\cy}{\cy})}{(\eadd{1}{\efun{\cx}{\cx}})}$
  \item
    $\eapp{(\efun{\cy}{\eadd{1}{2}})}{(\eadd{1}{\efun{\cx}{\cx}})}$
\end{enumerate}

\item Show the results of each expression in a CBV language and a CBN language.

\begin{enumerate}
  \item $\eadd{(\efun{\cx}{8})}{10}$
  \item $\eapp{(\efun{\cx}{8})}{(\eapp{1}{2})}$
  \item $\efun{\cx}{(\eapp{(\efun{\cy}{42})}{(\eapp{9}{2})})}$
  \item
    $\eadd{1}{(\eapp{(\efun{\cx}{\eadd{\cx}{13}})}{(\eadd{1}{\efun{\cy}{7}})})}$
  \item
    $\eadd{1}{(\eapp{(\efun{\cx}{\eadd{1}{13}})}{(\eadd{1}{\efun{\cy}{7}})})}$
\end{enumerate}

\item Note that there is a recursive call in the following function:

\begin{verbatim}
def strict(v: Value): Value = v match {
  case ev @ ExprV(e, env, None) =>
    val cache = strict(interp(e, env))
    ev.v = Some(cache)
    cache
  case ExprV(_, _, Some(cache)) => cache
  case _ => v
}
\end{verbatim}

Write an example \lang expression showing the need for the recursive \code{strict} call.

\item Consider the following expression:
\[
\begin{array}{l}
  \ebind{\cf}{\efun{\cx}{\eadd{\cy}{7}}}{\\
  \ebind{\cy}{5}{\\
  \eapp{\cf}{(\eadd{42}{\efun{\cy}{3}})}
  }
  }
\end{array}
\]

Explain the result of evaluating it under the following semantics:
\begin{enumerate}
  \item Lazy evaluation with static scope
  \item Lazy evaluation with dynamic scope
  \item Eager evaluation with static scope
  \item Eager evaluation with dynamic scope
\end{enumerate}

\item For each of the following \lang expressions, show how it is evaluated and its result.

\begin{enumerate}
  \item
    $\eapp{(\efun{\cx}{\eapp{\cx}{8}})}{(\eadd{\eapp{42}}{\efun{\cy}{\esub{\cy}{\cx}}})}$
  \item
    $\eapp{(\efun{\cy}{\eapp{(\efun{\cx}{\eadd{3}{\cx}})}{\cy})}}{5}$
\end{enumerate}

\item Consider the following language, LFAE:
\[
  \begin{array}{r@{\hskip6pt}r@{\hskip6pt}l@{\hskip40pt}
    r@{\hskip6pt}r@{\hskip6pt}l@{\hskip40pt}
    r@{\hskip4pt}c@{\hskip4pt}l}
    e &::=& n & v &::=& n & n &\in& \mathbb{Z} \\
    &|& e+e &&|& \clo{x}{e}{\sigma} & x &\in& \textit{Id} \\
    &|& x &&|& (e,\sigma) & v &\in& \textit{Val} \\
    &|& \lambda x.e &&&& e &\in& \textit{Expr} \\
    &|& e\ e &&&& \sigma &\in& \textit{Id}\rightarrow \textit{Val} \\
  \end{array}
\]
with the following ``strict'' evaluation of the form \fbox{$v \Downarrow v$}:
\[
\begin{array}{lll}
n \Downarrow n
&\qquad
\clo{x}{e}{\sigma} \Downarrow \clo{x}{e}{\sigma}
&\qquad
\inferrule
{\sigma \vdash e \Rightarrow v_1 \qquad
v_1 \Downarrow v_2 }
{(e, \sigma) \Downarrow v_2}
\end{array}
\]
Write the operational semantics of LFAE of the form
\fbox{$\sigma\vdash e\Rightarrow v$},
where a value is strictly evaluted
when it is used, that is, strict evaluation is deferred as much as possible.
Thus, the evaluation of ``$(\lambda x.x)\ y$'' results in ``$(y,\emptyset)$,''
rather than causing a run-time error.

\end{enumerate}
