\setchapterpreamble[u]{\margintoc}
\chapter{Lazy Evaluation}
\labch{lazy-evaluation}

The article deals with lazy evaluation. Lazy evaluation means delaying the
evaluation of an expression until the result is required. Real-world languages
support many features other than function applications. The evaluations of
expressions at various places can be subjects of lazy evaluation. For example,
Scala provides the \verb!lazy! keyword to delay the initialization of a variable.
The initialization of a variable declared with the \verb!lazy val! keyword
happens when the variable is first used, but not declared. However, discussing
the evaluations of function call arguments gives enough insight to understand
lazy evaluation. Therefore, the article focuses only on function applications.

Every hitherto language from the previous articles uses eager evaluation, or
strict evaluation. Eager evaluation is a strategy that evaluates an argument
before a function body. Call-by-value semantics of the last article belongs to
eager evaluation. Call by reference denotes semantics that arguments are passed
as references instead of values. Strictly speaking, it is orthogonal to the
distinction between eager and lazy evaluation. However, the intention of lazy
evaluation is delaying an evaluation, but call by reference does not intend it.
Thus, call-by-reference semantics also commonly belongs to eager evaluation.

On the other hand, lazy evaluation evaluates a function body first. The
evaluation of an argument happens when its corresponding parameter appears at the
function body. Consider the following Scala code:

\begin{verbatim}
def f(): Int = ...
def g(): Boolean = ...
def h(x: Int, b: Boolean): Int =
  if (b) x else 0

h(f(), g())
\end{verbatim}

Function \verb!h! has two parameters: \verb!x! and \verb!b!. The value of
\verb!x! is unnecessary when \verb!b! equals \verb!false!. In such cases, the
result always is \verb!0! regardless of the value of \verb!x!. In the last line
of the code, the arguments of the function call are \verb!f()! and \verb!g()!.
Suppose that \verb!g()! equals \verb!true!. Evaluating \verb!f()! is essential.
In contrast, when \verb!g()! equals \verb!false!, it is sure that the result is
\verb!0! whatever \verb!g()! results in. Scala uses eager evaluation. Since
\verb!f()! is always evaluated even though its result might be unessential, the
code is inefficient. However, lazy evaluation allows omitting the evaluation of
\verb!f()! when \verb!g()! equals \verb!false!. The code per se is efficient.
Lazy evaluation helps programmers to write concise and efficient code.

A way to implement lazy evaluation varies according to a criterion deciding the
necessity of the value of an argument. The simplest criterion is the occurrence
of the corresponding parameter in the function body. One can delay more than the
criterion. If the value does not participate in some calculation despite the
occurrence, the value is unnecessary. Consider the following code:

\begin{verbatim}
def f(x: Int): Int = x

f(1 + 1) + 1
\end{verbatim}

Follow the first criterion. Since \verb!x! appears in the body, the evaluation of
\verb!1 + 1! happens and yields \verb!2!. The final result is \verb!3!, which is
the result of \verb!2 + 1!.

Now, follow the second strategy. Despite the appearance of \verb!x!, its value is
unimportant. It is possible to omit the evaluation and to return \verb!1 + 1!.
The final result is \verb!3!, which is the result of \verb!3 + 1!.

One can delay more than the second strategy. The final result is \verb!(1 + 1) + 1!.
The expression is evaluated only when the result becomes observable, for
example being printed to a command line.

No single semantics is the best among them. The best depends on the context.
Language designers must carefully decide the semantics of their language based on
a domain whom it targets, performance, and ease of understanding by programmers.

The article discusses LFAE from the lecture at first. It lacks the formal
semantics of LFAE but shows the implementation of an interpreter. The last part
deals with call by name and call by need in detail. They are beyond the scope of
the course so that one may skip the part.

\section{LFAE
}

LFAE is a language applying lazy evaluation to FAE. They have the same syntax. In
LFAE, arguments are evaluated only when their values are necessary. The value is
necessary if it is a function of a function application or an operand for an
addition or a subtraction.

The following Scala code implements the abstract syntax of LFAE:

\begin{verbatim}
sealed trait Expr
case class Num(n: Int) extends Expr
case class Add(l: Expr, r: Expr) extends Expr
case class Sub(l: Expr, r: Expr) extends Expr
case class Id(x: String) extends Expr
case class Fun(x: String, b: Expr) extends Expr
case class App(f: Expr, a: Expr) extends Expr
\end{verbatim}

An environment is a map from strings to values. It contains the values of
arguments. However, the values are unknown when an environment is expanded.
Values must be able to denote delayed evaluations.

\begin{verbatim}
sealed trait Value
case class NumV(n: Int) extends Value
case class CloV(p: String, b: Expr, e: Env) extends Value
case class ExprV(e: Expr, env: Env) extends Value

type Env = Map[String, Value]
\end{verbatim}

An \verb!ExprV! instance denotes an expression whose value is unknown. Its value
is the result of evaluating \verb!e! under \verb!env!.

If an \verb!ExprV! instance is an operand or a function being applied to an
argument, then the real value of the instance is necessary. The following
\verb!strict! function calculates the value:

\begin{verbatim}
def strict(v: Value): Value = v match {
  case ExprV(e, env) => strict(interp(e, env))
  case _ => v
}
\end{verbatim}

An \verb!ExprV! instance denotes a value that is the result of evaluating its
expression under its environment. The value can be an \verb!ExprV! instance as
well. Recursive calls resolve such cases. When argument \verb!v! is a \verb!NumV!
or \verb!CloV! instance, it is the final result of the function.


\begin{verbatim}
def lookup(x: String, env: Env): Value =
  env.getOrElse(x, throw new Exception)

def interp(e: Expr, env: Env): Value = e match {
  case Num(n) => NumV(n)
  case Add(l, r) =>
    val NumV(n) = strict(interp(l, env))
    val NumV(m) = strict(interp(r, env))
    NumV(n + m)
  case Sub(l, r) =>
    val NumV(n) = strict(interp(l, env))
    val NumV(m) = strict(interp(r, env))
    NumV(n - m)
  case Id(x) => lookup(x, env)
  case Fun(x, b) => CloV(x, b, env)
  case App(f, a) =>
    val CloV(x, b, fEnv) = strict(interp(f, env))
    interp(b, fEnv + (x -> ExprV(a, env)))
}
\end{verbatim}

The \verb!Num!, \verb!Id!, and \verb!Fun! cases equal those of the FAE
interpreter. The \verb!Add! and \verb!Sub! cases need to apply the \verb!strict!
function to the results of \verb!interp! function calls. The \verb!Fun! case
applies the \verb!strict! function to the result of \verb!f! to get a closure.
There is no \verb!interp! call for an argument expression. A new \verb!ExprV!
instance consists of the argument expression and the current environment. The
instance is added to the environment.

The following code evaluates \((\lambda x.1)(1\ 1)\) with the \verb!interp!
function. The FAE interpreter will raise an error for the expression as the first
\(1\) of \(1\ 1\) is not a function. On the other hand, the LFAE interpreter
yields \(1\) because the function body lacks \(x\), the parameter. It never
evaluates \(1\ 1\), which is problematic.

The current implementation is efficient when a parameter appears once or less in
the function body. However, using a parameter twice or more leads to redundant
calculations. Consider applying the current implementation to the following Scala
code:

\begin{verbatim}
def f(x: Int): Int = x + x
def g(): Int = ...

f(g())
\end{verbatim}

\verb!x! appears twice in the body of function \verb!f!. \verb!g()! is evaluated
twice. Suppose that \verb!g! is a pure function, which lacks a side effect and
always produces the same result. Then, the second evaluation yields the same
result to the first evaluation. The redundant evaluation never affects other
evaluations. Therefore, using the first result of \verb!g()! again does not make
any differences. On the other hand, eager evaluation naturally makes \verb!g()!
to be evaluated only once.

By storing the value of an argument and using the value again, the interpreter
becomes efficient. It is as optimla as eager evaluation when a parameter appears
multiple times; it is as optimal as lazy evaluation when the value of a parameter
can be unnecessary. If a language features mutable boxes or variables, which
cause side effects, storing the value of an argument can change the result. The
modification in an interpreter changes the semantics. It is not an optimization.
However, LFAE lacks side effects, and it thus is just an optimization.

\begin{verbatim}
case class ExprV(
  e: Expr, env: Env, var v: Option[Value]
) extends Value
\end{verbatim}

The definition of the \verb!ExprV! class has changed. Field \verb!v! is mutable.
When the value of the expression is unknown, \verb!v! equals \verb!None!. After
the value is calculated once, \verb!v! equals a \verb!Some! instance containing
the value. To use the value again, reading the value of \verb!v! is enough. No
redundant evaluations happen. The \verb!strict! function requires the following
changes:

\begin{verbatim}
def strict(v: Value): Value = v match {
  case ev @ ExprV(e, env, None) =>
    val cache = strict(interp(e, env))
    ev.v = Some(cache)
    cache
  case ExprV(_, _, Some(cache)) => cache
  case _ => v
}
\end{verbatim}

The \verb!App! case of the \verb!interp! function must pass \verb!None!, which is
the initial value of \verb!v!, as an argument for the \verb!ExprV! constructor.

\begin{verbatim}
case App(f, a) =>
  val CloV(x, b, fEnv) = strict(interp(f, env))
  interp(b, fEnv + (x -> ExprV(a, env, None)))
\end{verbatim}

One may use a default parameter value for the \verb!ExprV! class instead of
revising the \verb!interp! function.

\begin{verbatim}
case class ExprV(
  e: Expr, env: Env, var v: Option[Value] = None
) extends Value
\end{verbatim}

In this case, passing only two arguments for a new \verb!ExprV! instance
initializes field \verb!v! to \verb!None! automatically. Hence, changing only
\verb!strict! function suffices. The \verb!interp! function remains the same.

\section{Substitution
}

The semantics of FAE needs revision. The formalization will change, but the
meaning remains the same. Substitution replaces environments. Substitution allows
defining lazy evaluation semantics easily and comparing eager and lazy
evaluations clearly.

Substitution is changing a particular subexpression in an expression into another
subexpression. Usually, it changes a variable into a specific expression. For
example, substituting \(x\) with \((z\ 3)\) in \(x+y\) results in \((z\ 3)+y\).

The following is the abstract syntax of FAE:

\[
\begin{array}{lrcl}
\text{Integer} & n & \in & \mathbb{Z} \\
\text{Variable} & x & \in & \textit{Id} \\
\text{Expression} & e & ::= & n \\
&& | & e + e \\
&& | & e - e \\
&& | & x \\
&& | & \lambda x.e \\
&& | & e\ e
\end{array}
\]

\(\lbrack e'/x\rbrack e\) denotes substituting \(x\) with \(e'\) in \(e\).
\(\lbrack /\rbrack\) is a function from an expression, a variable, and an
expression to an expression.

\[
\lbrack/\rbrack \in
\text{Expression}\times\text{Variable}\times\text{Expression}\rightarrow\text{Expression}
\]

Substitution retains an integer since it lacks variables.

\[\lbrack e/x \rbrack n  =  n\]

A variable changes into an expression if the subject of the substitution equals
the variable. Otherwise, it remains the same.

\[
\lbrack e/x' \rbrack x = \begin{cases} e& \text{if } x=x'\\ x & \text{if }
x\not=x'\end{cases}
\]

Substitution for sums, differences, and function applications has recursive
definitions. The result of substitution is applying the substitution to all the
subexpressions.

\[
\begin{array}{rcl}
\lbrack e/x \rbrack (e_1+e_2) = \lbrack e/x \rbrack e_1+\lbrack e/x \rbrack e_2
\\
\lbrack e/x \rbrack (e_1-e_2) = \lbrack e/x \rbrack e_1-\lbrack e/x \rbrack e_2
\\
\lbrack e/x \rbrack (e_1\ e_2)\ = \lbrack e/x \rbrack e_1\ \lbrack e/x \rbrack
e_2
\end{array}
\]

If the subject of substitution differs from the parameter of a lambda
abstraction, the result of the substitution is applying it to the body.
Otherwise, it remains the same.

The above rule is crucial for defining the semantics of function applications
with substitution. Substituting a parameter in the function body corresponds to
passing arguments to functions. If the name of the parameter is unchecked before
substitution, shadowing fails.

Now, the semantics of FAE uses substitution. Environments are needless. The
semantics is a relation over expressions and values. \(\Downarrow\) denotes the
semantics.

\[
\Downarrow \subseteq \text{Expression}\times\text{Value}
\]

\(e\Downarrow v\) implies that evaluating \(e\) yields \(v\).

A value is either an integer or a function value. Due to the absence of
environments, a function value lacks an environment. A lambda abstraction instead
of a closure is a function value.

\[
\begin{array}{lrcl}
\text{Value} & v & ::= & n \\
&& | & \lambda x.e
\end{array}
\]

The inference rules for integers, sums, and differences are similar to the
previous rules.

\[
n\Downarrow n
\]

\[
\inferrule
{
  e_1\Downarrow n_1 \\
  e_2\Downarrow n_2
}
{ e_1+e_2\Downarrow n_1+n_2 }
\]

\[
\inferrule
{
  e_1\Downarrow n_1 \\
  e_2\Downarrow n_2
}
{ e_1-e_2\Downarrow n_1-n_2 }
\]

Like an integer, a lambda abstraction produces itself.

\[
\lambda x.e\Downarrow\lambda x.e
\]

A function application evaluates the function and the argument. Then, it
evaluates an expression obtained by substituting the parameter with the value of
the argument in the function body.

\[
\inferrule{
  e_1\Downarrow \lambda x.e \\
  e_2\Downarrow v' \\
  \lbrack v'/x\rbrack e\Downarrow v
}
{ e_1\ e_2\Downarrow v }
\]

Since substitution changes parameters into values, evaluating expressions without
free variables never results in situations directly requiring the values of
variables. No rules for variables exist. If an evaluation fails because of a
variable, then the original expression contains a free variable. Consider
\(\lambda x.y)\ 1\). Variable \(y\) is free in the expression. Substituting \(x\)
with \(1\) in \(\lambda x.y\) results in \(y\). It is impossible to evaluate
\(y\).

To understand the definition of substitution, consider \((\lambda x.(\lambda
x.x)\ 1)\ 2\). \(\lambda x.x\) is an identity function. Since \((\lambda x.x\)\
\(1\) always results in \(1\), \(\lambda x.(\lambda x.x)\ 1\) is a function
returning \(1\) regardless of the value of an argument. According to the current
definition of substitution, substituting \(x\) with \(2\) in \((\lambda x.x)\ 1\)
results in \((\lambda x.x)\ 1\) because subject \(x\) equals parameter \(x\).
\((\lambda x.x)\ 1\) yields \(1\), which is correct. Assume that substitution
always changes the body of a lambda abstraction. Then the substitution results in
\((\lambda x.2)\ 1\). The final result is \(2\), which is wrong. It implies that
the innermost \(x\) is bound to parameter \(x\) of \(\lambda x.(\lambda x.x)\ 1\)
instead of parameter \(x\) of \(\lambda x.x\). Shadowing fails, and the original
definition thus is correct.

\subsection{Alpha Conversion
}

Alas, the semantics of FAE is incomplete. An expression containing a free
variable can result in an error because directly evaluating a variable is
impossible. However, under the current semantics, such an expression can
incorrectly yield a value. Consider \((\lambda x.\lambda y.x)\ (\lambda z.y)\ 1\
2\). \(\lambda z.y\) has free variable \(y\). \(\lambda x.\lambda y.x\) is a
function taking two arguments and returning the former. Thus, \((\lambda
x.\lambda y.x)\ (\lambda z.y)\ 1\) equals \(\lambda z.y\). Applying the function
to \(2\) produces an error as the value of \(y\) is unknown. However, the current
semantics evaluates the expression in a different way. Substituting \(x\) with
\(\lambda z.y\) in \(\lambda x.\lambda y.x\) results in \(\lambda y.\lambda
z.y\). It has already become problematic. Free variable \(y\) in \(\lambda z.y\)
has been unexpectedly bound to parameter \(y\) of \(\lambda y.x\). Continuing the
evaluation yields \(1\) since \(\lambda y.\lambda z.y\) takes two arguments and
returns the former. Why did it happen? \(y\) bound to parameter \(y\) of
\(\lambda y.x\) must syntactically belong to the body of the function. However,
the substitution syntactically replaces \(x\) with \(\lambda z.y\). As a
consequence, free variable \(y\), which is unrelated to parameter \(y\), has been
bound to parameter \(y\). Variable capture refers to that a free variable becomes
unexpectedly bound to an irrelevant binding occurrence.

Alpha conversion, or alpha renaming, resolves variable capture. It changes the
name of the parameter of a lambda abstraction while preserving the behavior of
the function. For example, alpha-converting \(\lambda x.x\) can yield \(\lambda
y.y\). Both functions are identity functions and thus have the same semantics but
differ in the names of the parameters. A lambda abstraction and a lambda
abstraction obtained by alpha-converting the former are alpha equivalent.

A new name of a parameter for alpha conversion must be an identifier that is not
in the lambda abstraction. Otherwise, the semantics of the function can change.
For example, replacing \(x\) with \(z\) in \(\lambda x.y\) yields \(\lambda z.y\)
and is correct alpha conversion. On the other hand, replacing \(x\) with \(y\)
yields \(\lambda y.y\), whose semantics differs from that of the original
function. Replacing \(x\) with \(z\) in \(\lambda x.\lambda y/x\) is correct
since the result is \(\lambda z.\lambda y.z\). However, replacing \(x\) with
\(y\) results in \(\lambda y.\lambda y.y\), which changes the semantics.

Function \(\mathit{id}\) takes an expression as an argument and returns the set
of all the identifiers in the expression.

\[
\mathit{id}\in \text{Expression}\rightarrow \mathcal{P}(\text{Variable})
\]

\[
\begin{array}{rcl}
\mathit{id}(n) & = & \emptyset \\
\mathit{id}(e_1+e_2) & = & \mathit{id}(e_1)\cup\mathit{id}(e_2) \\
\mathit{id}(e_1-e_2) & = & \mathit{id}(e_1)\cup\mathit{id}(e_2) \\
\mathit{id}(x) & = & \{x\} \\
\mathit{id}(\lambda x.e) & = & \mathit{id}(e)\cup\{x\} \\
\mathit{id}(e_1\ e_2) & = & \mathit{id}(e_1)\cup\mathit{id}(e_2)
\end{array}
\]

An integer lacks identifiers. A variable contains a single identifier that is its
name. When an expression is a sum, a difference, or a function application, the
union of the identifiers of its subexpressions is the result. A lambda
abstraction has the name of its parameter and every identifier in its body.

Alpha conversion is a relation over expressions and expressions. The definition
of alpha conversion follows:

\[
\equiv^\alpha \subseteq \text{Expression}\times\text{Expression}
\]

\[
\inferrule
{ x'\not\in\textit{ids}(\lambda x.e) }
{ \lambda x.e\equiv^\alpha\lambda x'.\lbrack x'/x\rbrack e }
\]

\[
\lambda x.e\equiv^\alpha\lambda x.e
\]

The domain of alpha conversion includes only lambda abstractions. For a given
function, it replaces every occurrence of the parameter with a fresh identifier,
which never occurs in the function, in the function. The result can be the
function itself.

Substitution avoids variable capture through alpha conversion. Before that, a
free variable of an expression must be formally defined. Function \(\mathit{fv}\)
takes an expression as an argument and returns the set of all the free variables
in the expression.

\[
\mathit{fv}\in \text{Expression}\rightarrow \mathcal{P}(\text{Variable})
\]

\[
\begin{array}{rcl}
\mathit{fv}(n) & = & \emptyset \\
\mathit{fv}(e_1+e_2) & = & \mathit{fv}(e_1)\cup\mathit{fv}(e_2) \\
\mathit{fv}(e_1-e_2) & = & \mathit{fv}(e_1)\cup\mathit{fv}(e_2) \\
\mathit{fv}(x) & = & \{x\} \\
\mathit{fv}(\lambda x.e) & = & \mathit{fv}(e)\setminus\{x\} \\
\mathit{fv}(e_1\ e_2) & = & \mathit{fv}(e_1)\cup\mathit{fv}(e_2)
\end{array}
\]

An integer lacks free variables. A variable has a single free variable that is
itself. When an expression is a sum, a difference, or a function application, the
union of the free variables of its subexpressions is the result. The free
variables of a lambda expression equal the free variables of the body except the
parameter.

Now, the definition of substitution changes. Since alpha conversion can produce
more than one expression, substitution is a relation, but not a function.

Substitution for lambda abstractions only needs changes. All the other cases
remain the same.

\[
\lbrack e/x \rbrack\lambda x.e' = \lambda x.e'
\]

\[
\inferrule
{ x\not=x' \\
  \lambda x'.e'\equiv^\alpha \lambda x'' .e'' \\
  x'' \not\in\mathit{fv}(e) }
{ \lbrack e/x \rbrack\lambda x'.e' =
  \lambda x'' .\lbrack e/x \rbrack e'' }
\]

Substitution correctly avoids variable capture. The semantics of FAE has been
perfect.

Many sorts of programming language research focus on topics irrelevant to
variable capture and alpha conversion. They usually assume alpha conversion to
prevent variable capture in their semantics. They typically say that they treat
expressions up to alpha conversion. It may seem naive but makes researchers be
able to concentrate on important points. Substitution often fits defining
semantics better than environments. As the assumption simply removes the concern
about variable capture, most researchers use substitution instead of environments
to define semantics.

Proof assistants, such as Coq, strictly deals with every insignificant detail.
Users cannot simply assume alpha conversion to resolve variable capture. They
commonly use De Bruijn indices to overcome the difficulty. Higher-order abstract
syntax, which represents variable binding with functions at the level of abstract
syntax trees, is another solution. Both topics are beyond the scope of the
article.

Interpreters can use substitution, but such interpreters are rare. Assuming alpha
conversion does not work for interpreters. Implementing correct substitution
requires some amount of effort. Besides, interpreters using environments
outperform interpreters using substitution. Maps efficiently implement
environments. Both extension and lookup require constant time. On the other hand,
applying substitution to an expression whose size is \(n\) requires time
complexity of \(O(n)\). The definition of the size of an expression is omitted as
it is unimportant.

\section{Call by Name
}

The inference rule for function applications under call-by-value semantics
follows:

\[
\inferrule{
  e_1\Downarrow \lambda x.e \\
  e_2\Downarrow v' \\
  \lbrack v'/x\rbrack e\Downarrow v
}
{ e_1\ e_2\Downarrow v }
\]

Call-by-name semantics changes the rule not to evaluate the argument and to
substitute the parameter with the argument expression.

All the other rules remain the same.

If evaluating an expression under call-by-value semantics results in a value,
then call-by-name semantics yields the same value. The following formally
rephrases it:

\[
\forall e.\forall v.e\Downarrow v\rightarrow e\Downarrow_l v
\]

It is a corollary of the standardization theorem~\cite{reynoldspl}. The theorem states that for
a given expression, if there is an order of evaluation that results in a value,
then normal-order evaluation produces the same value. Normal order is similar to
call by name. The only difference is that normal-order evaluation evaluates the
body of a function that is never called. Normal order is beyond the scope of the
article.

The above proposition is true because FAE and LFAE lack side effects. Expressions
of languages with side effects vary in the results according to the order of
evaluation. If side effects exist, redundant calculations can change states.
Under call-by-value semantics, every argument is evaluated once and only once.
However, under call-by-name semantics, an argument can be evaluated zero or more
times.

The converse of the proposition is false. Some expressions yield results under
only call-by-name semantics. Even though the evaluation of an argument raises an
error or does not terminate, the function application succeeds if the body does
not require the value of the argument. However, the evaluation always happens
under the call-by-value semantics.

\subsection{Implementing an Interpreter
}

Making some revision to the LFAE interpreter allows call by name for function
applications. The existing interpreter evaluates an argument when it appears as a
function of a function application or an operand of an addition or a subtraction.
Under the call-by-name semantics, an interpreter must evaluate an argument when
the corresponding parameter occurs.

\begin{verbatim}
sealed trait Expr
case class Num(n: Int) extends Expr
case class Add(l: Expr, r: Expr) extends Expr
case class Sub(l: Expr, r: Expr) extends Expr
case class Id(x: String) extends Expr
case class Fun(x: String, b: Expr) extends Expr
case class App(f: Expr, a: Expr) extends Expr
\end{verbatim}

The abstract syntax remains the same.

\begin{verbatim}
sealed trait Value
case class NumV(n: Int) extends Value
case class CloV(p: String, b: Expr, e: Env) extends Value

case class Expr(e: Expr, env: Env)
type Env = Map[String, Expr]
\end{verbatim}

The previous interpreter needs the \verb!ExprV! class to represent a delayed
evaluation as a value. Now, the result of the \verb!interp! function always
differs from a delayed evaluation because any occurrence of a parameter leads to
evaluation of the argument. The \verb!Expr! class replaces the \verb!ExprV!
class. The \verb!Expr! class takes the same role to the \verb!ExprV! class, but
its instance is not considered as a value of LFAE. An environment is a map from a
string to an \verb!Expr! instance.

\begin{verbatim}
def lookup(x: String, env: Env): Expr =
  env.getOrElse(x, throw new Exception)

def interp(e: Expr, env: Env): Value = e match {
  case Num(n) => NumV(n)
  case Add(l, r) =>
    val NumV(n) = interp(l, env)
    val NumV(m) = interp(r, env)
    NumV(n + m)
  case Sub(l, r) =>
    val NumV(n) = interp(l, env)
    val NumV(m) = interp(r, env)
    NumV(n - m)
  case Id(x) =>
    val Expr(e, eEnv) = lookup(x, env)
    interp(e, eEnv)
  case Fun(x, b) => CloV(x, b, env)
  case App(f, a) =>
    val CloV(x, b, fEnv) = interp(f, env)
    interp(b, fEnv + (x -> Expr(a, env)))
}
\end{verbatim}

Since the result of \verb!interp! function cannot be \verb!ExprV!, the
\verb!strict! function is useless. The \verb!Id! case finds an \verb!Expr!
instance from the environment and evaluates the expression in the instance under
the environment in the instance. The \verb!App! case creates an \verb!Expr!
instance to extend the environment.

\subsection{Call by Need
}

Call by need is semantics that evaluates an argument when the corresponding
parameter occurs in the function body, like call by name. The only difference
between two semantics is that call-by-need semantics stores a value of the
argument after the first evaluation and reuses the value for other occurrences of
the parameter. It applies the optimization for LFAE, which uses the option type,
to call by name. Two semantics always produces the same result for languages
without side effects. Under the presence of side effects, two semantics vary.

Call by need is formalizable as well as call by name. However, to represent
storing values of arguments, formalization needs a concept similar to a store of
BFAE or MFAE. It complexifies the semantics needlessly. Moreover, an LFAE
expression results in the same value regardless of choice between call by name
and call by need. Call by need for LFAE is an optimization of call by name for
LFAE. Thus, the article omits formalization of call by need. One can find
additionally from other articles, such as "An operational semantics of sharing in
lazy evaluation." Since modifying the interpreter to use call by need is
trivial, the article omits it as well.

\section{Exercises}
