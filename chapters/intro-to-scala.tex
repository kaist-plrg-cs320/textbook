\setchapterpreamble[u]{\margintoc}
\chapter{Introduction to Scala}
\labch{intro-to-scala}

\section{Scala}

Professor Odersky and his students in LAMP at EPFL have developed Scala. Scala
stands for ``\textbf{Sca}lable \textbf{La}nguage.'' The
``Growing a Language''\sidenote{\url{https://www.youtube.com/watch?v=_ahvzDzKdB0}}
talk given by Guy Steele has influenced Scala.
Scala supports both \term{functional} and \term{object-oriented} styles and aims a
growable language. Later articles will discuss functional programming more.

\begin{kaobox}[frametitle=Compilers and interpreters]
\term{Compilers} and \term{interpreters} are out of the topic but I believe that it is
the best point to discuss them.

Compilers are programs that translate one programming language into another
language. Usually, the term ``compilers'' is used for a narrower range of
programs: compilers get code written in high-level languages, whom people can
easily understand, as input and generate code written in low-level languages
like \term{machine languages} and \term{bytecode}. Physical machines can directly
interpret machine languages and \term{virtual machines} can directly interpret
bytecode. For example, GCC translates C and C++ into machine languages and
\verb+javac+ translates Java into Java bytecode. Since machines can directly
interpret the low-level languages, compiled forms of programs guarantee fast
execution. However, when programmers modify their code, they must re-compile
the code to execute programs so that checking results immediately after
changing the code takes a long time.

On the other hand, interpreters directly interpret code given as input and
show results instead of generating new code. Python and JavaScript are typical
languages using interpreters. Since given code is a \term{string}, interpreters do
complex procedures including \term{parsing} at run time and therefore execution
takes a long time. However, because changing code does not require compiling,
checking results after changing the code takes a short time.
\end{kaobox}

Scala compilers compile Scala code into Java bytecode executable by Java
virtual machines (JVM). Every Scala project can use the Java standard library
and any external Java libraries. Not only using compiled Java code but also
mixing Java code and Scala code in a single project is possible. In addition,
the Scala standard library supports
conversion\sidenote{\url{https://www.scala-lang.org/api/current/scala/jdk/CollectionConverters\$.html}}
between Java \term{collections} and
Scala collections. In practice, interoperability with Java is useful a lot.
Lack of Scala libraries is not problematic since libraries from the Java
ecosystem are available.

The most recent Scala version in July 2019 is 2.13.0. At the same time, Scala
, as known as Dotty, is coming. The most recent Dotty version in July 2019 is
.16.0. The official GitHub repositories are
\url{https://github.com/scala/scala}
and \url{https://github.com/lampepfl/dotty}, respectively.

\section{Scala in Industry}

Unfortunately, Scala is not one of the most popular languages. According to
GitHub Octoverse 2017\sidenote{\url{https://octoverse.github.com/2017}},
Scala was ranked the thirteenth place. The criterion was
the number of PRs in one year.
GitHub Octoverse 2018\sidenote{\url{https://octoverse.github.com/2018}}
and after showed until the tenth place and Scala did not appear on the list.
According to Stackoverflow Developer Survey 2019\sidenote{\url{https://insights.stackoverflow.com/survey/2019}},
Scala was ranked the twentieth place.

However, we can find several places using Scala. PLRG at KAIST has developed
JavaScript static analyzer
SAFE 2.0\sidenote{\url{https://github.com/sukyoung/safe}}
in Scala. Trivially, the official Scala and Dotty compilers have been
developed in Scala. In industry, Scala is popular for concurrent programming
and distributed computing.
Akka\sidenote{\url{https://akka.io/}} is a concurrent,
distributed computing library written in Scala. Many companies have been using Akka.
Apache Spark\sidenote{\url{https://spark.apache.org/}}, a well-known library for data
processing, also is written in Scala.
Play\sidenote{\url{https://www.playframework.com/}}
is a widely-used web framework based on Akka.

In another point of view, Scala is valuable since it is a good introduction to
functional programming. Nowadays, functional programming is popular in
industry as well as academia. Using one of functional languages like OCaml and
Haskell in future is highly possible for undergraduate students studying
computer science of these days. Therefore, if you started programming with
imperative* languages like C, Java, and Python, experiencing Scala will
increase the choices of your future job.

\section{Programming in Scala}

The official Scala web site\sidenote{\url{https://www.scala-lang.org/}} provides
instructions for installation. The site discusses also possible IDEs for
Scala.

\subsection{Scala REPL}

After installing Scala, type \verb+scala+ in a command line to execute the Scala
REPL. The term ``REPL'' stands for ``\textbf{r}ead, \textbf{e}val, \textbf{p}rint, and
\textbf{l}oop''.
It is a program that reads a line from a user, evaluates the code, prints
the result, and goes back to the first. Languages with interpreters used to
support REPLs but, nowadays, languages with compilers also supports REPLs. In
the Scala REPL, for every line of input, the Scala compiler compiles the code
and the REPL shows the result of the execution.

\begin{verbatim}
scala> 0
res0: Int = 0
\end{verbatim}

For input \term{expression} \verb+0+, the resulting \term{value} is \verb+0+ and the type of \verb+0+
is \verb+Int+.

\begin{verbatim}
scala> 1 * 1
res1: Int = 1
\end{verbatim}

It works well for more complex expressions.

\subsection{Variable Declarations}

\begin{verbatim}
scala> val x = 2
x: Int = 2
\end{verbatim}

To declare a variable, \verb+val+ can be used. \verb+val [name] = [expression]+ implies
that the result of evaluating the expression becomes the thing referred by the
name of the variable. The type of \verb+x+ is inferred as \verb+Int+.

\begin{verbatim}
scala> x = 3
<console>:12: error: reassignment to val
       x = 3
\end{verbatim}

Since \verb+val+ declares \term{immutable} variables, assigning an expression to an
already defined variable is impossible. Immutability is the most important
property of functional programming.

\begin{verbatim}
scala> val x = 4
x: Int = 4
\end{verbatim}

On the other hand, we can declare \verb+x+ again because, in the REPL, two \verb+x+'s
belong to different scopes. It is impossible to define two local variables of
the same name in a single scope.

\begin{verbatim}
scala> val y: Int = x + 1
y: Int = 5
\end{verbatim}

Mentioning explicitly the type of a variable is possible. Usually, due to
type inference*, type annotations are unnecessary but they are required in rare cases.

\begin{verbatim}
scala> var z = 6
z: Int = 6

scala> z = 7
z: Int = 7
\end{verbatim}

\verb+var+ allows to declare \term{mutable} variables.

\begin{verbatim}
scala> z = "8"
<console>:12: error: type mismatch;
 found   : String("8")
 required: Int
       z = "8"
           ^
\end{verbatim}

To assign an expression to a mutable variable, the expression must follow the
type of the variable. Functional programming avoids using mutable variables.
Actually, and surprisingly, prohibiting mutable variables does not decrease
expressivity of programs at all. Therefore, \verb+val+ should be used in most cases
but, for efficiency, \verb+var+ might be used. In the course, if there is no
special instruction, students cannot use \verb+var+ in their homework.

\subsection{Function Declarations}

\begin{verbatim}
scala> def add(x: Int, y: Int) = x + y
add: (x: Int, y: Int)Int
\end{verbatim}

\verb+def+ declares functions. \verb+def [name]([name]: [type], …) = [expression]+ is
the form. Like functions in mathematics, \verb+return+ is unnecessary. The return
value of a function call is the result of evaluating the expression at the
right-hand side. Since \verb+x + y+ has type \verb+Int+, the return type of \verb+add+ is
also \verb+Int+ due to type inference.

\begin{verbatim}
scala> add(4, 5)
res2: Int = 9
\end{verbatim}

The return value is correct.

\begin{verbatim}
scala> def add(x: Int, y: Int): Int = x + y
add: (x: Int, y: Int)Int
\end{verbatim}

Like variables, it is possible to annotate return types of functions.

\begin{verbatim}
scala> def add(x, y): Int = x + y
<console>:1: error: ':' expected but ',' found.
       def add(x, y): Int = x + y
                ^
\end{verbatim}

Unlike some languages including OCaml, types of function \term{parameters} are essential.

\begin{verbatim}
scala> def f(x: Int) = {
     |   val y = x + 1
     |   val z = x + 2
     |   y * y + z * z
     | }
f: (x: Int)Int
\end{verbatim}

If a function needs to use local variables, the body of the function is a
sequence of multiple expressions bundled in a pair of curly brackets. Every
expression inside the brackets is sequentially evaluated from top to bottom.
The result is equal to the result of the last expression.

\begin{verbatim}
scala> def g(x: Int) = {
     |   println(x)
     |   x * x
     |   x
     | }
g: (x: Int)Int

scala> g(10)
10
res3: Int = 10
\end{verbatim}

\verb+g+ shows the evaluation of bundled multiple expressions more clearly. \verb+g+
prints \verb+10+, calculates \verb+x * x+, whose result is discarded, and returns \verb+10+,
the result of evaluating \verb+x+.

\begin{verbatim}
scala> val w = {
     |   val a = 5
     |   a + a + 1
     | }
w: Int = 11

scala> add({
     |   val x = 6
     |   x + x
     | }, 0)
res4: Int = 12
\end{verbatim}

Since a sequence of bundled expressions is an expression, it can be used
anywhere expecting expressions.

\subsection{Conditionals}

\begin{verbatim}
scala> if (true) 13 else 14
res5: Int = 13
\end{verbatim}

\term{Conditional} expressions use \verb+if+ and \verb+else+ like many other languages.
Evaluating a conditional expression results in a value.

\begin{verbatim}
scala> val a = if (false) 13 else 14
a: Int = 14

scala> var b = 0
b: Int = 0

scala> if (false) b = 14 else b = 15

scala> b
res6: Int = 15
\end{verbatim}

In Scala, immutable variables are enough to assign conditional values. On the
other hand, imperative languages require mutable variables and reassignment.
\verb+if+ and \verb+else+ in Scala look similar to the ternary operator \verb+? :+ in
  imperative languages but are more general.

\begin{verbatim}
scala> if (true) {
     |   val x = 8
     |   x + x
     | } else {
     |   val x = 9
     |   x * x
     | }
res7: Int = 16
\end{verbatim}

Since a sequence of bundled expressions is an expression, conditional
expressions can use curly brackets to express complex computation while it is
impossible in imperative languages.

\subsection{Compiling Scala Code}

The Scala compiler \verb+scalac+ compiles Scala code into Java bytecode.

\begin{verbatim}
object App {
  def main(args: Array[String]) = println("Hello world!")
}
\end{verbatim}

Save the above code into \verb+App.scala+ file and type \verb+scala App.scala+ and
\verb+scala App+ in a command line. The console will print \verb+"Hello world!"+. \verb+scalac+
compiles \verb+App.scala+ and generates \verb+App.class+. \verb+scala+ uses JVM to execute
\verb+App.class+.

Like Java, the \verb+main+ \term{method} is the entry point of every Scala program. The
main` method must contain code has to be executed.

In practice, like other languages, programmers use build tools for their Scala
projects. Like CMake for C and C++ and Ant, Maven, and Gradle for Java, Scala
projects usually use SBT.
