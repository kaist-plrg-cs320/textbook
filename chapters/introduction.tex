\setchapterpreamble[u]{\margintoc}
\chapter{Introduction}
\labch{introduction}

What is a programming language?

The simplest answer is ``it is a language used for programming.'' However, this
answer does not help us understand programming languages. We need a better
question to get a better answer.

What does a programming language consist of?

There is a good answer for this question: ``in a narrow sense, a programming
language consists of syntax and semantics, and in a broad sense, it additionally
has a standard library and an ecosystem.''

Syntax and semantics are principal concepts to understand programming languages.
Syntax determines how a language looks like, and semantics fills the inside. If
we consider a programming language as a human, we can say that syntax is one’s
appearance, and semantics is one’s thoughts. Programmers write programs
according to syntax. Syntax decides characters used in source code. Once programs
are written, semantics decides what each program does. Without semantics, all
the programs are useless. Programs can work as being expected only after
semantics determines the meaning of them. A programming language with syntax and
semantics is complete. Programmers using that language can write programs with
the syntax and execute the programs with the semantics. From a theoretical
perspective, syntax and semantics are all of a programming language.

For programmers, syntax and semantics are not the only elements of a programming
language. First, the standard library of a language is another element. The
standard library provides various utilities required by applications: data
structures like lists and maps, functions handling file and network IO, and so
on. The standard library is like clothes for humans. A human without clothes is
a human; a programming language without a standard library is a programming
language. At the same time, clothes are important to humans as they make bodies
warm and protect bodies from dangerous objects. Similarly, a standard library is
important to a programming language as it supplies diverse functionalities for
applications. Each person wears clothes different from others, and each language
puts different things from other languages in its standard library. Some
languages include many utilities in their standard libraries, while others
include much less. Some languages treat lists and maps as built-in concepts in
their semantics, while others define them with other primitives in their standard libraries.
Programmers avoid using a language without a standard library because such a
language increases the effort to write programs.

Another important element to programmers is the ecosystem of a programming
language. The ecosystem includes everything related to the language: developers
and companies using the language, third-party libraries written in the language,
and so on. It is like a society for humans. If many programmers and companies
use a programming language, one can easily get help and find complementary
materials by using the same language. There will be more chances of cooperative
work and employment, too. Third-party libraries also take important roles in
software development. The standard library offers only general facilities and
often lacks domain-specific features. When a required functionality cannot be
found in the standard library, a third-party library can provide the exact
functionality. For these reasons, the ecosystem of a programming language is
important to programmers.

Practically, the standard library and the ecosystem of a language are important
elements. Unlike syntax and semantics, they are not essential. A programming
language can exist even without its standard library and ecosystem. However,
developers take standard libraries and ecosystems into account as well as syntax and
semantics to choose languages they use. From a practical perspective, a
programming language consists of syntax, semantics, a standard library, and an
ecosystem.

This book is not for helping readers use a specific programming language. It
does not recommend a specific programming language, either. This book helps
readers learn new programming languages easily. You can acquaint any programming
languages once you completely read and understand this book. Obviously, this
goal cannot be achieved if the book discusses various languages separately. It
is possible only by discussing the underlying principles of every programming
language.

The principles of programming languages can be found from their semantics. Each
language seems very different from the others, but it is actually not the case.
Precisely speaking, their insides are quite the same, while their appearances
look different. They look different because their syntax and standard libraries,
which determine the appearances, are different. However, their insides, the
semantics, fundamentally share the same mathematical principles. If you
understand essential concepts residing in the semantics of multiple languages,
it is easy to understand and learn new languages.

People who know the key principles and can separate the elements of a language
can easily learn programming languages. As an analogy, consider a man learning
how to use a computer. It is a big problem if he cannot distinguish a keyboard
from a computer. For example, he thinks ``to say hello, my right index finger
presses the keyboard, my left middle finger presses the keyboard, my right ring
finger presses the keyboard three times.'' If the layout of the keyboard changes,
he should learn the whole computer again. On the other hand, if he knows that a
keyboard is just a tool to input text, he will less suffer from the change of
the keyboard layout. As he thinks ``to say hello, I press H, E, L, L, and O,'' he
does not need to learn the whole computer again. Of course, he should learn the
new keyboard layout, but it will be much easier. In addition, it is
straightforward to apply his knowledge to do new things. For example, he will
easily figure out ``to say lol, I press L, O, and L.'' If he does not distinguish
a keyboard from a computer, he cannot find any common principles between saying
hello and saying lol. Learning programming languages is the same. People who
cannot distinguish syntax and semantics believe that they should learn the whole
language again when the syntax changes. On the other hand, people who can
distinguish syntax and semantics know that semantics remains the same even if
syntax may vary. They know that understanding the principles of semantics is
important to learn languages. Becoming familiar with the
new syntax is all they need to use a new language fluently.

This book explains the semantics of principal concepts in programming languages.
Chapters \chnum{introduction-to-scala}, \chnum{immutability}, \chnum{functions},
and \chnum{pattern-matching} introduce the Scala programming language. This book
uses Scala to implement interpreters and type checkers of languages introduced
in the book.  \refch{syntax-and-semantics} explains syntax and semantics. Then,
the book finally introduces various features of programming languages.
\begin{itemize}
    \item \refch{identifiers} introduces identifiers.
    \item \refchss{first-order-functions}{first-class-functions}{recursion} introduce functions.
    \item \refchs{boxes}{mutable-variables} introduce mutation.
    \item \refch{garbage-collection} introduces garbage collection.
    \item \refch{lazy-evaluation} introduces lazy evaluation.
    \item \refchss{continuations}{first-class-continuations}{first-order-representation-of-continuations} introduce continuations.
    \item \refch{nameless-representation-of-expressions} introduces De Bruijn indices.
    \item \refchs{type-systems}{typing-recursive-functions} introduce basic type systems.
    \item \refch{algebraic-data-types} introduces algebraic data types.
    \item \refch{parametric-polymorphism} introduces parametric polymorphism.
    \item \refch{subtype-polymorphism} introduces subtype polymorphism.
    \item \refch{type-inference} introduces type inference.
\end{itemize}
Each chapter explains a feature by defining a small language providing the
feature. Those languages may seem inconvenient in practice because they are too small.
However, the simplicity will allow us to focus on the topic of each chapter.

\section{Exercises}

\begin{enumerate}
\item Write the name of a programming language that you have used.
  What are the pros and cons of the language?
\item Write the names of two programming languages you know and compare them.
\end{enumerate}
