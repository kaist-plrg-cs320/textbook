\chapter{First-Class Continuations}
\labch{first-class-continuations}

\renewcommand{\plang}{\textsf{FAE}\xspace}
\renewcommand{\Lang}{\textsf{KFAE}\xspace}

The previous chapter defines the small-step semantics of \plang and implements
the interpreter of \plang in CPS. Conceptually,
continuations exist during the evaluation of \plang programs. However, they are
not exposed to programmers. Programmers cannot utilize continuations directly while
writing programs in \plang.

A first-class entity of a language is an entity treated as a
value. Since it is a value, it can be the value denoted by a variable, an argument for a
function call, and the return value of a function. For example, first-class
functions are functions used as values.

\textit{First-class continuations}\index{first-class continuation}
are continuations used as values. If a language
supports first-class continuations, continuations can be the value of a variable,
an argument for a function call, and the return value of a function. A
continuation can be considered as a function since it takes a value and performs
computation. Programmers can call continuations like calling functions. However,
continuations are different from functions. A function call returns a value, and
the execution continues with the return value. On the other hand, a continuation
call does not come back to its call site. The continuation at some point of execution
is the remaining computation. Once a continuation is called and evaluated,
the execution finishes. Calling a continuation changes the
current continuation to the called one. It changes the control flow of the
execution. First-class continuations allow programmers to express
computations with complex control flows concisely.

This chapter defines \Lang by extending \plang with first-class continuations.
It defines the small-step semantics of \Lang and implements an interpreter of
\Lang in CPS. While implementing the interpreter, you will see why CPS is required.
In addition, this chapter shows utilization of first-class continuations in
programming.

\section{Syntax}

The syntax of \Lang is as follows:
\footnote{We omit the common part to \plang.}

\[
  e\ ::=\ \cdots\ |\ \evcc{x}{e}
\]

An expression $\evcc{x}{e}$ evaluates $e$ while $x$ denotes the
continuation of $\evcc{x}{e}$. The term ``\textsf{cc}'' of \textsf{vcc} stands
for the \textbf{\large c}urrent \textbf{\large c}ontinuation.
The scope of $x$ equals $e$. When a continuation
is called, the continuation replaces the continuation of that point.

\section{Semantics}

Before going deep into the semantics of first-class continuations, it would be
better to understand the high-level idea with some examples.
Consider $1+(\evcc{\cx}{(\cx\ 2)+3})$. The
continuation of $(\evcc{\cx}{(\cx\ 2)+3})$ is to add the result to
$1$. Intuitively, we can use $\underline{1}+\square$ to represent the
continuation.\footnote{Like before, the line below $1$ denotes that $1$ is an
integer value, not an expression.}
The continuation is the value of $\cx$. After binding $\cx$, $\cx\ 2$ is
evaluated. The continuation of $\cx\ 2$ is $\underline{1}+(\square+3)$.
Therefore, if $\cx$ is a normal function, the result of function application
fills the hole, and the evaluation continues. However, $\cx$ is a
continuation, not a function. The evaluation of $\cx\ 2$ completely ignores the original
continuation $\underline{1}+(\square+3)$. It replaces the continuation with
the continuation denoted by $\cx$ and fills the hole with the argument, $2$.
Thus, $\cx\ 2$ results in evaluating $1+2$. Since the original continuation is
ignored, there is nothing more to do after the evaluation of $1+2$. The result
of the whole expression is $3$.

To compare first-class continuations and functions, consider the following
expression:

$1+(\ebind{\cx}{\efun{\cy}{1+\cy}}{(\cx\ 2)+3})$

In the previous expression,
$\cx$ denotes a continuation, but in this expression, $\cx$ denotes a function.
The continuation and the function are almost the same. Both take an argument
and add $1$ to the argument. However, continuations change the control flow,
while functions do not. Therefore, in this case, $\cx\ 2$ preserves its continuation,
$\underline{1}+(\square+3)$. The return value of the function application is
$3$, and it fills the hole in the original continuation. After the function
returns, $1+(3+3)$ is evaluated, and the whole expression results in $7$.

Let us consider another example:

$
\evcc{\cx}{\eapp{(\evcc{\cy}{\eapp{\cx}{(1+(\evcc{\cz}{\eapp{\cy}{\cz}}))}})}{3}}
$

What is the result of this expression?
The first thing happens during the evaluation is binding of $\cx$. $\cx$ denotes the
continuation of the whole expression, which is the identity function, i.e.
$\square$. Then,
$\eapp{(\evcc{\cy}{\eapp{\cx}{(1+(\evcc{\cz}{\eapp{\cy}{\cz}}))}})}{3}$ is
evaluated. Any function application evaluates the expression at the function
position first. Thus, the redex is
$\evcc{\cy}{\eapp{\cx}{(1+(\evcc{\cz}{\eapp{\cy}{\cz}}))}}$, and the
continuation is $\square\ 3$. The redex defines $\cy$, which denotes the
continuation, $\square\ 3$. Under the environment containing $\cx$ and $\cy$,
$\eapp{\cx}{(1+(\evcc{\cz}{\eapp{\cy}{\cz}}))}$ is evaluated. $\cx$ directly
evaluates to a continuation, and the argument expression becomes the redex.
At this point, the continuation is $(\cx\ \square)\ 3$. The argument
expression is $1+(\evcc{\cz}{\eapp{\cy}{\cz}})$, and $1$ evaluates to $1$.
Then, $\evcc{\cz}{\eapp{\cy}{\cz}}$ becomes the redex, and the continuation is
$(\cx\ (1+\square))\ 3$. Therefore, $\cz$ denotes $(\cx\ (1+\square))\ 3$.
When $\cy$ is applied to $\cz$, the original continuation is ignored, and $\cz$ fills
the hole in the continuation denoted by \cy. Now, the remaining computation
is $\cz\ 3$, which is obtained by filling the hole of $\square\ 3$ with $\cz$.
Applying $\cz$ to $3$ ignores the continuation again, and $(\cx\ (1+3))\
3$ is obtained by filling the hole of $(\cx\ (1+\square))\ 3$ with $3$.
Since $1+3$ evaluates to $4$, $\cx$ is applied to $4$. Then, $4$ fills the hole of
$\square$ and becomes the final result.

Now, we define the semantics of \Lang. First, since continuations are values,
values must be extended.

\[
  v\ ::=\ \cdots\ |\ \contv{k}{s}
\]

A continuation as a value is a pair of a computation
stack and a value stack. $\contv{k}{s}$ denotes a continuation whose computation
stack is $k$ and value stack is $s$. It corresponds to the state $k\ ||\ s$.
The previous chapter shows that applying a continuation $k\ ||\ s$ to a value
$v$ changes the state to $k\ ||\ v::s$. Therefore, applying $\contv{k}{s}$
to $v$ reduces the state to $k\ ||\ v::s$.

Since \Lang has a new sort of an expression, $\evcc{x}{e}$, we need to define
the reduction rule for $\evcc{x}{e}$.

\[
  \evalkd{\evcc{x}{e}}k\ ||\ s\rightarrow
  \evalk{\sigma[x\mapsto\contv{k}{s}]}{e}k\ ||\ s
  \quad\textsc{[Red-Vcc]}
\]

Expression $\evcc{x}{e}$ evaluates $e$ where $x$ denotes the
continuation of $\evcc{x}{e}$. If the current state is $\evalkd{\evcc{x}{e}}k\
||\ s$ and the redex is $\evcc{x}{e}$, the continuation is $k\ ||\ s$.
This continuation can be represented by a value $\contv{k}{s}$. Therefore,
reduction changes the top of the
computation stack to $\sigma[x\mapsto\contv{k}{s}]\vdash e$.

Adding Rule \textsc{Red-Vcc} is not enough to define the semantics of \Lang.
Due to the existence of first-class continuations, function application
expressions must be able to handle not only closures but also first-class
continuations. To define the reduction rule, recall the reduction rule that
handles function application.

\[
  \appk k\ ||\ \conss{v}\conss{\clov{x}{e}{\sigma}}s\rightarrow
  \evalk{\sigma\lbrack x\mapsto v\rbrack}{e}k\ ||\ s
  \quad\textsc{[Red-App2]}
\]

If a continuation is applied instead of a function,
the state should be $\appk k\ ||\ \conss{v}\conss{\contv{k'}{s'}}s$.
The current continuation is $k\ ||\ s$. Since a continuation is applied to a
value, the original continuation is ignored. The new continuation that replaces the original
one is $k'\ ||\ s'$, which comes from $\contv{k'}{s'}$ in the value stack.
The argument is $v$, which is at the top of the value stack.
Thus, reduction results in the state $k'\ ||\ v::s'$.

\[
  \appk k\ ||\ \conss{v}\conss{\contv{k'}{s'}}s\rightarrow k'\ ||\ v::s'
  \quad\textsc{[Red-App2-Cont]}
\]

The following shows that $1+(\textsf{vcc}\ \cx\ \textsf{in}\ ((\cx\ 2)+3))$
evaluates to $3$ by applying reduction according to the semantics:

\[
\begin{array}{lrcr}
& \emptyset\vdash 1+(\textsf{vcc}\ \cx\ \textsf{in}\ ((\cx\ 2)+3))::\square &||&
\blacksquare \\
\rightarrow & \emptyset\vdash 1::\emptyset\vdash \textsf{vcc}\ \cx\ \textsf{in}\
((\cx\ 2)+3)::(+)::\square &||& \blacksquare \\
\rightarrow & \emptyset\vdash \textsf{vcc}\ \cx\ \textsf{in}\ ((\cx\
2)+3)::(+)::\square &||& 1::\blacksquare \\
\rightarrow & \sigma\vdash (\cx\ 2)+3::(+)::\square &||& 1::\blacksquare \\
\rightarrow & \sigma\vdash \cx\ 2::\sigma\vdash 3::(+)::(+)::\square &||&
1::\blacksquare \\
\rightarrow & \sigma\vdash \cx::\sigma\vdash 2::(@)::\sigma\vdash
3::(+)::(+)::\square &||& 1::\blacksquare \\
\rightarrow & \sigma\vdash 2::(@)::\sigma\vdash 3::(+)::(+)::\square &||&
v:: 1::\blacksquare \\
\rightarrow & (@)::\sigma\vdash 3::(+)::(+)::\square &||& 2::v::1::\blacksquare \\
\rightarrow & (+)::\square &||& 2::1::\blacksquare \\
\rightarrow & \square &||& 3::\blacksquare \\
\end{array}
\]

where $\sigma$ is $\lbrack \cx\mapsto\langle(+)::\square\ ||\
1::\blacksquare\rangle\rbrack$ and $v$ is
$\langle(+)::\square \ ||\  1::\blacksquare\rangle$.

The following shows that
$\evcc{\cx}{\eapp{(\evcc{\cy}{\eapp{\cx}{(1+(\evcc{\cz}{\eapp{\cy}{\cz}}))}})}{3}}$
evaluates to $4$:

\[
  \footnotesize
\begin{array}{lrcr}
& \emptyset\vdash\textsf{vcc}\ \cx\ \textsf{in}\ (\textsf{vcc}\ \cy\ \textsf{in}\
\cx\ (1+(\textsf{vcc}\ \cz\ \textsf{in}\ \cy\ \cz)))\ 3
::\square &||& \blacksquare \\
\rightarrow& \sigma_1\vdash(\textsf{vcc}\ \cy\ \textsf{in}\ \cx\ (1+(\textsf{vcc}\ \cz\
\textsf{in}\ \cy\ \cz)))\ 3
::\square &||& \blacksquare \\
\rightarrow& \sigma_1\vdash\textsf{vcc}\ \cy\ \textsf{in}\ \cx\ (1+(\textsf{vcc}\ \cz\
\textsf{in}\ \cy\ \cz))::\sigma_1\vdash3::(@)
::\square &||& \blacksquare \\
\rightarrow& \sigma_2\vdash \cx\ (1+(\textsf{vcc}\ \cz\ \textsf{in}\ \cy\
\cz))::\sigma_1\vdash3::(@)
::\square &||& \blacksquare \\
\rightarrow& \sigma_2\vdash \cx::\sigma_2\vdash 1+(\textsf{vcc}\ \cz\ \textsf{in}\ \cy\
z)::(@)::\sigma_1\vdash3::(@)
::\square &||& \blacksquare \\
\rightarrow& \sigma_2\vdash 1+(\textsf{vcc}\ \cz\ \textsf{in}\ \cy\
\cz)::(@)::\sigma_1\vdash3::(@)
::\square &||& v_1::\blacksquare \\
\rightarrow& \sigma_2\vdash 1::\sigma_2\vdash\textsf{vcc}\ \cz\ \textsf{in}\ \cy\
\cz::(+)::(@)::\sigma_1\vdash3::(@)
::\square &||& v_1::\blacksquare \\
\rightarrow& \sigma_2\vdash\textsf{vcc}\ \cz\ \textsf{in}\ \cy\
\cz::(+)::(@)::\sigma_1\vdash3::(@)
::\square &||& 1::v_1::\blacksquare \\
\rightarrow& \sigma_3\vdash \cy\ \cz::(+)::(@)::\sigma_1\vdash3::(@)
::\square &||& 1::v_1::\blacksquare \\
\rightarrow& \sigma_3\vdash \cy::\sigma_3\vdash
\cz::(@)::(+)::(@)::\sigma_1\vdash3::(@)
::\square &||& 1::v_1::\blacksquare \\
\rightarrow& \sigma_3\vdash \cz::(@)::(+)::(@)::\sigma_1\vdash3::(@)
::\square &||& v_2::1::v_1::\blacksquare \\
\rightarrow& (@)::(+)::(@)::\sigma_1\vdash3::(@)
::\square &||& v_3::v_2::1::v_1::\blacksquare \\
\rightarrow& \sigma_1\vdash3::(@)
::\square &||& v_3::\blacksquare \\
\rightarrow& (@)
::\square &||& 3::v_3::\blacksquare \\
\rightarrow& (+)::(@)::\sigma_1\vdash3::(@)
::\square &||& 3::1::v_1::\blacksquare \\
\rightarrow& (@)::\sigma_1\vdash3::(@)
::\square &||& 4::v_1::\blacksquare \\
\rightarrow& \square &||& 4::\blacksquare \\
\end{array}
\]

where

$
\begin{array}{@{}r@{~}c@{~}l}
  v_1&=&\langle\square,\blacksquare\rangle\\
  v_2&=&\langle\sigma_1\vdash3::(@)::\square,\blacksquare\rangle\\
  v_3&=&\langle(+)::(@)::\sigma_1\vdash3::(@)::\square,1::v_1::\blacksquare\rangle\\
  \sigma_1&=&[\cx\mapsto v_1]\\
  \sigma_2&=&\sigma_1[\cy\mapsto v_2]\\
  \sigma_3&=&\sigma_2[\cz\mapsto v_3]\\
\end{array}
$

\section{Interpreter}

The following Scala code implements the syntax of \Lang:
\footnote{We omit the common part to \plang.}

\begin{verbatim}
sealed trait Expr
...
case class Vcc(x: String, b: Expr) extends Expr
\end{verbatim}

\code{Vcc($x$, $e$)} represents $\evcc{x}{e}$.

In addition, we add a new variant of \code{Value} to represent first-class
continuations.\footnote{We omit the common part to \plang.}

\begin{verbatim}
sealed trait Value
...
case class ContV(k: Cont) extends Value
\end{verbatim}

Since the interpreter treats a continuation as a Scala function,
a \code{ContV} instance has a single field whose type is \code{Cont}.
\code{ContV(k)} represents a continuation \code{k} as a value.

We need to add the \code{Vcc} case to \code{interp} and revise the \code{App}
case to handle continuation application properly.
Consider the \code{Vcc} case first.

\begin{verbatim}
case Vcc(x, b) =>
  interp(b, env + (x -> ContV(k)), k)
\end{verbatim}

The environment is extended with the continuation as a value,
\code{ContV(k)}. Then, the body is evaluated under the extended environment.

The above code clearly shows why the interpreter needs to use CPS. Without CPS,
the continuation cannot be accessed in the source code of the interpreter. There
is no way to construct \code{ContV(k)}. By using CPS, the \code{interp} function
always receives the continuation as an argument and becomes able to use the continuation to
construct \code{ContV(k)}. CPS takes the key role in implementing an interpreter of
a language that provides first-class continuations.

Now, let us fix the \code{App} case.
The \code{interp} function must handle continuation applications in the
\code{App} case.

\begin{verbatim}
case App(f, a) =>
  interp(f, env, fv =>
    interp(a, env, av => fv match {
      case CloV(x, e, fenv) =>
        interp(e, fenv + (x -> av), k)
      case ContV(nk) => nk(av)
    })
  )
\end{verbatim}

When \code{fv} is a \code{CloV} instance, the interpreter behaves just like before.
If \code{fv} is a \code{ContV} instance, the expression applies the continuation
to the argument. The applied continuation is the field of \code{fv}, and the argument
is \code{av}. Since the interpreter expresses a continuation with a Scala function,
applying the Scala function, the field of \code{fv}, to \code{av} is enough.
There is no \code{interp} call. It coincides with that Rule \textsc{Red-App2-Cont}
never adds $\sigma\vdash e$ to the computation stack.

\section{Use of First-Class Continuations}

Imperative languages provide control diverters, such as \code{return},
\code{break}, and \code{continue}, to allow programmers to change control
flows. However, \Lang supports only first-class continuations.

In fact, first-class continuations are the most general form of control flow
manipulation. Continuations represent control flows of programs because the
continuation at some point of execution is the remaining computation.
Changing the current continuation is equivalent to changing the control flow by
making the program compute different things.
In \Lang, programmers can change the current
continuation freely by calling continuations. Therefore, first-class
continuations allow arbitrary changes of control flows.
Expressions like \code{return} change control flows in a fixed way according
to their semantics. On the other hand, programmers using \Lang can make first-class
continuations with $\textsf{vcc}$ and call them at any points. The expressivity of
first-class continuations surpasses that of other control diverters.

Although first-class continuations are much more powerful than other control
diverters, it is difficult to utilize first-class continuations to correctly
implement desired control flow changes. To resolve the difficulty, language
designers can provide other control diverters as syntactic sugar. By doing so,
the designers can make their language convenient for programmers while
preventing the language from being complicated.

\subsection{Return}

A \code{return} expression makes a function immediately return.
Instead of adding \code{return} to \Lang, we can desugar \code{return} to
$\textsf{vcc}$ and a continuation application.

If a programmer writes
$\code{return}\ e$ in the body of a function, $\code{return}\ e$
can be desugared to $\code{r}\ e$, where $\code{r}$ is just an identifier.
\footnote{Assume that the rest of the expression does not use $\code{r}$ at all.}
Then, $\code{r}\ e$ is just an application expression.

Since $\code{r}$ is a free
identifier yet, it must be bound somewhere. The correct way to bind $\code{r}$
is to use $\textsf{vcc}$ because applying $\code{r}$ changes the control flow,
which is possible only by a first-class continuation. Then, where should we put
$\textsf{vcc}$? When there is no
\code{return}, the only way to return from a function is to finish the
evaluation of the function body. After the body is evaluated, its result is used
for the remaining compuation, which is the continuation of the function body.
Thus, applying the continuation of the function body to the return value is the same as
returning from the function. After adding \code{return}, applying $\code{r}$
to a value makes the function immediately return with the given value.
It implies that $\code{r}$ has to denote the continuation of the function body.
Therefore, every function that contains \code{return} in the body needs to be
desugared. An expression $\lambda x.e$ is desugared to
$\lambda x.\evcc{\code{r}}{e}$.

The following example uses $\code{return}$:

$((\lambda \cx.(\code{return}\ 1)+\cx)\ 2)+3$

By desugaring, the above expression becomes the following expression:

$((\lambda \cx.\evcc{\code{r}}{(\code{r}\ 1)+\cx})\ 2)+3$

While evaluating $(\lambda \cx.\evcc{\code{r}}{(\code{r}\ 1)+\cx})\ 2$,
$\code{r}$ denotes $\square+3$. When $\code{r}$ is applied to $1$,
the original continuation disappears, and the only remaining computation is
$1+3$. Therefore, the final result is $4$. The result matches the expected
semantics of \code{return}.

\subsection{Break and Continue}

Many imperative languages provide \code{break} and \code{continue}. Programmers
use them inside loops to change control flows. A \code{break} expression
immediately terminates the loop, and a \code{continue} expression makes the
loop skip the current iteration and directly move on to the beginning of the
next iteration.

Since \Lang lacks loops, we need to add loops first.
Suppose that $\textsf{while0}\ e_1\ e_2$ evaluates $e_2$ repeatedly while $e_1$
evaluates to $\textsf{0}$. When the evaluation terminates, we define the result
to be $0$, which is just an arbitrarily chosen value.

Now, we can add \code{break} and \code{continue} to \Lang as syntactic
sugar. They can occur only inside loops.

When a loop terminates, the continuation of the loop is applied to $0$, which is
the result of the loop. Since \code{break} terminates the loop, it
applies the continuation of the loop to $0$. Thus, when \code{b} denotes the
continuation of the loop, \code{break} can be desugared to $\code{b}\ 0$.
\footnote{Assume that the rest of the expression does not use $\code{b}$ at all.}
To make \code{b} the continuation of a loop, \textsf{vcc} that binds \code{b}
should enclose the loop. Therefore, every loop containing \code{break} must be
desugared. An expression $\textsf{while0}\ e_1\ e_2$ is desugared to
$\evcc{\code{b}}{\textsf{while0}\ e_1\ e_2}$.

For example, consider the following expression:

$\textsf{while0}\ 0\ \code{break}$

This expression is desugared to the following expression:

$\evcc{\code{b}}{\textsf{while0}\ 0\ (\code{b}\ 0)}$

Then, \code{b} is the continuation finishing the evaluation.
Inside the loop, \code{b} is applied to $0$, so the program terminates and
produces $0$ as a result. It coincides with the expected behavior of \code{break}.
Even though the condition of the loop never changes from $0$, the loop
terminates due to the use of \code{break}.

While \code{break} terminates the loop, \code{continue} just skips the current
iteration. It makes the program jump to the condition expression of the loop.
Evaluating the condition expression is the continuation of the body of the loop
because the condition is evaluated after the evaluation of the
body. Therefore, an expression $\textsf{while0}\ e_1\ e_2$ is desugared to
$\textsf{while0}\ e_1\ (\evcc{\code{c}}{e_2})$ when $e_2$ contains
\code{continue}, and \code{continue} is desugared to $\code{c}\ 0$.
\footnote{Assume that the rest of the expression does not use $\code{c}$ at all.}

For example, consider the following expression:

$\textsf{while0}\ 0\ (\code{continue};(1+\efun{\cx}{\cx}))$

This expression is desugared to the following expression:

$\textsf{while0}\ 0\ (\evcc{\code{c}}{((\code{c}\ 0);(1+\efun{\cx}{\cx}))})$

At each iteration, when $\code{c}\ 0$ is evaluated,
the result of whole $\evcc{\code{c}}{((\code{c}\ 0);(1+\efun{\cx}{\cx}))}$ becomes
$0$ without evaluating $1+\efun{\cx}{\cx}$. Then, the loop proceeds to the
next iteration without incurring a run-time error. Thus, the expression never
terminates. It is what we expect from \code{continue}. Without \code{continue},
the expression causes a run-time error because it is impossible to add a number to
a function. However, \code{continue} prevents the addition from being evaluated,
so the expression never terminates.

Note that the selection of $0$ in $\code{c}\ 0$ is completely arbitrary
since the result of the loop body is
never used. We may desugar \code{continue} to $\code{c}\ 42$ instead. It is
different from the case of \code{break}, which must apply \code{b} to $0$ to
make the result of the loop $0$.

\section{Exercises}

\begin{exercise}
\labex{first-class-continuations-result}

What is the result of each of the following expressions?

\begin{enumerate}
  \item $\evcc{\cx}{(\eadd{(\eapp{\cx}{3})}{4})}$
  \item $\eapp{(\evcc{\cx}{(2+(\eapp{\cx}{(\efun{\cy}{(\cy+\cy)})}))})}{3}$
  \item $\eapp{\eapp{(\evcc{\cx}{\cx})}{5}}$
  \item $\eapp{\eapp{(\evcc{\cx}{\cx})}{(\efun{\cy}{\cy})}}{5}$
  \item $\eapp{\eapp{(\evcc{\cx}{\cx})}{(\efun{\cy}{(\eapp{\cy}{\cy})})}}{5}$
\end{enumerate}

\end{exercise}

\begin{exercise}
\labex{first-class-continuations-print}

The following is an \code{interp} for \Lang that prints a given
expression each time it is called:

\begin{verbatim}
def interp(e: Expr, env: Env, k: Cont): Value = {
  println(e)
  e match { ... }
}
\end{verbatim}

Let $e$ be an expression $\eapp{(\evcc{\cx}{\cx})}{\efun{\cy}{\cy}}$, which is
\code{App(Vcc("x", Id("x")), Fun("y", Id("y")))}. What does \code{interp($e$,
Map(), x => x)} print?

\end{exercise}

\begin{exercise}
\labex{first-class-continuations-reduction}

Write the reduction steps of $(\evcc{\cx}{42+(\cx\ 2)})+8$.

\end{exercise}

\begin{exercise}
\labex{first-class-continuations-var}

This exercise extends \Lang with mutable variables.
Complete the following interpreter implementation:

\begin{verbatim}
type Env = Map[String, Addr]
type Sto = Map[Addr, Value]
type Cont = (Value, Sto) => (Value, Sto)

def interp(e: Expr, env: Env, sto: Sto, k: Cont): (Value, Sto) = e match {
  case Num(n) => ???
  case Id(x) => ???
  case Fun(x, b) => ???
  case App(f, a) => ???
  case Set(x, e) => ???
  case Vcc(x, b) => ???
}
\end{verbatim}

\end{exercise}
