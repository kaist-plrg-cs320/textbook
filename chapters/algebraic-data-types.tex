\chapter{Algebraic Data Types}
\labch{algebraic-data-types}

\renewcommand{\plang}{\textsf{TFAE}\xspace}
\renewcommand{\Lang}{\textsf{TVFAE}\xspace}

Algebraic Data Types (ADTs) are ubiquitous in functional programming. \refch{pattern-matching} explains
the concept of an ADT and how programmers can use ADTs in Scala. ADTs are useful
when a single type includes values of different structures. Such types are
common in computer science. For example, lists and options, which
almost every programming language provides, are typically implemented as ADTs. In addition, ASTs,
which all the compilers, interpreters, and static analyzers need, can be easily
implemented as ADTs as well.

Nonrecursive ADTs can be considered as syntactic sugar in
\plang.\footnote{Here, \plang means the extended version
defined in \refsec{extension}.}
Let us see how they can be desugared with an example. Consider the
following Scala code:

\begin{verbatim}
sealed trait Fruit
case class Apple(radius: Int) extends Fruit
case class Banana(radius: Int, height: Int) extends Fruit
\end{verbatim}

A fruit is either an apple or a banana. In this example, we are interested in
the sizes of fruits. An apple is approximated as a sphere and,
therefore, parametrized by its radius. A banana is approximated as a cylinder
and, therefore, parametrized by its radius and height.

We can easily create values that represent apples and bananas like below.

\begin{verbatim}
val apple = Apple(5)
val banana = Banana(2, 6)
\end{verbatim}

In the above code, \code{apple} represents an apple whose radius is \code{5},
and \code{banana} represents a banana whose radius is \code{2} and height is
\code{6}.

In \plang, we can represent a fruit as a value of
$\tnum\times(\tnum\times(\tnum\times\tnum))$. Thus, a fruit value is a pair.
The first value of the pair indicates which fruit it is. If the value is $0$,
it is an apple. Otherwise, it is a banana. The second value of the
pair is another pair, which represents the size of the fruit. If the fruit is an
apple, only the first value of the second pair is meaningful. The value denotes
the radius of the apple. Therefore, the following value represents an apple
whose radius is $5$:

$(0,(5,(0,0)))$

Note that $(0,0)$ can be replaced with any pair of integers. On the other hand,
if the fruit is a banana, only the second value, which is a pair again, of the
pair is meaningful. The pair consists of the radius and height of the banana.
The following value represents a banana whose radius is $2$ and height is $6$:

$(1,(0,(2,6)))$

Note that $1$ can be replaced with any nonzero integer, and $0$ can be replaced
with any integer.

It is tedious and error-prone to make fruit values like the above, while Scala
provides a simple way to construct fruit values. In \plang, we can define
functions to mimic constructors in Scala.

$
\begin{array}{@{}l}
  \ebind{\code{Apple}}{\efunt{\cx}{\tnum}{(0,(\cx,(0,0)))}}{\\
  \ebind{\code{Banana}}{\efunt{\cx}{(\tnum\times\tnum)}{(1,(0,\cx))}}{\\
  \cdots}}
\end{array}
$

$\code{Apple}$ is a function that takes an integer as an argument and returns an
apple whose radius is the given integer. Similarly, $\code{Banana}$ is a
function that takes a pair of integers as an argument and returns a banana whose
size is represented by the given pair. We can now easily create fruit values
with $\code{Apple}$ and $\code{Banana}$.

$
\begin{array}{@{}l}
  \ebind{\code{apple}}{\code{Apple}\ 5}{\\
  \ebind{\code{banana}}{\code{Banana}\ (6,2)}{\\
  \cdots}}
\end{array}
$

In Scala, a typical way to use a value of an ADT is pattern matching. For
instance, consider a function that computes the radius of a given fruit.
The function can be implemented like below.

\begin{verbatim}
def radius(f: Fruit): Int = f match {
  case Apple(r) => r
  case Banana(r, _) => r
}
\end{verbatim}

\plang does not have pattern matching, but we can exploit the fact that the
first value of a given pair indicates which fruit it is. We use a conditional
expression to perform a certain operation when the fruit is an apple, i.e. the
first value is $0$, and another opertaion when the fruit is a banana,
i.e. the first value is nonzero. The following expression defines the
$\code{radius}$ function:

$\ebind{\code{radius}}{\efunt{\cx}{(\tnum\times(\tnum\times(\tnum\times\tnum)))}
{\eifz{\cx\textsf{.1}}{\cx\textsf{.2.1}}{\cx\textsf{.2.2.1}}}}{\cdots}$

This example shows that we can desugar ADTs and pattern matching to pairs,
functions, and conditional expressions in \plang. The ADT of the example has
only two variants, which have one or two parameters. ADTs can have any number of
variants, and vairants can have any number of parameters. The same strategy can be
used to desugar ADTs with more variants and variants with more parameters.

Although nonrecursive ADTs can be desugared in \plang, there are a few flaws.
First, desugared programs have unnecessary values. Even when we make an apple, we
need $(0,0)$, which is a pair for the size of a banana. Similarly, a banana
value requires the size of an apple. They add unessential complexity and
computation to the code. Second, a single type may represent conceptually
different types when a single program uses multiple ADTs. In practice, it is
common to use multiple ADTs in a single program. Recall that the type of a fruit
is $\tnum\times(\tnum\times(\tnum\times\tnum))$. The same type may represent
other types as well. For example, the type of an electronic product can also be
$\tnum\times(\tnum\times(\tnum\times\tnum))$. In this case, the type system
allows a function intended to take a fruit to take an electronic product
as an argument. It does not incur any type errors at run time but can cause
undesirable behaviors.

These flaws can be resolved by adding primitive support for ADTs to the language.
After adding ADTs,
programs do not require unnecessary values to construct values of ADTs.
In addition, each ADT can be defined as a separate type, so
conceptually different types can be correctly distinguished even when they share
the same structure.

The most critical limitation of the current desugaring strategy is the missing
support for recursive ADTs. Consider the following Scala code:

\begin{verbatim}
sealed trait List
case object Nil extends List
case class Cons(h: Int, t: List) extends List
\end{verbatim}

which implements an integer list type. A list is one of the most famous
recursive types. Look at the definition of \code{Cons}. \code{Cons} is one
variant of \code{List}, so it defines \code{List}. At the same time, the
definition uses \code{List} as the type of the second parameter. Thus, \code{List}
is a recursively defined type, whose definition depends on itself.

Can we desugar the definition of a list in \plang? For desugaring, the first
thing to do is to determine the type of a list. Let the type of a list be
$\tau$. Then, $\tau$ equals $(\tnum, \tau')$ for some $\tau'$. The first element is an integer that
indicates which variant the value denotes. When the integer is $0$, the value is
$\code{Nil}$; otherwise, the value is $\code{Cons}$. When the value is
$\code{Nil}$, no other data is required since $\code{Nil}$ does not have any
parameters. Thus, the second element of a type $\tau'$ is for $\code{Cons}$. Since
$\code{Cons}$ has two parameters, an integer and a list, $\tau'$ equals
$(\tnum,\tau)$. Then, we obtain the equation $\tau=(\tnum,(\tnum,\tau))$.
However, as discussed in the previous chapter, no type in \plang can be the same
as a part of itself. Therefore, there is no such $\tau$. We can conclude that we
cannot desugar lists in \plang. In general, recursive ADTs cannot be expressed
in \plang.

This chapter defines \Lang by extending \plang\footnote{For the rest of the
chapter, local variable definitions and types are not parts of \plang. However,
we may keep using them in examples.} with ADTs, each of which can be
either nonrecursive or recursive. It allows programmers to represent ADTs
efficiently and concisely. In addition, many interesting recursive data types
become able to be used in programs.

\section{Syntax}

First, we introduce type identifiers, which are the names of types defined by
programmers. For example,
\code{Fruit} of the previous example is a type identifier. Let $\embox{TId}$ be
the set of every type identifier.

\[ t \in \embox{TId} \]

The metavariable $t$ ranges over type identifiers. Since $\embox{TId}$ includes only
the names of user-defined types, $\tnum$ is not a member of $\embox{TId}$.

The names of user-defined types can be used as types. For example, \code{Fruit},
which is a type name, is used as a type in \code{def radius(f: Fruit): Int = ...}.
Therefore, we extend the syntax of types as follows:

\[ \tau \ ::= \ \cdots\ |\  t \]

where $t$ is a type that includes every value of an ADT whose name is $t$.

Now, we define the syntax of expressions:
\[
e \ ::= \ \cdots
\ |\ \etdef{t}{x}{\tau}{x}{\tau}{e}
\ |\ \ematch{e}{x}{x}{e}{x}{x}{e}
\]

\begin{itemize}
  \item $\etdef{t}{x_1}{\tau_1}{x_2}{\tau_2}{e}$ is an expression that defines a new
    type. The name of the type is $t$, and the type has two variants. The name
    of the first variant is $x_1$, and it has a single parameter whose type is
    $\tau_1$. Similarly, the name of the second variant is $x_2$, and it has a
    single parameter whose type is $\tau_2$. The names of the variants serve
    as constructors in $e$. The type name $t$ can be used in $e$ as a type. In
    addition, since types can be recursively defined, $t$ can appear also in
    $\tau_1$ and $\tau_2$.
  \item $\ematch{e}{x_1}{x_3}{e_1}{x_2}{x_4}{e_2}$ is a pattern matching
    expression. $e$ is the target of the pattern matching. $x_1$ is the name of
    the variant handled by the first case, and $x_2$ is the name of the variant
    handled by the second case. In the first case, $x_3$ denotes the value held
    by the match target, and $e_1$ determines the result. Similarly, in the
    second case, $x_4$ denotes the value held by the match target, and $e_2$
    determines the result.
\end{itemize}

Note that each type can have only two variants, and each variant can have only
one parameter. This restriction can be easily removed. For brevity, this
chapter keeps the restriction.

Let us see some example expressions. The following expression defines the
\code{Fruit} type and the \code{radius} function:

$
\etdef{\code{Fruit}}{\code{Apple}}{\tnum}{\code{Banana}}{(\tnum\times\tnum)}{\\
\ebind{\code{radius}}{\efunt{\cx}{\code{Fruit}}{
  \ematch{\cx}{\code{Apple}}{\cy}{\cy}{\code{Banana}}{\cy}{\cy\textsf{.1}}
}}{\\
\cdots
}}$

The following expression defines the \code{List} type:

$
\etdef{\code{List}}{\code{Nil}}{\tnum}{\code{Cons}}{(\tnum\times\code{List})}
{\cdots}$

Note that \code{Nil} has one parameter since every variant of \Lang must have a
parameter. \code{Nil} can have any value because the value is not used at all anyway.

Recursive data types are typically used with recursive functions. If we add
recursive functions of \textsf{TRFAE} to the language, we can implement the
following \code{sum} function, which calculates the sum of every integer in a
given list:

$\erect{\code{sum}}{\cx}{\code{List}}{\tnum}{
  \ematch{\cx}{\code{Nil}}{\cy}{0}{\code{Cons}}{\cy}{
    \cy\textsf{.1}+(\code{sum}\ \cy\textsf{.2})
  }
}{\cdots}$

\section{Dynamic Semantics}

We should introduce two new sorts of a value: a variant value and a constructor.

\[ v \ ::= \ \cdots\ |\ x(v)\ |\ \langle x\rangle \]

\begin{itemize}
  \item $x(v)$ is a value of a variant named $x$. It contains one value $v$.
    A variant value can be the target of pattern matching.
  \item $\langle x\rangle$ is a constructor of a variant named $x$. A
    constructor can be considered as a special kind of a function because it can
    be applied to a value. When a constructor is applied to a value, a variant
    value is constructed.
\end{itemize}

For example, $\code{Apple}(5)$ is an apple value whose radius is $5$, and
$\code{Banana}((2,6))$ is a banana value whose radius is $2$ and height is $6$.
Both $\langle\code{Apple}\rangle$ and $\langle\code{Banana}\rangle$ are
constructors. When $\langle\code{Apple}\rangle$ is applied to $5$, the result is
$\code{Apple}(5)$, and when $\langle\code{Banana}\rangle$ is applied to $(2,6)$,
the result is $\code{Banana}((2,6))$.

Now, let us define the dynamic semantics of the added expressions. First,
consider an expression that defines a new type.

\semanticrule{TypeDef}{
  If
    \evaln{\sigma[x_1\mapsto \langle x_1\rangle,x_2\mapsto \langle x_2\rangle]}
    {e}{v}, \\
  then
    \evaldn{\etdef{t}{x_1}{\tau_1}{x_2}{\tau_2}{e}}{v}.
}

\vspace{-1em}

\[
  \inferrule
  { \eval{\sigma[x_1\mapsto \langle x_1\rangle,x_2\mapsto \langle x_2\rangle]}
    {e}{v} }
  { \evald{\etdef{t}{x_1}{\tau_1}{x_2}{\tau_2}{e}}{v} }
  \quad\textsc{[TypeDef]}
\]

The result of $\etdef{t}{x_1}{\tau_1}{x_2}{\tau_2}{e}$ equals the
result of $e$. The constructors of the variants have to be available during
the evaluation of $e$. For example, if the expression defines \code{Fruit}, then
$e$ should be able to construct values with the constructors,
$\langle\code{Apple}\rangle$ and $\langle\code{Banana}\rangle$. Programmers can
use the names of the variants to denote the constructors. They can
write code like $\code{Apple}\ 5$ and $\code{Banana}\ (2,6)$. It shows that
the identifier $\code{Apple}$ must denote the value $\langle\code{Apple}\rangle$ and
that the identifier $\code{Banana}$ must denote the value
$\langle\code{Banana}\rangle$. For this reason, the environment used for the
evaluation of $e$ contains a mapping from $x_1$ to $\langle x_1\rangle$ and
a mapping from $x_2$ to $\langle x_2\rangle$.

The other new expression is a pattern matching expression. Like the dynamic
semantics of a conditional expression, we define two rules: one for when the
target is handled by the first case and the other for when the target is handled
by the second case.

\semanticrule{Match-L}{
  If
    \evaldn{e}{x_1(v')} and
    \evaln{\sigma[x_3\mapsto v']}{e_1}{v}, \\
  then
    \evaldn{\ematch{e}{x_1}{x_3}{e_1}{x_2}{x_4}{e_2}}{v}.
}

\vspace{-1em}

\[
  \inferrule
  { \evald{e}{x_1(v')} \\
    \eval{\sigma[x_3\mapsto v']}{e_1}{v} }
  { \evald{\ematch{e}{x_1}{x_3}{e_1}{x_2}{x_4}{e_2}}{v} }
  \quad\textsc{[Match-L]}
\]

For pattern matching, the target has to be evaluated first. Therefore, $e$
is the first expression to be evaluated. If the result of $e$ is $x_1(v')$, then
it matches the first case. Thus, $e_1$ is the next expression to be evaluated.
During the evaluation of $e_1$, $x_3$ denotes $v'$. Therefore, the
environment has a mapping from $x_3$ to $v'$. The result of $e_1$ is the result
of the whole pattern matching expression.

\semanticrule{Match-R}{
  If
    \evaldn{e}{x_2(v')} and
    \evaln{\sigma[x_4\mapsto v']}{e_2}{v}, \\
  then
    \evaldn{\ematch{e}{x_1}{x_3}{e_1}{x_2}{x_4}{e_2}}{v}.
}

\vspace{-1em}

\[
  \inferrule
  { \evald{e}{x_2(v')} \\
    \eval{\sigma[x_4\mapsto v']}{e_2}{v} }
  { \evald{\ematch{e}{x_1}{x_3}{e_1}{x_2}{x_4}{e_2}}{v} }
  \quad\textsc{[Match-R]}
\]

On the other hand, if $e$ evaluates to $x_2(v')$, it matches the second case.
Then, $e_2$ is evaluated under the environment that contains a mapping from
$x_4$ to $v'$. The result of $e_2$ is the result of the whole pattern matching
expression.

In addition, we should define a new rule for function application. In \plang,
closures are the only values that can be applied to values. However, \Lang has
constructors, which also can be applied to values. We need a rule to handle
such cases.

\semanticrule{App-Cnstr}{
  If
    \evaldn{e_1}{\langle x\rangle} and
    \evaldn{e_2}{v}, \\
  then
    \evaldn{e_1\ e_2}{x(v)}.
}

\vspace{-1em}

\[
  \inferrule
  { \evald{e_1}{\langle x\rangle} \\
    \evald{e_2}{v} }
  { \evald{e_1\ e_2}{x(v)} }
  \quad\textsc{[App-Cnstr]}
\]

If $e_1$ evaluates to the constructor of a variant named $x$, the application
expression constructs a value of the variant.

\newcommand{\fruitexpr}{
\etdef{\code{Fruit}}{\code{Apple}}{\tnum}{\code{Banana}}{(\tnum\times\tnum)}{
\ematch{(\code{Apple}\
5)}{\code{Apple}}{\cy}{\cy}{\code{Banana}}{\cz}{\cz\textsf{.1}}
}}

Let $e$ be $\fruitexpr$.
The following proof tree proves that $e$ evaluates to $5$:

\[
  \inferrule
  {
    \inferrule
    {
      \inferrule
      {
        \inferrule
        { \code{Apple}\in\dom{\sigma_1} }
        { \eval{\sigma_1}{\code{Apple}}{\langle\code{Apple}\rangle} } \\
        \eval{\sigma_1}{5}{5}
      }
      { \eval{\sigma_1}{\code{Apple}\ 5}{\code{Apple}(5)} } \\
      \inferrule
      { \cy\in\dom{\sigma_2} }
      { \eval{\sigma_2}{\cy}{5} }
    }
    { \eval{\sigma_1}{
        \ematch{(\code{Apple}\
        5)}{\code{Apple}}{\cy}{\cy}{\code{Banana}}{\cz}{\cz\textsf{.1}}
      }{5} }
  }
  { \evale{e}{5} }
\]

where

$
  \begin{array}{@{}r@{~}c@{~}l}
    \sigma_1&=&[\code{Apple}\mapsto\langle\code{Apple}\rangle,\code{Banana}\mapsto\langle
    \code{Banana}\rangle] \\
    \sigma_2&=&\sigma_1[\cy\mapsto 5]
  \end{array}
$

\section{Interpreter}

The following code implements expressions of \Lang:

\begin{verbatim}
sealed trait Expr
...
case class TypeDef(
  t: String, v1: String, vt1: Type,
  v2: String, vt2: Type, b: Expr
) extends Expr
case class Match(
  e: Expr, v1: String, x1: String, e1: Expr,
  v2: String, x2: String, e2: Expr
) extends Expr
\end{verbatim}

\code{TypeDef($t$, $x_1$, $\tau_1$, $x_2$, $\tau_2$, $e$)} represents
$\etdef{t}{x_1}{\tau_1}{x_2}{\tau_2}{e}$, and
\code{Match($e$, $x_1$, $x_3$, $e_1$, $x_2$, $x_4$, $e_2$)} represents
$\ematch{e}{x_1}{x_3}{e_1}{x_2}{x_4}{e_2}$.

The following code implements values of \Lang:

\begin{verbatim}
sealed trait Value
...
case class VariantV(x: String, v: Value) extends Value
case class ConstructorV(x: String) extends Value
\end{verbatim}

\code{VariantV($x$, $v$)} represents $x(v)$, and
\code{ConstructorV($x$)} represents $\langle x\rangle$.

\begin{verbatim}
case TypeDef(_, v1, _, v2, _, b) =>
  interp(b, env + (v1 -> ConstructorV(v1)) + (v2 -> ConstructorV(v2)))
\end{verbatim}

An expression defining a type evaluates its body under the
environment with the constructors.

\begin{verbatim}
case Match(e, v1, x1, e1, v2, x2, e2) =>
  interp(e, env) match {
    case VariantV(`v1`, v) => interp(e1, env + (x1 -> v))
    case VariantV(`v2`, v) => interp(e2, env + (x2 -> v))
  }
\end{verbatim}

A pattern matching expression evaluates the target expression first.
If the result is a variant value and its name is the same as \code{v1},
then \code{e1} is evaluated under the environment with the value of
\code{x1}. If the name is the same as \code{v2}, then \code{e2} is
evaluated under the environment with the value of \code{x2}.

\begin{verbatim}
case App(f, a) =>
  val fv = interp(f, env)
  val av = interp(a, env)
  fv match {
    case CloV(x, b, fEnv) =>
      interp(b, fEnv + (x -> av))
    case ConstructorV(x) =>
      VariantV(x, av)
  }
\end{verbatim}

Function application allows a constructor to occur at the function position.
If a constructor appears, the result is a variant value that
contains the value denoted by the argument.

\section{Static Semantics}

To define the static semantics, the definition of a type environment should be
revised first. In \plang, type environments store the types of variables. They
are finite partial functions from identifiers to types. In \Lang, type environments
have more things to do. They have to store the information about user-defined types.
The type checking process uses the information. Let $\Gamma$ be the type
environment when $\etdef{t}{x_1}{\tau_1}{x_2}{\tau_2}{e}$ is
type-checked. Adding the information of $t$ to $\Gamma$ yields
$\Gamma[t:\{(x_1,\tau_1),(x_2,\tau_2)\}]$. The variants are elements of a set
since the order between them is unimportant. We will write
$t=x_1@\tau_1+x_2@\tau_2$ to denote $t:\{(x_1,\tau_1),(x_2,\tau_2)\}$ just for
the sake of intuitive notation.
The domain of a type environment now needs to include $t$, which is
a type identifier. Also, the codomain has to contain $x_1@\tau_1+x_2@\tau_2$.
Below is the revised definition. Note that $\mathcal{P}(A)$ denotes the power set
of $A$.

\[
  \embox{TEnv}=
(\embox{Id}\cup\embox{TId})\finto
(T\cup\mathcal{P}(\embox{Id}\times T))
\]

\subsection{Well-Formed Types}

An arbitrary type identifier can be a type in \Lang. For example, $\code{Fruit}$ is a
type. It is true regardless of whether the type $\code{Fruit}$ is bound by a type
definition. Programmers can write $\efunt{\cx}{\code{Fruit}}{\cx}$ without defining
\code{Fruit}. Such a function does not make sense at all since \code{Fruit} is a
free type identifier, whose information is missing. Expressions with free type
identifiers are weird and useless. Furthermore, they can break the type soundness of the
type system.

To prevent free type identifiers, we introduce the notion of a well-formed type.
A \textit{well-formed}\index{well-formed} type is a type that does not contain
any free type identifiers. The opposite of a well-formed type is an
\textit{ill-formed}\index{ill-formed} type, which contains a free type identifier.

When a type appears in an expression, the type must be well-formed. Any
expression that contains an ill-formed type is rejected by the type system.
In this way, weird programs like $\efunt{\cx}{\code{Fruit}}{\cx}$ can be
effectively prevented by type checking.

Now, we defined well-formed types. The well-formedness relation is a relation
over type environments and types because we need type information in a type
environment to decide whether a certain type identifier is free or not.

\[\vdash\subseteq\embox{TEnv}\times T\]

$\wftd{\tau}$ denotes that $\tau$ is well-formed under $\Gamma$.

Well-formedness rules prescribe which types are well-formed.

\typerule{Wf-NumT}{
  \wftdn{\tnum}.
}

\vspace{-1em}

\[
  \wftd{\tnum}
  \quad\textsc{[Wf-NumT]}
\]

The first well-formedness rule states that $\tnum$ is always well-formed.
$\tnum$ is neither a type identifier nor the name of a user-defined type. It is
a built-in type, which always exists. Thus, $\tnum$ is well-formed under any
type environment.

\typerule{Wf-ArrowT}{
  If
    \wftdn{\tau_1} and
    \wftdn{\tau_2}, \\
  then
    \wftdn{\tarrow{\tau_1}{\tau_2}}.
}

\vspace{-1em}

\[
  \inferrule
  { \wftd{\tau_1} \\
    \wftd{\tau_2} }
  { \wftd{\tarrow{\tau_1}{\tau_2}} }
  \quad\textsc{[Wf-ArrowT]}
\]

If both $\tau_1$ and $\tau_2$ are well-formed, then
$\tarrow{\tau_1}{\tau_2}$ also is well-formed. The reason is clear: if both
$\tau_1$ and $\tau_2$ lack free type identifiers, then $\tarrow{\tau_1}{\tau_2}$
also does.

\typerule{Wf-IdT}{
  If $t$ in the domain of $\Gamma$,\\
  then \wftdn{t}.
}

\vspace{-1em}

\[
  \inferrule
  { t\in\dom{\Gamma} }
  { \wftd{t} }
  \quad\textsc{[Wf-IdT]}
\]

If a type identifier can be found in the type environment, the type identifier is
a well-formed type. For example, if $\efunt{\cx}{\code{Fruit}}{\cx}$ is the
whole expression, \code{Fruit} is ill-formed since there is no \code{Fruit} in
the type environment. However, in
$\etdef{\code{Fruit}}{\code{Apple}}{\tnum}{\code{Banana}}{(\tnum\times\tnum)}{
  \efunt{\cx}{\code{Fruit}}{\cx}}$, \code{Fruit} is well-formed since the
expression puts the definition of \code{Fruit} into the type environment.

\subsection{Typing Rules}

Before defining the typing rules, we need to classify newly introduced values.
The type of a variant value is the type that defines the variant. For example,
if $x$ is a variant of $t$, then $x(v)$ is a value of $t$. The type of a
constructor is a function type because constructors can be applied to values.
Each constructor takes a value of the type that is specified in the type
definition and returns a value of the type that constructor belongs to. For
instance, if $x$ is a variant of $t$ and $t$ defines the parameter type of $x$
to be $\tau$, then the type of $x$ is $\tarrow{\tau}{t}$.

Now, let us define the typing rules. First, consider expressions that define
types.

\typerule{Typ-TypeDef}{
  \begin{tabular}{@{\hskip0pt}l@{\hskip-10pt}l}
    If \\
    & $\Gamma'$ denotes
    $\Gamma[t=x_1@\tau_1+x_2@\tau_2,x_1:\tarrow{\tau_1}{t},x_2:\tarrow{\tau_2}{t}]$, \\
    & \wftn{\Gamma'}{\tau_1}, \\
    & \wftn{\Gamma'}{\tau_2}, and \\
    & \typeofn{\Gamma'}{e}{\tau}, \\
    then \\
    & \typeofdn{\etdef{t}{x_1}{\tau_1}{x_2}{\tau_2}{e}}{\tau}.
  \end{tabular}
}

\vspace{-1em}

\[
  \inferrule
  {
    \Gamma'=\Gamma[t=x_1@\tau_1+x_2@\tau_2,x_1:\tarrow{\tau_1}{t},x_2:\tarrow{\tau_2}{t}] \\
    \wft{\Gamma'}{\tau_1} \\
    \wft{\Gamma'}{\tau_2} \\
    \typeof{\Gamma'}{e}{\tau} }
  { \typeofd{\etdef{t}{x_1}{\tau_1}{x_2}{\tau_2}{e}}{\tau} }
  \quad\textsc{[Typ-TypeDef]}
\]

Rule~\textsc{Typ-TypeDef} defines the type of an expression that defines a new type.
First, the definition of $t$ must
be added to the type environment. In addition, since $e$ can use the
constructors, the type environment should also contain the types of $x_1$ and
$x_2$. $\Gamma'$ denotes the type environment after adding the type definition
and the constructors.
Since $\tau_1$ and $\tau_2$ are user-written type annotations, they can be
ill-formed types. To rule out ill-formed types, the well-formedness of $\tau_1$
and $\tau_2$ is checked. For the well-formedness checking, $\Gamma'$ is used
instead of $\Gamma$. The use of $\Gamma'$ allows recursively defined types.
Since $\Gamma'$ contains the definition of $t$, $\tau_1$ and $\tau_2$ can be
well-formed even when $t$ occurs in them. Finally, $e$ is type-checked under
$\Gamma'$. The type of $e$ is the type of the whole type-defining expression.
Note that the type of $e$ does not need well-formedness checking since it is the
result of type checking, not a user-written type annotation. The same principle
can be applied to all the other typing rules in \Lang.

\typerule{Typ-Match}{
  \begin{tabular}{@{\hskip0pt}l@{\hskip-10pt}l}
    If \\
    & \typeofdn{e}{t}, \\
    & $t$ is in the domain of $\Gamma$, \\
    & $\Gamma(t)$ equals $x_1@\tau_1+x_2@\tau_2$, \\
    & \typeofn{\Gamma[x_3:\tau_1]}{e_1}{\tau}, and \\
    & \typeofn{\Gamma[x_4:\tau_2]}{e_2}{\tau}, \\
    then \\
    & \typeofdn{\ematch{e}{x_1}{x_3}{e_1}{x_2}{x_4}{e_2}}{\tau}.
  \end{tabular}
}

\vspace{-1em}

\[
  \inferrule
  { \typeofd{e}{t} \\
    t\in\dom{\Gamma} \\
    \Gamma(t)=x_1@\tau_1+x_2@\tau_2 \\
    \typeof{\Gamma[x_3:\tau_1]}{e_1}{\tau} \\
    \typeof{\Gamma[x_4:\tau_2]}{e_2}{\tau} }
  { \typeofd{\ematch{e}{x_1}{x_3}{e_1}{x_2}{x_4}{e_2}}{\tau} }
  \quad\textsc{[Typ-Match]}
\]

Rule~\textsc{Typ-Match} defines the type of a pattern matching expression.
First, the type of $e$, which is the target, is computed. Since pattern matching
in \Lang can be used for only user-defined types, the type of the target must be
$t$, which is a type identifier. In addition, its definition must be
found in the type environment. Since $x_1$ and $x_2$ in the cases denote the names
of variants, the names of $t$'s variants must be $x_1$ and $x_2$. Note that
the order does not need to be the same. The pattern matching expression can place
$x_1$ first while the type definition places $x_2$ first. The fact that the
order does not matter is reflected in the rule by representing variant
information as $x_1@\tau_1+x_2@\tau_2$, which actually denotes a set
$\{(x_1,\tau_1),(x_2,\tau_2)\}$. Finally, $e_1$ and $e_2$ are type-checked.
The type of $x_3$ is $\tau_1$ during the type checking of $e_1$, and the type of
$x_4$ is $\tau_2$ during the type checking of $e_2$. The type of $e_1$ must be
the same as the type of $e_2$ because an expression can have at most one type.
It is similar to the typing rule of a conditional expression. The common type to
$e_1$ and $e_2$ is the type of the whole pattern matching expression.

In addition, one typing rule from \plang needs a revision:
Rule~\textsc{Typ-Fun}.

\typerule{Typ-Fun}{
  If
    \wftdn{\tau_1} and
    \typeofn{\Gamma[x:\tau_1]}{e}{\tau_2}, \\
  then
    \typeofdn{\efunt{x}{\tau_1}{e}}{\tarrow{\tau_1}{\tau_2}}.
}

\vspace{-1em}

\[
  \inferrule
  { \wftd{\tau_1} \\
    \typeof{\Gamma[x:\tau_1]}{e}{\tau_2} }
  { \typeofd{\efunt{x}{\tau_1}{e}}{\tarrow{\tau_1}{\tau_2}} }
  \quad\textsc{[Typ-Fun]}
\]

A lambda abstraction also has a user-written type annotation, which might be
ill-formed. To rule out ill-formed types, the well-formedness of the type
annotation must be checked.

Let $e$ be $\fruitexpr$.
The following proof trees prove that the type of $e$ is $\tnum$:

\[
  \small
  \inferrule
  {
    \inferrule
    { \code{Apple}\in\dom{\Gamma_1} }
    { \typeof{\Gamma_1}{\code{Apple}}{\tarrow{\tnum}{\code{Fruit}}} }
    \\
    \typeof{\Gamma_1}{5}{\tnum}
  }
  { \typeof{\Gamma_1}{\code{Apple}\ 5}{\code{Fruit}} }
\]

\[
  \small
  \inferrule
  { \cy\in\dom{\Gamma_2} }
  { \typeof{\Gamma_2}{\cy}{\tnum} }
\]

\[
  \small
  \inferrule
  {
    \inferrule
    { \cz\in\dom{\Gamma_3} }
    { \typeof{\Gamma_3}{\cz}{\tnum\times\tnum} }
  }
  { \typeof{\Gamma_3}{\cz\textsf{.1}}{\tnum} }
\]

\[
  \small
  \inferrule
  {
    \typeof{\Gamma_1}{\code{Apple}\ 5}{\code{Fruit}}
    \\
    \code{Fruit}\in\dom{\Gamma_1}
    \\
    \Gamma_1(\code{Fruit})=\code{Apple}@\tnum+\code{Banana}@(\tnum\times\tnum)
    \\
    \typeof{\Gamma_2}{\cy}{\tnum}
    \\
    \typeof{\Gamma_3}{\cz\textsf{.1}}{\tnum}
  }
  { \typeof{\Gamma_1}{
      \ematch{(\code{Apple}\
      5)}{\code{Apple}}{\cy}{\cy}{\code{Banana}}{\cz}{\cz\textsf{.1}}
    }{\tnum} }
\]

\[
  \small
  \inferrule
  {
    \Gamma_1=\Gamma_1
    \\
    \wft{\Gamma_1}{\tnum}
    \\
    \inferrule
    {
      \wft{\Gamma_1}{\tnum}
      \\
      \wft{\Gamma_1}{\tnum}
    }
    { \wft{\Gamma_1}{\tnum\times\tnum} }
    \\
    \typeof{\Gamma_1}{
      \ematch{(\code{Apple}\
      5)}{\code{Apple}}{\cy}{\cy}{\code{Banana}}{\cz}{\cz\textsf{.1}}
    }{\tnum}
  }
  { \typeofe{e}{\tnum} }
\]

where

$
\small
\begin{array}{@{}r@{~}c@{~}l}
  \Gamma_1&=&[\code{Fruit}=\code{Apple}@\tnum+\code{Banana}@(\tnum\times\tnum),
  \code{Apple}:\tnum\rightarrow\code{Fruit},
  \code{Banana}:(\tnum\times\tnum)\rightarrow\code{Fruit}]\\
  \Gamma_2&=&\Gamma_1[\cy:\tnum] \\
  \Gamma_3&=&\Gamma_1[\cz:\tnum\times\tnum] \\
\end{array}
$

\section{Type Checker}

First, we extend the definition of a type since \Lang has a new sort of a type.

\begin{verbatim}
sealed trait Type
...
case class IdT(t: String) extends Type
\end{verbatim}

\code{IdT($t$)} represents $t$.

In \Lang, type environments store type definitions and the types of variables.
Thus, they cannot be represented by simple maps any longer. Now, they are
represented by \code{TEnv} instances, each of which contains two maps.

\begin{verbatim}
case class TEnv(
  vars: Map[String, Type],
  tbinds: Map[String, Map[String, Type]]
) {
  def add(x: String, t: Type): TEnv =
    TEnv(vars + (x -> t), tbinds)

  def add(x: String, m: Map[String, Type]): TEnv =
    TEnv(vars, tbinds + (x -> m))

  def contains(x: String): Boolean =
    tbinds.contains(x)
}
\end{verbatim}

\code{TEnv} has two fields: \code{vars} and \code{tbinds}.
The field \code{vars}, which is a map from strings to \Lang types, contains
the types of variables. The field \code{tbinds}, which is a map from strings to
maps, contains type definitions. Each map in \code{tbinds} maps strings, which
are the names of variants, to \Lang types, which are the parameter types
of variants. For example, \code{tbinds} containing the $\code{Fruit}$ type is
as follows:

\begin{verbatim}
Map("Fruit" -> Map("Apple" -> NumT, "Banana" -> PairT(NumT, NumT)))
\end{verbatim}

For the ease of adding type definitions and variables to type environments, the \code{TEnv}
class has two methods named \code{add}. Adding that the type of a variable $\cx$ is
$\tnum$ to \code{env} can be written like below.

\begin{verbatim}
env.add("x", NumT)
\end{verbatim}

Adding the $\code{Fruit}$ type to \code{env} can be written like below.

\begin{verbatim}
env.add("Fruit", Map("Apple" -> NumT, "Banana" -> PairT(NumT, NumT)))
\end{verbatim}

The \code{contains} method of the \code{TEnv} class checks whether a particular
type identifier is a bound type identifier. For instance,
the following code checks whether $\code{Fruit}$ is bound:

\begin{verbatim}
env.contains("Fruit")
\end{verbatim}

Now let us define a function that checks the well-formedness of a type.
The following \code{wfType} function checks whether a given type is well-formed
under a given type environment:

\begin{verbatim}
def wfType(t: Type, env: TEnv): Unit = t match {
  case NumT =>
  case ArrowT(p, r) =>
    wfType(p, env)
    wfType(r, env)
  case IdT(t) =>
    if (!env.contains(t))
      throw new Exception
}
\end{verbatim}

If the type is ill-formed under the type environment, the function throws
an exception.

Now, we add the \code{TypeDef} and \code{Match} cases to the \code{typeCheck} function.

\begin{verbatim}
 case TypeDef(t, v1, vt1, v2, vt2, b) =>
  val nenv = env
    .add(t, Map(v1 -> vt1, v2 -> vt2))
    .add(v1, ArrowT(vt1, IdT(t)))
    .add(v2, ArrowT(vt2, IdT(t)))
  wfType(vt1, nenv)
  wfType(vt2, nenv)
  typeCheck(b, nenv)
\end{verbatim}

First, the function adds the type definition and the constructors to the type
environment. Then, it checks the well-formedness of the parameter types of the
variants under the new type environment. If both are well-formed,
it type-checks the body expression. The type of the body is the type of the
whole type-defining expression.

\begin{verbatim}
case Match(e, v1, x1, e1, v2, x2, e2) =>
  val IdT(t) = typeCheck(e, env)
  val tdef = env.tbinds(t)
  val t1 = typeCheck(e1, env.add(x1, tdef(v1)))
  val t2 = typeCheck(e2, env.add(x2, tdef(v2)))
  mustSame(t1, t2)
  t1
\end{verbatim}

First, the function type-checks the target expression. The type must be
a type identifier. The definition of the type should be found in the type
environment. The definition gives the parameter type of each variant. The
function type-checks \code{e1} and \code{e2} under the type environments with the
type of \code{x1} and with the type of \code{x2}, respectively.
The types must be the same, and if it is the case, the common
type is the type of the whole pattern mathcing expression.

\begin{verbatim}
case Fun(x, t, b) =>
  wfType(t, env)
  ArrowT(t, typeCheck(b, env.add(x, t)))
\end{verbatim}

As discussed already, the \code{Fun} case needs a revision.
The well-formedness of the parameter type annotation needs to be checked.

\begin{verbatim}
case Id(x) => env.vars(x)
\end{verbatim}

The \code{Id} case also has a small change. Due to the new definition of a type
environment, a way to find the type of a variable is a bit different.

\section{Type Soundness of \Lang}

Actually, \Lang is not type sound. If an expression defines multiple types of
the same name, the expression can be well-typed and incur a run-time error at the
same time. Consider the following expression:

\vspace{0.5em}
$\begin{array}{@{}l}
  \etdef{\code{Fruit}}{\code{Apple}}{\tnum}{\code{Banana}}{(\tnum\times\tnum)}{( \\
  \ \ \ \ (\efunt{\cf}{\tarrow{\code{Fruit}}{\tnum}}{ \\
  \ \ \ \ \ \ \ \ \etdef{\code{Fruit}}{\code{Cherry}}{\tnum}{\code{Durian}}{(\tnum\times\tnum)}{ \\
  \ \ \ \ \ \ \ \ \cf\ (\code{Cherry}\ 1)} \\
  \ \ \ \ })\ (\efunt{\cx}{\code{Fruit}}{\ematch{\cx}{ \\
  \ \ \ \ \ \ \ \ \code{Apple}}{\cy}{\cy}{ \\
  \ \ \ \ \ \ \ \ \code{Banana}}{\cz}{\cz\textsf{.1}}} \\
  \ \ \ \ ) \\
  )}
\end{array}$
\vspace{0.5em}

The expression defines \code{Fruit} twice in a nested manner. The outer
\code{Fruit} has \code{Apple} and \code{Banana} as variants, and the inner
\code{Fruit} has \code{Cherry} and \code{Durian} as variants. The expression
applies a function of $\tarrow{(\tarrow{\code{Fruit}}{\tnum})}{\tnum}$ to a value
of ${\tarrow{\code{Fruit}}{\tnum}}$, so it is well-typed. However, the
expression causes a run-time error. The function of
$\tarrow{(\tarrow{\code{Fruit}}{\tnum})}{\tnum}$ applies a given function to a
value of the \code{Cherry} variant because \code{Fruit} has \code{Cherry} and
\code{Durian} inside the function. However, the inner definition of
\code{Fruit} is unavailable outside the function, so the argument given to the
function is a function that expects a value of \code{Apple} or \code{Banana}.
Thus, at run time, the pattern matching fails and incurs a run-time error.

The reason of broken type soundness is that the language allows multiple different
types of the same name, while its type checking depends solely on the names of
types to distinguish different types. Two different types may incorrectly
considered as the same type when they have the same name.

There are multiple ways to fix the problem. The first solution is to prohibit
local type definitions. Every type definition should be at top level, just like
functions in \textsf{F1VAE}. Then, types cannot be nested, and every type must
have a different name from each other. Since there cannot be different types of
the same name, the problem is resolved.

The second solution is to prevent interaction between different types of the
same name. It can be achieved by changing Rule~\textsc{Typ-TypeDef} like below.

\typerule{Typ-TypeDef'}{
  \begin{tabular}{@{\hskip0pt}l@{\hskip-10pt}l}
    If \\
    & $t$ is not in the domain of $\Gamma$, \\
    & $\Gamma'$ denotes
    $\Gamma[t=x_1@\tau_1+x_2@\tau_2,x_1:\tarrow{\tau_1}{t},x_2:\tarrow{\tau_2}{t}]$, \\
    & \wftn{\Gamma'}{\tau_1}, \\
    & \wftn{\Gamma'}{\tau_2}, \\
    & \typeofn{\Gamma'}{e}{\tau}, and \\
    & \wftn{\Gamma}{\tau}, \\
    then \\
    & \typeofdn{\etdef{t}{x_1}{\tau_1}{x_2}{\tau_2}{e}}{\tau}.
  \end{tabular}
}

\vspace{-1em}

\[
  \inferrule
  {
    t\not\in\dom{\Gamma} \\
    \Gamma'=\Gamma[t=x_1@\tau_1+x_2@\tau_2,x_1:\tarrow{\tau_1}{t},x_2:\tarrow{\tau_2}{t}] \\
    \wft{\Gamma'}{\tau_1} \\
    \wft{\Gamma'}{\tau_2} \\
    \typeof{\Gamma'}{e}{\tau} \\
    \wft{\Gamma}{\tau} }
  { \typeofd{\etdef{t}{x_1}{\tau_1}{x_2}{\tau_2}{e}}{\tau} }
  \quad\textsc{[Typ-TypeDef']}
\]

The rule has two new premises:
\begin{itemize}
  \item $t\not\in\dom{\Gamma}$
  \item $\wft{\Gamma}{\tau}$
\end{itemize}
The first one prevents nested types from having the same name. The second one
prevents each locally defined type from escaping its scope. In this way, we can
effectively solve the issue. A program still can have different types of the
same name, but different types of the same name cannot meet each other,
i.e. they cannot be used in the same place.

The third solution is to rename types before type checking in order to remove
any duplication in type names. Since the bound-bind relation between identifiers
can be easily determined with simple syntactic checking, it is possible to
rename types without changing the semantics of a given program. For instance,
the above example becomes the following expression after renaming:

\vspace{0.5em}
$\begin{array}{@{}l}
  \etdef{\code{Fruit}}{\code{Apple}}{\tnum}{\code{Banana}}{(\tnum\times\tnum)}{( \\
  \ \ \ \ (\efunt{\cf}{\tarrow{\code{Fruit}}{\tnum}}{ \\
  \ \ \ \ \ \ \ \ \etdef{\code{Fruit1}}{\code{Cherry}}{\tnum}{\code{Durian}}{(\tnum\times\tnum)}{ \\
  \ \ \ \ \ \ \ \ \cf\ (\code{Cherry}\ 1)} \\
  \ \ \ \ })\ (\efunt{\cx}{\code{Fruit}}{\ematch{\cx}{ \\
  \ \ \ \ \ \ \ \ \code{Apple}}{\cy}{\cy}{ \\
  \ \ \ \ \ \ \ \ \code{Banana}}{\cz}{\cz\textsf{.1}}} \\
  \ \ \ \ ) \\
  )}
\end{array}$
\vspace{0.5em}

This expression is certainly ill-typed because $\code{Cherry}\ 1$ is a value of
\code{Fruit1}, while $\cf$, which is applied to $\code{Cherry}\ 1$, is a
function that expects a value of \code{Fruit}.
This solution is desirable in the sense that it does not need any change in the
language. It only requires one additional step of transformation, which can be
easily implemented. On the other hand, the first solution changes the language
at syntax level, and the second solution changes the static semantics of the
language.

\section{Exercises}

\begin{exercise}
\labex{algebraic-data-types-eval}

What does each of the following expressions evaluate to?
If it is a run-time error, describe where the error occurs.

\begin{enumerate}
  \item
    $\small\etdef{\code{Fruit}}{\code{Apple}}{\tnum}{\code{Banana}}{\tnum}{\\
     \etdef{\code{Animal}}{\code{Apple}}{(\tarrow{\tnum}{\tnum})}{\code{Banana}}{(\tarrow{\tnum}{\tnum})}{\\
     (\efunt{\cx}{\code{Fruit}}{
       \ematch{\cx}{\code{Apple}}{\cy}{\cy}{\code{Banana}}{\cy}{\cy}
     })\ (\code{Banana}\ 10)
     }}
    $

  \item
    $\small\etdef{\code{Fruit}}{\code{Apple}}{\tnum}{\code{Banana}}{\tnum}{\\
     \etdef{\code{Fruit}}{\code{Apple}}{(\tarrow{\tnum}{\tnum})}{\code{Banana}}{(\tarrow{\tnum}{\tnum})}{\\
     (\efunt{\cx}{\code{Fruit}}{
       \ematch{\cx}{\code{Apple}}{\cy}{\cy}{\code{Banana}}{\cy}{\cy}
     })\ (\code{Banana}\ 10)
     }}
    $
\end{enumerate}

\end{exercise}

\begin{exercise}
\labex{algebraic-data-types-norec}

Suppose that Rule~\textsc{Typ-TypeDef} has been changed as the following:
\[
  \inferrule
  {
    \Gamma'=\Gamma[t=x_1@\tau_1+x_2@\tau_2,x_1:\tarrow{\tau_1}{t},x_2:\tarrow{\tau_2}{t}] \\
    \wft{\Gamma}{\tau_1} \\
    \wft{\Gamma}{\tau_2} \\
    \typeof{\Gamma'}{e}{\tau} }
  { \typeofd{\etdef{t}{x_1}{\tau_1}{x_2}{\tau_2}{e}}{\tau} }
\]
It checks the well-formedness of $\tau_1$ and $\tau_2$ under $\Gamma$, instead of
$\Gamma'$.

\begin{enumerate}
  \item
    Explain why the new typing rule is problematic.
  \item
    Write a \Lang expression that
    \begin{itemize}
      \item is accepted by the original type system, but
      \item is rejected by the new type system.
    \end{itemize}
\end{enumerate}

\end{exercise}

\begin{exercise}
\labex{algebraic-data-types-nowf}

Suppose that Rule~\textsc{Typ-TypeDef} has been changed as the following:
\[
  \inferrule
  {
    \Gamma'=\Gamma[t=x_1@\tau_1+x_2@\tau_2,x_1:\tarrow{\tau_1}{t},x_2:\tarrow{\tau_2}{t}] \\
    \wft{\Gamma'}{\tau_2} \\
    \typeof{\Gamma'}{e}{\tau} }
  { \typeofd{\etdef{t}{x_1}{\tau_1}{x_2}{\tau_2}{e}}{\tau} }
\]
It lacks the well-formedness check of $\tau_1$.
Write a \Lang expression that
\begin{itemize}
  \item is rejected by the original type system, but
  \item is accepted by the new type system, and
  \item results in a run-time error when evaluated.
\end{itemize}

\end{exercise}

\begin{exercise}
\labex{algebraic-data-types-nested}

Consider the following expression:

$
    \begin{array}{@{}l}
      \etdef{\code{Fruit}}{\code{Apple}}{\tnum}{\code{Banana}}{\tnum}{\\
      (\etdef{\code{Color}}{\code{Apple}}{\tnum}{\code{Banana}}{\tnum}{\\
      \efunt{\cx}{\code{Fruit}}{
        \ematch{\cx}
        {\\ \ \ \ \ \code{Apple}}{\cy}{\code{Apple}\ (\cy+1)}
        {\\ \ \ \ \ \code{Banana}}{\cy}{\code{Banana}\ (\cy+1)}
      }\\
      )\ (\code{Banana}\ 10)
      }}
    \end{array}
$

What is the result of type-checking the expression?
If the result is \code{Fruit}, explain why in detail.
Otherwise, revise the typing rules to make it \code{Fruit}.

\end{exercise}

\begin{exercise}
\labex{algebraic-data-types-mult}

This exercise extends \Lang to have
  \begin{itemize}
      \item functions with multiple parameters
      \item types with more than two variants
      \item variants with multiple parameters
  \end{itemize}
\[
  \begin{array}{@{}r@{~~}r@{~~}l}
    e & ::= & \cdots \\
    &|& \efun{(x{:}\tau,\cdots,x{:}\tau)}{e} \\
    &|& e(e,\cdots,e) \\
    &|& \textsf{type}\ t=x@(\tau,\cdots,\tau)+\cdots+x@(\tau,\cdots,\tau)\ \textsf{in}\ e \\
    &|& e\ \textsf{match}\ x(x,\cdots,x)\rightarrow e,\cdots,x(x,\cdots,x)\rightarrow e \\
    \tau & ::= & \cdots \\
    &|& \tarrow{(\tau,\cdots,\tau)}{\tau}
  \end{array}
\]

\begin{enumerate}
  \item Write the well-formedness rule for the added type.
  \item Write the typing rules for the added expressions.
  \item Draw the type derivation of the following expression:

    $
    \begin{array}{@{}l}
      \textsf{type}\ \code{Fruit}=
      \code{Apple}@()+
      \code{Banana}@(\tarrow{(\code{Fruit})}{\tnum},\code{Fruit})+
      \code{Cherry}@(\tnum)\ \textsf{in} \\
      \code{Apple}()\ \textsf{match} \\
      \ \ \ \ \code{Apple}()\rightarrow 42,\\
      \ \ \ \ \code{Banana}(\cf,\cx)\rightarrow \cf(\cx),\\
      \ \ \ \ \code{Cherry}(\cx)\rightarrow \cx
    \end{array}
    $
\end{enumerate}

\end{exercise}

\begin{exercise}
\labex{algebraic-data-types-nonterminate}

It is known that every \plang expression terminates. On the other
hand, \textsf{TRFAE} expressions can run forever by using \textsf{def}. Then, how
about \Lang? Is it possible to write a non-terminating \Lang
expression? The answer is yes, and this exercise asks you to construct such an
expression.

For simplicity, we extend \Lang to allow a type of a single variant:

$e\ ::=\ \cdots\ |\ \textsf{type}\ t=x@\tau;e\ |\ e\ \textsf{match}\ x(x)\rightarrow e$

The following \Lang expression passes type checking and runs forever:

$\begin{array}{@{}l@{}}
  \textsf{type}\ \cX=\code{toX}@(\tarrow{\cX}{\cX}); \\
  \textsf{val}\ \code{fromX}:\tarrow{\cX}{(\tarrow{\cX}{\cX})}=???;\\
  \textsf{val}\ \cf:\cX=???;\\
  ???
\end{array}$

Complete the expression. Use the fact that the following
\textsf{FAE} expression runs forever:

$\ebind{\cf}{\efun{\cx}{(\eapp{\cx}{\cx})}}{\eapp{\cf}{\cf}}$

\end{exercise}
