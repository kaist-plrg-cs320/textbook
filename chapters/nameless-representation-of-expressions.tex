\setchapterpreamble[u]{\margintoc}
\chapter{Nameless Representation of Expressions}
\labch{nameless-representation-of-expressions}

This article introduces de Bruijn indices, which allow representing expressions
without giving names to variables. By using de Bruijn indices, expressions become
nameless, and problems arising from naming conflicts can be avoided.

\section{Motivation}

All the previous articles distinguish different variables by naming them. For
example, $\lambda \code{x}.\lambda \code{y}.\code{x}$ is a function that takes an
argument twice and returns the first argument. Since two arguments have different
names, one can easily conclude that the first argument is the result. The first
argument is named $\cx$, and the second argument is named $\cy$. Therefore,
$\cx$ in the function body denotes the first argument.

Naming variables is an intuitive and practically useful way to represent
variables. However, it becomes problematic in some cases like formalizing the
semantics of languages and implementing interpreters and compilers, which take
source code as input.

First, two variables may not be distinguished when their names are the same.
Environments can easily deal with variables of the same name well, but
substitution is often used instead of environments to define the semantics of
languages. For instance, defining the semantics of function applications with
substitutions is as follows: evaluating $(\lambda \code{x}.\code{x} + \code{x})\ 1$
is the same as evaluating $1+1$, which is obtained by substituting $\cx$ with
$1$ in the function body $\code{x}+\code{x}$. Since the main purpose of mentioning
substitutions is explaining the problem of naming variables, I will not formally
define substitutions. In fact, it is difficult to define the semantics correctly
with substitutions. Consider the expression $(\lambda \code{f}.\lambda \code{
x}.\code{f})\ \lambda \code{y}.\code{x}$. By applying the same principle, evaluating
the expression is the same as evaluating $\lambda \code{x}.\lambda \code{y}.\code{
x}$, which is obtained by substituting $\cf$ with $\lambda \code{y}.\code{x}$ in
$\lambda \code{x}.\code{f}$. Alas, it is wrong. $\cx$ in the original argument
$\lambda \code{y}.\code{x}$ is a free identifier, while $\cx$ in $\lambda \code{
x}.\lambda \code{y}.\code{x}$ is a binding occurrence. The meaning of $\cx$
before and after the substitution is completely different. This example shows
that the current semantics is incorrect, and the root cause of the problem is
that two different variables are named $\cx$.

Such naming conflicts can be found even in type systems. The “Algebraic Data
Types” and “Parametric Polymorphism” articles explained various sources of
unsoundness in TVFAE and TPFAE. One of the sources was that the type systems
allowed defining types of the same names, and it was resolved by revising the
typing rules to disallow such types. Since names are used to distinguish
different types, different types can be incorrectly considered as the same one
when their names are the same.

Second, names hinder us from checking the semantic equivalence of expressions.
For example, both $\lambda \code{x}.\code{x}$ and $\lambda \code{y}.\code{y}$ are
identity functions. However, a naïve syntactic check cannot prove the semantic
equivalence of them, i.e. that their behaviors are the same, because the first
expression names the parameter $\cx$, while the second expression names the
parameter $\cy$. The ability to check semantic equivalence is valuable in many
places. Consider optimization of expressions.

\[
\begin{array}{l}
\textsf{val}\ \code{f}=\lambda \code{x}.\code{x}; \\
\textsf{val}\ \code{g}=\lambda \code{y}.\code{y}; \\
(\code{f}\ 1)+(\code{g}\ 2) \\
\end{array}
\]

The above expression defined the functions $\cf$ and $\code{g}$ and, then,
evaluate $(\code{f}\ 1)+(\code{g}\ 2)$. $\cf$ and $\code{g}$ are semantically
equivalent, but the names of their parameter are different. If a compiler is
aware of their equivalence, it can reduce the size of the binary code by
modifying the expression like below:

\[
\begin{array}{l}
\textsf{val}\ \code{f}=\lambda \code{x}.\code{x}; \\
(\code{f}\ 1)+(\code{f}\ 2) \\
\end{array}
\]

In languages with parametric polymorphism, the names of type parameters make the
comparison of types difficult. Consider TPFAE. Both
$\forall\alpha.\alpha\rightarrow\alpha$ and $\forall\beta.\beta\rightarrow\beta$
are the types of a polymorphic identity function. Therefore, an expression of the
type $\forall\beta.\beta\rightarrow\beta$ should be able to appear where an
expression of the type $\forall\alpha.\alpha\rightarrow\alpha$ is required.
Sadly, a naïve syntactic comparison of types cannot achieve the goal.

The above two examples show the importance of comparing two expressions (or two
types). Naming variables (or type variables) is not a good way for this purpose.

Names are often problematic in programming languages. If different entities are
named equally, they may not be distinguished correctly. The semantic equivalence
of functions may not be proved because of different parameter names. Multiple
solutions have been proposed to resolve the issue. This article introduces a de
Bruijn index, which is one of those solutions. De Bruijn indices represent
variables with indices, not names. For simplicity, the article only deals with de
Bruijn indices for variables. In fact, de Bruijn indices can be used anywhere
names lead to a problem.

\section{De Bruijn Indices}

De Bruijn indices represent variables with indices, which are natural numbers.
The number of $\lambda$ between a bound occurrence and the corresponding binding
occurrence represents the binding occurrence. For instance,
$\lambda.\underline{0}$ is the nameless version of $\lambda \code{x}.\code{x}$.
$\lambda.\underline{0}$ is a function with one parameter. Its body is
$\underline{0}$, which differs from a natural number $0$. $\underline{0}$ denotes
a variable whose distance from its definition is zero. The distance means the
number of $\lambda$. Therefore, the parameter of $\lambda.\underline{0}$ is the
one that $\underline{0}$ is bound to. In a similar fashion,
$\lambda.\lambda.\underline{1}$ is the nameless version of $\lambda \code{
x}.\lambda\code{y}.\code{x}$. $\lambda.\lambda.\underline{1}$ is a function with
one parameter and the body expression $\lambda.\underline{1}$.
$\lambda.\underline{1}$ also is a function with one parameter. Its body is
$\underline{1}$, which is a variable whose distance from the definition is one.
Thus, the parameter of $\lambda.\underline{1}$ cannot be denoted by
$\underline{1}$. There is no $\lambda$ between the parameter and $\underline{1}$.
$\underline{1}$ denotes the parameter of $\lambda.\lambda.\underline{1}$ because
there are one $\lambda$ in between. The following shows various examples of de
Bruijn indices.

* $\lambda \code{x}.\code{x}\rightarrow\lambda.\underline{0}$
* $\lambda \code{x}.\lambda \code{y}.\code{
x}\rightarrow\lambda.\lambda.\underline{1}$
* $\lambda \code{x}.\lambda \code{y}.\code{
y}\rightarrow\lambda.\lambda.\underline{0}$
* $\lambda \code{x}.\lambda \code{y}.\code{x}+\code{
y}\rightarrow\lambda.\lambda.\underline{1}+\underline{0}$
* $\lambda \code{x}.\lambda \code{y}.\code{x}+\code{
y}+42\rightarrow\lambda.\lambda.\underline{1}+\underline{0}+42$
* $\lambda \code{x}.(\code{x}\ \lambda \code{y}.(\code{x}\ \code{
y}))\rightarrow\lambda.(\underline{0}\ \lambda.(\underline{1}\ \underline{0}))$
* $\lambda \code{x}.((\lambda \code{y}.\code{x})\ (\lambda \code{z}.\code{
x}))\rightarrow\lambda.((\lambda.\underline{1})\ (\lambda.\underline{1}))$

It is important to notice that different indices can denote the same variable,
and the same indices can denote different variables. Consider the second example
from the bottom. The first $\underline{0}$ in $\lambda.(\underline{0}\
\lambda.(\underline{1}\ \underline{0}))$ denotes $\cx$ of the original
expression. At the same time, $\underline{1}$ also denotes $\cx$ of the
original expression. On the other hand, the second $\underline{0}$ denotes
$\cy$ of the original expression. The distance from the definition depends on the
location of a variable. Since de Bruijn indices represent variables with the
distances, the indices of a single variable can vary among places.

Note that expressions should be treated as trees, not strings, to calculate the
distances. Consider the last example. There are two $\lambda$’s between the last
$\cx$ and its definition when the expression is written as a string. However,
when the abstract syntax tree representing the expression is considered, there is
only one $\lambda$ in between. Therefore, the index of the last $\cx$ is
$\underline{1}$, not $\underline{2}$. We usually write expressions as strings for
convenience, but they always have tree structures in fact.

De Bruijn indices successfully resolve the issues arising from names. Consider
the comparison of expressions. $\lambda \code{x}.\code{x}$ and $\lambda \code{
y}.\code{y}$ are semantically equivalent but syntactically different expressions.
Both become $\lambda.\underline{0}$ when de Bruijn indices are used. By the help
of de Bruijn indices, a simple syntactic check will find out that two expressions
are equal.

Let us define the procedure that transform expressions with names into nameless
expressions. It will help understanding de Bruijn indices. At the same time, the
procedure is practically valuable. Use of names is the best way to denote
variables for programmers. Therefore, expressions written by programmers have
names. On the other hand, programs like interpreters and compilers sometimes need
to use de Bruijn indices to represent variables. In such cases, the procedure is
a part of the interpreter/compiler implementation. This article focuses on the
transform procedure for FAE. It is easy to modify the procedure so that it can
work for another language.

Below is the definition of an expression with and without names. Strictly
speaking, two different metavariables should denote each kind of expressions. For
brevity, I abuse the notation, so $e$ is used for both sorts of expressions.

\[
\begin{array}{lrcl}
\text{Expression} & e & ::= & x \\
&&|& \lambda x.e \\
&&|& e\ e \\
&&|& n \\
&&|& e+e \\
\end{array}
\]

\[
\begin{array}{lrcl}
\text{Index} & i & \in & \mathbb{N} \\
\text{Expression} & e & ::= & \underline{i} \\
&&|& \lambda.e \\
&&|& e\ e \\
&&|& n \\
&&|& e+e \\
\end{array}
\]

The former defines expressions with names; the latter defines nameless
expressions.

In nameless expressions, natural numbers represent variables. Those numbers have
underlines and, therefore, cannot be confused with integers in FAE. Lambda
abstractions $\lambda.e$ lack the names of their parameters. Note that
$\lambda.e$ does have a single parameter. It is not a function with zero
parameters.

In addition, a context, which is a partial function from a name to a natural
number, takes an important role during the transformation. A context gives the
distance between a variable and its definition.

\[\chi\in\embox{Id}\hookrightarrow\mathbb{N}\]

Let $[e]\chi$ be a nameless expression representing $e$ under a context $\chi$.
The definition of $[e]\chi$ is as follows:

\[
\begin{array}{rcl}
\lbrack x\rbrack \chi &=& \underline{i}\ \ \text{if}\ \chi(x)=i \\
\lbrack \lambda x.e\rbrack \chi &=& \lambda.\lbrack e\rbrack {\chi'}\ \ \text{where}\
\chi'=(\uparrow\chi)\lbrack x\mapsto 0\rbrack  \\
\lbrack e_1\ e_2\rbrack \chi &=& \lbrack e_1\rbrack \chi\ \lbrack e_2\rbrack \chi \\
\lbrack n\rbrack \chi &=& n \\
\lbrack e_1+e_2\rbrack \chi &=& \lbrack e_1\rbrack \chi+\lbrack e_2\rbrack \chi \\
\end{array}
\]

$[x]\chi$ is the result of transforming $x$. A natural number represents a
variable, and the natural number can be found in $\chi$. Therefore, when
$\chi(x)$ is $i$, $x$ is transformed in to $\underline{i}$.

$[\lambda x.e]\chi$ is the result of transforming $\lambda x.e$ and should look
like $\lambda.e$. However, $e$ uses names and, thus, needs to be transformed.
$\chi$ is not the correct context for the transformation of $e$. First, it lacks
the information of $x$. If $x$ appears in $e$ without any function definitions,
there is no $\lambda$ between the use and the definition. The context must know
that the index of $x$ is 0. In addition, indices in $\chi$ need changes. Suppose
that $x'$ is in $\chi$ and its index is 0. If $x'$ occurs in $e$, its index is
not 0 anymore. Since $e$ is the body of $\lambda x.e$, there is one $\lambda$
between $x'$ and is definition. During the transformation of $e$, the index of
$x'$ is 1, not 0. Similarly, if there is a variable whose index of 1 in $\chi$,
its index must be 2 during the transformation of $e$. In conclusion, every index
in $\chi$ has to increase by one. $\uparrow\chi$ denotes the context same as
$\chi$ but whose indices are one larger. The context used during the
transformation of $e$ is $(\uparrow\chi)[x\mapsto0]$. $[\lambda x.e]\chi$ is
$\lambda.[e]{\chi'}$ where $\chi'$ is $(\uparrow\chi)[x\mapsto0]$.

The remaining cases are straightforward. The transformations of $e_1\ e_2$ and
$e_1+e_2$ are recursively defined. Since $n$ does not contain variables, $n$
itself is the result.

Below shows how $\lambda \code{x}.\lambda \code{y}.\code{x}+\code{y}$ is transformed
by the procedure. In the beginning, the context is empty because there is no
variable yet.

\[
\begin{array}{cl}
& [\lambda \code{x}.\lambda \code{y}.\code{x}+\code{y}]\emptyset \\
= & \lambda.[\lambda \code{y}.\code{x}+\code{y}][\code{x}\mapsto 0] \\
= & \lambda.\lambda.[\code{x}+\code{y}][\code{x}\mapsto 1,\code{y}\mapsto 0] \\
= & \lambda.\lambda.[\code{x}][\code{x}\mapsto 1,\code{y}\mapsto 0]+[\code{y}][\code{
x}\mapsto 1,\code{y}\mapsto 0] \\
= & \lambda.\lambda.\underline{1}+[\code{y}][\code{x}\mapsto 1,\code{y}\mapsto 0] \\
= & \lambda.\lambda.\underline{1}+\underline{0} \\
\end{array}
\]

Now, let us implement the procedure in Scala.

\begin{verbatim}
sealed trait Expr
case class Id(x: String) extends Expr
case class Fun(x: String, e: Expr) extends Expr
case class App(f: Expr, a: Expr) extends Expr
case class Num(n: Int) extends Expr
case class Add(l: Expr, r: Expr) extends Expr
\end{verbatim}

The above defines expressions with names.

\begin{verbatim}
object Nameless {
  sealed trait Expr
  case class Id(i: Int) extends Expr
  case class Fun(e: Expr) extends Expr
  case class App(f: Expr, a: Expr) extends Expr
  case class Num(n: Int) extends Expr
  case class Add(l: Expr, r: Expr) extends Expr
}
\end{verbatim}

Nameless expressions are defined in the \code{Nameless} singleton object.
\code{Id(i)} is a variable whose index is \code{i}; \code{Fun(e)} is a function
with one parameter and the body expression \code{e}.

\begin{verbatim}
type Ctx = Map[String, Int]
\end{verbatim}

\code{Ctx}, the type of a context, is a map from a string to an integer.

Let the \code{transform} function recursively transform an expression with names
into a nameless expression.

\begin{verbatim}
def transform(e: Expr, ctx: Ctx): Nameless.Expr = e match {
  case Id(x) => Nameless.Id(ctx(x))
  case Fun(x, e) =>
    Nameless.Fun(transform(e, ctx.map{ case (x, i) => x -> (i + 1) } + (x -> 0)))
  case App(f, a) =>
    Nameless.App(transform(f, ctx), transform(a, ctx))
  case Num(n) => Nameless.Num(n)
  case Add(l, r) =>
    Nameless.Add(transform(l, ctx), transform(r, ctx))
}
\end{verbatim}

The function exactly looks like its mathematical definition, so it is easy to
understand the code.

The following program transforms $\lambda \code{x}.\lambda \code{y}.\code{x}+\code{
y}$ with \code{transform}:

\begin{verbatim}
// lambda x.lambda y.x+y
transform(Fun("x", Fun("y", Add(Id("x"), Id("y")))), Map())
// Fun(Fun(Add(Id(1),Id(0))))
// lambda.lambda._1+_0
\end{verbatim}

Lists can replace maps in the implementation. A context is a list of names, and
the index of a name is the location of the name in the list. Lists simplify the
implementation. When a name is added to a context, its index is always zero. It
means that the name is the head of the list. Adding a name is the same as making
the head of the list be the name. Increasing every index by one is the same as
moving each name backward by one slot. Therefore, if a context is a list,
prepending a new name in front of the list will do everything we need. For
example, consider a context containing $\cx$ and $\cy$. Let the indices of
$\cx$ and $\cy$ respectively be 0 and 1. The context is represented by the
list $[\code{x},\code{y}]$. It is enough to prepend $\cz$ to the list to add
$\cz$ to the context. The resulting list is $[\code{z},\code{x},\code{y}]$: $\cz$ at
index 0, $\cx$ at index 1, and $\cy$ at index 2. Since $\cz$ is the new
one, its index should be 0. At the same time, the indices of $\cx$ and $\cy$
should be greater by one than before. The new list does represent the new context
well.

\begin{verbatim}
type Ctx = List[String]
\end{verbatim}

Now, \code{Ctx} is a list of strings.

\begin{verbatim}
def transform(e: Expr, ctx: Ctx): Nameless.Expr = e match {
  case Id(x) => Nameless.Id(ctx.indexOf(x))
  case Fun(x, e) => Nameless.Fun(transform(e, x :: ctx))
  case App(f, a) =>
    Nameless.App(transform(f, ctx), transform(a, ctx))
  case Num(n) => Nameless.Num(n)
  case Add(l, r) =>
    Nameless.Add(transform(l, ctx), transform(r, ctx))
}
\end{verbatim}

The \code{Id} case needs to calculate the location of a given variable in a given
context. It is enough to use \code{indexOf}. In the \code{Fun} case, \code{x ::
ctx} is everything we need to add \code{x} to \code{ctx}.

The following program transforms $\lambda \code{x}.\lambda \code{y}.\code{x}+\code{
y}$ with \code{transform}:

\begin{verbatim}
// lambda x.lambda y.x+y
transform(Fun("x", Fun("y", Add(Id("x"), Id("y")))), Nil)
// Fun(Fun(Add(Id(1),Id(0))))
// lambda.lambda._1+_0
\end{verbatim}

\section{Evaluation of Nameless Expressions}

Evaluation of nameless expressions is similar to evaluation of expressions with
names. The definitions of values and environments are as follows:

\[
\begin{array}{rrcl}
\text{Value} & v & ::= & n \\
&&|& \langle\lambda.e,\sigma\rangle \\
\text{Environment} & \sigma & \in & \mathbb{N}\hookrightarrow\text{Value}
\end{array}
\]

As lambda abstractions lack parameter names, closures also lack parameter names.
Environments are partial functions from indices, which are natural numbers, to
values.

\[
\inferrule
{ i\in\dom{\sigma} }
{ \sigma\vdash\underline{i}\Rightarrow\sigma(i) }
\]

The value of a variable can be found in a given environment.

\[
\sigma\vdash\lambda.e\rightarrow\langle\lambda.e,\sigma\rangle
\]

A lambda abstraction evaluates to a closure without evaluating anything.

\[
\inferrule
{
  \sigma\vdash e_1\Rightarrow\langle\lambda.e,\sigma'\rangle \\
  \sigma\vdash e_2\Rightarrow v_2 \\
  (\uparrow\sigma')[0\mapsto v_2]\vdash e\Rightarrow v
}
{ \sigma\vdash e_1\ e_2\Rightarrow v }
\]

Evaluation of $e_1\ e_2$ evaluates both $e_1$ and $e_2$. Then, the body of the
closure is evaluated under the environment captured by the closure with the value
of the argument. If the parameter is used in the body, there is no $\lambda$
between the use and the definition. Its index is 0. Therefore, the value of the
argument has the index 0 in the new environment. In addition, every index in the
environment of the closure needs a change. Let a value $v$ correspond to the
index 0. The value is not the value of the argument, so it cannot correspond to
the index 0 anymore. As $\lambda$ from the closure exists between the use and the
definition, the index should increase by one. By the same principle, every index
in the environment increases by one. Let $\uparrow\sigma'$ denote the context
same as $\sigma'$ but whose indices are one larger. Then, the body of the closure
is evaluated under $(\uparrow\sigma')[0\mapsto v_2]$.

The rules for integers and sums are omitted because they are the same as those of
FAE.

The following proof tree proves that the reulst of
$(\lambda.\lambda.\underline{1}+\underline{0})\ 2\ 3$ is $5$.

% \[
% \inferrule
% {
%   \inferrule
%   {

% \emptyset\vdash\lambda.\lambda.\underline{1}+\underline{0}\Rightarrow\langle\lambda.\lambda.\underline{1}+\underline{0},\emptyset\rangle\\
%     \emptyset\vdash2\Rightarrow2\\

% [0\mapsto2]\vdash\lambda.\underline{1}+\underline{0}\Rightarrow\langle\lambda.\underline{1}+\underline{0},[0\mapsto2]\rangle
%   }
%   { \emptyset\vdash(\lambda.\lambda.\underline{1}+\underline{0})\
% 2\Rightarrow\langle\lambda.\underline{1}+\underline{0},[0\mapsto2]\rangle }
%   \\
%   \emptyset\vdash3\Rightarrow3 \\
%   \inferrule
%   {
%     \inferrule
%     { 1\in\dom{\sigma} }
%     { \sigma\vdash\underline{1}\Rightarrow2 }
%     \\
%     \inferrule
%     { 0\in\dom{\sigma} }
%     { \sigma\vdash\underline{0}\Rightarrow3 }
%   }
%   { \sigma\vdash\underline{1}+\underline{0}\Rightarrow5 }
% }
% { \emptyset\vdash(\lambda.\lambda.\underline{1}+\underline{0})\ 2\ 3\Rightarrow5
% }
% \]

% \[\sigma=[0\mapsto3,1\mapsto2]\]

Let us implement an interpreter of nameless expressions in Scala. Expressions has
been defined already. Below is the definitions of values and environments.

\begin{verbatim}
type Env = List[Value]

sealed trait Value
case class NumV(n: Int) extends Value
case class CloV(e: Expr, env: Env) extends Value
\end{verbatim}

An environment is a list of values. As shown by the implementation of
\code{transform}, lists are simpler than maps from integers to values.

\begin{verbatim}
def interp(e: Expr, env: Env): Value = e match {
  case Id(i) => env(i)
  case Fun(e) => CloV(e, env)
  case App(f, a) =>
    val CloV(b, fenv) = interp(f, env)
    interp(b, interp(a, env) :: fenv)
  case Num(n) => NumV(n)
  case Add(l, r) =>
    val NumV(n) = interp(l, env)
    val NumV(m) = interp(r, env)
    NumV(n + m)
}
\end{verbatim}

The \code{App} case is the only interesting case. The others are the same as
before. Since a closure lacks its parameter name and an environment does not need
the name, it is enough to prepend the value of the argument in front of the list.

The following program evaluates $(\lambda.\lambda.\underline{1}+\underline{0})\
2\ 3$ with \code{interp}. The result is $5$.

\begin{verbatim}
// (lambda.lambda._1+_0) 2 3
interp(
  App(
    App(
      Fun(Fun(Add(Id(1), Id(0)))),
      Num(2)
    ),
    Num(3)
  ),
  Nil
)
// 5
\end{verbatim}

Let $e$ be an expression with names. The result of evaluating $e$ is the same as
evaluating $e'$ where $e'$ is a nameless expression obtained by transforming $e$.
Mathematically, $\forall e,v.(\emptyset\vdash e\Rightarrow
v)\leftrightarrow(\emptyset\vdash[e]\emptyset\Rightarrow v)$. (Assume that the
equality of closures is defined properly.) In Scala implementation, given
\code{e}, which represents variables with names, \code{interp(e, Map())} and
\code{interp(transform(e, Nil), Nil)} result in the same value.

The interpreter of this article is an interpreter of FAE and does not use CPS.
Those who understand the implementation will be able to implement an interpreter
of another language or an interpreter with CPS by using de Bruijn indices.
