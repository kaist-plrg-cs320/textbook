\setchapterpreamble[u]{\margintoc}
\chapter{Nameless Representation of Expressions}
\labch{nameless-representation-of-expressions}

\renewcommand{\lang}{\textsf{FAE}\xspace}

In previous chapters, languages distinguish different variables by naming them
with different identifiers. For
example, $\lambda \cx.\lambda \cy.\cx$ is a function that takes an
argument twice and returns the first argument. Since two parameters have different
names, one can easily conclude that the first argument is the result. The first
parameter is named $\cx$, and the second parameter is named $\cy$. Therefore,
$\cx$ in the function body denotes the first parameter.

Naming variables is an intuitive and practically useful way to represent
variables. However, it becomes problematic in some cases like formalizing the
semantics of languages and implementing interpreters and compilers, which take
source code as input.

First, two variables may not be distinguished when their names are the same.
Environments can easily deal with variables of the same name well, but
substitution is often used instead of environments to define the semantics of
languages. For instance, defining the semantics of function applications with
substitutions is as follows: evaluating $(\lambda \cx.\cx + \cx)\ 1$
is the same as evaluating $1+1$, which is obtained by substituting $\cx$ with
$1$ in the function body $\cx+\cx$.\sidenote{Since the main purpose of mentioning
substitutions is explaining the problem of naming, we do not formally
define substitutions. To find more about substitution, see \refex{first-class-functions-subst}.}
In fact, it is difficult to define the semantics correctly
with substitutions. Consider the expression $(\lambda \cf.\lambda \cx.\cf)\ \lambda \cy.\cx$.
By applying the same principle, evaluating
the expression is the same as evaluating $\lambda \cx.\lambda \cy.\cx$,
which is obtained by substituting $\cf$ with $\lambda \cy.\cx$ in
$\lambda \cx.\cf$. Alas, it is wrong. $\cx$ in the original argument
$\lambda \cy.\cx$ is a free identifier, while $\cx$ in $\lambda \cx.\lambda \cy.\cx$
is a bound occurrence. The meaning of $\cx$
before and after the substitution is completely different. This example shows
that the current semantics is incorrect, and the root cause of the problem is
two different variables of the same name $\cx$.

% Such naming conflicts can be found even in type systems. The “Algebraic Data
% Types” and “Parametric Polymorphism” articles explained various sources of
% unsoundness in TVFAE and TPFAE. One of the sources was that the type systems
% allowed defining types of the same names, and it was resolved by revising the
% typing rules to disallow such types. Since names are used to distinguish
% different types, different types can be incorrectly considered as the same one
% when their names are the same.

Second, names hinder us from checking the semantic equivalence of expressions.
For example, both $\lambda \cx.\cx$ and $\lambda \cy.\cy$ are
identity functions. However, a naïve syntactic check cannot prove the semantic
equivalence of them, i.e. that their behaviors are the same, because the first
expression names the parameter $\cx$, while the second expression names the
parameter $\cy$. The ability to check semantic equivalence is valuable in many
places. Consider optimization of expressions.

\[
\begin{array}{l}
\textsf{val}\ \cf=\lambda \cx.\cx; \\
\textsf{val}\ \code{g}=\lambda \cy.\cy; \\
(\cf\ 1)+(\code{g}\ 2) \\
\end{array}
\]

The above expression defined the functions $\cf$ and $\code{g}$ and, then,
evaluate $(\cf\ 1)+(\code{g}\ 2)$. $\cf$ and $\code{g}$ are semantically
equivalent, but the names of their parameters are different. If a compiler is
aware of their equivalence, it can reduce the size of the program by
modifying the expression like below:

\[
\begin{array}{l}
\textsf{val}\ \cf=\lambda \cx.\cx; \\
(\cf\ 1)+(\cf\ 2) \\
\end{array}
\]

% In languages with parametric polymorphism, the names of type parameters make the
% comparison of types difficult. Consider TPFAE. Both
% $\forall\alpha.\alpha\rightarrow\alpha$ and $\forall\beta.\beta\rightarrow\beta$
% are the types of a polymorphic identity function. Therefore, an expression of the
% type $\forall\beta.\beta\rightarrow\beta$ should be able to appear where an
% expression of the type $\forall\alpha.\alpha\rightarrow\alpha$ is required.
% Sadly, a naïve syntactic comparison of types cannot achieve the goal.

As the example shows, comparing the semantic equivalence of expressions is an
important problem, but naming variables is not a good way for this purpose.

For these reasons, names are often problematic in programming languages.
Multiple solutions have been proposed to resolve the issue. This chapter introduces
de Bruijn indices, which are one of those solutions. \textit{De Bruijn indices}
\index{de Bruijn index} represent variables with indices, not names.
This chapter shows how de Bruijn indices can be used in \lang. Note that \lang
is just one possible use case of de Bruijn indices.
De Bruijn indices can be used anywhere names lead to a problem.

\section{De Bruijn Indices}

De Bruijn indices represent variables with indices, which are natural numbers.
The number of $\lambda$ between a bound occurrence and the corresponding binding
occurrence represents the binding occurrence. For instance,
$\lambda.\underline{0}$ is the nameless version of $\lambda \cx.\cx$.
$\lambda.\underline{0}$ is a function with one parameter. Its body is
$\underline{0}$, which differs from a natural number $0$. $\underline{0}$ denotes
a variable whose distance from its definition is zero. The distance means the
number of $\lambda$. Therefore, the parameter of $\lambda.\underline{0}$ is the
one that $\underline{0}$ is bound to. In a similar fashion,
$\lambda.\lambda.\underline{1}$ is the nameless version of $\lambda \cx
.\lambda\cy.\cx$. $\lambda.\lambda.\underline{1}$ is a function with
one parameter and the body expression $\lambda.\underline{1}$.
$\lambda.\underline{1}$ also is a function with one parameter. Its body is
$\underline{1}$, which is a variable whose distance from the definition is one.
Thus, the parameter of $\lambda.\underline{1}$ cannot be denoted by
$\underline{1}$. There is no $\lambda$ between the parameter and $\underline{1}$.
$\underline{1}$ denotes the parameter of $\lambda.\lambda.\underline{1}$ because
there is one $\lambda$ in between. The following table shows various examples of de
Bruijn indices.

\[
\begin{array}{c|c}
\text{With names} & \text{Without names} \\ \hline
\lambda \cx.\cx & \lambda.\underline{0} \\
\lambda \cx.\lambda \cy.\cx  & \lambda.\lambda.\underline{1} \\
\lambda \cx.\lambda \cy.\cy & \lambda.\lambda.\underline{0} \\
\lambda \cx.\lambda \cy.\cx+\cy & \lambda.\lambda.\underline{1}+\underline{0} \\
\lambda \cx.\lambda \cy.\cx+\cy+42 & \lambda.\lambda.\underline{1}+\underline{0}+42 \\
\lambda \cx.(\cx\ \lambda \cy.(\cx\ \cy)) & \lambda.(\underline{0}\ \lambda.(\underline{1}\ \underline{0})) \\
\lambda \cx.((\lambda \cy.\cx)\ (\lambda \cz.\cx)) & \lambda.((\lambda.\underline{1})\ (\lambda.\underline{1}))
\end{array}
\]

It is important to notice that different indices can denote the same variable,
and the same indices can denote different variables. Consider the second example
from the bottom. The first $\underline{0}$ in $\lambda.(\underline{0}\
\lambda.(\underline{1}\ \underline{0}))$ denotes $\cx$ of the original
expression. At the same time, $\underline{1}$ also denotes $\cx$ of the
original expression. On the other hand, the second $\underline{0}$ denotes
$\cy$ of the original expression. The distance from the definition depends on the
location of a variable. Since de Bruijn indices represent variables with their
distances, the indices of a single variable can vary among places.

Note that expressions should be treated as trees, not strings, to calculate the
distances. Consider the last example. There are two $\lambda$’s between the last
$\cx$ and its definition when the expression is written as a string. However,
when the abstract syntax tree representing the expression is considered, there is
only one $\lambda$ in between. Therefore, the index of the last $\cx$ is
$\underline{1}$, not $\underline{2}$. We usually write expressions as strings for
convenience, but they always have tree structures in fact.

De Bruijn indices successfully resolve the issues arising from names. Consider
the comparison of expressions. $\lambda \cx.\cx$ and $\lambda \cy.\cy$
are semantically equivalent but syntactically different expressions.
Both become $\lambda.\underline{0}$ when de Bruijn indices are used. By the help
of de Bruijn indices, a simple syntactic check will find out that two expressions
are equal.

Now, let us define the procedure that transforms named expressions into nameless
expressions. It helps readers understand de Bruijn indices. At the same time, the
procedure is practically valuable. Use of names is the best way to denote
variables for programmers. Therefore, expressions written by programmers have
names. On the other hand, programs like interpreters and compilers sometimes need
to use de Bruijn indices to represent variables. In such cases, the procedure is
a part of the interpreter/compiler implementation.

First, we define indices as follows:

\[ i \in \mathbb{N} \]

where the metavariable $i$ ranges over indices.

Then, we can define nameless expressions as follows:
\sidenote{This chapter uses $e$ to denote both named expressions and nameless
expressions. Strictly speaking, two different metavariables should be introduced
to denote each sort of an expression separately. However, for brevity, we abuse
the notation and use $e$ for both sorts of an expression.}

\[
e\ ::= \ \underline{i}
\ |\ \lambda.e
\ |\ e\ e
\ |\ n
\ |\ e+e
\]

In nameless expressions, natural numbers represent variables. Those numbers have
underlines and, therefore, cannot be confused with integers. A lambda
abstraction $\lambda.e$ lacks the name of its parameter. Note that
$\lambda.e$ does have a single parameter. It is not a function with zero
parameters.

A context, which is a finite partial function from names to natural
numbers, takes an important role during the transformation. A context gives the
distance between a variable and its definition.

\[\chi\in\embox{Id}\finto\mathbb{N}\]

The metavariable $\chi$ ranges over contexts.

Let $[e]\chi$ be a nameless expression representing $e$ under a context $\chi$.
The definition of $[e]\chi$ is as follows:

\[
  \begin{array}{r@{~}c@{~}l}
  \lbrack x\rbrack \chi &=& \underline{i}\ \ \text{if}\ \chi(x)=i \\
  \lbrack \lambda x.e\rbrack \chi &=& \lambda.\lbrack e\rbrack {\chi'}\ \ \text{where}\
  \chi'=(\uparrow\chi)\lbrack x\mapsto 0\rbrack  \\
  \lbrack e_1\ e_2\rbrack \chi &=& \lbrack e_1\rbrack \chi\ \lbrack e_2\rbrack \chi \\
  \lbrack n\rbrack \chi &=& n \\
  \lbrack e_1+e_2\rbrack \chi &=& \lbrack e_1\rbrack \chi+\lbrack e_2\rbrack \chi \\
  \end{array}
\]

$[x]\chi$ is the result of transforming $x$. A natural number represents a
variable, and the natural number can be found in $\chi$. Therefore, when
$\chi(x)$ is $i$, $x$ is transformed into $\underline{i}$.

$[\lambda x.e]\chi$ is the result of transforming $\lambda x.e$ and should look
like $\lambda.e$. However, $e$ uses names and, thus, needs to be transformed.
$\chi$ is not the correct context for the transformation of $e$. First, it lacks
the information of $x$. If $x$ appears in $e$ without any function definitions,
there is no $\lambda$ between the use and the definition. The context must know
that the index of $x$ is 0. In addition, indices in $\chi$ need changes. Suppose
that $x'$ is in $\chi$ and its index is 0. If $x'$ occurs in $e$, its index is
not 0 anymore. Since $e$ is the body of $\lambda x.e$, there is one $\lambda$
between $x'$ and its definition. During the transformation of $e$, the index of
$x'$ is 1, not 0. Similarly, if there is a variable whose index is 1 in $\chi$,
its index must be 2 during the transformation of $e$. In conclusion, every index
in $\chi$ has to increase by one. $\uparrow\chi$ denotes the context same as
$\chi$ but whose indices are one larger. The context used during the
transformation of $e$ is $(\uparrow\chi)[x\mapsto0]$. $[\lambda x.e]\chi$ is
$\lambda.[e]{\chi'}$ where $\chi'$ is $(\uparrow\chi)[x\mapsto0]$.

The remaining cases are straightforward. The transformations of $e_1\ e_2$ and
$e_1+e_2$ are recursively defined. Since $n$ does not contain variables, $n$
itself is the result.

Below shows how $\lambda \cx.\lambda \cy.\cx+\cy$ is transformed
by the procedure. In the beginning, the context is empty because there is no
variable yet.

\[
\begin{array}{cl}
& [\lambda \cx.\lambda \cy.\cx+\cy]\emptyset \\
= & \lambda.[\lambda \cy.\cx+\cy][\cx\mapsto 0] \\
= & \lambda.\lambda.[\cx+\cy][\cx\mapsto 1,\cy\mapsto 0] \\
= & \lambda.\lambda.[\cx][\cx\mapsto 1,\cy\mapsto 0]+[\cy][\cx
\mapsto 1,\cy\mapsto 0] \\
= & \lambda.\lambda.\underline{1}+[\cy][\cx\mapsto 1,\cy\mapsto 0] \\
= & \lambda.\lambda.\underline{1}+\underline{0} \\
\end{array}
\]

Now, let us implement the procedure in Scala. For named expressions, we can
reuse the previous implementation. The following code defines nameless
expressions:

\begin{verbatim}
object Nameless {
  sealed trait Expr
  case class Num(n: Int) extends Expr
  case class Add(l: Expr, r: Expr) extends Expr
  case class Id(i: Int) extends Expr
  case class Fun(e: Expr) extends Expr
  case class App(f: Expr, a: Expr) extends Expr
}
\end{verbatim}

\code{Id($i$)} represents $\underline{i}$, and
\code{Fun($e$)} represent $\lambda.e$.

Note that nameless expressions are defined in the \code{Nameless} singleton object.
Therefore, outside the object, \code{Expr} denotes the type of named
expressions, while \code{Nameless.Expr} denotes the type of nameless
expressions. Similarly, \code{Id} represents a variable represented by a name,
while \code{Nameless.Id} represents a variable represented by an index.

\begin{verbatim}
type Ctx = Map[String, Int]
\end{verbatim}

\code{Ctx}, the type of a context, is a map from strings to integers.

The following \code{transform} function recursively transforms a named expression
into a nameless expression:

\begin{verbatim}
def transform(e: Expr, ctx: Ctx): Nameless.Expr = e match {
  case Id(x) => Nameless.Id(ctx(x))
  case Fun(x, e) =>
    val nctx = ctx.map{ case (x, i) => x -> (i + 1) } + (x -> 0)
    Nameless.Fun(transform(e, nctx))
  case App(f, a) =>
    Nameless.App(transform(f, ctx), transform(a, ctx))
  case Num(n) => Nameless.Num(n)
  case Add(l, r) =>
    Nameless.Add(transform(l, ctx), transform(r, ctx))
}
\end{verbatim}

The function exactly looks like its mathematical definition, so it is easy to
understand the code.

Lists can replace maps in the implementation. A context is a list of names, and
the index of a name is the location of the name in the list. Lists simplify the
implementation. When a name is added to a context, its index is always zero. It
means that the name is the head of the list. Adding a name is the same as making
the head of the list be the name. Increasing every index by one is the same as
moving each name backward by one slot. Therefore, if a context is a list,
prepending a new name in front of the list does everything we need to extend
the context. For
example, consider a context containing $\cx$ and $\cy$. Let the indices of
$\cx$ and $\cy$, respectively, be 0 and 1. The context is represented by
\code{List("x", "y")}. It is enough to prepend $\cz$ to the list to add
$\cz$ to the context. The resulting list is \code{List("z", "x", "y")}---$\cz$ at
index 0, $\cx$ at index 1, and $\cy$ at index 2. Since $\cz$ is the new
name, its index should be 0. At the same time, the indices of $\cx$ and $\cy$
should be greater by one than before. The new list does represent the new context
well.

To use lists instead, we change the definition of \code{Ctx}.

\begin{verbatim}
type Ctx = List[String]
\end{verbatim}

Now, \code{Ctx} is a list of strings. Then, we can revise \code{transform}
accordingly.

\begin{verbatim}
def transform(e: Expr, ctx: Ctx): Nameless.Expr = e match {
  case Id(x) => Nameless.Id(locate(x, ctx))
  case Fun(x, e) => Nameless.Fun(transform(e, x :: ctx))
  case App(f, a) =>
    Nameless.App(transform(f, ctx), transform(a, ctx))
  case Num(n) => Nameless.Num(n)
  case Add(l, r) =>
    Nameless.Add(transform(l, ctx), transform(r, ctx))
}

def locate(x: String, ctx: Ctx): Int = ctx match {
  case Nil => error()
  case h :: t => if (h == x) 0 else 1 + locate(x, t)
}
\end{verbatim}

The \code{Id} case needs to calculate the location of a given variable in a
given context. For this purpose, we define \code{locate}. In the \code{Fun}
case, \code{x :: ctx} is everything we need to add \code{x} to \code{ctx}.

\section{Evaluation of Nameless Expressions}

Evaluation of nameless expressions is similar to evaluation of named
expressions. The definition of a value has a minor difference:

\[
  v\ ::=\ n\ |\ \langle\lambda.e,\sigma\rangle
\]

As lambda abstractions lack parameter names, closures also lack parameter names.

The definition of an environment also has an insignificant difference:

\[
  \sigma \in \mathbb{N}\finto V
\]

Environments are finite partial functions from indices, which are natural numbers, to
values.

Now, let us define the inference rules.

\semanticrule{Id}{
If
  $i$ is in the domain of $\sigma$,\\
then
  \evaldn{\underline{i}}{\sigma(i)}.
}

\vspace{-1em}

\[
  \inferrule
  { i\in\dom{\sigma} }
  { \sigma\vdash\underline{i}\Rightarrow\sigma(i) }
  \quad\textsc{[Id]}
\]

The value of a variable can be found in a given environment.

\semanticrule{Fun}{
  \evaldn{\lambda.e}{\langle\lambda.e,\sigma\rangle}.
}

\vspace{-1em}

\[
  \sigma\vdash\lambda.e\Rightarrow\langle\lambda.e,\sigma\rangle
  \quad\textsc{[Fun]}
\]

A lambda abstraction evaluates to a closure without evaluating anything.

\semanticrule{App}{
  \begin{tabular}{@{\hskip0pt}l@{\hskip0pt}l}
    If \\
    & \evaldn{e_1}{\langle\lambda.e,\sigma'\rangle}, \\
    & \evaldn{e_2}{v_2}, and \\
    & \evaln{(\uparrow\sigma')[0\mapsto v_2]}{e}{v}, \\
    then \\
    & \evaldn{\eapp{e_1}{e_2}}{v}.
  \end{tabular}
}

\vspace{-1em}

\[
  \inferrule
  {
    \sigma\vdash e_1\Rightarrow\langle\lambda.e,\sigma'\rangle \\
    \sigma\vdash e_2\Rightarrow v_2 \\
    (\uparrow\sigma')[0\mapsto v_2]\vdash e\Rightarrow v
  }
  { \sigma\vdash e_1\ e_2\Rightarrow v }
  \quad\textsc{[App]}
\]

Evaluation of $e_1\ e_2$ evaluates both $e_1$ and $e_2$. Then, the body of the
closure is evaluated under the environment captured by the closure with the value
of the argument. If the parameter is used in the body, there is no $\lambda$
between the use and the definition. Its index is 0. Therefore, the value of the
argument has the index 0 in the new environment. In addition, every index in the
environment of the closure needs a change. Let a value $v$ correspond to the
index 0. The value is not the value of the argument, so it cannot correspond to
the index 0 anymore. As $\lambda$ from the closure exists between the use and the
definition, the index should increase by one. By the same principle, every index
in the environment increases by one. Since $\uparrow\sigma'$ denotes the context
same as $\sigma'$ but whose indices are one larger, the body of the closure
is evaluated under $(\uparrow\sigma')[0\mapsto v_2]$.

The rules for integers and addition are omitted because they are the same as those
of \lang.

This new semantics for nameless expressions is equivalent to the previous
semantics for named expressions.
Let $e$ be a named expression. The result of evaluating $e$ is the same as
evaluating $e'$ where $e'$ is the nameless expression obtained by transforming $e$.
\sidenote{Assume that the equality of closures is defined properly.}
Mathematically, the following proposition is true:

$\forall e.\forall v.(\emptyset\vdash e\Rightarrow
v)\leftrightarrow(\emptyset\vdash[e]\emptyset\Rightarrow v)$.

Let us implement an interpreter of nameless expressions in Scala.
Below is the definitions of values and environments.\sidenote{At this point, we
do not consider named expressions, so we omit the \code{Nameless} singleton
object.}

\begin{verbatim}
type Env = List[Value]

sealed trait Value
case class NumV(n: Int) extends Value
case class CloV(e: Expr, env: Env) extends Value
\end{verbatim}

An environment is a list of values. As shown by the implementation of
\code{transform}, lists are simpler than maps from integers to values.

\begin{verbatim}
def interp(e: Expr, env: Env): Value = e match {
  case Id(i) => env(i)
  case Fun(e) => CloV(e, env)
  case App(f, a) =>
    val CloV(b, fenv) = interp(f, env)
    interp(b, interp(a, env) :: fenv)
  case Num(n) => NumV(n)
  case Add(l, r) =>
    val NumV(n) = interp(l, env)
    val NumV(m) = interp(r, env)
    NumV(n + m)
}
\end{verbatim}

The \code{App} case is the only interesting case. The others are the same as
before. Since a closure lacks its parameter name and an environment does not need
the name, it is enough to prepend the value of the argument in front of the list.

\section{Exercises}

\begin{exercise}
\labex{nameless-representation-of-expressions-trans}

\item Write the nameless representation of the following expression:
\[
  \eapp{\eapp{\eapp{(\efun{\cx}{\efun{\cy}{\efun{\cz}{\eadd{(\esub{\cz}{\cx})}{\cy}}}})}{42}}{0}}{10}
\]

\end{exercise}
