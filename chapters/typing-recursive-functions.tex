\setchapterpreamble[u]{\margintoc}
\chapter{Typing Recursive Functions}
\labch{typing-recursive-functions}

The article defines TRFAE, which extends TFAE with conditional expressions and
recursive functions. It defines the abstract syntax, the dynamic semantics, and
the type system of TRFAE, and implement a type checker and an interpreter of
TRFAE.

TFAE has the normalization property. Evaluation of every well-typed TFAE
expression terminates in a finite time. It implies that the fixed point
combinator is ill-typed in TFAE. Use of the fixed point combinator can create an
expression whose evaluation does not terminate. Therefore, programmers cannot
define recursive functions in TFAE. The article defines TRFAE, which features
recursive functions. The type system accepts some expressions using recursive
functions after the extension. The extension from FAE to RFAE does not increase
the expressivity. It is only for convenience of programmers. On the other hand,
the extension from TFAE to TRFAE does increase the expressivity by allowing
recursive functions.

\section{Syntax}

The following is the abstract syntax of TRFAE:

\[
\begin{array}{lrcl}
\text{Expression} & e & ::= & \cdots \\
&& | & \textsf{if0}\ e\ e\ e \\
&& | & \textsf{def}\ x(x:\tau):\tau=e\ \textsf{in}\ e \\
\text{Value} & v & ::= & n \\
&& | & \langle \lambda x.e,\sigma \rangle \\
\end{array}
\]

$\textsf{if0}\ e_1\ e_2\ e_3$ is the same as that of RFAE. $e_1$ is the
condition; $e_2$ is the true branch; $e_3$ is the false branch.

$\textsf{def}\ x_1(x_2:\tau_1):\tau_2=e_1\ \textsf{in}\ e_2$ defines a recursive
function. It is similar to a recursive function of RFAE. The only difference is
type annotation $\tau_1$ and $\tau_2$. $\tau_1$ denotes the parameter type of the
function; $\tau_2$ denotes the return type of the function. The function must
take an argument of type $\tau_1$ and return a value of type $\tau_2$. Type
annotations are used for type checking, just like type annotations in TFAE.

\section{Dynamic Semantics}

The dynamic semantics of TRFAE is similar to that of RFAE. The rules common to
TFAE are omitted.

\[
\inferrule
{ \sigma\vdash e_1\Rightarrow 0 \\
  \sigma\vdash e_2\Rightarrow v }
{ \sigma\vdash\textsf{if0}\ e_1\ e_2\ e_3\Rightarrow v}
\]

\[
\inferrule
{ \sigma\vdash e_1\Rightarrow v' \\
  v'\not=0 \\
  \sigma\vdash e_3\Rightarrow v }
{ \sigma\vdash\textsf{if0}\ e_1\ e_2\ e_3\Rightarrow v}
\]

\[
\inferrule
{ \sigma'=\sigma\lbrack x_1\mapsto\langle\lambda x_2.e,\sigma'\rangle\rbrack }
{ \sigma\vdash\mu x_1:\tau_1\rightarrow\tau_2.\lambda
x_2.e\Rightarrow\langle\lambda x_2.e,\sigma'\rangle}
\]

\section{Type System}

The rules common to TFAE are omitted.

\[
\inferrule
{ \Gamma\vdash e_1:\tau' \\
  \Gamma\vdash e_2:\tau \\
  \Gamma\vdash e_3:\tau }
{ \Gamma\vdash\textsf{if0}\ e_1\ e_2\ e_3:\tau}
\]

The condition of a conditional expression must be well-typed. If the result of
the condition is $0$, then the first branch is evaluated. Otherwise, the second
branch is evaluated. Therefore, the result can be any value. $\Gamma\vdash
e_1:\tau'$ denotes that. The rule cannot determine whether the first or the
second branch will be evaluated. Since every expression has at most one type,
$e_2$ and $e_3$ have the same type, $\tau$. The type of the whole expression is
$\tau$.

One may change the above rule to the following rule:

\[
\inferrule
{ \Gamma\vdash e_1:\textsf{num} \\
  \Gamma\vdash e_2:\tau \\
  \Gamma\vdash e_3:\tau }
{ \Gamma\vdash\textsf{if0}\ e_1\ e_2\ e_3:\tau}
\]

Both rules make the type system sound. The latter rejects more expressions than
the former. However, programmers usually want to use an integer as a condition.
Therefore, rejecting nonintegral values being used as a condition can prevent
mistakes of programmers.

\[
\inferrule
{
  \Gamma\lbrack x_1:\tau_1\rightarrow\tau_2,x_2:\tau_1\rbrack\vdash e_1:\tau_2
  \\
  \Gamma\lbrack x_1:\tau_1\rightarrow\tau_2\rbrack\vdash e_2:\tau
}
{ \Gamma\vdash\textsf{def}\ x_1(x_2:\tau_1):\tau_2=e_1\ \textsf{in}\ e_2:\tau}
\]

The static semantics of a recursive function is similar to that of a lambda
abstraction. The rule needs to check the type of the function body. The body can
use not only the parameter but also the function itself. The type of the function
is $\tau_1\rightarrow\tau_2$. The type of the parameter is $\tau_1$. Type
checking of the body uses the extended type environment, which has the type of
the function and the parameter. The type of $e_1$ has to be equal to the return
type of the function, which is $\tau_2$.

The following proof tree proves that the type of $\textsf{def}\
f(n:\textsf{num}):\textsf{
num}=\textsf{if0}\ n\ 0\ (n+(f\ (n-1)))\ \textsf{in}\ f\ 3$ is $\textsf{num}$:

\[\Gamma_1=\lbrack f:\textsf{num}\rightarrow\textsf{num},n:\textsf{num}\rbrack\]
\[\Gamma_2=\lbrack f:\textsf{num}\rightarrow\textsf{num}\rbrack\]

\[
\inferrule
{
  \Large
  \inferrule
  {
    \inferrule
    {n\in\mathit{Domain}(\Gamma_1)}
    {\Gamma_1\vdash n:\textsf{num}} \\
    \Gamma_1\vdash 0:\textsf{num} \\
    \inferrule
    {
      \inferrule
      {n\in\mathit{Domain}(\Gamma_1)}
      { \Gamma_1\vdash n:\textsf{num} } \\
      \inferrule
      {
        \inferrule
        {f\in\mathit{Domain}(\Gamma_1)}
        { \Gamma_1\vdash f:\textsf{num}\rightarrow\textsf{num} } \\
        \inferrule
        {
          \inferrule
          { n\in\mathit{Domain}(\Gamma_1) }
          { \Gamma_1\vdash n:\textsf{num} } \\
          \Gamma_1\vdash 1:\textsf{num}
        }
        { \Gamma_1\vdash n-1:\textsf{num} } \\
      }
      { \Gamma_1\vdash f\ (n-1):\textsf{num} }
    }
    { \Gamma_1\vdash n+(f\ (n-1)):\textsf{num} }
  }
  { \Gamma_1\vdash \textsf{if0}\ n\ 0\ (n+(f\ (n-1))):\textsf{num} }
  \\
  \inferrule
  {
    \inferrule
    { f\in\dom{\Gamma_2} }
    { \Gamma_2\vdash f:\textsf{num}\rightarrow\textsf{num} }
    \\
    \Gamma_2\vdash 3:\textsf{num}
  }
  { \Gamma_2\vdash f\ 3:\textsf{num} }
}
{
  \emptyset\vdash
  \textsf{def}\ f(n:\textsf{num}):\textsf{num}=\textsf{if0}\ n\ 0\ (n+(f\ (n-1)))\ \textsf{in}\ f\
3
  :\textsf{num}
}
\]

\section{Implementing a Type Checker}

The following Scala code implements the abstract syntax of TRFAE:

\begin{verbatim}
sealed trait Expr
case class Num(n: Int) extends Expr
case class Add(l: Expr, r: Expr) extends Expr
case class Sub(l: Expr, r: Expr) extends Expr
case class Id(x: String) extends Expr
case class Fun(x: String, t: Type, b: Expr) extends Expr
case class App(f: Expr, a: Expr) extends Expr
case class If0(c: Expr, t: Expr, f: Expr) extends Expr
case class Rec(f: String, x: String, p: Type, r: Type, b: Expr, e: Expr) extends
Expr

sealed trait Type
case object NumT extends Type
case class ArrowT(p: Type, r: Type) extends Type

type TEnv = Map[String, Type]

def mustSame(t1: Type, t2: Type): Type =
  if (t1 == t2) t1 else throw new Exception
\end{verbatim}

A \code{Expr} instance represent a TRFAE expression. The implementation is
similar to that of RFAE.

\begin{verbatim}
case If0(c, t, f) =>
  typeCheck(c, env)
  mustSame(typeCheck(t, env), typeCheck(f, env))
\end{verbatim}

\[
\inferrule
{ \Gamma\vdash e_1:\tau' \\
  \Gamma\vdash e_2:\tau \\
  \Gamma\vdash e_3:\tau }
{ \Gamma\vdash\textsf{if0}\ e_1\ e_2\ e_3:\tau}
\]

The condition of an expression must be well-typed. The \code{typeCheck} function
checks the type of \code{c}, and the result of type checking is unnecessary. The
types of the two branches must be the same. The \code{typeCheck} function checks
the types of \code{t} and \code{f}. The \code{mustSame} function compares the
results. If they are the same, then the type is the type of the whole expression.

\begin{verbatim}
case Rec(f, x, p, r, b, e) =>
  val t = ArrowT(p, r)
  val nenv = env + (f -> t)
  mustSame(r, typeCheck(b, nenv + (x -> p)))
  typeCheck(e, nenv)
\end{verbatim}

\[
\inferrule
{
  \Gamma\lbrack x_1:\tau_1\rightarrow\tau_2,x_2:\tau_1\rbrack\vdash e_1:\tau_2
  \\
  \Gamma\lbrack x_1:\tau_1\rightarrow\tau_2\rbrack\vdash e_2:\tau
}
{ \Gamma\vdash\textsf{def}\ x_1(x_2:\tau_1):\tau_2=e_1\ \textsf{in}\ e_2:\tau}
\]

The parameter type is \code{p}, and the return type is \code{r}. Thus, the type
of \code{f} is the function type from \code{p} to \code{r}. The type of \code{x}
is \code{p}. To type-check \code{b}, the type environment must have the types of
\code{f} and \code{x}. The type of \code{b} must equal \code{r}. The
\code{mustSame} function compares the types. The function can be used not only in
\code{b}, which is the body of the function, but also in \code{e}. On the other
hand, the parameter \code{x} cannot be used in \code{e}. Therefore, it is enough
to add only the type of \code{f} to the type environment used to type-check
\code{e}. The type of the whole expression is equal to the type of \code{e}.

The following shows the complete code of the function:

\begin{verbatim}
def typeCheck(e: Expr, env: TEnv): Type = e match {
  case Num(n) => NumT
  case Add(l, r) =>
    mustSame(mustSame(typeCheck(l, env), NumT), typeCheck(r, env))
  case Sub(l, r) =>
    mustSame(mustSame(typeCheck(l, env), NumT), typeCheck(r, env))
  case Id(x) => env(x)
  case Fun(x, t, b) =>
    ArrowT(t, typeCheck(b, env + (x -> t)))
  case App(f, a) =>
    val ArrowT(t1, t2) = typeCheck(f, env)
    val t3 = typeCheck(a, env)
    mustSame(t1, t3)
    t2
  case If0(c, t, f) =>
    typeCheck(c, env)
    mustSame(typeCheck(t, env), typeCheck(f, env))
  case Rec(f, x, p, r, b, e) =>
    val t = ArrowT(p, r)
    val nenv = env + (f -> t)
    mustSame(r, typeCheck(b, nenv + (x -> p)))
    typeCheck(e, nenv)
}
\end{verbatim}

The following code checks the type of $\textsf{def}\
f(n:\textsf{num}):\textsf{num}=\textsf{
if0}\ n\ 0\ (n+(f\ (n-1)))\ \textsf{in}\ f\ 3$:

\begin{verbatim}
// def f(x: num): num = if0 n 0 (n + (f (n-1))); 3
typeCheck(
  Rec(
    "f", "n", NumT, NumT,
    If0(Id("n"),
        Num(0),
        Add(
          Id("n"),
          App(Id("f"), Sub(Id("n"), Num(1)))
        )
    ),
    App(Id("f"), Num(3))
  ),
  Map.empty
)
// num
\end{verbatim}

\section{Implementing an Interpreter}

The interpreter is similar to that of RFAE.

\begin{verbatim}
sealed trait Value
case class NumV(n: Int) extends Value
case class CloV(p: String, b: Expr, var e: Env) extends Value

type Env = Map[String, Value]

def interp(e: Expr, env: Env): Value = e match {
  case Num(n) => NumV(n)
  case Add(l, r) =>
    val NumV(n) = interp(l, env)
    val NumV(m) = interp(r, env)
    NumV(n + m)
  case Sub(l, r) =>
    val NumV(n) = interp(l, env)
    val NumV(m) = interp(r, env)
    NumV(n - m)
  case Id(x) => env(x)
  case Fun(x, _, b) => CloV(x, b, env)
  case App(f, a) =>
    val CloV(x, b, fEnv) = interp(f, env)
    interp(b, fEnv + (x -> interp(a, env)))
  case If0(c, t, f) =>
    interp(if (interp(c, env) == NumV(0)) t else f, env)
  case Rec(f, x, _, _, b, e) =>
    val cloV = CloV(x, b, env)
    val nenv = env + (f -> cloV)
    cloV.e = nenv
    interp(e, nenv)
}

def run(e: Expr): Value = {
  typeCheck(e, Map.empty)
  interp(e, Map.empty)
}
\end{verbatim}

The following code executes $\textsf{def}\ f(n:\textsf{num}):\textsf{num}=\textsf{if0}\ n\ 0\
(n+(f\ (n-1)))\ \textsf{in}\ f\ 3$:

\begin{verbatim}
// def f(x: num): num = if0 n 0 (n + (f (n-1))); 3
run(
  Rec(
    "f", "n", NumT, NumT,
    If0(Id("n"),
        Num(0),
        Add(
          Id("n"),
          App(Id("f"), Sub(Id("n"), Num(1)))
        )
    ),
    App(Id("f"), Num(3))
  )
)
// 6
\end{verbatim}

\section{Exercises}

\begin{enumerate}
\item Write a \texttt{TRFAE} expression $e$ such that
only one of $e$ and $\efun{\texttt{x}{:}\textsf{num}}{(\eapp{e}{\texttt{x}})}$ terminates,
while both $e$ and $\efun{\texttt{x}{:}\textsf{num}}{(\eapp{e}{\texttt{x}})}$ are
well-typed.

\item Consider the following TRCFAE expression:
{
\begin{verbatim}
        {rec {f : (num -> num) f}
             {f 10}}
\end{verbatim}
}
\begin{itemize}
\item[a)] Draw the type derivation of the above expression.
Write every piece in the type derivation even in the presence of any type error.

\item[b)] What happens if we interpret it?  Describe it in detail.

\item[c)] How can we solve this type soundness problem without changing the soundness theorem?
\end{itemize}

\item Recall \MNSC of the homework.
Consider the following subset of \MNSC:
\Space{-1}
\[
\begin{array}{rllrllrl}
p ::= &s\ \cdots\ s\ e&\qquad&
e ::= & n &\qquad&\tau ::= & \verb+num+\\
s ::= &\verb+val +x\verb+:+\tau\verb+ = +e&&\mid& x&&\mid&\tau\verb+ => +\tau\\
\mid&\verb+var +x\verb+:+\tau\verb+ = +e &\qquad&
\mid& e\verb+(+e\verb+)+\\
\mid&\verb+def +x\verb+(+x\verb+:+\tau\verb+)+\verb+:+\tau\verb+ = +e&&
\mid& x\verb+ = +e&&\\
\end{array}
\]
with the following typing rules:
\[
\begin{array}{ccccc}
{ \typed{n}{\code{num}} }
&\qquad&
\newinfrule
{ x\in\mbox{Domain}(\Gamma) }
{ \typed{x}{\Gamma(x)} }
&\qquad&
\newinfrule
{ \emptyset\vdash s_1:\Gamma_1 \rulesep \cdots \rulesep
  \Gamma_{n-1}\vdash s_n:\Gamma_n \rulesep
  \Gamma_n\vdash e:\tau }
{ \vdash s_1\ \cdots\ s_n\ e:\tau }
\\[15pt]
\multicolumn{5}{c}{\begin{array}{ccc}
\newinfrule
{ \Gamma\vdash{\tau} \rulesep \typed{e}{\tau} }
{ \typed{\code{val}\ x\code{:}\tau\ \code{=}\ e}{\Gamma[x\mapsto \tau]} }
&\qquad&
\newinfrule
{ \Gamma\vdash{\tau} \rulesep \typed{e}{\tau} }
{ \typed{\code{var}\ x\code{:}\tau\ \code{=}\ e}{\Gamma[x\mapsto \tau,x:\textsf{mutable}]} }
\end{array}}
\end{array}
\]

\begin{itemize}
\item[a)] Complete the typing rules 
of the forms \fbox{$\typed{s}{\Gamma}$} and \fbox{$\typed{e}{\tau}$}
by filling the blanks:
\Space{-3}

{
\[
\newinfrule
{ \lmblank }
{ \typed{\code{def}\ x_1\code{(}x_2\code{:}\tau_1\code{)}\code{:}\tau_2\ \code{=}\ e}{\sblank} }
\]
\[
\newinfrule
{ \lmblank }
{ \typed{e_1\code{(}e_2\code{)}}{\sblank} }
\]
\[
\newinfrule
{ \lmblank }
{ \typed{x\ \code{=}\ e}{\sblank} }
\]
}
\item[b)]
The following three programs are well-typed.
The reason is that
`\code{x} $:$ \textsf{mutable},' which denotes that \code{x} is mutable,
is in the type environments when the assignment expressions are type-checked
because of the first statement, $\verb+var x: num = 1+$:

\begin{tabular}{|l@{\hskip35pt}|l@{\hskip35pt}|l}
\verb+var x: num = 1+&\verb+var x: num = 1+&\verb+var x: num = 1+\\
\verb+val x: num = 2+&\verb+def x(z: num): num = z+&\verb+def y(x: num): num = (x = 3)+\\
\verb+x = 3+&\verb+def y(z: num): num = z+&\verb+1+\\
&\verb+x = y+
\end{tabular}

Revise the following typing rules of the form \fbox{$\typed{s}{\Gamma}$}
to make the above programs not well-typed by filling the blanks.
You may use `$\Gamma\setminus(x:\textsf{mutable})$'
to remove the mutable tag from variable $x$.
For example, `$[\verb+x+\mapsto\verb+num+,\verb+x+:\textsf{mutable}]\setminus(x:\textsf{mutable})$'
equals `$[\verb+x+\mapsto\verb+num+]$.'
\Space{-3}

{
\[
\newinfrule
{ \lmblank }
{ \typed{\code{val}\ x\code{:}\tau\ \code{=}\ e}{\sblank} }
\]
\[
\newinfrule
{ \lmblank }
{ \typed{\code{def}\ x_1\code{(}x_2\code{:}\tau_1\code{)}\code{:}\tau_2\ \code{=}\ e}{\sblank} }
\]
}
\end{itemize}

\item Consider the following language:
\[
\begin{array}{rllrllrllr@{\hskip3pt}l}
e::= & n &\qquad& c::= & \verb+skip+ &\qquad& b::= & \verb+true+ &\qquad& n\in&\mathbb{Z}\\
\mid& b &&\mid& x\verb+:=+e &&\mid& \verb+false+ && x\in&\mathit{Id}\\
\mid& x &&\mid& \verb+if+\ e\ c\ c && v::= & n && v\in&\mathit{Val}\\
\mid& e\verb!+!e &&\mid& \verb+while+\ e\ c &&\mid& b && \tau\in&\mathit{Type}\\
\mid& e\verb!<!e &&\mid& c\verb+;+c &&\tau::= & \verb+num+ && \sigma\in&\mathit{Id}\rightarrow\mathit{Val}\\
&&&&&&\mid& \verb+bool+ && \Gamma\in&\mathit{Id}\rightarrow\mathit{Type}\\
\end{array}
\]

Metavariable $c$ ranges over commands.

Under given environment $\sigma$, evaluation of an expression yields a value and does not
change $\sigma$. The following is the operational semantics of the expressions:

\[
\begin{array}{ccc}
\multicolumn{3}{c}{\begin{array}{ccccc}
\sigma\vdash n\Rightarrow n &\qquad&
\sigma\vdash b\Rightarrow b &\qquad&
\newinfrule
{ x\in\Dom{\sigma} }
{ \sigma\vdash x\Rightarrow \sigma(x) }
\end{array}} \\[20pt]
\newinfrule
{ \sigma\vdash e_1\Rightarrow n_1 \rulesep
  \sigma\vdash e_2\Rightarrow n_2 }
{ \sigma\vdash e_1\code{+}e_2\Rightarrow n_1+n_2 }
&\qquad&
\newinfrule
{ \sigma\vdash e_1\Rightarrow n_1 \rulesep
  \sigma\vdash e_2\Rightarrow n_2 }
{ \sigma\vdash e_1\code{<}e_2\Rightarrow n_1<n_2 }
\end{array}
\]

\begin{itemize}
\item[a)] Complete the following typing rules
of the form \fbox{$\typed{e}{\tau}$}
by filling the blanks:
\end{itemize}
\Space{-2}

{
\[
\Gamma\vdash n:\code{num}
\]

\[
\newinfrule
{ x\in\Dom{\Gamma} }
{ \typed{x}{\Gamma(x)} }
\]

\[
\newinfrule
{ \lblank }
{ \typed{b}{\sblank} }
\]
\[
\newinfrule
{ \lblank }
{ \typed{e_1\code{+}e_2}{\sblank} }
\]
\[
\newinfrule
{ \lblank }
{ \typed{e_1\code{<}e_2}{\sblank} }
\]
}

Evaluation of a command produces a new environment.
The following is the operational semantics of the commands:

\[
\begin{array}{ccccc}
\sigma\vdash \code{skip}\Rightarrow \sigma
&\qquad&
\newinfrule
{ \sigma\vdash e\Rightarrow v }
{ \sigma\vdash x\code{:=}e\Rightarrow \sigma[x\mapsto v] }
&\qquad&
\newinfrule
{ \sigma\vdash e\Rightarrow \code{true} \rulesep
  \sigma\vdash c_1\Rightarrow \sigma_1 }
{ \sigma\vdash \code{if}\ e\ c_1\ c_2\Rightarrow \sigma_1 }
\\[20pt]
\multicolumn{3}{c}{
\newinfrule
{ \sigma\vdash e\Rightarrow \code{true} \rulesep
  \sigma\vdash c\Rightarrow \sigma_1 \rulesep
  \sigma_1\vdash \code{while}\ e\ c\Rightarrow \sigma_2 }
{ \sigma\vdash \code{while}\ e\ c\Rightarrow \sigma_2 }
}
&&
\newinfrule
{ \sigma\vdash e\Rightarrow \code{false} \rulesep
  \sigma\vdash c_2\Rightarrow \sigma_1 }
{ \sigma\vdash \code{if}\ e\ c_1\ c_2\Rightarrow \sigma_1 }
\\[20pt]
\multicolumn{3}{c}{
\newinfrule
{ \sigma\vdash e\Rightarrow \code{false} }
{ \sigma\vdash \code{while}\ e\ c\Rightarrow \sigma }
}
&&
\newinfrule
{ \sigma\vdash c_1\Rightarrow \sigma_1 \rulesep
  \sigma_1\vdash c_2\Rightarrow \sigma_2 }
{ \sigma\vdash c_1\code{;}c_2\Rightarrow \sigma_2 }
\end{array}
\]

\begin{itemize}
\item[b)] Complete the following typing rules
of the form \fbox{$\typed{c}{\Gamma}$}
by filling the blanks.
The following command must be well-typed: \verb+x := 1; x := 2+.
However, the following command must be not well-typed: \verb+x := 1; x := true+.
\end{itemize}

{
\[
\newinfrule
{ \lblank }
{ \typed{\code{skip}}{\sblank} }
\]
\[
\newinfrule
{ x\not\in\Dom{\Gamma} \rulesep \mblank }
{ \typed{x\code{:=}e}{\sblank} }
\]
\[
\newinfrule
{ x\in\Dom{\Gamma} \rulesep \mblank }
{ \typed{x\code{:=}e}{\sblank} }
\]
\[
\newinfrule
{ \lblank }
{ \typed{\code{if}\ e\ c_1\ c_2}{\sblank} }
\]
\[
\newinfrule
{ \lblank }
{ \typed{\code{while}\ e\ c}{\sblank} }
\]
\[
\newinfrule
{ \lblank }
{ \typed{c_1\code{;}c_2}{\sblank} }
\]
}

\end{enumerate}
