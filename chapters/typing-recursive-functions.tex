\setchapterpreamble[u]{\margintoc}
\chapter{Typing Recursive Functions}
\labch{typing-recursive-functions}

\renewcommand{\plang}{\textsf{TFAE}\xspace}
\renewcommand{\lang}{\textsf{TRFAE}\xspace}

In \textsf{FAE}, recursive functions can be considered as syntactic sugar.
For example, a function that computes the sum from one to a given integer
can be implemented as the following:

$Z\ (\efun{\cf}{\efun{\cv}{\sumbodyfv}})$

where

$
  Z=\efun{\cf}{\eapp
    {(\efun{\cx}{\eapp{\cf}{(\efun{\cv}{\eapp{\eapp{\cx}{\cx}}{\cv}})}})}
    {(\efun{\cx}{\eapp{\cf}{(\efun{\cv}{\eapp{\eapp{\cx}{\cx}}{\cv}})}})}
  }
$

Thus, \textsf{RFAE} has the exactly same expressive power as \textsf{FAE}.

However, we cannot implement recursive functions in \plang. Why cannot we
use the same approach as \textsf{FAE}? See the body of $Z$. We can find
$\eapp{\cx}{\cx}$. Unfortunately, such an expression is ill-typed in \plang.
Let us try to find the type of $\cx$. It is sure that the type cannot be $\tnum$
because $\cx$ is applied to a value. Therefore, the type must be
$\tarrow{\tau_1}{\tau_2}$ for some $\tau_1$ and $\tau_2$. Since a function
of $\tarrow{\tau_1}{\tau_2}$ is applied to $\cx$, the type of $\cx$ must be
$\tau_1$. It implies that the type of $\cx$ is $\tarrow{\tau_1}{\tau_2}$ and
$\tau_1$ at the same time. Since a value has at most one type in \plang,
$\tarrow{\tau_1}{\tau_2}$ must be the same as $\tau_1$. However, it is
impossible. In \plang, such a type does not exist. A type cannot be the same as
a part of itself. Since the fixed point combinator is ill-typed, we cannot
desugar recursive functions to nonrecursive functions in \plang.

More interestingly, not only recursive functions, but also any nonterminating
programs cannot be written in \plang. In other words, every well-typed \plang
expression evaluates to a value in a finite time. This is called the
\textit{normalization}\index{normalization} property of \plang. The
normalization property of \plang was proved by Tait in
1967~\cite{tait1967intensional}.\sidenote{Actually, the Tait's proof is about
the lazy version of \plang, but the same technique can be applied to \plang,
which is eager, as Pierce discussed~\cite{pierce2002types}.}

This chpater defines \lang, which extends \plang with recursive functions. By
adding recursive functions to the language, programmers become able to implement
recursive functions and nonterminating programs, which are impossible in \plang.
Thus, while the extension from \textsf{FAE} to \textsf{RFAE} does not increase
the expressivity, the extension from \plang to \lang dose.

\section{Syntax}

The following is the syntax of \lang:

\[
e\ ::=\ \cdots
\ | \ \eifz{e}{e}{e}
\ | \ \erect{x}{x}{\tau}{\tau}{e}{e}
\]

$\erect{x_1}{x_2}{\tau_1}{\tau_2}{e_1}{e_2}$ defines a recursive
function. It is similar to a recursive function in \textsf{RFAE} but
additionally has
type annotations $\tau_1$ and $\tau_2$. $\tau_1$ denotes the parameter type of the
function, and $\tau_2$ denotes the return type of the function. They are
used for type checking, just like type annotations in \plang.

\section{Dynamic Semantics}

The dynamic semantics of \lang is similar to that of \textsf{RFAE}, but type
annotations are ignored during closure construction.

\semanticrule{Rec}{
If
  \evaln{\sigma'}{e_2}{v}, where
  $\sigma'=\sigma[x_1\mapsto\clov{x_2}{e_1}{\sigma'}]$, \\
then
  \evaldn{\erect{x_1}{x_2}{\tau_1}{\tau_2}{e_1}{e_2}}{v}.
}

\vspace{-1em}

\[
  \inferrule
  {
    \sigma'=\sigma[x_1\mapsto\clov{x_2}{e_1}{\sigma'}] \\
    \eval{\sigma'}{e_2}{v}
  }
  { \evald{\erect{x_1}{x_2}{\tau_1}{\tau_2}{e_1}{e_2}}{v} }
  \quad\textsc{[Rec]}
\]

\section{Interpreter}

The following Scala code implements the syntax:

\begin{verbatim}
sealed trait Expr
...
case class If0(c: Expr, t: Expr, f: Expr) extends Expr
case class Rec(
  f: String, x: String, p: Type, r: Type, b: Expr, e: Expr
) extends Expr
\end{verbatim}

\code{If0($e_1$, $e_2$, $e_3$)} represents $\eifz{e_1}{e_2}{e_3}$, and
\code{Rec($x_1$, $x_2$, $\tau_1$, $\tau_2$, $e_1$, $e_2$)} represents
$\erect{x_1}{x_2}{\tau_1}{\tau_2}{e_1}{e_2}$.

The \code{interp} function is similar to that of \textsf{RFAE}, but type
annotations are ignored during closure construction.

\begin{verbatim}
def interp(e: Expr, env: Env): Value = e match {
  ...
  case Rec(f, x, _, _, b, e) =>
    val cloV = CloV(x, b, env)
    val nenv = env + (f -> cloV)
    cloV.e = nenv
    interp(e, nenv)
}
\end{verbatim}

\section{Static Semantics}

We need to define typing rules for conditional expressions and recursive
functions. Let us consider conditional expressions first.

\typerule{Typ-If0}{
  \begin{tabular}{@{\hskip0pt}l@{\hskip-10pt}l}
    If \\
    & \typeofdn{e_1}{\tnum}, \\
    & \typeofdn{e_2}{\tau}, and \\
    & \typeofdn{e_3}{\tau}, \\
    then \\
    & \typeofdn{\eifz{e_1}{e_2}{e_3}}{\tau} \\
  \end{tabular}
}

\vspace{-1em}

\[
  \inferrule
  { \typeofd{e_1}{\tnum} \\
    \typeofd{e_2}{\tau} \\
    \typeofd{e_3}{\tau} }
  { \typeofd{\eifz{e_1}{e_2}{e_3}}{\tau} }
  \quad\textsc{[Typ-If0]}
\]

The condition of a conditional expression must be a value of $\tnum$.
The rule cannot determine which branch will be evaluated at run time.
Since every expression has at most one type, $e_2$ and $e_3$ must have the same
type, $\tau$. The type of the whole expression also is $\tau$.

Actually, the condition can have any type. If the result of
the condition is $0$, then the true branch is evaluated. If the result is a
nonzero integer or a closure, then the false branch is evaluated.
A value of any type can be safely used as the condition.
Therefore, the type system may use the following rule instead of
Rule~\textsc{Typ-If0}:

\typerule{Typ-If0'}{
  \begin{tabular}{@{\hskip0pt}l@{\hskip-10pt}l}
    If \\
    & \typeofdn{e_1}{\tau'}, \\
    & \typeofdn{e_2}{\tau}, and \\
    & \typeofdn{e_3}{\tau}, \\
    then \\
    & \typeofdn{\eifz{e_1}{e_2}{e_3}}{\tau} \\
  \end{tabular}
}

\vspace{-1em}

\[
  \inferrule
  { \typeofd{e_1}{\tau'} \\
    \typeofd{e_2}{\tau} \\
    \typeofd{e_3}{\tau} }
  { \typeofd{\eifz{e_1}{e_2}{e_3}}{\tau} }
  \quad\textsc{[Typ-If0']}
\]

Both rules make the type system sound, but they are different from each other.
Rule~\textsc{Typ-If0} rejects more expressions than Rule~\textsc{Typ-If0'}
because the former allows only integers to be conditions, while the latter
allows functions as well. Therefore, from the perspective of reducing false positives,
Rule~\textsc{Typ-If0'} is better than Rule~\textsc{Typ-If0}.
However, if the type of the condition is a function type, it is highly likely to
be a mistake of the programmer. When the condition evaluates to a function, the
conditional expression always evaluates its false branch. The use of a
conditional expression is totally meaningless. Thus, from the perspective of
detecting programmers' mistakes, Rule~\textsc{Typ-If0} is better than
Rule~\textsc{Typ-If0'}.

Now, we define the typing rule of a recursive function.

\typerule{Typ-Rec}{
  \begin{tabular}{@{\hskip0pt}l@{\hskip-10pt}l}
    If \\
    & \typeofn{\Gamma[x_1:\tarrow{\tau_1}{\tau_2},x_2:\tau_1]}{e_1}{\tau_2}
    and \\
    & \typeofn{\Gamma[x_1:\tarrow{\tau_1}{\tau_2}]}{e_2}{\tau}, \\
    then \\
    & \typeofdn{\erect{x_1}{x_2}{\tau_1}{\tau_2}{e_1}{e_2}}{\tau}.
  \end{tabular}
}

\vspace{-1em}

\[
  \inferrule
  { \typeof{\Gamma[x_1:\tarrow{\tau_1}{\tau_2},x_2:\tau_1]}{e_1}{\tau_2} \\
    \typeof{\Gamma[x_1:\tarrow{\tau_1}{\tau_2}]}{e_2}{\tau} }
  { \typeofd{\erect{x_1}{x_2}{\tau_1}{\tau_2}{e_1}{e_2}}{\tau} }
  \quad\textsc{[Typ-Rec]}
\]

The principle of the rule is the same as the typing rule of a lambda
abstraction: given type annotations are used during the type checking of the
function body. Since the function itself can be used in the body, the type
checking of the body requires the type of the function. This is the reason that
the expression needs the return type annotation in addition to the parameter
type annotation. To type-check the body, the return type must be known.
For the type checking of the body, the type environment is extended with the
type of the function, $\tau_1\rightarrow\tau_2$, and the type of the parameter,
$\tau_1$. Since the type of the body is the return type, it must be $\tau_2$.
After the type checking of the body, $e_2$ is type-checked. For $e_2$, only the
type of the function is required; the parameter can be used only in the body.

The following proof trees prove that the type of
$\erect{\cf}{\cx}{\tnum}{\tnum}{\sumbodyf}{\cf\ 3}$
is $\tnum$:

\[
  \footnotesize
  \inferrule
  {
    \inferrule
    { \cx\in\dom{\Gamma_1} }
    { \typeof{\Gamma_1}{\cx}{\tnum} }
    \\
    \inferrule
    {
      \inferrule
      { \cf\in\dom{\Gamma_1} }
      { \typeof{\Gamma_1}{\cf}{\tarrow{\tnum}{\tnum}} }
      \\
      \inferrule
      {
        \inferrule
        { \cx\in\dom{\Gamma_1} }
        { \typeof{\Gamma_1}{\cx}{\tnum} }
        \\
        \typeof{\Gamma_1}{1}{\tnum}
      }
      { \typeof{\Gamma_1}{\cx-1}{\tnum} }
    }
    { \typeof{\Gamma_1}{\cf\ (\cx-1)}{\tnum} }
  }
  { \typeof{\Gamma_1}{\cx+(\cf\ (\cx-1))}{\tnum} }
\]

\[
  \footnotesize
  \inferrule
  {
    \inferrule
    { \cx\in\dom{\Gamma_1} }
    { \typeof{\Gamma_1}{\cx}{\tnum} }
    \\
    \typeof{\Gamma_1}{0}{\tnum}
    \\
    \typeof{\Gamma_1}{\cx+(\cf\ (\cx-1))}{\tnum}
  }
  { \typeof{\Gamma_1}{\sumbodyf}{\tnum} }
\]

\[
  \footnotesize
  \inferrule
  {
    \typeof{\Gamma_1}{\sumbodyf}{\tnum}
    \\
    \inferrule
    {
      \inferrule
      { \cf\in\dom{\Gamma_2} }
      { \typeof{\Gamma_2}{\cf}{\tarrow{\tnum}{\tnum}} }
      \\
      \typeof{\Gamma_2}{3}{\tnum}
    }
    { \typeof{\Gamma_2}{\cf\ 3}{\tnum} }
  }
  { \typeofe{\erect{\cf}{\cx}{\tnum}{\tnum}{\sumbodyf}{\cf\ 3}}{\tnum} }
\]

where
$\Gamma_1=[\cf:\tarrow{\tnum}{\tnum},\cx:\tnum]$
and
$\Gamma_2=[\cf:\tarrow{\tnum}{\tnum}]$.

\section{Type Checker}

To implement a type checker, we need to add the \code{If0} and \code{Rec} cases
to the \code{typeCheck} function for \plang.

\begin{verbatim}
case If0(c, t, f) =>
  mustSame(typeCheck(c, env), NumT)
  val tt = typeCheck(t, env)
  val tf = typeCheck(f, env)
  mustSame(tt, tf)
  tt
\end{verbatim}

The condition of an expression must belong to $\tnum$. The type of \code{c} is
computed with \code{typeCheck} and compared to \code{NumT} with \code{mustSame}.
The types of the branches must be the same. The \code{typeCheck} function
computes the types of \code{t} and \code{f}, and the \code{mustSame} function
compares them. If they are the same, then the type is the type of the whole expression.

\begin{verbatim}
case Rec(f, x, p, r, b, e) =>
  val t = ArrowT(p, r)
  val nenv = env + (f -> t)
  mustSame(typeCheck(b, nenv + (x -> p)), r)
  typeCheck(e, nenv)
\end{verbatim}

The parameter type is \code{p}, and the return type is \code{r}. Thus, the type
of \code{f} is the function type from \code{p} to \code{r}. The type of \code{x}
is \code{p}. To type-check \code{b}, the type environment must have the types of
\code{f} and \code{x}. The type of \code{b} must equal \code{r}. The
\code{mustSame} function compares the types. The function can be used not only in
\code{b}, which is the body of the function, but also in \code{e}. On the other
hand, the parameter \code{x} cannot be used in \code{e}. Therefore, it is enough
to add only the type of \code{f} to the type environment used to type-check
\code{e}. The type of the whole expression is equal to the type of \code{e}.

\section{Exercises}

\begin{exercise}
\labex{typing-recursive-functions-terminate}

Write a \lang expression $e$ such that only one of $e$ and
  $\efunt{\cx}{\tnum}{(\eapp{e}{\cx})}$ terminates, while both $e$ and
  $\efunt{\cx}{\tnum}{(\eapp{e}{\cx})}$ are well-typed.

\end{exercise}

\begin{exercise}

Consider the following language:
\[
\begin{array}{rllrllrll}
  e::= & n &\qquad& c::= & \textsf{skip} &\qquad& b::= & \textsf{true} \\
  |& b &&|& \eset{x}{e} &&|& \textsf{false} \\
  |& x &&|& \eif{e}{c}{c} && v::= & n \\
  |& e+e &&|& \textsf{while}\ e\ c &&|& b \\
  |& e<e &&|& \eseq{c}{c} &&\tau::= & \tnum \\
  &&&&&&|& \textsf{bool} \\
\end{array}
\]

The metavariable $c$ ranges over commands.

Under a given environment $\sigma$, evaluation of an expression yields a value and does not
change $\sigma$. The following is the operational semantics of the expressions:
    \vspace{-1em}
\[
  \begin{array}{c}
  \sigma\vdash n\Rightarrow n \qquad
  \sigma\vdash b\Rightarrow b \qquad
  \inferrule
  { x\in\dom{\sigma} }
  { \sigma\vdash x\Rightarrow \sigma(x) }
  \\[20pt]
  \inferrule
  { \sigma\vdash e_1\Rightarrow n_1 \\
    \sigma\vdash e_2\Rightarrow n_2 }
  { \sigma\vdash e_1+e_2\Rightarrow n_1+n_2 }
  \qquad
  \inferrule
  { \sigma\vdash e_1\Rightarrow n_1 \\
    \sigma\vdash e_2\Rightarrow n_2 }
  { \sigma\vdash e_1<e_2\Rightarrow n_1<n_2 }
  \end{array}
\]

\begin{enumerate}
  \item Write the typing rules of the form \fbox{$\typeofd{e}{\tau}$}.
\end{enumerate}

Evaluation of a command produces a new environment.
The following is the operational semantics of the commands:
    \vspace{-1em}
\[
  \begin{array}{c}
    \sigma\vdash \textsf{skip}\Rightarrow \sigma
    \quad
    \inferrule
    { \sigma\vdash e\Rightarrow v }
    { \sigma\vdash \eset{x}{e}\Rightarrow \sigma[x\mapsto v] }
    \\[20pt]
    \inferrule
    { \sigma\vdash e\Rightarrow \textsf{true} \\
      \sigma\vdash c_1\Rightarrow \sigma_1 }
    { \sigma\vdash \eif{e}{c_1}{c_2}\Rightarrow \sigma_1 }
    \quad
    \inferrule
    { \sigma\vdash e\Rightarrow \textsf{false} \\
      \sigma\vdash c_2\Rightarrow \sigma_1 }
    { \sigma\vdash \eif{e}{c_1}{c_2}\Rightarrow \sigma_1 }
    \\[20pt]
    \inferrule
    { \sigma\vdash e\Rightarrow \textsf{true} \\
      \sigma\vdash c\Rightarrow \sigma_1 \\
      \sigma_1\vdash \textsf{while}\ e\ c\Rightarrow \sigma_2 }
    { \sigma\vdash \textsf{while}\ e\ c\Rightarrow \sigma_2 }
    \\[20pt]
    \inferrule
    { \sigma\vdash e\Rightarrow \textsf{false} }
    { \sigma\vdash \textsf{while}\ e\ c\Rightarrow \sigma }
    \quad
    \inferrule
    { \sigma\vdash c_1\Rightarrow \sigma_1 \\
      \sigma_1\vdash c_2\Rightarrow \sigma_2 }
    { \sigma\vdash \eseq{c_1}{c_2}\Rightarrow \sigma_2 }
  \end{array}
\]

\begin{enumerate}
    \setcounter{enumi}{1}
\item Write the typing rules of the form \fbox{$\typeofd{c}{\Gamma}$}.
The following command must be well-typed: \verb+x := 1; x := 2+.
However, the following command must be ill-typed: \verb+x := 1; x := true+.
\end{enumerate}

\end{exercise}
