\setchapterpreamble[u]{\margintoc}
\chapter{Introduction to Functional Programming}
\labch{intro-to-fp}

\section{The Definition of Functional Programming}

What is functional programming? English Wikipedia says the following:

> Functional programming is a programming paradigm that treats computation as the
evaluation of mathematical functions and avoids changing-state and mutable data.

According to the book "Functional Programming in Scala":

> Functional programming (FP) is based on a simple premise with far-reaching
implications: we construct our programs using only pure functions—in other words,
functions that have no side effects.

It is the first sentence of the first chapter.

The above two sentences are enough to describe functional programming. Firstly,
consider the phrase 'the evaluation of mathematical functions.' In the last
article, I mentioned that everything is an expression in Scala and an expression
is evaluated into a value. In the perspective of functional programming, a
program is a single (mathematical) expression and the execution of the program is
finding a value denoted by the expression. The following code shows how
functional programming is different from imperative programming.

\begin{verbatim}
int x = 1;
int y = 2;
if (y < 3)
    x = x + 4;
else
    x = x - 5;
\end{verbatim}

\begin{verbatim}
let x = 1 in
let y = 2 in
if y < 3 then x + 4 else x - 5
\end{verbatim}

The first code is written in C, which represents imperative languages. Imperative
programming mimics a way that computers operate in. During the execution of a
program, a \term{state}, which can be interpreted as a memory of a computer,
exists and the execution modifies the state. The execution of the C program has
the following steps:

\begin{enumerate}
\item A state that both \verb+x+ and \verb+y+ are uninitialized
\item A state that \verb+x+ is \verb+1+ and \verb+y+ is uninitialized
\item A state that \verb+x+ is \verb+1+ and \verb+y+ is 2
\item Since \verb+y < 3+ is \verb+true+ under the state of the third step, go to
the next line.
\item A state that \verb+x+ is \verb+5+ and \verb+y+ is 2
\end{enumerate}

The second code is written in OCaml, which represents functional languages. A
program is an expression and the result of the execution is the result of
evaluating the expression. The execution does not require a notion of a state.
The execution of the OCaml program has the following steps:

\begin{enumerate}
\item Given a fact that \verb+x+ equals \verb+1+, evaluate
\verb!let y = 2 in if y < 3 then x + 4 else x – 5!.
\item Given a fact that \verb!x! equals \verb!1! and \verb!y! equals \verb!2!,
evaluate \verb!if y < 3 then x + 4 else x – 5!.
\item Given a fact that \verb!x! equals \verb!1! and \verb!y! equals \verb!2!,
evaluating \verb!y < 3! yields \verb!true! and thus evaluate \verb!x + 4!.
\item Given a fact that \verb!x! equals \verb!1! and \verb!y! equals \verb!2!,
evaluate \verb!x + 4!.
\item The result is \verb!5!.
\end{enumerate}

Since the programs are simple, two programs look similar but it is important to
understand two different perspectives of what a program is.

Secondly, look at the phrases 'avoids changing-state and mutable data' and 'using
only pure functions.' The last article said that functional programming avoids
mutable variables. Functional programming does not modify data including
variables and objects. States, which change throughout the execution of programs,
do not exist. Due to the lack of states, a function always does the same stuff
and for given the same argument, always returns the same value. Such functions
are called \term{pure functions}.

In practice, especially for large-scale projects, using only immutable things in
the whole code is usually inefficient. Most real-world functional languages
provide mutable variables or structures like \verb!var! of Scala, \verb!ref! of
OCaml, and \verb!set!! and \verb!box! of Racket. However, during the course, you
will see that functional programming uses immutable things in most cases but can
still express a lot of programs without difficulties. I discuss the advantages of
immutability in a later section of the article.

\section{Functional Programming in Industry}

Before going deeply into functional programming, let us find how people use
functional programming in industry.

\subsection{OCaml}

Facebook has developed Infer\sidenote{\url{https://fbinfer.com/}}, a static analyzer for Java,
C, C++, and Objective-C, in OCaml. Facebook and other companies including Amazon
and Mozila use Infer to find bugs statically in their programs. Facebook has
developed also Flow\sidenote{\url{https://flow.org/}}, a static type checker for JavaScript.

Jane Street\sidenote{\url{https://www.janestreet.com/}} is a financial company well-known in
the PL community and has developed its own software in OCaml.

According to the official OCaml web
site\sidenote{\url{http://ocaml.org/learn/companies.html}}, various companies including Docker
use OCaml.

\subsection{Haskell}

Haskell Wiki\sidenote{\url{http://wiki.haskell.org/Haskell_in_industry}} mentions that Google,
Facebook, Microsoft, Nvidia, and many others use Haskell.

\subsection{Erlang and Elixir}

Erlang is a functional language for concurrent and parallel computing. Elixir
operates on Erlang virtual machines and is used for the same purpose as Erlang.
An article from Code
Sync\sidenote{\url{https://codesync.global/media/successful-companies-using-elixir-and-erlang/}}
introduced companies including WhatsApp, Pinterest, and Goldman Sachs using
Erlang and Elixir.

Programs whose input has complicated and abstract structures like source code and
programs that have to be trustworthy or targets concurrent and parallel computing
are typically written in functional languages.

\section{The Advantages of Immutability}

The advantages of immutability per se become the advantages of functional
programming. The book "Programming in Scala" discusses four strength of
immutability:

> First, immutable objects are often easier to reason about than mutable ones,
because they do not have complex state spaces that change over time. Second, you
can pass immutable objects around quite freely, whereas you may need to make
defensive copies of mutable objects before passing them to other code. Third,
there is no way for two threads concurrently accessing an immutable to corrupt
its state once it has been properly constructed, because no thread can change the
state of an immutable. Fourth, immutable objects make safe hash table keys. If a
mutable object is mutated after it is placed into a \verb!HashSet!, for example,
that object may not be found the next time you look into the \verb!HashSet!.

\subsection{Is Easy to Reason about}

\begin{verbatim}
val x = 1
...
f(x)
\end{verbatim}

At the first line of the code, \verb!x! is \verb!1!. Since \verb!x! is immutable,
there is no doubt that \verb!x! is still \verb!1! when \verb!x! is passed as an
argument for \verb!f! at the last line of the code.

\begin{verbatim}
var x = 1
...
f(x)
\end{verbatim}

On the other hand, if \verb!x! is a mutable variable, one should read every line
of code in the middle to find the value of \verb!x! at the time when the function
call happens.

For small programs written by one person, whether \verb!x! is immutable or
mutable is unimportant. If the program does not expect any future usages, mutable
\verb!x! is fine. However, suppose the situation reading the code without any
prior knowledge about the code. A big difference between immutable and mutable
\verb!x! exists even though there is no person who reads only the first and the
last line of course. When \verb!x! is mutable, without tracking every
modification of \verb!x! throughout the code, the value of \verb!x! at the last
line is unknown. It makes programmers feel difficulties to understand the code so
possibly leads to more bugs. The program with immutable \verb!x! does not suffer
from such a problem. Remembering only one line of the code is enough to track the
value of \verb!x!.

Mutable data structures cause similar problems.

\begin{verbatim}
val x = List(1, 2)
...
f(x)
...
x
\end{verbatim}

\verb!List! is an immutable data structure came from the Scala standard library.
\verb!x! is always a list containing \verb!1! and \verb!2!.

\begin{verbatim}
import scala.collection.mutable.ListBuffer
val x = ListBuffer(1, 2)
...
f(x)
...
x
\end{verbatim}

On the other hand, \verb!ListBuffer! is a mutable data structure in the Scala
standard library. It is possible to add an item to or to remove an item from a
list referred by \verb!x!. Programmers are not sure about the content of \verb!x!
unless they read all the lines in between. Besides, function \verb!f! also is
able to change the content of \verb!x!. If one writes a program with a wrong
assumption that \verb!f! does not modify \verb!x!, then the program might be
buggy.

Mutable global variables make code much harder to understand than mutable local
variables.

\begin{verbatim}
def f(x: Int) = g(x, y)
\end{verbatim}

The return value of function \verb!f! depends on the value of global variable
\verb!y!. If \verb!y! is mutable, \verb!f! is not a pure function and expecting
the behavior of \verb!f! is nontrivial. \verb!y! can be declared in any arbitrary
file and all files are able to access \verb!y! and to change the value of
\verb!y!. In the worst case, an external library defines \verb!y! and source code
modifying \verb!y! is not available for reading.

The examples are small and seem artificial but immutability greatly improves
maintainability and readability of code in practice, especially for large
projects.

\subsection{Does not Need Defensive Copies}

\begin{verbatim}
val x = ListBuffer(1, 2)
...
f(x)
...
x
\end{verbatim}

Since \verb!ListBuffer! creates mutable lists, there is no guarantee that the
content of \verb!x! does not change by \verb!f!. If it is necessary to prevent
modification, copying \verb!x! is essential.

\begin{verbatim}
val x = ListBuffer(1, 2)
val y = x.clone
...
f(y)
...
x
\end{verbatim}

In cases that \verb!x! has many elements and the code is executed multiple times,
copying \verb!x! increases the execution time significantly.

In the code, using the \verb!clone! method is enough to copy the list because the
list contains only integral values. However, to pass lists containing mutable
objects safely to functions, defining additional methods for deep copy is
inevitable.

\subsection{Guarantees Safe Accesses from Multiple Threads}

\begin{verbatim}
val x = ListBuffer(1, 2)
def a() = f(x)
def b() = g(x)
\end{verbatim}

Consider two different threads calling functions \verb!a! and \verb!b!,
respectively. Functions \verb!f! and \verb!g! are arbitrarily overlapped so that
each execution can yield a different result from the result of each other. Some
results might be unwanted. Even worse, the \verb!ListBuffer! collection is unsafe
for concurrent accesses from multiple threads and therefore the program perhaps
crashes during execution.

By using \verb!List! instead of \verb!ListBuffer!, the program becomes safe.
Functions \verb!f! and \verb!g! can only read the content of \verb!x! and cannot
change the content. For any possible execution orders, the program behaves
equally and is safe to call functions \verb!a! and \verb!b! from two different
threads.

\subsection{Creates Safe Hash Values}

The hash value of an object is a key to find the object in hash tables. For the
purpose, the same object must have the same hash value all the time. However, the
hash value of a mutable object changes when the object is modified. It is
problematic when using data structures using hash tables like sets and maps.

\begin{verbatim}
val x = ListBuffer(0)
val y = Set(x, ...)
val z = Map(x -> 0, ...)

y.contains(x)
z(x)

x += 1

y.contains(x)
z(x)
\end{verbatim}

The former \verb!y.contains(x)! and \verb!z(x)! yield \verb!true! and \verb!0!,
respectively. After appending \verb!1! to \verb!x!, the hash value of \verb!x!
changes and therefore \verb!y.contains! results in \verb!false! and \verb!z(x)!
throws an exception named \verb!NoSuchElementException!. (Note that \verb!Set!
and \verb!Map! of the Scala standard library use hash tables only when they
contain more than four elements so that the explanation holds only for \verb!y!
and \verb!z! with more than four elements.) Using mutable objects as hash keys is
dangerous while immutable objects maintain the same hash values throughout
execution.

Immutability has several clear advantages. Immutability is important in
functional programming. Functional programs use immutable variables and objects
in most cases. However, mind that immutability is not the silver bullet for every
program. For example, implementing algorithms like sorts in a functional style is
highly inefficient. Use mutable data structures like arrays, mutable variables,
and loops like \verb!for! and \verb!while! to implement algorithms. It makes
programs much more readable and faster. Choosing a proper programming paradigm to
the purpose of a program is the key to write good code.

\section{Recursion}

Repeating the same computation multiple times is a common pattern in programming.
Loops allow concise code expressing such cases. However, if everything is
immutable, going back to the beginnings of loops does not change any states and
therefore it is impossible to apply the same operation on different values for
each iteration or to terminate the loops. As a consequence, loops are useless in
functional programming. Functional programs use recursive functions instead of
loops to rerun computation. A recursive function is a function that calls itself.
To do more computation, the function calls itself with proper arguments.
Otherwise, it terminates with some return value.

The below \verb!factorial! functions calculate the factorial of a given integral
argument. For simplicity, assume that the functions return one for negative
integers. The former uses a loop and the latter uses \term{recursion}.

\begin{verbatim}
def factorial(n: Int) = {
  var i = 1, res = 1
  while (i <= n) {
    res *= i
    i += 1
  }
  res
}

def factorial(n: Int): Int = if (n <= 0) 1 else n * factorial(n - 1)
\end{verbatim}

In Scala, recursive functions require explicit return types.

Recursive functions usually reveal their mathematical definitions more clearly
then functions using loops.

\[n!=\begin{cases}1 & \text{if } n=0\\n \times (n-1)! &
\text{otherwise}\end{cases}\]

The implementation of the \verb!factorial! function using recursion is identical
to the mathematical definition of factorial. Recursion allows not only repetitive
computation but also concise and intuitive descriptions of mathematical
functions. Recursive functions are easier to be verified that they are correct
than imperative versions of the functions. Even if mathematical verification is
unnecessary, recursion is better for intuitive reasoning of functions than loops.
However, some functions become efficient when their implementation uses loops
rather than recursion. Selecting an appropriate implementation strategy is
crucial.

Recursion has disadvantages: overheads of function calls and \term{stack
overflow}. Most modern CPUs have enough computing power to ignore function call
overheads but loops are ideal for functions with short computation time in
programs whose performance is extremely important. Stack overflow happens when a
stack lacks space due to repetitive function calls. It is a critical problem
since it causes immediate termination of execution without yielding meaningful
output. Moreover, programs like web servers do not finish their execution so that
stack overflow must happen. To resolve the problem, many functional languages try
\term{tail call optimization} to prevent stack overflow. The last part of the
article deals with tail call optimization in detail.

\section{Lists}

Almost all programs use lists. Understanding a way to define and to treat lists
in functional programming is good practice. This section defines lists and
functions dealing with lists. The Scala standard library contains \verb!List! but
the section defines its own version of \verb!List!.

\subsection{The Definition of a List}

A list contains finite elements. An order exists among the elements and
duplicates might exist. Simply, a list is an enumeration of a finite number of
values. However, defining lists in functional programming requires a recursive
definition of a list. The following is the definition of a list:

A list is

\begin{itemize}
\item \verb!Nil!: the empty list or
\item \verb!Cons!: a pair of a value and a list.
\end{itemize}

(Lisp, its variants, and some other languages use name \verb!Nil! and
\verb!Cons!. To describe functional lists, the names are typically chosen.)

Any nonempty list has at least one element. The list is divided into the first
element and the remaining elements. The first element is \term{head} and the list
of the remaining elements is \term{tail}. The definition describes every list and
a thing satisfying the definition is a list. The following expresses a list
containing \verb!1!, \verb!2!, and \verb!3! using \verb!Nil! and \verb!Cons!:

\begin{verbatim}
Cons(1, Cons(2, Cons(3, Nil)))
\end{verbatim}

For simple implementation, assume lists contain only integral values as elements.
The following Scala code implements lists:

\begin{verbatim}
trait List
case object Nil extends List
case class Cons(head: Int, tail: List) extends List
\end{verbatim}

The \verb!List! type is a typical \term{algebraic data type}.

\verb!trait List! defines type \verb!List!.

\verb!case object Nil extends List! defines value \verb!Nil! and declares that
\verb!Nil! is a value of type \verb!List!. Keyword \verb!object! implies that
\verb!Nil! is a \term{singleton object}. The code is \term{syntactic sugar} of
the following:

\begin{verbatim}
class Nil$ extends List
val Nil = new Nil$
\end{verbatim}

A singleton object is a unique instance of a class. The class cannot have any
instance other than the singleton object. Since all empty lists are identical to
the single empty list, rather than defining \verb!Nil! as a class and creating
instances whenever the empty list appears, it is more efficient to define the
single \verb!Nil! instance and to refer to the instance for any appearance of the
empty list. Using single \verb!Nil! is safe because it is immutable.

\verb!case class Cons(head: Int, tail: List) extends List! defines a \verb!Cons!
class. Every instance of \verb!Cons! belongs to \verb!List!. An instance has two
fields \verb!head!, whose type is \verb!Int!, and \verb!tail!, whose type is
\verb!List!.

The following code constructs list values:

\begin{verbatim}
Nil
Cons(1, Nil)
Cons(1, Cons(2, Nil))
\end{verbatim}

Functional lists are \term{singly linked lists} in the perspective of data
structures. Accessing the next element is possible while accessing the previous
element is impossible.

\subsection{Functions for Lists}

An arbitrary list is either the empty list or a nonempty list. Programmers use
\term{pattern matching} to split into the two cases, \verb!Nil! and \verb!Cons!.

\begin{verbatim}
def f(l: List) = l match {
  case Nil => "The empty list"
  case Cons(h, t) => "A pair of an integer and a list"
}
\end{verbatim}

Scala requires form \verb![expression] match { case [pattern] => [expression] ... }!
for pattern matching. Firstly, the first expression is evaluated. The result
is compared to the patterns. The result of the whole expression is the result of
an expression corresponding to the first matching pattern. If \verb!l! is
\verb!Nil!, the return value is \verb!"The empty list"!. Otherwise, the return
value is \verb!"A pair of an integer and a list"!. \verb!h! and \verb!t!
respectively refer to the head and tail of \verb!l!. The expression corresponding
to the \verb!Cons! pattern may use \verb!h! and \verb!t!.

The \verb!inc1! function takes a list as an argument and returns a list whose
elements are one larger than the elements of the given list.

\begin{verbatim}
def inc1(l: List): List = l match {
  case Nil => Nil
  case Cons(h, t) => Cons(h + 1, inc1(t))
}
\end{verbatim}

For the given empty list, the function returns the empty list. Otherwise, the
return value is a list whose head is one larger than the head of the given list
and tail has elements that are one larger than the elements of the tail of the
given list.

Define function \verb!square!, which takes a list as an argument and returns a
list whose elements are the squares of the elements of the given list. Check the
below code after trying to find an answer.

\begin{verbatim}
def square(l: List): List = l match {
  case Nil => Nil
  case Cons(h, t) => Cons(h * h, square(t))
}
\end{verbatim}

The \verb!odd! function takes a list as an argument and returns a list whose
every element is an odd integer.

\begin{verbatim}
def odd(l: List): List = l match {
  case Nil => Nil
  case Cons(h, t) =>
    if (h \% 2 != 0) Cons(h, odd(t))
    else odd(t)
}
\end{verbatim}

For a nonempty list, the function checks whether the head is odd or not. If the
head is odd, the resulting list contains the head and its tail is the tail with
only odd integers. Otherwise, the head is removed.

Define function \verb!positive!, which takes a list as an argument and returns a
list whose every element is a positive integer.

\begin{verbatim}
def positive(l: List): List = l match {
  case Nil => Nil
  case Cons(h, t) =>
    if (h > 0) Cons(h, positive(t))
    else positive(t)
}
\end{verbatim}

The \verb!length! function calculates the length of a given list.

\begin{verbatim}
def length(l: List): Int = l match {
  case Nil => 0
  case Cons(h, t) => 1 + length(t)
}
\end{verbatim}

The length of the empty list is zero. The length of a nonempty list is one larger
than the length of its tail.

Define functions \verb!sum! and \verb!product!, which respectively calculate the
sum and the product of the elements of a given list.

\begin{verbatim}
def sum(l: List): Int = l match {
  case Nil => 0
  case Cons(h, t) => h + sum(t)
}
\end{verbatim}

\begin{verbatim}
def product(l: List): Int = l match {
  case Nil => 1
  case Cons(h, t) => h * product(t)
}
\end{verbatim}

Define function \verb!addBack!, which takes a list and an integer as arguments
and return a list obtained by appending the integer at the end of the list.

\begin{verbatim}
def addBack(l: List, n: Int): List = l match {
  case Nil => Cons(n, Nil)
  case Cons(h, t) => Cons(h, addBack(t, n))
}
\end{verbatim}

Adding an element at the end of a list requires \term{time complexity} of
\(O(n)\). \term{Space complexity} also is \(O(n)\) since the function creates a
new entire list. In contrast, prepending an element at the beginning of a list
requires both time and space complexity of \(O(1)\). Therefore, adding an element
at the beginning instead of the end of a list is desirable in functional
programming.

\section{Tail Call Optimization}

If the last action of a function is calling a function, then the call is a tail
call. When a tail call happens, after the call, the \term{callee} does every
computation and thus the local variables of the \term{caller} have no need to
remain. The stack frame of the caller can be destroyed. Most functional languages
optimize tail calls. At compile time, compilers check whether calls are tail
calls. If a call is a tail call, the compilers generate code that eliminates the
stack frame of the caller before the call. They do not optimize non-tail function
calls because the local variables of the callers can be used after returning from
the callees. If every function call in a program is a tail call, a stack never
grows so that the program is safe from stack overflow.

\begin{verbatim}
def factorial(n: Int): Int = if (n <= 0) 1 else n * factorial(n - 1)
\end{verbatim}

The previous \verb!factorial! function multiplies \verb!n! and the return value
of the recursive \verb!factorial(n -1)! call. The multiplication is the last
action. The recursive call is not a tail call. The stack frame of the caller must
remain. The following process computes \verb!factorial(3)!:

\begin{itemize}
\item \verb!factorial(3)!
\item \verb!3 * factorial(2)!
\item \verb!3 * (2 * factorial(1))!
\item \verb!3 * (2 * (1 * factorial(0)))!
\item \verb!3 * (2 * (1 * 1))!
\item \verb!3 * (2 * 1)!
\item \verb!3 * 2!
\item \verb!6!
\end{itemize}

At most four stack frames coexist. For a large enough argument, a stack grows
again and again and finally overflows.

\begin{verbatim}
scala> def factorial(n: Int): Int = if (n <= 0) 1 else n * factorial(n - 1)
factorial: (n: Int)Int

scala> factorial(10000)
java.lang.StackOverflowError
  at .factorial(<console>:12)
\end{verbatim}

To implement the function using a tail call, instead of multiplying \verb!n! and
\verb!factorial(n - 1)!, the function has to pass both \verb!n! and \verb!n - 1!
as arguments and to make the callee multiply them. One can interpret the strategy
as passing an intermediate result.

\begin{itemize}
\item \verb!factorial(3)!
\item \verb!factorial(2, intermediate result = 3)!
\item \verb!factorial(1, intermediate result = 3 * 2)!
\item \verb!factorial(1, intermediate result = 6)!
\item \verb!factorial(0, intermediate result = 6 * 1)!
\item \verb!factorial(0, intermediate result = 6)!
\item \verb!6!
\end{itemize}

There is no need to return to the caller. The below code shows the
\verb!factorial! function with a tail call. The function needs one more parameter
that takes an intermediate result. \verb!factorial(n, i)! computes \(n!\times
i\).

\begin{verbatim}
def factorial(n: Int, inter: Int): Int =
  if (n <= 0) inter else factorial(n - 1, inter * n)
\end{verbatim}

The function uses the tail call. More precisely, the function is
\term{tail-recursive}. Its last action is calling itself. Unlike most functional
languages, Scala cannot optimize general tail calls. Scala optimizes only
tail-recursive calls.

Scala compilers generate Java bytecode, whom JVMs execute. The JVMs does not
allow to jump to the beginning of another function. In the JVMs, functions can
only either return or call functions. The Scala compilers cannot generate
optimized code by removing the stack frame of the caller. Instead, they transform
tail-recursive calls into loops. \verb!javap! disassembles Java bytecode files.
The compiled and disassembled \verb!factorial! function using tail recursion does
not call any function but jumps to instructions inside the function.

\begin{verbatim}
public int factorial(int, int);
  Code:
     0: iload_1
     1: iconst_0
     2: if_icmpgt     9
     5: iload_2
     6: goto          20
     9: iload_1
    10: iconst_1
    11: isub
    12: iload_2
    13: iload_1
    14: imul
    15: istore_2
    16: istore_1
    17: goto          0
    20: ireturn
\end{verbatim}

The function does not result in stack overflow.

(The result is still abnormal due to integer overflow. The \verb!BigInt! type
resolves it.)

The transformation not only prevents stack overflow but also removes the
overheads of function calls. The downside is that \term{mutually recursive
functions} using tail calls lie beyond the scope of the transformation. The below
functions may cause stack overflow.

\begin{verbatim}
def even(n: Int): Boolean = if (n <= 0) true else odd(n - 1)
def odd(n: Int): Boolean = if (n == 1) true else even(n - 1)
\end{verbatim}

In Scala, by using \term{annotations}, programmers ask the compilers to check
whether functions are tail-recursive. The annotations prevent non-tail-recursive
function made by mistakes.

\begin{verbatim}
import scala.annotation.tailrec
@tailrec def factorial(n: Int, inter: Int): Int =
  if (n <= 0) inter else factorial(n - 1, inter * n)
\end{verbatim}

A non-tail-recursive function with the \verb!tailrec! annotation results in a
compile error. The annotation does not affect the behaviors of the compilers.
Regardless of the existence of the annotation, the compilers always optimize
tail-recursive functions. Still, using the annotations is desirable to prevent
mistakes.

Calling the tail-recursive version of \verb!factorial! needs the unnecessary
second argument. The below code defines a new \verb!factorial! function with one
parameter and uses the tail-recursive one as a local function inside the
function.

\begin{verbatim}
def factorial(n: Int): Int = {
  @tailrec def aux(n: Int, inter: Int): Int =
    if (n <= 0) inter else aux(n - 1, inter * n)
  aux(n, 1)
}
\end{verbatim}

By revising \verb!length!, it becomes tail-recursive:

\begin{verbatim}
def length(l: List): Int = {
  @tailrec def aux(l: List, inter: Int): Int = l match {
    case Nil => inter
    case Cons(h, t) => aux(t, inter + 1)
  }
  aux(l, 0)
}
\end{verbatim}

The \verb!sum! and \verb!product! functions can be modified in the same manner.

\begin{verbatim}
def sum(l: List): Int = {
  @tailrec def aux(l: List, inter: Int): Int = l match {
    case Nil => inter
    case Cons(h, t) => aux(t, inter + h)
  }
  aux(l, 0)
}
\end{verbatim}

\begin{verbatim}
def product(l: List): Int = l match {
  @tailrec def aux(l: List, inter: Int): Int = l match {
    case Nil => inter
    case Cons(h, t) => aux(t, inter * h)
  }
  aux(l, 1)
}
\end{verbatim}

The following is tail-recursive \verb!inc1!.

\begin{verbatim}
def inc1(l: List): List = {
  @tailrec def aux(l: List, inter: List): List = l match {
    case Nil => inter
    case Cons(h, t) => aux(t, addBack(inter, h + 1))
  }
  aux(l, Nil)
}
\end{verbatim}

Its result is correct but takes \(O(n^2)\) time while the original version
requires only \(O(n)\).

It is possible to implement tail-recursive \verb!inc1! using \(O(n)\) time.

\begin{verbatim}
def reverse(l: List): List = {
  @tailrec def aux(l: List, inter: List): List = l match {
    case Nil => inter
    case Cons(h, t) => aux(t, Cons(h, inter))
  }
  aux(l, Nil)
}

def inc1(l: List): List = {
  @tailrec def aux(l: List, inter: List): List = l match {
    case Nil => inter
    case Cons(h, t) => aux(t, Cons(h + 1, inter))
  }
  reverse(aux(l, Nil))
}
\end{verbatim}
