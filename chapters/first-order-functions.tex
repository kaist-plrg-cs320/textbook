\setchapterpreamble[u]{\margintoc}
\chapter{First-Order Functions}
\labch{first-order-functions}

\renewcommand{\plang}{\textsf{VAE}\xspace}
\renewcommand{\lang}{\textsf{F1VAE}\xspace}

A function is one of the most important concepts in programming languages.
It is the key feature of functional languages, as the term functional
implies. Even in imperative languages, functions are important and widely-used.
This chapter focuses on first-order functions. \textit{First-order functions}
\index{first-order function} are functions that cannot take or return functions.
They are much restrictive than first-class functions but still very useful.

Consider the following Scala program:

\begin{verbatim}
def twice(x: Int): Int = x + x

println(twice(3) + twice(5))
\end{verbatim}

It defines a function, \code{twice}. The function takes one argument and
returns twice of the argument. The program can call the function whenever we want.
\code{twice(3)} passes \code{3} as an argument to \code{twice}. Its result is
\code{6}, which is twice of \code{3}. Similarly, \code{twice(5)} results in
\code{10}. Therefore, the program prints \code{16}.

This chapter defines \lang by adding first-order functions to \plang.
Every function in \lang is top-level. It means that a function definition cannot
be a part of an expression. We assume that a \lang program is a single expression
that is evaluated under an environment and a list of function definitions. This
design prevents us from exploring interesting topics like closures but enables us
to focus on the semantics of function calls. The next chapter will introduce
first-class functions and closures, which make functions more expressive.

\section{Syntax}

We can figure out the components of a function definition from the above
example. If we ignore the type annotations, the definition consists of three
parts: \code{twice}, \code{x}, and \code{x + x}. \code{twice} is the name of the
function; \code{x} is the parameter of the function; \code{x + x} is the body
of the function. Therefore, we can define the syntax of a function definition as
follows:

\[ d\ ::=\ \fundef{x}{x}{e} \]

The metavariable $d$ ranges over function definitions. Let $\embox{FunDef}$
denote the set of every function definition. A function definition
$\fundef{x_1}{x_2}{e}$ defines a function whose name is $x_1$, parameter is
$x_2$, and body is $e$. Both $x_1$ and $x_2$ are binding occurrences.
The scope of $x_1$ is the entire program; the scope of $x_2$ is $e$.
In many real-world languages, a function has zero or
more parameters. However, our syntax restricts a function to have one and only
one parameter. We adopt this design to make the semantics simple. Once you
understand a function with a single parameter, you can easily extend the concept
to a function with multiple parameters.

Using a function is to call the function. If we never call a function, the
function is useless. We need to add a new kind of expression to the language:
the function call expression. The following is the syntax of \lang:
\sidenote{We omit the common part to \plang.}

\[ e\ ::=\ \cdots\ |\ \eappfo{x}{e} \]

$x(e)$ is the function call expression. It calls a function named $x$. $e$
determines the value of the argument of the call. Here, $x$ is a bound
occurrence. A function call always has only one argument since every function in
\lang has only one parameter.

\section{Semantics}

Like that we have introduced environments to store the values of variables,
we need a new semantic element that associates functions with their names.
Let us call it a function environment, which is a finite partial function from
identifiers to function definitions.

\[\embox{FEnv}=\embox{Id}\finto\embox{FunDef}\]
\[\Lambda\in\embox{FEnv}\]

The metavariable $\Lambda$ ranges over function environments.

Evaluation of an expression requires not only an environment but also a function
environment to handle function calls properly.
Therefore, the semantics is a relation over $\embox{Env}$,
$\embox{FEnv}$, $E$, and $\mathbb{Z}$.

\[\Rightarrow\subseteq\embox{Env}\times\embox{FEnv}\times E\times\mathbb{Z}\]

$(\sigma,\Lambda,e,n)\in\Rightarrow$ is true if and only if $e$ evaluates to $n$
under $\sigma$ and $\Lambda$. We write $\eval{\sigma,\Lambda}{e}{n}$
instead of $(\sigma,\Lambda,e,n)\in\Rightarrow$.

The following rule describes the semantics of function calls:

\semanticrule{Call}{
\begin{tabular}{@{\hskip0pt}l@{\hskip0pt}l}
  If \\
  & $x$ is in the domain of $\Lambda$,\\
  & $\Lambda(x)$ is $\fundef{x}{x'}{e'}$,\\
  & \evaln{\sigma\tand\Lambda}{e}{n'}, and\\
  & \evaln{[x'\mapsto n']\tand\Lambda}{e'}{n},\\
  then \\
  & \evaln{\sigma\tand\Lambda}{\eappfo{x}{e}}{n}.
\end{tabular}
}

To evaluate $\eappfo{x}{e}$, we need to evaluate $e$ first to decide the value
of the argument. Then, we search for a function from the function environment
with a given function name, $x$. $x$ must be in the domain of the function
environment. Otherwise, $x$ is a free identifier, and a run-time error happens.
When $x$ is in the domain, we can get the corresponding function definition.
The function definition gives us the name of the parameter and the body. Since
every function is top-level, the body of each function does not belong to the
scope of any local variables. It can use no more than its own parameter. Thus,
the body should be evaluated under $[x'\mapsto n']$, not $\sigma[x'\mapsto n']$.
At the same time, since function calls do not affect function environments, the
same function environment is used for the evaluation of the body. The result of
the body is the result of the function call.

We can formulate the semantics as the following inference rule:

\[
  \inferrule
  {
    \eval{\sigma,\Lambda}{e}{n'} \\
    x\in\dom{\Lambda} \\
    \Lambda(x)=\fundef{x}{x'}{e'} \\
    \eval{[x'\mapsto n'],\Lambda}{e'}{n}
  }
  { \eval{\sigma,\Lambda}{\eappfo{x}{e}}{n} }
  \quad\textsc{[Call]}
\]

The other rules should be revised to consider function environments.
No expression modifies a function environment.

\semanticrule{Num}{
  \evaln{\sigma\tand\Lambda}{n}{n}.
}

\vspace{-1em}

\semanticrule{Add}{
If
  \evaln{\sigma\tand\Lambda}{e_1}{n_1}, and
  \evaln{\sigma\tand\Lambda}{e_2}{n_2},\\
then
  \evaln{\sigma\tand\Lambda}{\eadd{e_1}{e_2}}{n_1+n_2}.
}

\vspace{-1em}

\semanticrule{Sub}{
If
  \evaln{\sigma\tand\Lambda}{e_1}{n_1}, and
  \evaln{\sigma\tand\Lambda}{e_2}{n_2},\\
then
  \evaln{\sigma\tand\Lambda}{\esub{e_1}{e_2}}{n_1-n_2}.
}

\vspace{-1em}

\semanticrule{Val}{
If
  \evaln{\sigma\tand\Lambda}{e_1}{n_1}, and
  \evaln{\sigma[x\mapsto n_1]\tand\Lambda}{e_2}{n_2},\\
then
  \evaln{\sigma\tand\Lambda}{\ebind{x}{e_1}{e_2}}{n_2}.
}

\vspace{-1em}

\semanticrule{Id}{
If
  $x$ is in the domain of $\sigma$,\\
then
  \evaln{\sigma\tand\Lambda}{x}{\sigma(x)}.
}

We can fix the inference rules in a similar way.

\[
  \eval{\sigma,\Lambda}{n}{n}
  \quad\textsc{[Num]}
\]

\[
  \inferrule
  {
    \eval{\sigma,\Lambda}{e_1}{n_1} \\
    \eval{\sigma,\Lambda}{e_2}{n_2}
  }
  { \eval{\sigma,\Lambda}{\eadd{e_1}{e_2}}{n_1+n_2} }
  \quad\textsc{[Add]}
\]

\[
  \inferrule
  {
    \eval{\sigma,\Lambda}{e_1}{n_1} \\
    \eval{\sigma,\Lambda}{e_2}{n_2}
  }
  { \eval{\sigma,\Lambda}{\esub{e_1}{e_2}}{n_1-n_2} }
  \quad\textsc{[Sub]}
\]

\[
  \inferrule
  {
    \eval{\sigma,\Lambda}{e_1}{n_1} \\
    \eval{\sigma[x\mapsto n_1],\Lambda}{e_2}{n_2}
  }
  { \eval{\sigma,\Lambda}{\ebind{x}{e_1}{e_2}}{n_2} }
  \quad\textsc{[Val]}
\]

\[
  \inferrule
  { x\in\dom{\sigma} }
  { \eval{\sigma,\Lambda}{x}{\sigma(x)} }
  \quad\textsc{[Id]}
\]

\section{Interpreter}

The following Scala code expresses the abstract syntax of \lang:
\sidenote{We omit the common part to \plang.}

\begin{verbatim}
case class FunDef(f: String, x: String, b: Expr)

sealed trait Expr
...
case class Call(f: String, a: Expr) extends Expr
\end{verbatim}

Just like environments, function environments can be expressed as maps.
The type of a function environment is \code{Map[String, FunDef]} as it maps an
identifier to a function definition.

\begin{verbatim}
type FEnv = Map[String, FunDef]
\end{verbatim}

The following function evaluates a given expression under a given environment
and a given function environment.

\begin{verbatim}
def interp(e: Expr, env: Env, fEnv: FEnv): Int = e match {
  case Num(n) => n
  case Add(l, r) =>
    interp(l, env, fEnv) + interp(r, env, fEnv)
  case Sub(l, r) =>
    interp(l, env, fEnv) - interp(r, env, fEnv)
  case Val(x, i, b) =>
    interp(b, env + (x -> interp(i, env, fEnv)), fEnv)
  case Id(x) => env(x)
  case Call(f, a) =>
    val FunDef(_, x, e) = fEnv(f)
    interp(e, Map(x -> interp(a, env, fEnv)), fEnv)
}
\end{verbatim}

The implementation reflects the semantics exactly. You can easily check its
correctness with the case-wise comparison.

\section{Scope}

The current semantics is called static scope. Static scope allows the scope of a
binding occurrence to be determined statically, i.e. only by looking the code,
without executing it. In other words, a function body can use only variables
that have been defined already when the function is defined.
For example, consider the following code:

\[
  \fundef{\code{f}}{\cx}{\eadd{\cx}{\cy}}
\]

Since every function is top-level, while variables are local in \lang, \code{y}
does not belong to the scope of any binding occurrence of \code{y}. No variable
can be defined before the function. Therefore,
\code{y} is a free variable, and calling the function \code{f} must incur a
run-time error. It is true under the current semantics. The current function
call semantics evaluates the body of a function under the environment that has
only the value of the parameter. The environments at function call-sites never
affect the environment used for the evaluation of the body.

The opposite of static scope is dynamic scope, which makes every information in
the environment at each call-site available to the function body. The behavior
of a function depends on not only its argument but also its call-site.
An identifier in the body of a function becomes associated with a different
entity for each function call. It implies that the scope of a binding identifier
cannot be determined statically; it is determined at run time, i.e. dynamically.

For example, the following expression evaluates to $3$ when we assume the same
definition of \code{f} as before:

\[
  \eadd{(\ebind{\cy}{1}{\eappfo{\code{f}}{0})}}{(\ebind{\cy}{2}{\eappfo{\code{f}}{0})}}
\]

During the first function call, \code{y} in \code{f} is bound to the first
\code{y} and denotes $1$. However, during the second function call, it is
bound to the second one and denotes $2$. The scope of the first \code{y}
includes not only $\eappfo{\code{f}}{0}$, which is normal, but also the body of
\code{f}. It is the same for the second \code{y}. As you can see, under dynamic
scope, the scope of a binding identifier is not fixed; it becomes extended at
run time due to function calls.

To adopt dynamic scope to \lang, we need to change the function call semantics
as follows:

\semanticrule{Call-Dyn}{
\begin{tabular}{@{\hskip0pt}l@{\hskip0pt}l}
  If \\
  & $x$ is in the domain of $\Lambda$,\\
  & $\Lambda(x)$ is $\fundef{x}{x'}{e'}$,\\
  & \evaln{\sigma\tand\Lambda}{e}{n'}, and\\
  & \evaln{\sigma[x'\mapsto n']\tand\Lambda}{e'}{n},\\
  then \\
  & \evaln{\sigma\tand\Lambda}{\eappfo{x}{e}}{n}.
\end{tabular}
}

It is equivalent to the following inference rule:

\[
  \inferrule
  {
    \eval{\sigma,\Lambda}{e}{n'} \\
    x\in\dom{\Lambda} \\
    \Lambda(x)=\fundef{x}{x'}{e'} \\
    \eval{\sigma[x'\mapsto n'],\Lambda}{e'}{n}
  }
  { \eval{\sigma,\Lambda}{\eappfo{x}{e}}{n} }
  \quad\textsc{[Call-Dyn]}
\]

The interpreter can be fixed like below.

\begin{verbatim}
case Call(f, a) =>
  val FunDef(_, x, e) = fEnv(f)
  interp(e, env + (x -> interp(a, env, fEnv)), fEnv)
\end{verbatim}

Dynamic scope prevents programs from being modular. The environment at each call-site
affects the behavior of a function. It hinders programmers from reasoning about
the semantics of a function based on the definition. They need to additionally consider
every possible call-site. It implies that different parts of a program unexpectedly
interfere with each other. Therefore, dynamic scope makes programs error-prone.
Because of the harmfulness of dynamic scope, most modern languages adopt static
scope.

\section{Exercises}

\begin{exercise}
\labex{first-order-functions-eval}

With the following list of function definitions in \lang:

$
\begin{array}{l}
  \fundef{\code{twice}}{\cx}{\eadd{\cx}{\cx}} \\
  \fundef{\code{x}}{\code{y}}{\code{y}} \\
  \fundef{\code{f}}{\code{x}}{\eadd{\cx}{1}} \\
  \fundef{\code{g}}{\code{g}}{\code{g}} \\
\end{array}
$

Show the results of evaluating the following expressions under the empty environment.
When it is an error, describe which error it is.
\begin{enumerate}
  \item $\eappfo{\code{twice}}{\code{twice}}$
  \item $\ebind{\cx}{5}{\eappfo{\cx}{\cx}}$
  \item $\eappfo{\code{g}}{3}$
  \item $\eappfo{\code{g}}{\code{f}}$
  \item $\eappfo{\code{g}}{\code{g}}$
\end{enumerate}

\end{exercise}
