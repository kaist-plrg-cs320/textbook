\setchapterpreamble[u]{\margintoc}
\chapter{First-Order Functions}
\labch{first-order-functions}

The article defines F1VAE by adding \term{first-order functions} to VAE.
First-order functions cannot take functions as arguments or return functions. The
articles have followed the same pattern, extending a language by defining syntax
and semantics, since the previous one. This article and later articles omit
tedious details unless complex concepts appear.

F1VAE covered by the article differs from that of the lecture. The lecture
defines function definitions and expressions of F1VAE while the article
additionally defines programs of F1VAE. Defining programs makes the language
complete but is not the main topic of the article. Please focus on the syntax and
semantics of calling first-order functions.

\section{Syntax}

The following is the abstract syntax of F1VAE:

\[
\begin{array}{lrcl}
\text{Integer} & n & \in & \mathbb{Z} \\
\text{Variable} & x & \in & \textit{Id} \\
\text{Function Name} & f & \in & \textit{Id} \\
\text{Expression} & e & ::= & n \\
&& | & e + e \\
&& | & e - e \\
&& | & \textsf{val}\ x = e\ \textsf{in}\ e \\
&& | & x \\
&& | & f(e) \\
\text{Value} & v & ::= & n \\
\text{Function Definition} & d & ::= & f(x)=e \\
\text{Program} & P & ::= & e \\
&& | & d;P
\end{array}
\]

Expressions of F1VAE are function applications in addition to those of VAE.
\(f(e)\) is a function application, which applies a function named \(f\) to a
value denoted by \(e\).

The name, the name of a parameter, and a body expression defines a function.
Metavariable \(d\) and \(f\) respectively range over function definitions and
function names.

A program is either an expression or the pair of a function definition and a
program. In other words, it is an expression following an arbitrary number of
function definitions. Metavariable \(P\) ranges over programs.

The following is an example of an F1VAE program:

\[
\begin{array}{l}
id(x)=x; \\
twice(x)=x+x; \\
\textsf{val}\ x=1\ \textsf{in}\ twice(id(x))
\end{array}
\]

\section{Semantics}

\[
\begin{array}{lrcl}
\text{Environment} & \sigma & \in & \mathit{Id}\hookrightarrow \text{Value}
\end{array}
\]

An environment is a partial function from identifiers to values. It stores values
denoted by variables.

Evaluating an expression requires not only values denoted by variables but also
function definitions denoted by function names.

\[
\begin{array}{lrcl}
\text{Function Environment} & \Lambda & \in & \mathit{Id}\hookrightarrow
(\mathit{Id}\times\text{Expression})
\end{array}
\]

A function environment is a partial function from identifiers to pairs of
identifiers and expressions. It stores names of parameters and bodies denoted by
function names.

\[\Rightarrow\subseteq\text{Environment}\times\text{Function
Environment}\times\text{Expression}\times\text{Value}\]

An environment and a function environment are essential to evaluate an
expression. \(\Rightarrow\) is a relation over four sets. \(\sigma,\Lambda\vdash
e\Rightarrow v\) implies that evaluating \(e\) under \(\sigma\) and \(\Lambda\)
yields \(v\).

\[
\inferrule
{
  f\in\mathit{Domain}(\Lambda) \\
  \Lambda(f)=(x,e') \\
  \sigma,\Lambda\vdash e\Rightarrow v' \\
  \lbrack x\mapsto v'\rbrack,\Lambda\vdash e'\Rightarrow v
}
{ \sigma,\Lambda\vdash f(e)\Rightarrow v }
\]

The inference rule defines the semantics of a function application. An
environment used by a function body is an environment existing when the function
is defined but not called. Function definitions do not belong to the scopes of
the binding occurrences of any variables. Therefore, programs define every
function under the empty environment. The rule uses \(\lbrack x\mapsto
v'\rbrack\) instead of \(\sigma\lbrack x\mapsto v'\rbrack\) to evaluate \(e'\),
the body of a function. On the other hand, the scope of the binding occurrence of
every function name equals an entire program. A whole program is under the same
function environment. The rule uses \(\Lambda\) to evaluate both \(e\) and
\(e'\).

The other rules equal those of VAE except they need function environments.

\[
\sigma,\Lambda\vdash n\Rightarrow n
\]

\[
\inferrule
{ \sigma,\Lambda\vdash e_1\Rightarrow n_1 \\ \sigma,\Lambda\vdash
e_2\Rightarrow n_2 }
{ \sigma,\Lambda\vdash e_1+e_2\Rightarrow n_1+n_2 }
\]

\[
\inferrule
{ \sigma,\Lambda\vdash e_1\Rightarrow n_1 \\ \sigma,\Lambda\vdash
e_2\Rightarrow n_2 }
{ \sigma,\Lambda\vdash e_1-e_2\Rightarrow n_1-n_2 }
\]

\[
\inferrule
{
  \sigma,\Lambda\vdash e_1\Rightarrow v_1 \\
  \sigma\lbrack x\mapsto v_1\rbrack,\Lambda\vdash e_2\Rightarrow v_2
}
{ \sigma,\Lambda\vdash \textsf{val}\ x=e_1\ \textsf{in}\ e_2\Rightarrow v_2 }
\]

\[
\inferrule
{ x\in\mathit{Domain}(\sigma) }
{ \sigma,\Lambda\vdash x\Rightarrow \sigma(x)}
\]

The semantics of a program is a relation over function environments, programs,
and values. The semantics of an expression has already used \(\Rightarrow\), but
using \(\Rightarrow\) for also the semantics of a program retains clarity and
thus can be abused for convenience.

\[\Rightarrow\subseteq\text{Function
Environment}\times\text{Program}\times\text{Value}\]

\(\Lambda\vdash P\Rightarrow v\) implies that evaluating \(P\) under \(\Lambda\)
yields \(v\).

\[
\inferrule
{ \emptyset,\Lambda\vdash e\Rightarrow v }
{ \Lambda\vdash e\Rightarrow v }
\]

Evaluating a program without function definitions equals evaluating its
expression.

\[
\inferrule
{ \Lambda\lbrack f\mapsto(x,e)\rbrack\vdash P\Rightarrow v }
{ \Lambda\vdash f(x)=e;P\Rightarrow v }
\]

Evaluating a program that is the pair of a function definition and a program
equals evaluating the latter program under a function environment containing the
function definition.

\section{Implementing an Interpreter}

The following Scala code expresses the abstract syntax of F1VAE:

\begin{verbatim}
sealed trait Expr
case class Num(n: Int) extends Expr
case class Add(l: Expr, r: Expr) extends Expr
case class Sub(l: Expr, r: Expr) extends Expr
case class Val(x: String, i: Expr, b: Expr) extends Expr
case class Id(x: String) extends Expr
case class App(f: String, a: Expr) extends Expr
\end{verbatim}

Dictionaries encode both an environment and a function environment. The keys and
the values of an environment are strings and integers respectively. The keys and
the values of a function environment are strings and pairs of strings and
expressions of F1VAE respectively.

\begin{verbatim}
type Env = Map[String, Int]
type FEnv = Map[String, (String, Expr)]
\end{verbatim}

Function \verb!lookup! finds a value denoted by an identifier from an
environment. Function \verb!lookupFD! finds a function denoted by an identifier
from a function environment.

\begin{verbatim}
def lookup(x: String, env: Env): Int =
  env.getOrElse(x, throw new Exception)

def lookupFD(f: String, fEnv: FEnv): (String, Expr) =
  fEnv.getOrElse(f, throw new Exception)
\end{verbatim}

Function \verb!interp! takes an expression, an environment, and a function
environment as arguments and calculates a value denoted by the expression.

\begin{verbatim}
def interp(e: Expr, env: Env, fEnv: FEnv): Int = e match {
  case Num(n) => n
  case Add(l, r) => interp(l, env, fEnv) + interp(r, env, fEnv)
  case Sub(l, r) => interp(l, env, fEnv) - interp(r, env, fEnv)
  case Val(x, i, b) =>
    interp(b, env + (x -> interp(i, env, fEnv)), fEnv)
  case Id(x) => lookup(x, env)
  case App(f, a) =>
    val (x, e) = lookupFD(f, fEnv)
    interp(e, Map(x -> interp(a, env, fEnv)), fEnv)
}
\end{verbatim}

The \verb!App! case creates a new environment, which contains a single
identifier, to evaluate the body of a function.

The following is an example of calling \verb!interp!:

\begin{verbatim}
// id(x) = x;
// twice(x) = x + x;
// val x = 1 in twice(id(x))
interp(
  Val("x", Num(1),
    App("twice",
      App("id", Id("x"))
    )
  ),
  Map.empty,
  Map(
    "id" -> ("x", Id("x")),
    "twice" -> ("x", Add(Id("x"), Id("x")))
  )
)
// 2
\end{verbatim}

\section{Scope}

\subsection{Static Scope}

The semantics and the interpreter use \term{static scope}. Static scope enforces
function bodies to use environments existing when the functions are defined.
Calling below function \verb!f! always results in a run-time error.

\[f(x)=x+y\]

Static scope allows finding free identifiers and binding occurrences binding
bound occurrences without executing programs. Besides, every bound occurrence is
bound to a fixed single binding occurrence. Since code, not an execution,
determines entities referred by identifiers under static scope, \term{lexical
scope} is another name of static scope.

Most modern languages feature static scope.

\subsection{Dynamic Scope}

Unlike static scope, \term{dynamic scope} makes function bodies use environments
from function call-sites. A value denoted by identifier \(y\) of below function
\(f\) depends on a call-site. The identifier can be either free or bound.
Different binding occurrences may bind it during different calls.

\[f(x)=x+y\]

For example, the below expression denotes \(3\). At each call of \(f\), \(y\)
refers to a different value.

\[
\begin{array}{l}
f(x)=x+y; \\
(\textsf{val}\ y=1\ \textsf{in}\ f(0))\ +\ (\textsf{val}\ y=2\ \textsf{in}\ f(0))
\end{array}
\]

The following inference rule defines the semantics of a function application
using dynamic scope.

\[
\inferrule
{
  f\in\mathit{Domain}(\Lambda) \\
  \Lambda(f)=(x,e') \\
  \sigma,\Lambda\vdash e\Rightarrow v' \\
  \sigma\lbrack x\mapsto v'\rbrack,\Lambda\vdash e'\Rightarrow v
}
{ \sigma,\Lambda\vdash f(e)\Rightarrow v }
\]

Revising the \verb!App! case of the \verb!interp! function makes the interpreter
use dynamic scope.

\begin{verbatim}
def interp(e: Expr, env: Env, fEnv: FEnv): Int = e match {
  ...
  case App(f, a) =>
    val (x, e) = lookupFD(f, fEnv)
    interp(e, env + (x -> interp(a, env, fEnv)), fEnv)
}
\end{verbatim}

Dynamic scope prevents programs from being modular. An environment from a
call-site affects the behavior of a function under dynamic scope. Different parts
of a program unexpectedly interfere with each other. Programs show unexpected
behaviors and become error-prone.

Languages hardly feature dynamic scope. Common LISP is one example. \term{Macros}
in C behave similarly to functions under dynamic scope.

\begin{verbatim}
#define f(x) x + y

int main() {
    int y = 0;
    return f(0);
}
\end{verbatim}
