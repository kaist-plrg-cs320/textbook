\setchapterpreamble[u]{\margintoc}
\chapter{Subtype Polymorphism}
\labch{subtype-polymorphism}

\renewcommand{\plang}{\textsf{TFAE}\xspace}
\renewcommand{\lang}{\textsf{STFAE}\xspace}

This chapter introduces subtype polymorphism. Subtype polymorphism is often
found in object-oriented languages, but some functional languages also support
subtype polymorphism these days. For example, OCaml, which stands for Objective
Caml, is a functional language that provides subtype polymorphism. Understanding
subtype polymorphism is important in many real-world languages.

This chapter
defines \lang by extending \plang with subtype polymorphism. To illustrate the
need of subtype polymorphism, we start with adding records to \plang. We can add
subtype polymorphism to \plang without adding records together, but examples with
records can clearly show the benefits of subtype polymorphism.

\section{Records}

A record is a value that consists of multiple values. Records are similar to tuples,
but users can designate the names of the elements in a record. When we talk
about the elements of records, we often use the term
\textit{fields}\index{field}. From the perspective
of their expressivity and roles in programming, records are the same as tuples.
However, records help programmers express their high-level ideas in the code
more directly than tuples and prevent mistakes.

For example,
consider a tuple representing the height, score, and scholarship state of a
student. The tuple $(180,91, \textsf{true})$ represents a student who is
180-centimeter-tall, got 91 points from the recent exam, and is currently
receiving a scholarship. One disadvantage of using tuples is that programmers
may use wrong elements by mistake. When $\code{st}$ denotes the previously
mentioned tuple, one should write $\code{st}\textsf{.1}$ to get the height.
However, there is no reason to associate the first element with the height. If he or
she writes $\code{st}\textsf{.2}$ instead $\code{st}\textsf{.1}$, then the wrong
conclusion---the student is 91-centimeter-tall---will be obtained.

Records
effectively prevent such mistakes. We can represent the same student with the
record $\{\code{height}=180,\code{score}=91,\code{scholarship}=\textsf{true}\}$.
This record has three fields: \code{height}, \code{score}, and
\code{scholarship}.
Then, the height can be found by $\code{st}.\code{height}$. In this case, there
is a logical reason to associate the field whose name is \code{height} with the
height of the student. It is clear that $\code{st}.\code{score}$ does not denote
the height of the student.

\subsection{Syntax}

First, we
introduce labels, which are the names of fields in records. Let $\mathcal{L}$
be the set of every label and the metavariable $l$ ranges over labels.

\[ l \in \mathcal{L} \]

Now, we define the syntax of expressions related to records.

\[ e\ ::= \ \cdots \ |\ \{l=e,\cdots,l=e\} \ |\ e.l \]

\begin{itemize}
  \item
    $\{l_1=e_1,\cdots,l_n=e_n\}$ is an expression that creates a record. A
    record has zero or more fields. $l$'s are the names of the fields.
    We assume that the fields of a single record have distinct names.
    Since a record can have zero fields, $\{\}$ is an expression.
  \item
    $e.l$ is an expression that gives the value of a certain field from a record.
    It is usually called a
    \textit{projection}\index{projection}. $e$ determines the record,
    and $l$ is the name of the field whose value is acquired.
\end{itemize}

\subsection{Dynamic Semantics}

To make a language support records, we should add record values to the
language.

\[ v \ ::= \ \cdots \ |\ \{l=v,\cdots,l=v\} \]

A value $\{l_1=v_1,\cdots,l_n=v_n\}$ is a record value. $l$'s are the names of the
fields. $v_i$ is the value of the field $l_i$. For example, the result of
$\{\ca=1+2,\cb=3+4\}$ is $\{\ca=3,\cb=7\}$. The record has two fields: $\ca$ and
$\cb$. The value of $\ca$ is $3$, and the value of $\cb$ is $7$. The result of
$\{\}$ is $\{\}$, which is the empty record value.

Now, we add rules to define the dynamic semantics of the added expressions.

\semanticrule{Record}{
  If
    \evaldn{e_1}{v_1},
    $\cdots$, and
    \evaldn{e_n}{v_n}, \\
  then
    \evaldn{\{l_1=e_1,\cdots,l_n=e_n\}}{\{l_1=v_1,\cdots,l_n=v_n\}}.
}

\vspace{-1em}

\[
  \inferrule
  { \evald{e_1}{v_1} \\
    \cdots \\
    \evald{e_n}{v_n} }
  { \evald{\{l_1=e_1,\cdots,l_n=e_n\}}{\{l_1=v_1,\cdots,l_n=v_n\}} }
  \quad\textsc{[Record]}
\]

Every $e_i$ has to be evaluated for the evaluation of
$\{l_1=e_1,\cdots,l_n=e_n\}$. If the value of $e_i$ is $v_i$, the value of
the field $l_i$ is $v_i$. The result is $\{l_1=v_1,\cdots,l_n=v_n\}$.

\semanticrule{Proj}{
  If
    \evaldn{e}{\{\cdots,l=v,\cdots\}}, \\
  then
    \evaldn{e.l}{v}.
}

\vspace{-1em}

\[
  \inferrule
  { \evald{e}{\{\cdots,l=v,\cdots\}} }
  { \evald{e.l}{v} }
  \quad\textsc{[Proj]}
\]

For the evaluation of $e.l$, $e$ has to be evaluated first.
If the result of $e$ is a record that contains a field named $l$, then
$e.l$ results in the value of the field.
If $e$ evaluates to a nonrecord value or a record value that
does not contain $l$, $e.l$ incurs a run-time error.

\subsection{Static Semantics}

Since records are a new sort of a value, exsiting types cannot include records.
We need to add new types that records can belong to.

\[ \tau \ ::= \ \cdots \ |\ \{l:\tau,\cdots,l:\tau\} \]

A type $\{l_1:\tau_1,\cdots,l_n:\tau_n\}$ is a record type that includes
records whose fields follow the designated names and types. In other words, if a
record value has fields named $l_1$ from $l_n$ and the value of a field $l_i$ is
a value of $\tau_i$, then the value belongs to
$\{l_1:\tau_1,\cdots,l_n:\tau_n\}$. For example, the type of the value $\{\ca=3,\cb=7\}$ is
$\{\ca:\tnum,\cb:\tnum\}$. In addition, the type of the expression
$\{\ca=1+2,\cb=3+4\}$, which evaluates to $\{\ca=3,\cb=7\}$, also is
$\{\ca:\tnum,\cb:\tnum\}$. Similarly, the type of the empty record is $\{\}$.

Let us define the typing rules for the added expressions.

\typerule{Typ-Record}{
  If
    \typeofdn{e_1}{\tau_1}, $\cdots$,
    \typeofdn{e_n}{\tau_n}, \\
  then
    \typeofdn{\{l_1=e_1,\cdots,l_n=e_n\}}{\{l_1:\tau_1,\cdots,l_n:\tau_n\}}.
}

\vspace{-1em}

\[
  \inferrule
  { \typeofd{e_1}{\tau_1} \\ \cdots \\
    \typeofd{e_n}{\tau_n} }
  { \typeofd{\{l_1=e_1,\cdots,l_n=e_n\}}{\{l_1:\tau_1,\cdots,l_n:\tau_n\}} }
  \quad\textsc{[Typ-Record]}
\]

Let the type of $e_i$ be $\tau_i$. Then, $e_i$ evaluates to a value of $\tau_i$.
Thus, $\{l_1=e_1,\cdots,l_n=e_n\}$ evaluates to a value of
$\{l_1:\tau_1,\cdots,l_n:\tau_n\}$, and the type of
$\{l_1=e_1,\cdots,l_n=e_n\}$ is  $\{l_1:\tau_1,\cdots,l_n:\tau_n\}$.

\typerule{Typ-Proj}{
  If
    \typeofdn{e}{\{\cdots,l:\tau,\cdots\}}, \\
  then
    \typeofdn{e.l}{\tau}.
}

\vspace{-1em}

\[
  \inferrule
  { \typeofd{e}{\{\cdots,l:\tau,\cdots\}} }
  { \typeofd{e.l}{\tau} }
  \quad\textsc{[Typ-Proj]}
\]

$e.l$ can be evaluated without an error only if $e$
evaluates to a record containing a field named $l$. Therefore, the type
of $e$ has to be a record type that contains a field $l$. Then, the type of
$e.l$ is the type of $l$.

For example, the type of $\{\ca=1+2,\cb=3+4\}.\ca$ is $\tnum$ since
the type of $\{\ca=1+2,\cb=3+4\}$ is $\{\ca:\tnum,\cb:\tnum\}$.
On the other hand, $\{\ca=1+2,\cb=3+4\}.\code{c}$ is ill-typed because
$\{\ca:\tnum,\cb:\tnum\}$ lacks a field named $\code{c}$.

\section{Subtype Polymorphism}

The current type system is sound but not expressive enough. It rejects many
expressions that do not cause any run-time errors. Consider the following
expression:

$
  (\efunt{\cx}{\{\ca:\tnum\}}{\cx.\ca})\ \{\ca=1,\cb=2\}
$

The expression evaluates $\{\ca=1,\cb=2\}.\ca$, which yields $1$ without any
error. However, the type system rejects the expression. The type of
$\{\ca=1,\cb=2\}$ is $\{\ca:\tnum,\cb:\tnum\}$, while the parameter type of
the function is $\{\ca:\tnum\}$. Since the argument type is different
from the designated parameter type, the application expression is ill-typed.

Currently, the type $\{\ca:\tnum\}$ denotes the set of every record that has
only the integer-valued field \code{a}. However, this definition is too
restrictive. The type implies that its value can be used for any place that
accesses the field \code{a} and expects the value of the field to be an integer.
Thus, the type does not need to exclude values that have other fields in
addition to the field \code{a}.

To resolve the problem, we extend the meaning of $\{\ca:\tnum\}$. Now, the type
includes any records that have an integer-valued field \code{a}. Records that
have additional fields also can be values of $\{\ca:\tnum\}$. This change can be
attained by modifying Rule~\textsc{Typ-Record} like below.

\typerule{Typ-Record'}{
  If
    \typeofdn{e_1}{\tau_1}, $\cdots$,
    \typeofdn{e_n}{\tau_n}, $\cdots$,
    \typeofdn{e_{n+m}}{\tau_{n+m}}, \\
  then
    \typeofdn{\{l_1=e_1,\cdots,l_n=e_n,\cdots,l_{n+m}=e_{n+m}\}}{\{l_1:\tau_1,\cdots,l_n:\tau_n\}}.
}

\vspace{-1em}

\[
  \inferrule
  { \typeofd{e_1}{\tau_1} \\ \cdots \\
    \typeofd{e_n}{\tau_n} \\ \cdots \\
    \typeofd{e_{n+m}}{\tau_{n+m}} }
  { \typeofd{\{l_1=e_1,\cdots,l_n=e_n,\cdots,l_{n+m}=e_{n+m}\}}{\{l_1:\tau_1,\cdots,l_n:\tau_n\}} }
  \quad\textsc{[Typ-Record']}
\]

The rule allows forgetting the types of some fields if they are
unnecessary. Now, $\{\ca=1,\cb=2\}$ is a value of $\{\ca:\tnum\}$. Thus, the
previous example, $(\efunt{\cx}{\{\ca:\tnum\}}{\cx.\ca})\ \{\ca=1,\cb=2\}$, is
well-typed.

Does this fix solve all the problems? Unfortunately, no.
Consider the following expression:

$
\ebind{\cx}{\{\ca=1,\cb=2\}}{\\
\ebind{\cy}{(\efunt{\cx}{\{\ca:\tnum\}}{\cx.\ca})\ \cx}{\\
(\efunt{\cx}{\{\ca:\tnum,\cb:\tnum\}}{\cx.\ca+\cx.\cb})\ \cx
}}
$

This expression is still ill-typed though it does not incur any run-time errors.
If we say the type of $\cx$ is ${\{\ca:\tnum\}}$, the first function application
is well-typed. However, the second function application is ill-typed. If we say
the type of $\cx$ is $\{\ca:\tnum,\cb:\tnum\}$ instead, the second function application
becomes well-typed. However, the first function application becomes ill-typed. There is no
way to make both application expressions well-typed. We need a way to consider
$\cx$ as an expression of ${\{\ca:\tnum\}}$ and as an expression of
$\{\ca:\tnum,\cb:\tnum\}$ at the same time. In other words, we should assign
multiple types to a single entity, and this is the notion of polymorphism.

\textit{Subtype polymorphism}\index{subtype polymorphism} is one way of
polymorphism, which is based on the notion of subtyping. Recall that a
type is a set of values. Sometimes, one type is a subset of another type. For
example, any values that belong to $\{\ca:\tnum,\cb:\tnum\}$ are values of
${\{\ca:\tnum\}}$, so $\{\ca:\tnum,\cb:\tnum\}$ is a subset of
${\{\ca:\tnum\}}$. When $\tau_1$ is a subset of $\tau_2$,
we say that $\tau_1$ is a \textit{subtype}\index{subtype} of $\tau_2$ and
$\tau_2$ is a \textit{supertype}\index{supertype} of $\tau_1$.
For example, $\{\ca:\tnum,\cb:\tnum\}$ is a subtype of ${\{\ca:\tnum\}}$ and
$\{\ca:\tnum\}$ is a supertype of ${\{\ca:\tnum,\cb:\tnum\}}$.
This is the notion of subtyping.

Roughly speaking, subtyping is an ``A is a B'' relation betwen types. As an example,
consider \code{Animal} and \code{Cat}, which denote the type of every animal
and the type of every cat, respectively. We know that a cat is an animal. Then,
we can say that \code{Cat} is a subtype of \code{Animal}. On the other hand, can
we say that an animal is a cat? No, because there is an animal that is not a
cat. For instance, a dog is an animal, but not a cat. Thus, \code{Animal} is not
a subtype of \code{Cat}. We can do the same thing for record types. A record
that has fields $\ca$ and $\cb$ is a record that has $\ca$. (For brevity,
ignore the types of the fields here.) Therefore, $\{\ca:\tnum,\cb:\tnum\}$ is a
subtype of ${\{\ca:\tnum\}}$. On the other hand, a record that has $\ca$ is not
a record that has both $\ca$ and $\cb$ since it may lack $\cb$. As a consequence,
${\{\ca:\tnum\}}$ is not a subtype of $\{\ca:\tnum,\cb:\tnum\}$.

Mathematically, subtyping is a relation over types and types.

\[ <:\subseteq T\times T \]

We write $\subt{\tau_1}{\tau_2}$ to denote that $\tau_1$ is a subtype of
$\tau_2$. For example, both $\subt{\code{Cat}}{\code{Animal}}$ and
$\subt{\{\ca:\tnum,\cb:\tnum\}}{\{\ca:\tnum\}}$ are true.

Now, let us see how subtyping induces polymorphsim. The key insight is that
any expression of $\tau_1$ can be treated as an expression of $\tau_2$ without
breaking type soundness when $\tau_1$ is a subtype of $\tau_2$. For example,
suppose that there is an animal hospital that cures any animal. We can consider
the hospital as a function that takes a value of \code{Animal}. A cat is an
animal, so any cat can be cured in the hospital. Thus, if an expression
evaluates to a value of \code{Cat}, it can be considered as an expression
that evaluates to a value of \code{Animal} and safely given to the function
representing the hospital. On the other hand, the inverse is false. If a hospital
cures only cats and we know only that someone has an animal, then we cannot say
to him or her to carry the animal to the hospital. There is no guarantee that
the hospital will be able to cure the animal. Thus, the fact that $\tau_1$ is a
subtype of $\tau_2$ does not imply that any expression of $\tau_2$ can be treated
as an expression of $\tau_1$. In a similar fasion, any expression of
$\{\ca:\tnum,\cb:\tnum\}$ can be treated as an expression of ${\{\ca:\tnum\}}$,
but the inverse is false. We can express this idea with the following typing
rule:

\typerule{Typ-Sub}{
  If
  \typeofdn{e}{\tau'} and \subtn{\tau'}{\tau},\\
  then
  \typeofdn{e}\tau.
}

\vspace{-1em}

\[
  \inferrule
  { \typeofd{e}\tau' \\ \subt{\tau'}{\tau} }
  { \typeofd{e}\tau }
  \quad\textsc{[Typ-Sub]}
\]

The rule states that if $e$ is an expression of $\tau'$, then it is an
expression of $\tau$ at the same time when $\tau'$ is a subtype of $\tau$.

Subtyping has two important characteristics. It is reflexive and transitive.
The reason is clear. Every set is a subset of itself. Thus, every type is a
subtype of itself. In addition, if $A$ is a subset of $B$ and $B$ is a subset of
$C$, then $A$ is a subset of $C$. Therefore, if $\tau_1$ is a subtype of
$\tau_2$ and $\tau_2$ is a subtype of $\tau_3$, then $\tau_1$ is a subtype of
$\tau_3$. We can formally state these properties with the following
\textit{subtyping rules}\index{subtyping rule}, which describe when two types
are in the subtype relation.

\typerule{Sub-Refl}{
  \subtn{\tau}{\tau}.
}

\vspace{-1em}

\[
  \subt{\tau}{\tau}
  \quad\textsc{[Sub-Refl]}
\]

\typerule{Sub-Trans}{
  If
    \subtn{\tau_1}{\tau_2} and
    \subtn{\tau_2}{\tau_3}, \\
  then
    \subtn{\tau_1}{\tau_3}.
}

\[
  \inferrule
  { \subt{\tau_1}{\tau_2} \\
    \subt{\tau_2}{\tau_3} }
  { \subt{\tau_1}{\tau_3} }
  \quad\textsc{[Sub-Trans]}
\]

If we have only Rule~\textsc{Sub-Refl} and Rule~\textsc{Sub-Trans}, we cannot
prove any interesting subtype relationships, e.g.
$\subt{\{\ca:\tnum,\cb:\tnum\}}{\{\ca:\tnum\}}$, even though both of the rules
can help us prove interesting subtype relationships by being used with other
subtyping rules. Thus, we introduce subtyping rules for record types in the
next section.

\section{Subtyping of Record Types}

Consider the previous example again. The type system should be able to prove
$\subt{\{\ca:\tnum,\cb:\tnum\}}{\{\ca:\tnum\}}$. To achieve the goal, we define
the following subtyping rule:

\typerule{Sub-Width}{
  \subtn{\{l_1:\tau_1,\cdots,l_n:\tau_n,l:\tau\}}{\{l_1:\tau_1,\cdots,l_n:\tau_n\}}.
}

\[
  \subt{\{l_1:\tau_1,\cdots,l_n:\tau_n,l:\tau\}}{\{l_1:\tau_1,\cdots,l_n:\tau_n\}}
  \quad\textsc{[Sub-Width]}
\]

Any value of $\{l_1:\tau_1,\cdots,l_n:\tau_n,l:\tau\}$ is a value of
$\{l_1:\tau_1,\cdots,l_n:\tau_n\}$ because a record that has fields from $l_1$
to $l_n$ and $l$ additionally is a record that has fields from $l_1$ to $l_n$.
Therefore, the rule is valid.

Now, $\{\ca:\tnum,\cb:\tnum\}<:\{\ca:\tnum\}$ is true. Using this fact, the following
proof tree proves that $\{\ca:\tnum\}$ is a type of $\{\ca=1,\cb=2\}$:

\[
  \small
  \inferrule
  {
    \inferrule
    { \emptyset\vdash1:\tnum \\ \emptyset\vdash2:\tnum }
    { \emptyset\vdash\{\ca=1,\cb=2\}:\{\ca:\tnum,\cb:\tnum\} }
    \\
    \{\ca:\tnum,\cb:\tnum\}<:\{\ca:\tnum\}
  }
  { \emptyset\vdash\{\ca=1,\cb=2\}:\{\ca:\tnum\} }
\]

In addition, the following expression is now well-typed:

$
\ebind{\cx}{\{\ca=1,\cb=2\}}{\\
\ebind{\cy}{(\efunt{\cx}{\{\ca:\tnum\}}{\cx.\ca})\ \cx}{\\
(\efunt{\cx}{\{\ca:\tnum,\cb:\tnum\}}{\cx.\ca+\cx.\cb})\ \cx
}}
$

$\{\ca:\tnum,\cb:\tnum\}$ is a type of $\cx$, so the second function application
is well-typed. At the same time, by Rule~\textsc{Typ-Sub}, $\{\ca:\tnum\}$ also
is a type of $\cx$, so the first function application is well-typed as well.

If we use Rule~\textsc{Sub-Width} and Rule~\textsc{Sub-Trans} together,
other interesting subtype relationships can be proven. For example, the following
proof tree proves that $\{\ca:\tnum,\cb:\tnum,\code{c}:\tnum\}$ is a subtype of
$\{\ca:\tnum\}$.

\[
  \footnotesize
  \inferrule
  { \{\ca:\tnum,\cb:\tnum,\code{c}:\tnum\}<:\{\ca:\tnum,\cb:\tnum\} \\
    \{\ca:\tnum,\cb:\tnum\}<:\{\ca:\tnum\} }
  { \{\ca:\tnum,\cb:\tnum,\code{c}:\tnum\}<:\{\ca:\tnum\} }
\]

By the same principle, $\{\}$, which is the empty record type, is a supertype
of every record type. In other words, every record type is a subtype of
$\{\}$.

Alas, the type system is still restrictive. The following expression is ill-typed but
does not cause run-time errors:

$
\ebind{\cx}{\{\ca=1,\cb=2\}}{\\
\ebind{\cy}{(\efunt{\cx}{\{\cb:\tnum,\ca:\tnum\}}{\cx.\ca+\cx.\cb})\ \cx}{\\
(\efunt{\cx}{\{\ca:\tnum,\cb:\tnum\}}{\cx.\ca+\cx.\cb})\ \cx
}}
$

The type of $\cx$ is $\{\ca:\tnum,\cb:\tnum\}$. Therefore, the second function
application is well-typed, while the first function application is not.
We need to make $\cx$ be a value of $\{\ca:\tnum,\cb:\tnum\}$ and a value of
$\{\cb:\tnum,\ca:\tnum\}$ at the same time. Like before, fixing
Rule~\textsc{Typ-Record} cannot be a proper solution. The correct solution is to
add a new subtyping rule.

The key idea to define a new subtyping rule is that the order between fields
does not matter at all. For example, a record that has fields $\ca$ and $\cb$ is
a record that has fields $\cb$ and $\ca$. Thus, it is safe to consider
$\{\ca:\tnum,\cb:\tnum\}$ as a subtype of $\{\cb:\tnum,\ca:\tnum\}$. By
generalizing this observation, we define the following subtyping rule:

\typerule{Sub-Perm}{
  If
    $(l_1,\tau_1)$, $\cdots$, $(l_n,\tau_n)$
     is a permutation of
    $(l'_1,\tau'_1)$, $\cdots$, $(l'_n,\tau'_n)$, \\
  then
    \subtn{\{l_1:\tau_1,\cdots,l_n:\tau_n\}}{\{l'_1:\tau'_1,\cdots,l'_n:\tau'_n\}}.
}

\vspace{-1em}

\[
  \inferrule
  { (l_1,\tau_1),\cdots,(l_n,\tau_n)
    \text{ is a permutation of }
    (l'_1,\tau'_1),\cdots,(l'_n,\tau'_n) }
  { \subt{\{l_1:\tau_1,\cdots,l_n:\tau_n\}}{\{l'_1:\tau'_1,\cdots,l'_n:\tau'_n\}} }
  \quad\textsc{[Sub-Perm]}
\]

The rule states that altering the order between the fields of a record type
results in a subtype of the record type.

The following proof tree proves that
$\{\ca:\tnum,\cb:\tnum\}$ is a subtype of $\{\cb:\tnum,\ca:\tnum\}$:

\[
  \inferrule
  { (\ca,\tnum),(\cb,\tnum)\text{ is a permutation of }(\cb,\tnum),(\ca,\tnum) }
  { \{\ca:\tnum,\cb:\tnum\}<:\{\cb:\tnum,\ca:\tnum\} }
\]

If we use multiple subtyping rules together,
other interesting subtype relationships can be proven. For example, the following
proof tree proves that $\{\ca:\tnum,\cb:\tnum\}$ is a subtype of
$\{\cb:\tnum\}$.

\[
  \scriptsize
  \inferrule
  {
    \inferrule
    { (\ca,\tnum),(\cb,\tnum)\text{ is a permutation of }(\cb,\tnum),(\ca,\tnum) }
    { \{\ca:\tnum,\cb:\tnum\}<:\{\cb:\tnum,\ca:\tnum\} }
    \\
    \{\cb:\tnum,\ca:\tnum\}<:\{\cb:\tnum\}
  }
  { \{\ca:\tnum,\cb:\tnum\}<:\{\cb:\tnum\} }
\]

Even after the addition of Rule~\textsc{Sub-Width} and Rule~\textsc{Sub-Perm},
the type system still can be improved more. Consider the following expression:

$
\ebind{\cx}{\{\ca=\{\ca=1,\cb=2\}\}}{\\
\ebind{\cy}{(\efunt{\cx}{\{\ca:\{\ca:\tnum\}\}}{\cx.\ca.\ca})\ \cx}{\\
(\efunt{\cx}{\{\ca:\{\ca:\tnum,\cb:\tnum\}\}}{\cx.\ca.\ca+\cx.\ca.\cb})\ \cx
}}
$

The above expression does not incur any run-time errors. However, the first
function application is ill-typed, while the second function application is
well-typed. We need to make $\{\ca=\{\ca=1,\cb=2\}\}$ be a value of
$\{\ca:\{\ca:\tnum\}\}$ and a value of $\{\ca:\{\ca:\tnum,\cb:\tnum\}\}$ at the same time by adding
a subtyping rule.

The current type system is too strict about the types of fields in records.
For example, consider $\{\ca:\{\ca:\tnum,\cb:\tnum\}\}$ and
$\{\ca:\{\ca:\tnum\}\}$. A value of $\{\ca:\{\ca:\tnum,\cb:\tnum\}\}$ has at
least one field, whose name is $\ca$. The value of the field is a value of
$\{\ca:\tnum,\cb:\tnum\}$. We already know that any value of
$\{\ca:\tnum,\cb:\tnum\}$ is a value of $\{\ca:\tnum\}$. Therefore, we can
say that a value of $\{\ca:\{\ca:\tnum,\cb:\tnum\}\}$ has at least one field,
whose name is $\ca$ and type is $\{\ca:\tnum\}$. In fact, it is the characteristic of a
value of $\{\ca:\{\ca:\tnum\}\}$. As a result, any value of
$\{\ca:\{\ca:\tnum,\cb:\tnum\}\}$ is a value of $\{\ca:\{\ca:\tnum\}\}$ at the
same time, so $\{\ca:\{\ca:\tnum,\cb:\tnum\}\}$ must be a subtype of
$\{\ca:\{\ca:\tnum\}\}$. By
generalizing this observation, we define the following subtyping rule:

\typerule{Sub-Depth}{
  If
    \subtn{\tau_1}{\tau'_1}, $\cdots$,
    \subtn{\tau_n}{\tau'_n}, \\
  then
    \subtn{\{l_1:\tau_1,\cdots,l_n:\tau_n\}}{\{l_1:\tau'_1,\cdots,l_n:\tau'_n\}}.
}

\[
  \inferrule
  { \subt{\tau_1}{\tau'_1} \\ \cdots \\
    \subt{\tau_n}{\tau'_n} }
  { \subt{\{l_1:\tau_1,\cdots,l_n:\tau_n\}}{\{l_1:\tau'_1,\cdots,l_n:\tau'_n\}} }
  \quad\textsc{[Sub-Depth]}
\]

The rule states that strengthening\sidenote{If we strengthen a type, a subtype
is obtained in the sense that a subtype imposes a stronger condition on its
value than the original type.} the type of each field in a record type
results in a subtype of the record type.

By using Rule~\textsc{Sub-Width} and Rule~\textsc{Sub-Depth} together, we can
prove that $\{\ca:\{\ca:\tnum,\cb:\tnum\}\}$ is a subtype of
$\{\ca:\{\ca:\tnum\}\}$.

\[
  \inferrule
  { \{\ca:\tnum,\cb:\tnum\}<:\{\ca:\tnum\} }
  { \{\ca:\{\ca:\tnum,\cb:\tnum\}\}<:\{\ca:\{\ca:\tnum\}\} }
\]

\section{Subtyping of Function Types}

It is time to consider a subtyping rule for function types. A function type
consists of a parameter type and a return type. Let us discuss
return types first.

Consider two function types: $\tarrow{\tau_1}{\tau_2}$ and
$\tarrow{\tau_1}{\tau_2'}$. Assume that $\tau_2$ is a subtype of
$\tau_2'$. A value of $\tarrow{\tau_1}{\tau_2}$ is a function that takes a value
of $\tau_1$ and returns a value of $\tau_2$. Since $\tau_2$ is a subtype of
$\tau_2'$, any value of $\tau_2$ is a value of $\tau_2'$.
Therefore, we can say that a function of $\tarrow{\tau_1}{\tau_2}$ returns a
value of $\tau_2'$. It implies that the function is a value of
$\tarrow{\tau_1}{\tau_2'}$ at the same time. Thus, any value of
$\tarrow{\tau_1}{\tau_2}$ is a value of $\tarrow{\tau_1}{\tau_2'}$, and
$\tarrow{\tau_1}{\tau_2}$ is a subtype of $\tarrow{\tau_1}{\tau_2'}$.
The following rule formalizes this fact:

\typerule{Sub-Ret}{
  If
    \subtn{\tau_2}{\tau_2'}, \\
  then
    \subtn{\tarrow{\tau_1}{\tau_2}}{\tarrow{\tau_1}{\tau_2'}}.
}

\vspace{-1em}

\[
  \inferrule
  { \subt{\tau_2}{\tau_2'} }
  { \subt{\tarrow{\tau_1}{\tau_2}}{\tarrow{\tau_1}{\tau_2'}} }
  \quad\textsc{[Sub-Ret]}
\]

Function types preserve the subtype relationship between their return types.
Suppose that there are two function types. If their parameter types are the same
and the return type of the former is a subtype of the return type of the latter, then the
former is a subtype of the latter.
For example, $\tarrow{\tnum}{\{\ca:\tnum,\cb:\tnum\}}$
is a subtype of $\tarrow{\tnum}{\{\ca:\tnum\}}$.

Now, let us discuss parameter types.
Consider two function types: $\tau_1\rightarrow\tau_2$ and
$\tau_1'\rightarrow\tau_2$. Assume that $\tau_1'$ is a subtype of
$\tau_1$. A value of $\tarrow{\tau_1}{\tau_2}$ is a function that takes a value
of $\tau_1$ and returns a value of $\tau_2$. Since $\tau_1'$ is a subtype of
$\tau_1$, any value of $\tau_1'$ is a value of $\tau_1$.
Therefore, a function of $\tarrow{\tau_1}{\tau_2}$ should work properly
when a value of $\tau_1'$ is given. We can say that the function takes a value
of $\tau_1'$ and returns a value of $\tau_2$. It implies that the function is a value of
$\tarrow{\tau_1'}{\tau_2}$ at the same time. Thus, any value of
$\tarrow{\tau_1}{\tau_2}$ is a value of $\tarrow{\tau_1'}{\tau_2}$, and
$\tarrow{\tau_1}{\tau_2}$ is a subtype of $\tarrow{\tau_1'}{\tau_2}$.
The following rule formalizes this fact:

\typerule{Sub-Param}{
  If
    \subtn{\tau_1'}{\tau_1}, \\
  then
    \subtn{\tarrow{\tau_1}{\tau_2}}{\tarrow{\tau_1'}{\tau_2}}.
}

\vspace{-1em}

\[
  \inferrule
  { \subt{\tau_1'}{\tau_1} }
  { \subt{\tarrow{\tau_1}{\tau_2}}{\tarrow{\tau_1'}{\tau_2}} }
  \quad\textsc{[Sub-Param]}
\]

Function types reverse the subtype relationship between their parameter types.
Suppose that there are two function types. If their return types are the same
and the parameter type of the former is a supertype of the parameter type of the latter, then
the former is a subtype of the latter.
For example, $\tarrow{\{\ca:\tnum\}}{\tnum}$
is a subtype of $\tarrow{\{\ca:\tnum,\cb:\tnum\}}{\tnum}$.

We can combine Rule~\textsc{Sub-Ret} and Rule~\textsc{Sub-Param} to form a
single rule.

\typerule{Sub-ArrowT}{
  If
    \subtn{\tau_1'}{\tau_1} and
    \subtn{\tau_2}{\tau_2'}, \\
  then
    \subtn{\tarrow{\tau_1}{\tau_2}}{\tarrow{\tau_1'}{\tau_2'}}.
}

\vspace{-1em}

\[
  \inferrule
  { \subt{\tau_1'}{\tau_1} \\
    \subt{\tau_2}{\tau_2'} }
  { \subt{\tarrow{\tau_1}{\tau_2}}{\tarrow{\tau_1'}{\tau_2'}} }
  \quad\textsc{[Sub-ArrowT]}
\]

\section{Top and Bottom Types}

Now, we add two types to \lang.

\[ \tau \ ::= \ \cdots \ |\ \ttop \ |\ \tbot \]

$\ttop$ is the \textit{top type}.\index{top type} The type denotes the set of
every value. The set is a superset of any set of values. Thus, the top type is a
supertype of every type. In other words, every type is a subtype of the top
type. The following is the subtyping rule for the top type:

\typerule{Sub-TopT}{
  \subtn{\tau}{\ttop}.
}

\[
  \subt{\tau}{\ttop}
  \quad\textsc{[Sub-TopT]}
\]

The top type can be used to give a single type to two or more completely
irrelevant expressions. Suppose that the language has conditional expressions.
Then, the type of the following expression is $\{\ca:\tnum\}$:

$\eifz{0}{\{\ca=1\}}{\{\ca=1,\cb=2\}}$

However, the following expression is ill-typed in \lang when the top type does
not exist:

$\eifz{0}{\{\ca=1\}}{1}$

By extending \lang with the top type, the type of the above expression can be
$\ttop$.

$\tbot$ is the \textit{bottom type}\index{bottom type}, which is the dual of
the top type. The bottom type denotes the empty set. Since the empty set is a subset of
any set, the bottom type is a subtype of every type, and
every type is a supertype of the bottom type. The following is the subtyping rule
for the bottom type:

\typerule{Sub-BottomT}{
  \subtn{\tbot}{\tau}.
}

\[
  \subt{\tbot}{\tau}
  \quad\textsc{[Sub-BottomT]}
\]

Even though no value is a value of the the bottom type,
the bottom type is useful. It can be the type of
expressions that throw exceptions or call first-class continuations. Those
expressions do not evaluate to any values. They just change control flows.
Thus, it is quite natural to say that the type of such an expression
is the bottom type.

\section{Exercises}

\begin{exercise}
\labex{subtype-polymorphism-welltyped}

Write whether each expression is well-typed in \lang without the top type,
If so, draw the type derivation. Otherwise, explain why.
\begin{enumerate}
\item $\eifz{1}{\{ \}}{2}$
\item $\eifz{1}{\{ \}}{\{\ca=2\}}$
\end{enumerate}

\end{exercise}

\begin{exercise}
\labex{subtype-polymorphism-arrow}

If we change Rule~\textsc{Sub-ArrowT} like below, the language is not type
  sound.
  \[
    \inferrule
    { \tau_1<:\tau_1' \\ \tau_2<:\tau_2' }
    { \tau_1\rightarrow\tau_2<:\tau_1'\rightarrow\tau_2' }
  \]
Write an expression that is accepted by the new type system but
causes a run-time error.

\end{exercise}

\begin{exercise}
\labex{subtype-polymorphism-subtyper}

Complete the following implementation of an \lang subtype checker:

\begin{verbatim}
sealed trait Type
case object TopT extends Type
case object BottomT extends Type
case object NumT extends Type
case class ArrowT(p: Type, r: Type) extends Type
case class RecordT(fs: Map[String, Type]) extends Type

def subtype(t1: Type, t2: Type): Boolean = (t1, t2) match {
  case ??? => true
  case ??? => true
  case ??? => true
  case (ArrowT(p1, r1), ArrowT(p2, r2)) => ???
  case (RecordT(fs1), RecordT(fs2)) => ???
  case _ => false
}
\end{verbatim}

where \code{RecordT(Map($l_1$ -> $\tau_1$, $\cdots$, $l_n$ -> $\tau_n$))}
denotes $\{l_1:\tau_1,\cdots,l_n:\tau_n\}$.

The specification of \code{subtype} is as follows:
\begin{itemize}
  \item \code{subtype($\tau_1$, $\tau_2$)} is \code{true} when $\subt{\tau_1}{\tau_2}$
    is true.
  \item \code{subtype($\tau_1$, $\tau_2$)} is \code{false} when $\subt{\tau_1}{\tau_2}$
    is false.
\end{itemize}

\end{exercise}

\begin{exercise}
\labex{subtype-polymorphism-typer}

Complete the following implementation of a \lang type checker:

\begin{verbatim}
sealed trait Type
case object NumT extends Type
case class ArrowT(p: Type, r: Type) extends Type
case object TopT extends Type
case object BottomT extends Type

def typeCheck(e: Expr, tenv: TEnv): Type = e match {
  case Num(n) => NumT
  case Add(l, r) =>
    val lt = typeCheck(l, tenv)
    val rt = typeCheck(r, tenv)
    ???
  case Id(x) => tenv(x)
  case Fun(x, t, b) =>
    ArrowT(t, typeCheck(b, tenv + (x -> t)))
  case App(f, a) =>
    val ft = typeCheck(f, tenv)
    val at = typeCheck(a, tenv)
    ft match {
      case NumT => ???
      case ArrowT(pt, rt) => ???
      case TopT => ???
      case BottomT => ???
    }
  case Throw => BottomT
}
\end{verbatim}

We say that $\tau$ is the minimal type of $e$ under $\Gamma$ if and only if
the following conditions are satisfied:

\begin{itemize}
  \item $\typeofd{e}{\tau}$
  \item for every $\tau'$ satisfying $\typeofd{e}{\tau'}$, $\subt{\tau}{\tau'}$
\end{itemize}

If $\tau$ is the minimal type of $e$ under $\Gamma$, \code{typeCheck($e$,
$\Gamma$)} returns $\tau$. If there is no $\tau$ satisfying $\typeofd{e}{\tau}$,
\code{typeCheck($e$, $\Gamma$)} throws an exception.

You may use the following helper function without defining it:

\begin{verbatim}
// returns true if t1 is a subtype of t2
// returns false otherwise
def subtype(t1: Type, t2: Type): Boolean
\end{verbatim}

\end{exercise}

\begin{exercise}
\labex{subtype-polymorphism-lattice}

\code{Type} is defined as the same as \refex{subtype-polymorphism-subtyper}.
Consider the following implementation:

\begin{verbatim}
def lub(t1: Type, t2: Type): Type = (t1, t2) match {
  case (BottomT, t) => ???
  case (t, BottomT) => ???
  case (NumT, NumT) => ???
  case (ArrowT(p1, r1), ArrowT(p2, r2)) => ???
  case (RecordT(fs1), RecordT(fs2)) => ???
  case _ => TopT
}

def glb(t1: Type, t2: Type): Type = (t1, t2) match {
  case (TopT, t) => ???
  case (t, TopT) => ???
  case (NumT, NumT) => ???
  case (ArrowT(p1, r1), ArrowT(p2, r2)) => ???
  case (RecordT(fs1), RecordT(fs2)) => ???
  case _ => BottomT
}

def typeCheck(e: Expr, tenv: TEnv): Type = e match {
  ...
  case If0(c, t, f) => ???
}
\end{verbatim}

\code{lub} and \code{glb} compute the \emph{least upper bound} and the
\emph{greatest lower bound} of given types, respectively.
The least upper bound of $\tau_1$ and $\tau_2$ is $\tau_3$ if and only if
the following three conditions are satisfied:

\begin{itemize}
  \item $\subt{\tau_1}{\tau_3}$
  \item $\subt{\tau_2}{\tau_3}$
  \item for every $\tau$, if $\subt{\tau_1}{\tau}$ and $\subt{\tau_2}{\tau}$,
    then $\subt{\tau_3}{\tau}$
\end{itemize}

The greatest lower bound of $\tau_1$ and $\tau_2$ is $\tau_3$ if and only if
the following three conditions are satisfied:

\begin{itemize}
  \item $\subt{\tau_3}{\tau_1}$
  \item $\subt{\tau_3}{\tau_2}$
  \item for every $\tau$, if $\subt{\tau}{\tau_1}$ and $\subt{\tau}{\tau_2}$,
    then $\subt{\tau}{\tau_3}$
\end{itemize}

\code{typeCheck} satisfies the same condition as that of
\refex{subtype-polymorphism-typer}.

Complete the implementation.

\end{exercise}

\begin{exercise}
\labex{subtype-polymorphism-box}

Consider \plang with boxes in \refex{type-systems-box}.
  When can $\textsf{box}\ \tau_1$ be a subtype of $\textsf{box}\ \tau_2$?
  Write a new subtyping rule for box types.

\end{exercise}

\begin{exercise}
\labex{subtype-polymorphism-list}

Consider \plang with lists in \refex{type-systems-list}.
  \begin{enumerate}
    \item When can $\textsf{list}\ \tau_1$ be a subtype of $\textsf{list}\
      \tau_2$? Write a new subtyping rule for list types.
    \item Suppose that we extend the language as follows:

      \vspace{0.5em}
      $\begin{array}{@{}r@{~}r@{~}l@{}}
        e & ::= & \cdots\ |\ e[e]:=e \\
        \tau & ::= & \cdots\ |\ \ttop \\
      \end{array}$
      \vspace{0.5em}

      The typing rule for the new expression, which mutates an element of a
      list, is as follows:
      \[
        \inferrule
        { \typeofd{e_1}{\textsf{list}\ \tau} \\
          \typeofd{e_2}{\tnum} \\
          \typeofd{e_3}{\tau}
        }
        { \typeofd{e_1[e_2]:=e_3}{\tau} }
      \]

      When can $\textsf{list}\ \tau_1$ be a subtype of $\textsf{list}\
      \tau_2$? Write a new subtyping rule for list types.
  \end{enumerate}

\end{exercise}

\begin{exercise}
\labex{subtype-polymorphism-vcc}

This exercise extends \lang with first-class continuations.

    \vspace{0.5em}
    $e\ ::=\ \cdots\ |\ (\textsf{vcc}\ x\ \textsf{in}\ e){:}\tau$
    \vspace{0.5em}

    The type of
    $(\textsf{vcc}\ x\ \textsf{in}\ e){:}\tau$ is $\tau$ when it is well-typed.

\begin{enumerate}
  \item
    Write the typing rule of $(\textsf{vcc}\ x\ \textsf{in}\ e){:}\tau$
    of the form \fbox{$\Gamma\vdash{e}:{\tau}$}.
  \item Draw the type derivation tree of
    $(\textsf{vcc}\ {\code{x}}\ \textsf{in}\ {\eapp{(\eapp{\code{x}}{1})}{42}}){:}\tnum$.
\end{enumerate}

\end{exercise}

\begin{exercise}
\labex{subtype-polymorphism-intersection}

Many real-world languages, such as Scala and TypeScript, provide
intersection types. The intersection type of two types is a type containing values that
belong to both of the types. In other words, $v$ is an element of
$\tau_1\land\tau_2$, the intersection type of $\tau_1$ and $\tau_2$, if and only
if $v$ is an element of both $\tau_1$ and $\tau_2$.

\begin{enumerate}
  \item Write one or more subtyping rules each of which has
    $\subt{\tau_1\land\tau_2}{\tau_3}$ as a conclusion.
  \item Write one or more subtyping rules each of which has
    $\subt{\tau_3}{\tau_1\land\tau_2}$ as a conclusion.
  \item Draw a subtype derivation tree proving that
    $\subt{\tau_1\land\tau_2}{\tau_2\land\tau_1}$.
\end{enumerate}

\end{exercise}

\begin{exercise}
\labex{subtype-polymorphism-union}

Many real-world languages, such as Scala and TypeScript, provide
union types. The union type of two types is a type containing values that
belong to at least one of the types. In other words, $v$ is an element of
$(\tau_1\lor\tau_2)$, the union type of $\tau_1$ and $\tau_2$, if and only
if $v$ is an element of either $\tau_1$ or $\tau_2$.

\begin{enumerate}
  \item Write one or more subtyping rules each of which has
    $\subt{(\tau_1\lor\tau_2)}{\tau_3}$ as a conclusion.
  \item Write one or more subtyping rules each of which has
    $\subt{\tau_3}{(\tau_1\lor\tau_2)}$ as a conclusion.
  \item Draw a subtype derivation tree proving that
    $\subt{((\tau_1\lor\tau_2)\lor\tau_3)}{(\tau_1\lor(\tau_2\lor\tau_3))}$.
\end{enumerate}

\end{exercise}

\begin{exercise}
\labex{subtype-polymorphism-adt}

This exercise extends \textsf{TVFAE} to allow types with any number (including zero) of variants.

\vspace{0.5em}
$
    \small
  \begin{array}{@{}l}
    e ::=\ \cdots\ |\ {\textsf{type}}\ t=x@\tau+\cdots+x@\tau;\ e\ |\
           e\ {\textsf{match}}\ x(x)\rightarrow e,\cdots,x(x)\rightarrow e \\
  \end{array}
$

For example, you can write the following code in the extended language:

\vspace{0.5em}
$
  \begin{array}{@{}l}
    {\textsf{type}}\ \code{fruit}=\code{apple}@\tnum+\code{banana}@\tnum+\code{cherry}@\tnum;\\
    \cdots
  \end{array}
$
\vspace{0.5em}

Suppose that the operational semantics and the typing rules are the same as
those of \code{TVFAE} except that some rules are revised to handle zero or more
variants properly.

Some expressions are rejected by the type system even though
they do not cause run-time errors. The following expression is such an example:

\vspace{0.5em}
$
  \begin{array}{@{}l}
    \textsf{type}\ \code{abc}=\code{apple}@\tnum+\code{banana}@\tnum+\code{cherry}@\tnum;\\
    \textsf{val}\ \code{f}=\efun{\code{x}{:}\code{abc}}{(}\\
    \ \ \ \ \code{x}\ \textsf{match} \\
    \ \ \ \ \ \ \ \ \code{apple}(\code{a})\rightarrow\code{a} \\
    \ \ \ \ \ \ \ \ \code{banana}(\code{b})\rightarrow\code{b} \\
    \ \ \ \ \ \ \ \ \code{cherry}(\code{c})\rightarrow\code{c} \\
    );\\
    \textsf{type}\ \code{ab}=\code{apple}@\tnum+\code{banana}@\tnum;\\
    \eapp{\code{f}}{(\eapp{\code{apple}}{42})}
  \end{array}
$
\vspace{0.5em}

We want to add subtyping to the language to allow more expressions including
the above one. Add subtyping rule(s) of the form \fbox{$\Gamma\vdash\subt{\tau}{\tau}$}
to the language. Assume that the following rules already exist:

\[
  \inferrule{}{\Gamma\vdash\subt{\tau}{\tau}}
  \qquad
  \inferrule
  {\Gamma\vdash\subt{\tau_1}{\tau_2} \\ \Gamma\vdash\subt{\tau_2}{\tau_3}}
  {\Gamma\vdash\subt{\tau_1}{\tau_3}}
\]
\[
  \inferrule
  { \Gamma\vdash\subt{\tau_1'}{\tau_1} \\ \Gamma\vdash\subt{\tau_2}{\tau_2'} }
  { \Gamma\vdash\subt{{\tau_1}\rightarrow{\tau_2}}{{\tau_1'}\rightarrow{\tau_2'}} }
  \qquad
  \inferrule
  { \typeofd{e}{\tau'} \\ \Gamma\vdash\subt{\tau'}{\tau} }
  { \typeofd{e}{\tau} }
\]

\end{exercise}

\begin{exercise}
\labex{subtype-polymorphism-exception}

Consider a language with exceptions. Its syntax and semantics are as
  follows:
\[
  \begin{array}{rrl}
    e & ::= & n\ |\ \eadd{e}{e}\ |\ x\ |\ \efunt{x}{\tau}{e}\ |\
    \eapp{e}{e}\ |\ \textsf{throw}\ e\ |\ \textsf{try}\ e\ \textsf{catch}\ e \\
    v & ::= & n\ |\ \clov{x}{e}{\sigma} \\
    r & ::= & v\ |\ \uparrow v \\
    \tau & ::= & \tnum\ |\ \tarrow{\tau}{\tau/\uparrow\tau} \ |\ \ttop\ |\ \tbot
  \end{array}
\]
\[
\begin{array}{@{}c}
  \inferrule
  {}
  { \evald{n}{n} }
  \quad
  \inferrule
  { \evald{e_1}{\uparrow v} }
  { \evald{\eadd{e_1}{e_2}}{\uparrow v} }
  \quad
  \inferrule
  { \evald{e_1}{v_1} \\
    \evald{e_2}{\uparrow v_2} }
  { \evald{\eadd{e_1}{e_2}}{\uparrow v_2} }
  \\[2em]
  \inferrule
  { \evald{e_1}{n_1} \\
    \evald{e_2}{n_2} }
  { \evald{\eadd{e_1}{e_2}}{n_1+n_2} }
  \quad
  \inferrule
  { x\in\dom{\sigma} }
  { \evald{x}{\sigma(x)} }
  \\[2em]
  \inferrule
  {}
  { \evald{\efunt{x}{\tau}{e}}{\clov{x}{e}{\sigma}} }
  \quad
  \inferrule
  { \evald{e_1}{\uparrow v} }
  { \evald{\eapp{e_1}{e_2}}{\uparrow v} }
  \\[2em]
  \inferrule
  { \evald{e_1}{v_1} \\
    \evald{e_2}{\uparrow v_2} }
  { \evald{\eapp{e_1}{e_2}}{\uparrow v_2} }
  \\[2em]
  \inferrule
  { \evald{e_1}{\clov{x}{e}{\sigma'}} \\
    \evald{e_2}{v_2} \\
    \eval{\sigma'[x\mapsto v_2]}{e}{r} }
  { \evald{\eapp{e_1}{e_2}}{r} }
\end{array}
\]
\[
\begin{array}{@{}c}
  \inferrule
  { \evald{e}{\uparrow v} }
  { \evald{\textsf{throw}\ e}{\uparrow v} }
  \quad
  \inferrule
  { \evald{e}{v} }
  { \evald{\textsf{throw}\ e}{\uparrow v} }
  \\[2em]
  \inferrule
  { \evald{e_1}{\uparrow v_1} \\ \evald{e_2}{\uparrow v_2} }
  { \evald{\textsf{try}\ e_1\ \textsf{catch}\ e_2}{\uparrow v_2} }
  \\[2em]
  \inferrule
  { \evald{e_1}{\uparrow v_1} \\
    \evald{e_2}{\clov{x}{e}{\sigma'}} \\
    \eval{\sigma'[x\mapsto v_1]}{e}{r} }
  { \evald{\textsf{try}\ e_1\ \textsf{catch}\ e_2}{r} }
  \\[2em]
  \inferrule
  { \evald{e_1}{v} }
  { \evald{\textsf{try}\ e_1\ \textsf{catch}\ e_2}{v} }
\end{array}
\]

where
\begin{itemize}
  \item $\evald{e}{v}$ implies that $e$ evaluates to $v$ under $\sigma$, and
  \item $\evald{e}{\uparrow v}$ implies that $e$ throws an exception carrying $v$ under
$\sigma$.
\end{itemize}
Note that each exception carries a single value, which can be used by
exception handlers.

A type $\tarrow{\tau_1}{\tau_2/\uparrow\tau_3}$ is a type of a function that takes a
value of $\tau_1$ as an argument and either returns a value of $\tau_2$ or
throws an exception carrying a value of $\tau_3$.
The following subtyping rules are given:
\[
\begin{array}{@{}c}
  \subt{\tau}{\tau}
  \quad
  \inferrule
  { \subt{\tau_1}{\tau_2} \\ \subt{\tau_2}{\tau_3} }
  { \subt{\tau_1}{\tau_3} }
  \quad
  \inferrule
  { \subt{\tau_1'}{\tau_1} \\ \subt{\tau_2}{\tau_2'} \\ \subt{\tau_3}{\tau_3'} }
  { \subt{\tarrow{\tau_1}{\tau_2/\uparrow\tau_3}}{\tarrow{\tau_1'}{\tau_2'/\uparrow\tau_3'}} }
  \\[2em]
  \subt{\tau}{\ttop}
  \qquad
  \subt{\tbot}{\tau}
\end{array}
\]
$\typeofd{e}{\tau_1\uparrow\tau_2}$ implies that $e$ either evaluates to
a value of $\tau_1$ or throws an exception carrying a value of $\tau_2$.
Write the typing rules of this language, where one typing rule is
given as follows:
\[
  \inferrule
  {
    \typeofd{e}{\tau_1'\uparrow\tau_2'} \\
    \subt{\tau_1'}{\tau_1} \\
    \subt{\tau_2'}{\tau_2} \\
  }
  { \typeofd{e}{\tau_1\uparrow\tau_2} }
\]

\end{exercise}
